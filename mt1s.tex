% Like most advanced LaTeX files, this one begins with a lot of
% boilerplate. You don't need to understand (or even read) most of it.
% All you need to do is fill in your name, UMN ID, email address,
% and the number of the pset. (Search for "METADATA" to find the place
% for this.) Then, you can go straight to the "EXERCISE 1"
% section and start writing your solutions.
% The "VARIOUS USEFUL COMMANDS" section is probably worth taking a
% look at at some point.

%----------------------------------------------------------------------------------------
%	PACKAGES AND OTHER DOCUMENT CONFIGURATIONS
%----------------------------------------------------------------------------------------
\documentclass[paper=a4, fontsize=12pt]{scrartcl} % A4 paper and 12pt font size
\usepackage[T1]{fontenc} % Use 8-bit encoding that has 256 glyphs
\usepackage[english]{babel} % English language/hyphenation
\usepackage{amsmath,amsfonts,amsthm,amssymb} % Math packages
\usepackage{mathrsfs}    % More math packages
\usepackage{sectsty}  % Allows customizing section commands
\allsectionsfont{\centering \normalfont\scshape} % Make all section titles centered, the default font and small caps %remove this to left align section tites
\usepackage{hyperref} % Turns cross-references into hyperlinks,
                      % and defines \url and \href commands.
\usepackage{graphicx} % For embedding graphics files.
\usepackage{framed}   % For the "leftbar" environment used below.
\usepackage{ifthen}   % Used for the \powset command below.
\usepackage{lastpage} % for counting the number of pages
\usepackage[headsepline,footsepline,manualmark]{scrlayer-scrpage}
\usepackage[height=10in,a4paper,hmargin={1in,0.8in}]{geometry}
\usepackage[usenames,dvipsnames]{xcolor}
\usepackage{tikz}     % This is a powerful tool to draw vector
                      % graphics inside LaTeX. In particular, you can
                      % use it to draw graphs.
\usepackage{verbatim} % For the "verbatim" environment, in which
                      % special symbols can be used freely without
                      % confusing the compiler. (And it's typeset in
                      % a constant-width font.)
                      % Useful, e.g., for quoting code (or ASCII art).

%\numberwithin{table}{section} % Number tables within sections (i.e. 1.1, 1.2, 2.1, 2.2 instead of 1, 2, 3, 4)

\setlength\parindent{20pt} % Makes indentation for paragraphs longer.
                           % This makes paragraphs stand out more.

%----------------------------------------------------------------------------------------
%	VARIOUS USEFUL COMMANDS
%----------------------------------------------------------------------------------------
% The commands below might be convenient. For example, you probably
% prefer to write $\powset[2]{V}$ for the set of $2$-element subsets
% of $V$, rather than writing $\mathcal{P}_2(V)$.
% Notice that you can easily define your own commands like this.
% Caveat: Some of these commands need to be properly "guarded" when
% they occur in subscripts or superscripts. So you should not write
% $K_\CC$, but rather $K_{\CC}$.
\newcommand{\CC}{\mathbb{C}} % complex numbers
\newcommand{\RR}{\mathbb{R}} % real numbers
\newcommand{\QQ}{\mathbb{Q}} % rational numbers
\newcommand{\NN}{\mathbb{N}} % nonnegative integers
\newcommand{\PP}{\mathbb{P}} % positive integers
\newcommand{\Z}[1]{\mathbb{Z}/#1\mathbb{Z}} % integers modulo k
                                            % (syntax: "\Z{k}")
\newcommand{\ZZ}{\mathbb{Z}} % integers
\newcommand{\id}{\operatorname{id}} % identity map
\newcommand{\lcm}{\operatorname{lcm}}
% Lowest common multiple. For historical reasons, LaTeX has a \gcd
% command built in, but not an \lcm command. The preceding line
% rectifies that.
\newcommand{\set}[1]{\left\{ #1 \right\}}
% $\set{...}$ compiles to {...} (set-brackets).
\newcommand{\abs}[1]{\left| #1 \right|}
% $\abs{...}$ compiles to |...| (absolute value, or size of a set).
\newcommand{\tup}[1]{\left( #1 \right)}
% $\tup{...}$ compiles to (...) (parentheses, or tuple-brackets).
\newcommand{\ive}[1]{\left[ #1 \right]}
% $\ive{...}$ compiles to [...] (Iverson bracket, aka truth value; also, set of first n integers).
\newcommand{\floor}[1]{\left\lfloor #1 \right\rfloor}
% $\floor{...}$ compiles to |_..._| (floor function).
\newcommand{\underbrack}[2]{\underbrace{#1}_{\substack{#2}}}
% $\underbrack{...1}{...2}$ yields
% $\underbrace{...1}_{\substack{...2}}$. This is useful for doing
% local rewriting transformations on mathematical expressions with
% justifications. For example, try this out:
% $ \underbrack{(a+b)^2}{= a^2 + 2ab + b^2 \\ \text{(by the binomial formula)}} $
\newcommand{\powset}[2][]{\ifthenelse{\equal{#2}{}}{\mathcal{P}\left(#1\right)}{\mathcal{P}_{#1}\left(#2\right)}}
% $\powset[k]{S}$ stands for the set of all $k$-element subsets of
% $S$. The argument $k$ is optional, and if not provided, the result
% is the whole powerset of $S$.
\newcommand{\horrule}[1]{\rule{\linewidth}{#1}} % Create horizontal rule command with 1 argument of height
\newcommand{\nnn}{\nonumber\\} % Don't number this line in an "align" environment, and move on to the next line.

%----------------------------------------------------------------------------------------
%	MAKING SUMMATION SIGNS ALWAYS PUT THEIR BOUNDS ABOVE AND BELOW
%	THE SIGN
%----------------------------------------------------------------------------------------
% The following are hacks to ensure that sums (such as
% $\sum_{k=1}^n k$) always put their bounds (i.e., the $k=1$ and the
% $n$) underneath and above the sign, as opposed to on its right.
% Same for products (\prod), set unions (\bigcup) and set
% intersections (\bigcap). Remove the 8 lines below if you do not want
% this behavior.
\let\sumnonlimits\sum
\let\prodnonlimits\prod
\let\cupnonlimits\bigcup
\let\capnonlimits\bigcap
\renewcommand{\sum}{\sumnonlimits\limits}
\renewcommand{\prod}{\prodnonlimits\limits}
\renewcommand{\bigcup}{\cupnonlimits\limits}
\renewcommand{\bigcap}{\capnonlimits\limits}

%----------------------------------------------------------------------------------------
%	ENVIRONMENTS
%----------------------------------------------------------------------------------------
% The incantations below define how theorem environments
% (\begin{theorem} ... \end{theorem}) and their likes will look like.
\newtheoremstyle{plainsl}% <name>
  {8pt plus 2pt minus 4pt}% <Space above>
  {8pt plus 2pt minus 4pt}% <Space below>
  {\slshape}% <Body font>
  {0pt}% <Indent amount>
  {\bfseries}% <Theorem head font>
  {.}% <Punctuation after theorem head>
  {5pt plus 1pt minus 1pt}% <Space after theorem headi>
  {}% <Theorem head spec (can be left empty, meaning `normal')>

% Environments which make the text inside them slanted:
\theoremstyle{plainsl}
  \newtheorem{theorem}{Theorem}[section]
  \newtheorem{proposition}[theorem]{Proposition}
  \newtheorem{lemma}[theorem]{Lemma}
  \newtheorem{corollary}[theorem]{Corollary}
  \newtheorem{conjecture}[theorem]{Conjecture}
% Environments that don't:
\theoremstyle{definition}
  \newtheorem{definition}[theorem]{Definition}
  \newtheorem{example}[theorem]{Example}
  \newtheorem{exercise}[theorem]{Exercise}
  \newtheorem{examples}[theorem]{Examples}
  \newtheorem{algorithm}[theorem]{Algorithm}
  \newtheorem{question}[theorem]{Question}
 \theoremstyle{remark}
  \newtheorem{remark}[theorem]{Remark}
\newenvironment{statement}{\begin{quote}}{\end{quote}}
\newenvironment{fineprint}{\begin{small}}{\end{small}}

%----------------------------------------------------------------------------------------
%	METADATA
%----------------------------------------------------------------------------------------
\newcommand{\myname}{Darij Grinberg} % ENTER YOUR NAME HERE
\newcommand{\myid}{00000000} % ENTER YOUR UMN ID HERE
\newcommand{\mymail}{dgrinber@umn.edu} % ENTER YOUR EMAIL HERE
\newcommand{\psetnumber}{1} % ENTER THE NUMBER OF THIS PSET HERE

%----------------------------------------------------------------------------------------
%	HEADER AND FOOTER
%----------------------------------------------------------------------------------------
\ihead{Solutions to midterm \#\psetnumber} % Page header left
\ohead{page \thepage\ of \pageref{LastPage}} % Page header right
\ifoot{\myname, \myid} % left footer
\ofoot{\mymail} % right footer

%----------------------------------------------------------------------------------------
%	TITLE SECTION
%----------------------------------------------------------------------------------------
\title{	
\normalfont \normalsize 
\textsc{University of Minnesota, School of Mathematics} \\ [25pt] % Your university, school and/or department name(s)
\horrule{0.5pt} \\[0.4cm] % Thin top horizontal rule
\huge Math 4281: Introduction to Modern Algebra, \\
Spring 2019:
Midterm \psetnumber\\% The assignment title
\horrule{2pt} \\[0.5cm] % Thick bottom horizontal rule
}
\author{\myname}

\begin{document}

\maketitle % Print the title

\begin{center} % Delete this if you want to save space!
{\large due date: \textbf{Wednesday, 6 March 2019} at the beginning of class, \\
or before that through Canvas.

\textbf{No collaboration allowed} -- this is a midterm.

Please solve \textbf{at most 3 of the 6 exercises}!}
\end{center}

%----------------------------------------------------------------------------------------
%	EXERCISE 1
%----------------------------------------------------------------------------------------
\horrule{0.3pt} \\[0.4cm]

\section{Exercise 1: Gcds and lcms together}

\subsection{Problem}

Let $a, b, c$ be three integers.

\begin{enumerate}

\item[\textbf{(a)}]
Prove that
$\gcd\tup{a, \lcm\tup{b, c}}
= \lcm\tup{\gcd\tup{a, b}, \gcd\tup{a, c}}$.

\item[\textbf{(b)}]
Prove that
$\lcm\tup{a, \gcd\tup{b, c}}
= \gcd\tup{\lcm\tup{a, b}, \lcm\tup{a, c}}$.

\end{enumerate}

\subsection{Solution}

See
\href{http://www-users.math.umn.edu/~dgrinber/19s/notes.pdf}{the class notes},
where this is Exercise 2.13.11.

%----------------------------------------------------------------------------------------
%	EXERCISE 2
%----------------------------------------------------------------------------------------
\horrule{0.3pt} \\[0.4cm]

\section{Exercise 2: $p$-adic valuations of rationals}

\subsection{Problem}

Fix a prime $p$.
For each nonzero rational number $r$, define the
\textit{extended $p$-adic valuation} $w_p\tup{r}$ as
follows:
We write $r$ in the form $r = a/b$
for two nonzero integers $a$ and $b$, and set
$w_p\tup{r} = v_p\tup{a} - v_p\tup{b}$.
(It also makes sense to set $w_p\tup{0} = \infty$,
but we shall not concern ourselves with this border case
in this exercise.)

\begin{enumerate}

\item[\textbf{(a)}]
Prove that this is well-defined -- i.e., that
$w_p\tup{r}$ does not depend on the precise choice of
$a$ and $b$ satisfying $r = a/b$.

\item[\textbf{(b)}]
Prove that $w_p\tup{n} = v_p\tup{n}$ for each nonzero
integer $n$.

\item[\textbf{(c)}]
Prove that $w_p\tup{ab} = w_p\tup{a} + w_p\tup{b}$ for
any two nonzero rational numbers $a$ and $b$.

\item[\textbf{(d)}]
Prove that $w_p\tup{a+b} \geq \min\set{w_p\tup{a}, w_p\tup{b}}$
for any two nonzero rational numbers $a$ and $b$
if $a+b \neq 0$.

\end{enumerate}

\subsection{Remark}

The claim of part \textbf{(d)} has a curious (and, if you are a
number theorist, important) consequence:
Each prime $p$ can be used to define a ``distance function'' on $\QQ$
that is very different from the usual distance function
($\tup{a,b} \mapsto \abs{a-b}$):
Namely, for any two rational numbers $a$ and $b$, we define
the \textit{$p$-adic distance} $d_p\tup{a, b}$ between $a$ and $b$ by
\[
d_p\tup{a, b}
= \begin{cases}
     p^{w_p\tup{a-b}}, & \text{if $a-b \neq 0$}; \\
     0,                & \text{if $a-b = 0$}.
  \end{cases}
\]
For instance, $d_3\tup{5, 1/2} = 3^{w_3\tup{5 - 1/2}} = 3^2$,
since $w_3\tup{5 - 1/2} = w_3\tup{9/2} = v_3\tup{9} - v_3\tup{2}
= 2 - 0 = 2$.

The $p$-adic distance deserves the name ``distance'', as it
does satisfy the triangle inequality:
\[
d_p\tup{a, c} \leq d_p\tup{a, b} + d_p\tup{b, c}
\qquad \text{for any $a, b, c \in \QQ$} .
\]
Actually, the following stronger inequality (called
\textit{ultrametric triangle inequality}) holds:
\[
d_p\tup{a, c} \leq \max\set{d_p\tup{a, b}, d_p\tup{b, c}}
\qquad \text{for any $a, b, c \in \QQ$} .
\]
Indeed, this follows easily from part \textbf{(d)} of the
exercise (applied to $a-b$ and $b-c$ instead of $a$ and $b$).

You might remember that the real numbers were defined as the
completion of the rational numbers with respect to the usual
distance (i.e., a real number is actually an equivalence class
of Cauchy sequences defined with respect to the usual distance).
Similarly one can consider the completion of the rational numbers
with respect to the $p$-adic distance (i.e., again consider
Cauchy sequences, but this time the distances are replaced by
the $p$-adic distances).
This leads to \href{https://en.wikipedia.org/wiki/P-adic_number}{the $p$-adic numbers}.
See \cite{Gouvea97} for an elementary introduction to this
subject.

\subsection{Solution}

See
\href{http://www-users.math.umn.edu/~dgrinber/19s/notes.pdf}{the class notes},
where this is Exercise 3.4.1.

%----------------------------------------------------------------------------------------
%	EXERCISE 3
%----------------------------------------------------------------------------------------
\horrule{0.3pt} \\[0.4cm]

\section{Exercise 3: How often does a prime divide a factorial?}

\subsection{Problem}

In this exercise, we shall use
\href{https://en.wikipedia.org/wiki/Iverson_bracket}{the \textit{Iverson bracket notation}}:
If $\mathcal{A}$ is any statement, then $\ive{\mathcal{A}}$
stands for the integer
$\begin{cases}
1, & \text{if $\mathcal{A}$ is true;} \\
0, & \text{if $\mathcal{A}$ is false}
\end{cases}$
(which is also known as the \textit{truth value} of
$\mathcal{A}$).
For instance, $\ive{1+1 = 2} = 1$ and $\ive{1+1 = 1} = 0$.

\begin{enumerate}

\item[\textbf{(a)}]
Prove that $n // k = \sum_{i=1}^n \ive{k \mid i}$
for any $n \in \NN$ and any positive integer $k$.

\item[\textbf{(b)}]
Prove that $v_p \tup{n} = \sum_{i \geq 1} \ive{p^i \mid n}$
for any prime $p$ and any nonzero integer $n$.
Here, the sum $\sum_{i \geq 1} \ive{p^i \mid n}$ is
a sum over all positive integers; but it is
well-defined, since it has only finitely many nonzero
addends.

\item[\textbf{(c)}]
Prove that
$v_p \tup{n!} = \sum_{i \geq 1} n // p^i$ for any
prime $p$ and any $n \in \NN$.

\end{enumerate}

\subsection{Solution}

See
\href{http://www-users.math.umn.edu/~dgrinber/19s/notes.pdf}{the class notes},
where this is the first three parts of Exercise 2.17.2.

%----------------------------------------------------------------------------------------
%	EXERCISE 4
%----------------------------------------------------------------------------------------
\horrule{0.3pt} \\[0.4cm]

\section{Exercise 4: Wilson with a twist}

\subsection{Problem}

Let $p$ be a prime.
Prove that
\[
\tup{p-1}! \equiv p-1
\mod 1 + 2 + \cdots + \tup{p-1} .
\]

\subsection{Solution}

See
\href{http://www-users.math.umn.edu/~dgrinber/19s/notes.pdf}{the class notes},
where this is Exercise 2.15.4.

% This exercise is also more or less clearly
% equivalent to §2.1, problem 51 in Niven/Zuckerman/Montgomery
% (see kim-NT/sol3.pdf §2).

%----------------------------------------------------------------------------------------
%	EXERCISE 5
%----------------------------------------------------------------------------------------
\horrule{0.3pt} \\[0.4cm]

\section{Exercise 5: gcds in exponents}

\subsection{Problem}

Let $a$ be an integer.
Let $n, m \in \NN$.
Prove that
\[
\gcd\tup{a^n - 1, a^m - 1}
= \abs{a^{\gcd\tup{n, m}} - 1}.
\]

[\textbf{Hint:} Strong induction. First show that
$a^n - 1 \equiv a^m - 1 \mod a^{n-m} - 1$ if $n \geq m$.]

\subsection{Solution}

See
\href{http://www-users.math.umn.edu/~dgrinber/19s/notes.pdf}{the class notes},
where this is Exercise 2.9.3 (with slightly changed notations).

%----------------------------------------------------------------------------------------
%	EXERCISE 6
%----------------------------------------------------------------------------------------
\horrule{0.3pt} \\[0.4cm]

\section{Exercise 6: Remainder arithmetic}

\subsection{Problem}

Let $u$ and $v$ be two integers.
Let $n$ be a positive integer.
Prove that
\[
u \% n + v \% n - \tup{u+v} \% n \in \set{0, n} .
\]

\subsection{Solution}

See
\href{http://www-users.math.umn.edu/~dgrinber/19s/notes.pdf}{the class notes},
where this is Exercise 2.6.3 \textbf{(a)}.

\begin{thebibliography}{99999999}                                                                                         %

% Feel free to add your sources -- or copy some from the source code
% of the class notes ( http://www-users.math.umn.edu/~dgrinber/19s/notes.tex ).

\bibitem[Gouvea97]{Gouvea97}Fernando Q. Gouvea,
\textit{$p$-adic numbers: An introduction},
2nd edition, Springer 1997.
\url{https://www.springer.com/us/book/9783540629115} or
\url{https://link.springer.com/book/10.1007/F978-3-642-59058-0} .

% \bibitem[GrKnPa94]{GKP}Ronald L. Graham, Donald E. Knuth, Oren Patashnik,
% \textit{Concrete Mathematics, Second Edition}, Addison-Wesley 1994.\\
% See \url{https://www-cs-faculty.stanford.edu/~knuth/gkp.html} for errata.

% \bibitem[Grinbe19]{detnotes}Darij Grinberg,
% \textit{Notes on the combinatorial fundamentals of algebra},
% 10 January 2019. \\
% \url{http://www.cip.ifi.lmu.de/~grinberg/primes2015/sols.pdf}
% \\
% The numbering of theorems and formulas in this link might shift
% when the project gets updated; for a ``frozen'' version whose
% numbering is guaranteed to match that in the citations above, see
% \url{https://github.com/darijgr/detnotes/releases/tag/2019-01-10} .

\end{thebibliography}

\end{document}

