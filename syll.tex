\documentclass[numbers=enddot,12pt,final,onecolumn,notitlepage]{scrartcl}%
\usepackage[headsepline,footsepline,manualmark]{scrlayer-scrpage}
\usepackage[all,cmtip]{xy}
\usepackage{amsfonts}
\usepackage{amssymb}
\usepackage{framed}
\usepackage{amsmath}
\usepackage{comment}
\usepackage{color}
\usepackage{hyperref}
\usepackage[sc]{mathpazo}
\usepackage[T1]{fontenc}
\usepackage{amsthm}
%TCIDATA{OutputFilter=latex2.dll}
%TCIDATA{Version=5.50.0.2960}
%TCIDATA{LastRevised=Tuesday, May 07, 2019 21:32:56}
%TCIDATA{SuppressPackageManagement}
%TCIDATA{<META NAME="GraphicsSave" CONTENT="32">}
%TCIDATA{<META NAME="SaveForMode" CONTENT="1">}
%TCIDATA{BibliographyScheme=Manual}
%BeginMSIPreambleData
\providecommand{\U}[1]{\protect\rule{.1in}{.1in}}
%EndMSIPreambleData
\theoremstyle{definition}
\newtheorem{theo}{Theorem}[section]
\newenvironment{theorem}[1][]
{\begin{theo}[#1]\begin{leftbar}}
{\end{leftbar}\end{theo}}
\newtheorem{lem}[theo]{Lemma}
\newenvironment{lemma}[1][]
{\begin{lem}[#1]\begin{leftbar}}
{\end{leftbar}\end{lem}}
\newtheorem{prop}[theo]{Proposition}
\newenvironment{proposition}[1][]
{\begin{prop}[#1]\begin{leftbar}}
{\end{leftbar}\end{prop}}
\newtheorem{defi}[theo]{Definition}
\newenvironment{definition}[1][]
{\begin{defi}[#1]\begin{leftbar}}
{\end{leftbar}\end{defi}}
\newtheorem{remk}[theo]{Remark}
\newenvironment{remark}[1][]
{\begin{remk}[#1]\begin{leftbar}}
{\end{leftbar}\end{remk}}
\newtheorem{coro}[theo]{Corollary}
\newenvironment{corollary}[1][]
{\begin{coro}[#1]\begin{leftbar}}
{\end{leftbar}\end{coro}}
\newtheorem{conv}[theo]{Convention}
\newenvironment{condition}[1][]
{\begin{conv}[#1]\begin{leftbar}}
{\end{leftbar}\end{conv}}
\newtheorem{quest}[theo]{Question}
\newenvironment{algorithm}[1][]
{\begin{quest}[#1]\begin{leftbar}}
{\end{leftbar}\end{quest}}
\newtheorem{warn}[theo]{Warning}
\newenvironment{conclusion}[1][]
{\begin{warn}[#1]\begin{leftbar}}
{\end{leftbar}\end{warn}}
\newtheorem{conj}[theo]{Conjecture}
\newenvironment{conjecture}[1][]
{\begin{conj}[#1]\begin{leftbar}}
{\end{leftbar}\end{conj}}
\newtheorem{exmp}[theo]{Example}
\newenvironment{example}[1][]
{\begin{exmp}[#1]\begin{leftbar}}
{\end{leftbar}\end{exmp}}
\newenvironment{statement}{\begin{quote}}{\end{quote}}
\newenvironment{fineprint}{\begin{small}}{\end{small}}
\iffalse
\newenvironment{proof}[1][Proof]{\noindent\textbf{#1.} }{\ \rule{0.5em}{0.5em}}
\fi
\newenvironment{verlong}{}{}
\newenvironment{vershort}{}{}
\newenvironment{noncompile}{}{}
\excludecomment{verlong}
\includecomment{vershort}
\excludecomment{noncompile}
\newcommand{\kk}{\mathbf{k}}
\newcommand{\id}{\operatorname{id}}
\newcommand{\ev}{\operatorname{ev}}
\newcommand{\Comp}{\operatorname{Comp}}
\newcommand{\bk}{\mathbf{k}}
\newcommand{\Nplus}{\mathbb{N}_{+}}
\newcommand{\NN}{\mathbb{N}}
\let\sumnonlimits\sum
\let\prodnonlimits\prod
\renewcommand{\sum}{\sumnonlimits\limits}
\renewcommand{\prod}{\prodnonlimits\limits}
\setlength\textheight{22.5cm}
\setlength\textwidth{15cm}
\ihead{Math 4281, Spring 2019, Darij Grinberg: Syllabus}
\ohead{\today}
\begin{document}

\begin{center}
Math 4281, Spring 2019: \textbf{Introduction to Modern Algebra}

-- \textit{Syllabus} --

Darij Grinberg, VinH 203B, \texttt{dgrinber@umn.edu}

last update: \today

\end{center}

\bigskip

\#\#\#\#\#\#\#\#\#\#\#\#\#\#\#\#\#\#\#\#\#\#\#\#\#\#\#\#\#\#\#\#\#\#\#\#\#\#\#\#\#\#\#\#\#\#\#\#\#

\textbf{WARNING!} You are reading the syllabus of a class that lies in the
past. If you're looking for the current iteration of Math 4281, you are in the
wrong place.

\#\#\#\#\#\#\#\#\#\#\#\#\#\#\#\#\#\#\#\#\#\#\#\#\#\#\#\#\#\#\#\#\#\#\#\#\#\#\#\#\#\#\#\#\#\#\#\#\#

\bigskip

\section{Time \& Place}%

\begin{tabular}
[c]{|ll|}\hline
\textbf{Lectures:} & \textbf{Section 001:} MWF 9:05--9:55,
\href{http://campusmaps.umn.edu/vincent-hall}{Vincent Hall} 211.\\
& \textbf{Section 002:} MWF 11:15--12:05,
\href{http://campusmaps.umn.edu/vincent-hall}{Vincent Hall} 211.\\\hline
\textbf{Office hours:} & Monday 14:35--15:35.\\
& Tuesday 10:00--11:00.\\
& Tuesday 12:00--13:00.\\
& Friday 14:30--15:00 (half an hour).\\\hline
\end{tabular}


\begin{noncompile}
\vspace{0.1cm}Office hours are usually in
\href{http://campusmaps.umn.edu/lind-hall}{Lind Hall 4th floor (center of the
hall)}. If I am not there, try
\href{http://campusmaps.umn.edu/vincent-hall}{Vincent Hall 203B}. Otherwise,
by appointment (email).
\end{noncompile}

\bigskip

This class has \href{http://www-users.math.umn.edu/~dgrinber/19s/}{a website:
http://www-users.math.umn.edu/\symbol{126}dgrinber/19s/} and two Canvas boards:

\qquad\href{https://canvas.umn.edu/courses/99199}{Section 001:
https://canvas.umn.edu/courses/99199}.

\qquad\href{Section 002: https://canvas.umn.edu/courses/99188}{Section 002:
https://canvas.umn.edu/courses/99188}.

\bigskip

\textbf{Homework} will usually be due on \textbf{Wednesday} (sometimes weekly,
sometimes every other week) \textbf{at the beginning of class}. You can also
submit your homework electronically via Canvas (see above) \textbf{provided
that the problem set is submitted as 1 single PDF file}. See \textquotedblleft
Grading\textquotedblright\ and \textquotedblleft Coursework\textquotedblright%
\ below for details.

\bigskip

\textbf{Midterms} (3 in total) are like homework, but they count for more, and
collaboration is not allowed (see below for details). There will be \textbf{no
final exam}.

\section{Requirements}

This is a pure mathematics class and relies heavily on proofs. You have to
feel at home reading and writing mathematical proofs. You can catch up on this from:

\begin{itemize}
\item {}[LeLeMe] Eric Lehman, F. Thomson Leighton, Albert R. Meyer,
\textit{Mathematics for Computer Science}, \newline%
\url{https://courses.csail.mit.edu/6.042/spring18/mcs.pdf} . (You should know
the material from Chapters 1--5, minus the CS parts.)

\item {}[Hammack] Richard Hammack, \textit{Book of Proof},\newline\url{http://www.people.vcu.edu/~rhammack/BookOfProof/}

\item {}[Day] Martin V. Day, \textit{An Introduction to Proofs and the
Mathematical Vernacular},\newline%
\url{https://www.math.vt.edu/people/day/ProofsBook/IPaMV.pdf} .
\end{itemize}

You should also know some linear algebra: Gaussian elimination, linear
independence, vector spaces, dimension, rank of matrices. The more you know,
the better. One of the best texts to learn linear algebra from is:

\begin{itemize}
\item {}[Hefferon] Jim Hefferon, \textit{Linear Algebra}, 2017,\newline\url{http://joshua.smcvt.edu/linearalgebra/}
\end{itemize}

\noindent if you have the time for it (it is long and thorough). If not, make
sure you know at least the topics named above. We won't start using linear
algebra until a few weeks into this class.

\section{Texts}

\subsection{required:}

\begin{noncompile}
I will write lecture notes accompanying the class:%
\[
\text{\url{http://www.cip.ifi.lmu.de/~grinberg/t/19s/notes.pdf}}%
\]
(just started; they'll grow). I am not sure how detailed they will be; I might
also supplement them with other freely available sources. Everything
examinable will be in these notes (and homework).
\end{noncompile}

There is no required textbook if you attend class. I won't follow a text
directly anyway. I will post lecture notes on the
\href{http://www-users.math.umn.edu/~dgrinber/19s/}{class website}. These will
not be as polished as a book, but will contain everything needed to do
homeworks and midterms.

\begin{noncompile}
class notes (= scans of what I'm projecting on the screen in class)
\end{noncompile}

\subsection{recommended:}

Here are some good texts I know:\footnote{The bracketed names (such as
\textquotedblleft\lbrack Goodman]\textquotedblright) are the abbreviations by
which I will refer to the texts in class.}

\begin{itemize}
\item {}[Goodman]
\href{http://homepage.divms.uiowa.edu/~goodman/algebrabook.dir/algebrabook.html}{Frederick
M. Goodman, \textit{Algebra: Abstract and Concrete}, edition 2.6}.\newline
This is a long book going beyond what we will do in this class. We won't
follow it directly either, but it has a lot of the material we care about.

\item {}[Siksek]
\href{http://homepages.warwick.ac.uk/staff/S.Siksek/teaching/aa/aanotes.pdf}{Samir
Siksek, \textit{Introduction to Abstract Algebra}, 2015}.\newline This is
probably the most amusing (though occasionally overdoing the goofiness) book
on abstract algebra I've ever seen. It is also really good at motivating definitions.

\item {}[Pinter]
\href{https://www.amazon.com/Book-Abstract-Algebra-Second-Mathematics/dp/0486474178}{Charles
C. Pinter, \textit{A book of abstract algebra}, 2nd edition, Dover
2010}.\newline The link goes to an ebook version that doesn't look very good,
but \href{http://store.doverpublications.com/0486474178.html}{you can get the
original for less than \$20}.

\item {}[Armstrong]
\href{http://www.math.miami.edu/~armstrong/561fa18.php}{Drew Armstrong,
\textit{Abstract Algebra I}, 2018}.\newline(Click on
\href{http://www.math.miami.edu/~armstrong/561fa18/561fa18notes.pdf}{\textquotedblleft
Course Notes\textquotedblright} for the main text.) This is new and I haven't
had much experience with it, but Armstrong is a good expositor. Note that this
only goes until group theory.
\end{itemize}

On specific topics:

\begin{itemize}
\item {}[Conrad-Gauss]
\href{http://www.math.uconn.edu/~kconrad/blurbs/ugradnumthy/Zinotes.pdf}{Keith
Conrad, \textit{The Gaussian integers}}.\newline This is a topic we'll spend
some time with early on in this course. And Conrad is a really good writer;
check out \href{http://www.math.uconn.edu/~kconrad/blurbs/}{his expository
papers} for various other pieces of algebra.

\item {}[Strickland]
\href{https://neil-strickland.staff.shef.ac.uk/courses/MAS201/MAS201.pdf}{Neil
Strickland, \textit{Linear mathematics for applications}}.\newline Would you
have guessed by the title that these lecture notes have one of the most
rigorous treatments of determinants I've seen in the linear algebra literature?

\item {}[Gallier-RSA]
\href{https://www.cis.upenn.edu/~jean/RSA-primality-testing.pdf}{Jean Gallier,
Jocelyn Quaintance, \textit{Notes on Primality Testing And Public Key
Cryptography, Part 1}}.\newline This contains most of the elementary number
theory we'll be going through (modular arithmetic, RSA, groups), plus a lot
that we won't (primality checking).
\end{itemize}

\begin{noncompile}
You don't have to buy any of these, but I assume any of them would look nice
on your shelf. (You can download [Bogart] and [GKP] from the URLs above, and
you might be able to get [Aigner] for free from the UMN network. Of course,
there are ways to get the other two as well if you know your way around the
internet\footnote{Search for \textquotedblleft reddit
textbooks\textquotedblright, for example.}.)
\end{noncompile}

\begin{fineprint}
For the sake of completeness, here are links to some old editions of this
class: \href{http://www-users.math.umn.edu/~kim00657/teaching/18f4281/}{Fall
2018}, \href{http://mchmutov.net/teaching/4281teaching.html}{Spring 2018},
\href{http://www-users.math.umn.edu/~musiker/4281/}{Fall 2017},
\href{http://www-users.math.umn.edu/~musiker/4281/}{Spring 2017}. Note that
they were structured differently from mine and used different texts.
\end{fineprint}

\section{Contact}

All material regarding the course (including homework) can be found on
\[
\text{\href{http://www-users.math.umn.edu/~dgrinber/19s/}{http://www-users.math.umn.edu/\symbol{126}%
dgrinber/19s/}}%
\]


The best way to reach me is by email to \texttt{dgrinber@umn.edu} .

After I leave UMN (in Summer), the best way to reach me will be by email to
\texttt{darijgrinberg@gmail.com} , and the class materials will migrate to
\url{http://www.cip.ifi.lmu.de/~grinberg/t/19s/} .

\section{Topics (tentative)}

This is still far from finished and decided.

Topics marked with an * \textbf{may} be excluded. Topics marked with an **
probably \textbf{will} be excluded.

\begin{enumerate}
\item Introduction and motivating questions.

\begin{enumerate}
\item $n=x^{2}+y^{2}$.

\item Algebraic vs. transcendental numbers; sums of algebraics.

\item \href{https://en.wikipedia.org/wiki/Shamir%27s_Secret_Sharing}{Shamir's
secret sharing scheme} and the need for finite fields.

\item * Angle trisection, cube duplication, circle squaring.

\item * Solving degree-$3$ equations; higher orders.
\end{enumerate}

\item Elementary number theory.

\begin{enumerate}
\item Divisibility.

\item Congruences mod $n$.

\item Division with remainder.

\item gcd, Bezout, Euclidean algorithm.

\item Primes and prime factorization.

\item Modular inverses \& cancellation mod $n$.

\item Fermat's little theorem.

\item Euler's $\phi$-function and its properties.

\item Chinese Remainder Theorem.
\end{enumerate}

\item $\mathbb{Z}/n$.

\begin{enumerate}
\item Equivalence relations.

\item Equivalence classes and quotient sets.

\item $\mathbb{Z}/n$.

\item RSA and applications of the Chinese Remainder Theorem.

\item Primitive roots (no proofs).
\end{enumerate}

\item Complex numbers and Gaussian integers.

\begin{enumerate}
\item Complex numbers (the very basics).

\item Gaussian integers (follow Keith Conrad).

\item $n=x^{2}+y^{2}$ (follow Keith Conrad).

\item Finish with discussion of other quadratic and higher-order number rings.
\end{enumerate}

\item Rings and fields I.

\begin{enumerate}
\item Define rings, commutative rings and fields.

\item Examples (but no structure or homomorphisms). Include $\mathbb{Z}/p$,
$\mathbb{Z}\left[  i\right]  $, dual numbers, etc. Also $\mathbb{Z}\left[
\sqrt[3]{2}\right]  $, but note that just adding $\sqrt[3]{2}$ does not work.

\item General properties (e.g., finite sums).
\end{enumerate}

\item Linear algebra.

\begin{enumerate}
\item Modules and vector spaces.

\item Linear algebra (crash course, following Hefferon? Charlier? Artin?
whoever does it most slickly).

\item Direct sums/products of vector spaces.

\item Quotient vector spaces.

\item Linear algebra over $\mathbb{Z}/p$.

\item $\operatorname*{XOR}$ as vector addition over $\mathbb{Z}/2$.

\item Solve \href{https://en.wikipedia.org/wiki/Lights_Out_(game)}{lights-out}
using $\mathbb{Z}/2$-linear algebra.

\item What works and what fails over the base ring $\mathbb{Z}$ ? (Leave some
harder stuff unproven.)

\item * $\mathbb{Z}/26$ and affine ciphers as $2\times2$-matrices (follow Conrad).

\item The Smith normal form over $\mathbb{Z}$, proven by merging Gaussian
elimination with the Euclidean algorithm.
\end{enumerate}

\item Permutations and determinants.

\begin{enumerate}
\item Permutation basics (follow [detnotes]).

\item Lengths and signs (follow [detnotes]).

\item Cycles and transpositions (follow [detnotes]).

\item Cycle decompositions (follow Loeh?).

\item Determinants.

\item Properties of determinants.

\item $\det\left(  AB\right)  $.

\item Adjugate matrices.
\end{enumerate}

\item Groups.

\begin{enumerate}
\item Definition of a group.

\item Subgroups.

\item Homomorphisms.

\item Direct products.
\end{enumerate}

\item Group actions. (Include necklace counting.)

\item Quotient groups and homomorphism theorems.

\item Rings.

\begin{enumerate}
\item Recalling the definition.

\item Subrings.

\item Ring homomorphisms.

\item Ideals and quotient rings.
\end{enumerate}

\item Polynomials and their rings.

\begin{enumerate}
\item Define formal power series.

\item Define polynomials.

\item Polynomials over a field as vector space.

\item Division with remainder modulo monic (or invertible-LT) polynomial.

\item Use as generating functions.

\item Lucas's theorem reproved.

\item Companion matrices.

\item Cayley-Hamilton theorem.

\item Application: sums/products of algebraic integers.
\end{enumerate}

\item Fields.

\item Field extensions and angle trisection.

\item Finite fields. (Prove existence by counting irreducible polys.)

\item * Properties of rings (PID, UFD, etc.).

\item * Symmetric polynomials and the Fundamental Theorem.

\item * Galois theory introduction. (Apply to deg-$3$ maybe.)

\item * Modules and their basic applications. (Apply to algebraic integers again?)
\end{enumerate}

\section{Schedule (tentative)}

\begin{noncompile}
This is my \textbf{best guess} at this point (and \textquotedblleft
best\textquotedblright\ does not mean \textquotedblleft good\textquotedblright%
). Even the midterm dates are \textbf{not set in stone}. Topics inside
parentheses are the most likely to be sacrificed if I run out of time.

Asterisks (*) mean days when I'll be away; class will be substituted.
\end{noncompile}

\textbf{The due dates with question marks are not set in stone.}%

\[%
\begin{tabular}
[c]{||c||c||c||}\hline\hline
\textbf{week} & \textbf{material} & \textbf{due}\\\hline\hline
Jan 23, 25 &  & \\\hline\hline
Jan 28, 30, 1 &  & \\\hline\hline
Feb 4, 6, 8 &  & hw1\\\hline\hline
Feb 11, 13, 15 &  & hw2\\\hline\hline
Feb 18, 20, 22 &  & hw3\\\hline\hline
Feb 25, 27, 1 &  & \\\hline\hline
Mar 4, 6, 8 &  & MT1\\\hline\hline
Mar 11, 13, 15 &  & hw4\\\hline\hline
\textit{Mar 18--24} & \textit{break} & \\\hline\hline
Mar 25, 27, 29 &  & \\\hline\hline
Apr 1, 3, 5 &  & hw5 (Mon)\\\hline\hline
Apr 8, 10, 12 &  & MT2\\\hline\hline
Apr 15, 17, 19 &  & hw6?\\\hline\hline
Apr 22, 24, 26 &  & \\\hline\hline
Apr 29, 1, 3 &  & MT3\\\hline\hline
May 5 &  & \\\hline\hline
\end{tabular}
\ \ \ \ \ \ \ \ \ \ \ \
\]


\bigskip

\section{Grading}

The grade will be computed based on three take-home \textbf{midterms}
(totalling to \textbf{60\%} of the final grade, each giving 20\% of the final
grade) and somewhere between 5 and 10 \textbf{homework sets} (totalling to
\textbf{40\%} of the final grade, but the lowest score will be dropped).

Points will be deducted if your proofs are ambiguously worded or otherwise
hard to understand. Writing readable arguments is part of mathematics; you can
learn this from the references in the \textquotedblleft
Requirements\textquotedblright\ section above and you can practice it on
\href{https://math.stackexchange.com/}{math.stackexchange}.

\section{Coursework}

Collaboration on homework is allowed, as long as:

\begin{itemize}
\item you \textbf{write} up the solutions autonomously and in your own words
(in particular, this means that you have to \textbf{understand} them), and

\item you \textbf{list the names of your collaborators} (there will be no
penalties for collaboration, so you don't lose anything doing this!).
\end{itemize}

On the midterms, you have to \textbf{work alone} (you can \textbf{read}
whatever you want, but you must \textbf{not contact} anyone about the midterm
problems\footnote{It is OK to contact \textbf{me} with questions.}; in
particular, you must \textbf{not ask} them on the internet).

Homework and midterms should be submitted either in person during class, or
via Canvas.

\textbf{If you handwrite your solutions:}

\begin{itemize}
\item Make sure that your writing is legible.

\item If you submit your solutions by email, make sure that your submission is
\textbf{1 single PDF file} for a given homework set (not many 1-page JPGs!).
Double-check that your scans are readable and aren't missing any relevant text
near the margins.
\end{itemize}

\textbf{If you type up your solutions:}

\begin{itemize}
\item Again, make sure that your submission is \textbf{1 single PDF file} for
a given homework set.

\item Double-check that your text doesn't go over the margins (something that
often happens when using LaTeX). If something is not on the page, we cannot
grade it...
\end{itemize}

Calculators and computer algebra systems may be used, but are not necessary
(and you are responsible for any resulting errors).

\begin{noncompile}
For emails, I suggest using \textquotedblleft\lbrack Math 4281] Homework set
\#$n$ submission\textquotedblright\ ($n$ = the number of the problem set) as
the subject line.
\end{noncompile}

\textbf{Late} homework or late midterms are \textbf{not accepted} in any
situation; if you are not finished, submit whatever you have before the
deadline. If you want to update your submission, you can do so (before the
deadline!) by sending me an email that includes the whole updated submission
(not just the parts you want changed).

See also the following university policies:

\begin{itemize}
\item
\href{https://policy.umn.edu/education/gradingtranscripts}{https://policy.umn.edu/education/gradingtranscripts}%


\item
\href{https://policy.umn.edu/research/academicmisconduct}{https://policy.umn.edu/research/academicmisconduct}%

\end{itemize}

\begin{noncompile}
-----

Things I wanted to get to:

(a) angle trisection impossible, $\leftarrow$ This should take 1 extra hour
(once ring extensions and gcds of polynomials are done), but is probably not
worth the hassle.

(b) sum\&prod of algebraic numbers/integers, $\leftarrow$ This requires either
Cayley-Hamilton + modules, or symmetric polynomials. I'd love to get here but
I didn't.

(c) finite fields GF(q), $\leftarrow$ Gregg proved the existence, which is
really nice (use necklace enumeration to prove the existence of irreducible
polynomials of each degree; then use adjunction via polynomial quotients). But
I didn't get here. Uniqueness is prolly not worth it.

(d) the Lagrange interpolation cryptoscheme (Shamir) $\leftarrow$
\textbf{done}.

(e) Solving systems of linear equations over $\mathbb{Z}$ (a simple version of
Smith normal form). (elementary, Euclidean alg). $\leftarrow$ I sketched the
rough idea of this without going into details.

(f) Gaussian integers and writing $n$ as $x^{2}+y^{2}$ (this is in KConrad
except for counting), $\leftarrow$ \textbf{done}.

(g) Fundamental theorem of algebra, proven using Sylow (may be too advanced).
$\leftarrow$ Not tried. I think it isn't worth it, since algebra doesn't use it.

(h) quaternions (what exactly?). $\leftarrow$ I've done the very basics. Not
sure whether there are any good applications to go into short of the
four-squares theorem (which would probably take too long.)

(i) Solving deg-3 and deg-4 equations. (Gregg: \textquotedblleft solving
degree 3 polys and once when I taught it, I got a little further in Galois
theory and contrasted deg 4 polys with Galois group V\_4 vs C\_4 (but did not
discuss solving degree 4 in general)\textquotedblright) $\leftarrow$ not done,
and Vic confirms my suspicion that doing this is fairly orthogonal to most of
abstract algebra.

(j) Insolvability of deg-5 equation, probably with bare hints to its proof.
$\leftarrow$ not even tried; not worth it.
\end{noncompile}


\end{document}