% The LaTeX below is mostly computer-generated (for reasons of speed); don't expect it to be very readable. Sorry.

\documentclass[paper=a4, fontsize=12pt]{scrartcl}%
\usepackage[T1]{fontenc}
\usepackage[english]{babel}
\usepackage{amsmath,amsfonts,amsthm,amssymb}
\usepackage{mathrsfs}
\usepackage{sectsty}
\usepackage{hyperref}
\usepackage{graphicx}
\usepackage{framed}
\usepackage{ifthen}
\usepackage{lastpage}
\usepackage{mathdots}
\usepackage[headsepline,footsepline,manualmark]{scrlayer-scrpage}
\usepackage[height=10in,a4paper,hmargin={1in,0.8in}]{geometry}
\usepackage[usenames,dvipsnames]{xcolor}
\usepackage{tikz}
\usepackage{verbatim}
\usepackage{amsmath}
\usepackage{amsfonts}
\usepackage{amssymb}%
\setcounter{MaxMatrixCols}{30}
%TCIDATA{OutputFilter=latex2.dll}
%TCIDATA{Version=5.50.0.2960}
%TCIDATA{LastRevised=Thursday, May 09, 2019 19:19:28}
%TCIDATA{<META NAME="GraphicsSave" CONTENT="32">}
%TCIDATA{<META NAME="SaveForMode" CONTENT="1">}
%TCIDATA{BibliographyScheme=Manual}
%BeginMSIPreambleData
\providecommand{\U}[1]{\protect\rule{.1in}{.1in}}
%EndMSIPreambleData
\allsectionsfont{\centering \normalfont\scshape}
\setlength\parindent{20pt}
\newcommand{\CC}{\mathbb{C}}
\newcommand{\RR}{\mathbb{R}}
\newcommand{\QQ}{\mathbb{Q}}
\newcommand{\NN}{\mathbb{N}}
\newcommand{\DD}{{\mathbb{D}}}
\newcommand{\PP}{\mathbb{P}}
\newcommand{\KK}{\mathbb{K}}
\newcommand{\LL}{\mathbb{L}}
\newcommand{\MM}{\mathbb{M}}
\newcommand{\FF}{\mathbb{F}}
\newcommand{\Z}[1]{\mathbb{Z}/#1\mathbb{Z}}
\newcommand{\ZZ}{\mathbb{Z}}
\newcommand{\id}{\operatorname{id}}
\newcommand{\op}{\operatorname{op}}
\newcommand{\tildot}{\left. \widetilde{\cdot} \right.}
\newcommand{\lcm}{\operatorname{lcm}}
\newcommand{\set}[1]{\left\{ #1 \right\}}
\newcommand{\abs}[1]{\left| #1 \right|}
\newcommand{\tup}[1]{\left( #1 \right)}
\newcommand{\ive}[1]{\left[ #1 \right]}
\newcommand{\floor}[1]{\left\lfloor #1 \right\rfloor}
\newcommand{\underbrack}[2]{\underbrace{#1}_{\substack{#2}}}
\newcommand{\powset}[2][]{\ifthenelse{\equal{#2}{}}{\mathcal{P}\left(#1\right)}{\mathcal{P}_{#1}\left(#2\right)}}
\newcommand{\mapeq}[1]{\underset{#1}{\equiv}}
\newcommand{\eps}{\varepsilon}
\newcommand{\N}{\operatorname{N}}
\newcommand{\horrule}[1]{\rule{\linewidth}{#1}}
\newcommand{\nnn}{\nonumber\\}
\let\sumnonlimits\sum
\let\prodnonlimits\prod
\let\cupnonlimits\bigcup
\let\capnonlimits\bigcap
\renewcommand{\sum}{\sumnonlimits\limits}
\renewcommand{\prod}{\prodnonlimits\limits}
\renewcommand{\bigcup}{\cupnonlimits\limits}
\renewcommand{\bigcap}{\capnonlimits\limits}
\newtheoremstyle{plainsl}
{8pt plus 2pt minus 4pt}
{8pt plus 2pt minus 4pt}
{\slshape}
{0pt}
{\bfseries}
{.}
{5pt plus 1pt minus 1pt}
{}
\theoremstyle{plainsl}
\newtheorem{theorem}{Theorem}[section]
\newtheorem{proposition}[theorem]{Proposition}
\newtheorem{lemma}[theorem]{Lemma}
\newtheorem{corollary}[theorem]{Corollary}
\newtheorem{conjecture}[theorem]{Conjecture}
\theoremstyle{definition}
\newtheorem{definition}[theorem]{Definition}
\newtheorem{example}[theorem]{Example}
\newtheorem{exercise}[theorem]{Exercise}
\newtheorem{examples}[theorem]{Examples}
\newtheorem{algorithm}[theorem]{Algorithm}
\newtheorem{question}[theorem]{Question}
\theoremstyle{remark}
\newtheorem{remark}[theorem]{Remark}
\newenvironment{statement}{\begin{quote}}{\end{quote}}
\newenvironment{fineprint}{\begin{small}}{\end{small}}
\newcommand{\myname}{Darij Grinberg}
\newcommand{\myid}{00000000}
\newcommand{\mymail}{dgrinber@umn.edu}
\newcommand{\psetnumber}{6}
\ihead{Solutions to homework set \#\psetnumber}
\ohead{page \thepage\ of \pageref{LastPage}}
\ifoot{\myname, \myid}
\ofoot{\mymail}
\begin{document}

\title{ \normalfont {\normalsize \textsc{University of Minnesota, School of
Mathematics} }\\[25pt] \rule{\linewidth}{0.5pt} \\[0.4cm] {\huge Math 4281: Introduction to Modern Algebra, }\\Spring 2019: Homework 6\\\rule{\linewidth}{2pt} \\[0.5cm] }
\author{Darij Grinberg}
\maketitle

%----------------------------------------------------------------------------------------
%	EXERCISE 1
%----------------------------------------------------------------------------------------
\rule{\linewidth}{0.3pt} \\[0.4cm]

\section{Exercise 1: The opposite ring}

Let $\mathbb{K}$ be a ring. We define a new binary operation $\left.
\widetilde{\cdot} \right.  $ on $\mathbb{K}$ by setting
\[
a \left.  \widetilde{\cdot} \right.  b = ba \qquad\text{for all } a, b
\in\mathbb{K} .
\]
(Thus, $\left.  \widetilde{\cdot} \right.  $ is the multiplication of
$\mathbb{K}$, but with the arguments switched.)

\begin{enumerate}
\item[\textbf{(a)}] Prove that the set $\mathbb{K}$, equipped with the
addition $+$, the multiplication $\left.  \widetilde{\cdot} \right.  $, the
zero $0_{\mathbb{K}}$ and the unity $1_{\mathbb{K}}$, is a ring.
\end{enumerate}

\noindent This new ring is called the \textit{opposite ring} of $\mathbb{K}$,
and is denoted by $\mathbb{K}^{\operatorname{op}}$.

Note that the \textbf{sets} $\mathbb{K}$ and $\mathbb{K}^{\operatorname{op}}$
are identical (so a map from $\mathbb{K}$ to $\mathbb{K}$ is the same as a map
from $\mathbb{K}$ to $\mathbb{K}^{\operatorname{op}}$); but the \textbf{rings}
$\mathbb{K}$ and $\mathbb{K}^{\operatorname{op}}$ are generally not the same
(so a ring homomorphism from $\mathbb{K}$ to $\mathbb{K}$ is not the same as a
ring homomorphism from $\mathbb{K}$ to $\mathbb{K}^{\operatorname{op}}$).

\begin{enumerate}
\item[\textbf{(b)}] Prove that the identity map $\operatorname{id} :
\mathbb{K} \to\mathbb{K}$ is a ring isomorphism from $\mathbb{K}$ to
$\mathbb{K}^{\operatorname{op}}$ if and only if $\mathbb{K}$ is commutative.

\item[\textbf{(c)}] Now, assume that $\mathbb{K}$ is the matrix ring
$\mathbb{L}^{n \times n}$ for some commutative ring $\mathbb{L}$ and some $n
\in\mathbb{N}$. Prove that the map
\[
\mathbb{K} \to\mathbb{K}^{\operatorname{op}}, \qquad A \mapsto A^{T}
\]
(where $A^{T}$, as usual, denotes the transpose of a matrix $A$) is a ring isomorphism.
\end{enumerate}

[\textbf{Hint:} In \textbf{(a)}, you only have to check the ring axioms that
have to do with multiplication. Similarly, in \textbf{(b)}, you are free to
check the one axiom relating to multiplication only. In \textbf{(c)}, you can
use \cite[Exercise 6.5]{detnotes} without proof.]

\subsection{Remark}

This exercise gives some examples of rings $\mathbb{K}$ that are isomorphic to
their opposite rings $\mathbb{K}^{\operatorname{op}}$. See
\url{https://mathoverflow.net/questions/64370/} for examples of rings that are not.

\subsection{Solution}

We shall follow the PEMDAS convention for the order of operations, treating
the new multiplication $\left.  \widetilde{\cdot}\right.  $ operation as a
multiplicative operation. Thus, the expression \textquotedblleft$a\left.
\widetilde{\cdot}\right.  b+c\left.  \widetilde{\cdot}\right.  d$%
\textquotedblright\ will mean \textquotedblleft$\left(  a\left.
\widetilde{\cdot}\right.  b\right)  +\left(  c\left.  \widetilde{\cdot
}\right.  d\right)  $\textquotedblright\ rather than \textquotedblleft%
$a\left.  \widetilde{\cdot}\right.  \left(  b+c\right)  \left.
\widetilde{\cdot}\right.  d$\textquotedblright.

We are in the slightly confusing situation of having two different
\textquotedblleft multiplications\textquotedblright\ on one and the same set
$\mathbb{K}$: the original multiplication $\cdot$ of the ring $\mathbb{K}$,
and the new multiplication $\left.  \widetilde{\cdot}\right.  $ of the ring
$\mathbb{K}^{\operatorname*{op}}$ (although we still have not shown that
$\mathbb{K}^{\operatorname*{op}}$ is actually a ring). Let us agree that if
$a,b\in\mathbb{K}$, then the notation \textquotedblleft$ab$\textquotedblright%
\ shall always mean \textquotedblleft$a\cdot b$\textquotedblright\ (that is,
the image of the pair $\left(  a,b\right)  $ under the original multiplication
$\cdot$, not under the new multiplication $\left.  \widetilde{\cdot}\right.  $).

The original ring $\mathbb{K}$ satisfies all eight ring axioms (since it is a ring).

\bigskip

\textbf{(a)} Clearly, the addition $+$ and the multiplication $\left.
\widetilde{\cdot}\right.  $ are binary operations on $\mathbb{K}$, and the
elements $0_{\mathbb{K}}$ and $1_{\mathbb{K}}$ indeed belong to $\mathbb{K}$.
It remains to prove that these two operations and these two elements make
$\mathbb{K}$ into a ring. In order to do so, we need to verify the ring
axioms. These axioms are the following:

\begin{itemize}
\item \textbf{Commutativity of addition:} We have $a+b=b+a$ for all
$a,b\in\mathbb{K}$.

\item \textbf{Associativity of addition:} We have $a+\left(  b+c\right)
=\left(  a+b\right)  +c$ for all $a,b,c\in\mathbb{K}$.

\item \textbf{Neutrality of zero:} We have $a+0_{\mathbb{K}}=0_{\mathbb{K}%
}+a=a$ for all $a\in\mathbb{K}$.

\item \textbf{Existence of additive inverses:} For any $a\in\mathbb{K}$, there
exists an element $a^{\prime}\in\mathbb{K}$ such that $a+a^{\prime}=a^{\prime
}+a=0_{\mathbb{K}}$.

\item \textbf{Associativity of multiplication:} We have $a\left.
\widetilde{\cdot}\right.  \left(  b\left.  \widetilde{\cdot}\right.  c\right)
=\left(  a\left.  \widetilde{\cdot}\right.  b\right)  \left.  \widetilde{\cdot
}\right.  c$ for all $a,b,c\in\mathbb{K}$. (Of course, we \textbf{cannot} use
\textquotedblleft$ab$\textquotedblright\ as an abbreviation for
\textquotedblleft$a\left.  \widetilde{\cdot}\right.  b$\textquotedblright,
since \textquotedblleft$ab$\textquotedblright\ already stands for the
different product $a\cdot b$.)

\item \textbf{Neutrality of one:} We have $a\left.  \widetilde{\cdot}\right.
1_{\mathbb{K}}=1_{\mathbb{K}}\left.  \widetilde{\cdot}\right.  a=a$ for all
$a\in\mathbb{K}$.

\item \textbf{Annihilation:} We have $a\left.  \widetilde{\cdot}\right.
0_{\mathbb{K}}=0_{\mathbb{K}}\left.  \widetilde{\cdot}\right.  a=0_{\mathbb{K}%
}$ for all $a\in\mathbb{K}$.

\item \textbf{Distributivity:} We have%
\[
a\left.  \widetilde{\cdot}\right.  \left(  b+c\right)  =a\left.
\widetilde{\cdot}\right.  b+a\left.  \widetilde{\cdot}\right.
c\ \ \ \ \ \ \ \ \ \ \text{and}\ \ \ \ \ \ \ \ \ \ \left(  a+b\right)  \left.
\widetilde{\cdot}\right.  c=a\left.  \widetilde{\cdot}\right.  c+b\left.
\widetilde{\cdot}\right.  c
\]
for all $a,b,c\in\mathbb{K}$.
\end{itemize}

The first four of these eight axioms do not involve the new multiplication
$\left.  \widetilde{\cdot}\right.  $. Thus, they say exactly the same thing as
the corresponding axioms for the original ring $\mathbb{K}$ (with the original
operations $+$ and $\cdot$). Hence, they are satisfied (since the
corresponding axioms for the original ring $\mathbb{K}$ are satisfied). It
thus remains to prove that the remaining four axioms are satisfied. Let us
check this:

[\textit{Proof of the \textquotedblleft Associativity of
multiplication\textquotedblright\ axiom:} Let $a,b,c\in\mathbb{K}$. We must
prove that $a\left.  \widetilde{\cdot}\right.  \left(  b\left.
\widetilde{\cdot}\right.  c\right)  =\left(  a\left.  \widetilde{\cdot
}\right.  b\right)  \left.  \widetilde{\cdot}\right.  c$.

The definition of the operation $\left.  \widetilde{\cdot}\right.  $ yields
$b\left.  \widetilde{\cdot}\right.  c=cb$ and $a\left.  \widetilde{\cdot
}\right.  b=ba$ and%
\begin{equation}
a\left.  \widetilde{\cdot}\right.  \left(  b\left.  \widetilde{\cdot}\right.
c\right)  =\underbrace{\left(  b\left.  \widetilde{\cdot}\right.  c\right)
}_{=cb}a=\left(  cb\right)  a \label{sol.ring.op.a.ass.1}%
\end{equation}
and%
\begin{equation}
\left(  a\left.  \widetilde{\cdot}\right.  b\right)  \left.  \widetilde{\cdot
}\right.  c=c\underbrace{\left(  a\left.  \widetilde{\cdot}\right.  b\right)
}_{=ba}=c\left(  ba\right)  . \label{sol.ring.op.a.ass.2}%
\end{equation}
But the original ring $\mathbb{K}$ satisfies the \textquotedblleft
Associativity of multiplication\textquotedblright\ axiom (since it is a ring);
thus, $\left(  cb\right)  a=c\left(  ba\right)  $. In other words, the right
hand sides of the two equalities \eqref{sol.ring.op.a.ass.1} and
\eqref{sol.ring.op.a.ass.2} are equal. Thus, their left hand sides are also
equal. In other words, $a\left.  \widetilde{\cdot}\right.  \left(  b\left.
\widetilde{\cdot}\right.  c\right)  =\left(  a\left.  \widetilde{\cdot
}\right.  b\right)  \left.  \widetilde{\cdot}\right.  c$. Thus, the
\textquotedblleft Associativity of multiplication\textquotedblright\ axiom is proven.]

[\textit{Proof of the \textquotedblleft Neutrality of one\textquotedblright%
\ axiom:} Let $a\in\mathbb{K}$. We must prove that $a\left.  \widetilde{\cdot
}\right.  1_{\mathbb{K}}=1_{\mathbb{K}}\left.  \widetilde{\cdot}\right.  a=a$.

But the original ring $\mathbb{K}$ satisfies the \textquotedblleft Neutrality
of one\textquotedblright\ axiom (since it is a ring); thus, $a1_{\mathbb{K}%
}=1_{\mathbb{K}}a=a$.

The definition of the operation $\left.  \widetilde{\cdot}\right.  $ yields
$a\left.  \widetilde{\cdot}\right.  1_{\mathbb{K}}=1_{\mathbb{K}}a=a$ and
$1_{\mathbb{K}}\left.  \widetilde{\cdot}\right.  a=a1_{\mathbb{K}}=a$.
Combining these two equalities, we find $a\left.  \widetilde{\cdot}\right.
1_{\mathbb{K}}=1_{\mathbb{K}}\left.  \widetilde{\cdot}\right.  a=a$. Thus, the
\textquotedblleft Neutrality of one\textquotedblright\ axiom is proven.]

[\textit{Proof of the \textquotedblleft Annihilation\textquotedblright%
\ axiom:} Let $a\in\mathbb{K}$. We must prove that $a\left.  \widetilde{\cdot
}\right.  0_{\mathbb{K}}=0_{\mathbb{K}}\left.  \widetilde{\cdot}\right.
a=0_{\mathbb{K}}$.

But the original ring $\mathbb{K}$ satisfies the \textquotedblleft
Annihilation\textquotedblright\ axiom (since it is a ring); thus,
$a0_{\mathbb{K}}=0_{\mathbb{K}}a=0_{\mathbb{K}}$.

The definition of the operation $\left.  \widetilde{\cdot}\right.  $ yields
$a\left.  \widetilde{\cdot}\right.  0_{\mathbb{K}}=0_{\mathbb{K}%
}a=0_{\mathbb{K}}$ and $0_{\mathbb{K}}\left.  \widetilde{\cdot}\right.
a=a0_{\mathbb{K}}=0_{\mathbb{K}}$. Combining these two equalities, we find
$a\left.  \widetilde{\cdot}\right.  0_{\mathbb{K}}=0_{\mathbb{K}}\left.
\widetilde{\cdot}\right.  a=0_{\mathbb{K}}$. Thus, the \textquotedblleft
Annihilation\textquotedblright\ axiom is proven.]

[\textit{Proof of the \textquotedblleft Distributivity\textquotedblright%
\ axiom:} Let $a,b,c\in\mathbb{K}$. We must prove that%
\[
a\left.  \widetilde{\cdot}\right.  \left(  b+c\right)  =a\left.
\widetilde{\cdot}\right.  b+a\left.  \widetilde{\cdot}\right.
c\ \ \ \ \ \ \ \ \ \ \text{and}\ \ \ \ \ \ \ \ \ \ \left(  a+b\right)  \left.
\widetilde{\cdot}\right.  c=a\left.  \widetilde{\cdot}\right.  c+b\left.
\widetilde{\cdot}\right.  c.
\]


But the original ring $\mathbb{K}$ satisfies the \textquotedblleft
Distributivity\textquotedblright\ axiom (since it is a ring); thus,%
\[
c\left(  a+b\right)  =ca+cb\ \ \ \ \ \ \ \ \ \ \text{and}%
\ \ \ \ \ \ \ \ \ \ \left(  b+c\right)  a=ba+ca.
\]


The definition of the operation $\left.  \widetilde{\cdot}\right.  $ yields
$a\left.  \widetilde{\cdot}\right.  \left(  b+c\right)  =\left(  b+c\right)
a$ and $a\left.  \widetilde{\cdot}\right.  b=ba$ and $a\left.
\widetilde{\cdot}\right.  c=ca$. Thus,%
\[
a\left.  \widetilde{\cdot}\right.  \left(  b+c\right)  =\left(  b+c\right)
a=ba+ca.
\]
Comparing this with $\underbrace{a\left.  \widetilde{\cdot}\right.  b}%
_{=ba}+\underbrace{a\left.  \widetilde{\cdot}\right.  c}_{=ca}=ba+ca$, we
obtain $a\left.  \widetilde{\cdot}\right.  \left(  b+c\right)  =a\left.
\widetilde{\cdot}\right.  b+a\left.  \widetilde{\cdot}\right.  c$.

The definition of the operation $\left.  \widetilde{\cdot}\right.  $ yields
$\left(  a+b\right)  \left.  \widetilde{\cdot}\right.  c=c\left(  a+b\right)
$ and $a\left.  \widetilde{\cdot}\right.  c=ca$ and $b\left.  \widetilde{\cdot
}\right.  c=cb$. Thus,%
\[
\left(  a+b\right)  \left.  \widetilde{\cdot}\right.  c=c\left(  a+b\right)
=ca+cb.
\]
Comparing this with $\underbrace{a\left.  \widetilde{\cdot}\right.  c}%
_{=ca}+\underbrace{b\left.  \widetilde{\cdot}\right.  c}_{=cb}=ca+cb$, we
obtain $\left(  a+b\right)  \left.  \widetilde{\cdot}\right.  c=a\left.
\widetilde{\cdot}\right.  c+b\left.  \widetilde{\cdot}\right.  c$.

Thus, we have proven the equalities%
\[
a\left.  \widetilde{\cdot}\right.  \left(  b+c\right)  =a\left.
\widetilde{\cdot}\right.  b+a\left.  \widetilde{\cdot}\right.
c\ \ \ \ \ \ \ \ \ \ \text{and}\ \ \ \ \ \ \ \ \ \ \left(  a+b\right)  \left.
\widetilde{\cdot}\right.  c=a\left.  \widetilde{\cdot}\right.  c+b\left.
\widetilde{\cdot}\right.  c.
\]
Hence, the \textquotedblleft Associativity of multiplication\textquotedblright%
\ axiom is proven.]

We have now shown that the set $\mathbb{K}$, equipped with the addition $+$,
the multiplication $\left.  \widetilde{\cdot}\right.  $, the zero
$0_{\mathbb{K}}$ and the unity $1_{\mathbb{K}}$, satisfies all the eight ring
axioms. Hence, it is a ring. This solves part \textbf{(a)} of the problem.

\bigskip

\textbf{(b)} $\Longrightarrow:$ Assume that $\operatorname{id}:\mathbb{K}%
\rightarrow\mathbb{K}$ is a ring isomorphism from $\mathbb{K}$ to
$\mathbb{K}^{\operatorname{op}}$. We must prove that $\mathbb{K}$ is commutative.

We have assumed that $\operatorname*{id}$ is a ring isomorphism from
$\mathbb{K}$ to $\mathbb{K}^{\operatorname{op}}$. Thus, in particular,
$\operatorname*{id}$ is a ring homomorphism from $\mathbb{K}$ to
$\mathbb{K}^{\operatorname{op}}$ (since any ring isomorphism must be a ring homomorphism).

Recall that if $\mathbb{U}$ and $\mathbb{V}$ are two rings, and if $f$ is a
ring homomorphism from $\mathbb{U}$ to $\mathbb{V}$, then%
\begin{equation}
f\left(  a\cdot b\right)  =f\left(  a\right)  \cdot f\left(  b\right)
\ \ \ \ \ \ \ \ \ \ \text{for all }a,b\in\mathbb{U}.
\label{sol.ring.op.b.fwd.1}%
\end{equation}
(Indeed, this is one of the four axioms in our definition of a ring
homomorphism.) But keep in mind that the two \textquotedblleft$\cdot
$\textquotedblright\ signs in the equality \eqref{sol.ring.op.b.fwd.1} have
different meanings: The \textquotedblleft$\cdot$\textquotedblright\ sign on
the left hand side stands for the multiplication of the ring $\mathbb{U}$,
whereas the \textquotedblleft$\cdot$\textquotedblright\ sign on the right hand
side stands for the multiplication of the ring $\mathbb{V}$. Thus,
\eqref{sol.ring.op.b.fwd.1} (applied to $\mathbb{U}=\mathbb{K}$,
$\mathbb{V}=\mathbb{K}^{\operatorname*{op}}$ and $f=\operatorname*{id}$)
yields%
\begin{equation}
\operatorname*{id}\left(  a\cdot b\right)  =\operatorname*{id}\left(
a\right)  \left.  \widetilde{\cdot}\right.  \operatorname*{id}\left(
b\right)  \ \ \ \ \ \ \ \ \ \ \text{for all }a,b\in\mathbb{K}
\label{sol.ring.op.b.fwd.2}%
\end{equation}
(since $\operatorname*{id}$ is a ring homomorphism from $\mathbb{K}$ to
$\mathbb{K}^{\operatorname{op}}$, and since the multiplication of the ring
$\mathbb{K}$ is denoted by \textquotedblleft$\cdot$\textquotedblright\ whereas
the multiplication of the ring $\mathbb{K}^{\operatorname*{op}}$ is denoted by
\textquotedblleft$\left.  \widetilde{\cdot}\right.  $\textquotedblright).

Now, if $a,b\in\mathbb{K}$, then%
\begin{align*}
ab  &  =a\cdot b=\operatorname*{id}\left(  a\cdot b\right)
=\underbrace{\operatorname*{id}\left(  a\right)  }_{=a}\left.
\widetilde{\cdot}\right.  \underbrace{\operatorname*{id}\left(  b\right)
}_{=b}\qquad\left(  \text{by \eqref{sol.ring.op.b.fwd.2}}\right) \\
&  =a\left.  \widetilde{\cdot}\right.  b=ba\qquad\left(  \text{by the
definition of the operation }\left.  \widetilde{\cdot}\right.  \right)  .
\end{align*}
In other words, the ring $\mathbb{K}$ satisfies the \textquotedblleft
Commutativity of multiplication\textquotedblright\ axiom. In other words, the
ring $\mathbb{K}$ is commutative. This proves the \textquotedblleft%
$\Longrightarrow$\textquotedblright\ direction of part \textbf{(b)}.

$\Longleftarrow:$ Assume that $\mathbb{K}$ is commutative. We must prove that
$\operatorname{id}:\mathbb{K}\rightarrow\mathbb{K}$ is a ring isomorphism from
$\mathbb{K}$ to $\mathbb{K}^{\operatorname{op}}$.

If $a,b\in\mathbb{K}$, then%
\begin{align*}
a\left.  \widetilde{\cdot}\right.  b  &  =ba\qquad\left(  \text{by the
definition of the operation }\left.  \widetilde{\cdot}\right.  \right) \\
&  =ab\qquad\left(  \text{since the ring }\mathbb{K}\text{ is commutative}%
\right) \\
&  =a\cdot b.
\end{align*}
Thus, the binary operation $\left.  \widetilde{\cdot}\right.  $ is identical
with the binary operation $\cdot$.

But the only difference between the rings $\mathbb{K}$ and $\mathbb{K}%
^{\operatorname*{op}}$ is that $\mathbb{K}^{\operatorname*{op}}$ has the
multiplication $\left.  \widetilde{\cdot}\right.  $ while $\mathbb{K}$ has the
multiplication $\cdot$. (All the remaining structure of $\mathbb{K}$ and
$\mathbb{K}^{\operatorname*{op}}$ is the same.) But since we have shown that
$\left.  \widetilde{\cdot}\right.  $ is identical with $\cdot$, we see that
this difference is not actually a difference either; the multiplications of
$\mathbb{K}$ and $\mathbb{K}^{\operatorname*{op}}$ are also the same. Hence,
the ring $\mathbb{K}^{\operatorname*{op}}$ is completely identical to the ring
$\mathbb{K}$ (not just as sets, but as rings with all their structure).

But recall that $\operatorname*{id}:\mathbb{K}\rightarrow\mathbb{K}$ is a ring
isomorphism from $\mathbb{K}$ to $\mathbb{K}$. Since the ring $\mathbb{K}%
^{\operatorname*{op}}$ is completely identical to the ring $\mathbb{K}$, we
can replace the last \textquotedblleft$\mathbb{K}$\textquotedblright\ in this
sentence by \textquotedblleft$\mathbb{K}^{\operatorname*{op}}$%
\textquotedblright\ without changing its meaning. Thus, we obtain that
$\operatorname*{id}:\mathbb{K}\rightarrow\mathbb{K}$ is a ring isomorphism
from $\mathbb{K}$ to $\mathbb{K}^{\operatorname*{op}}$. This proves the
\textquotedblleft$\Longleftarrow$\textquotedblright\ direction of part
\textbf{(b)}.

\bigskip

\textbf{(c)} Let us quote the following fact from \cite[Exercise
6.5]{detnotes} (except that we are replacing $\mathbb{K}$ by $\mathbb{L}$):

\begin{proposition}
\label{prop.sol.ring.op.c.transp.helper}Let $\mathbb{L}$ be a commutative
ring. In this proposition, all matrices are over $\mathbb{L}$.

\textbf{(a)} If $u$, $v$ and $w$ are three nonnegative integers, if $P$ is a
$u\times v$-matrix, and if $Q$ is a $v\times w$-matrix, then%
\[
\left(  PQ\right)  ^{T}=Q^{T}P^{T}.
\]


\textbf{(b)} Every $u\in\mathbb{N}$ satisfies%
\[
\left(  I_{u}\right)  ^{T}=I_{u}.
\]


\textbf{(c)} If $u$ and $v$ are two nonnegative integers, if $P$ is a $u\times
v$-matrix, and if $\lambda\in\mathbb{L}$, then%
\[
\left(  \lambda P\right)  ^{T}=\lambda P^{T}.
\]


\textbf{(d)} If $u$ and $v$ are two nonnegative integers, and if $P$ and $Q$
are two $u\times v$-matrices, then%
\[
\left(  P+Q\right)  ^{T}=P^{T}+Q^{T}.
\]


\textbf{(e)} If $u$ and $v$ are two nonnegative integers, and if $P$ is a
$u\times v$-matrix, then%
\[
\left(  P^{T}\right)  ^{T}=P.
\]

\end{proposition}

Now, let $\mathbf{T}$ be the map%
\[
\mathbb{K}\rightarrow\mathbb{K}^{\operatorname{op}},\qquad A\mapsto A^{T}.
\]
We must prove that $\mathbf{T}$ is a ring isomorphism.

In class\footnote{specifically, Proposition 5.10.5 in
\href{http://www.cip.ifi.lmu.de/~grinberg/t/19s/notes.pdf}{the class notes};
but the numbering may change}, we have proven that any invertible ring
homomorphism is a ring isomorphism. Hence, it suffices to prove that
$\mathbf{T}$ is an invertible ring homomorphism.

Let us first prove that $\mathbf{T}$ is a ring homomorphism. In order to do
so, we need to verify the following four claims:

\begin{statement}
\textit{Claim 1:} We have $\mathbf{T}\left(  a+b\right)  =\mathbf{T}\left(
a\right)  +\mathbf{T}\left(  b\right)  $ for all $a,b\in\mathbb{K}$.
\end{statement}

\begin{statement}
\textit{Claim 2:} We have $\mathbf{T}\left(  0_{\mathbb{K}}\right)
=0_{\mathbb{K}^{\operatorname*{op}}}$.
\end{statement}

\begin{statement}
\textit{Claim 3:} We have $\mathbf{T}\left(  ab\right)  =\mathbf{T}\left(
a\right)  \left.  \widetilde{\cdot}\right.  \mathbf{T}\left(  b\right)  $ for
all $a,b\in\mathbb{K}$.
\end{statement}

\begin{statement}
\textit{Claim 4:} We have $\mathbf{T}\left(  1_{\mathbb{K}}\right)
=1_{\mathbb{K}^{\operatorname*{op}}}$.
\end{statement}

(Note the \textquotedblleft$\left.  \widetilde{\cdot}\right.  $%
\textquotedblright\ sign on the right hand side of Claim 3; this is because
$\mathbf{T}\left(  a\right)  $ and $\mathbf{T}\left(  b\right)  $ are being
considered as elements of $\mathbb{K}^{\operatorname*{op}}$, and the
multiplication of the ring $\mathbb{K}^{\operatorname*{op}}$ is $\left.
\widetilde{\cdot}\right.  $.)

Let us now prove these claims:

[\textit{Proof of Claim 3:} Let $a,b\in\mathbb{K}$. Then, $a\in\mathbb{K}%
=\mathbb{L}^{n\times n}$ and $b\in\mathbb{K}=\mathbb{L}^{n\times n}$. Hence,
$a$ and $b$ are two $n\times n$-matrices over $\mathbb{L}$. The definition of
$\mathbf{T}$ yields $\mathbf{T}\left(  ab\right)  =\left(  ab\right)  ^{T}$
and $\mathbf{T}\left(  a\right)  =a^{T}$ and $\mathbf{T}\left(  b\right)
=b^{T}$. The definition of the operation $\left.  \widetilde{\cdot}\right.  $
yields $\mathbf{T}\left(  a\right)  \left.  \widetilde{\cdot}\right.
\mathbf{T}\left(  b\right)  =\underbrace{\mathbf{T}\left(  b\right)  }%
_{=b^{T}}\underbrace{\mathbf{T}\left(  a\right)  }_{=a^{T}}=b^{T}a^{T}$. But
$\mathbf{T}\left(  ab\right)  =\left(  ab\right)  ^{T}=b^{T}a^{T}$ (by
Proposition \ref{prop.sol.ring.op.c.transp.helper} \textbf{(a)}, applied to
$u=n$, $v=n$, $w=n$, $P=a$ and $Q=b$). Comparing these two equalities, we
obtain $\mathbf{T}\left(  ab\right)  =\mathbf{T}\left(  a\right)  \left.
\widetilde{\cdot}\right.  \mathbf{T}\left(  b\right)  $. This proves Claim 3.]

[\textit{Proof of Claim 1:} Let $a,b\in\mathbb{K}$. Then, $a\in\mathbb{K}%
=\mathbb{L}^{n\times n}$ and $b\in\mathbb{K}=\mathbb{L}^{n\times n}$. Hence,
$a$ and $b$ are two $n\times n$-matrices over $\mathbb{L}$. The definition of
$\mathbf{T}$ yields $\mathbf{T}\left(  a+b\right)  =\left(  a+b\right)  ^{T}$
and $\mathbf{T}\left(  a\right)  =a^{T}$ and $\mathbf{T}\left(  b\right)
=b^{T}$. But $\underbrace{\mathbf{T}\left(  a\right)  }_{=a^{T}}%
+\underbrace{\mathbf{T}\left(  b\right)  }_{=b^{T}}=a^{T}b^{T}$. But
$\mathbf{T}\left(  a+b\right)  =\left(  a+b\right)  ^{T}=a^{T}+b^{T}$ (by
Proposition \ref{prop.sol.ring.op.c.transp.helper} \textbf{(d)}, applied to
$u=n$, $v=n$, $P=a$ and $Q=b$). Comparing these two equalities, we obtain
$\mathbf{T}\left(  a+b\right)  =\mathbf{T}\left(  a\right)  +\mathbf{T}\left(
b\right)  $. This proves Claim 1.]

[\textit{Proof of Claim 2:} We have $0_{\mathbb{K}}=0_{n\times n}$ (by the
definition of the ring $\mathbb{K}=\mathbb{L}^{n\times n}$). Applying the map
$\mathbf{T}$ to both sides of this equality, we obtain $\mathbf{T}\left(
0_{\mathbb{K}}\right)  =\mathbf{T}\left(  0_{n\times n}\right)  =\left(
0_{n\times n}\right)  ^{T}$ (by the definition of $\mathbf{T}$). But the
definition of the transpose of a matrix easily yields $\left(  0_{n\times
n}\right)  ^{T}=0_{n\times n}$. Hence, $\mathbf{T}\left(  0_{\mathbb{K}%
}\right)  =\left(  0_{n\times n}\right)  ^{T}=0_{n\times n}$. But the
definition of the ring $\mathbb{K}^{\operatorname*{op}}$ yields $0_{\mathbb{K}%
^{\operatorname*{op}}}=0_{\mathbb{K}}=0_{n\times n}$. Comparing the latter two
equalities, we obtain $\mathbf{T}\left(  0_{\mathbb{K}}\right)  =0_{\mathbb{K}%
^{\operatorname*{op}}}$. This proves Claim 2.]

[\textit{Proof of Claim 4:} We have $1_{\mathbb{K}}=I_{n}$ (by the definition
of the ring $\mathbb{K}=\mathbb{L}^{n\times n}$). Applying the map
$\mathbf{T}$ to both sides of this equality, we obtain $\mathbf{T}\left(
1_{\mathbb{K}}\right)  =\mathbf{T}\left(  I_{n}\right)  =\left(  I_{n}\right)
^{T}$ (by the definition of $\mathbf{T}$). But Proposition
\ref{prop.sol.ring.op.c.transp.helper} \textbf{(b)} (applied to $u=n$) yields
$\left(  I_{n}\right)  ^{T}=I_{n}$. Hence, $\mathbf{T}\left(  1_{\mathbb{K}%
}\right)  =\left(  I_{n}\right)  ^{T}=I_{n}$. But the definition of the ring
$\mathbb{K}^{\operatorname*{op}}$ yields $1_{\mathbb{K}^{\operatorname*{op}}%
}=1_{\mathbb{K}}=I_{n}$. Comparing the latter two equalities, we obtain
$\mathbf{T}\left(  1_{\mathbb{K}}\right)  =1_{\mathbb{K}^{\operatorname*{op}}%
}$. This proves Claim 4.]

We have now proven all four Claims 1, 2, 3 and 4. Hence, $\mathbf{T}$ is a
ring homomorphism from $\mathbb{K}$ to $\mathbb{K}^{\operatorname*{op}}$ (by
the definition of a ring homomorphism).

Let us next prove that the map $\mathbf{T}$ is invertible. In proving this, we
do not need to concern ourselves with the ring structures (i.e., the
additions, multiplications, zeroes and unities) of $\mathbb{K}$ and
$\mathbb{K}^{\operatorname*{op}}$, but can simply consider $\mathbb{K}$ and
$\mathbb{K}^{\operatorname*{op}}$ as sets (because the invertibility of a map
has nothing to do with any ring structures).

Recall that $\mathbb{K}^{\operatorname*{op}}=\mathbb{K}$ \textbf{as sets}.
Thus, the map $\mathbf{T}$ is a map from $\mathbb{K}$ to $\mathbb{K}$ (since
$\mathbf{T}$ is a map from $\mathbb{K}$ to $\mathbb{K}^{\operatorname*{op}}$).
Hence, the map $\mathbf{T}\circ\mathbf{T}:\mathbb{K}\rightarrow\mathbb{K}$ is
well-defined. Moreover, each $P\in\mathbb{K}$ satisfies%
\begin{align*}
\left(  \mathbf{T}\circ\mathbf{T}\right)  \left(  P\right)   &  =\mathbf{T}%
\left(  \underbrace{\mathbf{T}\left(  P\right)  }_{\substack{=P^{T}\\\text{(by
the definition of }\mathbf{T}\text{)}}}\right)  =\mathbf{T}\left(
P^{T}\right)  =\left(  P^{T}\right)  ^{T}\qquad\left(  \text{by the definition
of }\mathbf{T}\right) \\
&  =P\qquad\left(  \text{by Proposition \ref{prop.sol.ring.op.c.transp.helper}
\textbf{(e)} (applied to }u=n\text{ and }v=n\text{)}\right) \\
&  =\operatorname*{id}\left(  P\right)  .
\end{align*}
In other words, $\mathbf{T}\circ\mathbf{T}=\operatorname*{id}$. Hence, the
maps $\mathbf{T}:\mathbb{K}\rightarrow\mathbb{K}$ and $\mathbf{T}%
:\mathbb{K}\rightarrow\mathbb{K}$ are mutually inverse. Thus, the map
$\mathbf{T}:\mathbb{K}\rightarrow\mathbb{K}$ is invertible. In other words,
the map $\mathbf{T}:\mathbb{K}\rightarrow\mathbb{K}^{\operatorname*{op}}$ is
invertible (since $\mathbb{K}=\mathbb{K}^{\operatorname*{op}}$ as sets).

So we have proven that the map $\mathbf{T}:\mathbb{K}\rightarrow
\mathbb{K}^{\operatorname*{op}}$ is an invertible ring homomorphism from
$\mathbb{K}$ to $\mathbb{K}^{\operatorname*{op}}$. Thus, this map $\mathbf{T}$
is a ring isomorphism from $\mathbb{K}$ to $\mathbb{K}^{\operatorname*{op}}$
(since any invertible ring homomorphism is a ring isomorphism). This solves
part \textbf{(c)} of the exercise.

%----------------------------------------------------------------------------------------
%	EXERCISE 2
%----------------------------------------------------------------------------------------
\rule{\linewidth}{0.3pt} \\[0.4cm]

\section{Exercise 2: More ring isomorphisms}

\subsection{Problem}

\begin{enumerate}
\item[\textbf{(a)}] Let $\mathbb{L}$ be a ring. Let $w \in\mathbb{L}$ be an
invertible element. Prove that the map
\[
\mathbb{L} \to\mathbb{L} , \qquad a \mapsto waw^{-1}
\]
is a ring isomorphism.

\item[\textbf{(b)}] Let $\mathbb{K}$ be a ring. Let $W$ be the $n \times
n$-matrix
\[
\left(  \left[  i + j = n + 1 \right]  \right)  _{1\leq i\leq n,\ 1\leq j\leq
n} =
\begin{pmatrix}
0 & \cdots & 0 & 0 & 1\\
0 & \cdots & 0 & 1 & 0\\
0 & \cdots & 1 & 0 & 0\\
\vdots & \iddots & \vdots & \vdots & \vdots\\
1 & \cdots & 0 & 0 & 0
\end{pmatrix}
\in\mathbb{K}^{n \times n}
\]
(where we are using \href{https://en.wikipedia.org/wiki/Iverson_bracket}{the
Iverson bracket notation} again).

Prove that $W = W^{-1}$.

\item[\textbf{(c)}] Let $A = \left(  a_{i,j} \right)  _{1\leq i\leq n,\ 1\leq
j\leq n} \in\mathbb{K}^{n \times n}$ be any $n \times n$-matrix. Prove that
\[
WAW^{-1} = \left(  a_{n+1-i,n+1-j} \right)  _{1\leq i\leq n,\ 1\leq j\leq n}.
\]
(In other words, $WAW^{-1}$ is the $n \times n$-matrix obtained from $A$ by
reversing the order of the rows and also reversing the order of the columns.)
\end{enumerate}

\subsection{Remark}

The map
\[
\mathbb{L} \to\mathbb{L} , \qquad a \mapsto waw^{-1}
\]
in part \textbf{(a)} of this exercise is called \textit{conjugation by $w$}.
It is best known in the case of a matrix ring, where it corresponds to a
change of basis for an endomorphism of a vector space. When $\mathbb{K}$ is a
field, the \textbf{only} ring isomorphisms $\mathbb{K}^{n \times n}
\to\mathbb{K}^{n \times n}$ are conjugations by invertible matrices; this is
the Noether--Skolem theorem (in one of its less general variants).

\subsection{Solution}

\textbf{(a)} Let $f$ be the map%
\[
\mathbb{L}\rightarrow\mathbb{L},\qquad a\mapsto waw^{-1}.
\]
We must prove that $f$ is a ring isomorphism.

In class, we have proven that any invertible ring homomorphism is a ring
isomorphism. Hence, it suffices to prove that $f$ is an invertible ring homomorphism.

Let us first prove that $f$ is a ring homomorphism. In order to do so, we need
to verify the following four claims:

\begin{statement}
\textit{Claim 1:} We have $f\left(  a+b\right)  =f\left(  a\right)  +f\left(
b\right)  $ for all $a,b\in\mathbb{L}$.
\end{statement}

\begin{statement}
\textit{Claim 2:} We have $f\left(  0\right)  =0$.
\end{statement}

\begin{statement}
\textit{Claim 3:} We have $f\left(  ab\right)  =f\left(  a\right)  f\left(
b\right)  $ for all $a,b\in\mathbb{L}$.
\end{statement}

\begin{statement}
\textit{Claim 4:} We have $f\left(  1\right)  =1$.
\end{statement}

Let us now prove these claims:

[\textit{Proof of Claim 1:} Let $a,b\in\mathbb{L}$. The definition of $f$
yields $f\left(  a\right)  =waw^{-1}$ and $f\left(  b\right)  =wbw^{-1}$ and
$f\left(  a+b\right)  =w\left(  a+b\right)  w^{-1}$. Hence,%
\begin{align*}
f\left(  a+b\right)   &  =w\underbrace{\left(  a+b\right)  w^{-1}%
}_{\substack{=aw^{-1}+bw^{-1}\\\text{(by distributivity)}}}=w\left(
aw^{-1}+bw^{-1}\right)  =\underbrace{waw^{-1}}_{=f\left(  a\right)
}+\underbrace{wbw^{-1}}_{=f\left(  b\right)  }\qquad\left(  \text{by
distributivity}\right) \\
&  =f\left(  a\right)  +f\left(  b\right)  .
\end{align*}
This proves Claim 1.]

[\textit{Proof of Claim 2:} The definition of $f$ yields $f\left(  0\right)
=w\underbrace{0w^{-1}}_{=0}=w0=0$. This proves Claim 2.]

[\textit{Proof of Claim 3:} Let $a,b\in\mathbb{L}$. The definition of $f$
yields $f\left(  a\right)  =waw^{-1}$ and $f\left(  b\right)  =wbw^{-1}$ and
$f\left(  ab\right)  =w\left(  ab\right)  w^{-1}$. Hence,%
\[
\underbrace{f\left(  a\right)  }_{=waw^{-1}}\ \ \underbrace{f\left(  b\right)
}_{=wbw^{-1}}=wa\underbrace{w^{-1}w}_{=1}bw^{-1}=wabw^{-1}=w\left(  ab\right)
w^{-1}=f\left(  ab\right)  .
\]
In other words, $f\left(  ab\right)  =f\left(  a\right)  f\left(  b\right)  $.
This proves Claim 3.]

[\textit{Proof of Claim 4:} The definition of $f$ yields $f\left(  1\right)
=w\underbrace{1w^{-1}}_{=w^{-1}}=ww^{-1}=1$. This proves Claim 4.]

We have now proven all four Claims 1, 2, 3 and 4. Hence, $f$ is a ring
homomorphism from $\mathbb{L}$ to $\mathbb{L}$ (by the definition of a ring homomorphism).

Let us next prove that the map $f$ is invertible.

Indeed, let $g$ be the map%
\[
\mathbb{L}\rightarrow\mathbb{L},\qquad a\mapsto w^{-1}aw.
\]


Then, each $a\in\mathbb{L}$ satisfies%
\begin{align*}
\left(  g\circ f\right)  \left(  a\right)   &  =g\left(  f\left(  a\right)
\right)  =w^{-1}\underbrace{f\left(  a\right)  }_{\substack{=waw^{-1}%
\\\text{(by the definition of }f\text{)}}}w\qquad\left(  \text{by the
definition of }g\right) \\
&  =\underbrace{w^{-1}w}_{=1}a\underbrace{w^{-1}w}_{=1}=a=\operatorname*{id}%
\left(  a\right)  .
\end{align*}
In other words, $g\circ f=\operatorname*{id}$.

Also, each $a\in\mathbb{L}$ satisfies%
\begin{align*}
\left(  f\circ g\right)  \left(  a\right)   &  =f\left(  g\left(  a\right)
\right)  =w\underbrace{g\left(  a\right)  }_{\substack{=w^{-1}aw\\\text{(by
the definition of }g\text{)}}}w^{-1}\qquad\left(  \text{by the definition of
}f\right) \\
&  =\underbrace{ww^{-1}}_{=1}a\underbrace{ww^{-1}}_{=1}=a=\operatorname*{id}%
\left(  a\right)  .
\end{align*}
In other words, $f\circ g=\operatorname*{id}$.

Now, the two maps $f$ and $g$ are mutually inverse (since $f\circ
g=\operatorname*{id}$ and $g\circ f=\operatorname*{id}$). Thus, the map $f$ is invertible.

So we have proven that the map $f$ is an invertible ring homomorphism. Thus,
this map $f$ is a ring isomorphism (since any invertible ring homomorphism is
a ring isomorphism). This solves part \textbf{(a)} of the exercise.

\bigskip

\textbf{(b)} We first show two auxiliary claims about how multiplication by
$W$ changes a matrix:

\begin{statement}
\textit{Claim 5:} Let $A=\left(  a_{i,j}\right)  _{1\leq i\leq n,\ 1\leq j\leq
n}\in\mathbb{K}^{n\times n}$ be any $n\times n$-matrix. Then,%
\[
WA=\left(  a_{n+1-i,j}\right)  _{1\leq i\leq n,\ 1\leq j\leq n}.
\]

\end{statement}

\begin{statement}
\textit{Claim 6:} Let $A=\left(  a_{i,j}\right)  _{1\leq i\leq n,\ 1\leq j\leq
n}\in\mathbb{K}^{n\times n}$ be any $n\times n$-matrix. Then,%
\[
AW=\left(  a_{i,n+1-j}\right)  _{1\leq i\leq n,\ 1\leq j\leq n}.
\]

\end{statement}

[\textit{Proof of Claim 5:} We have $W=\left(  \left[  i+j=n+1\right]
\right)  _{1\leq i\leq n,\ 1\leq j\leq n}$ and $A=\left(  a_{i,j}\right)
_{1\leq i\leq n,\ 1\leq j\leq n}$. Hence, the definition of the multiplication
of matrices yields%
\begin{equation}
WA=\left(  \sum_{k=1}^{n}\left[  i+k=n+1\right]  a_{k,j}\right)  _{1\leq i\leq
n,\ 1\leq j\leq n}. \label{sol.ring.matrix.W-conj.b.c5.pf.1}%
\end{equation}


Now, let $\left(  i,j\right)  \in\left\{  1,2,\ldots,n\right\}  ^{2}$. Thus,
$i,j\in\left\{  1,2,\ldots,n\right\}  $. From $i\in\left\{  1,2,\ldots
,n\right\}  $, we obtain $n+1-i\in\left\{  1,2,\ldots,n\right\}  $. Now,%
\begin{align}
&  \sum_{k=1}^{n}\left[  i+k=n+1\right]  a_{k,j}\nonumber\\
&  =\sum_{k\in\left\{  1,2,\ldots,n\right\}  }\left[  i+k=n+1\right]
a_{k,j}\nonumber\\
&  =\underbrace{\left[  i+\left(  n+1-i\right)  =n+1\right]  }%
_{\substack{=1\\\text{(since }i+\left(  n+1-i\right)  =n+1\text{)}%
}}a_{n+1-i,j}+\sum_{\substack{k\in\left\{  1,2,\ldots,n\right\}  ;\\k\neq
n+1-i}}\underbrace{\left[  i+k=n+1\right]  }_{\substack{=0\\\text{(since
}i+k\neq n+1\\\text{(because }k\neq n+1-i\text{))}}}a_{k,j}\nonumber\\
&  \qquad\qquad\left(
\begin{array}
[c]{c}%
\text{here, we have split off the addend for }k=n+1-i\text{ from the sum}\\
\text{(since }n+1-i\in\left\{  1,2,\ldots,n\right\}  \text{)}%
\end{array}
\right) \nonumber\\
&  =a_{n+1-i,j}+\underbrace{\sum_{\substack{k\in\left\{  1,2,\ldots,n\right\}
;\\k\neq n+1-i}}0a_{k,j}}_{=0}=a_{n+1-i,j}.
\label{sol.ring.matrix.W-conj.b.c5.pf.2}%
\end{align}


Now, forget that we fixed $\left(  i,j\right)  $. We thus have proven
\eqref{sol.ring.matrix.W-conj.b.c5.pf.2} for each $\left(  i,j\right)
\in\left\{  1,2,\ldots,n\right\}  ^{2}$. Thus, we have%
\[
\left(  \sum_{k=1}^{n}\left[  i+k=n+1\right]  a_{k,j}\right)  _{1\leq i\leq
n,\ 1\leq j\leq n}=\left(  a_{n+1-i,j}\right)  _{1\leq i\leq n,\ 1\leq j\leq
n}.
\]
Hence, \eqref{sol.ring.matrix.W-conj.b.c5.pf.2} becomes%
\[
WA=\left(  \sum_{k=1}^{n}\left[  i+k=n+1\right]  a_{k,j}\right)  _{1\leq i\leq
n,\ 1\leq j\leq n}=\left(  a_{n+1-i,j}\right)  _{1\leq i\leq n,\ 1\leq j\leq
n}.
\]
This proves Claim 5.]

[\textit{Proof of Claim 6:} We have $A=\left(  a_{i,j}\right)  _{1\leq i\leq
n,\ 1\leq j\leq n}$ and $W=\left(  \left[  i+j=n+1\right]  \right)  _{1\leq
i\leq n,\ 1\leq j\leq n}$. Hence, the definition of the multiplication of
matrices yields%
\begin{equation}
AW=\left(  \sum_{k=1}^{n}a_{i,k}\left[  k+j=n+1\right]  \right)  _{1\leq i\leq
n,\ 1\leq j\leq n}. \label{sol.ring.matrix.W-conj.b.c6.pf.1}%
\end{equation}


Now, let $\left(  i,j\right)  \in\left\{  1,2,\ldots,n\right\}  ^{2}$. Thus,
$i,j\in\left\{  1,2,\ldots,n\right\}  $. From $j\in\left\{  1,2,\ldots
,n\right\}  $, we obtain $n+1-j\in\left\{  1,2,\ldots,n\right\}  $. Now,%
\begin{align}
&  \sum_{k=1}^{n}a_{i,k}\left[  k+j=n+1\right] \nonumber\\
&  =\sum_{k\in\left\{  1,2,\ldots,n\right\}  }a_{i,k}\left[  k+j=n+1\right]
\nonumber\\
&  =a_{i,n+1-j}\underbrace{\left[  \left(  n+1-j\right)  +j=n+1\right]
}_{\substack{=1\\\text{(since }\left(  n+1-j\right)  +j=n+1\text{)}}%
}+\sum_{\substack{k\in\left\{  1,2,\ldots,n\right\}  ;\\k\neq n+1-j}%
}a_{i,k}\underbrace{\left[  k+j=n+1\right]  }_{\substack{=0\\\text{(since
}k+j\neq n+1\\\text{(because }k\neq n+1-j\text{))}}}\nonumber\\
&  \qquad\qquad\left(
\begin{array}
[c]{c}%
\text{here, we have split off the addend for }k=n+1-j\text{ from the sum}\\
\text{(since }n+1-j\in\left\{  1,2,\ldots,n\right\}  \text{)}%
\end{array}
\right) \nonumber\\
&  =a_{i,n+1-j}+\underbrace{\sum_{\substack{k\in\left\{  1,2,\ldots,n\right\}
;\\k\neq n+1-j}}a_{i,k}0}_{=0}=a_{i,n+1-j}.
\label{sol.ring.matrix.W-conj.b.c6.pf.2}%
\end{align}


Now, forget that we fixed $\left(  i,j\right)  $. We thus have proven
\eqref{sol.ring.matrix.W-conj.b.c6.pf.2} for each $\left(  i,j\right)
\in\left\{  1,2,\ldots,n\right\}  ^{2}$. Thus, we have%
\[
\left(  \sum_{k=1}^{n}a_{i,k}\left[  k+j=n+1\right]  \right)  _{1\leq i\leq
n,\ 1\leq j\leq n}=\left(  a_{i,n+1-j}\right)  _{1\leq i\leq n,\ 1\leq j\leq
n}.
\]
Hence, \eqref{sol.ring.matrix.W-conj.b.c6.pf.2} becomes%
\[
AW=\left(  \sum_{k=1}^{n}a_{i,k}\left[  k+j=n+1\right]  \right)  _{1\leq i\leq
n,\ 1\leq j\leq n}=\left(  a_{i,n+1-j}\right)  _{1\leq i\leq n,\ 1\leq j\leq
n}.
\]
This proves Claim 6.]

Let us now come back to part \textbf{(b)} of this exercise. Recall the
definition of the identity matrix $I_{n}\in\mathbb{K}^{n\times n}$. Namely,
$I_{n}$ is defined by
\[
I_{n}=\left(  \delta_{i,j}\right)  _{1\leq i\leq n,\ 1\leq j\leq n}%
,\qquad\text{where }\delta_{i,j}=%
\begin{cases}
1, & \text{if }i=j;\\
0, & \text{if }i\neq j
\end{cases}
.
\]
(Note that $\delta_{i,j}$ can also be written as $\left[  i=j\right]  $ using
the Iverson bracket notation.)

Now, $W=\left(  \left[  i+j=n+1\right]  \right)  _{1\leq i\leq n,\ 1\leq j\leq
n}$. Hence, Claim 6 (applied to $A=W$ and $a_{i,j}=\left[  i+j=n+1\right]  $)
yields%
\begin{equation}
WW=\left(  \left[  \left(  n+1-i\right)  +j=n+1\right]  \right)  _{1\leq i\leq
n,\ 1\leq j\leq n}. \label{sol.ring.matrix.W-conj.b.WW=}%
\end{equation}
Now, let $\left(  i,j\right)  \in\left\{  1,2,\ldots,n\right\}  ^{2}$. Thus,
$i,j\in\left\{  1,2,\ldots,n\right\}  $. Now, the statement \textquotedblleft%
$\left(  n+1-i\right)  +j=n+1$\textquotedblright\ is equivalent to
\textquotedblleft$i=j$\textquotedblright\ (since $\left(  \left(
n+1-i\right)  +j\right)  -\left(  n+1\right)  =j-i$). Thus,%
\begin{align}
\left[  \left(  n+1-i\right)  +j=n+1\right]   &  =\left[  i=j\right]  =%
\begin{cases}
1, & \text{if }i=j\text{ is true;}\\
0, & \text{if }i=j\text{ is false}%
\end{cases}
\nonumber\\
&  \qquad\left(  \text{by the definition of the Iverson bracket notation}%
\right) \nonumber\\
&  =%
\begin{cases}
1, & \text{if }i=j;\\
0, & \text{if }i\neq j
\end{cases}
=\delta_{i,j}. \label{sol.ring.matrix.W-conj.b.1}%
\end{align}


Forget that we fixed $\left(  i,j\right)  $. We thus have proven
\eqref{sol.ring.matrix.W-conj.b.1} for each $\left(  i,j\right)  \in\left\{
1,2,\ldots,n\right\}  ^{2}$. Thus, we have%
\[
\left(  \left[  \left(  n+1-i\right)  +j=n+1\right]  \right)  _{1\leq i\leq
n,\ 1\leq j\leq n}=\left(  \delta_{i,j}\right)  _{1\leq i\leq n,\ 1\leq j\leq
n}.
\]
Hence, \eqref{sol.ring.matrix.W-conj.b.WW=} becomes%
\[
WW=\left(  \left[  \left(  n+1-i\right)  +j=n+1\right]  \right)  _{1\leq i\leq
n,\ 1\leq j\leq n}=\left(  \delta_{i,j}\right)  _{1\leq i\leq n,\ 1\leq j\leq
n}=I_{n}.
\]


Now, the matrix $W$ is an inverse of $W$ (since $WW=I_{n}$ and $WW=I_{n}$).
Thus, the matrix $W$ is invertible, and its inverse is $W^{-1}=W$. This solves
part \textbf{(b)} of the exercise.

\bigskip

\textbf{(c)} We have $A=\left(  a_{i,j}\right)  _{1\leq i\leq n,\ 1\leq j\leq
n}$. Thus, Claim 5 yields%
\[
WA=\left(  a_{n+1-i,j}\right)  _{1\leq i\leq n,\ 1\leq j\leq n}.
\]
Hence, Claim 6 (applied to $WA$ and $a_{n+1-i,j}$ instead of $A$ and $a_{i,j}%
$) yields%
\[
WAW=\left(  a_{n+1-i,n+1-j}\right)  _{1\leq i\leq n,\ 1\leq j\leq n}.
\]
But part \textbf{(b)} of this exercise yields $W=W^{-1}$. Hence,
$WA\underbrace{W}_{=W^{-1}}=WAW^{-1}$, so that
\[
WAW^{-1}=WAW=\left(  a_{n+1-i,n+1-j}\right)  _{1\leq i\leq n,\ 1\leq j\leq
n}.
\]
This solves part \textbf{(c)} of the exercise.

%----------------------------------------------------------------------------------------
%	EXERCISE 3
%----------------------------------------------------------------------------------------
\rule{\linewidth}{0.3pt} \\[0.4cm]

\section{Exercise 3: Entangled inverses}

Let $\mathbb{K}$ be a ring.

A \textit{left inverse} of an element $x \in\mathbb{K}$ is defined to be a $y
\in\mathbb{K}$ such that $yx = 1$.

A \textit{right inverse} of an element $x \in\mathbb{K}$ is defined to be a $y
\in\mathbb{K}$ such that $xy = 1$.

Let $a$ and $b$ be two elements of $\mathbb{K}$. Prove the following:

\begin{enumerate}
\item[\textbf{(a)}] If $c$ is a left inverse of $1 - ab$, then $1 + bca$ is a
left inverse of $1 - ba$.

\item[\textbf{(b)}] If $c$ is a right inverse of $1 - ab$, then $1 + bca$ is a
right inverse of $1 - ba$.

\item[\textbf{(c)}] If $c$ is an inverse of $1 - ab$, then $1 + bca$ is an
inverse of $1 - ba$.
\end{enumerate}

Here and in the following, the word ``\textit{inverse}'' (unless qualified
with an adjective) means ``multiplicative inverse''.

\subsection{Solution}

\textbf{(a)} Assume that $c$ is a left inverse of $1-ab$. Thus, $c\left(
1-ab\right)  =1$ (by the definition of a left inverse).

Now, the laws of distributivity\footnote{When we say \textquotedblleft the
laws of distributivity\textquotedblright\ here, we mean not just the axiom of
distributivity (which says that $u\left(  v+w\right)  =uv+uw$ and $\left(
u+v\right)  w=uw+vw$ for all $u,v,w\in\mathbb{K}$), but also its analogue for
subtraction (which says that $u\left(  v-w\right)  =uv-uw$ and $\left(
u-v\right)  w=uw-vw$ for all $u,v,w\in\mathbb{K}$). The latter analogue is not
one of the ring axioms, but follows easily from them.} yield%
\[
a\left(  1-ba\right)  =a-aba=\left(  1-ab\right)  a,
\]
thus%
\[
c\underbrace{a\left(  1-ba\right)  }_{=\left(  1-ab\right)  a}%
=\underbrace{c\left(  1-ab\right)  }_{=1}a=1a=a.
\]
Hence, using the distributivity axiom, we obtain%
\[
\left(  1+bca\right)  \left(  1-ba\right)  =\left(  1-ba\right)
+b\underbrace{ca\left(  1-ba\right)  }_{=a}=\left(  1-ba\right)  +ba=1.
\]
In other words, $1+bca$ is a left inverse of $1-ba$ (by the definition of a
left inverse). This solves part \textbf{(a)} of the exercise.

\bigskip

\textbf{(b)} Assume that $c$ is a right inverse of $1-ab$. Thus, $\left(
1-ab\right)  c=1$ (by the definition of a right inverse).

Now, the laws of distributivity yield%
\[
\left(  1-ba\right)  b=b-bab=b\left(  1-ab\right)  ,
\]
thus%
\[
\underbrace{\left(  1-ba\right)  b}_{=b\left(  1-ab\right)  }%
c=b\underbrace{\left(  1-ab\right)  c}_{=1}=b1=b.
\]
Hence, using the distributivity axiom, we obtain%
\[
\left(  1-ba\right)  \left(  1+bca\right)  =\left(  1-ba\right)
+\underbrace{\left(  1-ba\right)  bc}_{=b}a=\left(  1-ba\right)  +ba=1.
\]
In other words, $1+bca$ is a right inverse of $1-ba$ (by the definition of a
right inverse). This solves part \textbf{(b)} of the exercise.

\bigskip

\textbf{(c)} Assume that $c$ is an inverse of $1-ab$. In other words, $c$ is a
multiplicative inverse of $1-ab$. Thus, $\left(  1-ab\right)  c=c\left(
1-ab\right)  =1$ (by the definition of a multiplicative inverse).

From $c\left(  1-ab\right)  =1$, we conclude that $c$ is a left inverse of
$1-ab$. Hence, part \textbf{(a)} of this exercise shows that $1+bca$ is a left
inverse of $1-ba$. In other words, $\left(  1+bca\right)  \left(  1-ba\right)
=1$.

From $\left(  1-ab\right)  c=1$, we conclude that $c$ is a right inverse of
$1-ab$. Hence, part \textbf{(b)} of this exercise shows that $1+bca$ is a
right inverse of $1-ba$. In other words, $\left(  1-ba\right)  \left(
1+bca\right)  =1$.

Combining $\left(  1+bca\right)  \left(  1-ba\right)  =1$ with $\left(
1-ba\right)  \left(  1+bca\right)  =1$, we obtain
\[
\left(  1-ba\right)  \left(  1+bca\right)  =\left(  1+bca\right)  \left(
1-ba\right)  =1.
\]
In other words, $1+bca$ is a multiplicative inverse of $1-ba$ (by the
definition of a multiplicative inverse). In other words, $1+bca$ is an inverse
of $1-ba$. This solves part \textbf{(c)} of the exercise.

%----------------------------------------------------------------------------------------
%	EXERCISE 4
%----------------------------------------------------------------------------------------
\rule{\linewidth}{0.3pt} \\[0.4cm]

\section{Exercise 4: Composition of ring homomorphisms}

\subsection{Problem}

Let $\mathbb{K}$, $\mathbb{L}$ and $\mathbb{M}$ be three rings. Prove the following:

\begin{enumerate}
\item[\textbf{(a)}] If $f : \mathbb{K} \to\mathbb{L}$ and $g : \mathbb{L}
\to\mathbb{M}$ are two ring homomorphisms, then $g \circ f : \mathbb{K}
\to\mathbb{M}$ is a ring homomorphism.

\item[\textbf{(b)}] If $f : \mathbb{K} \to\mathbb{L}$ and $g : \mathbb{L}
\to\mathbb{M}$ are two ring isomorphisms, then $g \circ f : \mathbb{K}
\to\mathbb{M}$ is a ring isomorphism.
\end{enumerate}

\subsection{Solution}

\textbf{(a)} Let $f:\mathbb{K}\rightarrow\mathbb{L}$ and $g:\mathbb{L}%
\rightarrow\mathbb{M}$ be two ring homomorphisms. We must prove that $g\circ
f:\mathbb{K}\rightarrow\mathbb{M}$ is a ring homomorphism.

We have assumed that $f:\mathbb{K}\rightarrow\mathbb{L}$ is a ring
homomorphism. In other words, $f$ satisfies the four axioms in our definition
of a ring homomorphism. In other words, the following four claims hold:

\begin{statement}
\textit{Claim 1:} We have $f\left(  a+b\right)  =f\left(  a\right)  +f\left(
b\right)  $ for all $a,b\in\mathbb{K}$.
\end{statement}

\begin{statement}
\textit{Claim 2:} We have $f\left(  0\right)  =0$.
\end{statement}

\begin{statement}
\textit{Claim 3:} We have $f\left(  ab\right)  =f\left(  a\right)  f\left(
b\right)  $ for all $a,b\in\mathbb{K}$.
\end{statement}

\begin{statement}
\textit{Claim 4:} We have $f\left(  1\right)  =1$.
\end{statement}

Similarly, from the assumption that $g:\mathbb{L}\rightarrow\mathbb{M}$ is a
ring homomorphism, we conclude that the following four claims hold:

\begin{statement}
\textit{Claim 5:} We have $g\left(  a+b\right)  =g\left(  a\right)  +g\left(
b\right)  $ for all $a,b\in\mathbb{L}$.
\end{statement}

\begin{statement}
\textit{Claim 6:} We have $g\left(  0\right)  =0$.
\end{statement}

\begin{statement}
\textit{Claim 7:} We have $g\left(  ab\right)  =g\left(  a\right)  g\left(
b\right)  $ for all $a,b\in\mathbb{L}$.
\end{statement}

\begin{statement}
\textit{Claim 8:} We have $g\left(  1\right)  =1$.
\end{statement}

Now, we must prove that $g\circ f:\mathbb{K}\rightarrow\mathbb{M}$ is a ring
homomorphism. In other words, we must prove that $g\circ f$ satisfies the four
axioms in our definition of a ring homomorphism. In other words, we must prove
that the following four claims hold:

\begin{statement}
\textit{Claim 9:} We have $\left(  g\circ f\right)  \left(  a+b\right)
=\left(  g\circ f\right)  \left(  a\right)  +\left(  g\circ f\right)  \left(
b\right)  $ for all $a,b\in\mathbb{K}$.
\end{statement}

\begin{statement}
\textit{Claim 10:} We have $\left(  g\circ f\right)  \left(  0\right)  =0$.
\end{statement}

\begin{statement}
\textit{Claim 11:} We have $\left(  g\circ f\right)  \left(  ab\right)
=\left(  g\circ f\right)  \left(  a\right)  \left(  g\circ f\right)  \left(
b\right)  $ for all $a,b\in\mathbb{K}$.
\end{statement}

\begin{statement}
\textit{Claim 12:} We have $\left(  g\circ f\right)  \left(  1\right)  =1$.
\end{statement}

But this is straightforward:

[\textit{Proof of Claim 9:} For all $a,b\in\mathbb{K}$, we have%
\begin{align*}
\left(  g\circ f\right)  \left(  a+b\right)   &  =g\left(
\underbrace{f\left(  a+b\right)  }_{\substack{=f\left(  a\right)  +f\left(
b\right)  \\\text{(by Claim 1)}}}\right)  =g\left(  f\left(  a\right)
+f\left(  b\right)  \right)  =\underbrace{g\left(  f\left(  a\right)  \right)
}_{=\left(  g\circ f\right)  \left(  a\right)  }+\underbrace{g\left(  f\left(
b\right)  \right)  }_{=\left(  g\circ f\right)  \left(  b\right)  }\\
&  \qquad\left(  \text{by Claim 5, applied to }f\left(  a\right)  \text{ and
}f\left(  b\right)  \text{ instead of }a\text{ and }b\right) \\
&  =\left(  g\circ f\right)  \left(  a\right)  +\left(  g\circ f\right)
\left(  b\right)  .
\end{align*}
Thus, Claim 9 is proven.]

[\textit{Proof of Claim 10:} We have $\left(  g\circ f\right)  \left(
0\right)  =g\left(  \underbrace{f\left(  0\right)  }_{\substack{=0\\\text{(by
Claim 2)}}}\right)  =g\left(  0\right)  =0$ (by Claim 6). Thus, Claim 10 is proven.]

[\textit{Proof of Claim 11:} The proof of Claim 11 is analogous to the proof
of Claim 9, except that we need to use Claims 3 and 7 instead of Claims 1 and 5.]

[\textit{Proof of Claim 12:} The proof of Claim 12 is analogous to the proof
of Claim 10, except that we need to use Claims 4 and 8 instead of Claims 2 and 6.]

Thus, all four Claims 9, 10, 11 and 12 are proven. As we explained, this shows
that $g\circ f$ is a ring homomorphism. Hence, part \textbf{(a)} of the
exercise is solved.

\bigskip

\textbf{(b)} Let $f:\mathbb{K}\rightarrow\mathbb{L}$ and $g:\mathbb{L}%
\rightarrow\mathbb{M}$ be two ring isomorphisms. We must show that $g\circ
f:\mathbb{K}\rightarrow\mathbb{M}$ is a ring isomorphism.

The map $f$ is a ring isomorphism. In other words, $f$ is invertible and both
$f$ and $f^{-1}$ are ring homomorphisms (by the definition of a ring isomorphism).

The map $g$ is a ring isomorphism. In other words, $g$ is invertible and both
$g$ and $g^{-1}$ are ring homomorphisms (by the definition of a ring isomorphism).

Now we know that $f:\mathbb{K}\rightarrow\mathbb{L}$ and $g:\mathbb{L}%
\rightarrow\mathbb{M}$ are two ring homomorphisms. Hence, part \textbf{(a)} of
this exercise shows that $g\circ f:\mathbb{K}\rightarrow\mathbb{M}$ is a ring homomorphism.

Also, we know that $g^{-1}:\mathbb{M}\rightarrow\mathbb{L}$ and $f^{-1}%
:\mathbb{L}\rightarrow\mathbb{K}$ are two ring homomorphisms. Hence, part
\textbf{(a)} of this exercise (applied to $\mathbb{M}$, $\mathbb{K}$, $g^{-1}$
and $f^{-1}$ instead of $\mathbb{K}$, $\mathbb{M}$, $f$ and $g$) shows that
$f^{-1}\circ g^{-1}:\mathbb{M}\rightarrow\mathbb{K}$ is a ring homomorphism.

But the maps $f$ and $g$ are invertible. Hence, it is well-known that their
composition $g\circ f$ is invertible as well, and its inverse is $\left(
g\circ f\right)  ^{-1}=f^{-1}\circ g^{-1}$. Hence, $\left(  g\circ f\right)
^{-1}$ is a ring homomorphism (since $f^{-1}\circ g^{-1}$ is a ring homomorphism).

Now, we know that the map $g\circ f$ is invertible and both $g\circ f$ and
$\left(  g\circ f\right)  ^{-1}$ are ring homomorphisms. In other words,
$g\circ f$ is a ring isomorphism (by the definition of a ring isomorphism).
This solves part \textbf{(b)} of the exercise.

%----------------------------------------------------------------------------------------
%	EXERCISE 5
%----------------------------------------------------------------------------------------
\rule{\linewidth}{0.3pt} \\[0.4cm]

\section{Exercise 5: Squares in finite fields I}

\subsection{Problem}

Let $\mathbb{F}$ be a field.

\begin{enumerate}
\item[\textbf{(a)}] Prove that if $a, b \in\mathbb{F}$ satisfy $ab = 0$, then
$a = 0$ or $b = 0$.

\item[\textbf{(b)}] Prove that if $a, b \in\mathbb{F}$ satisfy $a^{2} = b^{2}%
$, then $a = b$ or $a = -b$.
\end{enumerate}

Recall that an element $\eta\in\mathbb{F}$ is called a \textit{square} if
there exists some $\alpha\in\mathbb{F}$ such that $\eta= \alpha^{2}$.

From now on, assume that $2 \cdot1_{\mathbb{F}} \neq0_{\mathbb{F}}$ (that is,
$1_{\mathbb{F}} + 1_{\mathbb{F}} \neq0_{\mathbb{F}}$). Note that this is
satisfied whenever $\mathbb{F} = \mathbb{Z} / p$ for a prime $p > 2$ (but also
for various other finite fields), but fails when $\mathbb{F} = \mathbb{Z} / 2$.

\begin{enumerate}
\item[\textbf{(c)}] Prove that $a \neq-a$ for every nonzero $a \in\mathbb{F}$.
\end{enumerate}

From now on, assume that $\mathbb{F}$ is finite.

\begin{enumerate}
\item[\textbf{(d)}] Prove that the number of squares in $\mathbb{F}$ is
$\dfrac{1}{2} \left(  \left|  \mathbb{F} \right|  + 1 \right)  $.

\item[\textbf{(e)}] Conclude that $\left|  \mathbb{F} \right|  $ is odd.
\end{enumerate}

[\textbf{Hint:} For part \textbf{(d)}, argue that each nonzero square in
$\mathbb{F}$ can be written as $\alpha^{2}$ for exactly two $\alpha
\in\mathbb{F}$.]

\subsection{Solution}

We have assumed that $\mathbb{F}$ is a field. Hence, $\mathbb{F}$ is a
commutative skew field (by the definition of a field). Every nonzero element
of $\mathbb{F}$ is invertible (since $\mathbb{F}$ is a skew field).

\bigskip

\textbf{(a)} Let $a,b\in\mathbb{F}$ be such that $ab=0$. We must prove that
$a=0$ or $b=0$.

Assume the contrary. Thus, neither $a=0$ nor $b=0$ holds. In other words, we
have $a\neq0$ and $b\neq0$. Thus, the elements $a$ and $b$ of $\mathbb{F}$ are
nonzero, and therefore invertible (since every nonzero element of $\mathbb{F}$
is invertible). Hence, their inverses $a^{-1}$ and $b^{-1}$ are well-defined.
Comparing the equalities $\underbrace{a^{-1}a}_{=1}b=b$ and $a^{-1}%
\underbrace{ab}_{=0}=a^{-1}0=0$, we obtain $b=0$. This contradicts $b\neq0$.
This contradiction shows that our assumption was false. This completes the
solution to part \textbf{(a)} of the exercise.

\bigskip

\textbf{(b)} Let $a,b\in\mathbb{F}$ satisfy $a^{2}=b^{2}$. We must prove that
$a=b$ or $a=-b$.

Since $\mathbb{F}$ is commutative, we have $ab=ba$. Now, multiplying out
$\left(  a-b\right)  \left(  a+b\right)  $ (by applying the distributivity
laws several times), we obtain%
\[
\left(  a-b\right)  \left(  a+b\right)  =\underbrace{aa}_{=a^{2}=b^{2}%
}+\underbrace{ab}_{=ba}-ba-\underbrace{bb}_{=b^{2}}=b^{2}+ba-ba-b^{2}=0.
\]
Thus, part \textbf{(a)} of this exercise (applied to $a-b$ and $a+b$ instead
of $a$ and $b$) shows that $a-b=0$ or $a+b=0$. In other words, $a=b$ or
$a=-b$. Thus, part \textbf{(b)} of the exercise is solved.

\bigskip

\textbf{(c)} Let $a\in\mathbb{F}$ be nonzero. We must prove that $a\neq-a$.

Assume the contrary. Thus, $a=-a$, so that $a+a=0$. Now,%
\[
\underbrace{\left(  2\cdot1_{\mathbb{F}}\right)  }_{=1_{\mathbb{F}%
}+1_{\mathbb{F}}}a=\left(  1_{\mathbb{F}}+1_{\mathbb{F}}\right)
a=\underbrace{1_{\mathbb{F}}a}_{=a}+\underbrace{1_{\mathbb{F}}a}_{=a}=a+a=0.
\]


The element $2\cdot1_{\mathbb{F}}$ of $\mathbb{F}$ is nonzero (since
$2\cdot1_{\mathbb{F}}\neq0_{\mathbb{F}}$), and thus invertible (since every
nonzero element of $\mathbb{F}$ is invertible). Hence, it has a well-defined
inverse $\left(  2\cdot1_{\mathbb{F}}\right)  ^{-1}$.

Now,%
\[
\left(  2\cdot1_{\mathbb{F}}\right)  ^{-1}\cdot\underbrace{\left(
2\cdot1_{\mathbb{F}}\right)  a}_{=0}=\left(  2\cdot1_{\mathbb{F}}\right)
^{-1}\cdot0=0.
\]
Comparing this with $\underbrace{\left(  2\cdot1_{\mathbb{F}}\right)
^{-1}\cdot\left(  2\cdot1_{\mathbb{F}}\right)  }_{=1}a=1a=a$, we obtain $a=0$.
This contradicts the fact that $a$ is nonzero. This contradiction shows that
our assumption was false. Hence, $a\neq-a$. Thus, part \textbf{(c)} of the
exercise is solved.

\bigskip

\textbf{(d)} We have the following:

\begin{statement}
\textit{Claim 1:} Let $c\in\mathbb{F}$. Then:

\textbf{(i)} If $c$ is a nonzero square, then%
\[
\left\vert \left\{  d\in\mathbb{F}\ \mid\ c=d^{2}\right\}  \right\vert =2.
\]


\textbf{(ii)} If $c$ is not a square, then%
\[
\left\vert \left\{  d\in\mathbb{F}\ \mid\ c=d^{2}\right\}  \right\vert =0.
\]


\textbf{(iii)} If $c=0$, then%
\[
\left\vert \left\{  d\in\mathbb{F}\ \mid\ c=d^{2}\right\}  \right\vert =1.
\]

\end{statement}

[\textit{Proof of Claim 1:} \textbf{(i)} Assume that $c$ is a nonzero square.
Thus, there exists a $g\in\mathbb{F}$ such that $c=g^{2}$ (since $c$ is a
square). Consider this $g$. Moreover,
\[
\left(  -g\right)  ^{2}=\left(  -g\right)  \left(  -g\right)
=-\underbrace{\left(  \left(  -g\right)  g\right)  }_{=-gg}=-\left(
-gg\right)  =gg=g^{2}=c
\]
(since $c=g^{2}$). Hence, $c=\left(  -g\right)  ^{2}$.

If we had $g=0$, then we would have $c=\underbrace{g}_{=0}\ ^{2}=0^{2}=0$,
which would contradict our assumption that $c$ is nonzero. Hence, we cannot
have $g=0$. Thus, $g$ is nonzero. Therefore, $g\neq-g$ (by part \textbf{(c)}
of this exercise, applied to $a=g$). Hence, the elements $g$ and $-g$ of
$\mathbb{F}$ are distinct. Thus, $\left\vert \left\{  g,-g\right\}
\right\vert =2$.

But $g\in\left\{  d\in\mathbb{F}\ \mid\ c=d^{2}\right\}  $ (since
$g\in\mathbb{F}$ and $c=g^{2}$) and $-g\in\left\{  d\in\mathbb{F}%
\ \mid\ c=d^{2}\right\}  $ (since $-g\in\mathbb{F}$ and $c=\left(  -g\right)
^{2}$). Combining these two facts, we obtain
\begin{equation}
\left\{  g,-g\right\}  \subseteq\left\{  d\in\mathbb{F}\ \mid\ c=d^{2}%
\right\}  . \label{sol.ffields.squares1.c1.pf.1}%
\end{equation}


On the other hand, let us prove that $\left\{  d\in\mathbb{F}\ \mid
\ c=d^{2}\right\}  \subseteq\left\{  g,-g\right\}  $. Indeed, let
$a\in\left\{  d\in\mathbb{F}\ \mid\ c=d^{2}\right\}  $. Thus, $a$ is a
$d\in\mathbb{F}$ such that $c=d^{2}$. In other words, $a$ is an element of
$\mathbb{F}$ and satisfies $c=a^{2}$. Hence, $a^{2}=c=g^{2}$. Thus, part
\textbf{(b)} of this exercise (applied to $b=g$) yields that $a=g$ or $a=-g$.
In other words, $a\in\left\{  g,-g\right\}  $. Now, forget that we fixed $a$.
We thus have shown that $a\in\left\{  g,-g\right\}  $ for each $a\in\left\{
d\in\mathbb{F}\ \mid\ c=d^{2}\right\}  $. In other words, $\left\{
d\in\mathbb{F}\ \mid\ c=d^{2}\right\}  \subseteq\left\{  g,-g\right\}  $.
Combining this with \eqref{sol.ffields.squares1.c1.pf.1}, we obtain%
\[
\left\{  d\in\mathbb{F}\ \mid\ c=d^{2}\right\}  =\left\{  g,-g\right\}  .
\]
Hence,%
\[
\left\vert \left\{  d\in\mathbb{F}\ \mid\ c=d^{2}\right\}  \right\vert
=\left\vert \left\{  g,-g\right\}  \right\vert =2.
\]
This proves Claim 1 \textbf{(i)}.

\textbf{(ii)} Assume that $c$ is not a square. Then, there exists no
$\alpha\in\mathbb{F}$ such that $c=\alpha^{2}$ (by the definition of a
square). In other words, there exists no $d\in\mathbb{F}$ such that $c=d^{2}$
(here, we have renamed the index $\alpha$ as $d$). In other words, $\left\{
d\in\mathbb{F}\ \mid\ c=d^{2}\right\}  =\varnothing$. Hence, $\left\vert
\left\{  d\in\mathbb{F}\ \mid\ c=d^{2}\right\}  \right\vert =\left\vert
\varnothing\right\vert =0$. This proves Claim 1 \textbf{(ii)}.

\textbf{(iii)} Assume that $c=0$. Then, $0\in\left\{  d\in\mathbb{F}%
\ \mid\ c=d^{2}\right\}  $ (since $0\in\mathbb{F}$ and $c=0=0^{2}$) and thus
$\left\{  0\right\}  \subseteq\left\{  d\in\mathbb{F}\ \mid\ c=d^{2}\right\}
$.

On the other hand, let us show that $\left\{  d\in\mathbb{F}\ \mid
\ c=d^{2}\right\}  \subseteq\left\{  0\right\}  $.

Indeed, let $a\in\left\{  d\in\mathbb{F}\ \mid\ c=d^{2}\right\}  $. Then, $a$
is a $d\in\mathbb{F}$ such that $c=d^{2}$. In other words, $a$ is an element
of $\mathbb{F}$ and satisfies $c=a^{2}$. Hence, $aa=a^{2}=c=0$. Thus, part
\textbf{(a)} of this exercise (applied to $b=a$) yields that $a=0$ or $a=0$.
In other words, $a=0$. In other words, $a\in\left\{  0\right\}  $. Now, forget
that we fixed $a$. We thus have shown that $a\in\left\{  0\right\}  $ for each
$a\in\left\{  d\in\mathbb{F}\ \mid\ c=d^{2}\right\}  $. In other words,
$\left\{  d\in\mathbb{F}\ \mid\ c=d^{2}\right\}  \subseteq\left\{  0\right\}
$. Combining this with $\left\{  0\right\}  \subseteq\left\{  d\in
\mathbb{F}\ \mid\ c=d^{2}\right\}  $, we obtain $\left\{  d\in\mathbb{F}%
\ \mid\ c=d^{2}\right\}  =\left\{  0\right\}  $. Hence, $\left\vert \left\{
d\in\mathbb{F}\ \mid\ c=d^{2}\right\}  \right\vert =\left\vert \left\{
0\right\}  \right\vert =1$. This proves Claim 1 \textbf{(iii)}.]

Now, let us count all pairs $\left(  c,d\right)  \in\mathbb{F}\times
\mathbb{F}$ satisfying $c=d^{2}$. We shall count these pairs in two ways:

\begin{itemize}
\item The first way is to split this count according to the value of $c$ (that
is, first count all such pairs $\left(  c,d\right)  $ with a given $c$, and
then sum the result up over all $c\in\mathbb{F}$). Thus, we find%
\begin{align*}
&  \left(  \text{the number of all }\left(  c,d\right)  \in\mathbb{F}%
\times\mathbb{F}\text{ such that }c=d^{2}\right) \\
&  =\sum_{c\in\mathbb{F}}\underbrace{\left(  \text{the number of all }%
d\in\mathbb{F}\text{ such that }c=d^{2}\right)  }_{=\left\vert \left\{
d\in\mathbb{F}\ \mid\ c=d^{2}\right\}  \right\vert }\\
&  =\sum_{c\in\mathbb{F}}\left\vert \left\{  d\in\mathbb{F}\ \mid
\ c=d^{2}\right\}  \right\vert \\
&  =\sum_{\substack{c\in\mathbb{F};\\c=0}}\underbrace{\left\vert \left\{
d\in\mathbb{F}\ \mid\ c=d^{2}\right\}  \right\vert }_{\substack{=1\\\text{(by
Claim 1 \textbf{(iii)})}}}+\sum_{\substack{c\in\mathbb{F};\\c\text{ is a
nonzero}\\\text{square}}}\underbrace{\left\vert \left\{  d\in\mathbb{F}%
\ \mid\ c=d^{2}\right\}  \right\vert }_{\substack{=2\\\text{(by Claim 1
\textbf{(i)})}}}\\
&  \qquad+\sum_{\substack{c\in\mathbb{F};\\c\text{ is not a}\\\text{square}%
}}\underbrace{\left\vert \left\{  d\in\mathbb{F}\ \mid\ c=d^{2}\right\}
\right\vert }_{\substack{=0\\\text{(by Claim 1 \textbf{(ii)})}}}\\
&  \qquad\left(
\begin{array}
[c]{c}%
\text{because each }c\in\mathbb{F}\text{ satisfies exactly one of the three
statements}\\
\text{\textquotedblleft}c=0\text{\textquotedblright, \textquotedblleft}c\text{
is a nonzero square\textquotedblright\ and \textquotedblleft}c\text{ is not a
square\textquotedblright}%
\end{array}
\right) \\
&  =\underbrace{\sum_{\substack{c\in\mathbb{F};\\c=0}}1}%
_{\substack{=1\\\text{(since this sum has}\\\text{exactly one addend)}%
}}+\underbrace{\sum_{\substack{c\in\mathbb{F};\\c\text{ is a nonzero}%
\\\text{square}}}2}_{=2\cdot\left(  \text{the number of nonzero squares in
}\mathbb{F}\right)  }+\underbrace{\sum_{\substack{c\in\mathbb{F};\\c\text{ is
not a}\\\text{square}}}0}_{=0}\\
&  =1+2\cdot\left(  \text{the number of nonzero squares in }\mathbb{F}\right)
+0\\
&  =1+2\cdot\left(  \text{the number of nonzero squares in }\mathbb{F}\right)
.
\end{align*}


\item The second way is to split this count according to the value of $d$
(that is, first count all such pairs $\left(  c,d\right)  $ with a given $d$,
and then sum the result up over all $d\in\mathbb{F}$). Thus, we find%
\begin{align*}
&  \left(  \text{the number of all }\left(  c,d\right)  \in\mathbb{F}%
\times\mathbb{F}\text{ such that }c=d^{2}\right) \\
&  =\sum_{d\in\mathbb{F}}\underbrace{\left(  \text{the number of all }%
c\in\mathbb{F}\text{ such that }c=d^{2}\right)  }_{\substack{=1\\\text{(since
there is exactly one }c\in\mathbb{F}\text{ such that }c=d^{2}\text{ (namely,
}c=d^{2}\text{))}}}\\
&  =\sum_{d\in\mathbb{F}}1=\left\vert \mathbb{F}\right\vert \cdot1=\left\vert
\mathbb{F}\right\vert .
\end{align*}

\end{itemize}

Comparing these two equalities, we obtain%
\[
\left\vert \mathbb{F}\right\vert =1+2\cdot\left(  \text{the number of nonzero
squares in }\mathbb{F}\right)  .
\]
Solving this for $\left(  \text{the number of nonzero squares in }%
\mathbb{F}\right)  $, we find%
\[
\left(  \text{the number of nonzero squares in }\mathbb{F}\right)
=\dfrac{\left\vert \mathbb{F}\right\vert -1}{2}.
\]


Now, there are two kinds of squares in $\mathbb{F}$: namely, the nonzero
squares (of which there are exactly $\dfrac{\left\vert \mathbb{F}\right\vert
-1}{2}$ many, as we just proved) and the zero squares (of which there is only
$1$, namely $0^{2}=0$). Thus, the total number of squares in $\mathbb{F}$ is
$\dfrac{\left\vert \mathbb{F}\right\vert -1}{2}+1=\dfrac{\left\vert
\mathbb{F}\right\vert +1}{2}=\dfrac{1}{2}\left(  \left\vert \mathbb{F}%
\right\vert +1\right)  $. This solves part \textbf{(d)} of the exercise.

\bigskip

\textbf{(e)} Part \textbf{(d)} of this exercise shows that the number of
squares in $\mathbb{F}$ is $\dfrac{1}{2}\left(  \left\vert \mathbb{F}%
\right\vert +1\right)  $. Thus,%
\[
\dfrac{1}{2}\left(  \left\vert \mathbb{F}\right\vert +1\right)  =\left(
\text{the number of squares in }\mathbb{F}\right)  \in\mathbb{N}%
\]
(since a number that counts something is always $\in\mathbb{N}$). Therefore,
$\dfrac{1}{2}\left(  \left\vert \mathbb{F}\right\vert +1\right)  \in
\mathbb{N}\subseteq\mathbb{Z}$, so that the integer $\left\vert \mathbb{F}%
\right\vert +1$ is even. This shows that $\left\vert \mathbb{F}\right\vert $
is odd. This solves part \textbf{(e)} of the exercise.

%----------------------------------------------------------------------------------------
%	EXERCISE 6
%----------------------------------------------------------------------------------------
\rule{\linewidth}{0.3pt} \\[0.4cm]

\section{Exercise 6: The characteristic of a field}

\subsection{Problem}

Let $\mathbb{F}$ be a field. Recall that we have defined $na$ to mean
$\underbrace{a + a + \cdots+ a}_{n \text{ times}}$ whenever $n \in\mathbb{N}$
and $a \in\mathbb{F}$.

Assume that there exists a positive integer $n$ such that $n \cdot
1_{\mathbb{F}} = 0$. Let $p$ be the \textbf{smallest} such $n$.

Prove that $p$ is prime.

[\textbf{Hint:} $\left(  a \cdot1_{\mathbb{F}} \right)  \cdot\left(  b
\cdot1_{\mathbb{F}} \right)  = ab \cdot1_{\mathbb{F}}$ for all $a, b
\in\mathbb{N}$.]

\subsection{Remark}

The $p$ we just defined is called the \textit{characteristic} of the field
$\mathbb{F}$ when it exists. (Otherwise, the characteristic of the field
$\mathbb{F}$ is defined to be $0$.)

Thus, for each prime $p$, the finite field $\mathbb{Z} / p$, as well as the
finite field of size $p^{2}$ that we constructed in class, have characteristic
$p$.

\subsection{Solution sketch}

We have assumed that $\mathbb{F}$ is a field. Hence, $\mathbb{F}$ is a
commutative skew field (by the definition of a field). We have $0_{\mathbb{F}%
}\neq1_{\mathbb{F}}$ (since $\mathbb{F}$ is a skew field).

We have defined $p$ to be the \textbf{smallest} positive integer $n$ such that
$n\cdot1_{\mathbb{F}}=0$. Thus, $p$ is a positive integer which itself
satisfies $p\cdot1_{\mathbb{F}}=0$. Furthermore, if $n$ is a positive integer
such that $n\cdot1_{\mathbb{F}}=0$, then%
\begin{equation}
n\geq p \label{sol.field.char.prime.1}%
\end{equation}
(since $p$ is the \textbf{smallest} positive integer $n$ such that
$n\cdot1_{\mathbb{F}}=0$).

If we had $p=1$, then we would have $p\cdot1_{\mathbb{F}}=1\cdot1_{\mathbb{F}%
}=1_{\mathbb{F}}\neq0_{\mathbb{F}}$ (since $0_{\mathbb{F}}\neq1_{\mathbb{F}}%
$), which would contradict $p\cdot1_{\mathbb{F}}=0=0_{\mathbb{F}}$. Thus, we
cannot have $p=1$. Therefore, we have $p>1$ (since $p$ is a positive integer).

We shall now show that the only positive divisors of $p$ are $1$ and $p$.
Indeed, assume the contrary. Thus, $p$ has a positive divisor other than $1$
and $p$. Consider such a divisor, and denote it by $d$. Thus, $d$ is a
positive divisor of $p$ that is distinct from $1$ and $p$. In other words, $d$
is a positive divisor of $p$ and satisfies $d\neq1$ and $d\neq p$. We have
$d\leq p$ (since $d$ is a positive divisor of the positive integer $p$).
Combining this with $d\neq p$, we obtain $d<p$. Also, $d\in\mathbb{Z}$ (since
$d$ is an integer) and $\dfrac{p}{d}\in\mathbb{Z}$ (since $d$ is a divisor of
$p$).

Now, for all $a,b\in\mathbb{Z}$, we have%
\[
\left(  a\cdot1_{\mathbb{F}}\right)  \cdot\left(  b\cdot1_{\mathbb{F}}\right)
=a\cdot\underbrace{\left(  1_{\mathbb{F}}\cdot\left(  b\cdot1_{\mathbb{F}%
}\right)  \right)  }_{=b\cdot1_{\mathbb{F}}}=a\cdot\left(  b\cdot
1_{\mathbb{F}}\right)  =ab\cdot1_{\mathbb{F}}.
\]
Applying this to $a=d$ and $b=\dfrac{p}{d}$, we obtain%
\[
\left(  d\cdot1_{\mathbb{F}}\right)  \cdot\left(  \dfrac{p}{d}\cdot
1_{\mathbb{F}}\right)  =\underbrace{d\cdot\dfrac{p}{d}}_{=p}\cdot
1_{\mathbb{F}}=p\cdot1_{\mathbb{F}}=0.
\]
Thus, Exercise 5 \textbf{(a)} (applied to $a=d\cdot1_{\mathbb{F}}$ and
$b=\dfrac{p}{d}\cdot1_{\mathbb{F}}$) shows that $d\cdot1_{\mathbb{F}}=0$ or
$\dfrac{p}{d}\cdot1_{\mathbb{F}}=0$.

If we had $d\cdot1_{\mathbb{F}}=0$, then we would have $d\geq p$ (by
\eqref{sol.field.char.prime.1}, applied to $n=d$), which would contradict
$d<p$. Hence, we cannot have $d\cdot1_{\mathbb{F}}=0$. Thus, we have
$\dfrac{p}{d}\cdot1_{\mathbb{F}}=0$ (since $d\cdot1_{\mathbb{F}}=0$ or
$\dfrac{p}{d}\cdot1_{\mathbb{F}}=0$). But $\dfrac{p}{d}$ is an integer (since
$\dfrac{p}{d}\in\mathbb{Z}$) and is positive (since $p$ and $d$ are positive);
thus, $\dfrac{p}{d}$ is a positive integer. Hence,
\eqref{sol.field.char.prime.1} (applied to $n=\dfrac{p}{d}$) yields $\dfrac
{p}{d}\geq p$ (since $\dfrac{p}{d}\cdot1_{\mathbb{F}}=0$). Since $d$ is
positive, we can multiply this inequality by $d$, and thus obtain $p\geq pd$.
Since $p$ is positive, we can divide this inequality by $p$, and thus obtain
$1\geq d$. Hence, $d=1$ (since $d$ is a positive integer). This contradicts
$d\neq1$.

This contradiction shows that our assumption was false. Hence, the only
positive divisors of $p$ are $1$ and $p$. Thus, $p$ is a prime (since $p$ is
an integer satisfying $p>1$). Qed.

\subsection{Remark}

We have never used the commutativity of multiplication (in $\mathbb{F}$) in
the above proof. Thus, we can replace \textquotedblleft
field\textquotedblright\ by \textquotedblleft skew field\textquotedblright\ in
this exercise.

\begin{thebibliography}{99999999}                                                                                         %


%Feel free to add your sources -- or copy some from the source code
%of the class notes ( http://www.cip.ifi.lmu.de/~grinberg/t/19s/notes.tex ).


%This is the bibliography: The list of papers/books/articles/blogs/...
%cited. The syntax is: "\bibitem[name]{tag}Reference",
%where "name" is the name that will appear in the compiled
%bibliography, and "tag" is the tag by which you will refer to
%the source in the TeX file. For example, the following source
%has name "GrKnPa94" (so you will see it referenced as
%"[GrKnPa94]" in the compiled PDF) and tag "GKP" (so you
%can cite it by writing "\cite{GKP}").


%I've commented the below references out, since I'm not citing
%them above.


%\bibitem[GrKnPa94]{GKP}Ronald L. Graham, Donald E. Knuth, Oren Patashnik,
%\textit{Concrete Mathematics, Second Edition}, Addison-Wesley 1994.\\
%See \url{https://www-cs-faculty.stanford.edu/~knuth/gkp.html} for errata.


\bibitem[Grinbe19]{detnotes}Darij Grinberg, \textit{Notes on the combinatorial
fundamentals of algebra}, 10 January 2019. \newline%
\url{http://www.cip.ifi.lmu.de/~grinberg/primes2015/sols.pdf} \newline The
numbering of theorems and formulas in this link might shift when the project
gets updated; for a ``frozen'' version whose numbering is guaranteed to match
that in the citations above, see
\url{https://github.com/darijgr/detnotes/releases/tag/2019-01-10} .
\end{thebibliography}


\end{document}