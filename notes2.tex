\documentclass[numbers=enddot,12pt,final,onecolumn,notitlepage]{scrartcl}%
\usepackage[headsepline,footsepline,manualmark]{scrlayer-scrpage}
\usepackage[all,cmtip]{xy}
\usepackage{amssymb}
\usepackage{amsmath}
\usepackage{amsthm}
\usepackage{framed}
\usepackage{comment}
\usepackage{color}
\usepackage{hyperref}
\usepackage[sc]{mathpazo}
\usepackage[T1]{fontenc}
\usepackage{tikz}
\usepackage{needspace}
\usepackage{tabls}
\usepackage{wasysym}
%TCIDATA{OutputFilter=latex2.dll}
%TCIDATA{Version=5.50.0.2960}
%TCIDATA{LastRevised=Friday, January 25, 2019 12:09:34}
%TCIDATA{SuppressPackageManagement}
%TCIDATA{<META NAME="GraphicsSave" CONTENT="32">}
%TCIDATA{<META NAME="SaveForMode" CONTENT="1">}
%TCIDATA{BibliographyScheme=Manual}
%TCIDATA{Language=American English}
%BeginMSIPreambleData
\providecommand{\U}[1]{\protect\rule{.1in}{.1in}}
%EndMSIPreambleData
\usetikzlibrary{arrows}
\newcounter{exer}
\newcounter{exera}
\numberwithin{exer}{section}
\theoremstyle{definition}
\newtheorem{theo}{Theorem}[section]
\newenvironment{theorem}[1][]
{\begin{theo}[#1]\begin{leftbar}}
{\end{leftbar}\end{theo}}
\newtheorem{lem}[theo]{Lemma}
\newenvironment{lemma}[1][]
{\begin{lem}[#1]\begin{leftbar}}
{\end{leftbar}\end{lem}}
\newtheorem{prop}[theo]{Proposition}
\newenvironment{proposition}[1][]
{\begin{prop}[#1]\begin{leftbar}}
{\end{leftbar}\end{prop}}
\newtheorem{defi}[theo]{Definition}
\newenvironment{definition}[1][]
{\begin{defi}[#1]\begin{leftbar}}
{\end{leftbar}\end{defi}}
\newtheorem{remk}[theo]{Remark}
\newenvironment{remark}[1][]
{\begin{remk}[#1]\begin{leftbar}}
{\end{leftbar}\end{remk}}
\newtheorem{coro}[theo]{Corollary}
\newenvironment{corollary}[1][]
{\begin{coro}[#1]\begin{leftbar}}
{\end{leftbar}\end{coro}}
\newtheorem{conv}[theo]{Convention}
\newenvironment{convention}[1][]
{\begin{conv}[#1]\begin{leftbar}}
{\end{leftbar}\end{conv}}
\newtheorem{quest}[theo]{Question}
\newenvironment{question}[1][]
{\begin{quest}[#1]\begin{leftbar}}
{\end{leftbar}\end{quest}}
\newtheorem{warn}[theo]{Warning}
\newenvironment{conclusion}[1][]
{\begin{warn}[#1]\begin{leftbar}}
{\end{leftbar}\end{warn}}
\newtheorem{conj}[theo]{Conjecture}
\newenvironment{conjecture}[1][]
{\begin{conj}[#1]\begin{leftbar}}
{\end{leftbar}\end{conj}}
\newtheorem{exam}[theo]{Example}
\newenvironment{example}[1][]
{\begin{exam}[#1]\begin{leftbar}}
{\end{leftbar}\end{exam}}
\newtheorem{exmp}[exer]{Exercise}
\newenvironment{exercise}[1][]
{\begin{exmp}[#1]\begin{leftbar}}
{\end{leftbar}\end{exmp}}
\newenvironment{statement}{\begin{quote}}{\end{quote}}
\iffalse
\newenvironment{proof}[1][Proof]{\noindent\textbf{#1.} }{\ \rule{0.5em}{0.5em}}
\newenvironment{question}[1][Question]{\noindent\textbf{#1.} }{\ \rule{0.5em}{0.5em}}
\fi
\let\sumnonlimits\sum
\let\prodnonlimits\prod
\let\cupnonlimits\bigcup
\let\capnonlimits\bigcap
\renewcommand{\sum}{\sumnonlimits\limits}
\renewcommand{\prod}{\prodnonlimits\limits}
\renewcommand{\bigcup}{\cupnonlimits\limits}
\renewcommand{\bigcap}{\capnonlimits\limits}
\setlength\tablinesep{3pt}
\setlength\arraylinesep{3pt}
\setlength\extrarulesep{3pt}
\voffset=0cm
\hoffset=-0.7cm
\setlength\textheight{22.5cm}
\setlength\textwidth{15.5cm}
\newcommand\arxiv[1]{\href{http://www.arxiv.org/abs/#1}{\texttt{arXiv:#1}}}
\newenvironment{verlong}{}{}
\newenvironment{vershort}{}{}
\newenvironment{noncompile}{}{}
\excludecomment{verlong}
\includecomment{vershort}
\excludecomment{noncompile}
\newcommand{\CC}{\mathbb{C}}
\newcommand{\RR}{\mathbb{R}}
\newcommand{\QQ}{\mathbb{Q}}
\newcommand{\NN}{\mathbb{N}}
\newcommand{\ZZ}{\mathbb{Z}}
\newcommand{\id}{\operatorname{id}}
\newcommand{\lcm}{\operatorname{lcm}}
\newcommand{\rev}{\operatorname{rev}}
\newcommand{\powset}[2][]{\ifthenelse{\equal{#2}{}}{\mathcal{P}\left(#1\right)}{\mathcal{P}_{#1}\left(#2\right)}}
\newcommand{\set}[1]{\left\{ #1 \right\}}
\newcommand{\abs}[1]{\left| #1 \right|}
\newcommand{\tup}[1]{\left( #1 \right)}
\newcommand{\ive}[1]{\left[ #1 \right]}
\newcommand{\floor}[1]{\left\lfloor #1 \right\rfloor}
\newcommand{\lf}[2]{#1^{\underline{#2}}}
\newcommand{\underbrack}[2]{\underbrace{#1}_{\substack{#2}}}
\newcommand{\horrule}[1]{\rule{\linewidth}{#1}}
\newcommand{\nnn}{\nonumber\\}
\newcommand{\sslash}{\mathbin{/\mkern-6mu/}}
\ihead{Math 4281 notes}
\ohead{page \thepage}
\cfoot{}
\begin{document}

\title{UMN Spring 2019 Math 4281 notes}
\author{Darij Grinberg}
\date{
%TCIMACRO{\TeXButton{today}{\today} }%
%BeginExpansion
\today
%EndExpansion
}
\maketitle
\tableofcontents

\section{Introduction}

This file will contain the notes from the Math 4281 class (``Introduction to
Modern Algebra'') I am teaching at UMN in Spring 2019. I will type the first
draft directly in the classroom, and subsequently expand it into proper
writing. Occasionally, I will also add extra sections not covered in class.

The website of the class is
\url{http://www-users.math.umn.edu/~dgrinber/19s/index.html} ; you will find
homework sets there.

\subsection{Organisation (this will be moved into the syllabus)}

\textbf{Office hours:} See
\href{http://www-users.math.umn.edu/~dgrinber/19s/syll.pdf}{the syllabus}.

\textbf{Homework:}

Not sure when HW1 will be due. Depends on when I get a grader.

Expect HW0 soon, but this is not an actual homework, but rather a ``demo''
file that you should read or glance over. It should give you an idea of how
certain kinds of proofs can be written and about the $\sum$ notation. You can
also use it as a template if you want to write your homework in \LaTeX\ (which
I highly encourage).

\subsection{Literature}

Many books have been written about abstract algebra. I have only a passing
familiarity with most of them. Some of the ``bibles'' of the subject (bulky
texts covering lots of material) are Dummit/Foote \cite{Dummit-Foote}, Knapp
\cite{Knapp1} and \cite{Knapp2} (both freely available), van der Waerden
\cite{Waerden1} and \cite{Waerden2} (one of the oldest texts on modern
algebra, thus rather dated, but still as readable as ever).
%Two other textbooks are Bosch \cite{Bosch} and Artin \cite{Artin}.


Of course, any book longer than 200 pages likely goes further than our course
will (unless it is full of details or solved exercises or printed in really
large letters). Thus, let me recommend some more introductory sources.
Siksek's lecture notes \cite{Siksek} are a readable introduction that is a lot
more amusing than I had ever expected an algebra text to be. Goodman's free
book \cite{Goodman} combines introductory material with geometric motivation
and applications, such as the classification of regular polyhedra and
2-dimensional crystals. In a sense, it is a great complement to our
ungeometric course. Pinter's \cite{Pinter} often gets used in classes like
ours. Armstrong's notes \cite{Armstrong} cover a significant part of what we
do (and he will likely have notes for a second course written up by the end of
this semester).

Keith Conrad's blurbs \cite{Conrad*} are not a book, as they only cover
selected topics. But at pretty much every topic they cover, they are one of
the best sources (clear, full of examples, and often going fairly deep). We
shall follow one of them particularly closely: the one on Gaussian integers
\cite{Conrad-Gauss}.

We will use some basic linear algebra, all of which can be found in Hefferon's
book \cite{Hefferon} (but we won't need all of this book). As far as
determinants are concerned, we will briefly build up their theory; we refer to
\cite[Section 12 \& Appendix B]{Strickland} for proofs (and to \cite[Chapter
6]{detnotes} for a really detailed and formal treatment).

This course will begin (after some motivating questions) with a survey of
elementary number theory. This is in itself a deep subject (despite the name)
with a long history (\href{https://en.wikipedia.org/wiki/Plimpton_322}{perhaps
as old as mathematics}), and of course we will just scratch the surface. Books
like \cite{NiZuMo}, \cite{Burton} and \cite{Uspensky-Heaslet} cover a lot more
than we can do. The Gallier/Quaintance survey \cite{Gallier-RSA} covers a good
amount of basics and more.

We assume that the reader is familiar with the commonplaces of mathematical
argumentation, such as induction (including strong induction),
\textquotedblleft WLOG\textquotedblright\ arguments, proof by contradiction,
summation signs ($\sum$) and polynomials (a vague notion of polynomials will
suffice; we will give a precise definition when it becomes necessary). If not,
several texts can be helpful in achieving such familiarity: e.g.,
\cite[particularly Chapters 1--5]{LeLeMe}, \cite{Hammack}, \cite{Day}.

\begin{center}
\textbf{2019-01-23 lecture}
\end{center}

\subsection{The plan}

The material I am going to cover is mostly standard. However, the order in
which I will go through it is somewhat unusual: I will spend a lot of time
studying the basic examples before defining abstract notions such as
``group'', ``monoid'', ``ring'' and ``field''. This way, once I come to these
notions, you'll already have many examples to work with. (Don't be fooled by
the word ``example'': We will prove a lot about them, much of which is neither
straightforward nor easy.)

First, I will show some motivating questions that are easy to state yet
require abstract algebra to prove. We will hopefully see their answers by the
end of this class. (Some of them can also be answered elementarily, without
using abstract algebra, but such answers usually take more work and are harder
to find.)

\subsection{\label{subsect.intro.sum-of-2sq}Motivation: $n=x^{2}+y^{2}$}

A \textit{perfect square} means the square of an integer. Thus, the perfect
squares are
\[
0^{2} = 0, \qquad1^{2} = 1, \qquad2^{2} = 4, \qquad3^{2} = 9, \qquad4^{2} =
16, \qquad\ldots.
\]


Here is an old problem (first solved by Pierre de Fermat in 1640, but
apparently already studied by Diophantus in the 3rd Century):

\begin{question}
\label{quest.intro.sum-of-2sq.1} What integers can be written as sums of two
perfect squares?
\end{question}

For example, $5$ can be written in this way, since $5=2^{2}+1^{2}$.

So can $4$, since $4=2^{2}+0^{2}$. (Keep in mind that $0$ is a perfect square.)

However, $7$ cannot be written in this way. In fact, if we had $7 = a^{2} +
b^{2}$ for two integers $a$ and $b$, then $a^{2}$ and $b^{2}$ would have to be
$\leq7$ (since $a^{2}$ and $b^{2}$ are always $\geq0$, no matter what sign $a$
and $b$ have); but the only perfect squares that are $\leq7$ are $0,1,4$, and
there is no way to write $7$ as a sum of two of these perfect squares (just
check all the possibilities).

For a similar but simpler reason, no negative number can be written as a sum
of two perfect squares.

We can of course approach Question~\ref{quest.intro.sum-of-2sq.1} using a
computer: It is very easy to check, for a given integer $n$, whether $n$ is a
sum of two perfect squares. (Just check all possibilities for $a$ and $b$ for
the validity of the equation $n = a^{2} + b^{2}$. You only need to try $a$ and
$b$ belonging to $\left\{  0, 1, \ldots, \left\lfloor \sqrt{n} \right\rfloor
\right\}  $, where $\left\lfloor y \right\rfloor $ (for a real number $y$)
denotes the smallest integer that is less or equal than $y$ (also known as
``$y$ rounded down'').) If you do this, you will see that among the first
$101$ nonnegative integers, the ones that can be written as sums of two
perfect squares are precisely
\begin{align*}
&  0, 1, 2, 4, 5, 8, 9, 10, 13, 16, 17, 18, 20, 25, 26, 29,\\
&  32, 34, 36, 37, 40, 41, 45, 49, 50, 52, 53, 58, 61, 64,\\
&  65, 68, 72, 73, 74, 80, 81, 82, 85, 89, 90, 97, 98, 100 .
\end{align*}
Having this data, you can look up the sequence in \href{https://oeis.org/}{the
Online Encyclopedia of Integer Sequences (short OEIS)}, and see that the
sequence of these integers is known as \href{https://oeis.org/A001481}{OEIS
Sequence A001481}. In the ``Comments'' field, you can read a lot of what is
known about it (albeit in telegraphic style).

For example, one of the comments says ``Closed under multiplication''. This is
short for ``if you multiply two entries of the sequence, then the product will
again be an entry of the sequence''. In other words, if you multiply two
integers that are sums of two perfect squares, then you get another sum of two
perfect squares. Why is this so?

It turns out that there is a ``simple'' reason for this: the identity
\[
\left(  a^{2}+b^{2}\right)  \left(  c^{2}+d^{2}\right)  =\left(  ad+bc\right)
^{2}+\left(  ac-bd\right)  ^{2} ,
\]
which holds for arbitrary reals $a,b,c,d$ (and thus, in particular, for
integers). This is known as
\href{https://en.wikipedia.org/wiki/Brahmagupta-Fibonacci_identity}{the
Brahmagupta-Fibonacci identity}, and of course can easily be proven by
expanding both sides. But how would you come up with such an identity?

If you stare at the above sequence long enough, you may also discover another
pattern: An integer of the form $4k+3$ with integer $k$ (that is, an integer
that is larger by $3$ than a multiple of $4$) can never be written as a sum of
two perfect squares. (Thus, $3, 7, 11, 15, 19, 23, \ldots$ cannot be written
in this way.) This does not account for all integers that cannot be written in
this way, but it does provide some clues to the answer that we will later see.
In order to prove this observation, we shall need basic modular arithmetic (or
at least division with remainder); we will see this proof very soon.

Further questions can be asked. One of them is: Given an integer $n$, how many
ways are there to represent $n$ as a sum of two perfect squares? This is
actually several questions masquerading as one, since it is not so clear what
a ``way'' is. Do $5 = 1^{2} + 2^{2}$ and $5 = 2^{2} + 1^{2}$ count as two
different ways? What about $5 = 1^{2} + 2^{2}$ versus $5 = \left(  -1 \right)
^{2} + 2^{2}$ (here, the perfect squares are the same, but do we really want
to count the squares or rather the numbers we are squaring?).

Let me formalize the question as follows:

\begin{question}
\label{quest.intro.sum-of-2sq.2} Let $n$ be an integer.

\textbf{(a)} How many pairs $\left(  a, b \right)  \in\mathbb{N}^{2}$ are
there that satisfy $n = a^{2} + b^{2}$ ? Here, and in the following,
$\mathbb{N}$ denotes the set $\left\{  0, 1, 2, \ldots\right\}  $ of all
nonnegative integers.

\textbf{(b)} How many pairs $\left(  a, b \right)  \in\mathbb{Z}^{2}$ are
there that satisfy $n = a^{2} + b^{2}$ ? Here, and in the following,
$\mathbb{Z}$ denotes the set $\left\{  \ldots, -2, -1, 0, 1, 2, \ldots
\right\}  $ of all integers.

\textbf{(c)} How do these counts change if we count \textbf{unordered} pairs
instead (i.e., count $\left(  a, b \right)  $ and $\left(  b, a \right)  $ as
one only)?
\end{question}

Note that when I say ``pair'', I always mean ``ordered pair'' by default,
unless I explicitly say ``unordered pair''.

Again, a little bit of programming easily yields answers to all three parts of
this question for small values of $n$, and the resulting data can be plugged
into the OEIS and yields lots of information.

\begin{proof}
[First steps toward answering Question~\ref{quest.intro.sum-of-2sq.2}%
.]\textbf{(a)} I claim that the number of such pairs is even unless $n$ is
twice a perfect square (i.e., unless $n = 2m^{2}$ for some integer $m$); in
the latter case, this number is odd instead.

Why? Let me define a \textit{solution} to be a pair $\left(  a, b \right)  $
such that $n = a^{2} + b^{2}$. So I want to know whether the number of
solutions is even or odd. But we have $a^{2} + b^{2} = b^{2} + a^{2}$ for all
$a$ and $b$. Thus, if $\left(  a, b \right)  $ is a solution, then so is
$\left(  b, a \right)  $. Hence, the solutions themselves ``come in pairs'',
with each solution $\left(  a, b \right)  $ being matched to the solution
$\left(  b, a \right)  $, unless there is a solution $\left(  a, b \right)  $
with $a = b$ (because such a solution would be matched to itself, and thus not
form an actual pair). But solutions $\left(  a, b \right)  $ with $a = b$ are
easy to classify: If $n$ is twice a perfect square, then there is exactly one
such solution (namely, $\left(  \sqrt{n/2}, \sqrt{n/2} \right)  $); otherwise
there is none (because $n = a^{2} + b^{2}$ with $a = b$ leads to $n = b^{2} +
b^{2} = 2b^{2}$). Since we know that all the other solutions ``come in
pairs'', we thus conclude that the number of solutions is odd if $n$ is twice
a perfect square and even otherwise. This proves our claim.

Of course, we have not made much headway into
Question~\ref{quest.intro.sum-of-2sq.2}; knowing whether a number is even or
odd is far from knowing the number itself. But I think the argument above was
worth showing; similar reasoning is used a lot in algebra.

\textbf{(b)} By reasoning analogous to the one we used in part \textbf{(a)},
we can see that the number of such pairs will be divisible by $8$ whenever $n$
is neither a perfect square nor twice a perfect square. Indeed, this relies on
the fact that
\begin{align*}
a^{2} + b^{2}  &  = b^{2} + a^{2} = \left(  -a \right)  ^{2} + b^{2} = b^{2} +
\left(  -a \right)  ^{2} = a^{2} + \left(  -b \right)  ^{2} = \left(  -b
\right)  ^{2} + a^{2}\\
&  = \left(  -a \right)  ^{2} + \left(  -b \right)  ^{2} = \left(  -b \right)
^{2} + \left(  -a \right)  ^{2}%
\end{align*}
for all $a$ and $b$. Thus the pairs $\left(  a, b \right)  \in\mathbb{Z}^{2}$
that satisfy $n = a^{2} + b^{2}$ don't just come in pairs; they come in sets
of $8$ (namely, each $\left(  a, b \right)  $ comes in a set with $\left(  b,
a \right)  $, $\left(  -a, b \right)  $, $\left(  b, -a \right)  $, $\left(
a, -b \right)  $, $\left(  -b, a \right)  $, $\left(  -a, -b \right)  $ and
$\left(  -b, -a \right)  $). These sets of $8$ can ``degenerate'' to smaller
sets when some of their elements coincide, but this can only happen when $n$
is a perfect square (in which case we can have $\left(  a, b \right)  =
\left(  -a, b \right)  $ for example) or twice a perfect square (in which case
we can have $\left(  a, b \right)  = \left(  b, a \right)  $ or $\left(  a, b
\right)  = \left(  -b, -a \right)  $ or other such coincidences). (Check this!)

\textbf{(c)} We can reduce this to parts \textbf{(a)} and \textbf{(b)}.
Indeed:\footnote{In the rest of this argument, \textquotedblleft
pair\textquotedblright\ will always mean \textquotedblleft pair $\left(
a,b\right)  $ satisfying $n=a^{2}+b^{2}$\textquotedblright.}

\begin{itemize}
\item When $n$ is not twice a perfect square, the number of unordered pairs
will be half the number of ordered pairs, since each unordered pair $\left(
u,v\right)  _{\text{unordered}}$ corresponds to precisely two ordered pairs
$\left(  u,v\right)  $ and $\left(  v,u\right)  $.

\item When $n$ is twice a perfect square, we have%
\begin{align*}
&  \left(  \text{the number of unordered pairs}\right) \\
&  =\dfrac{\left(  \text{the number of ordered pairs}\right)  +\left(
\text{the number of pairs with }a=b\right)  }{2}.
\end{align*}
Indeed, each unordered pair $\left(  u,v\right)  _{\text{unordered}}$
corresponds to precisely two ordered pairs $\left(  u,v\right)  $ and $\left(
v,u\right)  $ unless $u=v$, in which case it corresponds to only one ordered
pair. Thus, if we multiply the number of unordered pairs by $2$, then we
\textbf{overcount} the number of ordered pairs, because we are counting the
pairs $\left(  u,v\right)  $ with $u=v$ (that is, the pairs with $a=b$) twice.
So we get $\left(  \text{the number of ordered pairs}\right)  +\left(
\text{the number of pairs with }a=b\right)  $. This proves our above formula.

What is the number of pairs with $a=b$ ? If $n=0$, then it is $1$ (and the
only such pair is $\left(  0,0\right)  $). Otherwise, it is $1$ if we are
counting pairs in $\mathbb{N}^{2}$ (and the only such pair is $\left(
\sqrt{n/2},\sqrt{n/2}\right)  $), and is $2$ if we are counting pairs in
$\mathbb{Z}^{2}$ (and the only two such pairs are $\left(  \sqrt{n/2}%
,\sqrt{n/2}\right)  $ and $\left(  -\sqrt{n/2},-\sqrt{n/2}\right)  $).
\qedhere

\end{itemize}
\end{proof}

Note that sums of squares have a geometric meaning (going back to Pythagoras):
Two real numbers $a$ and $b$ satisfy $a^{2} + b^{2} = n$ (for a given integer
$n \geq0$) if and only if the point with Cartesian coordinates $\left(  a, b
\right)  $ lies on the circle with center $0$ and radius $\sqrt{n}$. This will
actually prove a valuable insight that will lead us to the answers to the
above questions.

Just as a teaser: There are formulas for all three parts of
Question~\ref{quest.intro.sum-of-2sq.2}, in terms of divisors of $n$ of the
forms $4k+1$ and $4k+3$. We will see these formulas after we have properly
understood the concept of Gaussian integers.

\subsection{\label{subsect.intro.algnum}Motivation: Algebraic numbers}

\begin{noncompile}
Recall how the number system was constructed. In a way, each extension was
done in order to allow a certain operation to proceed: The natural numbers
were extended to the integers in order to allow subtraction (in all cases, not
just when we are subtracting a smaller number from a larger). Then, the
integers were extended to the rational numbers in order to allow division (in
all reasonable cases\footnote{``Reasonable'' in this case means that division
by $0$ is still forbidden. If we allowed division by $0$ as well, then our
``rational numbers'' would all be equal to each other and therefore a huge
step back from the integers.}, not just when the division works out
remainder-less). Then, the rational numbers were extended to the real numbers
in order to allow limits (in all reasonable cases). Finally, the real numbers
were (or will be -- we will see this in more detail) extended to the complex
numbers in order to allow square roots.

From an algebraic point of view, the step from the rational numbers to the
real numbers is somewhat of an overkill. Algebraists often want to work with
roots, particularly roots of polynomials; ideally, every polynomial of degree
$n$ should have ``all'' $n$ roots (counted with multiplicity), so it can be
factored into linear factors. This does indeed happen once you get to complex
numbers (the so-called ``Fundamental Theorem of Algebra''), but the road there
is bumpy and non-algebraic (at the very least, you need continuity to prove
the ``Fundamental Theorem of Algebra''). So algebraists have wondered whether
there is a cheaper way to buy roots for their polynomials -- without having to
pay the price of analysis. (The question became even more relevant when they
started working over arbitrary fields and even commutative rings -- in a
sense, ``alternative number systems'' in which analysis won't help you.)

The answer is ``yes'', and we will eventually see how. But for now, let me
focus on a simple problem that is already interesting if one works inside the
real numbers.
\end{noncompile}

A real number $z$ is said to be \textit{algebraic} if there exists a nonzero
polynomial $P$ with rational coefficients such that $P\left(  z \right)  = 0$.
In other words, a real number $z$ is algebraic if and only if it is a root of
a nonzero polynomial with rational coefficients.

(If you know the complex numbers, you can replace ``real'' by ``complex'' in
this definition; but we shall only see real numbers in this little
motivational subsection.)

Examples:

\begin{itemize}
\item Each rational number $a$ is algebraic (being a root of the nonzero
polynomial $x-a$ with rational coefficients).

\item The number $\sqrt{2}$ is algebraic (being a root of the nonzero
polynomial $x^{2}-2$).

\item The number $\sqrt[3]{5}$ is algebraic (being a root of $x^{3}-5$).

\item All the roots of the polynomial $f\left(  x \right)  := \dfrac{3}%
{2}x^{4}+17x^{3}-12x+\dfrac{9}{4}$ (whatever they are) are algebraic. \newline
Speaking of these roots, what are they? Using a computer, one can show that
this polynomial $f\left(  x \right)  $ has $4$ real roots ($-11.269\ldots,
-0.960\ldots, 0.198\ldots, 0.697\ldots$), which can be written as complicated
expressions with radicals (i.e., $\sqrt[k]{}$ signs), though complex numbers
appear in these expressions (despite the roots being real!). All this does not
matter to the fact that they are algebraic :)

\item All the roots of the polynomial $g\left(  x \right)  := x^{7} - x^{5} +
1$ are algebraic. \newline This polynomial has only one real root. This root
cannot be written as an expression with radicals (as can be proven using
\href{https://en.wikipedia.org/wiki/Galois_theory}{Galois theory} -- indeed,
the discovery of this theory greatly motivated the development of abstract
algebra).
%(Nor can the remaining $6$ complex roots be.)
Nevertheless, it is algebraic, by definition. (The same holds for the
remaining $6$ complex roots of $g$ -- we are working with real numbers here
only for the sake of familiarity.)

\item The most famous number that is not algebraic is $\pi$. This is a famous
result of Lindemann, but it belongs to analysis, not to algebra, because $\pi$
is not defined algebraically in the first place (it is defined as the length
of a curve or as an area of a curved region -- but either of these definitions
boils down to a limit of a sequence).

\item The second most famous number that is not algebraic is
\href{https://en.wikipedia.org/wiki/E_(mathematical_constant)}{Euler's number
$e$} (the basis of the natural logarithm). Again, analysis is needed to define
$e$, and thus also to prove its non-algebraicity.
\end{itemize}

Numbers that are not algebraic are called
\href{https://en.wikipedia.org/wiki/Transcendental_number}{\textit{transcendental}%
}. We shall not study them much, since most of them do not come from algebra.
Instead, we shall try our hands at the following question:

\begin{question}
\label{quest.intro.algnum.1} \textbf{(a)} Is the sum of two (or, more
generally, finitely many) algebraic numbers always algebraic?

\textbf{(b)} What if we replace ``sum'' by ``difference'' or ``product''?
\end{question}

Let me motivate why this is a natural question to ask. The sum of two integers
is still an integer; the sum of two rational numbers is still a rational
number. These facts are fundamental; without them we could hardly work with
integers and rational numbers. If a similar fact would not hold for algebraic
numbers, it would mean that the algebraic numbers are not a good ``number
system'' to work in; on a practical level, it would mean that (e.g.) if we
defined a function on the set of all algebraic numbers, then we could not plug
a sum of algebraic numbers into it.

\begin{proof}
[Attempts at answering Question~\ref{quest.intro.algnum.1} \textbf{(a)}.]Let
us try a particularly simple example of a sum of two algebraic numbers: Let
$w$ be $\sqrt{2} + \sqrt{3}$. Is $w$ algebraic?

To answer this question affirmatively, we need to find a nonzero polynomial
$f\left(  x \right)  $ with rational coefficients that has $w$ as a root.

Just looking at the equality $w = \sqrt{2} + \sqrt{3}$, we cannot directly
eyeball such an $f$. The problem, in a sense, is that there are too many
(namely, two) square roots in this equality.

However, if we square this equality, then we obtain
\[
w^{2}=\left(  \sqrt{2}+\sqrt{3}\right)  ^{2}=2+2\sqrt{2}\cdot\sqrt{3}+3
=5+2\sqrt{6},
\]
which is an equality with only one square root (a sign of progress).
Subtracting $5$ from this equality (in order to ``isolate'' this remaining
square root), we obtain $w^{2}-5=2\sqrt{6}$. If we now square this equality,
then we obtain $\left(  w^{2}-5\right)  ^{2}=\left(  2\sqrt{6}\right)
^{2}=24$. At this point all square roots are gone, and we are left with an
equality that contains rational numbers and $w$ only! We can further rewrite
it as $\left(  w^{2} - 5 \right)  ^{2} - 24 = 0$. Thus, $w$ is a root of the
polynomial $f\left(  x \right)  := \left(  x^{2}-5\right)  ^{2}-24 =
x^{4}-10x^{2}+1$. This means that $w$ is algebraic (since $f$ is nonzero).

Let us try a more complicated example: Let $z$ be the number $\sqrt
{2}+\sqrt[3]{2}$. Is $z$ algebraic? The squaring trick no longer works, since
squaring $\sqrt{2} + \sqrt[3]{2}$ does not reduce the number of radicals (=
root signs). Let's instead try rewriting $z = \sqrt{2} + \sqrt[3]{2}$ as
$z-\sqrt{2}=\sqrt[3]{2}$. Cubing this equality, we obtain $\left(  z-\sqrt
{2}\right)  ^{3} =2$. In view of
\begin{align*}
\left(  z-\sqrt{2}\right)  ^{3} = z^{3}-3z^{2}\sqrt{2}+3z\left(  \sqrt
{2}\right)  ^{2}-\left(  \sqrt{2}\right)  ^{3}%
\end{align*}
(this is a particular case of the identity $\left(  a-b\right)  ^{3}%
=a^{3}-3a^{2}b+3ab^{2}-b^{3}$, which is one form of the Binomial Theorem for
exponent $3$), this becomes
\[
z^{3}-3z^{2}\sqrt{2}+3z\left(  \sqrt{2}\right)  ^{2}-\left(  \sqrt{2}\right)
^{3} = 2.
\]
This simplifies to%
\begin{align*}
z^{3}-3\sqrt{2}z^{2}+6z-2\sqrt{2}  &  =2.
\end{align*}
Let us transform this inequality in such a way that all terms with a $\sqrt
{2}$ in them end up on the right hand side while all the remaining terms end
up on the left. We thus obtain
\[
z^{3}+6z-2 \left(  3z^{2}+2\right)  \sqrt{2} .
\]
Now, squaring this equality yields
\[
\left(  z^{3}+6z-2\right)  ^{2}=\left(  3z^{2}+2\right)  ^{2}2.
\]
Hence, $z$ is a root of the polynomial
\[
g\left(  x \right)  := \left(  x^{3}+6x-2\right)  ^{2}-2\left(  3x^{2}%
+2\right)  ^{2}=x^{6}-6x^{4}-4x^{3}+12x^{2}-24x-4.
\]
This is a nonzero polynomial with rational coefficients; hence, $z$ is algebraic.

We thus have verified that the sum of two algebraic numbers is algebraic in
two cases. What about more complicated cases, such as
\[
\sqrt{2}+\sqrt{3}+\sqrt[7]{11}\text{ ?}%
\]
This is a sum of two algebraic numbers (since we already know that $\sqrt
{2}+\sqrt{3}=w$ is algebraic). Is it algebraic? Neither of our above two
methods properly works here; do we have to come up with new ad-hoc tricks?
\end{proof}

\begin{center}
\textbf{2019-01-25 lecture}
\end{center}

\subsection{Motivation: Shamir's Secret Sharing Scheme}

\subsubsection{The problem}

Adi Shamir is one of the first modern cryptographers.

Shamir's Secret Sharing Scheme is a way in which a secret (a piece of data)
can be distributed among $n$ people in such a way that

\begin{itemize}
\item \textit{Requirement 1:} any $k$ of them (collaboratively) can
reconstruct it uniquely, but

\item \textit{Requirement 2:} any $k-1$ of them (even collaboratively) cannot
gain \textbf{any} insight about it (so not only cannot they reconstruct it,
but they cannot even tell that some data are more likely to be the secret than others).
\end{itemize}

To formalize this, we encode the secret as a bitstring. A \textit{bit} is an
element of $\left\{  0,1\right\}  $. A \textit{bitstring} is a finite sequence
of bits -- such as $\left(  0,1,1,0\right)  $ or $\left(  1,0\right)  $ or
$\left(  1,1,0,1,0,0,1\right)  $ or $\left(  {}\right)  $. We shall omit the
commas and the parentheses when we write bitstrings -- so $\left(
1,1,0,1,0,0,1\right)  $ will become $1101001$.

So we have $n$ people $1,2,\ldots,n$, a positive integer $k\in\left\{
1,2,\ldots,n\right\}  $ and a secret piece of data $\mathbf{a}$ encoded as a
bitstring. Our goal is to give each of the $n$ people some bitstring in such a
way that Requirements 1 and 2 are satisfied. We denote the bitstrings given to
people $1,2,\ldots,n$ by $\mathbf{a}_{1},\mathbf{a}_{2},\ldots,\mathbf{a}_{n}$.

(Note: The length of $\mathbf{a}$ is not assumed to be a secret.)

\subsubsection{The $k=1$ case}

Let us first solve the $k=1$ case. In this case, it suffices to give each
person the full secret $\mathbf{a}$ (so $\mathbf{a}_{i}=\mathbf{a}$).

\subsubsection{The $k=n$ case}

Let us now deal with the $k=n$ case.

What happens if we give each person \textquotedblleft one $n$%
-th\textquotedblright\ of the secret? (Person $1$ gets the first $n$-th,
person $2$ the second, ...)

Requirement 1 is satisfied.

Requirement 2 is \textquotedblleft half-satisfied\textquotedblright: they
cannot reconstruct the full secret, but they know much about it.

So we need something else.

\begin{definition}
A \textit{binary operation} on a set $S$ is a map $f:S\times S\rightarrow S$.
We shall write $afb$ instead of $f\left(  a,b\right)  $.
\end{definition}

\begin{example}
Addition of integers is a binary operation on $\mathbb{Z}$ (the set of all integers).

So are multiplication and subtraction.

Division is not a binary operation -- not even on $\mathbb{Q}$ (the set of all
rational numbers).

Define a binary operation $\#$ on $\mathbb{Q}$ by $a\#b=\dfrac{a-b}%
{1+a^{2}+b^{2}}$. Well-defined since the denominator is positive.
\end{example}

\begin{definition}
Define a binary operation $\operatorname*{XOR}$ on $\left\{  0,1\right\}  $ by%
\begin{align*}
0\operatorname*{XOR}0  & =0,\\
0\operatorname*{XOR}1  & =1,\\
1\operatorname*{XOR}0  & =1,\\
1\operatorname*{XOR}1  & =0.
\end{align*}
(So $a\operatorname*{XOR}b=%
\begin{cases}
1, & \text{if }a\neq b;\\
0, & \text{if }a=b.
\end{cases}
$)
\end{definition}

XOR is short for \textquotedblleft exclusive or\textquotedblright.

\begin{definition}
For each nonnegative integer $n$, we define a binary operation
$\operatorname*{XOR}$ on $\left\{  0,1\right\}  ^{n}$ by%
\begin{align*}
& \left(  a_{1}a_{2}\cdots a_{n}\right)  \operatorname*{XOR}\left(  b_{1}%
b_{2}\cdots b_{n}\right)  \\
& =\left(  a_{1}\operatorname*{XOR}b_{1},\ a_{2}\operatorname*{XOR}%
b_{2},\ \ldots,\ a_{n}\operatorname*{XOR}b_{n}\right)  .
\end{align*}

\end{definition}

\begin{example}%
\[
\left(  11010\right)  \operatorname*{XOR}\left(  01011\right)  =10001.
\]
Note: The $0$'s in $\mathbf{a}\operatorname*{XOR}\mathbf{b}$ (where
$\mathbf{a}$ and $\mathbf{b}$ are two bitstrings) are at the positions where
$\mathbf{a}$ and $\mathbf{b}$ have equal entries, and the $1$'s are at the
positions where they have different entries.
\end{example}

Rules:

\begin{itemize}
\item $a\operatorname*{XOR}0=a$ for any bit $a$.

\item Thus, $\mathbf{a}\operatorname*{XOR}\mathbf{0}=\mathbf{a}$ for any
bitstring $\mathbf{a}$. Here, $\mathbf{0}$ means the bitstring $00\cdots0$ (of
appropriate length).

\item $a\operatorname*{XOR}a=0$ for any bit $a$.

\item Thus, $\mathbf{a}\operatorname*{XOR}\mathbf{a}=\mathbf{0}$ for any
bitstring $\mathbf{a}$.

\item $a\operatorname*{XOR}b=b\operatorname*{XOR}a$ for any bits $a,b$.

\item Thus, $\mathbf{a}\operatorname*{XOR}\mathbf{b}=\mathbf{b}%
\operatorname*{XOR}\mathbf{a}$ for any bitstrings $\mathbf{a},\mathbf{b}$.

\item $a\operatorname*{XOR}\left(  b\operatorname*{XOR}c\right)  =\left(
a\operatorname*{XOR}b\right)  \operatorname*{XOR}c$ for any bits $a,b,c$.
(Easy to check.)

\item Thus, $\mathbf{a}\operatorname*{XOR}\left(  \mathbf{b}%
\operatorname*{XOR}\mathbf{c}\right)  =\left(  \mathbf{a}\operatorname*{XOR}%
\mathbf{b}\right)  \operatorname*{XOR}\mathbf{c}$ for any bitstrings
$\mathbf{a},\mathbf{b},\mathbf{c}$. (The associativity law.)

\item If $\mathbf{a}_{1},\mathbf{a}_{2},\ldots,\mathbf{a}_{n}$ are $n$
bitstrings, then the \textquotedblleft$\operatorname*{XOR}$%
-product\textquotedblright%
\[
\mathbf{a}_{1}\operatorname*{XOR}\mathbf{a}_{2}\operatorname*{XOR}%
\mathbf{a}_{3}\operatorname*{XOR}\cdots\operatorname*{XOR}\mathbf{a}_{n}%
\]
is well-defined, in the sense that it does not depend on the parenthesization.

For example, for $n=4$, there are $5$ choices of parenthesization:%
\begin{align*}
& \left(  \mathbf{a}_{1}\operatorname*{XOR}\mathbf{a}_{2}\right)
\operatorname*{XOR}\left(  \mathbf{a}_{3}\operatorname*{XOR}\mathbf{a}%
_{4}\right)  ,\\
& \left(  \left(  \mathbf{a}_{1}\operatorname*{XOR}\mathbf{a}_{2}\right)
\operatorname*{XOR}\mathbf{a}_{3}\right)  \operatorname*{XOR}\mathbf{a}%
_{4},\\
& \mathbf{a}_{1}\operatorname*{XOR}\left(  \left(  \mathbf{a}_{2}%
\operatorname*{XOR}\mathbf{a}_{3}\right)  \operatorname*{XOR}\mathbf{a}%
_{4}\right)  ,\\
& \mathbf{a}_{1}\operatorname*{XOR}\left(  \mathbf{a}_{2}\operatorname*{XOR}%
\left(  \mathbf{a}_{3}\operatorname*{XOR}\mathbf{a}_{4}\right)  \right)  ,\\
& \left(  \mathbf{a}_{1}\operatorname*{XOR}\left(  \mathbf{a}_{2}%
\operatorname*{XOR}\mathbf{a}_{3}\right)  \right)  \operatorname*{XOR}%
\mathbf{a}_{4}.
\end{align*}
Why do they all give the same result? For example,%
\begin{align*}
& \left(  \mathbf{a}_{1}\operatorname*{XOR}\mathbf{a}_{2}\right)
\operatorname*{XOR}\left(  \mathbf{a}_{3}\operatorname*{XOR}\mathbf{a}%
_{4}\right)  \\
& =\mathbf{a}_{1}\operatorname*{XOR}\left(  \mathbf{a}_{2}\operatorname*{XOR}%
\left(  \mathbf{a}_{3}\operatorname*{XOR}\mathbf{a}_{4}\right)  \right)  \\
& =\mathbf{a}_{1}\operatorname*{XOR}\left(  \left(  \mathbf{a}_{2}%
\operatorname*{XOR}\mathbf{a}_{3}\right)  \operatorname*{XOR}\mathbf{a}%
_{4}\right)  \\
& =\left(  \mathbf{a}_{1}\operatorname*{XOR}\left(  \mathbf{a}_{2}%
\operatorname*{XOR}\mathbf{a}_{3}\right)  \right)  \operatorname*{XOR}%
\mathbf{a}_{4},
\end{align*}
where we used $\left(  \mathbf{a}\operatorname*{XOR}\mathbf{b}\right)
\operatorname*{XOR}\mathbf{c}=\mathbf{a}\operatorname*{XOR}\left(
\mathbf{b}\operatorname*{XOR}\mathbf{c}\right)  $ in each step.

For the general ($n=n$) case, see HW\#0 exercise 4.
\end{itemize}

Now, we split the secret as follows: The first $n-1$ persons get uniformly
random bitstrings $\mathbf{a}_{1},\mathbf{a}_{2},\ldots,\mathbf{a}_{n-1}$. The
last person gets%
\[
\mathbf{a}_{n}=\mathbf{a}\operatorname*{XOR}\mathbf{a}_{1}\operatorname*{XOR}%
\mathbf{a}_{2}\operatorname*{XOR}\cdots\operatorname*{XOR}\mathbf{a}_{n-1}.
\]


Then,%
\begin{align*}
& \mathbf{a}_{n}\operatorname*{XOR}\mathbf{a}_{n-1}\operatorname*{XOR}%
\mathbf{a}_{n-2}\operatorname*{XOR}\cdots\operatorname*{XOR}\mathbf{a}_{1}\\
& =\mathbf{a}\operatorname*{XOR}\mathbf{a}_{1}\operatorname*{XOR}%
\mathbf{a}_{2}\operatorname*{XOR}\cdots\operatorname*{XOR}%
\underbrace{\mathbf{a}_{n-1}\operatorname*{XOR}\mathbf{a}_{n-1}}_{=\mathbf{0}%
}\operatorname*{XOR}\mathbf{a}_{n-2}\operatorname*{XOR}\cdots
\operatorname*{XOR}\mathbf{a}_{1}\\
& =\mathbf{a}\operatorname*{XOR}\mathbf{a}_{1}\operatorname*{XOR}%
\mathbf{a}_{2}\operatorname*{XOR}\cdots\operatorname*{XOR}\mathbf{a}%
_{n-2}\underbrace{\operatorname*{XOR}\mathbf{0}}_{\text{disappears}%
}\operatorname*{XOR}\mathbf{a}_{n-2}\operatorname*{XOR}\cdots
\operatorname*{XOR}\mathbf{a}\\
& =\mathbf{a}\operatorname*{XOR}\mathbf{a}_{1}\operatorname*{XOR}%
\mathbf{a}_{2}\operatorname*{XOR}\cdots\operatorname*{XOR}%
\underbrace{\mathbf{a}_{n-2}\operatorname*{XOR}\mathbf{a}_{n-2}}_{=\mathbf{0}%
}\operatorname*{XOR}\cdots\operatorname*{XOR}\mathbf{a}_{1}\\
& =\cdots\\
& =\mathbf{a}.
\end{align*}
Hence, Requirement 1 is satisfied.

As for Requirement 2: The first $n-1$ people $1,2,\ldots,n-1$ are clueless
about $\mathbf{a}$, since they just have random bitstring. What about
$2,3,\ldots,n$ ? They know $\mathbf{a}_{2},\mathbf{a}_{3},\ldots
,\mathbf{a}_{n-1}$ and%
\[
\mathbf{a}_{n}=\mathbf{a}\operatorname*{XOR}\mathbf{a}_{1}\operatorname*{XOR}%
\mathbf{a}_{2}\operatorname*{XOR}\cdots\operatorname*{XOR}\mathbf{a}_{n-1}.
\]
They can recover $\mathbf{a}\operatorname*{XOR}\mathbf{a}_{1}$ but neither
$\mathbf{a}$ nor $\mathbf{a}_{1}$. Since $\mathbf{a}_{1}$ is uniformly random,
$\mathbf{a}$ can also be anything to them, with equal probabilities. So they
are also clueless. Similarly for any other choice of $n-1$ people. So both
requirements hold, and we are done in the $k=n$ case.

\subsubsection{The case $n=3$ and $k=2$}

Now let us work in the case $n=3$ and $k=2$. So we have $3$ people $1,2,3$,
and any $2$ should be able to recover the secret.

Suggestion: \textquotedblleft Factor\textquotedblright\ $\mathbf{a}$ as
$\mathbf{a}=\mathbf{x}_{1}\operatorname*{XOR}\mathbf{x}_{2}\operatorname*{XOR}%
\mathbf{x}_{3}$ (by picking $\mathbf{x}_{1}$ and $\mathbf{x}_{2}$ randomly and
setting $\mathbf{x}_{3}=\mathbf{a}\operatorname*{XOR}\mathbf{x}_{1}%
\operatorname*{XOR}\mathbf{x}_{2}$), and give person $1$ the bitstrings
$\mathbf{x}_{2}$ and $\mathbf{x}_{3}$ etc.? (Nathan.)

What about $k=2$ in general? \textquotedblleft Factor\textquotedblright%
\ $\mathbf{a}$ as $\mathbf{a}=\mathbf{x}_{1}\operatorname*{XOR}\mathbf{x}%
_{2}\operatorname*{XOR}\cdots\operatorname*{XOR}\mathbf{x}_{n}$. Each person
$i$ gets all $\mathbf{x}_{j}$ with $j\neq i$. This works!

Okay, I think we need $k=3$ to make this \textquotedblleft
interesting\textquotedblright.

\subsubsection{The case $n=5$ and $k=3$}

Recall: A polynomial $p\left(  x\right)  =cx^{2}+bx+a$ of degree $\leq2$ is
uniquely determined by any three values.

Choose two \textbf{uniformly random} bitstrings $\mathbf{c}$ and $\mathbf{b}$
(of the same length as $\mathbf{a}$) and encode them as numbers $c$ and $b$
(reading them as binary). Encode the secret $\mathbf{a}$ as a number $a$.

Give each person $i\in\left\{  1,2,\ldots,5\right\}  $ the value $p\left(
i\right)  $.

Requirement 1 is satisfied, since any three of the values uniquely determine
$p$. For example, $p\left(  1\right)  ,p\left(  2\right)  ,p\left(  5\right)
$ uniquely determine $p$. Thus, three people recover $p$ and thus $a$.

Requirement 2 is \textquotedblleft sort of\textquotedblright\ satisfied: Given
only two values -- say, $p\left(  1\right)  $ and $p\left(  3\right)  $ --, we
have infinitely many $p$'s that could give them, and these $p$'s have
different constant terms $a$ (this can be checked). This looks like no
information on $\mathbf{a}$.

Does this solve the problem?

(In general, the polynomial should be of degree $\leq k-1$.)

Almost. We have to be careful with our randomness.

\begin{itemize}
\item If we do as above -- choosing uniformly random $\mathbf{c}$ and
$\mathbf{b}$ -- then $c$ and $b$ will be integers between $0$ and $2^{N}-1$
for some $N\geq0$. These aren't arbitrary numbers. Now, person $2$ knows
$p\left(  2\right)  =c2^{2}+b2+a=4c+2b+a$, and thus knows whether $a$ is even
or odd. This means she knows the last bit of the secret $\mathbf{a}$. This is
not \textquotedblleft clueless\textquotedblright.

\item Okay, let us pick $c$ and $b$ to be uniformly random rational numbers instead.

Oops! There is no such thing as \textquotedblleft uniformly random rational
numbers\textquotedblright.

\item Okay, let us pick $c$ and $b$ from $\left[  0,1\right]  $ (for example).

Now, if person $3$ gets $p\left(  3\right)  =$
\end{itemize}


\end{document}