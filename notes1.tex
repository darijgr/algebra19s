\documentclass[numbers=enddot,12pt,final,onecolumn,notitlepage]{scrartcl}%
\usepackage[headsepline,footsepline,manualmark]{scrlayer-scrpage}
\usepackage[all,cmtip]{xy}
\usepackage{amssymb}
\usepackage{amsmath}
\usepackage{amsthm}
\usepackage{framed}
\usepackage{comment}
\usepackage{color}
\usepackage{hyperref}
\usepackage[sc]{mathpazo}
\usepackage[T1]{fontenc}
\usepackage{tikz}
\usepackage{needspace}
\usepackage{tabls}
\usepackage{wasysym}
%TCIDATA{OutputFilter=latex2.dll}
%TCIDATA{Version=5.50.0.2960}
%TCIDATA{LastRevised=Monday, February 18, 2019 09:56:19}
%TCIDATA{SuppressPackageManagement}
%TCIDATA{<META NAME="GraphicsSave" CONTENT="32">}
%TCIDATA{<META NAME="SaveForMode" CONTENT="1">}
%TCIDATA{BibliographyScheme=Manual}
%TCIDATA{Language=American English}
%BeginMSIPreambleData
\providecommand{\U}[1]{\protect\rule{.1in}{.1in}}
%EndMSIPreambleData
\usetikzlibrary{arrows}
\newcounter{exer}
\newcounter{exera}
\numberwithin{exer}{subsection}
\theoremstyle{definition}
\newtheorem{theo}{Theorem}[subsection]
\newenvironment{theorem}[1][]
{\begin{theo}[#1]\begin{leftbar}}
{\end{leftbar}\end{theo}}
\newtheorem{lem}[theo]{Lemma}
\newenvironment{lemma}[1][]
{\begin{lem}[#1]\begin{leftbar}}
{\end{leftbar}\end{lem}}
\newtheorem{prop}[theo]{Proposition}
\newenvironment{proposition}[1][]
{\begin{prop}[#1]\begin{leftbar}}
{\end{leftbar}\end{prop}}
\newtheorem{defi}[theo]{Definition}
\newenvironment{definition}[1][]
{\begin{defi}[#1]\begin{leftbar}}
{\end{leftbar}\end{defi}}
\newtheorem{remk}[theo]{Remark}
\newenvironment{remark}[1][]
{\begin{remk}[#1]\begin{leftbar}}
{\end{leftbar}\end{remk}}
\newtheorem{coro}[theo]{Corollary}
\newenvironment{corollary}[1][]
{\begin{coro}[#1]\begin{leftbar}}
{\end{leftbar}\end{coro}}
\newtheorem{conv}[theo]{Convention}
\newenvironment{convention}[1][]
{\begin{conv}[#1]\begin{leftbar}}
{\end{leftbar}\end{conv}}
\newtheorem{quest}[theo]{Question}
\newenvironment{question}[1][]
{\begin{quest}[#1]\begin{leftbar}}
{\end{leftbar}\end{quest}}
\newtheorem{warn}[theo]{Warning}
\newenvironment{conclusion}[1][]
{\begin{warn}[#1]\begin{leftbar}}
{\end{leftbar}\end{warn}}
\newtheorem{conj}[theo]{Conjecture}
\newenvironment{conjecture}[1][]
{\begin{conj}[#1]\begin{leftbar}}
{\end{leftbar}\end{conj}}
\newtheorem{exam}[theo]{Example}
\newenvironment{example}[1][]
{\begin{exam}[#1]\begin{leftbar}}
{\end{leftbar}\end{exam}}
\newtheorem{exmp}[exer]{Exercise}
\newenvironment{exercise}[1][]
{\begin{exmp}[#1]\begin{leftbar}}
{\end{leftbar}\end{exmp}}
\newenvironment{statement}{\begin{quote}}{\end{quote}}
\newenvironment{fineprint}{\begin{small}}{\end{small}}
\iffalse
\newenvironment{proof}[1][Proof]{\noindent\textbf{#1.} }{\ \rule{0.5em}{0.5em}}
\newenvironment{question}[1][Question]{\noindent\textbf{#1.} }{\ \rule{0.5em}{0.5em}}
\fi
\let\sumnonlimits\sum
\let\prodnonlimits\prod
\let\cupnonlimits\bigcup
\let\capnonlimits\bigcap
\renewcommand{\sum}{\sumnonlimits\limits}
\renewcommand{\prod}{\prodnonlimits\limits}
\renewcommand{\bigcup}{\cupnonlimits\limits}
\renewcommand{\bigcap}{\capnonlimits\limits}
\setlength\tablinesep{3pt}
\setlength\arraylinesep{3pt}
\setlength\extrarulesep{3pt}
\voffset=0cm
\hoffset=-0.7cm
\setlength\textheight{22.5cm}
\setlength\textwidth{15.5cm}
\newcommand\arxiv[1]{\href{http://www.arxiv.org/abs/#1}{\texttt{arXiv:#1}}}
\newenvironment{verlong}{}{}
\newenvironment{vershort}{}{}
\newenvironment{noncompile}{}{}
\excludecomment{verlong}
\includecomment{vershort}
\excludecomment{noncompile}
\newcommand{\CC}{\mathbb{C}}
\newcommand{\RR}{\mathbb{R}}
\newcommand{\QQ}{\mathbb{Q}}
\newcommand{\NN}{\mathbb{N}}
\newcommand{\ZZ}{\mathbb{Z}}
\newcommand{\id}{\operatorname{id}}
\newcommand{\lcm}{\operatorname{lcm}}
\newcommand{\rev}{\operatorname{rev}}
\newcommand{\powset}[2][]{\ifthenelse{\equal{#2}{}}{\mathcal{P}\left(#1\right)}{\mathcal{P}_{#1}\left(#2\right)}}
\newcommand{\set}[1]{\left\{ #1 \right\}}
\newcommand{\abs}[1]{\left| #1 \right|}
\newcommand{\tup}[1]{\left( #1 \right)}
\newcommand{\ive}[1]{\left[ #1 \right]}
\newcommand{\floor}[1]{\left\lfloor #1 \right\rfloor}
\newcommand{\lf}[2]{#1^{\underline{#2}}}
\newcommand{\underbrack}[2]{\underbrace{#1}_{\substack{#2}}}
\newcommand{\horrule}[1]{\rule{\linewidth}{#1}}
\newcommand{\nnn}{\nonumber\\}
\newcommand{\sslash}{\mathbin{/\mkern-6mu/}}
\ihead{Math 4281 notes}
\ohead{page \thepage}
\cfoot{}
\begin{document}

\title{UMN Spring 2019 Math 4281 notes}
\author{Darij Grinberg}
\date{
%TCIMACRO{\TeXButton{today}{\today} }%
%BeginExpansion
\today
%EndExpansion
}
\maketitle
\tableofcontents

\section{Introduction}

This file will contain the notes from the Math 4281 class (``Introduction to
Modern Algebra'') I am teaching at UMN in Spring 2019. I will type the first
draft directly in the classroom, and subsequently expand it into proper
writing. Occasionally, I will also add extra sections not covered in class.

The website of the class is
\url{http://www-users.math.umn.edu/~dgrinber/19s/index.html} ; you will find
homework sets there.

\subsection{Organisation}

See \href{http://www-users.math.umn.edu/~dgrinber/19s/syll.pdf}{the syllabus}
for the organization of this class and for the homework.

\subsection{Literature}

Many books have been written about abstract algebra. I have only a passing
familiarity with most of them. Some of the ``bibles'' of the subject (bulky
texts covering lots of material) are Dummit/Foote \cite{Dummit-Foote}, Knapp
\cite{Knapp1} and \cite{Knapp2} (both freely available), van der Waerden
\cite{Waerden1} and \cite{Waerden2} (one of the oldest texts on modern
algebra, thus rather dated, but still as readable as ever).
%Two other textbooks are Bosch \cite{Bosch} and Artin \cite{Artin}.


Of course, any book longer than 200 pages likely goes further than our course
will (unless it is full of details or solved exercises or printed in really
large letters). Thus, let me recommend some more introductory sources.
Siksek's lecture notes \cite{Siksek} are a readable introduction that is a lot
more amusing than I had ever expected an algebra text to be. Goodman's free
book \cite{Goodman} combines introductory material with geometric motivation
and applications, such as the classification of regular polyhedra and
2-dimensional crystals. In a sense, it is a great complement to our
ungeometric course. Pinter's \cite{Pinter} often gets used in classes like
ours. Armstrong's notes \cite{Armstrong} cover a significant part of what we
do (and he will likely have notes for a second course written up by the end of
this semester).

Keith Conrad's blurbs \cite{Conrad*} are not a book, as they only cover
selected topics. But at pretty much every topic they cover, they are one of
the best sources (clear, full of examples, and often going fairly deep). We
shall follow one of them particularly closely: the one on Gaussian integers
\cite{Conrad-Gauss}.

We will use some basic linear algebra, all of which can be found in Hefferon's
book \cite{Hefferon} (but we won't need all of this book). As far as
determinants are concerned, we will briefly build up their theory; we refer to
\cite[Section 12 \& Appendix B]{Strickland} for proofs (and to \cite[Chapter
6]{detnotes} for a really detailed and formal treatment).

This course will begin (after some motivating questions) with a survey of
elementary number theory. This is in itself a deep subject (despite the name)
with a long history (\href{https://en.wikipedia.org/wiki/Plimpton_322}{perhaps
as old as mathematics}), and of course we will just scratch the surface. Books
like \cite{NiZuMo91}, \cite{Burton} and \cite{Uspensky-Heaslet} cover a lot
more than we can do. The Gallier/Quaintance survey \cite{Gallier-RSA} covers a
good amount of basics and more.

We assume that the reader is familiar with the commonplaces of mathematical
argumentation, such as induction (including strong induction),
\textquotedblleft WLOG\textquotedblright\ arguments, proof by contradiction,
summation signs ($\sum$) and polynomials (a vague notion of polynomials will
suffice; we will give a precise definition when it becomes necessary). If not,
several texts can be helpful in achieving such familiarity: e.g.,
\cite[particularly Chapters 1--5]{LeLeMe}, \cite{Hammack}, \cite{Day}.

\begin{center}
\textbf{2019-01-23 lecture}
\end{center}

\subsection{The plan}

The material I am going to cover is mostly standard. However, the order in
which I will go through it is somewhat unusual: I will spend a lot of time
studying the basic examples before defining abstract notions such as
``group'', ``monoid'', ``ring'' and ``field''. This way, once I come to these
notions, you'll already have many examples to work with. (Don't be fooled by
the word ``example'': We will prove a lot about them, much of which is neither
straightforward nor easy.)

First, I will show some motivating questions that are easy to state yet
require abstract algebra to prove. We will hopefully see their answers by the
end of this class. (Some of them can also be answered elementarily, without
using abstract algebra, but such answers usually take more work and are harder
to find.)

\subsection{\label{subsect.intro.sum-of-2sq}Motivation: $n=x^{2}+y^{2}$}

A \textit{perfect square} means the square of an integer. Thus, the perfect
squares are
\[
0^{2} = 0, \qquad1^{2} = 1, \qquad2^{2} = 4, \qquad3^{2} = 9, \qquad4^{2} =
16, \qquad\ldots.
\]


Here is an old problem (first solved by Pierre de Fermat in 1640, but
apparently already studied by Diophantus in the 3rd Century):

\begin{question}
\label{quest.intro.sum-of-2sq.1} What integers can be written as sums of two
perfect squares?
\end{question}

For example, $5$ can be written in this way, since $5=2^{2}+1^{2}$.

So can $4$, since $4=2^{2}+0^{2}$. (Keep in mind that $0$ is a perfect square.)

However, $7$ cannot be written in this way. In fact, if we had $7 = a^{2} +
b^{2}$ for two integers $a$ and $b$, then $a^{2}$ and $b^{2}$ would have to be
$\leq7$ (since $a^{2}$ and $b^{2}$ are always $\geq0$, no matter what sign $a$
and $b$ have); but the only perfect squares that are $\leq7$ are $0,1,4$, and
there is no way to write $7$ as a sum of two of these perfect squares (just
check all the possibilities).

For a similar but simpler reason, no negative number can be written as a sum
of two perfect squares.

We can of course approach Question~\ref{quest.intro.sum-of-2sq.1} using a
computer: It is very easy to check, for a given integer $n$, whether $n$ is a
sum of two perfect squares. (Just check all possibilities for $a$ and $b$ for
the validity of the equation $n = a^{2} + b^{2}$. You only need to try $a$ and
$b$ belonging to $\left\{  0, 1, \ldots, \left\lfloor \sqrt{n} \right\rfloor
\right\}  $, where $\left\lfloor y \right\rfloor $ (for a real number $y$)
denotes the smallest integer that is less or equal than $y$ (also known as
``$y$ rounded down'').) If you do this, you will see that among the first
$101$ nonnegative integers, the ones that can be written as sums of two
perfect squares are precisely
\begin{align*}
&  0, 1, 2, 4, 5, 8, 9, 10, 13, 16, 17, 18, 20, 25, 26, 29,\\
&  32, 34, 36, 37, 40, 41, 45, 49, 50, 52, 53, 58, 61, 64,\\
&  65, 68, 72, 73, 74, 80, 81, 82, 85, 89, 90, 97, 98, 100 .
\end{align*}
Having this data, you can look up the sequence in \href{https://oeis.org/}{the
Online Encyclopedia of Integer Sequences (short OEIS)}, and see that the
sequence of these integers is known as \href{https://oeis.org/A001481}{OEIS
Sequence A001481}. In the ``Comments'' field, you can read a lot of what is
known about it (albeit in telegraphic style).

For example, one of the comments says ``Closed under multiplication''. This is
short for ``if you multiply two entries of the sequence, then the product will
again be an entry of the sequence''. In other words, if you multiply two
integers that are sums of two perfect squares, then you get another sum of two
perfect squares. Why is this so?

It turns out that there is a \textquotedblleft simple\textquotedblright%
\ reason for this: the identity
\begin{equation}
\left(  a^{2}+b^{2}\right)  \left(  c^{2}+d^{2}\right)  =\left(  ad+bc\right)
^{2}+\left(  ac-bd\right)  ^{2}, \label{eq.intro.sum-of-2sq.sum*sum}%
\end{equation}
which holds for arbitrary reals $a,b,c,d$ (and thus, in particular, for
integers). This is known as
\href{https://en.wikipedia.org/wiki/Brahmagupta-Fibonacci_identity}{the
Brahmagupta-Fibonacci identity}, and of course can easily be proven by
expanding both sides. But how would you come up with such an identity?

If you stare at the above sequence long enough, you may also discover another
pattern: An integer of the form $4k+3$ with integer $k$ (that is, an integer
that is larger by $3$ than a multiple of $4$) can never be written as a sum of
two perfect squares. (Thus, $3,7,11,15,19,23,\ldots$ cannot be written in this
way.) This does not account for all integers that cannot be written in this
way, but it does provide some clues to the answer that we will later see. In
order to prove this observation, we shall need basic modular arithmetic (or at
least division with remainder); we will see this proof very soon (see Exercise
\ref{exe.ent.even-odd-sumsq} \textbf{(c)}).

Further questions can be asked. One of them is: Given an integer $n$, how many
ways are there to represent $n$ as a sum of two perfect squares? This is
actually several questions masquerading as one, since it is not so clear what
a ``way'' is. Do $5 = 1^{2} + 2^{2}$ and $5 = 2^{2} + 1^{2}$ count as two
different ways? What about $5 = 1^{2} + 2^{2}$ versus $5 = \left(  -1 \right)
^{2} + 2^{2}$ (here, the perfect squares are the same, but do we really want
to count the squares or rather the numbers we are squaring?).

Let me formalize the question as follows:

\begin{question}
\label{quest.intro.sum-of-2sq.2} Let $n$ be an integer.

\textbf{(a)} How many pairs $\left(  a, b \right)  \in\mathbb{N}^{2}$ are
there that satisfy $n = a^{2} + b^{2}$ ? Here, and in the following,
$\mathbb{N}$ denotes the set $\left\{  0, 1, 2, \ldots\right\}  $ of all
nonnegative integers.

\textbf{(b)} How many pairs $\left(  a, b \right)  \in\mathbb{Z}^{2}$ are
there that satisfy $n = a^{2} + b^{2}$ ? Here, and in the following,
$\mathbb{Z}$ denotes the set $\left\{  \ldots, -2, -1, 0, 1, 2, \ldots
\right\}  $ of all integers.

\textbf{(c)} How do these counts change if we count \textbf{unordered} pairs
instead (i.e., count $\left(  a, b \right)  $ and $\left(  b, a \right)  $ as
one only)?
\end{question}

Note that when I say ``pair'', I always mean ``ordered pair'' by default,
unless I explicitly say ``unordered pair''.

Again, a little bit of programming easily yields answers to all three parts of
this question for small values of $n$, and the resulting data can be plugged
into the OEIS and yields lots of information.

\begin{proof}
[First steps toward answering Question~\ref{quest.intro.sum-of-2sq.2}%
.]\textbf{(a)} I claim that the number of such pairs is even unless $n$ is
twice a perfect square (i.e., unless $n = 2m^{2}$ for some integer $m$); in
the latter case, this number is odd instead.

Why? Let me define a \textit{solution} to be a pair $\left(  a,b\right)  $
such that $n=a^{2}+b^{2}$. So I want to know whether the number of solutions
is even or odd. But we have $a^{2}+b^{2}=b^{2}+a^{2}$ for all $a$ and $b$.
Thus, if $\left(  a,b\right)  $ is a solution, then so is $\left(  b,a\right)
$. Hence, the solutions themselves \textquotedblleft come in
pairs\textquotedblright, with each solution $\left(  a,b\right)  $ being
matched to the solution $\left(  b,a\right)  $, unless there is a solution
$\left(  a,b\right)  $ with $a=b$ (because such a solution would be matched to
itself, and thus not form an actual pair). But solutions $\left(  a,b\right)
$ with $a=b$ are easy to classify: If $n$ is twice a perfect square, then
there is exactly one such solution (namely, $\left(  \sqrt{n/2},\sqrt
{n/2}\right)  $); otherwise there is none (because $n=a^{2}+b^{2}$ with $a=b$
leads to $n=b^{2}+b^{2}=2b^{2}$, which can only happen when $n$ is twice a
perfect square). Since we know that all the other solutions \textquotedblleft
come in pairs\textquotedblright, we thus conclude that the number of solutions
is odd if $n$ is twice a perfect square and even otherwise. This proves our claim.

Of course, we have not made much headway into
Question~\ref{quest.intro.sum-of-2sq.2}; knowing whether a number is even or
odd is far from knowing the number itself. But I think the argument above was
worth showing; similar reasoning is used a lot in algebra.

\textbf{(b)} By reasoning analogous to the one we used in part \textbf{(a)},
we can see that the number of such pairs will be divisible by $8$ whenever $n$
is neither a perfect square nor twice a perfect square. Indeed, this relies on
the fact that
\begin{align*}
a^{2} + b^{2}  &  = b^{2} + a^{2} = \left(  -a \right)  ^{2} + b^{2} = b^{2} +
\left(  -a \right)  ^{2} = a^{2} + \left(  -b \right)  ^{2} = \left(  -b
\right)  ^{2} + a^{2}\\
&  = \left(  -a \right)  ^{2} + \left(  -b \right)  ^{2} = \left(  -b \right)
^{2} + \left(  -a \right)  ^{2}%
\end{align*}
for all $a$ and $b$. Thus the pairs $\left(  a, b \right)  \in\mathbb{Z}^{2}$
that satisfy $n = a^{2} + b^{2}$ don't just come in pairs; they come in sets
of $8$ (namely, each $\left(  a, b \right)  $ comes in a set with $\left(  b,
a \right)  $, $\left(  -a, b \right)  $, $\left(  b, -a \right)  $, $\left(
a, -b \right)  $, $\left(  -b, a \right)  $, $\left(  -a, -b \right)  $ and
$\left(  -b, -a \right)  $). These sets of $8$ can ``degenerate'' to smaller
sets when some of their elements coincide, but this can only happen when $n$
is a perfect square (in which case we can have $\left(  a, b \right)  =
\left(  -a, b \right)  $ for example) or twice a perfect square (in which case
we can have $\left(  a, b \right)  = \left(  b, a \right)  $ or $\left(  a, b
\right)  = \left(  -b, -a \right)  $ or other such coincidences). (Check this!)

\textbf{(c)} We can reduce this to parts \textbf{(a)} and \textbf{(b)}.
Indeed:\footnote{In the rest of this argument, \textquotedblleft
pair\textquotedblright\ will always mean \textquotedblleft pair $\left(
a,b\right)  $ satisfying $n=a^{2}+b^{2}$\textquotedblright.}

\begin{itemize}
\item When $n$ is not twice a perfect square, the number of unordered pairs
will be half the number of ordered pairs, since each unordered pair $\left(
u,v\right)  _{\text{unordered}}$ corresponds to precisely two ordered pairs
$\left(  u,v\right)  $ and $\left(  v,u\right)  $.

\item When $n$ is twice a perfect square, we have%
\begin{align*}
&  \left(  \text{the number of unordered pairs}\right) \\
&  =\dfrac{\left(  \text{the number of ordered pairs}\right)  +\left(
\text{the number of pairs with }a=b\right)  }{2}.
\end{align*}
Indeed, each unordered pair $\left(  u,v\right)  _{\text{unordered}}$
corresponds to precisely two ordered pairs $\left(  u,v\right)  $ and $\left(
v,u\right)  $ unless $u=v$, in which case it corresponds to only one ordered
pair. Thus, if we multiply the number of unordered pairs by $2$, then we
\textbf{overcount} the number of ordered pairs, because we are counting the
pairs $\left(  u,v\right)  $ with $u=v$ (that is, the pairs with $a=b$) twice.
So we get $\left(  \text{the number of ordered pairs}\right)  +\left(
\text{the number of pairs with }a=b\right)  $. This proves our above formula.

What is the number of pairs with $a=b$ ? If $n=0$, then it is $1$ (and the
only such pair is $\left(  0,0\right)  $). Otherwise, it is $1$ if we are
counting pairs in $\mathbb{N}^{2}$ (and the only such pair is $\left(
\sqrt{n/2},\sqrt{n/2}\right)  $), and is $2$ if we are counting pairs in
$\mathbb{Z}^{2}$ (and the only two such pairs are $\left(  \sqrt{n/2}%
,\sqrt{n/2}\right)  $ and $\left(  -\sqrt{n/2},-\sqrt{n/2}\right)  $).
\qedhere

\end{itemize}
\end{proof}

Note that sums of squares have a geometric meaning (going back to Pythagoras):
Two real numbers $a$ and $b$ satisfy $a^{2}+b^{2}=n$ (for a given integer
$n\geq0$) if and only if the point with Cartesian coordinates $\left(
a,b\right)  $ lies on the circle with center $0$ and radius $\sqrt{n}$. This
will actually prove a valuable insight that will lead us to the answers to the
above questions.

Just as a teaser: There are formulas for all three parts of
Question~\ref{quest.intro.sum-of-2sq.2}, in terms of divisors of $n$ of the
forms $4k+1$ and $4k+3$. We will see these formulas after we have properly
understood the concept of Gaussian integers.

\subsection{\label{subsect.intro.algnum}Motivation: Algebraic numbers}

\begin{noncompile}
Recall how the number system was constructed. In a way, each extension was
done in order to allow a certain operation to proceed: The natural numbers
were extended to the integers in order to allow subtraction (in all cases, not
just when we are subtracting a smaller number from a larger). Then, the
integers were extended to the rational numbers in order to allow division (in
all reasonable cases\footnote{``Reasonable'' in this case means that division
by $0$ is still forbidden. If we allowed division by $0$ as well, then our
``rational numbers'' would all be equal to each other and therefore a huge
step back from the integers.}, not just when the division works out
remainder-less). Then, the rational numbers were extended to the real numbers
in order to allow limits (in all reasonable cases). Finally, the real numbers
were (or will be -- we will see this in more detail) extended to the complex
numbers in order to allow square roots.

From an algebraic point of view, the step from the rational numbers to the
real numbers is somewhat of an overkill. Algebraists often want to work with
roots, particularly roots of polynomials; ideally, every polynomial of degree
$n$ should have ``all'' $n$ roots (counted with multiplicity), so it can be
factored into linear factors. This does indeed happen once you get to complex
numbers (the so-called ``Fundamental Theorem of Algebra''), but the road there
is bumpy and non-algebraic (at the very least, you need continuity to prove
the ``Fundamental Theorem of Algebra''). So algebraists have wondered whether
there is a cheaper way to buy roots for their polynomials -- without having to
pay the price of analysis. (The question became even more relevant when they
started working over arbitrary fields and even commutative rings -- in a
sense, ``alternative number systems'' in which analysis won't help you.)

The answer is ``yes'', and we will eventually see how. But for now, let me
focus on a simple problem that is already interesting if one works inside the
real numbers.
\end{noncompile}

A real number $z$ is said to be \textit{algebraic} if there exists a nonzero
polynomial $P$ with rational coefficients such that $P\left(  z \right)  = 0$.
In other words, a real number $z$ is algebraic if and only if it is a root of
a nonzero polynomial with rational coefficients.

(If you know the complex numbers, you can replace ``real'' by ``complex'' in
this definition; but we shall only see real numbers in this little
motivational subsection.)

Examples:

\begin{itemize}
\item Each rational number $a$ is algebraic (being a root of the nonzero
polynomial $x-a$ with rational coefficients).

\item The number $\sqrt{2}$ is algebraic (being a root of the nonzero
polynomial $x^{2}-2$).

\item The number $\sqrt[3]{5}$ is algebraic (being a root of $x^{3}-5$).

\item All the roots of the polynomial $f\left(  x \right)  := \dfrac{3}%
{2}x^{4}+17x^{3}-12x+\dfrac{9}{4}$ (whatever they are) are algebraic. \newline
Speaking of these roots, what are they? Using a computer, one can show that
this polynomial $f\left(  x \right)  $ has $4$ real roots ($-11.269\ldots,
-0.960\ldots, 0.198\ldots, 0.697\ldots$), which can be written as complicated
expressions with radicals (i.e., $\sqrt[k]{}$ signs), though complex numbers
appear in these expressions (despite the roots being real!). All this does not
matter to the fact that they are algebraic :)

\item All the roots of the polynomial $g\left(  x \right)  := x^{7} - x^{5} +
1$ are algebraic. \newline This polynomial has only one real root. This root
cannot be written as an expression with radicals (as can be proven using
\href{https://en.wikipedia.org/wiki/Galois_theory}{Galois theory} -- indeed,
the discovery of this theory greatly motivated the development of abstract
algebra).
%(Nor can the remaining $6$ complex roots be.)
Nevertheless, it is algebraic, by definition. (The same holds for the
remaining $6$ complex roots of $g$ -- we are working with real numbers here
only for the sake of familiarity.)

\item The most famous number that is not algebraic is $\pi$. This is a famous
result of Lindemann, but it belongs to analysis, not to algebra, because $\pi$
is not defined algebraically in the first place (it is defined as the length
of a curve or as an area of a curved region -- but either of these definitions
boils down to a limit of a sequence).

\item The second most famous number that is not algebraic is
\href{https://en.wikipedia.org/wiki/E_(mathematical_constant)}{Euler's number
$e$} (the basis of the natural logarithm). Again, analysis is needed to define
$e$, and thus also to prove its non-algebraicity.
\end{itemize}

Numbers that are not algebraic are called
\href{https://en.wikipedia.org/wiki/Transcendental_number}{\textit{transcendental}%
}. We shall not study them much, since most of them do not come from algebra.
Instead, we shall try our hands at the following question:

\begin{question}
\label{quest.intro.algnum.1} \textbf{(a)} Is the sum of two (or, more
generally, finitely many) algebraic numbers always algebraic?

\textbf{(b)} What if we replace ``sum'' by ``difference'' or ``product''?
\end{question}

Let me motivate why this is a natural question to ask. The sum of two integers
is still an integer; the sum of two rational numbers is still a rational
number. These facts are fundamental; without them we could hardly work with
integers and rational numbers. If a similar fact would not hold for algebraic
numbers, it would mean that the algebraic numbers are not a good ``number
system'' to work in; on a practical level, it would mean that (e.g.) if we
defined a function on the set of all algebraic numbers, then we could not plug
a sum of algebraic numbers into it.

\begin{proof}
[Attempts at answering Question~\ref{quest.intro.algnum.1} \textbf{(a)}.]Let
us try a particularly simple example of a sum of two algebraic numbers: Let
$w$ be $\sqrt{2} + \sqrt{3}$. Is $w$ algebraic?

To answer this question affirmatively, we need to find a nonzero polynomial
$f\left(  x \right)  $ with rational coefficients that has $w$ as a root.

Just looking at the equality $w = \sqrt{2} + \sqrt{3}$, we cannot directly
eyeball such an $f$. The problem, in a sense, is that there are too many
(namely, two) square roots in this equality.

However, if we square this equality, then we obtain
\[
w^{2}=\left(  \sqrt{2}+\sqrt{3}\right)  ^{2}=2+2\sqrt{2}\cdot\sqrt{3}+3
=5+2\sqrt{6},
\]
which is an equality with only one square root (a sign of progress).
Subtracting $5$ from this equality (in order to ``isolate'' this remaining
square root), we obtain $w^{2}-5=2\sqrt{6}$. If we now square this equality,
then we obtain $\left(  w^{2}-5\right)  ^{2}=\left(  2\sqrt{6}\right)
^{2}=24$. At this point all square roots are gone, and we are left with an
equality that contains rational numbers and $w$ only! We can further rewrite
it as $\left(  w^{2} - 5 \right)  ^{2} - 24 = 0$. Thus, $w$ is a root of the
polynomial $f\left(  x \right)  := \left(  x^{2}-5\right)  ^{2}-24 =
x^{4}-10x^{2}+1$. This means that $w$ is algebraic (since $f$ is nonzero).

Let us try a more complicated example: Let $z$ be the number $\sqrt
{2}+\sqrt[3]{2}$. Is $z$ algebraic? The squaring trick no longer works, since
squaring $\sqrt{2}+\sqrt[3]{2}$ does not reduce the number of radicals (= root
signs). Let's instead try rewriting $z=\sqrt{2}+\sqrt[3]{2}$ as $z-\sqrt
{2}=\sqrt[3]{2}$. Cubing this equality, we obtain $\left(  z-\sqrt{2}\right)
^{3}=2$. In view of
\[
\left(  z-\sqrt{2}\right)  ^{3}=z^{3}-3z^{2}\sqrt{2}+3z\left(  \sqrt
{2}\right)  ^{2}-\left(  \sqrt{2}\right)  ^{3}%
\]
(this is a particular case of the identity $\left(  a-b\right)  ^{3}%
=a^{3}-3a^{2}b+3ab^{2}-b^{3}$, which is one form of the Binomial Theorem for
exponent $3$), this rewrites a
\[
z^{3}-3z^{2}\sqrt{2}+3z\left(  \sqrt{2}\right)  ^{2}-\left(  \sqrt{2}\right)
^{3}=2.
\]
This simplifies to%
\[
z^{3}-3\sqrt{2}z^{2}+6z-2\sqrt{2}=2.
\]
Let us transform this inequality in such a way that all terms with a $\sqrt
{2}$ in them end up on the right hand side while all the remaining terms end
up on the left. We thus obtain
\[
z^{3}+6z-2=\sqrt{2}\left(  3z^{2}+2\right)  .
\]
Now, squaring this equality yields
\[
\left(  z^{3}+6z-2\right)  ^{2}=2\left(  3z^{2}+2\right)  ^{2}.
\]
Hence, $z$ is a root of the polynomial
\[
g\left(  x\right)  :=\left(  x^{3}+6x-2\right)  ^{2}-2\left(  3x^{2}+2\right)
^{2}=x^{6}-6x^{4}-4x^{3}+12x^{2}-24x-4.
\]
This is a nonzero polynomial with rational coefficients; hence, $z$ is algebraic.

We thus have verified that the sum of two algebraic numbers is algebraic in
two cases. What about more complicated cases, such as
\[
\sqrt{2}+\sqrt{3}+\sqrt[7]{11}\text{ ?}%
\]
This is a sum of two algebraic numbers (since we already know that $\sqrt
{2}+\sqrt{3}=w$ is algebraic). Is it algebraic? Neither of our above two
methods properly works here; do we have to come up with new ad-hoc tricks?
\end{proof}

\begin{center}
\textbf{2019-01-25 lecture}
\end{center}

\subsection{Motivation: Shamir's Secret Sharing Scheme}

\subsubsection{The problem}

Adi Shamir is one of the founders of modern mathematical cryptography (famous
in particular for \href{https://en.wikipedia.org/wiki/RSA_(cryptosystem)}{the
RSA cryptosystem}, see later).

Shamir's Secret Sharing Scheme is a way in which a secret $\mathbf{a}$ (a
piece of data -- e.g., nuclear launch codes) can be distributed among $n$
people in such a way that

\begin{itemize}
\item any $k$ of them can (if they come together) reconstruct it uniquely, but

\item any $k-1$ of them (if they come together) cannot gain \textbf{any}
insight about it (i.e., not only cannot they reconstruct it, but they cannot
even tell that some values are more likely than others to be $\mathbf{a}$).
\end{itemize}

Here $n$ and $k$ are fixed positive integers.

Understanding this scheme completely will require some abstract algebra, but
we can already start thinking about the problem and get reasonably far.

So we have $n$ people $1,2,\ldots,n$, a positive integer $k\in\left\{
1,2,\ldots,n\right\}  $ and a secret piece of data $\mathbf{a}$. We assume
that this data $\mathbf{a}$ is encoded as a \textit{bitstring} -- i.e., a
finite sequence of bits. A \textit{bit} is an element of the set $\left\{
0,1\right\}  $. Thus, examples of bitstrings are $\left(  0,1,1,0\right)  $
and $\left(  1,0\right)  $ and $\left(  1,1,0,1,0,0,0\right)  $ as well as the
empty sequence $\left(  {}\right)  $. When writing bitstring, we shall usually
omit both the commas and the parentheses; thus, e.g., the bitstring $\left(
1,1,0,1,0,0,0\right)  $ will become $1101000$. Make sure you don't mistake it
for a number. Our goal is to give each of the $n$ people $1,2,\ldots,n$ some
bitstring in such a way that:

\begin{itemize}
\item \textit{Requirement 1:} Any $k$ of the $n$ people can (if they come
together) reconstruct $\mathbf{a}$ uniquely.

\item \textit{Requirement 2:} Any $k-1$ of the $n$ people are unable to gain
any insight about $\mathbf{a}$ (even if they collaborate).
\end{itemize}

We denote the bitstrings given to the people $1,2,\ldots,n$ by $\mathbf{a}%
_{1},\mathbf{a}_{2},\ldots,\mathbf{a}_{n}$, respectively.

We assume that the length of our secret bitstring $\mathbf{a}$ is known in
advance to all parties; i.e., it is not a secret. Thus, when we say
\textquotedblleft$k-1$ persons cannot gain any insight about $\mathbf{a}%
$\textquotedblright, we do not mean that they don't know the length; and when
we say \textquotedblleft some values are more likely than others to be
$\mathbf{a}$\textquotedblright, we only mean values that fit this length.

\subsubsection{The $k=1$ case}

One simple special case of our problem is when $k=1$. In this case, it
suffices to give each of the $n$ people the full secret $\mathbf{a}$ (that is,
we set $\mathbf{a}_{i}=\mathbf{a}$ for all $i$). Then, Requirement 1 is
satisfied (since any $1$ of the $n$ people already knows $\mathbf{a}$), while
Requirement 2 is satisfied as well ($0$ people know nothing).

\subsubsection{The $k=n$ case: what doesn't work}

Let us now consider the case when $k=n$. This case will not help us solve the
general problem, but it will show some ideas that we will encounter again and
again in abstract algebra.

We want to ensure that all $n$ people needed to reconstruct the secret
$\mathbf{a}$, while any $n-1$ of them will be completely clueless.

It sounds reasonable to split $\mathbf{a}$ into $n$ parts, and give each
person one of these parts\footnote{assuming that $\mathbf{a}$ is long enough
for that} (i.e., we let $\mathbf{a}_{i}$ be the $i$-th part of $\mathbf{a}$
for each $i\in\left\{  1,2,\ldots,n\right\}  $). This method satisfies
Requirement 1 (indeed, all $n$ people together can reconstruct $\mathbf{a}$
simply by fusing the $n$ parts back together), but fails Requirement 2
(indeed, any $n-1$ people know $n-1$ parts of the secret $\mathbf{a}$, which
is a far from being clueless about $\mathbf{a}$). So this method doesn't work.
It is not that easy.

\subsubsection{The $\operatorname*{XOR}$ operations}

One way to solve the $k=n$ case is using the $\operatorname*{XOR}$ operation.

Let us first define some basic language. A \textit{binary operation} on a set
$S$ is (informally speaking) a function that takes two elements of $S$ and
assigns a new element of $S$ to them. More formally:

\begin{definition}
A \textit{binary operation} on a set $S$ is a map $f$ from $S\times S$ to $S$.
When $f$ is a binary operation on $S$ and $a$ and $b$ are two elements of $S$,
we shall write $afb$ for the value $f\left(  a,b\right)  $.
\end{definition}

\begin{example}
Addition, subtraction and multiplication of integers are three binary
operations on the set $\mathbb{Q}$ (the set of all rational numbers). For
example, addition is the map from $\mathbb{Q}\times\mathbb{Q}$ to $\mathbb{Q}$
that sends each pair $\left(  a,b\right)  \in\mathbb{Q}\times\mathbb{Q}$ to
$a+b$.

Division is not a binary operation on the set $\mathbb{Q}$. Indeed, if it was,
then it would send the pair $\left(  1,0\right)  $ to some integer called
$1/0$; but there is no such integer.

There are myriad more complicated binary operations around waiting for someone
to name them. For example, you could define a binary operation $\smiley{}$ on
the set $\mathbb{Q}$ by $a\smiley{}b=\dfrac{a-b}{1+a^{2}+b^{2}}$. Indeed, you
can do this because $1+a^{2}+b^{2}$ is always nonzero when $a,b\in\mathbb{Q}$
(after all, squares are nonnegative, so that $1+\underbrace{a^{2}}_{\geq
0}+\underbrace{b^{2}}_{\geq0}\geq1>0$). I am not saying that you should...
\end{example}

Now, we define some specific binary operations on the set $\left\{
0,1\right\}  $ of all bits, and on the set $\left\{  0,1\right\}  ^{n}$ of all
length-$n$ bitstrings (for a given $n$).

\begin{definition}
We define a binary operation $\operatorname*{XOR}$ on the set $\left\{
0,1\right\}  $ by setting%
\begin{align*}
0\operatorname*{XOR}0  &  =0,\\
0\operatorname*{XOR}1  &  =1,\\
1\operatorname*{XOR}0  &  =1,\\
1\operatorname*{XOR}1  &  =0.
\end{align*}
This is a valid definition, because there are only four pairs $\left(
a,b\right)  \in\left\{  0,1\right\}  \times\left\{  0,1\right\}  $, and we
have just defined $a\operatorname*{XOR}b$ for each of these four options. We
can also rewrite this definition as follows:%
\[
a\operatorname*{XOR}b=%
\begin{cases}
1, & \text{if }a\neq b;\\
0, & \text{if }a=b
\end{cases}
=%
\begin{cases}
1, & \text{if \textbf{exactly} one of }a\text{ and }b\text{ is }1;\\
0, & \text{otherwise.}%
\end{cases}
\]
For lack of a better name, we refer to $a\operatorname*{XOR}b$ as the
\textquotedblleft XOR of $a$ and $b$\textquotedblright.
\end{definition}

The name \textquotedblleft$\operatorname*{XOR}$\textquotedblright\ is short
for \textquotedblleft exclusive or\textquotedblright. In fact, if you identify
bits with boolean truth values (so the bit $0$ stands for \textquotedblleft
False\textquotedblright\ and the bit $1$ stands for \textquotedblleft
True\textquotedblright), then $a\operatorname*{XOR}b$ is precisely the truth
value for \textquotedblleft exactly one of $a$ and $b$ is
True\textquotedblright, which is also known as \textquotedblleft$a$
exclusive-or $b$\textquotedblright.

\begin{definition}
Let $m$ be a nonnegative integer. We define a binary operation
$\operatorname*{XOR}$ on the set $\left\{  0,1\right\}  ^{m}$ (this is the set
of all length-$m$ bitstrings) by%
\[
\left(  a_{1},a_{2},\ldots,a_{m}\right)  \operatorname*{XOR}\left(
b_{1},b_{2},\ldots,b_{m}\right)  =\left(  a_{1}\operatorname*{XOR}b_{1}%
,a_{2}\operatorname*{XOR}b_{2},\ldots,a_{m}\operatorname*{XOR}b_{m}\right)  .
\]
In other words, if $\mathbf{a}$ and $\mathbf{b}$ are two length-$m$
bitstrings, then $\mathbf{a}\operatorname*{XOR}\mathbf{b}$ is obtained by
taking the XOR of each entry of $\mathbf{a}$ with the corresponding entry of
$\mathbf{b}$, and packing these $m$ XORs into a new length-$m$ bitstring.
\end{definition}

For example,%
\begin{align*}
\left(  1001\right)  \operatorname*{XOR}\left(  1100\right)   &  =0101;\\
\left(  11011\right)  \operatorname*{XOR}\left(  10101\right)   &  =01110;\\
\left(  11010\right)  \operatorname*{XOR}\left(  01011\right)   &  =10001;\\
\left(  1\right)  \operatorname*{XOR}\left(  0\right)   &  =1;\\
\left(  {}\right)  \operatorname*{XOR}\left(  {}\right)   &  =\left(
{}\right)  .
\end{align*}


Note that if $\mathbf{a}$ and $\mathbf{b}$ are two length-$m$ bitstrings, then
the $0$'s in the bitstring $\mathbf{a}\operatorname*{XOR}\mathbf{b}$ are at
the positions where $\mathbf{a}$ and $\mathbf{b}$ have equal entries, and the
$1$'s in $\mathbf{a}\operatorname*{XOR}\mathbf{b}$ are at the positions where
$\mathbf{a}$ and $\mathbf{b}$ have different entries. Thus, the operation
$\operatorname*{XOR}$ on bitstring essentially pinpoints the differences
between $\mathbf{a}$ and $\mathbf{b}$.

We observe the following simple properties of these operations
$\operatorname*{XOR}$ on bits and on bitstrings\footnote{As a mnemonic, we
shall try to use boldfaced letters like $\mathbf{a}$ and $\mathbf{b}$ for
bitstrings and regular italic letters like $a$ and $b$ for single bits.}:

\begin{itemize}
\item We have $a\operatorname*{XOR}0=a$ for any bit $a$. (This can be
trivially checked by considering both possibilities for $a$.)

\item Thus, $\mathbf{a}\operatorname*{XOR}\mathbf{0}=\mathbf{a}$ for any
bitstring $\mathbf{a}$, where $\mathbf{0}$ denotes the bitstring
$00\cdots0=\left(  0,0,\ldots,0\right)  $ (of appropriate length -- i.e., of
the same length as $\mathbf{a}$).

\item We have $a\operatorname*{XOR}a=0$ for any bit $a$. (This can be
trivially checked by considering both possibilities for $a$.)

\item Thus, $\mathbf{a}\operatorname*{XOR}\mathbf{a}=\mathbf{0}$ for any
bitstring $\mathbf{a}$. We shall refer to this as the
\textit{self-cancellation law}.

\item We have $a\operatorname*{XOR}b=b\operatorname*{XOR}a$ for any bits
$a,b$. (Again, this is easy to check by going through all four options for $a$
and $b$.)

\item Thus, $\mathbf{a}\operatorname*{XOR}\mathbf{b}=\mathbf{b}%
\operatorname*{XOR}\mathbf{a}$ for any bitstrings $\mathbf{a},\mathbf{b}$.

\item We have $a\operatorname*{XOR}\left(  b\operatorname*{XOR}c\right)
=\left(  a\operatorname*{XOR}b\right)  \operatorname*{XOR}c$ for any bits
$a,b,c$. (Again, this is easy to check by going through all eight options for
$a,b,c$.)

\item Thus, $\mathbf{a}\operatorname*{XOR}\left(  \mathbf{b}%
\operatorname*{XOR}\mathbf{c}\right)  =\left(  \mathbf{a}\operatorname*{XOR}%
\mathbf{b}\right)  \operatorname*{XOR}\mathbf{c}$ for any bitstrings
$\mathbf{a},\mathbf{b},\mathbf{c}$.

\item Thus, for any bitstrings $\mathbf{a}$ and $\mathbf{b}$, we have%
\[
\left(  \mathbf{a}\operatorname*{XOR}\mathbf{b}\right)  \operatorname*{XOR}%
\mathbf{b}=\mathbf{a}\operatorname*{XOR}\underbrace{\left(  \mathbf{b}%
\operatorname*{XOR}\mathbf{b}\right)  }_{\substack{=\mathbf{0}\\\text{(by the
self-cancellation law)}}}=\mathbf{a}\operatorname*{XOR}\mathbf{0}=\mathbf{a}.
\]


This observation gives rise to a primitive cryptosystem (known as a
\textit{\href{https://en.wikipedia.org/wiki/One-time_pad}{\textit{one-time
pad}}}): If you have a secret bitstring $\mathbf{a}$ that you want to encrypt,
and another secret bitstring $\mathbf{b}$ that can be used as a key, then you
can encrypt\ $\mathbf{a}$ by XORing it with $\mathbf{b}$ (that is, you
transform it into $\mathbf{a}\operatorname*{XOR}\mathbf{b}$). Then, you can
decrypt it again by XORing it with $\mathbf{b}$ again; indeed, if you do this,
you will obtain $\left(  \mathbf{a}\operatorname*{XOR}\mathbf{b}\right)
\operatorname*{XOR}\mathbf{b}=\mathbf{a}$. This is a highly safe cryptosystem
as long as you can safely communicate the key $\mathbf{b}$ to whomever needs
to be able to decrypt (or encrypt) your secrets, and as long as you are able
to generate uniformly random keys $\mathbf{b}$ of sufficient length. Its only
weakness is its impracticality (in many situations): If the secret you want to
encrypt is long (say, a whole book), your key will need to be equally long.
Even storing such keys can become difficult.
\end{itemize}

We shall refer to the properties $a\operatorname*{XOR}b=b\operatorname*{XOR}a$
and $\mathbf{a}\operatorname*{XOR}\mathbf{b}=\mathbf{b}\operatorname*{XOR}%
\mathbf{a}$ as \textit{laws of commutativity}, and we shall refer to the
properties $a\operatorname*{XOR}\left(  b\operatorname*{XOR}c\right)  =\left(
a\operatorname*{XOR}b\right)  \operatorname*{XOR}c$ and $\mathbf{a}%
\operatorname*{XOR}\left(  \mathbf{b}\operatorname*{XOR}\mathbf{c}\right)
=\left(  \mathbf{a}\operatorname*{XOR}\mathbf{b}\right)  \operatorname*{XOR}%
\mathbf{c}$ as \textit{laws of associativity}. These are, of course, similar
to well-known facts like $\alpha+\beta=\beta+\alpha$ and $\alpha+\left(
\beta+\gamma\right)  =\left(  \alpha+\beta\right)  +\gamma$ for numbers
$\alpha,\beta,\gamma$ (which is why we are giving them the same name). This
similarity is not coincidental. Just as for addition or multiplication of
numbers, these laws lead to a notion of \textquotedblleft
XOR-products\textquotedblright:

\begin{proposition}
\label{prop.intro.xor.prodm}Let $m$ be a positive integer. Let $\mathbf{a}%
_{1},\mathbf{a}_{2},\ldots,\mathbf{a}_{m}$ be $m$ bitstrings. Then, the
\textquotedblleft$\operatorname*{XOR}$-product\textquotedblright\ expression%
\[
\mathbf{a}_{1}\operatorname*{XOR}\mathbf{a}_{2}\operatorname*{XOR}%
\mathbf{a}_{3}\operatorname*{XOR}\cdots\operatorname*{XOR}\mathbf{a}_{m}%
\]
is well-defined, in the sense that it does not depend on the parenthesization.
\end{proposition}

What do we mean by \textquotedblleft parenthesization\textquotedblright? To
clarify things, let us set $m=4$. In this case, we want to make sense of the
expression $\mathbf{a}_{1}\operatorname*{XOR}\mathbf{a}_{2}\operatorname*{XOR}%
\mathbf{a}_{3}\operatorname*{XOR}\mathbf{a}_{4}$. This expression does not
make sense a priori, since it is a $\operatorname*{XOR}$ of \textbf{four}
bitstrings, whereas we have defined only the $\operatorname*{XOR}$ of
\textbf{two} bitstrings. But there are five ways to put parentheses around
some of its sub-expressions such that the expression becomes meaningful:
\begin{align*}
&  \left(  \mathbf{a}_{1}\operatorname*{XOR}\mathbf{a}_{2}\right)
\operatorname*{XOR}\left(  \mathbf{a}_{3}\operatorname*{XOR}\mathbf{a}%
_{4}\right)  ,\\
&  \left(  \left(  \mathbf{a}_{1}\operatorname*{XOR}\mathbf{a}_{2}\right)
\operatorname*{XOR}\mathbf{a}_{3}\right)  \operatorname*{XOR}\mathbf{a}_{4},\\
&  \mathbf{a}_{1}\operatorname*{XOR}\left(  \left(  \mathbf{a}_{2}%
\operatorname*{XOR}\mathbf{a}_{3}\right)  \operatorname*{XOR}\mathbf{a}%
_{4}\right)  ,\\
&  \mathbf{a}_{1}\operatorname*{XOR}\left(  \mathbf{a}_{2}\operatorname*{XOR}%
\left(  \mathbf{a}_{3}\operatorname*{XOR}\mathbf{a}_{4}\right)  \right)  ,\\
&  \left(  \mathbf{a}_{1}\operatorname*{XOR}\left(  \mathbf{a}_{2}%
\operatorname*{XOR}\mathbf{a}_{3}\right)  \right)  \operatorname*{XOR}%
\mathbf{a}_{4}.
\end{align*}
Each of these five parenthesizations (= placements of parentheses) turns our
expression $\mathbf{a}_{1}\operatorname*{XOR}\mathbf{a}_{2}\operatorname*{XOR}%
\mathbf{a}_{3}\operatorname*{XOR}\mathbf{a}_{4}$ into a combination of
$\operatorname*{XOR}$'s of \textbf{two} bitstrings each, and thus gives it
meaning. The question is: Do these five parenthesizations give it the
\textbf{same} meaning?

Well, let us calculate:%
\begin{align*}
&  \left(  \mathbf{a}_{1}\operatorname*{XOR}\mathbf{a}_{2}\right)
\operatorname*{XOR}\left(  \mathbf{a}_{3}\operatorname*{XOR}\mathbf{a}%
_{4}\right) \\
&  =\mathbf{a}_{1}\operatorname*{XOR}\underbrace{\left(  \mathbf{a}%
_{2}\operatorname*{XOR}\left(  \mathbf{a}_{3}\operatorname*{XOR}\mathbf{a}%
_{4}\right)  \right)  }_{=\left(  \mathbf{a}_{2}\operatorname*{XOR}%
\mathbf{a}_{3}\right)  \operatorname*{XOR}\mathbf{a}_{4}}\\
&  =\mathbf{a}_{1}\operatorname*{XOR}\left(  \left(  \mathbf{a}_{2}%
\operatorname*{XOR}\mathbf{a}_{3}\right)  \operatorname*{XOR}\mathbf{a}%
_{4}\right) \\
&  =\underbrace{\left(  \mathbf{a}_{1}\operatorname*{XOR}\left(
\mathbf{a}_{2}\operatorname*{XOR}\mathbf{a}_{3}\right)  \right)  }_{=\left(
\mathbf{a}_{1}\operatorname*{XOR}\mathbf{a}_{2}\right)  \operatorname*{XOR}%
\mathbf{a}_{3}}\operatorname*{XOR}\mathbf{a}_{4}\\
&  =\left(  \left(  \mathbf{a}_{1}\operatorname*{XOR}\mathbf{a}_{2}\right)
\operatorname*{XOR}\mathbf{a}_{3}\right)  \operatorname*{XOR}\mathbf{a}_{4},
\end{align*}
where we used the law of associativity in each step. This shows that our five
parenthesizations yield the same result. Thus, they all give our
\textquotedblleft$\operatorname*{XOR}$-product\textquotedblright\ expression
$\mathbf{a}_{1}\operatorname*{XOR}\mathbf{a}_{2}\operatorname*{XOR}%
\mathbf{a}_{3}\operatorname*{XOR}\mathbf{a}_{4}$ the same meaning; so we can
say that this expression is well-defined. This confirms Proposition
\ref{prop.intro.xor.prodm} for $m=4$.

Of course, proving Proposition \ref{prop.intro.xor.prodm} is less simple. Such
a proof will appear in Exercise 4 on homework set \#0.

\subsubsection{The $k=n$ case: an answer}

Let us now return to our problem. We have $n$ persons $1,2,\ldots,n$ and a
secret $\mathbf{a}$ (encoded as a bitstring). We want to give each person $i$
some bitstring $\mathbf{a}_{i}$ such that only all $n$ of them can recover
$\mathbf{a}$ but any $n-1$ of them cannot gain any insight about $\mathbf{a}$.

We let $\mathbf{a}_{1},\mathbf{a}_{2},\ldots,\mathbf{a}_{n-1}$ be $n-1$
\textbf{uniformly} random bitstrings of the same length as $\mathbf{a}$.
(Think of them as random gibberish.) Set%
\[
\mathbf{a}_{n}=\mathbf{a}\operatorname*{XOR}\mathbf{a}_{1}\operatorname*{XOR}%
\mathbf{a}_{2}\operatorname*{XOR}\cdots\operatorname*{XOR}\mathbf{a}_{n-1}.
\]
(This expression makes sense because of Proposition \ref{prop.intro.xor.prodm}.)

Then,%
\begin{align*}
&  \mathbf{a}_{n}\operatorname*{XOR}\mathbf{a}_{n-1}\operatorname*{XOR}%
\mathbf{a}_{n-2}\operatorname*{XOR}\cdots\operatorname*{XOR}\mathbf{a}_{1}\\
&  =\left(  \mathbf{a}\operatorname*{XOR}\mathbf{a}_{1}\operatorname*{XOR}%
\mathbf{a}_{2}\operatorname*{XOR}\cdots\operatorname*{XOR}\mathbf{a}%
_{n-1}\right)  \operatorname*{XOR}\mathbf{a}_{n-1}\operatorname*{XOR}%
\mathbf{a}_{n-2}\operatorname*{XOR}\cdots\operatorname*{XOR}\mathbf{a}_{1}\\
&  =\mathbf{a}\operatorname*{XOR}\mathbf{a}_{1}\operatorname*{XOR}%
\mathbf{a}_{2}\operatorname*{XOR}\cdots\operatorname*{XOR}%
\underbrace{\mathbf{a}_{n-1}\operatorname*{XOR}\mathbf{a}_{n-1}}_{=\mathbf{0}%
}\operatorname*{XOR}\mathbf{a}_{n-2}\operatorname*{XOR}\cdots
\operatorname*{XOR}\mathbf{a}_{1}\\
&  =\mathbf{a}\operatorname*{XOR}\mathbf{a}_{1}\operatorname*{XOR}%
\mathbf{a}_{2}\operatorname*{XOR}\cdots\operatorname*{XOR}%
\underbrace{\mathbf{a}_{n-2}\operatorname*{XOR}\mathbf{0}}_{=\mathbf{a}_{n-2}%
}\operatorname*{XOR}\mathbf{a}_{n-2}\operatorname*{XOR}\cdots
\operatorname*{XOR}\mathbf{a}_{1}\\
&  =\mathbf{a}\operatorname*{XOR}\mathbf{a}_{1}\operatorname*{XOR}%
\mathbf{a}_{2}\operatorname*{XOR}\cdots\operatorname*{XOR}%
\underbrace{\mathbf{a}_{n-2}\operatorname*{XOR}\mathbf{a}_{n-2}}_{=\mathbf{0}%
}\operatorname*{XOR}\cdots\operatorname*{XOR}\mathbf{a}_{1}\\
&  =\cdots\\
&  =\mathbf{a}%
\end{align*}
(here, we have been unravelling the big $\operatorname*{XOR}$-product from the
middle on, by cancelling equal bitstrings using the self-cancellation law and
then removing the resulting $\mathbf{0}$ using the $\mathbf{a}%
\operatorname*{XOR}\mathbf{0}=\mathbf{a}$ law). Hence, the $n$ people together
can decrypt the secret $\mathbf{a}$.

Can $n-1$ people gain any insight about it? The $n-1$ people $1,2,\ldots,n-1$
certainly cannot, since all they know are the random bitstrings $\mathbf{a}%
_{1},\mathbf{a}_{2},\ldots,\mathbf{a}_{n-1}$. But the $n-1$ people
$2,3,\ldots,n$ cannot gain any insight about $\mathbf{a}$ either: In fact, all
they know are the random bitstrings $\mathbf{a}_{2},\mathbf{a}_{3}%
,\ldots,\mathbf{a}_{n-1}$ and the bitstring%
\[
\mathbf{a}_{n}=\mathbf{a}\operatorname*{XOR}\mathbf{a}_{1}\operatorname*{XOR}%
\mathbf{a}_{2}\operatorname*{XOR}\cdots\operatorname*{XOR}\mathbf{a}_{n-1};
\]
therefore, all the information they have about $\mathbf{a}$ and $\mathbf{a}%
_{1}$ comes to them through $\mathbf{a}\operatorname*{XOR}\mathbf{a}_{1}$,
which says nothing about $\mathbf{a}$ as long as they know nothing about
$\mathbf{a}_{1}$. (We used a bit of handwaving in this argument, but then
again we never formally defined what it means to \textquotedblleft gain no
insight\textquotedblright; this is done in courses on cryptography and
information theory.) Similar arguments show that any other choice of $n-1$
persons remains equally clueless about $\mathbf{a}$. So we have solved the
problem in the case $k=n$.

\subsubsection{The $k=2$ case}

The next simple case is when $k=2$. So we want to ensure that any $2$ of our
$n$ people can together recover the secret, but no $1$ person can learn
anything about it alone.

A really nice approach was suggested by Nathan in class: We pick $n$ random
bitstrings $\mathbf{x}_{1},\mathbf{x}_{2},\ldots,\mathbf{x}_{n-1}$ of the same
length as $\mathbf{a}$. Set
\[
\mathbf{x}_{n}=\mathbf{a}\operatorname*{XOR}\mathbf{x}_{1}\operatorname*{XOR}%
\mathbf{x}_{2}\operatorname*{XOR}\cdots\operatorname*{XOR}\mathbf{x}_{n-1};
\]
thus, as in the $k=n$ case, we have%
\begin{equation}
\mathbf{x}_{n}\operatorname*{XOR}\mathbf{x}_{n-1}\operatorname*{XOR}%
\mathbf{x}_{n-2}\operatorname*{XOR}\cdots\operatorname*{XOR}\mathbf{x}%
_{1}=\mathbf{a}. \label{eq.intro.shamir.k=2.2}%
\end{equation}


Each person $i$ now receives the bitstring%
\[
\mathbf{a}_{i}=\mathbf{x}_{1}\mathbf{x}_{2}\cdots\mathbf{x}_{i-1}%
\mathbf{x}_{i+1}\mathbf{x}_{i+2}\cdots\mathbf{x}_{n},
\]
where the product stands for \textit{concatenation} (i.e., the bitstring
$\mathbf{a}_{i}$ is formed by writing down all of the bitstring $\mathbf{x}%
_{1},\mathbf{x}_{2},\ldots,\mathbf{x}_{n}$ one after the other but skipping
$\mathbf{x}_{i}$). Thus, each person $i$ can recover all the $n-1$ bitstrings
$\mathbf{x}_{1},\mathbf{x}_{2},\ldots,\mathbf{x}_{i-1},\mathbf{x}%
_{i+1},\mathbf{x}_{i+2},\ldots,\mathbf{x}_{n}$ (because their lengths are the
length of $\mathbf{a}$, which is known), but knows nothing about
$\mathbf{x}_{i}$ (his \textquotedblleft blind spot\textquotedblright). Hence,
$2$ people together can recover all the $n$ bitstrings $\mathbf{x}%
_{1},\mathbf{x}_{2},\ldots,\mathbf{x}_{n}$ and therefore recover the secret
$\mathbf{a}$ (by (\ref{eq.intro.shamir.k=2.2})). On the other hand, each
single person has no insight about $\mathbf{a}$ (this is proven similarly to
the $k=n$ case). So again, the problem is solved in this case.

\subsubsection{The $k=3$ case}

Now, let us come to the case when $k=3$. Now I think the usefulness of the
$\operatorname*{XOR}$ approach has come to its end: at least I don't know how
to make it work here. Instead, out of the blue, I will invoke something
completely different: polynomials (let's say with rational coefficients).

Recall a fact you might have heard in high school: A polynomial $p\left(
x\right)  =cx^{2}+bx+a$ of degree $\leq2$ is uniquely determined by any three
of its values. More precisely: If $u,v,w$ are three fixed distinct numbers,
then a polynomial $p\left(  x\right)  =cx^{2}+bx+a$ of degree $\leq2$ is
uniquely determined by the values $p\left(  u\right)  ,p\left(  v\right)
,p\left(  w\right)  $. We will put this to use now, and sort-of solve the problem.

Also recall that any bitstring of given length $N$ can be encoded as an
integer in $\left\{  0,1,\ldots,2^{N}-1\right\}  $; just read it as a number
in binary. More precisely, any bitstring $a_{N-1}a_{N-2}\cdots a_{0}$ of
length $N$ becomes the integer $a_{N-1}\cdot2^{N-1}+a_{N-2}\cdot2^{N-2}%
+\cdots+a_{0}\cdot2^{0}\in\left\{  0,1,\ldots,2^{N}-1\right\}  $. For example,
the bitstring $010110$ of length $6$ becomes the integer%
\[
0\cdot2^{5}+1\cdot2^{4}+0\cdot2^{3}+1\cdot2^{2}+1\cdot2^{1}+0\cdot2^{0}%
=22\in\left\{  0,1,\ldots,2^{6}-1\right\}  .
\]


Choose two \textbf{uniformly random} bitstrings $\mathbf{c}$ and $\mathbf{b}$
(of the same length as $\mathbf{a}$) and encode them as numbers $c$ and $b$
(as just explained). Encode the secret $\mathbf{a}$ as a number $a$ as well
(in the same way). Define the polynomial $p\left(  x\right)  =cx^{2}+bx+a$.
Reveal to each person $i\in\left\{  1,2,\ldots,n\right\}  $ the value
$p\left(  i\right)  $ -- or, rather, a bitstring that encodes it in binary --
as $\mathbf{a}_{i}$.

As we know, any three of the values $p\left(  i\right)  $ uniquely determine
the polynomial $p$. Thus, any three people can use their bitstrings
$\mathbf{a}_{i}$ to recover three values $p\left(  i\right)  $ and therefore
$p$ and therefore $a$ (as the constant term of $p$) and therefore $\mathbf{a}$
(by decoding $a$). So our method satisfies Requirement 1.

Now, let us see whether it satisfies Requirement 2. Any $2$ people can recover
two values $p\left(  i\right)  $, which generally do not determine $p$
uniquely. It is not hard to show that they do not even determine $a$ uniquely;
thus, they do not determine $\mathbf{a}$ uniquely. What's better: If you know
just two values of $p$, there are infinitely many possible choices for $p$,
and all of them have distinct constant terms (unless one of the two values you
know is $p\left(  0\right)  $, which of course pins down the constant term).
So we get infinitely many possible values for $a$, and thus infinitely many
possible values for $\mathbf{a}$. This means that our $2$ people don't gain
any insight about $\mathbf{a}$, right?

Not so fast! We cannot really have \textquotedblleft infinitely many possible
values for $\mathbf{a}$\textquotedblright, since $\mathbf{a}$ is bound to be a
bitstring of a given length -- there are only finitely many of those! You can
only get infinitely many possible values for $p$ if you forget how $p$ was
constructed (from $c$, $b$ and $a$) and pretend that $p$ is just a
\textquotedblleft uniformly random\textquotedblright\ polynomial (whatever
this means). But no one can force the $2$ people to do this; it is certainly
not in their interest! Here are some things they might do with this knowledge:

\begin{itemize}
\item Let $N$ be the length of $\mathbf{a}$ (which, as we said, is known).
Thus, $\mathbf{c}$ and $\mathbf{b}$ are bitstrings of length $N$, so that $c$
and $b$ are integers in $\left\{  0,1,\ldots,2^{N}-1\right\}  $. Assume that
one of the $2$ people is person $2$. Now, person $2$ knows $p\left(  2\right)
=c2^{2}+b2+a=4c+2b+a$, and thus knows whether $a$ is even or odd (because $a$
is even resp. odd if and only if $4c+2b+a$ is even resp. odd). This means she
knows the last bit of the secret $\mathbf{a}$. This is not \textquotedblleft
clueless\textquotedblright.

\item You might try to fix this by picking $c$ and $b$ to be uniformly random
rational numbers instead (rather than using uniformly random bitstrings
$\mathbf{c}$ and $\mathbf{b}$).

Unfortunately, there is no such thing as a \textquotedblleft uniformly random
rational number\textquotedblright\ (in the sense that, e.g., larger numbers
aren't less likely to be picked than smaller numbers). Any probability
distribution will make some numbers more likely than others, and this will
usually cause information about $\mathbf{a}$ to \textquotedblleft
leak\textquotedblright. For example, if $c$ and $b$ are chosen from the
interval $\left[  0,2^{N}-1\right]  $, then person $1$'s knowledge of
$p\left(  1\right)  =c1^{2}+b1+a=c+b+a$ will sometimes reveal to person $1$
that $a\geq0.5\cdot\left(  2^{N}-1\right)  $ (namely, this will happen when
$p\left(  1\right)  \geq2.5\cdot\left(  2^{N}-1\right)  $, which occasionally
happens). This, again, is nontrivial information about the secret $\mathbf{a}%
$, which a single person (or even two people) should not be having.
\end{itemize}

So we cannot make Requirement 2 hold, and the culprit is that there are too
many numbers (namely, infinitely many). What would help is a finite
\textquotedblleft number system\textquotedblright\ in which we can add,
subtract, multiply and divide (so that we can define polynomials over it, and
a polynomial of degree $\leq2$ is still uniquely determined by any $3$
values). Assuming that this \textquotedblleft number system\textquotedblright%
\ is large enough that we can encode bitstrings using \textquotedblleft
numbers\textquotedblright\ of this system (instead of integers), we can then
play the above game using this \textquotedblleft number
system\textquotedblright\ and obtain actually uniformly random numbers.

It turns out that such \textquotedblleft number systems\textquotedblright%
\ exist. They are called \textit{finite fields}, and we will construct them
later in this course.

Assuming that they can be constructed, we thus obtain a method of solving the
problem for $k=3$. A similar method works for arbitrary $k$, using polynomials
of degree $\leq k-1$. This is called \textit{Shamir's secret sharing scheme}.

\begin{center}
\textbf{2019-01-30 lecture (virtual)}
\end{center}

\section{Elementary number theory}

Let us now begin a systematic introduction to algebra. We start with studying
integers and their divisibility properties -- the beginnings of number theory.
Part of these will be used directly in what will follow; part of these will
inspire more general results and proofs.

\subsection{Notations}

\begin{definition}
Let $\mathbb{N}=\left\{  0,1,2,\ldots\right\}  $ be the set of
\textbf{nonnegative} integers.

Let $\mathbb{P}=\left\{  1,2,3,\ldots\right\}  $ be the set of
\textbf{positive} integers.

Let $\mathbb{Z}=\left\{  \ldots,-1,0,1,\ldots\right\}  $ be the set of integers.

Let $\mathbb{Q}$ be the set of rational numbers.

Let $\mathbb{R}$ be the set of real numbers.
\end{definition}

Be careful with the notation $\mathbb{N}$: While I use it for $\left\{
0,1,2,\ldots\right\}  $, various other authors use it for $\left\{
1,2,3,\ldots\right\}  $ instead. There is no consensus in sight on what
$\mathbb{N}$ should mean.

Same holds for the word \textquotedblleft natural number\textquotedblright%
\ (which I will avoid): It means \textquotedblleft element of $\mathbb{N}%
$\textquotedblright, so again its ultimate meaning depends on the author.

\subsection{Divisibility}

We now go through the basics of divisibility of integers.

\begin{definition}
\label{def.ent.div.div}Let $a$ and $b$ be two integers. We say that $a\mid b$
(or \textquotedblleft$a$ \textit{divides} $b$\textquotedblright\ or
\textquotedblleft$b$ is \textit{divisible by }$a$\textquotedblright\ or
\textquotedblleft$b$ is a \textit{multiple} of $a$\textquotedblright) if there
exists an integer $c$ such that $b=ac$.

We furthermore say that $a\nmid b$ if $a$ does not divide $b$.
\end{definition}

Some authors define the \textquotedblleft divisibility\textquotedblright%
\ relation a bit differently, in that they forbid $a=0$. From the viewpoint of
abstract algebra, this feels like an unnecessary exception, so we don't follow them.

\begin{example}
\label{exa.ent.div.triv}\textbf{(a)} We have $4\mid12$, since $12=4\cdot3$.

\textbf{(b)} We have $a\mid0$ for any $a\in\mathbb{Z}$, since $0=a\cdot0$.

\textbf{(c)} An integer $b$ satisfies $0\mid b$ only when $b=0$, since $0\mid
b$ implies $b=0c=0$ (for some $c\in\mathbb{Z}$).

\textbf{(d)} We have $a\mid a$ for any $a\in\mathbb{Z}$, since $a=a\cdot1$.

\textbf{(e)} We have $1\mid b$ for each $b\in\mathbb{Z}$, since $b=1\cdot b$.
\end{example}

I apologize in advance for the next proposition, in which vertical bars stand
both for the \textquotedblleft divides\textquotedblright\ relation and for the
absolute value of a number. Unfortunately, both of these uses are standard
notation. Confusion is possible, but hopefully will not happen
much\footnote{Unfortunately, the use of vertical bars for absolute values
alone suffices to generate confusion! Just think of the meaning of
\textquotedblleft$\left\vert a\right\vert b\left\vert c\right\vert
$\textquotedblright\ when $a$, $b$ and $c$ are three numbers. Does it stand
for \textquotedblleft$\left(  \left\vert a\right\vert \right)  \cdot
b\cdot\left(  \left\vert c\right\vert \right)  $\textquotedblright\ (where I
am using parentheses to make the ambiguity disappear) or for \textquotedblleft%
$\left\vert \left(  a\cdot\left\vert b\right\vert \cdot c\right)  \right\vert
$\textquotedblright? If you see any expressions in my notes that allow for
more than one meaningful interpretation, please let me know!}.

\begin{proposition}
\label{prop.ent.div.1}Let $a$ and $b$ be two integers.

\textbf{(a)} We have $a\mid b$ if and only $\left\vert a\right\vert
\mid\left\vert b\right\vert $. (Here, \textquotedblleft$\left\vert
a\right\vert \mid\left\vert b\right\vert $\textquotedblright\ means
\textquotedblleft$\left\vert a\right\vert $ divides $\left\vert b\right\vert
$\textquotedblright.)

\textbf{(b)} If $a\mid b$ and $b\neq0$, then $\left\vert a\right\vert
\leq\left\vert b\right\vert $.

\textbf{(c)} Assume that $a\neq0$. Then, $a\mid b$ if and only if $\dfrac
{b}{a}\in\mathbb{Z}$.
\end{proposition}

Before we prove this proposition, let us recall a well-known fact: We have
\begin{equation}
\left\vert xy\right\vert =\left\vert x\right\vert \cdot\left\vert y\right\vert
\label{eq.ent.div.abs(xy)}%
\end{equation}
for any two integers\footnote{or real numbers} $x$ and $y$. (This can be
easily proven by case distinction: $x$ is either nonnegative or negative, and
so is $y$.)

\begin{proof}
[Proof of Proposition \ref{prop.ent.div.1}.]\textbf{(a)} $\Longrightarrow
:$\footnote{If you are unfamiliar with the shorthand notation
\textquotedblleft$\Longrightarrow:$\textquotedblright, let me explain it. Our
goal is to prove that $a\mid b$ if and only if $\left\vert a\right\vert
\mid\left\vert b\right\vert $. In other words, we need to prove the
equivalence $\left(  a\mid b\right)  \Longleftrightarrow\left(  \left\vert
a\right\vert \mid\left\vert b\right\vert \right)  $. In order to prove this
equivalence, it suffices to prove the two implications $\left(  a\mid
b\right)  \Longrightarrow\left(  \left\vert a\right\vert \mid\left\vert
b\right\vert \right)  $ (called the \textquotedblleft forward
implication\textquotedblright\ or the \textquotedblleft$\Longrightarrow$
direction\textquotedblright\ of the equivalence) and $\left(  a\mid b\right)
\Longleftarrow\left(  \left\vert a\right\vert \mid\left\vert b\right\vert
\right)  $ (called the \textquotedblleft backward
implication\textquotedblright\ or the \textquotedblleft$\Longleftarrow$
direction\textquotedblright). The shorthand \textquotedblleft$\Longrightarrow
:$\textquotedblright\ simply marks the beginning of the proof of the forward
implication; similarly, the symbol \textquotedblleft$\Longleftarrow
:$\textquotedblright\ heralds in the proof of the backward implication.}
Assume that $a\mid b$. Thus, there exists an integer $d$ such that $b=ad$ (by
Definition \ref{def.ent.div.div}). Consider\footnote{Me saying
\textquotedblleft Consider this $d$\textquotedblright\ means that I am picking
some integer $d$ such that $b=ad$ (this can be done, since we have just proven
that such a $d$ exists), and will be referring to it as $d$ from now on.} this
$d$. We have $b=ad$ and thus $\left\vert b\right\vert =\left\vert
ad\right\vert =\left\vert a\right\vert \cdot\left\vert d\right\vert $ (by
(\ref{eq.ent.div.abs(xy)})). Thus, there exists an integer $c$ such that
$\left\vert b\right\vert =\left\vert a\right\vert \cdot c$ (namely,
$c=\left\vert d\right\vert $). In other words, $\left\vert a\right\vert
\mid\left\vert b\right\vert $. This proves the \textquotedblleft%
$\Longrightarrow$\textquotedblright\ direction of Proposition
\ref{prop.ent.div.1} \textbf{(a)}.

$\Longleftarrow:$ Assume that $\left\vert a\right\vert \mid\left\vert
b\right\vert $. Thus, there exists an integer $f$ such that $\left\vert
b\right\vert =\left\vert a\right\vert \cdot f$ (by Definition
\ref{def.ent.div.div}). Consider this $f$.

The definition of $\left\vert b\right\vert $ shows that $\left\vert
b\right\vert $ equals either $b$ or $-b$. Hence, $b$ equals either $\left\vert
b\right\vert $ or $-\left\vert b\right\vert $. In other words, $b$ equals
either $1\left\vert b\right\vert $ or $\left(  -1\right)  \left\vert
b\right\vert $. In other words, $b=q\left\vert b\right\vert $ for some
$q\in\left\{  1,-1\right\}  $. Similarly, $a=r\left\vert a\right\vert $ for
some $r\in\left\{  1,-1\right\}  $. Consider these $q$ and $r$.

From $r\in\left\{  1,-1\right\}  $, we obtain $r^{2}=1$. Now, $r\underbrace{a}%
_{=r\left\vert a\right\vert }=\underbrace{rr}_{=r^{2}=1}\left\vert
a\right\vert =\left\vert a\right\vert $.

Now, $b=a\cdot qfr$ (since $a\cdot qfr=qf\underbrace{ra}_{=\left\vert
a\right\vert }=q\underbrace{f\left\vert a\right\vert }_{=\left\vert
a\right\vert \cdot f=\left\vert b\right\vert }=q\left\vert b\right\vert =b$).
Hence, there exists an integer $c$ such that $b=ac$ (namely, $c=qfr$). In
other words, $a\mid b$. This proves the \textquotedblleft$\Longleftarrow
$\textquotedblright\ direction of Proposition \ref{prop.ent.div.1}
\textbf{(a)}.

Thus, the proof of Proposition \ref{prop.ent.div.1} \textbf{(a)} is complete.

\textbf{(b)} Assume that $a\mid b$ and $b\neq0$.

From $a\mid b$, we conclude that there exists an integer $c$ such that $b=ac$.
Consider this $c$. We have $ac=b\neq0$, thus $c\neq0$. Hence, $\left\vert
c\right\vert >0$, and thus $\left\vert c\right\vert \geq1$ (since $\left\vert
c\right\vert $ is an integer). We can multiply this inequality by $\left\vert
a\right\vert $ (since $\left\vert a\right\vert \geq0$), and obtain $\left\vert
a\right\vert \cdot\left\vert c\right\vert \geq\left\vert a\right\vert
\cdot1=\left\vert a\right\vert $.

From $b=ac$, we obtain $\left\vert b\right\vert =\left\vert ac\right\vert
=\left\vert a\right\vert \cdot\left\vert c\right\vert $ (by
(\ref{eq.ent.div.abs(xy)})). Hence, $\left\vert b\right\vert =\left\vert
a\right\vert \cdot\left\vert c\right\vert \geq\left\vert a\right\vert $. This
proves Proposition \ref{prop.ent.div.1} \textbf{(b)}.

\textbf{(c)} $\Longrightarrow:$ Assume that $a\mid b$. Thus, there exists an
integer $d$ such that $b=ad$. Consider this $d$. We can divide the equality
$b=ad$ by $a$ (since $a\neq0$), and thus obtain $\dfrac{b}{a}=d\in\mathbb{Z}$.
This proves the $\Longrightarrow$ direction of Proposition
\ref{prop.ent.div.1} \textbf{(c)}.

$\Longleftarrow:$ Assume that $\dfrac{b}{a}\in\mathbb{Z}$. Thus, there exists
an integer $c$ such that $b=ac$ (namely, $c=\dfrac{b}{a}$). In other words,
$a\mid b$. This proves the $\Longleftarrow$ direction of Proposition
\ref{prop.ent.div.1} \textbf{(c)}. Hence, the proof of Proposition
\ref{prop.ent.div.1} \textbf{(c)} is complete.
\end{proof}

Proposition \ref{prop.ent.div.1} \textbf{(a)} shows that both $a$ and $b$ in
\textquotedblleft the statement $a\mid b$\textquotedblright\ can be replaced
by their absolute values. Thus, when we talk about divisibility of integers,
the sign of the integers does not really matter -- it usually suffices to work
with nonnegative integers. We will often use this (tacitly, after a couple
times) in proofs.

The next proposition shows some basic properties of the divisibility relation:

\begin{proposition}
\label{prop.ent.div.2}\textbf{(a)} We have $a\mid a$ for every $a\in
\mathbb{Z}$. (This is called the \textit{reflexivity of divisibility}.)

\textbf{(b)} If $a,b,c\in\mathbb{Z}$ satisfy $a\mid b$ and $b\mid c$, then
$a\mid c$. (This is called the \textit{transitivity of divisibility}.)

\textbf{(c)} If $a_{1},a_{2},b_{1},b_{2}\in\mathbb{Z}$ satisfy $a_{1}\mid
b_{1}$ and $a_{2}\mid b_{2}$, then $a_{1}a_{2}\mid b_{1}b_{2}$.
\end{proposition}

\begin{proof}
\textbf{(a)} Let $a\in\mathbb{Z}$. Then, there exists an integer $c$ such that
$a=ac$ (namely, $c=1$). In other words, $a\mid a$. This proves Proposition
\ref{prop.ent.div.2} \textbf{(a)}.

\textbf{(b)} Let $a,b,c\in\mathbb{Z}$ satisfy $a\mid b$ and $b\mid c$.

From $a\mid b$, we conclude that there exists an integer $d$ such that $b=ad$.
Consider this $d$.

From $b\mid c$, we conclude that there exists an integer $e$ such that $c=be$.
Consider this $e$.

We have $c=\underbrace{b}_{=ad}e=ade$. Hence, there exists an integer $f$ such
that $c=af$ (namely, $f=de$). In other words, $a\mid c$ (by Definition
\ref{def.ent.div.div}). This proves Proposition \ref{prop.ent.div.2}
\textbf{(b)}.

\textbf{(c)} Let $a_{1},a_{2},b_{1},b_{2}\in\mathbb{Z}$ satisfy $a_{1}\mid
b_{1}$ and $a_{2}\mid b_{2}$.

From $a_{1}\mid b_{1}$, we conclude that there exists an integer $d$ such that
$b_{1}=a_{1}d$. Consider this $d$.

From $a_{2}\mid b_{2}$, we conclude that there exists an integer $e$ such that
$b_{2}=a_{2}e$. Consider this $e$.

We have $\underbrace{b_{1}}_{=a_{1}d}\underbrace{b_{2}}_{=a_{2}e}=a_{1}%
da_{2}e=a_{1}a_{2}de$. Hence, there exists an integer $f$ such that
$b_{1}b_{2}=a_{1}a_{2}f$ (namely, $f=de$). In other words, $a_{1}a_{2}\mid
b_{1}b_{2}$ (by Definition \ref{def.ent.div.div}). This proves Proposition
\ref{prop.ent.div.2} \textbf{(c)}.
\end{proof}

\begin{exercise}
\label{exe.ent.div.aabs}Let $a\in\mathbb{Z}$.

\textbf{(a)} Prove that $a\mid\left\vert a\right\vert $. (This means
\textquotedblleft$a$ divides $\left\vert a\right\vert $\textquotedblright.)

\textbf{(b)} Prove that $\left\vert a\right\vert \mid a$. (This means
\textquotedblleft$\left\vert a\right\vert $ divides $a$\textquotedblright.)
\end{exercise}

\begin{fineprint}
\begin{proof}
[Solution to Exercise \ref{exe.ent.div.aabs}.]The definition of $\left\vert
a\right\vert $ shows that $\left\vert a\right\vert $ equals either $a$ or
$-a$. In other words, $\left\vert a\right\vert $ equals either $1a$ or
$\left(  -1\right)  a$. In other words, $\left\vert a\right\vert =qa$ for some
$q\in\left\{  1,-1\right\}  $. Consider this $q$. Clearly, $q$ is an integer.
Now, from $\left\vert a\right\vert =qa=aq$, we conclude that $a\mid\left\vert
a\right\vert $ (since $q$ is an integer). This solves Exercise
\ref{exe.ent.div.aabs} \textbf{(a)}.

\textbf{(b)} From $q\in\left\{  1,-1\right\}  $, we obtain $q^{2}\in\left\{
1^{2},\left(  -1\right)  ^{2}\right\}  =\left\{  1,1\right\}  =\left\{
1\right\}  $, so that $q^{2}=1$. Now, multiplying the equality $\left\vert
a\right\vert =qa$ by $q$, we obtain $q\left\vert a\right\vert =\underbrace{qq}%
_{=q^{2}=1}a=a$. Hence, $a=q\left\vert a\right\vert =\left\vert a\right\vert
\cdot q$. Thus, $\left\vert a\right\vert \mid a$ (since $q$ is an integer).
This solves Exercise \ref{exe.ent.div.aabs} \textbf{(b)}.
\end{proof}
\end{fineprint}

\begin{exercise}
\label{exe.ent.div.abba}Let $a$ and $b$ be two integers such that $a\mid b$
and $b\mid a$. Prove that $\left\vert a\right\vert =\left\vert b\right\vert $.
\end{exercise}

\begin{fineprint}
\begin{proof}
[Solution to Exercise \ref{exe.ent.div.abba}.]We are in one of the following
two cases:

\textit{Case 1:} We have $b\neq0$.

\textit{Case 2:} We have $b=0$.

Let us first consider Case 1. In this case, we have $b\neq0$. Thus,
Proposition \ref{prop.ent.div.1} \textbf{(b)} yields $\left\vert a\right\vert
\leq\left\vert b\right\vert $ (since $a\mid b$).

We have $a\mid b$. In other words, there exists an integer $c$ such that
$b=ac$. Consider this $c$. If we had $a=0$, then we would have
$b=\underbrace{a}_{=0}c=0$, which would contradict $b\neq0$. Thus, we cannot
have $a=0$. Hence, $a\neq0$. Thus, Proposition \ref{prop.ent.div.1}
\textbf{(b)} (applied to $b$ and $a$ instead of $a$ and $b$) yields
$\left\vert b\right\vert \leq\left\vert a\right\vert $ (since $b\mid a$).
Combining this with $\left\vert a\right\vert \leq\left\vert b\right\vert $, we
obtain $\left\vert a\right\vert =\left\vert b\right\vert $. Thus, Exercise
\ref{exe.ent.div.abba} is solved in Case 1.

Let us now consider Case 2. In this case, we have $b=0$. But we have $b\mid
a$. In other words, there exists an integer $c$ such that $a=bc$. Consider
this $c$. Hence, $a=\underbrace{b}_{=0}c=0c=0=b$ (since $b=0$). Thus,
$\left\vert a\right\vert =\left\vert b\right\vert $. Hence, Exercise
\ref{exe.ent.div.abba} is solved in Case 2.

Now, we have solved Exercise \ref{exe.ent.div.abba} in both Cases 1 and 2.
Hence, Exercise \ref{exe.ent.div.abba} always holds.
\end{proof}
\end{fineprint}

\begin{exercise}
\label{exe.ent.div.acbc}Let $a,b,c$ be three integers such that $c\neq0$.
Prove that $a\mid b$ holds if and only if $ac\mid bc$.
\end{exercise}

\begin{fineprint}
\begin{proof}
[Solution to Exercise \ref{exe.ent.div.acbc}.]$\Longrightarrow:$ Assume that
$a\mid b$ holds. We must prove that $ac\mid bc$.

It is easy to do this straight from the definition of divisibility, but here
is a shorter argument: Proposition \ref{prop.ent.div.2} \textbf{(a)} (applied
to $c$ instead of $a$) yields $c\mid c$. Also, $a\mid b$. Hence, Proposition
\ref{prop.ent.div.2} \textbf{(c)} (applied to $a_{1}=a$, $b_{1}=b$, $a_{2}=c$
and $b_{2}=c$) yields $ac\mid bc$. This proves the \textquotedblleft%
$\Longrightarrow$\textquotedblright\ direction of Exercise
\ref{exe.ent.div.acbc}.

$\Longleftarrow:$ Assume that $ac\mid bc$ holds. We must prove that $a\mid b$.

We have $ac\mid bc$. In other words, there exists an integer $d$ such that
$bc=\left(  ac\right)  d$ (by Definition \ref{def.ent.div.div}). Consider this
$d$. We have $bc=\left(  ac\right)  d=adc$. We can divide both sides of this
equality by $c$ (since $c\neq0$), and thus obtain $b=ad$. Thus, there exists
an integer $e$ such that $b=ae$ (namely, $e=d$). In other words, $a\mid b$ (by
Definition \ref{def.ent.div.div}). This proves the \textquotedblleft%
$\Longleftarrow$\textquotedblright\ direction of Exercise
\ref{exe.ent.div.acbc}.
\end{proof}
\end{fineprint}

\begin{exercise}
\label{exe.ent.div.powers}Let $n\in\mathbb{Z}$. Let $a,b\in\mathbb{N}$ be such
that $a\leq b$. Prove that $n^{a}\mid n^{b}$.
\end{exercise}

\begin{fineprint}
\begin{proof}
[Solution to Exercise \ref{exe.ent.div.powers}.]We have $b-a\geq0$ (since
$a\leq b$), thus $b-a\in\mathbb{N}$. Hence, $n^{b-a}$ is a well-defined
integer. Now, $n^{b}=n^{a}n^{b-a}$ (since $n^{a}n^{b-a}=n^{a+\left(
b-a\right)  }=n^{b}$). Hence, there exists an integer $c$ such that
$n^{b}=n^{a}c$ (namely, $c=n^{b-a}$). In other words, $n^{a}\mid n^{b}$ (by
the definition of divisibility). This solves Exercise \ref{exe.ent.div.powers}.
\end{proof}
\end{fineprint}

\subsection{Congruence modulo $n$}

The next definition is simple but crucial:

\begin{definition}
\label{def.ent.cong}Let $n,a,b\in\mathbb{Z}$. We say that $a$ \textit{is
congruent to }$b$ \textit{modulo }$n$ if and only if $n\mid a-b$. We shall use
the notation \textquotedblleft$a\equiv b\operatorname{mod}n$\textquotedblright%
\ for \textquotedblleft$a$ is congruent to $b$ modulo $n$\textquotedblright.

We furthermore shall use the notation \textquotedblleft$a\not \equiv
b\operatorname{mod}n$\textquotedblright\ for \textquotedblleft$a$ is not
congruent to $b$ modulo $n$\textquotedblright.
\end{definition}

\begin{example}
\label{exa.ent.cong.triv}\textbf{(a)} Is $3\equiv7\operatorname{mod}2$ ? Yes,
since $2\mid3-7=-4$.

\textbf{(b)} Is $3\equiv6\operatorname{mod}2$ ? No, since $2\nmid3-6=-3$. So
we have $3\not \equiv 6\operatorname{mod}2$.

Now, let $a$ and $b$ be two integers.

\textbf{(c)} We have $a\equiv b\operatorname{mod}0$ if and only if $a=b$.
(Indeed, $a\equiv b\operatorname{mod}0$ is defined to mean $0\mid a-b$, but
the latter divisibility happens only when $a-b=0$, which is tantamount to
saying $a=b$.)

\textbf{(d)} We have $a\equiv b\operatorname{mod}1$ always, since $1\mid a-b$
always holds (remember: $1$ divides everything).
\end{example}

Note that being congruent modulo $2$ means having the same parity: i.e., two
even numbers will be congruent modulo $2$, and two odd numbers will be, but an
even number will never be congruent to an odd number modulo $2$. (To be
rigorous: This is not quite obvious at this point yet; but it will be easy
once we have properly introduced division with remainder. See Exercise
\ref{exe.ent.even-odd.1} \textbf{(i)} below for the proof.)

\href{https://en.wikipedia.org/wiki/Modulo_(jargon)}{The word
\textquotedblleft modulo\textquotedblright}\ in the phrase \textquotedblleft%
$a$ is congruent to $b$ modulo $n$\textquotedblright\ is due to Gauss and
means something like \textquotedblleft with respect to\textquotedblright. You
should think of \textquotedblleft$a$ is congruent to $b$ modulo $n$%
\textquotedblright\ as a relation between all three of the numbers $a$, $b$
and $n$, but $a$ and $b$ are the \textquotedblleft main
characters\textquotedblright\ and $n$ sets the scenery.

\begin{exercise}
\label{exe.ent.mod.a+b=a-b}Let $a,b\in\mathbb{Z}$. Prove that $a+b\equiv
a-b\operatorname{mod}2$.
\end{exercise}

\begin{fineprint}
\begin{proof}
[Solution to Exercise \ref{exe.ent.mod.a+b=a-b}.]According to Definition
\ref{def.ent.cong}, we have $a+b\equiv a-b\operatorname{mod}2$ if and only if
$2\mid\left(  a+b\right)  -\left(  a-b\right)  $. Thus, it remains to prove
that $2\mid\left(  a+b\right)  -\left(  a-b\right)  $. But this follows
immediately from $\left(  a+b\right)  -\left(  a-b\right)  =2b$. Thus Exercise
\ref{exe.ent.mod.a+b=a-b} is solved.
\end{proof}
\end{fineprint}

We begin with a proposition so fundamental that we will always use it without saying:

\begin{proposition}
\label{prop.ent.mod.0}Let $n\in\mathbb{Z}$ and $a\in\mathbb{Z}$. Then,
$a\equiv0\operatorname{mod}n$ if and only if $n\mid a$.
\end{proposition}

\begin{proof}
[Proof of Proposition \ref{prop.ent.mod.0}.]We have the following chain of
equivalences:%
\begin{align*}
\left(  a\equiv0\operatorname{mod}n\right)  \  &  \Longleftrightarrow\ \left(
n\mid a-0\right)  \ \ \ \ \ \ \ \ \ \ \left(  \text{by Definition
\ref{def.ent.cong}}\right) \\
&  \Longleftrightarrow\ \left(  n\mid a\right)  \ \ \ \ \ \ \ \ \ \ \left(
\text{since }a-0=a\right)  .
\end{align*}
This proves Proposition \ref{prop.ent.mod.0}.
\end{proof}

Next come some staple properties of congruences:

\begin{proposition}
\label{prop.ent.mod.basics}Let $n\in\mathbb{Z}$.

\textbf{(a)} We have $a\equiv a\operatorname{mod}n$ for every $a\in\mathbb{Z}$.

\textbf{(b)} If $a,b,c\in\mathbb{Z}$ satisfy $a\equiv b\operatorname{mod}n$
and $b\equiv c\operatorname{mod}n$, then $a\equiv c\operatorname{mod}n$.

\textbf{(c)} If $a,b\in\mathbb{Z}$ satisfy $a\equiv b\operatorname{mod}n$,
then $b\equiv a\operatorname{mod}n$.

\textbf{(d)} If $a_{1},a_{2},b_{1},b_{2}\in\mathbb{Z}$ satisfy $a_{1}\equiv
b_{1}\operatorname{mod}n$ and $a_{2}\equiv b_{2}\operatorname{mod}n$, then%
\begin{align}
a_{1}+a_{2}  &  \equiv b_{1}+b_{2}\operatorname{mod}%
n;\label{eq.prop.ent.mod.basics.d.1}\\
a_{1}-a_{2}  &  \equiv b_{1}-b_{2}\operatorname{mod}%
n;\label{eq.prop.ent.mod.basics.d.2}\\
a_{1}a_{2}  &  \equiv b_{1}b_{2}\operatorname{mod}n.
\label{eq.prop.ent.mod.basics.d.3}%
\end{align}


\textbf{(e)} Let $m\in\mathbb{Z}$ be such that $m\mid n$. If $a,b\in
\mathbb{Z}$ satisfy $a\equiv b\operatorname{mod}n$, then $a\equiv
b\operatorname{mod}m$.
\end{proposition}

\begin{proof}
\textbf{(a)} Let $a\in\mathbb{Z}$. Recall that $a\equiv a\operatorname{mod}n$
is defined to mean $n\mid a-a$. Since $n\mid a-a$ holds (because
$a-a=0=n\cdot0$), we thus see that $a\equiv a\operatorname{mod}n$ holds. This
proves Proposition \ref{prop.ent.mod.basics} \textbf{(a)}.

\textbf{(b)} Let $a,b,c\in\mathbb{Z}$ satisfy $a\equiv b\operatorname{mod}n$
and $b\equiv c\operatorname{mod}n$.

We have $a\equiv b\operatorname{mod}n$. In other words, $n\mid a-b$ (by
Definition \ref{def.ent.cong}). In other words, there exists an integer $p$
such that $a-b=np$ (by Definition \ref{def.ent.div.div}). Consider this $p$.

We have $b\equiv c\operatorname{mod}n$. In other words, $n\mid b-c$ (by
Definition \ref{def.ent.cong}). In other words, there exists an integer $q$
such that $b-c=nq$ (by Definition \ref{def.ent.div.div}). Consider this $q$.

Now,
\[
a-c=\underbrace{\left(  a-b\right)  }_{=np}+\underbrace{\left(  b-c\right)
}_{=nq}=np+nq=n\left(  p+q\right)  .
\]
Hence, there exists an integer $r$ such that $a-c=nr$ (namely, $r=p+q$). In
other words, $n\mid a-c$ (by Definition \ref{def.ent.div.div}). In other
words, $a\equiv c\operatorname{mod}n$ (by Definition \ref{def.ent.cong}). This
proves Proposition \ref{prop.ent.mod.basics} \textbf{(b)}.

\textbf{(c)} Let $a,b\in\mathbb{Z}$ satisfy $a\equiv b\operatorname{mod}n$.

We have $a\equiv b\operatorname{mod}n$. In other words, $n\mid a-b$ (by
Definition \ref{def.ent.cong}). In other words, there exists an integer $p$
such that $a-b=np$ (by Definition \ref{def.ent.div.div}). Consider this $p$.
Now,%
\[
b-a=-\underbrace{\left(  a-b\right)  }_{=np}=-np=n\left(  -p\right)  .
\]
Hence, there exists an integer $c$ such that $b-a=nc$ (namely, $c=-p$). In
other words, $n\mid b-a$ (by Definition \ref{def.ent.div.div}). In other
words, $b\equiv a\operatorname{mod}n$ (by Definition \ref{def.ent.cong}). This
proves Proposition \ref{prop.ent.mod.basics} \textbf{(c)}.

\textbf{(d)} Let $a_{1},a_{2},b_{1},b_{2}\in\mathbb{Z}$ satisfy $a_{1}\equiv
b_{1}\operatorname{mod}n$ and $a_{2}\equiv b_{2}\operatorname{mod}n$.

We have $a_{1}\equiv b_{1}\operatorname{mod}n$. In other words, $n\mid
a_{1}-b_{1}$ (by Definition \ref{def.ent.cong}). In other words, there exists
an integer $p$ such that $a_{1}-b_{1}=np$ (by Definition \ref{def.ent.div.div}%
). Consider this $p$.

We have $a_{2}\equiv b_{2}\operatorname{mod}n$. In other words, $n\mid
a_{2}-b_{2}$ (by Definition \ref{def.ent.cong}). In other words, there exists
an integer $q$ such that $a_{2}-b_{2}=nq$ (by Definition \ref{def.ent.div.div}%
). Consider this $q$.

We have%
\[
\left(  a_{1}+a_{2}\right)  -\left(  b_{1}+b_{2}\right)  =\underbrace{\left(
a_{1}-b_{1}\right)  }_{=np}+\underbrace{\left(  a_{2}-b_{2}\right)  }%
_{=nq}=np+nq=n\left(  p+q\right)  .
\]
Hence, there exists an integer $c$ such that $\left(  a_{1}+a_{2}\right)
-\left(  b_{1}+b_{2}\right)  =nc$ (namely, $c=p+q$). In other words,
$n\mid\left(  a_{1}+a_{2}\right)  -\left(  b_{1}+b_{2}\right)  $ (by
Definition \ref{def.ent.div.div}). In other words, $a_{1}+a_{2}\equiv
b_{1}+b_{2}\operatorname{mod}n$ (by Definition \ref{def.ent.cong}). A similar
argument (using $p-q$ instead of $p+q$) shows that $a_{1}-a_{2}\equiv
b_{1}-b_{2}\operatorname{mod}n$. It thus remains to show that $a_{1}%
a_{2}\equiv b_{1}b_{2}\operatorname{mod}n$.

Let us first show that $a_{1}a_{2}\equiv a_{1}b_{2}\operatorname{mod}n$.
Indeed, $a_{1}a_{2}-a_{1}b_{2}=a_{1}\underbrace{\left(  a_{2}-b_{2}\right)
}_{=nq}=a_{1}nq=n\left(  a_{1}q\right)  $. Hence, there exists an integer $c$
such that $a_{1}a_{2}-a_{1}b_{2}=nc$ (namely, $c=a_{1}q$). In other words,
$n\mid a_{1}a_{2}-a_{1}b_{2}$ (by Definition \ref{def.ent.div.div}). In other
words, $a_{1}a_{2}\equiv a_{1}b_{2}\operatorname{mod}n$ (by Definition
\ref{def.ent.cong}).

Next, let us show that $a_{1}b_{2}\equiv b_{1}b_{2}\operatorname{mod}n$.
Indeed, $a_{1}b_{2}-b_{1}b_{2}=b_{2}\underbrace{\left(  a_{1}-b_{1}\right)
}_{=np}=b_{2}np=n\left(  b_{2}p\right)  $. Hence, there exists an integer $c$
such that $a_{1}b_{2}-b_{1}b_{2}=nc$ (namely, $c=b_{2}p$). In other words,
$n\mid a_{1}b_{2}-b_{1}b_{2}$ (by Definition \ref{def.ent.div.div}). In other
words, $a_{1}b_{2}\equiv b_{1}b_{2}\operatorname{mod}n$ (by Definition
\ref{def.ent.cong}).

From $a_{1}a_{2}\equiv a_{1}b_{2}\operatorname{mod}n$ and $a_{1}b_{2}\equiv
b_{1}b_{2}\operatorname{mod}n$, we now conclude that $a_{1}a_{2}\equiv
b_{1}b_{2}\operatorname{mod}n$ (by Proposition \ref{prop.ent.mod.basics}
\textbf{(c)}, applied to $a=a_{1}a_{2}$, $b=a_{1}b_{2}$ and $c=b_{1}b_{2}$).
This completes the proof of Proposition \ref{prop.ent.mod.basics} \textbf{(d)}.

\textbf{(e)} Let $a,b\in\mathbb{Z}$ satisfy $a\equiv b\operatorname{mod}n$.

We have $a\equiv b\operatorname{mod}n$. In other words, $n\mid a-b$ (by
Definition \ref{def.ent.cong}). From $m\mid n$ and $n\mid a-b$, we obtain
$m\mid a-b$ (by Proposition \ref{prop.ent.div.2} \textbf{(b)}, applied to $m$,
$n$ and $a-b$ instead of $a$, $b$ and $c$). In other words, $a\equiv
b\operatorname{mod}m$ (by Definition \ref{def.ent.cong}). This proves
Proposition \ref{prop.ent.mod.basics} \textbf{(e)}.
\end{proof}

In the above proof, we took care to explicitly cite Definition
\ref{def.ent.div.div} and Definition \ref{def.ent.cong} whenever we used them;
in the following, we will omit references like this.

Proposition \ref{prop.ent.mod.basics} \textbf{(d)} is saying that congruences
modulo $n$ (for a fixed integer $n$) can be added, subtracted and multiplied
together. This does not mean that you can do everything with them that you can
do with equalities. The next exercise shows that dividing congruences and
taking a congruence to the power of another does not generally work:

\begin{exercise}
\label{exe.ent.mod.basics-nope}Let $n,a_{1},a_{2},b_{1},b_{2}\in\mathbb{Z}$
satisfy $a_{1}\equiv b_{1}\operatorname{mod}n$ and $a_{2}\equiv b_{2}%
\operatorname{mod}n$. Then, \textbf{in general}, neither $a_{1}/a_{2}\equiv
b_{1}/b_{2}\operatorname{mod}n$ nor $a_{1}^{a_{2}}\equiv b_{1}^{b_{2}%
}\operatorname{mod}n$ is necessarily true. Of course, this is partly due to
the fact that $a_{1}/a_{2}$, $b_{1}/b_{2}$ and $a_{1}^{a_{2}}$ and
$b_{1}^{b_{2}}$ are not always integers in the first place (and being
congruent modulo $n$ only makes sense for integers, at least for now). But
even when $a_{1}/a_{2}$, $b_{1}/b_{2}$ and $a_{1}^{a_{2}}$ and $b_{1}^{b_{2}}$
are integers, the congruences $a_{1}/a_{2}\equiv b_{1}/b_{2}\operatorname{mod}%
n$ nor $a_{1}^{a_{2}}\equiv b_{1}^{b_{2}}\operatorname{mod}n$ are often false.
Find examples of $n,a_{1},a_{2},b_{1},b_{2}$ such that $a_{1}/a_{2}$,
$b_{1}/b_{2}$ and $a_{1}^{a_{2}}$ and $b_{1}^{b_{2}}$ are integers but the
congruences $a_{1}/a_{2}\equiv b_{1}/b_{2}\operatorname{mod}n$ and
$a_{1}^{a_{2}}\equiv b_{1}^{b_{2}}\operatorname{mod}n$ are false.
\end{exercise}

\begin{fineprint}
\begin{proof}
[Solution to Exercise \ref{exe.ent.mod.basics-nope}.]There are many such
examples. Here is one:%
\[
n=8,\ \ \ \ \ \ \ \ \ \ a_{1}=10,\ \ \ \ \ \ \ \ \ \ a_{2}%
=2,\ \ \ \ \ \ \ \ \ \ b_{1}=10,\ \ \ \ \ \ \ \ \ \ b_{2}=10.
\]
These satisfy $a_{1}\equiv b_{1}\operatorname{mod}n$ and $a_{2}\equiv
b_{2}\operatorname{mod}n$ but neither $a_{1}/a_{2}\equiv b_{1}/b_{2}%
\operatorname{mod}n$ nor $a_{1}^{a_{2}}\equiv b_{1}^{b_{2}}\operatorname{mod}%
n$.

It is much easier to find examples which fail only one of the two congruences
$a_{1}/a_{2}\equiv b_{1}/b_{2}\operatorname{mod}n$ and $a_{1}^{a_{2}}\equiv
b_{1}^{b_{2}}\operatorname{mod}n$.
\end{proof}
\end{fineprint}

However, we can divide a congruence $a\equiv b\operatorname{mod}n$ by a
nonzero integer $d$ when all of $a,b,n$ are divisible by $d$:

\begin{exercise}
\label{exe.ent.mod.basics.2}Let $n,d,a,b\in\mathbb{Z}$, and assume that
$d\neq0$. Assume that $d$ divides each of $a,b,n$, and assume that $a\equiv
b\operatorname{mod}n$. Prove that $a/d\equiv b/d\operatorname{mod}n/d$.
\end{exercise}

\begin{fineprint}
\begin{proof}
[Solution to Exercise \ref{exe.ent.mod.basics.2}.]We have $a\equiv
b\operatorname{mod}n$. In other words, $n\mid a-b$ (by the definition of
congruence). Note that all of $a/d$, $b/d$ and $n/d$ are integers (since $d$
divides each of $a,b,n$). Hence, $\left(  a-b\right)  /d=a/d-b/d$ is an
integer as well. Hence, Exercise \ref{exe.ent.div.acbc} (applied to $n/d$,
$\left(  a-b\right)  /d$ and $d$ instead of $a$, $b$ and $c$) shows that
$n/d\mid\left(  a-b\right)  /d$ holds if and only if $\left(  n/d\right)
d\mid\left(  \left(  a-b\right)  /d\right)  d$. Since $\left(  n/d\right)
d\mid\left(  \left(  a-b\right)  /d\right)  d$ does hold (indeed, this is just
a complicated way to say $n\mid a-b$), we thus conclude that $n/d\mid\left(
a-b\right)  /d$ holds. In other words, $n/d\mid a/d-b/d$ (since $\left(
a-b\right)  /d=a/d-b/d$). In other words, $a/d\equiv b/d\operatorname{mod}n/d$
(by the definition of congruence). This solves Exercise
\ref{exe.ent.mod.basics.2}.
\end{proof}
\end{fineprint}

We can also take a congruence to the $k$-th power when $k\in\mathbb{N}$:

\begin{exercise}
\label{exe.ent.mod.basics.k-power}Let $n,a,b\in\mathbb{Z}$ be such that
$a\equiv b\operatorname{mod}n$. Prove that $a^{k}\equiv b^{k}%
\operatorname{mod}n$ for each $k\in\mathbb{N}$.
\end{exercise}

(Note that the \textquotedblleft$n$\textquotedblright\ is not being taken to
the $k$-th power here.)

\begin{fineprint}
\begin{proof}
[First solution to Exercise \ref{exe.ent.mod.basics.k-power}.]We want to prove
that
\begin{equation}
a^{k}\equiv b^{k}\operatorname{mod}n\ \ \ \ \ \ \ \ \ \ \text{for each }%
k\in\mathbb{N}. \label{sol.ent.mod.basics.k-power.goal}%
\end{equation}
We shall prove this by induction on $k$:

\textit{Induction base:} Proposition \ref{prop.ent.mod.basics} \textbf{(a)}
yields $1\equiv1\operatorname{mod}n$. In view of $a^{0}=1$ and $b^{0}=1$, this
rewrites as $a^{0}\equiv b^{0}\operatorname{mod}n$. In other words,
(\ref{sol.ent.mod.basics.k-power.goal}) holds for $k=0$. This completes the
induction base.

\textit{Induction step:} Let $\ell\in\mathbb{N}$. Assume that
(\ref{sol.ent.mod.basics.k-power.goal}) holds for $k=\ell$. We must prove that
(\ref{sol.ent.mod.basics.k-power.goal}) holds for $k=\ell+1$.

We have assumed that (\ref{sol.ent.mod.basics.k-power.goal}) holds for
$k=\ell$. In other words, we have $a^{\ell}\equiv b^{\ell}\operatorname{mod}%
n$. Also, recall that $a\equiv b\operatorname{mod}n$. Hence,
(\ref{eq.prop.ent.mod.basics.d.3}) (applied to $c=a^{\ell}$ and $d=b^{\ell}$)
yields $aa^{\ell}\equiv bb^{\ell}\operatorname{mod}n$. In other words,
$a^{\ell+1}\equiv b^{\ell+1}\operatorname{mod}n$ (since $aa^{\ell}=a^{\ell+1}$
and $bb^{\ell}=b^{\ell+1}$). In other words,
(\ref{sol.ent.mod.basics.k-power.goal}) holds for $k=\ell+1$. This completes
the induction step. Thus, (\ref{sol.ent.mod.basics.k-power.goal}) is proven by
induction. Therefore, Exercise \ref{exe.ent.mod.basics.k-power} is solved.
\end{proof}

\begin{proof}
[Second solution to Exercise \ref{exe.ent.mod.basics.k-power}.]Recall that%
\begin{equation}
\left(  a-b\right)  \left(  a^{k-1}+a^{k-2}b+a^{k-3}b^{2}+\cdots
+ab^{k-2}+b^{k-1}\right)  =a^{k}-b^{k}
\label{sol.exe.ent.mod.basics.k-power.2nd.gs}%
\end{equation}
for every $k\in\mathbb{N}$. (This is a well-known identity, and it appears
(with $k$ renamed as $n$) as the first half of Exercise 1 on
\href{http://www-users.math.umn.edu/~dgrinber/19s/hw0s.pdf}{homework set \#0}.)

Now, let $k\in\mathbb{N}$. We have assumed that $a\equiv b\operatorname{mod}%
n$. In other words, $n\mid a-b$. In other words, there exists an integer $c$
such that $a-b=nc$. Consider this $c$. Now,
(\ref{sol.exe.ent.mod.basics.k-power.2nd.gs}) yields%
\begin{align*}
a^{k}-b^{k}  &  =\underbrace{\left(  a-b\right)  }_{=nc}\left(  a^{k-1}%
+a^{k-2}b+a^{k-3}b^{2}+\cdots+ab^{k-2}+b^{k-1}\right) \\
&  =nc\left(  a^{k-1}+a^{k-2}b+a^{k-3}b^{2}+\cdots+ab^{k-2}+b^{k-1}\right)  .
\end{align*}
The right hand side of this equality is clearly divisible by $n$. Hence, so is
the left hand side. In other words, $n\mid a^{k}-b^{k}$. In other words,
$a^{k}\equiv b^{k}\operatorname{mod}n$. Hence, Exercise
\ref{exe.ent.mod.basics.k-power} is solved again.
\end{proof}
\end{fineprint}

We can add not just two, but any number of congruences (where
\textquotedblleft number\textquotedblright\ means \textquotedblleft finite
number\textquotedblright):

\begin{exercise}
\label{exe.ent.mod.k-sum}Let $n$ be an integer. Let $S$ be a finite set. For
each $s\in S$, let $a_{s}$ and $b_{s}$ be two integers. Assume that%
\begin{equation}
a_{s}\equiv b_{s}\operatorname{mod}n\ \ \ \ \ \ \ \ \ \ \text{for each }s\in
S. \label{eq.exe.ent.mod.k-sum.ass}%
\end{equation}


\textbf{(a)} Prove that%
\begin{equation}
\sum_{s\in S}a_{s}\equiv\sum_{s\in S}b_{s}\operatorname{mod}n.
\label{eq.exe.ent.mod.k-sum.a}%
\end{equation}


\textbf{(b)} Prove that
\begin{equation}
\prod_{s\in S}a_{s}\equiv\prod_{s\in S}b_{s}\operatorname{mod}n.
\label{eq.exe.ent.mod.k-sum.b}%
\end{equation}


(Keep in mind that if the set $S$ is empty, then $\sum_{s\in S}a_{s}%
=\sum_{s\in S}b_{s}=0$ and $\prod_{s\in S}a_{s}=\prod_{s\in S}b_{s}=1$; this
holds by the definition of empty sums and of empty products.)
\end{exercise}

\begin{fineprint}
\begin{proof}
[Solution to Exercise \ref{exe.ent.mod.k-sum}.]\textbf{(a)} We shall solve
Exercise \ref{exe.ent.mod.k-sum} \textbf{(a)} by induction on $\left\vert
S\right\vert $:

\textit{Induction base:} Exercise \ref{exe.ent.mod.k-sum} \textbf{(a)} holds
whenever $\left\vert S\right\vert =0$\ \ \ \ \footnote{\textit{Proof.} Let
$n$, $S$, $a_{s}$ and $b_{s}$ be as in Exercise \ref{exe.ent.mod.k-sum}, and
assume that $\left\vert S\right\vert =0$. Then, the set $S$ is empty (since
$\left\vert S\right\vert =0$), and thus we have $\sum_{s\in S}a_{s}=\left(
\text{empty sum}\right)  =0$. Similarly, $\sum_{s\in S}b_{s}=0$. Now,
Proposition \ref{prop.ent.mod.basics} \textbf{(a)} yields $0\equiv
0\operatorname{mod}n$. In view of $\sum_{s\in S}a_{s}=0$ and $\sum_{s\in
S}b_{s}=0$, this rewrites as $\sum_{s\in S}a_{s}\equiv\sum_{s\in S}%
b_{s}\operatorname{mod}n$. Thus, Exercise \ref{exe.ent.mod.k-sum} \textbf{(a)}
holds in our case.
\par
So we have shown that Exercise \ref{exe.ent.mod.k-sum} \textbf{(a)} holds
whenever $\left\vert S\right\vert =0$.}. This completes the induction base.

\textit{Induction step:} Fix $k\in\mathbb{N}$. Assume that Exercise
\ref{exe.ent.mod.k-sum} \textbf{(a)} holds whenever $\left\vert S\right\vert
=k$. We must prove that Exercise \ref{exe.ent.mod.k-sum} \textbf{(a)} holds
whenever $\left\vert S\right\vert =k+1$.

We have assumed that Exercise \ref{exe.ent.mod.k-sum} \textbf{(a)} holds
whenever $\left\vert S\right\vert =k$. In other words, the following statement
is true:

\begin{statement}
\textit{Statement 1:} Let $n$, $S$, $a_{s}$ and $b_{s}$ be as in Exercise
\ref{exe.ent.mod.k-sum}. Assume that $\left\vert S\right\vert =k$. Then,
$\sum_{s\in S}a_{s}\equiv\sum_{s\in S}b_{s}\operatorname{mod}n$.
\end{statement}

Now, we must prove that Exercise \ref{exe.ent.mod.k-sum} \textbf{(a)} holds
whenever $\left\vert S\right\vert =k+1$. In other words, we must prove the
following statement:

\begin{statement}
\textit{Statement 2:} Let $n$, $S$, $a_{s}$ and $b_{s}$ be as in Exercise
\ref{exe.ent.mod.k-sum}. Assume that $\left\vert S\right\vert =k+1$. Then,
$\sum_{s\in S}a_{s}\equiv\sum_{s\in S}b_{s}\operatorname{mod}n$.
\end{statement}

[\textit{Proof of Statement 2:} We have $\left\vert S\right\vert =k+1>k\geq0$;
thus, the set $S$ is nonempty. Hence, there exists some $t\in S$. Pick such a
$t$. Thus, $\left\vert S\setminus\left\{  t\right\}  \right\vert =\left\vert
S\right\vert -1=k$ (since $\left\vert S\right\vert =k+1$). Moreover, from
(\ref{eq.exe.ent.mod.k-sum.ass}), we immediately obtain that
\[
a_{s}\equiv b_{s}\operatorname{mod}n\ \ \ \ \ \ \ \ \ \ \text{for each }s\in
S\setminus\left\{  t\right\}
\]
(since each $s\in S\setminus\left\{  t\right\}  $ belongs to $S$). Hence, we
can apply Statement 1 to $S\setminus\left\{  t\right\}  $ instead of $S$. We
thus obtain
\[
\sum_{s\in S\setminus\left\{  t\right\}  }a_{s}\equiv\sum_{s\in S\setminus
\left\{  t\right\}  }b_{s}\operatorname{mod}n.
\]
Also, we have
\[
a_{t}\equiv b_{t}\operatorname{mod}n
\]
(by (\ref{eq.exe.ent.mod.k-sum.ass}), applied to $s=t$). Adding these two
congruences together, we obtain%
\[
\sum_{s\in S\setminus\left\{  t\right\}  }a_{s}+a_{t}\equiv\sum_{s\in
S\setminus\left\{  t\right\}  }b_{s}+b_{t}\operatorname{mod}n.
\]
In view of%
\[
\sum_{s\in S}a_{s}=\sum_{s\in S\setminus\left\{  t\right\}  }a_{s}%
+a_{t}\ \ \ \ \ \ \ \ \ \ \left(
\begin{array}
[c]{c}%
\text{here, we have split off the addend}\\
\text{for }s=t\text{ from the sum}%
\end{array}
\right)
\]
and%
\[
\sum_{s\in S}b_{s}=\sum_{s\in S\setminus\left\{  t\right\}  }b_{s}%
+b_{t}\ \ \ \ \ \ \ \ \ \ \left(
\begin{array}
[c]{c}%
\text{here, we have split off the addend}\\
\text{for }s=t\text{ from the sum}%
\end{array}
\right)  ,
\]
this can be rewritten as%
\[
\sum_{s\in S}a_{s}\equiv\sum_{s\in S}b_{s}\operatorname{mod}n.
\]
This proves Statement 2.]

We have now proven Statement 2; this means that Exercise
\ref{exe.ent.mod.k-sum} \textbf{(a)} holds whenever $\left\vert S\right\vert
=k+1$. This completes the induction step; thus, Exercise
\ref{exe.ent.mod.k-sum} \textbf{(a)} is solved.

\textbf{(b)} The solution to Exercise \ref{exe.ent.mod.k-sum} \textbf{(b)} is
analogous to the one we gave above for Exercise \ref{exe.ent.mod.k-sum}
\textbf{(a)}; the main difference is that we have to replace sums by products
(and $0$ by $1$).
\end{proof}
\end{fineprint}

\begin{exercise}
\label{exe.ent.mod.prod-wrong}Is it true that if $a_{1},a_{2},b_{1}%
,b_{2},n_{1},n_{2}\in\mathbb{Z}$ satisfy $a_{1}\equiv b_{1}\operatorname{mod}%
n_{1}$ and $a_{2}\equiv b_{2}\operatorname{mod}n_{2}$, then $a_{1}a_{2}\equiv
b_{1}b_{2}\operatorname{mod}n_{1}n_{2}$ ?
\end{exercise}

\begin{fineprint}
\begin{proof}
[Solution to Exercise \ref{exe.ent.mod.prod-wrong}.]No, it is not true. For
example, $a_{1}=1$, $a_{2}=1$, $b_{1}=1$, $b_{2}=0$, $n_{1}=0$ and $n_{2}=1$
yield a counterexample.
\end{proof}
\end{fineprint}

\subsection{\label{sect.ent.subst-chain}Chains of congruences}

For this whole Section \ref{sect.ent.subst-chain}, we fix an integer $n$.

Chains of equalities are a fundamental piece of notation used throughout
mathematics. For example, here is a chain of equalities:%
\begin{align*}
&  \left(  ad+bc\right)  ^{2}+\left(  ac-bd\right)  ^{2}\\
&  =\left(  ad\right)  ^{2}+2ad\cdot bc+\left(  bc\right)  ^{2}+\left(
ac\right)  ^{2}-2ac\cdot bd+\left(  bd\right)  ^{2}\\
&  =a^{2}d^{2}+2abcd+b^{2}c^{2}+a^{2}c^{2}-2abcd+b^{2}d^{2}\\
&  =a^{2}c^{2}+a^{2}d^{2}+b^{2}c^{2}+b^{2}d^{2}\\
&  =\left(  a^{2}+b^{2}\right)  \left(  c^{2}+d^{2}\right)
\end{align*}
(where $a,b,c,d$ are arbitrary numbers). This chain proves the equality
(\ref{eq.intro.sum-of-2sq.sum*sum}). But why does it really? If we look
closely at this chain of equalities, we see that it has the form
\textquotedblleft$A=B=C=D=E$\textquotedblright, where $A,B,C,D,E$ are five
numbers (namely, $A=\left(  ad+bc\right)  ^{2}+\left(  ac-bd\right)  ^{2}$ and
$B=\left(  ad\right)  ^{2}+2ad\cdot bc+\left(  bc\right)  ^{2}+\left(
ac\right)  ^{2}-2ac\cdot bd+\left(  bd\right)  ^{2}$ and so on). This kind of
statement is called a \textquotedblleft chain of equalities\textquotedblright,
and, a priori, it simply means that any two \textbf{adjacent} numbers in this
chain are equal: $A=B$ and $B=C$ and $C=D$ and $D=E$. Without as much as
noticing it, we have concluded that \textbf{any} two numbers in this chain are
equal; thus, in particular, $A=E$, which is precisely the equality
(\ref{eq.intro.sum-of-2sq.sum*sum}) we wanted to prove.

That this kind of \textquotedblleft chaining\textquotedblright\ is possible is
one of the most basic facts in mathematics. Let us define a chain of
equalities formally:

\begin{definition}
If $a_{1},a_{2},\ldots,a_{k}$ are $k$ objects\footnotemark, then the statement
\textquotedblleft$a_{1}=a_{2}=\cdots=a_{k}$\textquotedblright\ shall mean
that
\[
a_{i}=a_{i+1}\text{ holds for each }i\in\left\{  1,2,\ldots,k-1\right\}  .
\]
(In other words, it shall mean that $a_{1}=a_{2}$ and $a_{2}=a_{3}$ and
$a_{3}=a_{4}$ and $\cdots$ and $a_{k-1}=a_{k}$. This is vacuously true when
$k\leq1$. If $k=2$, then it simply means that $a_{1}=a_{2}$.)

Such a statement will be called a \textit{chain of equalities}.
\end{definition}

\footnotetext{\textquotedblleft Objects\textquotedblright\ can be numbers,
sets, tuples or any other well-defined things in mathematics.}

\begin{proposition}
\label{prop.mod.chain-eq}Let $a_{1},a_{2},\ldots,a_{k}$ be $k$ objects such
that $a_{1}=a_{2}=\cdots=a_{k}$. Let $u$ and $v$ be two elements of $\left\{
1,2,\ldots,k\right\}  $. Then, $a_{u}=a_{v}$.
\end{proposition}

So we have defined a chain of equalities to be true if and only if any two
adjacent terms in this chain are equal (i.e., if \textquotedblleft each
equality sign in the chain is satisfied\textquotedblright). Proposition
\ref{prop.mod.chain-eq} shows that in such a chain, \textbf{any two} terms are
equal. This is intuitively rather clear, but can also be formally proven by
induction using the basic properties of equality
(transitivity\footnote{\textit{Transitivity of equality} says that if $a,b,c$
are three objects satisfying $a=b$ and $b=c$, then $a=c$.},
reflexivity\footnote{\textit{Reflexivity of equality} says that every object
$a$ satisfies $a=a$.} and symmetry\footnote{\textit{Symmetry of equality} says
that if $a,b$ are two objects satisfying $a=b$, then $b=a$.}).

But our goal is to understand basic number theory, not to scrutinize the
foundations of mathematics. So let us recall that we have fixed an integer
$n$, and consider congruences modulo $n$. We claim that these can be chained
just as equalities:

\begin{definition}
If $a_{1},a_{2},\ldots,a_{k}$ are $k$ integers, then the statement
\textquotedblleft$a_{1}\equiv a_{2}\equiv\cdots\equiv a_{k}\operatorname{mod}%
n$\textquotedblright\ shall mean that
\[
a_{i}\equiv a_{i+1}\operatorname{mod}n\text{ holds for each }i\in\left\{
1,2,\ldots,k-1\right\}  .
\]
(In other words, it shall mean that $a_{1}\equiv a_{2}\operatorname{mod}n$ and
$a_{2}\equiv a_{3}\operatorname{mod}n$ and $a_{3}\equiv a_{4}%
\operatorname{mod}n$ and $\cdots$ and $a_{k-1}\equiv a_{k}\operatorname{mod}%
n$. This is vacuously true when $k\leq1$. If $k=2$, then it simply means that
$a_{1}\equiv a_{2}\operatorname{mod}n$.)

Such a statement will be called a \textit{chain of congruences modulo }$n$.
\end{definition}

\begin{proposition}
\label{prop.mod.chain}Let $a_{1},a_{2},\ldots,a_{k}$ be $k$ integers such that
$a_{1}\equiv a_{2}\equiv\cdots\equiv a_{k}\operatorname{mod}n$. Let $u$ and
$v$ be two elements of $\left\{  1,2,\ldots,k\right\}  $. Then, $a_{u}\equiv
a_{v}\operatorname{mod}n$.
\end{proposition}

Proposition \ref{prop.mod.chain} shows that any two terms in a chain of
congruences modulo $n$ must be congruent to each other modulo $n$. Again, this
can be formally proven by induction; see \cite[proof of Proposition
2.16]{detnotes}. The ingredients of the proof are basic properties of
congruence modulo $n$: transitivity, reflexivity and symmetry. These are fancy
names for parts \textbf{(b)}, \textbf{(a)} and \textbf{(c)} of Proposition
\ref{prop.ent.mod.basics}.

We will use Proposition \ref{prop.mod.chain} tacitly (just as you would use
Proposition \ref{prop.mod.chain-eq}): i.e., every time we prove a chain of
congruences like $a_{1}\equiv a_{2}\equiv\cdots\equiv a_{k}\operatorname{mod}%
n$, we assume that the reader will automatically conclude that any two of its
terms are congruent to each other modulo $n$ (and will remember this
conclusion). For instance, if we show that $1\equiv4\equiv34\equiv
334\equiv304\operatorname{mod}3$, then we automatically get the congruences
$1\equiv304\operatorname{mod}3$ and $334\equiv1\operatorname{mod}3$ and
$4\equiv334\operatorname{mod}3$ and several others out of this chain.

Chains of congruences can also include equality signs. For example, if
$a,b,c,d$ are integers, then \textquotedblleft$a\equiv b=c\equiv
d\operatorname{mod}n$\textquotedblright\ means that $a\equiv
b\operatorname{mod}n$ and $b=c$ and $c\equiv d\operatorname{mod}n$. Such a
chain is still a chain of congruences, because $b=c$ implies $b\equiv
c\operatorname{mod}n$ (by Proposition \ref{prop.ent.mod.basics} \textbf{(a)}).

Just as there are chains of equalities and chains of congruences, there are
chains of divisibilities:

\begin{definition}
If $a_{1},a_{2},\ldots,a_{k}$ are $k$ integers, then the statement
\textquotedblleft$a_{1}\mid a_{2}\mid\cdots\mid a_{k}$\textquotedblright%
\ shall mean that
\[
a_{i}\mid a_{i+1}\text{ holds for each }i\in\left\{  1,2,\ldots,k-1\right\}
.
\]
(In other words, it shall mean that $a_{1}\mid a_{2}$ and $a_{2}\mid a_{3}$
and $a_{3}\mid a_{4}$ and $\cdots$ and $a_{k-1}\mid a_{k}$. This is vacuously
true when $k\leq1$. If $k=2$, then it simply means that $a_{1}\mid a_{2}$.)

Such a statement will be called a \textit{chain of divisibilities}.
\end{definition}

\begin{proposition}
\label{prop.ent.div.chain}Let $a_{1},a_{2},\ldots,a_{k}$ be $k$ integers such
that $a_{1}\mid a_{2}\mid\cdots\mid a_{k}$. Let $u$ and $v$ be two elements of
$\left\{  1,2,\ldots,k\right\}  $ such that $u\leq v$. Then, $a_{u}\mid a_{v}$.
\end{proposition}

Note that we had to require $u\leq v$ in this proposition, unlike the
analogous propositions for chains of equalities and chains of congruences,
because there is no \textquotedblleft symmetry of
divisibility\textquotedblright\ (i.e., if $a\mid b$, then we don't generally
have $b\mid a$). The proof of Proposition \ref{prop.ent.div.chain} relies on
the reflexivity of divisibility (Proposition \ref{prop.ent.div.2}
\textbf{(a)}) and on the transitivity of divisibility (Proposition
\ref{prop.ent.div.2} \textbf{(b)}).

Again, chains of divisibilities can include equality signs.

\subsection{\label{sect.ent.subst-mod}Substitutivity for congruences}

In\ Section \ref{sect.ent.subst-chain}, we have learnt that congruences modulo
an integer $n$ can be chained together like equalities. A further important
feature of congruences is the principle of \textit{substitutivity for
congruences}. This is yet another way in which congruences behave like
equalities. We are not going to state it fully formally (as it is a
meta-mathematical principle), but merely explain its meaning. Later on, once
we understand what the rings $\mathbb{Z}/n$ (for integer $n$) are, we will no
longer need this principle, since it will just boil down to \textquotedblleft
equal things can be substituted for one another\textquotedblright\ (the whole
point of $\mathbb{Z}/n$ is to \textquotedblleft make congruent numbers
equal\textquotedblright); but for now, we cannot treat \textquotedblleft
congruent modulo $n$\textquotedblright\ as \textquotedblleft
equal\textquotedblright, so we have to state it.

You are probably used to making computations like these:%
\begin{align*}
\underbrace{\left(  a+b\right)  ^{2}}_{=a^{2}+2ab+b^{2}}+\underbrace{\left(
a-b\right)  ^{2}}_{=a^{2}-2ab+b^{2}}  &  =\left(  a^{2}+2ab+b^{2}\right)
+\left(  a^{2}-2ab+b^{2}\right) \\
&  =\underbrace{a^{2}+a^{2}}_{=2a^{2}}+\underbrace{b^{2}+b^{2}}_{=2b^{2}%
}=2a^{2}+2b^{2}%
\end{align*}
(for any two numbers $a$ and $b$). What is going on in these underbraces (like
\textquotedblleft$\underbrace{\left(  a+b\right)  ^{2}}_{=a^{2}+2ab+b^{2}}%
$\textquotedblright)? Something pretty simple is going on: You are replacing a
number (in this case, $\left(  a+b\right)  ^{2}$) by an equal number (in this
case, $a^{2}+2ab+b^{2}$). This relies on a fundamental principle of
mathematics (called the \textit{principle of substitutivity for equalities}),
which says that an object in an expression can indeed be replaced by any
object equal to it (without changing the value of the expression). (This is
also known as \textit{Leibniz's equality law}.) To be precise, we are using
this principle twice in some of our equality signs above, since we are making
several replacements at the same time; but this is fine (we can just do the
replacement one by one instead).

We would like to have a similar principle for congruences modulo $n$: We would
like to be able to replace any integer by an integer congruent to it modulo
$n$. For example, we would like to be able to say that if seven integers
$a,a^{\prime},b,b^{\prime},c,c^{\prime},n$ satisfy $a\equiv a^{\prime
}\operatorname{mod}n$ and $b\equiv b^{\prime}\operatorname{mod}n$ and $c\equiv
c^{\prime}\operatorname{mod}n$, then%
\[
\underbrace{b}_{\equiv b^{\prime}\operatorname{mod}n}\ \ \underbrace{c}%
_{\equiv c^{\prime}\operatorname{mod}n}+\underbrace{c}_{\equiv c^{\prime
}\operatorname{mod}n}\ \ \underbrace{a}_{\equiv a^{\prime}\operatorname{mod}%
n}+\underbrace{a}_{\equiv a^{\prime}\operatorname{mod}n}\ \ \underbrace{b}%
_{\equiv b^{\prime}\operatorname{mod}n}\equiv b^{\prime}c^{\prime}+c^{\prime
}a^{\prime}+a^{\prime}b^{\prime}\operatorname{mod}n.
\]


We have to be careful with this: For example, we run into troubles if division
is involved in our expressions. For example, we have $6\equiv
9\operatorname{mod}3$, but we do not have $\underbrace{6}_{\equiv
9\operatorname{mod}3}/3\equiv9/3\operatorname{mod}3$. Similarly,
exponentiation can be problematic. So we need to state the principle we are
using here in clearer terms, so that we know what we can do.

For this whole Section \ref{sect.ent.subst-mod}, we fix an integer $n$.

The \textit{principle of substitutivity for equalities} says the following:

\begin{statement}
\textit{Principle of substitutivity for equalities (PSE):} If two objects $x$
and $x^{\prime}$ are equal, and if we have any expression $A$ that involves
the object $x$, then we can replace this $x$ (or, more precisely, any
arbitrary appearance of $x$ in $A$) in $A$ by $x^{\prime}$; the resulting
expression $A^{\prime}$ will be equal to $A$.
\end{statement}

Here are two examples of how this principle can be used:

\begin{itemize}
\item If $a,b,c,d,e,c^{\prime}$ are numbers such that $c=c^{\prime}$, then the
PSE says that we can replace $c$ by $c^{\prime}$ in the expression $a\left(
b-\left(  c+d\right)  e\right)  $, and the resulting expression $a\left(
b-\left(  c^{\prime}+d\right)  e\right)  $ will be equal to $a\left(
b-\left(  c+d\right)  e\right)  $; that is, we have%
\begin{equation}
a\left(  b-\left(  c+d\right)  e\right)  =a\left(  b-\left(  c^{\prime
}+d\right)  e\right)  . \label{eq.mod.substitutivity-nums.1}%
\end{equation}


\item If $a,b,c,a^{\prime}$ are numbers such that $a=a^{\prime}$, then
\begin{equation}
\left(  a-b\right)  \left(  a+b\right)  =\left(  a^{\prime}-b\right)  \left(
a+b\right)  , \label{eq.mod.substitutivity-nums.2}%
\end{equation}
because the PSE allows us to replace the first $a$ appearing in the expression
$\left(  a-b\right)  \left(  a+b\right)  $ by an $a^{\prime}$. (We can also
replace the second $a$ by $a^{\prime}$, of course.)
\end{itemize}

More generally, we can make several such replacements at the same time.

The PSE is one of the headstones of mathematical logic; it is the essence of
what it means for two objects to be equal.

The \textit{principle of substitutivity for congruences} is similar, but far
less fundamental; it says the following:

\begin{statement}
\textit{Principle of substitutivity for congruences (PSC):} If two numbers $x$
and $x^{\prime}$ are congruent to each other modulo $n$ (that is, $x\equiv
x^{\prime}\operatorname{mod}n$), and if we have any expression $A$ that
involves only integers, addition, subtraction and multiplication, and involves
the object $x$, then we can replace this $x$ (or, more precisely, any
arbitrary appearance of $x$ in $A$) in $A$ by $x^{\prime}$; the resulting
expression $A^{\prime}$ will be congruent to $A$ modulo $n$.
\end{statement}

This principle is less general than the PSE, since it only applies to
expressions that are built from integers and certain operations (note that
division is not one of these operations). But it still lets us prove analogues
of our above examples (\ref{eq.mod.substitutivity-nums.1}) and
(\ref{eq.mod.substitutivity-nums.2}):

\begin{itemize}
\item If $a,b,c,d,e,c^{\prime}$ are integers such that $c\equiv c^{\prime
}\operatorname{mod}n$, then the PSC says that we can replace $c$ by
$c^{\prime}$ in the expression $a\left(  b-\left(  c+d\right)  e\right)  $,
and the resulting expression $a\left(  b-\left(  c^{\prime}+d\right)
e\right)  $ will be congruent to $a\left(  b-\left(  c+d\right)  e\right)  $
modulo $n$; that is, we have%
\begin{equation}
a\left(  b-\left(  c+d\right)  e\right)  \equiv a\left(  b-\left(  c^{\prime
}+d\right)  e\right)  \operatorname{mod}n.
\label{eq.mod.substitutivity-congs.1}%
\end{equation}


\item If $a,b,c,a^{\prime}$ are integers such that $a\equiv a^{\prime
}\operatorname{mod}n$, then
\begin{equation}
\left(  a-b\right)  \left(  a+b\right)  \equiv\left(  a^{\prime}-b\right)
\left(  a+b\right)  \operatorname{mod}n, \label{eq.mod.substitutivity-congs.2}%
\end{equation}
because the PSC allows us to replace the first $a$ appearing in the expression
$\left(  a-b\right)  \left(  a+b\right)  $ by an $a^{\prime}$. (We can also
replace the second $a$ by $a^{\prime}$, of course.)
\end{itemize}

We shall not prove the PSC, since we have not formalized it (after all, we
have not defined what an \textquotedblleft expression\textquotedblright\ is).
But we shall prove the specific congruences
(\ref{eq.mod.substitutivity-congs.1}) and (\ref{eq.mod.substitutivity-congs.2}%
) using Proposition \ref{prop.ent.mod.basics}; the way in which we prove these
congruences is symptomatic: Every congruence obtained from the PSC can be
proven in a manner like these. Thus, the proofs of
(\ref{eq.mod.substitutivity-congs.1}) and (\ref{eq.mod.substitutivity-congs.2}%
) given below can serve as templates which can easily be adapted to any other
situation in which an application of the PSC needs to be justified.

\begin{proof}
[Proof of (\ref{eq.mod.substitutivity-congs.1}).]Let $n$ be any integer, and
let $a,b,c,d,e,c^{\prime}$ be integers such that $c\equiv c^{\prime
}\operatorname{mod}n$.

Adding the congruence\footnote{Proposition \ref{prop.ent.mod.basics}
\textbf{(d)} shows that we can add, subtract and multiply congruences modulo
$n$ at will. We are using this freedom here and will use it many times below.}
$c\equiv c^{\prime}\operatorname{mod}n$ with the congruence $d\equiv
d\operatorname{mod}n$ (which follows from Proposition
\ref{prop.ent.mod.basics} \textbf{(a)}), we obtain $c+d\equiv c^{\prime
}+d\operatorname{mod}n$. Multiplying this congruence with the congruence
$e\equiv e\operatorname{mod}n$ (which follows from Proposition
\ref{prop.ent.mod.basics} \textbf{(a)}), we obtain $\left(  c+d\right)
e\equiv\left(  c^{\prime}+d\right)  e\operatorname{mod}n$. Subtracting this
congruence from the congruence $b\equiv b\operatorname{mod}n$ (which, again,
follows from Proposition \ref{prop.ent.mod.basics} \textbf{(a)}), we obtain
$b-\left(  c+d\right)  e\equiv b-\left(  c^{\prime}+d\right)
e\operatorname{mod}n$. Multiplying the congruence $a\equiv a\operatorname{mod}%
n$ (which follows from Proposition \ref{prop.ent.mod.basics} \textbf{(a)})
with this congruence, we obtain $a\left(  b-\left(  c+d\right)  e\right)
\equiv a\left(  b-\left(  c^{\prime}+d\right)  e\right)  \operatorname{mod}n$.
This proves (\ref{eq.mod.substitutivity-congs.1}).
\end{proof}

\begin{proof}
[Proof of (\ref{eq.mod.substitutivity-congs.2}).]Let $n$ be any integer, and
let $a,b,c,a^{\prime}$ be integers such that $a\equiv a^{\prime}%
\operatorname{mod}n$.

Subtracting the congruence $b\equiv b\operatorname{mod}n$ (which follows from
Proposition \ref{prop.ent.mod.basics} \textbf{(a)}) from the congruence
$a\equiv a^{\prime}\operatorname{mod}n$, we obtain $a-b\equiv a^{\prime
}-b\operatorname{mod}n$. Multiplying this congruence with the congruence
$a+b\equiv a+b\operatorname{mod}n$ (which follows from Proposition
\ref{prop.ent.mod.basics} \textbf{(a)}), we obtain $\left(  a-b\right)
\left(  a+b\right)  \equiv\left(  a^{\prime}-b\right)  \left(  a+b\right)
\operatorname{mod}n$. This proves (\ref{eq.mod.substitutivity-congs.2}).
\end{proof}

As we said, these two proofs are exemplary: Any congruence obtained from the
PSC can be proven in such a way (starting with the congruence $x\equiv
x^{\prime}\operatorname{mod}n$, and then \textquotedblleft
wrapping\textquotedblright\ it up in the expression $A$ by repeatedly adding,
multiplying and subtracting congruences that follow from Proposition
\ref{prop.ent.mod.basics} \textbf{(a)}).

When we apply the PSC, we shall use underbraces to point out which integers we
are replacing. For example, when deriving (\ref{eq.mod.substitutivity-congs.1}%
) from this principle, we shall write%
\[
a\left(  b-\left(  \underbrace{c}_{\equiv c^{\prime}\operatorname{mod}%
n}+d\right)  e\right)  \equiv a\left(  b-\left(  c^{\prime}+d\right)
e\right)  \operatorname{mod}n,
\]
in order to stress that we are replacing $c$ by $c^{\prime}$. Likewise, when
deriving (\ref{eq.mod.substitutivity-congs.2}) from the PSC, we shall write%
\[
\left(  \underbrace{a}_{\equiv a^{\prime}\operatorname{mod}n}-b\right)
\left(  a+b\right)  \equiv\left(  a^{\prime}-b\right)  \left(  a+b\right)
\operatorname{mod}n,
\]
in order to stress that we are replacing the first $a$ (but not the second
$a$) by $a^{\prime}$.

The PSC allows us to replace a \textbf{single} integer $x$ appearing in an
expression by another integer $x^{\prime}$ that is congruent to $x$ modulo
$n$. Applying this principle many times, we thus conclude that we can also
replace \textbf{several} integers at the same time (because we can get to the
same result by performing these replacements one at a time, and Proposition
\ref{prop.mod.chain} shows that the final result will be congruent to the
original result). For example, if seven integers $a,a^{\prime},b,b^{\prime
},c,c^{\prime},n$ satisfy $a\equiv a^{\prime}\operatorname{mod}n$ and $b\equiv
b^{\prime}\operatorname{mod}n$ and $c\equiv c^{\prime}\operatorname{mod}n$,
then%
\begin{equation}
bc+ca+ab\equiv b^{\prime}c^{\prime}+c^{\prime}a^{\prime}+a^{\prime}b^{\prime
}\operatorname{mod}n, \label{eq.mod.substitutivity-congs.3}%
\end{equation}
because we can replace all the six integers $b,c,c,a,a,b$ in the expression
$bc+ca+ab$ (listed in the order of their appearance in this expression) by
$b^{\prime},c^{\prime},c^{\prime},a^{\prime},a^{\prime},b^{\prime}$,
respectively. If we want to derive this from the PSC, then we must perform the
replacements one at a time, e.g., as follows:%
\begin{align*}
\underbrace{b}_{\equiv b^{\prime}\operatorname{mod}n}c+ca+ab  &  \equiv
b^{\prime}\underbrace{c}_{\equiv c^{\prime}\operatorname{mod}n}+ca+ab\equiv
b^{\prime}c^{\prime}+\underbrace{c}_{\equiv c^{\prime}\operatorname{mod}%
n}a+ab\\
&  \equiv b^{\prime}c^{\prime}+c^{\prime}\underbrace{a}_{\equiv a^{\prime
}\operatorname{mod}n}+ab\equiv b^{\prime}c^{\prime}+c^{\prime}a^{\prime
}+\underbrace{a}_{\equiv a^{\prime}\operatorname{mod}n}b\\
&  \equiv b^{\prime}c^{\prime}+c^{\prime}a^{\prime}+a^{\prime}\underbrace{b}%
_{\equiv b^{\prime}\operatorname{mod}n}\equiv b^{\prime}c^{\prime}+c^{\prime
}a^{\prime}+a^{\prime}b^{\prime}\operatorname{mod}n.
\end{align*}
Of course, we shall always just show the replacements as a single step:%
\[
\underbrace{b}_{\equiv b^{\prime}\operatorname{mod}n}\ \ \underbrace{c}%
_{\equiv c^{\prime}\operatorname{mod}n}+\underbrace{c}_{\equiv c^{\prime
}\operatorname{mod}n}\ \ \underbrace{a}_{\equiv a^{\prime}\operatorname{mod}%
n}+\underbrace{a}_{\equiv a^{\prime}\operatorname{mod}n}\ \ \underbrace{b}%
_{\equiv b^{\prime}\operatorname{mod}n}\equiv b^{\prime}c^{\prime}+c^{\prime
}a^{\prime}+a^{\prime}b^{\prime}\operatorname{mod}n.
\]


The PSC can be extended: The expression $A$ can be allowed to involve not only
integers, addition, subtraction, multiplication and $x$, but also $k$-th
powers for $k\in\mathbb{N}$ (as long as $k$ remains unchanged in our
replacement) as well as finite sums and products (as long as the bounds of the
sums and products are unchanged). This follows from Exercise
\ref{exe.ent.mod.basics.k-power} and Exercise \ref{exe.ent.mod.k-sum}.

\begin{exercise}
\label{exe.ent.mod.substitutivity.7div}Let $n\in\mathbb{N}$. Show that
$7\mid3^{2n+1}+2^{n+2}$.
\end{exercise}

\begin{fineprint}
\begin{proof}
[Solution to Exercise \ref{exe.ent.mod.substitutivity.7div}.]We have
$3^{2n+1}=\left(  \underbrace{3^{2}}_{=9\equiv2\operatorname{mod}7}\right)
^{n}\cdot3\equiv2^{n}\cdot3\operatorname{mod}7$. (This follows from the PSC,
in its extended form that allows $k$-th powers in the expression $A$.
Alternatively, you can argue by hand as follows: We have $3^{2}=9\equiv
2\operatorname{mod}7$. Thus, Exercise \ref{exe.ent.mod.basics.k-power}
(applied to $7$, $3^{2}$, $2$ and $n$ instead of $n$, $a$, $b$ and $k$) yields
$\left(  3^{2}\right)  ^{n}\equiv2^{n}\operatorname{mod}7$. Multiplying this
congruence by the obvious congruence $3\equiv3\operatorname{mod}7$, we obtain
$\left(  3^{2}\right)  ^{n}\cdot3\equiv2^{n}\cdot3\operatorname{mod}7$. Thus,
$3^{2n+1}=\left(  3^{2}\right)  ^{n}\cdot3\equiv2^{n}\cdot3\operatorname{mod}%
7$.)

Hence, again using the PSC, we obtain%
\[
\underbrace{3^{2n+1}}_{\equiv2^{n}\cdot3\operatorname{mod}7}%
+\underbrace{2^{n+2}}_{=2^{n}\cdot2^{2}=2^{n}\cdot4}\equiv2^{n}\cdot
3+2^{n}\cdot4=2^{n}\cdot\underbrace{\left(  3+4\right)  }_{=7}=2^{n}%
\cdot7\equiv0\operatorname{mod}7
\]
(since $2^{n}\cdot7$ is clearly divisible by $7$). In other words,
$7\mid3^{2n+1}+2^{n+2}$.

[\textit{Remark:} Here is a sketch of a different solution: If we set
$a_{n}=3^{2n+1}+2^{n+2}$ for each $n\in\mathbb{N}$, then we must prove that
$7\mid a_{n}$ for all $n\in\mathbb{N}$. But a straightforward computation
reveals that%
\begin{equation}
a_{n}=11a_{n-1}-18a_{n-2}\ \ \ \ \ \ \ \ \ \ \text{for each }n\geq2.
\label{sol.ent.mod.substitutivity.7div.rec}%
\end{equation}
Thus, once we check that $7\mid a_{0}$ and $7\mid a_{1}$, and then a
straightforward strong induction on $n$ shows that $7\mid a_{n}$ for all
$n\in\mathbb{N}$, which is exactly the claim of Exercise
\ref{exe.ent.mod.substitutivity.7div}. Of course, \textbf{finding} the
relation (\ref{sol.ent.mod.substitutivity.7div.rec}) was the main trick in
this solution; it becomes somewhat natural once you know the theory of linear
recurrences (such as the Fibonacci sequence).]
\end{proof}
\end{fineprint}

\begin{center}
\textbf{2019-02-01 lecture}
\end{center}

\subsection{\label{sect.ent.quorem}Division with remainder}

The following fact you likely remember from high school:

\begin{theorem}
\label{thm.ent.quorem.full}Let $n$ be a positive integer. Let $u\in\mathbb{Z}%
$. Then, there exists a unique pair $\left(  q,r\right)  \in\mathbb{Z}%
\times\left\{  0,1,\ldots,n-1\right\}  $ such that $u=qn+r$.
\end{theorem}

Before we prove this theorem, let us introduce the notations that it justifies:

\begin{definition}
\label{def.ent.quorem}Let $n$ be a positive integer. Let $u\in\mathbb{Z}$.
Theorem \ref{thm.ent.quorem.full} shows that there exists a unique pair
$\left(  q,r\right)  \in\mathbb{Z}\times\left\{  0,1,\ldots,n-1\right\}  $
such that $u=qn+r$. Consider this pair.

\textbf{(a)} We denote the integer $q$ by $u//n$, and refer to it as the
\textit{quotient of the division of }$u$ \textit{by }$n$.

\textbf{(b)} We denote the integer $r$ by $u\%n$, and refer to it as the
\textit{remainder of the division of }$u$ \textit{by }$n$.
\end{definition}

The words \textquotedblleft quotient\textquotedblright\ and \textquotedblleft
remainder\textquotedblright\ are standard, but the notations \textquotedblleft%
$u//n$\textquotedblright\ and \textquotedblleft$u\%n$\textquotedblright\ are
not (I have taken them from the Python programming language); be prepared to
see other notations in the literature (e.g., the notations \textquotedblleft%
$\operatorname*{quo}\left(  u,n\right)  $\textquotedblright\ and
\textquotedblleft$\operatorname*{rem}\left(  u,n\right)  $\textquotedblright%
\ for $u//n$ and $u\%n$, respectively).

\begin{example}
\textbf{(a)} We have $14//3=4$ and $14\%3=2$, because $\left(  4,2\right)  $
is the unique pair $\left(  q,r\right)  \in\mathbb{Z}\times\left\{
0,1,2\right\}  $ satisfying $14=q\cdot3+r$.

\textbf{(b)} We have $18//3=6$ and $18\%3=0$, because $\left(  6,0\right)  $
is the unique pair $\left(  q,r\right)  \in\mathbb{Z}\times\left\{
0,1,2\right\}  $ satisfying $18=q\cdot3+r$.

\textbf{(c)} We have $\left(  -2\right)  //3=-1$ and $\left(  -2\right)
\%3=1$, because $\left(  -1,1\right)  $ is the unique pair $\left(
q,r\right)  \in\mathbb{Z}\times\left\{  0,1,2\right\}  $ satisfying
$-2=q\cdot3+r$.

\textbf{(d)} For each $u\in\mathbb{Z}$, we have $u//1=u$ and $u\%1=0$, because
$\left(  u,0\right)  $ is the unique pair $\left(  q,r\right)  \in
\mathbb{Z}\times\left\{  0\right\}  $ satisfying $u=q\cdot1+r$.
\end{example}

But we have gotten ahead of ourselves: We need to prove Theorem
\ref{thm.ent.quorem.full} before we can use the notations \textquotedblleft%
$u//n$\textquotedblright\ and \textquotedblleft$u\%n$\textquotedblright.

Let us split Theorem \ref{thm.ent.quorem.full} into two parts: existence and uniqueness:

\begin{lemma}
\label{lem.ent.quorem.exist}Let $n$ be a positive integer. Let $u\in
\mathbb{Z}$. Then, there exists \textbf{at least one} pair $\left(
q,r\right)  \in\mathbb{Z}\times\left\{  0,1,\ldots,n-1\right\}  $ such that
$u=qn+r$.
\end{lemma}

\begin{lemma}
\label{lem.ent.quorem.unique}Let $n$ be a positive integer. Let $u\in
\mathbb{Z}$. Then, there exists \textbf{at most one} pair $\left(  q,r\right)
\in\mathbb{Z}\times\left\{  0,1,\ldots,n-1\right\}  $ such that $u=qn+r$.
\end{lemma}

We begin by proving Lemma \ref{lem.ent.quorem.unique} (which is the easier one):

\begin{proof}
[Proof of Lemma \ref{lem.ent.quorem.unique}.]Let $\left(  q_{1},r_{1}\right)
$ and $\left(  q_{2},r_{2}\right)  $ be two pairs $\left(  q,r\right)
\in\mathbb{Z}\times\left\{  0,1,\ldots,n-1\right\}  $ such that $u=qn+r$. We
shall show that $\left(  q_{1},r_{1}\right)  =\left(  q_{2},r_{2}\right)  $.

We know that $\left(  q_{1},r_{1}\right)  $ is a pair $\left(  q,r\right)
\in\mathbb{Z}\times\left\{  0,1,\ldots,n-1\right\}  $ such that $u=qn+r$. In
other words, $\left(  q_{1},r_{1}\right)  \in\mathbb{Z}\times\left\{
0,1,\ldots,n-1\right\}  $ and $u=q_{1}n+r_{1}$. Similarly, $\left(
q_{2},r_{2}\right)  \in\mathbb{Z}\times\left\{  0,1,\ldots,n-1\right\}  $ and
$u=q_{2}n+r_{2}$.

From $\left(  q_{1},r_{1}\right)  \in\mathbb{Z}\times\left\{  0,1,\ldots
,n-1\right\}  $, we obtain $q_{1}\in\mathbb{Z}$ and $r_{1}\in\left\{
0,1,\ldots,n-1\right\}  $. Similarly, $q_{2}\in\mathbb{Z}$ and $r_{2}%
\in\left\{  0,1,\ldots,n-1\right\}  $. Thus, in particular, $q_{1},q_{2}%
,r_{1},r_{2}$ are integers.

From $r_{1}\in\left\{  0,1,\ldots,n-1\right\}  $ and $r_{2}\in\left\{
0,1,\ldots,n-1\right\}  $, we can easily derive%
\begin{equation}
\left\vert r_{2}-r_{1}\right\vert \leq n-1.
\label{pf.lem.ent.quorem.unique.ineq}%
\end{equation}


\begin{fineprint}
[\textit{Proof of (\ref{pf.lem.ent.quorem.unique.ineq}):} Intuitively, this
should be clear: Both $r_{1}$ and $r_{2}$ belong to the integer interval
$\left\{  0,1,\ldots,n-1\right\}  $, and thus the unsigned distance between
$r_{1}$ and $r_{2}$ is at most $n-1$ (with the worst case being when $r_{1}$
and $r_{2}$ are at opposite ends of this interval).

Here is a formal restatement of this argument: We have $r_{1}\in\left\{
0,1,\ldots,n-1\right\}  $, thus $r_{1}\geq0$. Also, $r_{2}\in\left\{
0,1,\ldots,n-1\right\}  $, hence $r_{2}\leq n-1$. Hence, $\underbrace{r_{2}%
}_{\leq n-1}-\underbrace{r_{1}}_{\geq0}\leq\left(  n-1\right)  -0=n-1$.
Similarly, $r_{1}-r_{2}\leq n-1$. But recall that $\left\vert x\right\vert
\in\left\{  x,-x\right\}  $ for each $x\in\mathbb{Z}$. Applying this to
$x=r_{2}-r_{1}$, we obtain
\[
\left\vert r_{2}-r_{1}\right\vert \in\left\{  r_{2}-r_{1},\underbrace{-\left(
r_{2}-r_{1}\right)  }_{=r_{1}-r_{2}}\right\}  =\left\{  r_{2}-r_{1}%
,r_{1}-r_{2}\right\}  .
\]
In other words, $\left\vert r_{2}-r_{1}\right\vert $ is one of the two numbers
$r_{2}-r_{1}$ and $r_{1}-r_{2}$. Since both of these numbers $r_{2}-r_{1}$ and
$r_{1}-r_{2}$ are $\leq n-1$ (as we have just shown), we thus conclude that
$\left\vert r_{2}-r_{1}\right\vert \leq n-1$. This proves
(\ref{pf.lem.ent.quorem.unique.ineq}).]
\end{fineprint}

We have $q_{1}n+r_{1}=u=q_{2}n+r_{2}$, thus $q_{1}n-q_{2}n=r_{2}-r_{1}$.
Hence,%
\begin{equation}
r_{2}-r_{1}=q_{1}n-q_{2}n=\left(  q_{1}-q_{2}\right)  n.
\label{pf.lem.ent.quorem.unique.1}%
\end{equation}


Assume (for the sake of contradiction) that $q_{1}\neq q_{2}$. Thus,
$q_{1}-q_{2}\neq0$, so that $\left\vert q_{1}-q_{2}\right\vert >0$ and
therefore $\left\vert q_{1}-q_{2}\right\vert \geq1$ (since $\left\vert
q_{1}-q_{2}\right\vert $ is an integer). We can multiply this inequality by
$n$ (since $n$ is positive) and thus obtain $\left\vert q_{1}-q_{2}\right\vert
n\geq1n=n$. But from (\ref{pf.lem.ent.quorem.unique.1}), we obtain%
\begin{align*}
\left\vert r_{2}-r_{1}\right\vert  &  =\left\vert \left(  q_{1}-q_{2}\right)
n\right\vert =\left\vert q_{1}-q_{2}\right\vert \cdot\underbrace{\left\vert
n\right\vert }_{\substack{=n\\\text{(since }n\text{ is positive)}%
}}\ \ \ \ \ \ \ \ \ \ \left(  \text{by (\ref{eq.ent.div.abs(xy)})}\right) \\
&  =\left\vert q_{1}-q_{2}\right\vert n\geq n>n-1.
\end{align*}
This contradicts (\ref{pf.lem.ent.quorem.unique.ineq}). This contradiction
shows that our assumption (that $q_{1}\neq q_{2}$) was false. Hence, we have
$q_{1}=q_{2}$. Thus, $q_{1}-q_{2}=0$, so that
(\ref{pf.lem.ent.quorem.unique.1}) becomes $r_{2}-r_{1}=\underbrace{\left(
q_{1}-q_{2}\right)  }_{=0}n=0$ and thus $r_{2}=r_{1}$, so that $r_{1}=r_{2}$.
Combining this with $q_{1}=q_{2}$, we obtain $\left(  q_{1},r_{1}\right)
=\left(  q_{2},r_{2}\right)  $.

Now, forget that we have fixed $\left(  q_{1},r_{1}\right)  $ and $\left(
q_{2},r_{2}\right)  $. We thus have proven that if $\left(  q_{1}%
,r_{1}\right)  $ and $\left(  q_{2},r_{2}\right)  $ are two pairs $\left(
q,r\right)  \in\mathbb{Z}\times\left\{  0,1,\ldots,n-1\right\}  $ such that
$u=qn+r$, then $\left(  q_{1},r_{1}\right)  =\left(  q_{2},r_{2}\right)  $. In
other words, any two pairs $\left(  q,r\right)  \in\mathbb{Z}\times\left\{
0,1,\ldots,n-1\right\}  $ such that $u=qn+r$ must be equal. In other words,
there exists at most one such pair. This proves Lemma
\ref{lem.ent.quorem.unique}.
\end{proof}

But we also need to prove Lemma \ref{lem.ent.quorem.exist}. This lemma can be
proven by induction on $u$, but not without some complications: Since it is
stated for all integers $u$ (rather than just for nonnegative or positive
integers), the classical induction principle (with an induction base and a
\textquotedblleft$u$ to $u+1$\textquotedblright\ step) cannot prove it
directly. Instead, we have to either add a \textquotedblleft$u$ to
$u-1$\textquotedblright\ step to our induction (resulting in a
\textquotedblleft two-sided induction\textquotedblright\ or \textquotedblleft
up- and down-induction\textquotedblright\ argument), or to treat the case of
negative $u$ separately. A proof using the first of these two methods can be
found in \cite[proof of Proposition 2.150]{detnotes} (where $n$ and $u$ are
denoted by $N$ and $n$). We shall instead give a proof using the second
method; thus, we first state the particular case of Lemma
\ref{lem.ent.quorem.exist} when $u$ is nonnegative:

\begin{lemma}
\label{lem.ent.quorem.existN}Let $n$ be a positive integer. Let $u\in
\mathbb{N}$. Then, there exists \textbf{at least one} pair $\left(
q,r\right)  \in\mathbb{Z}\times\left\{  0,1,\ldots,n-1\right\}  $ such that
$u=qn+r$.
\end{lemma}

This lemma can be proven by induction on $u$ as in \cite[proof of Proposition
2.150]{detnotes}. Let us instead prove it by \textbf{strong} induction on $u$.
See \cite[\S 2.8]{detnotes} for an introduction to strong induction; in
particular, recall that a strong induction needs no induction base (but often
contains a case distinction in its \textquotedblleft induction
step\textquotedblright\ that, in some way, does give the first few values a
special treatment). The proof of Lemma \ref{lem.ent.quorem.existN} that we
give below follows a stupid but valid method of finding the pair $\left(
q,r\right)  $: Keep subtracting $n$ from $u$ until $u$ becomes $<n$; then $r$
will be the resulting number, whereas $q$ will be the number of times you have
subtracted $n$.

\begin{proof}
[Proof of Lemma \ref{lem.ent.quorem.existN}.]We proceed by strong induction on
$u$.

Let $U\in\mathbb{N}$. Assume (as the induction hypothesis) that Lemma
\ref{lem.ent.quorem.existN} holds for every $u\in\mathbb{N}$ satisfying $u<U$.
We must prove that Lemma \ref{lem.ent.quorem.existN} also holds for $u=U$. In
other words, we must prove that there exists \textbf{at least one} pair
$\left(  q,r\right)  \in\mathbb{Z}\times\left\{  0,1,\ldots,n-1\right\}  $
such that $U=qn+r$.

We are in one of the following two cases:

\textit{Case 1:} We have $U<n$.

\textit{Case 2:} We have $U\geq n$.

Let us first consider Case 1. In this case, we have $U<n$. Thus, $U\leq n-1$
(since $U$ and $n$ are integers), so that $U\in\left\{  0,1,\ldots
,n-1\right\}  $ (since $U\in\mathbb{N}$). Combining this with $0\in\mathbb{Z}%
$, we obtain $\left(  0,U\right)  \in\mathbb{Z}\times\left\{  0,1,\ldots
,n-1\right\}  $. Hence, $\left(  0,U\right)  $ is a pair $\left(  q,r\right)
\in\mathbb{Z}\times\left\{  0,1,\ldots,n-1\right\}  $ such that $U=qn+r$
(since $U=0n+U$). Thus, there exists \textbf{at least one} pair $\left(
q,r\right)  \in\mathbb{Z}\times\left\{  0,1,\ldots,n-1\right\}  $ such that
$U=qn+r$ (namely, $\left(  q,r\right)  =\left(  0,U\right)  $).

Let us now consider Case 2. In this case, we have $U\geq n$. Hence, $U-n\geq
0$, so that $U-n\in\mathbb{N}$ (remember that $\mathbb{N}=\left\{
0,1,2,\ldots\right\}  $). Also, $U-n<U$ (since $n$ is positive). But our
induction hypothesis said that Lemma \ref{lem.ent.quorem.existN} holds for
every $u\in\mathbb{N}$ satisfying $u<U$. Hence, in particular, Lemma
\ref{lem.ent.quorem.existN} holds for $u=U-n$ (since $U-n\in\mathbb{N}$ and
$U-n<U$). In other words, there exists \textbf{at least one} pair $\left(
q,r\right)  \in\mathbb{Z}\times\left\{  0,1,\ldots,n-1\right\}  $ such that
$U-n=qn+r$. Fix such a pair and denote it by $\left(  q_{0},r_{0}\right)  $.
Thus, $\left(  q_{0},r_{0}\right)  \in\mathbb{Z}\times\left\{  0,1,\ldots
,n-1\right\}  $ and $U-n=q_{0}n+r_{0}$.

From $U-n=q_{0}n+r_{0}$, we obtain $U=n+\left(  q_{0}n+r_{0}\right)  =\left(
q_{0}+1\right)  n+r_{0}$. Also, from $\left(  q_{0},r_{0}\right)
\in\mathbb{Z}\times\left\{  0,1,\ldots,n-1\right\}  $, we obtain $q_{0}%
\in\mathbb{Z}$ and $r_{0}\in\left\{  0,1,\ldots,n-1\right\}  $, and thus
$\left(  q_{0}+1,r_{0}\right)  \in\mathbb{Z}\times\left\{  0,1,\ldots
,n-1\right\}  $. Thus, the pair $\left(  q_{0}+1,r_{0}\right)  $ is a pair
$\left(  q,r\right)  \in\mathbb{Z}\times\left\{  0,1,\ldots,n-1\right\}  $
such that $U=qn+r$ (since $U=\left(  q_{0}+1\right)  n+r_{0}$). Therefore,
there exists \textbf{at least one} pair $\left(  q,r\right)  \in
\mathbb{Z}\times\left\{  0,1,\ldots,n-1\right\}  $ such that $U=qn+r$ (namely,
$\left(  q,r\right)  =\left(  q_{0}+1,r_{0}\right)  $).

Now, in each of the two Cases 1 and 2, we have shown that there exists
\textbf{at least one} pair $\left(  q,r\right)  \in\mathbb{Z}\times\left\{
0,1,\ldots,n-1\right\}  $ such that $U=qn+r$. Hence, this holds always. In
other words, Lemma \ref{lem.ent.quorem.existN} holds for $u=U$. This completes
the induction step; thus, Lemma \ref{lem.ent.quorem.existN} is proven by
strong induction.
\end{proof}

In order to derive Lemma \ref{lem.ent.quorem.exist} from Lemma
\ref{lem.ent.quorem.existN} (that is, to extend Lemma
\ref{lem.ent.quorem.existN} to the case of negative $u$), we shall need a
simple but important trick:

\begin{lemma}
\label{lem.ent.cong-to-nonneg}Let $n$ be a positive integer. Let
$u\in\mathbb{Z}$. Then, there exists a $v\in\mathbb{N}$ such that $u\equiv
v\operatorname{mod}n$.
\end{lemma}

\begin{proof}
[Proof of Lemma \ref{lem.ent.cong-to-nonneg}.]We are in one of the following
two cases:

\textit{Case 1:} We have $u\geq0$.

\textit{Case 2:} We have $u<0$.

Let us first consider Case 1. In this case, we have $u\geq0$. Thus,
$u\in\mathbb{N}$. Also, $u\equiv u\operatorname{mod}n$ (by Proposition
\ref{prop.ent.mod.basics} \textbf{(a)}). Thus, there exists a $v\in\mathbb{N}$
such that $u\equiv v\operatorname{mod}n$ (namely, $v=u$). This proves Lemma
\ref{lem.ent.cong-to-nonneg} in Case 1.

Let us now consider Case 2. In this case, we have $u<0$. Hence, $-u>0$. Now,
$u-\left(  n-1\right)  \left(  -u\right)  =nu$ is divisible by $n$ (since
$u\in\mathbb{Z}$). In other words, $n\mid u-\left(  n-1\right)  \left(
-u\right)  $. In other words, $u\equiv\left(  n-1\right)  \left(  -u\right)
\operatorname{mod}n$. Moreover, $n\geq1$ (since $n$ is a positive integer), so
that $n-1\geq0$. We can multiply this inequality with $-u$ (since $-u>0$), and
thus obtain $\left(  n-1\right)  \left(  -u\right)  \geq0\left(  -u\right)
=0$. In other words, $\left(  n-1\right)  \left(  -u\right)  \in\mathbb{N}$.
Thus, there exists a $v\in\mathbb{N}$ such that $u\equiv v\operatorname{mod}n$
(namely, $v=\left(  n-1\right)  \left(  -u\right)  $). This proves Lemma
\ref{lem.ent.cong-to-nonneg} in Case 2.

We have now proven Lemma \ref{lem.ent.cong-to-nonneg} in both Cases 1 and 2;
hence, Lemma \ref{lem.ent.cong-to-nonneg} always holds.
\end{proof}

\begin{proof}
[Proof of Lemma \ref{lem.ent.quorem.exist}.]Lemma \ref{lem.ent.cong-to-nonneg}
shows that there exists a $v\in\mathbb{N}$ such that $u\equiv
v\operatorname{mod}n$. Consider this $v$.

Note that $n\mid u-v$ (since $u\equiv v\operatorname{mod}n$). In other words,
there exists an integer $c$ such that $u-v=nc$. Consider this $c$. From
$u-v=nc$, we obtain $u=v+nc$.

Lemma \ref{lem.ent.quorem.existN} (applied to $v$ instead of $u$) yields that
there exists \textbf{at least one} pair $\left(  q,r\right)  \in
\mathbb{Z}\times\left\{  0,1,\ldots,n-1\right\}  $ such that $v=qn+r$. Fix
such a pair, and denote it by $\left(  q_{0},r_{0}\right)  $. Thus, $\left(
q_{0},r_{0}\right)  \in\mathbb{Z}\times\left\{  0,1,\ldots,n-1\right\}  $ and
$v=q_{0}n+r_{0}$. Now,%
\[
u=\underbrace{v}_{=q_{0}n+r_{0}}+nc=\left(  q_{0}n+r_{0}\right)  +nc=\left(
q_{0}+c\right)  n+r_{0}.
\]
Also, from $\left(  q_{0},r_{0}\right)  \in\mathbb{Z}\times\left\{
0,1,\ldots,n-1\right\}  $, we obtain $q_{0}\in\mathbb{Z}$ and $r_{0}%
\in\left\{  0,1,\ldots,n-1\right\}  $, and thus $\left(  q_{0}+c,r_{0}\right)
\in\mathbb{Z}\times\left\{  0,1,\ldots,n-1\right\}  $. Thus, the pair $\left(
q_{0}+c,r_{0}\right)  $ is a pair $\left(  q,r\right)  \in\mathbb{Z}%
\times\left\{  0,1,\ldots,n-1\right\}  $ such that $u=qn+r$ (since $u=\left(
q_{0}+c\right)  n+r_{0}$). Therefore, there exists \textbf{at least one} pair
$\left(  q,r\right)  \in\mathbb{Z}\times\left\{  0,1,\ldots,n-1\right\}  $
such that $u=qn+r$ (namely, $\left(  q,r\right)  =\left(  q_{0}+c,r_{0}%
\right)  $). This proves Lemma \ref{lem.ent.quorem.exist}.
\end{proof}

\begin{proof}
[Proof of Theorem \ref{thm.ent.quorem.full}.]Theorem \ref{thm.ent.quorem.full}
follows by combining Lemma \ref{lem.ent.quorem.exist} with Lemma
\ref{lem.ent.quorem.unique}.
\end{proof}

The following properties of the quotient and the remainder are simple but will
be used all the time:

\begin{corollary}
\label{cor.ent.quo-rem.remmod}Let $n$ be a positive integer. Let
$u\in\mathbb{Z}$.

\textbf{(a)} Then, $u\%n\in\left\{  0,1,\ldots,n-1\right\}  $ and $u\%n\equiv
u\operatorname{mod}n$.

\textbf{(b)} We have $n\mid u$ if and only if $u\%n=0$.

\textbf{(c)} If $c\in\left\{  0,1,\ldots,n-1\right\}  $ is such that $c\equiv
u\operatorname{mod}n$, then $c=u\%n$.

\textbf{(d)} We have $u=\left(  u//n\right)  n+\left(  u\%n\right)  $.
\end{corollary}

Before we prove this corollary, let us explain its purpose. Corollary
\ref{cor.ent.quo-rem.remmod} \textbf{(a)} says that $u\%n$ is a number in the
set $\left\{  0,1,\ldots,n-1\right\}  $ that is congruent to $u$ modulo $n$.
Corollary \ref{cor.ent.quo-rem.remmod} \textbf{(c)} says that $u\%n$ is the
\textbf{only} such number (as it says that any further such number $c$ must be
equal to $u\%n$). Corollary \ref{cor.ent.quo-rem.remmod} \textbf{(b)} gives an
algorithm to check whether $n\mid u$ holds (namely, compute $u\%n$ and check
whether $u\%n=0$). Corollary \ref{cor.ent.quo-rem.remmod} \textbf{(d)} is a
trivial consequence of the definition of quotient and remainder.

\begin{proof}
[Proof of Corollary \ref{cor.ent.quo-rem.remmod}.]Theorem
\ref{thm.ent.quorem.full} says that there is a unique pair $\left(
q,r\right)  \in\mathbb{Z}\times\left\{  0,1,\ldots,n-1\right\}  $ such that
$u=qn+r$. Consider this pair $\left(  q,r\right)  $. The uniqueness of this
pair can be restated as follows: If $\left(  q^{\prime},r^{\prime}\right)
\in\mathbb{Z}\times\left\{  0,1,\ldots,n-1\right\}  $ is any further pair such
that $u=q^{\prime}n+r^{\prime}$, then%
\begin{equation}
\left(  q^{\prime},r^{\prime}\right)  =\left(  q,r\right)  .
\label{pf.cor.ent.quo-rem.remmod.uni}%
\end{equation}


Recall that $u\%n$ was defined to be $r$ (in Definition \ref{def.ent.quorem}
\textbf{(b)}). Thus, $u\%n=r$. Now, $n\mid qn=u-r$ (since $u=qn+r$). In other
words, $u\equiv r\operatorname{mod}n$. Hence, $r\equiv u\operatorname{mod}n$
(by Proposition \ref{prop.ent.mod.basics} \textbf{(c)}). This rewrites as
$u\%n\equiv u\operatorname{mod}n$ (since $r=u\%n$).

Furthermore, $u\%n=r\in\left\{  0,1,\ldots,n-1\right\}  $ (since $\left(
q,r\right)  \in\mathbb{Z}\times\left\{  0,1,\ldots,n-1\right\}  $). This
completes the proof of Corollary \ref{cor.ent.quo-rem.remmod} \textbf{(a)}.

Also, $u//n$ was defined to be $q$ (in Definition \ref{def.ent.quorem}
\textbf{(a)}). Hence, $u//n=q$. Now,%
\[
u=\underbrace{q}_{=u//n}n+\underbrace{r}_{=u\%n}=\left(  u//n\right)
n+\left(  u\%n\right)  .
\]
This proves Corollary \ref{cor.ent.quo-rem.remmod} \textbf{(d)}.

\textbf{(b)} $\Longrightarrow:$ Assume that $n\mid u$. We must prove that
$u\%n=0$.

We have $n\mid u$. In other words, there exists some integer $w$ such that
$u=nw$. Consider this $w$.

We have $n-1\in\mathbb{N}$ (since $n$ is a positive integer), thus
$0\in\left\{  0,1,\ldots,n-1\right\}  $. Hence, $\left(  w,0\right)
\in\mathbb{Z}\times\left\{  0,1,\ldots,n-1\right\}  $ (since $w\in\mathbb{Z}%
$). Also, $u=nw=wn=wn+0$. Hence, (\ref{pf.cor.ent.quo-rem.remmod.uni})
(applied to $\left(  q^{\prime},r^{\prime}\right)  =\left(  w,0\right)  $)
yields $\left(  w,0\right)  =\left(  q,r\right)  $. In other words, $w=q$ and
$0=r$. Hence, $r=0$, so that $u\%n=r=0$. This proves the \textquotedblleft%
$\Longrightarrow$\textquotedblright\ implication of Corollary
\ref{cor.ent.quo-rem.remmod} \textbf{(b)}.

$\Longleftarrow:$ Assume that $u\%n=0$. We must prove that $n\mid u$.

We have $u=qn+\underbrace{r}_{=u\%n=0}=qn=nq$. Thus, $n\mid u$. This proves
the \textquotedblleft$\Longleftarrow$\textquotedblright\ implication of
Corollary \ref{cor.ent.quo-rem.remmod} \textbf{(b)}.

\textbf{(c)} Let $c\in\left\{  0,1,\ldots,n-1\right\}  $ be such that $c\equiv
u\operatorname{mod}n$.

We have $c\equiv u\operatorname{mod}n$. In other words, $n\mid c-u$. In other
words, there exists some integer $w$ such that $c-u=nw$. Consider this $w$.

From $-w\in\mathbb{Z}$ and $c\in\left\{  0,1,\ldots,n-1\right\}  $, we obtain
$\left(  -w,c\right)  \in\mathbb{Z}\times\left\{  0,1,\ldots,n-1\right\}  $.
Also, from $c-u=nw$, we obtain $u=c-nw=\left(  -w\right)  n+c$. Hence,
(\ref{pf.cor.ent.quo-rem.remmod.uni}) (applied to $\left(  q^{\prime
},r^{\prime}\right)  =\left(  -w,c\right)  $) yields $\left(  -w,c\right)
=\left(  q,r\right)  $. In other words, $-w=q$ and $c=r$. Hence, $c=r=u\%n$.
This proves Corollary \ref{cor.ent.quo-rem.remmod} \textbf{(c)}.
\end{proof}

\begin{exercise}
\label{exe.ent.quo-rem.mod=rem}Let $n$ be a positive integer. Let $u$ and $v$
be integers. Prove that $u\equiv v\operatorname{mod}n$ if and only if
$u\%n=v\%n$.
\end{exercise}

\begin{fineprint}
\begin{proof}
[Solution to Exercise \ref{exe.ent.quo-rem.mod=rem}.]$\Longrightarrow:$ Assume
that $u\equiv v\operatorname{mod}n$. We must prove that $u\%n=v\%n$.

Corollary \ref{cor.ent.quo-rem.remmod} \textbf{(a)} yields that $u\%n\in
\left\{  0,1,\ldots,n-1\right\}  $ and $u\%n\equiv u\operatorname{mod}n$.
Hence, $u\%n\equiv u\equiv v\operatorname{mod}n$.

But Corollary \ref{cor.ent.quo-rem.remmod} \textbf{(c)} (applied to $v$
instead of $u$) yields that if $c\in\left\{  0,1,\ldots,n-1\right\}  $ is such
that $c\equiv v\operatorname{mod}n$, then $c=v\%n$. Applying this to $c=u\%n$,
we obtain $u\%n=v\%n$ (since $u\%n\in\left\{  0,1,\ldots,n-1\right\}  $ and
$u\%n\equiv v\operatorname{mod}n$). This proves the \textquotedblleft%
$\Longrightarrow$\textquotedblright\ direction of Exercise
\ref{exe.ent.quo-rem.mod=rem}.

$\Longleftarrow:$ Assume that $u\%n=v\%n$. We must prove that $u\equiv
v\operatorname{mod}n$.

Corollary \ref{cor.ent.quo-rem.remmod} \textbf{(a)} yields that $u\%n\in
\left\{  0,1,\ldots,n-1\right\}  $ and $u\%n\equiv u\operatorname{mod}n$.
Corollary \ref{cor.ent.quo-rem.remmod} \textbf{(a)} (applied to $v$ instead of
$u$) yields that $v\%n\in\left\{  0,1,\ldots,n-1\right\}  $ and $v\%n\equiv
v\operatorname{mod}n$.

From $u\%n\equiv u\operatorname{mod}n$, we obtain $u\equiv u\%n=v\%n\equiv
v\operatorname{mod}n$. Thus, we have proven $u\equiv v\operatorname{mod}n$.
This proves the \textquotedblleft$\Longleftarrow$\textquotedblright\ direction
of Exercise \ref{exe.ent.quo-rem.mod=rem}.
\end{proof}
\end{fineprint}

\subsection{Even and odd numbers}

Recall the following:

\begin{definition}
\label{def.ent.even-odd}Let $u$ be an integer.

\textbf{(a)} We say that $u$ is \textit{even} if $u$ is divisible by $2$.

\textbf{(b)} We say that $u$ is \textit{odd }if $u$ is not divisible by $2$.
\end{definition}

So an integer is either even or odd (but not both at the same time). The
following exercise collects various properties of even and odd integers:

\begin{exercise}
\label{exe.ent.even-odd.1}Let $u$ be an integer.

\textbf{(a)} Prove that $u$ is even if and only if $u\%2=0$.

\textbf{(b)} Prove that $u$ is odd if and only if $u\%2=1$.

\textbf{(c)} Prove that $u$ is even if and only if $u\equiv0\operatorname{mod}%
2$.

\textbf{(d)} Prove that $u$ is odd if and only if $u\equiv1\operatorname{mod}%
2$.

\textbf{(e)} Prove that $u$ is odd if and only if $u+1$ is even.

\textbf{(f)} Prove that exactly one of the two numbers $u$ and $u+1$ is even.

\textbf{(g)} Prove that $u\left(  u+1\right)  \equiv0\operatorname{mod}2$.

\textbf{(h)} Prove that $u^{2}\equiv-u\equiv u\operatorname{mod}2$.

\textbf{(i)} Let $v$ be a further integer. Prove that $u\equiv
v\operatorname{mod}2$ holds if and only if $u$ and $v$ are either both odd or
both even.
\end{exercise}

\begin{fineprint}
\begin{proof}
[Solution to Exercise \ref{exe.ent.even-odd.1}.]Corollary
\ref{cor.ent.quo-rem.remmod} \textbf{(b)} (applied to $n=2$) shows that we
have $2\mid u$ if and only if $u\%2=0$. In other words, we have the logical
equivalence
\begin{equation}
\left(  2\mid u\right)  \ \Longleftrightarrow\ \left(  u\%2=0\right)  .
\label{sol.ent.even-odd.1.1}%
\end{equation}


Corollary \ref{cor.ent.quo-rem.remmod} \textbf{(a)} (applied to $n=2$) yields
that $u\%2\in\left\{  0,1,\ldots,2-1\right\}  $ and $u\%2\equiv
u\operatorname{mod}2$. Thus, in particular, $u\%2\in\left\{  0,1,\ldots
,2-1\right\}  =\left\{  0,1\right\}  $. Hence, $u\%2$ is either $0$ or $1$.
Thus, the number $u\%2$ is $1$ if and only if it is not $0$. In other words,
we have the equivalence%
\begin{equation}
\left(  u\%2=1\right)  \ \Longleftrightarrow\ \left(  u\%2\neq0\right)  .
\label{sol.ent.even-odd.1.2}%
\end{equation}


Proposition \ref{prop.ent.mod.0} (applied to $a=u$ and $n=2$) shows that
$u\equiv0\operatorname{mod}2$ if and only if $2\mid u$. In other words, we
have the equivalence%
\begin{equation}
\left(  u\equiv0\operatorname{mod}2\right)  \ \Longleftrightarrow\ \left(
2\mid u\right)  . \label{sol.ent.even-odd.1.3}%
\end{equation}


\textbf{(a)} We have the following chain of equivalences:%
\begin{align*}
\left(  u\text{ is even}\right)  \  &  \Longleftrightarrow\ \left(  u\text{ is
divisible by }2\right)  \ \ \ \ \ \ \ \ \ \ \left(  \text{by the definition of
\textquotedblleft even\textquotedblright}\right) \\
&  \Longleftrightarrow\ \left(  2\mid u\right)  \ \Longleftrightarrow\ \left(
u\%2=0\right)  \ \ \ \ \ \ \ \ \ \ \left(  \text{by
(\ref{sol.ent.even-odd.1.1})}\right)  .
\end{align*}
In other words, $u$ is even if and only if $u\%2=0$. This solves Exercise
\ref{exe.ent.even-odd.1} \textbf{(a)}.

\textbf{(b)} We have the following chain of equivalences:%
\begin{align}
\left(  u\text{ is odd}\right)  \  &  \Longleftrightarrow\ \left(  u\text{ is
not divisible by }2\right)  \ \ \ \ \ \ \ \ \ \ \left(  \text{by the
definition of \textquotedblleft odd\textquotedblright}\right) \nonumber\\
&  \Longleftrightarrow\ \left(  \text{we don't have }2\mid u\right)
\ \Longleftrightarrow\ \left(  \text{we don't have }u\%2=0\right) \nonumber\\
&  \ \ \ \ \ \ \ \ \ \ \left(  \text{because of the equivalence }\left(  2\mid
u\right)  \ \Longleftrightarrow\ \left(  u\%2=0\right)  \right) \nonumber\\
&  \Longleftrightarrow\ \left(  u\%2\neq0\right)
\label{sol.ent.even-odd.1.b.1}\\
&  \Longleftrightarrow\ \left(  u\%2=1\right)  \ \ \ \ \ \ \ \ \ \ \left(
\text{by (\ref{sol.ent.even-odd.1.2})}\right)  .\nonumber
\end{align}
In other words, $u$ is odd if and only if $u\%2=1$. This solves Exercise
\ref{exe.ent.even-odd.1} \textbf{(b)}.

\textbf{(c)} We have the following chain of equivalences:%
\begin{align*}
\left(  u\text{ is even}\right)  \  &  \Longleftrightarrow\ \left(  u\text{ is
divisible by }2\right)  \ \ \ \ \ \ \ \ \ \ \left(  \text{by the definition of
\textquotedblleft even\textquotedblright}\right) \\
&  \Longleftrightarrow\ \left(  2\mid u\right)  \ \Longleftrightarrow\ \left(
u\equiv0\operatorname{mod}2\right)  \ \ \ \ \ \ \ \ \ \ \left(  \text{by
(\ref{sol.ent.even-odd.1.3})}\right)  .
\end{align*}
In other words, $u$ is even if and only if $u\equiv0\operatorname{mod}2$. This
solves Exercise \ref{exe.ent.even-odd.1} \textbf{(c)}.

\textbf{(d)} $\Longrightarrow:$ Assume that $u$ is odd. We must prove that
$u\equiv1\operatorname{mod}2$.

We know that $u$ is odd. In other words, $u\%2=1$ (by Exercise
\ref{exe.ent.even-odd.1} \textbf{(b)}). But recall that $u\%2\equiv
u\operatorname{mod}2$. Thus, $u\equiv u\%2=1\operatorname{mod}2$. This proves
the \textquotedblleft$\Longrightarrow$\textquotedblright\ direction of
Exercise \ref{exe.ent.even-odd.1} \textbf{(d)}.

$\Longleftarrow:$ Assume that $u\equiv1\operatorname{mod}2$. We must prove
that $u$ is odd.

We have $1\equiv u\operatorname{mod}2$ (since $u\equiv1\operatorname{mod}2$)
and $1\in\left\{  0,1,\ldots,2-1\right\}  $. But Corollary
\ref{cor.ent.quo-rem.remmod} \textbf{(c)} (applied to $n=2$) says that if
$c\in\left\{  0,1,\ldots,2-1\right\}  $ satisfies $c\equiv u\operatorname{mod}%
2$, then $c=u\%2$. Applying this to $c=1$, we find $1=u\%2$ (since
$1\in\left\{  0,1,\ldots,2-1\right\}  $ and $1\equiv u\operatorname{mod}2$).
In other words, $u\%2=1$. According to Exercise \ref{exe.ent.even-odd.1}
\textbf{(b)}, this means that $u$ is odd. This proves the \textquotedblleft%
$\Longleftarrow$\textquotedblright\ direction of Exercise
\ref{exe.ent.even-odd.1} \textbf{(d)}.

\textbf{(e)} $\Longrightarrow:$ Assume that $u$ is odd. We must prove that
$u+1$ is even.

We have assumed that $u$ is odd. According to Exercise
\ref{exe.ent.even-odd.1} \textbf{(d)}, this means that $u\equiv
1\operatorname{mod}2$. On the other hand, $1\equiv-1\operatorname{mod}2$
(since $2\mid1-\left(  -1\right)  $). Adding these two congruences together,
we find $u+1\equiv1+\left(  -1\right)  =0\operatorname{mod}2$.

But Exercise \ref{exe.ent.even-odd.1} \textbf{(c)} (applied to $u+1$ instead
of $u$) shows that $u+1$ is even if and only if $u+1\equiv0\operatorname{mod}%
2$. Hence, $u+1$ is even (since $u+1\equiv0\operatorname{mod}2$). This proves
the \textquotedblleft$\Longrightarrow$\textquotedblright\ direction of
Exercise \ref{exe.ent.even-odd.1} \textbf{(e)}.

$\Longleftarrow:$ Assume that $u+1$ is even. We must prove that $u$ is odd.

We know that $u+1$ is even. But Exercise \ref{exe.ent.even-odd.1} \textbf{(c)}
(applied to $u+1$ instead of $u$) shows that $u+1$ is even if and only if
$u+1\equiv0\operatorname{mod}2$. Hence, $u+1\equiv0\operatorname{mod}2$ (since
$u+1$ is even). On the other hand, $-1\equiv1\operatorname{mod}2$ (since
$2\mid\left(  -1\right)  -1$). Adding these two congruences together, we
obtain $\left(  u+1\right)  +\left(  -1\right)  \equiv0+1=1\operatorname{mod}%
2$. In view of $\left(  u+1\right)  +\left(  -1\right)  =u$, this rewrites as
$u\equiv1\operatorname{mod}2$. According to Exercise \ref{exe.ent.even-odd.1}
\textbf{(d)}, this means that $u$ is odd. This proves the \textquotedblleft%
$\Longleftarrow$\textquotedblright\ direction of Exercise
\ref{exe.ent.even-odd.1} \textbf{(e)}.

\textbf{(f)} We have the equivalence $\left(  u\text{ is divisible by
}2\right)  \ \Longleftrightarrow\ \left(  u\text{ is even}\right)  $ (by the
definition of \textquotedblleft even\textquotedblright).

Exercise \ref{exe.ent.even-odd.1} \textbf{(e)} shows that $u$ is odd if and
only if $u+1$ is even. Thus, we have the following chain of equivalences:%
\begin{align*}
&  \ \left(  u+1\text{ is even}\right) \\
&  \Longleftrightarrow\ \left(  u\text{ is odd}\right)  \ \Longleftrightarrow
\ \left(  u\text{ is not divisible by }2\right)  \ \ \ \ \ \ \ \ \ \ \left(
\text{by the definition of \textquotedblleft odd\textquotedblright}\right) \\
&  \Longleftrightarrow\ \left(  u\text{ is not even}\right)
\end{align*}
(because of the equivalence $\left(  u\text{ is divisible by }2\right)
\ \Longleftrightarrow\ \left(  u\text{ is even}\right)  $). In other words,
$u+1$ is even if and only if $u$ is not. In other words, exactly one of the
two numbers $u$ and $u+1$ is even. This solves Exercise
\ref{exe.ent.even-odd.1} \textbf{(f)}.

\textbf{(g)} Exercise \ref{exe.ent.even-odd.1} \textbf{(f)} shows that exactly
one of the two numbers $u$ and $u+1$ is even. Thus, in particular, \textbf{at
least} one of these two numbers is even. Hence, the product $u\left(
u+1\right)  $ has \textbf{at least} one even factor. But a product of any even
integer with any integer is even\footnote{\textit{Proof.} We must prove that
if $a$ is an even integer, and if $b$ is an integer, then the product $ab$ is
even.
\par
So let $a$ be an even integer, and let $b$ be an integer. Then, $a$ is even;
in other words, $2\mid a$ (by the definition of \textquotedblleft
even\textquotedblright). But $a\mid ab$. Hence, $2\mid a\mid ab$; in other
words, $ab$ is even. Qed.}. Hence, a product that has at least one even factor
is always even. Thus, $u\left(  u+1\right)  $ is even (since $u\left(
u+1\right)  $ is a product that has at least one even factor). In other words,
$2\mid u\left(  u+1\right)  $. In other words, $u\left(  u+1\right)
\equiv0\operatorname{mod}2$. This solves Exercise \ref{exe.ent.even-odd.1}
\textbf{(g)}.

\textbf{(h)} We have $u^{2}-\left(  -u\right)  =u^{2}+u=u\left(  u+1\right)
\equiv0\operatorname{mod}2$ (by Exercise \ref{exe.ent.even-odd.1}
\textbf{(g)}). In other words, $2\mid u^{2}-\left(  -u\right)  $. In other
words, $u^{2}\equiv-u\operatorname{mod}2$.

Also, $2\mid\left(  -u\right)  -u$ (since $\left(  -u\right)  -u=2\left(
-u\right)  $ is clearly divisible by $2$); in other words, $-u\equiv
u\operatorname{mod}2$. Hence, $u^{2}\equiv-u\equiv u\operatorname{mod}2$. This
solves Exercise \ref{exe.ent.even-odd.1} \textbf{(h)}.

\textbf{(i)} Exercise \ref{exe.ent.quo-rem.mod=rem} (applied to $n=2$) shows
that $u\equiv v\operatorname{mod}2$ if and only if $u\%2=v\%2$.

We are in one of the following four cases:

\textit{Case 1:} We have $u\%2=0$ and $v\%2=0$.

\textit{Case 2:} We have $u\%2=0$ and $v\%2\neq0$.

\textit{Case 3:} We have $u\%2\neq0$ and $v\%2=0$.

\textit{Case 4:} We have $u\%2\neq0$ and $v\%2\neq0$.

Let us first consider Case 1. In this case, we have $u\%2=0$ and $v\%2=0$.
Thus, $u\%2=0=v\%2$ and therefore $u\equiv v\operatorname{mod}2$ (since we
know that $u\equiv v\operatorname{mod}2$ if and only if $u\%2=v\%2$). But
recall that $u\%2=0$. Equivalently, $u$ is even (because of Exercise
\ref{exe.ent.even-odd.1} \textbf{(a)}). Similarly, from $v\%2=0$, we conclude
that $v$ is even. Thus, $u$ and $v$ are either both odd or both even (namely,
they are both even).

Thus, $u\equiv v\operatorname{mod}2$ holds if and only if $u$ and $v$ are
either both odd or both even (because both statements \textquotedblleft%
$u\equiv v\operatorname{mod}2$\textquotedblright\ and \textquotedblleft$u$ and
$v$ are either both odd or both even\textquotedblright\ hold). Hence, Exercise
\ref{exe.ent.even-odd.1} \textbf{(i)} is solved in Case 1.

Let us now consider Case 2. In this case, we have $u\%2=0$ and $v\%2\neq0$.
Thus, $u\%2=0\neq v\%2$. In other words, \textquotedblleft$u\%2=v\%2$%
\textquotedblright\ is false. Thus, \textquotedblleft$u\equiv
v\operatorname{mod}2$\textquotedblright\ is false as well (since we know that
$u\equiv v\operatorname{mod}2$ if and only if $u\%2=v\%2$). But recall that
$u\%2=0$. Equivalently, $u$ is even (because of Exercise
\ref{exe.ent.even-odd.1} \textbf{(a)}). Hence, $u$ is not odd\footnote{because
an integer is either even or odd (but not both at the same time)}. Thus, $u$
and $v$ are not both even. Also, Exercise \ref{exe.ent.even-odd.1}
\textbf{(a)} (applied to $v$ instead of $u$) shows that $v$ is even if and
only if $v\%2=0$. Since we don't have $v\%2=0$ (because $v\%2\neq0$), we thus
conclude that $v$ is not even. Thus, $u$ and $v$ are not both odd.

So $u$ and $v$ are neither both odd nor both even. In other words, the
statement \textquotedblleft$u$ and $v$ are either both odd or both
even\textquotedblright\ is false.

Thus, $u\equiv v\operatorname{mod}2$ holds if and only if $u$ and $v$ are
either both odd or both even (because both statements \textquotedblleft%
$u\equiv v\operatorname{mod}2$\textquotedblright\ and \textquotedblleft$u$ and
$v$ are either both odd or both even\textquotedblright\ are false). Hence,
Exercise \ref{exe.ent.even-odd.1} \textbf{(i)} is solved in Case 2.

Case 3 is analogous to Case 2 (it differs from Case 2 only in that $u$ and $v$
trade places).

Let us finally consider Case 4. In this case, we have $u\%2\neq0$ and
$v\%2\neq0$. By (\ref{sol.ent.even-odd.1.b.1}), we have the logical
equivalence $\left(  u\text{ is odd}\right)  \ \Longleftrightarrow\ \left(
u\%2\neq0\right)  $. Hence, $u$ is odd (since $u\%2\neq0$). Similarly, $v$ is
odd. Thus, $u$ and $v$ are both odd. Thus, $u$ and $v$ are either both odd or
both even (namely, they are both odd). Moreover, we know that $u$ is odd;
equivalently, $u\%2=1$ (by Exercise \ref{exe.ent.even-odd.1} \textbf{(b)}).
Similarly, $v\%2=1$. Hence, $u\%2=1=v\%2$. Therefore, $u\equiv
v\operatorname{mod}2$ (since we know that $u\equiv v\operatorname{mod}2$ if
and only if $u\%2=v\%2$).

Thus, $u\equiv v\operatorname{mod}2$ holds if and only if $u$ and $v$ are
either both odd or both even (because both statements \textquotedblleft%
$u\equiv v\operatorname{mod}2$\textquotedblright\ and \textquotedblleft$u$ and
$v$ are either both odd or both even\textquotedblright\ hold). Hence, Exercise
\ref{exe.ent.even-odd.1} \textbf{(i)} is solved in Case 4.

We have now solved Exercise \ref{exe.ent.even-odd.1} \textbf{(i)} in all four
Cases 1, 2, 3 and 4. Hence, Exercise \ref{exe.ent.even-odd.1} \textbf{(i)} is solved.
\end{proof}
\end{fineprint}

\begin{exercise}
\label{exe.ent.even-odd-sumsq}\textbf{(a)} Prove that each even integer $u$
satisfies $u^{2}\equiv0\operatorname{mod}4$.

\textbf{(b)} Prove that each odd integer $u$ satisfies $u^{2}\equiv
1\operatorname{mod}4$.

\textbf{(c)} Prove that no two integers $x$ and $y$ satisfy $x^{2}+y^{2}%
\equiv3\operatorname{mod}4$.

\textbf{(d)} Prove that if $x$ and $y$ are two integers satisfying
$x^{2}+y^{2}\equiv2\operatorname{mod}4$, then $x$ and $y$ are both odd.
\end{exercise}

\begin{fineprint}
\begin{proof}
[Solution to Exercise \ref{exe.ent.even-odd-sumsq}.]\textbf{(a)} Let $u$ be an
even integer. Thus, $u$ is even. In other words, $u$ is divisible by $2$. In
other words, there exists some integer $c$ such that $u=2c$. Consider this $c$.

From $u=2c$, we obtain $u^{2}=\left(  2c\right)  ^{2}=4c^{2}$, which is
clearly divisible by $4$. So we have $4\mid u^{2}=u^{2}-0$. In other words,
$u^{2}\equiv0\operatorname{mod}4$. This solves Exercise
\ref{exe.ent.even-odd-sumsq} \textbf{(a)}.

\textbf{(b)} Let $u$ be an odd integer. Thus, $u$ is odd. Equivalently,
$u\equiv1\operatorname{mod}2$ (by Exercise \ref{exe.ent.even-odd.1}
\textbf{(d)}). In other words, $2\mid u-1$. In other words, there exists some
integer $c$ such that $u-1=2c$. Consider this $c$.

From $u-1=2c$, we obtain $u=2c+1$ and thus $u^{2}=\left(  2c+1\right)
^{2}=4c^{2}+4c+1$. Hence, $u^{2}-1=4c^{2}+4c=4\left(  c^{2}+c\right)  $, which
is clearly divisible by $4$. So we have $4\mid u^{2}-1$. In other words,
$u^{2}\equiv1\operatorname{mod}4$. This solves Exercise
\ref{exe.ent.even-odd-sumsq} \textbf{(b)}.

\textbf{(c)} Let $x$ and $y$ be two integers such that $x^{2}+y^{2}%
\equiv3\operatorname{mod}4$. We shall derive a contradiction.

Recall that an integer is always either even or odd. Thus, $x$ is either even
or odd. Similarly, $y$ is either even or odd. Thus, we are in one of the
following four cases:

\textit{Case 1:} The integer $x$ is even, and the integer $y$ is even.

\textit{Case 2:} The integer $x$ is even, and the integer $y$ is odd.

\textit{Case 3:} The integer $x$ is odd, and the integer $y$ is even.

\textit{Case 4:} The integer $x$ is odd, and the integer $y$ is odd.

Let us first consider Case 1. In this case, the integer $x$ is even, and the
integer $y$ is even. Hence, Exercise \ref{exe.ent.even-odd-sumsq} \textbf{(a)}
(applied to $u=x$) yields $x^{2}\equiv0\operatorname{mod}4$ (since $x$ is
even). Also, Exercise \ref{exe.ent.even-odd-sumsq} \textbf{(a)} (applied to
$u=y$) yields $y^{2}\equiv0\operatorname{mod}4$ (since $y$ is even). Thus,
$\underbrace{x^{2}}_{\equiv0\operatorname{mod}4}+\underbrace{y^{2}}%
_{\equiv0\operatorname{mod}4}\equiv0+0=0\operatorname{mod}4$. Hence, $0\equiv
x^{2}+y^{2}\equiv3\operatorname{mod}4$. But Exercise
\ref{exe.ent.quo-rem.mod=rem} (applied to $n=4$, $u=0$ and $v=3$) shows that
$0\equiv3\operatorname{mod}4$ if and only if $0\%4=3\%4$. Hence, $0\%4=3\%4$
(since $0\equiv3\operatorname{mod}4$). This contradicts the fact that
$0\%4=0\neq3=3\%4$. Hence, we have obtained a contradiction in Case 1.

Let us next consider Case 2. In this case, the integer $x$ is even, and the
integer $y$ is odd. Hence, Exercise \ref{exe.ent.even-odd-sumsq} \textbf{(a)}
(applied to $u=x$) yields $x^{2}\equiv0\operatorname{mod}4$ (since $x$ is
even). Also, Exercise \ref{exe.ent.even-odd-sumsq} \textbf{(b)} (applied to
$u=y$) yields $y^{2}\equiv1\operatorname{mod}4$ (since $y$ is odd). Thus,
$\underbrace{x^{2}}_{\equiv0\operatorname{mod}4}+\underbrace{y^{2}}%
_{\equiv1\operatorname{mod}4}\equiv0+1=1\operatorname{mod}4$. Hence, $1\equiv
x^{2}+y^{2}\equiv3\operatorname{mod}4$. But Exercise
\ref{exe.ent.quo-rem.mod=rem} (applied to $n=4$, $u=1$ and $v=3$) shows that
$1\equiv3\operatorname{mod}4$ if and only if $1\%4=3\%4$. Hence, $1\%4=3\%4$
(since $1\equiv3\operatorname{mod}4$). This contradicts the fact that
$1\%4=1\neq3=3\%4$. Hence, we have obtained a contradiction in Case 2.

The arguments in Cases 3 and 4 are completely analogous (in Case 3, we obtain
$x^{2}+y^{2}\equiv1\operatorname{mod}4$ again, whereas in Case 4 we obtain
$x^{2}+y^{2}\equiv2\operatorname{mod}4$). Thus, we have obtained a
contradiction in each of the four Cases 1, 2, 3 and 4. Hence, we always have a contradiction.

Now, forget that we fixed $x$ and $y$. We thus have obtained a contradiction
whenever $x$ and $y$ are two integers such that $x^{2}+y^{2}\equiv
3\operatorname{mod}4$. Thus, there are no such two integers. This solves
Exercise \ref{exe.ent.even-odd-sumsq} \textbf{(c)}.

\textbf{(d)} The solution of Exercise \ref{exe.ent.even-odd-sumsq}
\textbf{(d)} is very similar to the above solution of Exercise
\ref{exe.ent.even-odd-sumsq} \textbf{(c)} (indeed, we have to consider the
same four cases, but this time we don't get a contradiction in Case 4) and is
left to the reader.
\end{proof}
\end{fineprint}

Exercise \ref{exe.ent.even-odd-sumsq} \textbf{(c)} establishes our previous
experimental observation that an integer of the form $4k+3$ with integer $k$
(that is, an integer that is larger by $3$ than a multiple of $4$) can never
be written as a sum of two perfect squares.

\begin{center}
\textbf{2019-02-04 lecture}
\end{center}

\subsection{The floor function}

\begin{definition}
\label{def.ent.floor}Let $x$ be a real number. Then, $\left\lfloor
x\right\rfloor $ is defined to be the unique integer $n$ satisfying $n\leq
x<n+1$. This integer $\left\lfloor x\right\rfloor $ is called the
\textit{floor} of $x$, or the \textit{integer part} of $x$.
\end{definition}

\begin{remark}
\label{rmk.ent.floor}\textbf{(a)} Why is $\left\lfloor x\right\rfloor $
well-defined? I mean, why does the unique integer $n$ in Definition
\ref{def.ent.floor} exist, and why is it unique? This question is trickier
than it sounds and relies on the construction of real numbers. However, in the
case when $x$ is rational, the well-definedness of $\left\lfloor
x\right\rfloor $ follows from Proposition \ref{prop.ent.floor.quorem} below.

\textbf{(b)} What we call $\left\lfloor x\right\rfloor $ is typically called
$\left[  x\right]  $ in older books (such as \cite{NiZuMo91}). I suggest
avoiding the notation $\left[  x\right]  $ wherever possible; it has too many
different meanings (whereas $\left\lfloor x\right\rfloor $ almost always means
the floor of $x$).

\textbf{(c)} The map $\mathbb{R}\rightarrow\mathbb{Z},\ x\mapsto\left\lfloor
x\right\rfloor $ is called the \textit{floor function} or the \textit{greatest
integer function}.

There is also a \textit{ceiling function}, which sends each $x\in\mathbb{R}$
to the unique integer $n$ satisfying $n-1<x\leq n$; this latter integer is
called $\left\lceil x\right\rceil $. The two functions are connected by the
rule $\left\lceil x\right\rceil =-\left\lfloor -x\right\rfloor $ (for all
$x\in\mathbb{R}$).

The floor and the ceiling functions are some of the simplest examples of
discontinuous functions.

\textbf{(d)} Here are some examples of floors:%
\begin{align*}
\left\lfloor n\right\rfloor  &  =n\ \ \ \ \ \ \ \ \ \ \text{for every }%
n\in\mathbb{Z};\\
\left\lfloor 1.32\right\rfloor  &  =1;\ \ \ \ \ \ \ \ \ \ \left\lfloor
\pi\right\rfloor =3;\ \ \ \ \ \ \ \ \ \ \left\lfloor 0.98\right\rfloor =0;\\
\left\lfloor -2.3\right\rfloor  &  =-3;\ \ \ \ \ \ \ \ \ \ \left\lfloor
-0.4\right\rfloor =-1.
\end{align*}


\textbf{(e)} You might have the impression that $\left\lfloor x\right\rfloor $
is \textquotedblleft what remains from $x$ if the digits behind the comma are
removed\textquotedblright. This impression is highly imprecise. For one, it is
completely broken for negative $x$ (for example, $\left\lfloor
-2.3\right\rfloor $ is $-3$, not $-2$). But more importantly, the operation of
\textquotedblleft removing the digits behind the comma\textquotedblright\ from
a number is not well-defined; the periodic decimal representations
$0.999\ldots$ and $1.000\ldots$ belong to the same real number ($1$), but
removing their digits behind the comma leaves us with different integers.

\textbf{(f)} A related map is the map $\mathbb{R}\rightarrow\mathbb{Z}%
,\ x\mapsto\left\lfloor x+\dfrac{1}{2}\right\rfloor $. It sends each real $x$
to the integer that is closest to $x$, choosing the larger one in the case of
a tie. This is one of the many things that are commonly known as
\textquotedblleft rounding\textquotedblright\ a number.
\end{remark}

\begin{proposition}
\label{prop.ent.floor.quorem}Let $a$ and $b$ be integers such that $b>0$.
Then, $\left\lfloor \dfrac{a}{b}\right\rfloor $ is well-defined and equals
$a//b$.
\end{proposition}

\begin{proof}
[Proof of Proposition \ref{prop.ent.floor.quorem}.]This is a rather easy and
neat exercise. A full proof can be found in \cite[proof of Proposition
1.1.3]{floor}.
\end{proof}

\subsection{Common divisors, the Euclidean algorithm and the Bezout theorem}

\subsubsection{Divisors}

\begin{definition}
Let $b\in\mathbb{Z}$. The \textit{divisors} of $b$ are defined as the integers
that divide $b$.
\end{definition}

Be aware that some authors use a mildly different definition of
\textquotedblleft divisors\textquotedblright; namely, they additionally
require them to be positive. We don't make such a requirement.

For example, the divisors of $6$ are $-6,-3,-2,-1,1,2,3,6$. Of course, the
negative divisors of an integer $b$ are merely the reflections of the positive
divisors through the origin\footnote{\textquotedblleft Reflection through the
origin\textquotedblright\ is just a poetic way to say \textquotedblleft
negative\textquotedblright; i.e., the reflection of a number $a$ through the
origin is $-a$.} (this follows easily from Proposition \ref{prop.ent.div.1}
\textbf{(a)}); thus, the positive divisors are usually the only ones of interest.

Here are some basic properties of divisors:

\begin{proposition}
\label{prop.ent.divisors.find}\textbf{(a)} If $b\in\mathbb{Z}$, then $1$ and
$b$ are divisors of $b$.

\textbf{(b)} The divisors of $0$ are all the integers.

\textbf{(c)} Let $b\in\mathbb{Z}$ be nonzero. Then, all divisors of $b$ belong
to the set $\left\{  -\left\vert b\right\vert ,-\left\vert b\right\vert
+1,\ldots,\left\vert b\right\vert \right\}  \setminus\left\{  0\right\}  $.
\end{proposition}

\begin{proof}
[Proof of Proposition \ref{prop.ent.divisors.find}.]\textbf{(a)} Clearly,
$1\mid b$ (since $b=1b$), so that $1$ is a divisor of $b$. Also, $b\mid b$
(since $b=b\cdot1$), so that $b$ is a divisor of $b$.

\textbf{(b)} Each integer $a$ divides $0$ (since $0=a\cdot0$) and thus is a
divisor of $0$. This proves Proposition \ref{prop.ent.divisors.find}
\textbf{(b)}.

\textbf{(c)} Let $a$ be a divisor of $b$. Then, $a$ divides $b$. In other
words, $a\mid b$. Hence, Proposition \ref{prop.ent.div.1} \textbf{(b)} yields
$\left\vert a\right\vert \leq\left\vert b\right\vert $ (since $b\neq0$). But
$\left\vert a\right\vert \geq a$ (since $\left\vert x\right\vert \geq x$ for
each $x\in\mathbb{R}$), so that $a\leq\left\vert a\right\vert \leq\left\vert
b\right\vert $. Also, $\left\vert a\right\vert \geq-a$ (since $\left\vert
x\right\vert \geq-x$ for each $x\in\mathbb{R}$) and thus $-a\leq\left\vert
a\right\vert \leq\left\vert b\right\vert $, so that $a\geq-\left\vert
b\right\vert $. Combining this with $a\leq\left\vert b\right\vert $, we obtain
$-\left\vert b\right\vert \leq a\leq\left\vert b\right\vert $ and thus
$a\in\left\{  -\left\vert b\right\vert ,-\left\vert b\right\vert
+1,\ldots,\left\vert b\right\vert \right\}  $ (since $a$ is an integer).

From Example \ref{exa.ent.div.triv} \textbf{(c)}, we know that $0\mid b$ only
when $b=0$. Thus, we don't have $0\mid b$ (since $b\neq0$).

If we had $a=0$, then we would have $0=a\mid b$, which would contradict the
fact that we don't have $0\mid b$. Thus, we cannot have $a=0$. Hence, $a\neq
0$. Combining $a\in\left\{  -\left\vert b\right\vert ,-\left\vert b\right\vert
+1,\ldots,\left\vert b\right\vert \right\}  $ with $a\neq0$, we obtain
$a\in\left\{  -\left\vert b\right\vert ,-\left\vert b\right\vert
+1,\ldots,\left\vert b\right\vert \right\}  \setminus\left\{  0\right\}  $.

We have proven this for each divisor $a$ of $b$. Thus, we conclude that all
divisors of $b$ belong to the set $\left\{  -\left\vert b\right\vert
,-\left\vert b\right\vert +1,\ldots,\left\vert b\right\vert \right\}
\setminus\left\{  0\right\}  $. This proves Proposition
\ref{prop.ent.divisors.find} \textbf{(c)}.
\end{proof}

Thanks to Proposition \ref{prop.ent.divisors.find}, we have a method to find
all divisors of an integer $b$: If $b=0$, then Proposition
\ref{prop.ent.divisors.find} \textbf{(b)} directly yields the result;
otherwise, Proposition \ref{prop.ent.divisors.find} \textbf{(c)} shows that
there is only a finite set of numbers we have to check. When $b$ is large,
this is slow, but to some extent that is because the problem is
computationally hard (or at least suspected to be hard).

\subsubsection{Common divisors}

It is somewhat more interesting to consider the common divisors of two or more integers:

\begin{definition}
\label{def.ent.Div}Let $b_{1},b_{2},\ldots,b_{k}$ be integers. Then, the
\textit{common divisors} of $b_{1},b_{2},\ldots,b_{k}$ are defined to be the
integers $a$ that satisfy%
\begin{equation}
\left(  a\mid b_{i}\text{ for all }i\in\left\{  1,2,\ldots,k\right\}  \right)
\label{eq.def.ent.Div.cond}%
\end{equation}
(in other words, that divide all of the integers $b_{1},b_{2},\ldots,b_{k}$).
We let $\operatorname*{Div}\left(  b_{1},b_{2},\ldots,b_{k}\right)  $ denote
the set of these common divisors.
\end{definition}

Note that the concept of common divisors encompasses the concept of divisors:
The common divisors of a single integer $b$ are merely the divisors of $b$.
Thus, $\operatorname*{Div}\left(  b\right)  $ is the set of all divisors of
$b$ whenever $b\in\mathbb{Z}$. (Of course, speaking of \textquotedblleft
common divisors\textquotedblright\ of just one integer is like speaking of a
conspiracy of just one person. But the definition fits, and we algebraists
don't exclude cases just because they are ridiculous.)

(Also, the common divisors of an empty list of integers are all the integers,
because the requirement (\ref{eq.def.ent.Div.cond}) is vacuously true for
$k=0$. In other words, $\operatorname*{Div}\left(  {}\right)  =\mathbb{Z}$.)

Here are some more interesting examples of common divisors:

\begin{example}
\textbf{(a)} The common divisors of $6$ and $8$ are $-2,-1,1,2$. (In order to
see this, just observe that the divisors of $6$ are $-6,-3,-2,-1,1,2,3,6$,
whereas the divisors of $8$ are $-8,-4,-2,-1,1,2,4,8$; now you can find the
common divisors of $6$ and $8$ by taking the numbers common to these two
lists.) Thus,%
\[
\operatorname*{Div}\left(  6,8\right)  =\left\{  -2,-1,1,2\right\}  .
\]


\textbf{(b)} The common divisors of $6$ and $14$ are $-2,-1,1,2$ again. (In
order to see this, just observe that the divisors of $6$ are
$-6,-3,-2,-1,1,2,3,6$, whereas the divisors of $14$ are
$-14,-7,-2,-1,1,2,7,14$.)

\textbf{(c)} The common divisors of $6$, $10$ and $15$ are $-1$ and $1$. (In
order to see this, note that:

\begin{itemize}
\item The divisors of $6$ are $-6,-3,-2,-1,1,2,3,6$.

\item The divisors of $10$ are $-10,-5,-2,-1,1,2,5,10$.

\item The divisors of $15$ are $-15,-5,-3,-1,1,3,5,15$.
\end{itemize}

\noindent The only numbers common to these three lists are $-1$ and $1$.) However:

\begin{itemize}
\item The common divisors of $6$ and $10$ are $-2,-1,1,2$.

\item The common divisors of $6$ and $15$ are $-3,-1,1,3$.

\item The common divisors of $10$ and $15$ are $-5,-1,1,5$.
\end{itemize}

\noindent This illustrates the fact that three numbers can have pairwise
nontrivial common divisors (where \textquotedblleft
nontrivial\textquotedblright\ means \textquotedblleft distinct from $1$ and
$-1$\textquotedblright), but the only common divisors of all three of them may
still be just $1$ and $-1$.
\end{example}

\begin{proposition}
\label{prop.ent.Div.fin}Let $b_{1},b_{2},\ldots,b_{k}$ be finitely many
integers that are not all $0$. Then, the set $\operatorname*{Div}\left(
b_{1},b_{2},\ldots,b_{k}\right)  $ has a largest element, and this largest
element is a positive integer.
\end{proposition}

\begin{proof}
[Proof of Proposition \ref{prop.ent.Div.fin}.]The integer $1$ satisfies
$\left(  1\mid b_{i}\text{ for all }i\in\left\{  1,2,\ldots,k\right\}
\right)  $. Thus, $1$ is a common divisor of $b_{1},b_{2},\ldots,b_{k}$ (by
the definition of a \textquotedblleft common divisor\textquotedblright). In
other words, $1\in\operatorname*{Div}\left(  b_{1},b_{2},\ldots,b_{k}\right)
$ (by the definition of $\operatorname*{Div}\left(  b_{1},b_{2},\ldots
,b_{k}\right)  $). Hence, the set $\operatorname*{Div}\left(  b_{1}%
,b_{2},\ldots,b_{k}\right)  $ is nonempty.

Moreover, it is easy to see that the set $\operatorname*{Div}\left(
b_{1},b_{2},\ldots,b_{k}\right)  $ is finite.

\begin{fineprint}
[\textit{Proof:} We have assumed that $b_{1},b_{2},\ldots,b_{k}$ are not all
$0$. In other words, there exists a $j\in\left\{  1,2,\ldots,k\right\}  $ such
that $b_{j}$ is nonzero. Consider such a $j$.

Let $d\in\operatorname*{Div}\left(  b_{1},b_{2},\ldots,b_{k}\right)  $. Thus,
$d$ is a common divisor of $b_{1},b_{2},\ldots,b_{k}$ (by the definition of
$\operatorname*{Div}\left(  b_{1},b_{2},\ldots,b_{k}\right)  $). In other
words, $d\mid b_{i}$ for all $i\in\left\{  1,2,\ldots,k\right\}  $ (by the
definition of \textquotedblleft common divisor\textquotedblright). Applying
this to $i=j$, we obtain $d\mid b_{j}$. Hence, $d$ is a divisor of $b_{j}$.
But Proposition \ref{prop.ent.divisors.find} \textbf{(c)} (applied to
$b=b_{j}$) shows that all divisors of $b_{j}$ belong to the set $\left\{
-\left\vert b_{j}\right\vert ,-\left\vert b_{j}\right\vert +1,\ldots
,\left\vert b_{j}\right\vert \right\}  \setminus\left\{  0\right\}  $. Hence,
$d$ must belong to this set (since $d$ is a divisor of $b_{j}$). In other
words, $d\in\left\{  -\left\vert b_{j}\right\vert ,-\left\vert b_{j}%
\right\vert +1,\ldots,\left\vert b_{j}\right\vert \right\}  \setminus\left\{
0\right\}  $.

Now, forget that we fixed $d$. We thus have shown that $d\in\left\{
-\left\vert b_{j}\right\vert ,-\left\vert b_{j}\right\vert +1,\ldots
,\left\vert b_{j}\right\vert \right\}  \setminus\left\{  0\right\}  $ for each
$d\in\operatorname*{Div}\left(  b_{1},b_{2},\ldots,b_{k}\right)  $. In other
words,
\[
\operatorname*{Div}\left(  b_{1},b_{2},\ldots,b_{k}\right)  \subseteq\left\{
-\left\vert b_{j}\right\vert ,-\left\vert b_{j}\right\vert +1,\ldots
,\left\vert b_{j}\right\vert \right\}  \setminus\left\{  0\right\}  .
\]
Thus, the set $\operatorname*{Div}\left(  b_{1},b_{2},\ldots,b_{k}\right)  $
is finite (since the set $\left\{  -\left\vert b_{j}\right\vert ,-\left\vert
b_{j}\right\vert +1,\ldots,\left\vert b_{j}\right\vert \right\}
\setminus\left\{  0\right\}  $ is finite).]
\end{fineprint}

Now we know that the set $\operatorname*{Div}\left(  b_{1},b_{2},\ldots
,b_{k}\right)  $ is a nonempty finite set of integers. Thus, this set
$\operatorname*{Div}\left(  b_{1},b_{2},\ldots,b_{k}\right)  $ has a largest
element (since every nonempty finite set of integers has a largest element).
It remains to prove that this largest element is a positive integer.

Let $g$ be this largest element. Thus, we must prove that $g$ is a positive
integer. Clearly, $g$ is an integer (since all elements of
$\operatorname*{Div}\left(  b_{1},b_{2},\ldots,b_{k}\right)  $ are integers);
it thus remains to show that $g$ is positive.

The element $g$ is the largest element of the set $\operatorname*{Div}\left(
b_{1},b_{2},\ldots,b_{k}\right)  $, and thus is $\geq$ to every element of
this set. In other words, $g\geq x$ for each $x\in\operatorname*{Div}\left(
b_{1},b_{2},\ldots,b_{k}\right)  $. Applying this to $x=1$, we obtain $g\geq1$
(since $1\in\operatorname*{Div}\left(  b_{1},b_{2},\ldots,b_{k}\right)  $).
Hence, $g$ is positive. This completes the proof of Proposition
\ref{prop.ent.Div.fin}.
\end{proof}

\subsubsection{Greatest common divisors}

Proposition \ref{prop.ent.Div.fin} allows us to make a crucial definition:

\begin{definition}
\label{def.ent.gcd.gcd}Let $b_{1},b_{2},\ldots,b_{k}$ be finitely many
integers. The \textit{greatest common divisor} of $b_{1},b_{2},\ldots,b_{k}$
is defined as follows:

\begin{itemize}
\item If $b_{1},b_{2},\ldots,b_{k}$ are not all $0$, then it is defined as the
largest element of the set $\operatorname*{Div}\left(  b_{1},b_{2}%
,\ldots,b_{k}\right)  $. This largest element is well-defined (by Proposition
\ref{prop.ent.Div.fin}), and is a positive integer (by Proposition
\ref{prop.ent.Div.fin} again).

\item If $b_{1},b_{2},\ldots,b_{k}$ are all $0$, then it is defined to be $0$.
(This is a slight abuse of the word \textquotedblleft greatest common
divisor\textquotedblright, because $0$ is not actually the greatest among the
common divisors of $b_{1},b_{2},\ldots,b_{k}$ in this case. In fact, when
$b_{1},b_{2},\ldots,b_{k}$ are all $0$, \textbf{every} integer is a common
divisor of $b_{1},b_{2},\ldots,b_{k}$, so that there is no greatest among
these common divisors, because there is no \textquotedblleft greatest
integer\textquotedblright. Nevertheless, defining the greatest common divisor
of $b_{1},b_{2},\ldots,b_{k}$ to be $0$ in this case will prove to be a good
decision, as it will greatly reduce the number of exceptions in our results.)
\end{itemize}

Thus, in either case, this greatest common divisor is a nonnegative integer.
We denote it by $\gcd\left(  b_{1},b_{2},\ldots,b_{k}\right)  $.

We shall also use the word \textquotedblleft\textit{gcd}\textquotedblright\ as
shorthand for \textquotedblleft greatest common divisor\textquotedblright.
\end{definition}

The greatest common divisors you will most commonly see are those of two
integers. Indeed, any other gcd can be rewritten in terms of these: for
example,%
\[
\gcd\left(  a,b,c,d,e\right)  =\gcd\left(  a,\gcd\left(  b,\gcd\left(
c,\gcd\left(  d,e\right)  \right)  \right)  \right)
\]
for all $a,b,c,d,e\in\mathbb{Z}$. This is, in fact, a consequence of
Proposition \ref{thm.ent.gcd.uniprop-mul} \textbf{(d)} (which we will prove
later), applied several times.

First, let us observe several properties of greatest common divisors:

\begin{proposition}
\label{prop.ent.gcd.props1}\textbf{(a)} We have $\gcd\left(  a,0\right)
=\gcd\left(  a\right)  =\left\vert a\right\vert $ for all $a\in\mathbb{Z}$.

\textbf{(b)} We have $\gcd\left(  a,b\right)  =\gcd\left(  b,a\right)  $ for
all $a,b\in\mathbb{Z}$.

\textbf{(c)} We have $\gcd\left(  a,ua+b\right)  =\gcd\left(  a,b\right)  $
for all $a,b,u\in\mathbb{Z}$.

\textbf{(d)} If $a,b,c\in\mathbb{Z}$ satisfy $b\equiv c\operatorname{mod}a$,
then $\gcd\left(  a,b\right)  =\gcd\left(  a,c\right)  $.

\textbf{(e)} If $a,b\in\mathbb{Z}$ are such that $a$ is positive, then
$\gcd\left(  a,b\right)  =\gcd\left(  a,b\%a\right)  $.

\textbf{(f)} We have $\gcd\left(  a,b\right)  \mid a$ and $\gcd\left(
a,b\right)  \mid b$ for all $a,b\in\mathbb{Z}$.

\textbf{(g)} We have $\gcd\left(  -a,b\right)  =\gcd\left(  a,b\right)  $ for
all $a,b\in\mathbb{Z}$.

\textbf{(h)} We have $\gcd\left(  a,-b\right)  =\gcd\left(  a,b\right)  $ for
all $a,b\in\mathbb{Z}$.

\textbf{(i)} If $a,b\in\mathbb{Z}$ satisfy $a\mid b$, then $\gcd\left(
a,b\right)  =\left\vert a\right\vert $.

\textbf{(j)} The greatest common divisor of the empty list of integers is
$\gcd\left(  {}\right)  =0$.
\end{proposition}

Proposition \ref{prop.ent.gcd.props1} is not difficult and we could start
proving it right away. However, such a proof would require some annoying case
distinctions due to the special treatment that the \textquotedblleft%
$b_{1},b_{2},\ldots,b_{k}$ are all $0$\textquotedblright\ case required in
Definition \ref{def.ent.gcd.gcd}. Fortunately, we can circumnavigate these
annoyances by stating a simple rule how the gcd of $k$ integers $b_{1}%
,b_{2},\ldots,b_{k}$ can be computed from their set of common divisors
(including the case when $b_{1},b_{2},\ldots,b_{k}$ are all $0$):

\begin{lemma}
\label{lem.ent.gcd.through-Div}Let $b_{1},b_{2},\ldots,b_{k}$ be finitely many
integers. Then,%
\[
\gcd\left(  b_{1},b_{2},\ldots,b_{k}\right)  =%
\begin{cases}
\max\left(  \operatorname*{Div}\left(  b_{1},b_{2},\ldots,b_{k}\right)
\right)  , & \text{if }0\notin\operatorname*{Div}\left(  b_{1},b_{2}%
,\ldots,b_{k}\right)  ;\\
0, & \text{if }0\in\operatorname*{Div}\left(  b_{1},b_{2},\ldots,b_{k}\right)
.
\end{cases}
\]
(Here, $\max S$ denotes the largest element of a set $S$ of integers, whenever
this largest element exists.)
\end{lemma}

\begin{proof}
[Proof of Lemma \ref{lem.ent.gcd.through-Div}.]We are in one of the following
two cases:

\textit{Case 1:} The integers $b_{1},b_{2},\ldots,b_{k}$ are not all $0$.

\textit{Case 2:} The integers $b_{1},b_{2},\ldots,b_{k}$ are all $0$.

Let us consider Case 1 first. In this case, the integers $b_{1},b_{2}%
,\ldots,b_{k}$ are not all $0$. Hence, $\gcd\left(  b_{1},b_{2},\ldots
,b_{k}\right)  $ is defined as the largest element of the set
$\operatorname*{Div}\left(  b_{1},b_{2},\ldots,b_{k}\right)  $ (by Definition
\ref{def.ent.gcd.gcd}). In other words,
\begin{equation}
\gcd\left(  b_{1},b_{2},\ldots,b_{k}\right)  =\max\left(  \operatorname*{Div}%
\left(  b_{1},b_{2},\ldots,b_{k}\right)  \right)  .
\label{pf.lem.ent.gcd.through-Div.c1.1}%
\end{equation}


On the other hand, $0\notin\operatorname*{Div}\left(  b_{1},b_{2},\ldots
,b_{k}\right)  $\ \ \ \ \footnote{\textit{Proof.} Assume the contrary. Thus,
$0\in\operatorname*{Div}\left(  b_{1},b_{2},\ldots,b_{k}\right)  $. In other
words, $0$ is a common divisor of $b_{1},b_{2},\ldots,b_{k}$ (by the
definition of $\operatorname*{Div}\left(  b_{1},b_{2},\ldots,b_{k}\right)  $).
In other words, $0\mid b_{i}$ for all $i\in\left\{  1,2,\ldots,k\right\}  $
(by the definition of \textquotedblleft common divisor\textquotedblright).
Thus, for all $i\in\left\{  1,2,\ldots,k\right\}  $, we have $b_{i}=0$ (since
$0\mid b_{i}$, so that $b_{i}=0c$ for some integer $c$; but this yields
$b_{i}=0c=0$). In other words, $b_{1},b_{2},\ldots,b_{k}$ are all $0$. But
this contradicts the fact that $b_{1},b_{2},\ldots,b_{k}$ are not all $0$.
This contradiction shows that our assumption was false, qed.}. Hence,%
\[%
\begin{cases}
\max\left(  \operatorname*{Div}\left(  b_{1},b_{2},\ldots,b_{k}\right)
\right)  , & \text{if }0\notin\operatorname*{Div}\left(  b_{1},b_{2}%
,\ldots,b_{k}\right)  ;\\
0, & \text{if }0\in\operatorname*{Div}\left(  b_{1},b_{2},\ldots,b_{k}\right)
\end{cases}
=\max\left(  \operatorname*{Div}\left(  b_{1},b_{2},\ldots,b_{k}\right)
\right)  .
\]
Comparing this with (\ref{pf.lem.ent.gcd.through-Div.c1.1}), we obtain%
\[
\gcd\left(  b_{1},b_{2},\ldots,b_{k}\right)  =%
\begin{cases}
\max\left(  \operatorname*{Div}\left(  b_{1},b_{2},\ldots,b_{k}\right)
\right)  , & \text{if }0\notin\operatorname*{Div}\left(  b_{1},b_{2}%
,\ldots,b_{k}\right)  ;\\
0, & \text{if }0\in\operatorname*{Div}\left(  b_{1},b_{2},\ldots,b_{k}\right)
.
\end{cases}
\]
Hence, Lemma \ref{lem.ent.gcd.through-Div} is proven in Case 1.

Let us now consider Case 2. In this case, the integers $b_{1},b_{2}%
,\ldots,b_{k}$ are all $0$. Hence, $\gcd\left(  b_{1},b_{2},\ldots
,b_{k}\right)  $ is defined as $0$ (by Definition \ref{def.ent.gcd.gcd}). In
other words,
\begin{equation}
\gcd\left(  b_{1},b_{2},\ldots,b_{k}\right)  =0.
\label{pf.lem.ent.gcd.through-Div.c2.1}%
\end{equation}


On the other hand, $0\in\operatorname*{Div}\left(  b_{1},b_{2},\ldots
,b_{k}\right)  $\ \ \ \ \footnote{\textit{Proof.} The integers $b_{1}%
,b_{2},\ldots,b_{k}$ are all $0$. In other words, $b_{i}=0$ for all
$i\in\left\{  1,2,\ldots,k\right\}  $. Hence, $0\mid b_{i}$ for all
$i\in\left\{  1,2,\ldots,k\right\}  $ (since each $i\in\left\{  1,2,\ldots
,k\right\}  $ satisfies $b_{i}=0=0\cdot0$). In other words, $0$ is a common
divisor of $b_{1},b_{2},\ldots,b_{k}$ (by the definition of \textquotedblleft
common divisor\textquotedblright). In other words, $0\in\operatorname*{Div}%
\left(  b_{1},b_{2},\ldots,b_{k}\right)  $ (by the definition of
$\operatorname*{Div}\left(  b_{1},b_{2},\ldots,b_{k}\right)  $).}. Hence,%
\[%
\begin{cases}
\max\left(  \operatorname*{Div}\left(  b_{1},b_{2},\ldots,b_{k}\right)
\right)  , & \text{if }0\notin\operatorname*{Div}\left(  b_{1},b_{2}%
,\ldots,b_{k}\right)  ;\\
0, & \text{if }0\in\operatorname*{Div}\left(  b_{1},b_{2},\ldots,b_{k}\right)
\end{cases}
=0.
\]
Comparing this with (\ref{pf.lem.ent.gcd.through-Div.c2.1}), we obtain%
\[
\gcd\left(  b_{1},b_{2},\ldots,b_{k}\right)  =%
\begin{cases}
\max\left(  \operatorname*{Div}\left(  b_{1},b_{2},\ldots,b_{k}\right)
\right)  , & \text{if }0\notin\operatorname*{Div}\left(  b_{1},b_{2}%
,\ldots,b_{k}\right)  ;\\
0, & \text{if }0\in\operatorname*{Div}\left(  b_{1},b_{2},\ldots,b_{k}\right)
.
\end{cases}
\]
Hence, Lemma \ref{lem.ent.gcd.through-Div} is proven in Case 2.

We have now proven Lemma \ref{lem.ent.gcd.through-Div} in both Cases 1 and 2.
Thus, Lemma \ref{lem.ent.gcd.through-Div} always holds.
\end{proof}

A corollary of Lemma \ref{lem.ent.gcd.through-Div} is the following:

\begin{lemma}
\label{lem.ent.gcd.through-Divc}Let $b_{1},b_{2},\ldots,b_{k}$ be finitely
many integers. Let $c_{1},c_{2},\ldots,c_{\ell}$ be finitely many integers. If%
\[
\operatorname*{Div}\left(  b_{1},b_{2},\ldots,b_{k}\right)
=\operatorname*{Div}\left(  c_{1},c_{2},\ldots,c_{\ell}\right)  ,
\]
then%
\[
\gcd\left(  b_{1},b_{2},\ldots,b_{k}\right)  =\gcd\left(  c_{1},c_{2}%
,\ldots,c_{\ell}\right)  .
\]

\end{lemma}

\begin{proof}
[Proof of Lemma \ref{lem.ent.gcd.through-Div}.]Assume that
$\operatorname*{Div}\left(  b_{1},b_{2},\ldots,b_{k}\right)
=\operatorname*{Div}\left(  c_{1},c_{2},\ldots,c_{\ell}\right)  $. Lemma
\ref{lem.ent.gcd.through-Div} yields%
\begin{align*}
\gcd\left(  b_{1},b_{2},\ldots,b_{k}\right)   &  =%
\begin{cases}
\max\left(  \operatorname*{Div}\left(  b_{1},b_{2},\ldots,b_{k}\right)
\right)  , & \text{if }0\notin\operatorname*{Div}\left(  b_{1},b_{2}%
,\ldots,b_{k}\right)  ;\\
0, & \text{if }0\in\operatorname*{Div}\left(  b_{1},b_{2},\ldots,b_{k}\right)
\end{cases}
\\
&  =%
\begin{cases}
\max\left(  \operatorname*{Div}\left(  c_{1},c_{2},\ldots,c_{\ell}\right)
\right)  , & \text{if }0\notin\operatorname*{Div}\left(  c_{1},c_{2}%
,\ldots,c_{\ell}\right)  ;\\
0, & \text{if }0\in\operatorname*{Div}\left(  c_{1},c_{2},\ldots,c_{\ell
}\right)
\end{cases}
\end{align*}
(since $\operatorname*{Div}\left(  b_{1},b_{2},\ldots,b_{k}\right)
=\operatorname*{Div}\left(  c_{1},c_{2},\ldots,c_{\ell}\right)  $). But Lemma
\ref{lem.ent.gcd.through-Div} (applied to $c_{1},c_{2},\ldots,c_{\ell}$
instead of $b_{1},b_{2},\ldots,b_{k}$) yields%
\[
\gcd\left(  c_{1},c_{2},\ldots,c_{\ell}\right)  =%
\begin{cases}
\max\left(  \operatorname*{Div}\left(  c_{1},c_{2},\ldots,c_{\ell}\right)
\right)  , & \text{if }0\notin\operatorname*{Div}\left(  c_{1},c_{2}%
,\ldots,c_{\ell}\right)  ;\\
0, & \text{if }0\in\operatorname*{Div}\left(  c_{1},c_{2},\ldots,c_{\ell
}\right)  .
\end{cases}
\]
Comparing these two equalities, we obtain $\gcd\left(  b_{1},b_{2}%
,\ldots,b_{k}\right)  =\gcd\left(  c_{1},c_{2},\ldots,c_{\ell}\right)  $. This
proves Lemma \ref{lem.ent.gcd.through-Div}.
\end{proof}

\begin{proof}
[Proof of Proposition \ref{prop.ent.gcd.props1}.]\textbf{(a)} Here is a sketch
of the proof: The number $0$ is a \textquotedblleft joker\textquotedblright%
\ when it comes to common divisors: For example, if $a\in\mathbb{Z}$, then the
common divisors of $a$ and $0$ are the same as the divisors of $a$, because
every integer divides $0$. Thus, if $a\in\mathbb{Z}$ is nonzero, then the
greatest common divisor of $a$ and $0$ is the greatest divisor of $a$, which
is $\left\vert a\right\vert $ (an easy consequence of Proposition
\ref{prop.ent.divisors.find} \textbf{(b)}).

For the sake of completeness, let us give a detailed proof of Proposition
\ref{prop.ent.gcd.props1} \textbf{(a)}:

\begin{fineprint}
Let $a\in\mathbb{Z}$. Definition \ref{def.ent.gcd.gcd} (specifically, its case
when $b_{1},b_{2},\ldots,b_{k}$ are all $0$) shows that $\gcd\left(
0,0\right)  =0$ and $\gcd\left(  0\right)  =0$. Combining this with
$\left\vert 0\right\vert =0$, we obtain $\gcd\left(  0,0\right)  =\gcd\left(
0\right)  =\left\vert 0\right\vert $. In other words, Proposition
\ref{prop.ent.gcd.props1} \textbf{(a)} holds if $a=0$. Thus, for the rest of
this proof, we WLOG assume that $a\neq0$. Hence, the two integers $a,0$ are
not all zero. Thus, $\gcd\left(  a,0\right)  $ is defined to be the largest
element of the set $\operatorname*{Div}\left(  a,0\right)  $ (by Definition
\ref{def.ent.gcd.gcd}). Likewise, $\gcd\left(  a\right)  $ is the largest
element of the set $\operatorname*{Div}\left(  a\right)  $.

We shall now prove that $\operatorname*{Div}\left(  a,0\right)
=\operatorname*{Div}\left(  a\right)  $. Indeed, for any integer $x$, we have
the following chain of equivalences:%
\begin{align*}
&  \ \left(  x\in\operatorname*{Div}\left(  a,0\right)  \right) \\
&  \Longleftrightarrow\ \left(  x\text{ is a common divisor of }a\text{ and
}0\right)  \ \ \ \ \ \ \ \ \ \ \left(  \text{by the definition of
}\operatorname*{Div}\left(  a,0\right)  \right) \\
&  \Longleftrightarrow\ \left(  x\mid a\text{ and }x\mid0\right)
\ \ \ \ \ \ \ \ \ \ \left(  \text{by the definition of a \textquotedblleft
common divisor\textquotedblright}\right) \\
&  \Longleftrightarrow\ \left(  x\mid a\right)  \ \ \ \ \ \ \ \ \ \ \left(
\text{since }x\mid0\text{ always holds (since }0=x\cdot0\text{)}\right) \\
&  \Longleftrightarrow\ \left(  x\text{ is a common divisor of }a\right)
\ \ \ \ \ \ \ \ \ \ \left(  \text{by the definition of a \textquotedblleft
common divisor\textquotedblright}\right) \\
&  \Longleftrightarrow\ \left(  x\in\operatorname*{Div}\left(  a\right)
\right)  \ \ \ \ \ \ \ \ \ \ \left(  \text{by the definition of }%
\operatorname*{Div}\left(  a\right)  \right)  .
\end{align*}
In other words, an integer belongs to $\operatorname*{Div}\left(  a,0\right)
$ if and only if it belongs to $\operatorname*{Div}\left(  a\right)  $. Thus,
$\operatorname*{Div}\left(  a,0\right)  =\operatorname*{Div}\left(  a\right)
$ (since both $\operatorname*{Div}\left(  a,0\right)  $ and
$\operatorname*{Div}\left(  a\right)  $ are sets of integers). Thus, Lemma
\ref{lem.ent.gcd.through-Div} (applied to $\left(  a,0\right)  $ and $\left(
a\right)  $ instead of $\left(  b_{1},b_{2},\ldots,b_{k}\right)  $ and
$\left(  c_{1},c_{2},\ldots,c_{\ell}\right)  $) yields $\gcd\left(
a,0\right)  =\gcd\left(  a\right)  $.

For any integer $x$, we have the following chain of equivalences:
\begin{align*}
&  \ \left(  x\in\operatorname*{Div}\left(  a\right)  \right) \\
&  \Longleftrightarrow\ \left(  x\text{ is a common divisor of }a\right)
\ \ \ \ \ \ \ \ \ \ \left(  \text{by the definition of }\operatorname*{Div}%
\left(  a\right)  \right) \\
&  \Longleftrightarrow\ \left(  x\mid a\right)  \ \ \ \ \ \ \ \ \ \ \left(
\text{by the definition of a \textquotedblleft common
divisor\textquotedblright}\right) \\
&  \Longleftrightarrow\ \left(  x\text{ is a divisor of }a\right)  .
\end{align*}
Thus, $\operatorname*{Div}\left(  a\right)  $ is the set of all divisors of
$a$.

Exercise \ref{exe.ent.div.aabs} \textbf{(b)} yields $\left\vert a\right\vert
\mid a$. In other words, $\left\vert a\right\vert $ is a divisor of $a$.

Moreover, $a$ is nonzero (since $a\neq0$). Hence, Proposition
\ref{prop.ent.divisors.find} \textbf{(b)} (applied to $b=a$) shows that all
divisors of $a$ belong to the set $\left\{  -\left\vert a\right\vert
,-\left\vert a\right\vert +1,\ldots,\left\vert a\right\vert \right\}
\setminus\left\{  0\right\}  $. Hence, they belong to the set $\left\{
-\left\vert a\right\vert ,-\left\vert a\right\vert +1,\ldots,\left\vert
a\right\vert \right\}  $, and thus are $\leq\left\vert a\right\vert $.

Recall that $\left\vert a\right\vert $ is a divisor of $a$. Since we also know
that all divisors of $a$ are $\leq\left\vert a\right\vert $, we can thus
conclude that $\left\vert a\right\vert $ is the \textbf{largest} divisor of
$a$. In other words, $\left\vert a\right\vert $ is the largest element of the
set $\operatorname*{Div}\left(  a\right)  $ (since $\operatorname*{Div}\left(
a\right)  $ is the set of all divisors of $a$). In other words, $\left\vert
a\right\vert $ is $\gcd\left(  a\right)  $ (since $\gcd\left(  a\right)  $ is
the largest element of the set $\operatorname*{Div}\left(  a\right)  $). Thus,
$\gcd\left(  a\right)  =\left\vert a\right\vert $. Combining this with
$\gcd\left(  a,0\right)  =\gcd\left(  a\right)  $, this yields $\gcd\left(
a,0\right)  =\gcd\left(  a\right)  =\left\vert a\right\vert $. Thus,
Proposition \ref{prop.ent.gcd.props1} \textbf{(a)} is finally proven.
\end{fineprint}

\textbf{(b)} For any integer $x$, we have the following chain of equivalences:%
\begin{align*}
&  \ \left(  x\in\operatorname*{Div}\left(  a,b\right)  \right) \\
&  \Longleftrightarrow\ \left(  x\text{ is a common divisor of }a\text{ and
}b\right)  \ \ \ \ \ \ \ \ \ \ \left(  \text{by the definition of
}\operatorname*{Div}\left(  a,b\right)  \right) \\
&  \Longleftrightarrow\ \left(  x\mid a\text{ and }x\mid b\right)
\ \ \ \ \ \ \ \ \ \ \left(  \text{by the definition of a \textquotedblleft
common divisor\textquotedblright}\right) \\
&  \Longleftrightarrow\ \left(  x\mid b\text{ and }x\mid a\right) \\
&  \Longleftrightarrow\ \left(  x\text{ is a common divisor of }b\text{ and
}a\right) \\
&  \ \ \ \ \ \ \ \ \ \ \left(  \text{by the definition of a \textquotedblleft
common divisor\textquotedblright}\right) \\
&  \Longleftrightarrow\ \left(  x\in\operatorname*{Div}\left(  b,a\right)
\right)  \ \ \ \ \ \ \ \ \ \ \left(  \text{by the definition of }%
\operatorname*{Div}\left(  b,a\right)  \right)  .
\end{align*}
In other words, an integer belongs to $\operatorname*{Div}\left(  a,b\right)
$ if and only if it belongs to $\operatorname*{Div}\left(  b,a\right)  $.
Thus, $\operatorname*{Div}\left(  a,b\right)  =\operatorname*{Div}\left(
b,a\right)  $ (since both $\operatorname*{Div}\left(  a,b\right)  $ and
$\operatorname*{Div}\left(  b,a\right)  $ are sets of integers). Thus, Lemma
\ref{lem.ent.gcd.through-Div} (applied to $\left(  a,b\right)  $ and $\left(
b,a\right)  $ instead of $\left(  b_{1},b_{2},\ldots,b_{k}\right)  $ and
$\left(  c_{1},c_{2},\ldots,c_{\ell}\right)  $) yields $\gcd\left(
a,b\right)  =\gcd\left(  b,a\right)  $. This proves Proposition
\ref{prop.ent.gcd.props1} \textbf{(b)}.

Let us prove part \textbf{(d)} now, and then derive part \textbf{(c)} from it.

\textbf{(d)} Let $a,b,c\in\mathbb{Z}$ satisfy $b\equiv c\operatorname{mod}a$.
We must prove that $\gcd\left(  a,b\right)  =\gcd\left(  a,c\right)  $. To do
so, we shall first prove that $\operatorname*{Div}\left(  a,b\right)
=\operatorname*{Div}\left(  a,c\right)  $.

From $b\equiv c\operatorname{mod}a$, we obtain $c\equiv b\operatorname{mod}a$
(by Proposition \ref{prop.ent.mod.basics} \textbf{(c)}). Hence, our situation
is symmetric with respect to $b$ and $c$.

We shall now show that $\operatorname*{Div}\left(  a,b\right)  \subseteq
\operatorname*{Div}\left(  a,c\right)  $. Indeed, let $x\in\operatorname*{Div}%
\left(  a,b\right)  $. Then, $x$ is a common divisor of $a$ and $b$ (by the
definition of $\operatorname*{Div}\left(  a,b\right)  $). In other words,
$x\mid a$ and $x\mid b$ (by the definition of a \textquotedblleft common
divisor\textquotedblright). From $x\mid b$, we obtain $b\equiv
0\operatorname{mod}x$. But from $x\mid a$ and $c\equiv b\operatorname{mod}a$,
we obtain $c\equiv b\operatorname{mod}x$ (by Proposition
\ref{prop.ent.mod.basics} \textbf{(e)}, applied to $a$, $x$, $c$ and $b$
instead of $n$, $m$, $c$ and $b$). Thus, $c\equiv b\equiv0\operatorname{mod}%
x$, so that $x\mid c$. Combining $x\mid a$ and $x\mid c$, we see that $x$ is a
common divisor of $a$ and $c$ (by the definition of a \textquotedblleft common
divisor\textquotedblright). In other words, $x\in\operatorname*{Div}\left(
a,c\right)  $ (by the definition of $\operatorname*{Div}\left(  a,c\right)  $).

Now, forget that we fixed $x$. We thus have proven that $x\in
\operatorname*{Div}\left(  a,c\right)  $ for each $x\in\operatorname*{Div}%
\left(  a,b\right)  $. In other words, $\operatorname*{Div}\left(  a,b\right)
\subseteq\operatorname*{Div}\left(  a,c\right)  $.

The same argument (but with the roles of $b$ and $c$ swapped) shows that
$\operatorname*{Div}\left(  a,c\right)  \subseteq\operatorname*{Div}\left(
a,b\right)  $ (since our situation is symmetric with respect to $b$ and $c$).
Combining this with $\operatorname*{Div}\left(  a,b\right)  \subseteq
\operatorname*{Div}\left(  a,c\right)  $, we obtain $\operatorname*{Div}%
\left(  a,b\right)  =\operatorname*{Div}\left(  a,c\right)  $. Thus, Lemma
\ref{lem.ent.gcd.through-Div} (applied to $\left(  a,b\right)  $ and $\left(
a,c\right)  $ instead of $\left(  b_{1},b_{2},\ldots,b_{k}\right)  $ and
$\left(  c_{1},c_{2},\ldots,c_{\ell}\right)  $) yields $\gcd\left(
a,b\right)  =\gcd\left(  a,c\right)  $. This proves Proposition
\ref{prop.ent.gcd.props1} \textbf{(d)}.

\textbf{(c)} Let $a,b,u\in\mathbb{Z}$. Then, $ua+b\equiv b\operatorname{mod}a$
(since $\left(  ua+b\right)  -b=ua$ is clearly divisible by $a$). Thus,
Proposition \ref{prop.ent.gcd.props1} \textbf{(d)} (applied to $ua+b$ and $b$
instead of $b$ and $c$) yields $\gcd\left(  a,ua+b\right)  =\gcd\left(
a,b\right)  $. This proves Proposition \ref{prop.ent.gcd.props1} \textbf{(c)}.

\textbf{(e)} Let $a,b\in\mathbb{Z}$ be such that $a$ is positive. Then,
$b\%a\equiv b\operatorname{mod}a$ (by Corollary \ref{cor.ent.quo-rem.remmod}
\textbf{(a)}, applied to $a$ and $b$ instead of $n$ and $u$), thus $b\equiv
b\%a\operatorname{mod}a$. Hence, $\gcd\left(  a,b\right)  =\gcd\left(
a,b\%a\right)  $ (by Proposition \ref{prop.ent.gcd.props1} \textbf{(d)},
applied to $c=b\%a$). This proves Proposition \ref{prop.ent.gcd.props1}
\textbf{(e)}.

\textbf{(f)} Let $a,b\in\mathbb{Z}$. We must prove that $\gcd\left(
a,b\right)  \mid a$ and $\gcd\left(  a,b\right)  \mid b$.

If the two integers $a,b$ are all $0$, then this is
obvious\footnote{\textit{Proof.} Assume that $a,b$ are all $0$. Then,
$a=0=\gcd\left(  a,b\right)  \cdot0$, so that $\gcd\left(  a,b\right)  \mid
a$; similarly, $\gcd\left(  a,b\right)  \mid b$. Hence, we have proven that
$\gcd\left(  a,b\right)  \mid a$ and $\gcd\left(  a,b\right)  \mid b$ if the
integers $a,b$ are all $0$.}. Hence, for the rest of this proof, we WLOG
assume that $a,b$ are not all $0$. Thus, $\gcd\left(  a,b\right)  $ is defined
to be the largest element of the set $\operatorname*{Div}\left(  a,b\right)  $
(by Definition \ref{def.ent.gcd.gcd}). Hence, $\gcd\left(  a,b\right)  $ is an
element of this set $\operatorname*{Div}\left(  a,b\right)  $. In other words,
$\gcd\left(  a,b\right)  $ is a common divisor of $a$ and $b$ (by the
definition of $\operatorname*{Div}\left(  a,b\right)  $). In other words,
$\gcd\left(  a,b\right)  \mid a$ and $\gcd\left(  a,b\right)  \mid b$. This
proves Proposition \ref{prop.ent.gcd.props1} \textbf{(f)}.

\textbf{(g)} Let $a,b\in\mathbb{Z}$. We must prove that $\gcd\left(
-a,b\right)  =\gcd\left(  a,b\right)  $. Again, we shall achieve this via
showing that $\operatorname*{Div}\left(  -a,b\right)  =\operatorname*{Div}%
\left(  a,b\right)  $.

First, we will show that $\operatorname*{Div}\left(  a,b\right)
\subseteq\operatorname*{Div}\left(  -a,b\right)  $. Indeed, let $x\in
\operatorname*{Div}\left(  a,b\right)  $. Then, $x$ is a common divisor of $a$
and $b$ (by the definition of $\operatorname*{Div}\left(  a,b\right)  $). In
other words, $x\mid a$ and $x\mid b$ (by the definition of a \textquotedblleft
common divisor\textquotedblright). We have $a\mid-a$ (since $-a=a\cdot\left(
-1\right)  $). Thus, $x\mid a\mid-a$. Combining $x\mid-a$ and $x\mid b$, we
see that $x$ is a common divisor of $-a$ and $b$ (by the definition of a
\textquotedblleft common divisor\textquotedblright). In other words,
$x\in\operatorname*{Div}\left(  -a,b\right)  $ (by the definition of
$\operatorname*{Div}\left(  -a,b\right)  $).

Now, forget that we fixed $x$. We thus have proven that $x\in
\operatorname*{Div}\left(  -a,b\right)  $ for each $x\in\operatorname*{Div}%
\left(  a,b\right)  $. In other words, $\operatorname*{Div}\left(  a,b\right)
\subseteq\operatorname*{Div}\left(  -a,b\right)  $.

The same argument (but applied to $-a$ instead of $a$) shows that
$\operatorname*{Div}\left(  -a,b\right)  \subseteq\operatorname*{Div}\left(
-\left(  -a\right)  ,b\right)  $. Since $-\left(  -a\right)  =a$, this
rewrites as $\operatorname*{Div}\left(  -a,b\right)  \subseteq
\operatorname*{Div}\left(  a,b\right)  $. Combining this with
$\operatorname*{Div}\left(  a,b\right)  \subseteq\operatorname*{Div}\left(
-a,b\right)  $, we obtain $\operatorname*{Div}\left(  -a,b\right)
=\operatorname*{Div}\left(  a,b\right)  $. Thus, Lemma
\ref{lem.ent.gcd.through-Div} (applied to $\left(  -a,b\right)  $ and $\left(
a,b\right)  $ instead of $\left(  b_{1},b_{2},\ldots,b_{k}\right)  $ and
$\left(  c_{1},c_{2},\ldots,c_{\ell}\right)  $) yields $\gcd\left(
-a,b\right)  =\gcd\left(  a,b\right)  $. This proves Proposition
\ref{prop.ent.gcd.props1} \textbf{(g)}.

\textbf{(h)} We can prove this similarly to how we just proved Proposition
\ref{prop.ent.gcd.props1} \textbf{(g)}, but it is easier to derive it from
what was already shown.

Let $a,b\in\mathbb{Z}$. Proposition \ref{prop.ent.gcd.props1} \textbf{(b)}
(applied to $-b$ instead of $b$) yields%
\begin{align*}
\gcd\left(  a,-b\right)   &  =\gcd\left(  -b,a\right)  =\gcd\left(
b,a\right)  \ \ \ \ \ \ \ \ \ \ \left(
\begin{array}
[c]{c}%
\text{by Proposition \ref{prop.ent.gcd.props1} \textbf{(g),}}\\
\text{applied to }b\text{ and }a\text{ instead of }a\text{ and }b
\end{array}
\right) \\
&  =\gcd\left(  a,b\right)  \ \ \ \ \ \ \ \ \ \ \left(  \text{by Proposition
\ref{prop.ent.gcd.props1} \textbf{(b)}}\right)  .
\end{align*}
This proves Proposition \ref{prop.ent.gcd.props1} \textbf{(h)}.

\textbf{(i)} Let $a,b\in\mathbb{Z}$ satisfy $a\mid b$. From $a\mid b$, we
obtain $b\equiv0\operatorname{mod}a$. Hence, Proposition
\ref{prop.ent.gcd.props1} \textbf{(d)} (applied to $c=0$) yields $\gcd\left(
a,b\right)  =\gcd\left(  a,0\right)  =\left\vert a\right\vert $ (by
Proposition \ref{prop.ent.gcd.props1} \textbf{(a)}). This proves Proposition
\ref{prop.ent.gcd.props1} \textbf{(i)}.

\textbf{(j)} The empty list of integers $\left(  {}\right)  $ has the property
that all its entries are $0$ (indeed, this is vacuously true because it has no
entries at all). Thus, its greatest common divisor is defined to be $0$ (by
the \textquotedblleft If $b_{1},b_{2},\ldots,b_{k}$ are not all $0$%
\textquotedblright\ case of Definition \ref{def.ent.gcd.gcd}). In other words,
$\gcd\left(  {}\right)  =0$. This proves Proposition \ref{prop.ent.gcd.props1}
\textbf{(j)}.
\end{proof}

\begin{remark}
Proposition \ref{prop.ent.gcd.props1} \textbf{(c)} says that if we add a
multiple of $a$ to $b$, then $\gcd\left(  a,b\right)  $ does not change.
Similarly, if we add a multiple of $b$ to $a$, then $\gcd\left(  a,b\right)  $
does not change (i.e., we have $\gcd\left(  vb+a,b\right)  =\gcd\left(
a,b\right)  $ for all $a,b,v\in\mathbb{Z}$).

However, if we \textbf{simultaneously} add a multiple of $a$ to $b$ and a
multiple of $b$ to $a$, then $\gcd\left(  a,b\right)  $ may well change: i.e.,
we may have $\gcd\left(  vb+a,ua+b\right)  \neq\gcd\left(  a,b\right)  $ for
all $a,b,u,v\in\mathbb{Z}$. Examples are easy to find (just take $v=1$ and
$u=1$).
\end{remark}

Proposition \ref{prop.ent.gcd.props1} gives a quick way to compute
$\gcd\left(  a,b\right)  $ for two nonnegative integers $a$ and $b$, by
repeatedly applying division with remainder. For example, let us compute
$\gcd\left(  210,45\right)  $ as follows:%
\begin{align*}
\gcd\left(  210,45\right)   &  =\gcd\left(  45,210\right)
\ \ \ \ \ \ \ \ \ \ \left(  \text{by Proposition \ref{prop.ent.gcd.props1}
\textbf{(b)}}\right) \\
&  =\gcd\left(  45,\underbrace{210\%45}_{=30}\right)
\ \ \ \ \ \ \ \ \ \ \left(  \text{by Proposition \ref{prop.ent.gcd.props1}
\textbf{(e)}}\right) \\
&  =\gcd\left(  45,30\right) \\
&  =\gcd\left(  30,45\right)  \ \ \ \ \ \ \ \ \ \ \left(  \text{by Proposition
\ref{prop.ent.gcd.props1} \textbf{(b)}}\right) \\
&  =\gcd\left(  30,\underbrace{45\%30}_{=15}\right)
\ \ \ \ \ \ \ \ \ \ \left(  \text{by Proposition \ref{prop.ent.gcd.props1}
\textbf{(e)}}\right) \\
&  =\gcd\left(  30,15\right) \\
&  =\gcd\left(  15,30\right)  \ \ \ \ \ \ \ \ \ \ \left(  \text{by Proposition
\ref{prop.ent.gcd.props1} \textbf{(b)}}\right) \\
&  =\gcd\left(  15,\underbrace{30\%15}_{=0}\right)
\ \ \ \ \ \ \ \ \ \ \left(  \text{by Proposition \ref{prop.ent.gcd.props1}
\textbf{(e)}}\right) \\
&  =\gcd\left(  15,0\right)  =\left\vert 15\right\vert
\ \ \ \ \ \ \ \ \ \ \left(  \text{by Proposition \ref{prop.ent.gcd.props1}
\textbf{(a)}}\right) \\
&  =15.
\end{align*}
This method of computing $\gcd\left(  a,b\right)  $ is called the
\textit{Euclidean algorithm}, and is usually much faster than the divisors of
$a$ or the divisors of $b$ can be found!

\subsubsection{Bezout's theorem}

The following fact about gcds is one of the most important facts in number theory:

\begin{theorem}
\label{thm.ent.gcd.bezout}Let $a$ and $b$ be two integers. Then, there exist
integers $x\in\mathbb{Z}$ and $y\in\mathbb{Z}$ such that%
\[
\gcd\left(  a,b\right)  =xa+yb.
\]

\end{theorem}

Theorem \ref{thm.ent.gcd.bezout} is often stated as follows: \textquotedblleft
If $a$ and $b$ are two integers, then $\gcd\left(  a,b\right)  $ is a
$\mathbb{Z}$-linear combination of $a$ and $b$\textquotedblright. The notion
\textquotedblleft$\mathbb{Z}$-linear combination of $a$ and $b$%
\textquotedblright\ simply means \textquotedblleft a number of the form
$xa+yb$ with $x\in\mathbb{Z}$ and $y\in\mathbb{Z}$\textquotedblright\ (this is
exactly the notion of a \textquotedblleft linear combination\textquotedblright%
\ in linear algebra, except that now the scalars must come from $\mathbb{Z}$),
so this is just a restatement of Theorem \ref{thm.ent.gcd.bezout}.

Theorem \ref{thm.ent.gcd.bezout} is known as \textit{Bezout's theorem} (or
\textit{Bezout's identity})\footnote{or \textit{Bezout's theorem for integers}
if you want to be more precise (as there are similar theorems for other
objects)}. We shall prove it in several steps. The first step is to show it
when $a$ and $b$ are nonnegative:

\begin{lemma}
\label{lem.ent.gcd.bezout.++}Let $a\in\mathbb{N}$ and $b\in\mathbb{N}$. Then,
there exist integers $x\in\mathbb{Z}$ and $y\in\mathbb{Z}$ such that%
\[
\gcd\left(  a,b\right)  =xa+yb.
\]

\end{lemma}

\begin{proof}
[Proof of Lemma \ref{lem.ent.gcd.bezout.++}.]The following proof uses a
strategy similar to the Euclidean algorithm (making one of $a$ and $b$ smaller
repeatedly until one of $a$ and $b$ becomes $0$), and can in fact be viewed as
a \textquotedblleft protocol\textquotedblright\ of the algorithm.

We use strong induction on $a+b$. Thus, we fix an $n\in\mathbb{N}$, and assume
(as induction hypothesis) that Lemma \ref{lem.ent.gcd.bezout.++} holds
whenever $a+b<n$. We must now prove that Lemma \ref{lem.ent.gcd.bezout.++}
holds whenever $a+b=n$.

We have assumed that Lemma \ref{lem.ent.gcd.bezout.++} holds whenever $a+b<n$.
In other words, the following statement holds:

\begin{statement}
\textit{Statement 1:} Let $a\in\mathbb{N}$ and $b\in\mathbb{N}$ be such that
$a+b<n$. Then, there exist integers $x\in\mathbb{Z}$ and $y\in\mathbb{Z}$ such
that $\gcd\left(  a,b\right)  =xa+yb$.
\end{statement}

Now, we must prove that Lemma \ref{lem.ent.gcd.bezout.++} holds whenever
$a+b=n$. Let us first prove this in the case when $b\geq a$:

\begin{statement}
\textit{Statement 2:} Let $a\in\mathbb{N}$ and $b\in\mathbb{N}$ be such that
$a+b=n$ and $b\geq a$. Then, there exist integers $x\in\mathbb{Z}$ and
$y\in\mathbb{Z}$ such that $\gcd\left(  a,b\right)  =xa+yb$.
\end{statement}

[\textit{Proof of Statement 2:} We are in one of the following two cases:

\textit{Case 1:} We have $a=0$.

\textit{Case 2:} We have $a\neq0$.

Let us first consider Case 1. In this case, we have $a=0$. Now, Proposition
\ref{prop.ent.gcd.props1} \textbf{(a)} (applied to $b$ instead of $a$) yields
$\gcd\left(  b,0\right)  =\gcd\left(  b\right)  =\left\vert b\right\vert
\in\left\{  b,-b\right\}  $. In other words, $\gcd\left(  b,0\right)  =ub$ for
some $u\in\left\{  1,-1\right\}  $. Consider this $u$. Now, Proposition
\ref{prop.ent.gcd.props1} \textbf{(b)} yields%
\[
\gcd\left(  a,b\right)  =\gcd\left(  b,\underbrace{a}_{=0}\right)
=\gcd\left(  b,0\right)  =ub=0a+ub.
\]
Hence, there exist integers $x\in\mathbb{Z}$ and $y\in\mathbb{Z}$ such that
$\gcd\left(  a,b\right)  =xa+yb$ (namely, $x=0$ and $y=u$). Thus, Statement 2
is proven in Case 1.

Let us next consider Case 2. In this case, we have $a\neq0$. Hence, $a>0$
(since $a\in\mathbb{N}$), so that $a+b>b$. Hence, $b<a+b=n$.

From $b\geq a$, we obtain $b-a\in\mathbb{N}$. Moreover, $a\in\mathbb{N}$ and
$b-a\in\mathbb{N}$ satisfy $a+\left(  b-a\right)  =b<n$. Therefore, we can
apply Statement 1 \textbf{to }$b-a$ \textbf{instead of }$b$. Thus we obtain
that there exist integers $x\in\mathbb{Z}$ and $y\in\mathbb{Z}$ such that
$\gcd\left(  a,b-a\right)  =xa+y\left(  b-a\right)  $. Fix two such integers
$x$ and $y$, and denote them by $x_{0}$ and $y_{0}$. Thus, $x_{0}$ and $y_{0}$
are two integers such that $\gcd\left(  a,b-a\right)  =x_{0}a+y_{0}\left(
b-a\right)  $.

Also, Proposition \ref{prop.ent.gcd.props1} \textbf{(c)} (applied to $u=-1$)
yields $\gcd\left(  a,\left(  -1\right)  a+b\right)  =\gcd\left(  a,b\right)
$. Hence,%
\begin{align*}
\gcd\left(  a,b\right)   &  =\gcd\left(  a,\underbrace{\left(  -1\right)
a+b}_{=b-a}\right)  =\gcd\left(  a,b-a\right)  =x_{0}a+y_{0}\left(  b-a\right)
\\
&  =x_{0}a+y_{0}b-y_{0}a=\left(  x_{0}-y_{0}\right)  a+y_{0}b.
\end{align*}
Hence, there exist integers $x\in\mathbb{Z}$ and $y\in\mathbb{Z}$ such that
$\gcd\left(  a,b\right)  =xa+yb$ (namely, $x=x_{0}-y_{0}$ and $y=y_{0}$).
Thus, Statement 2 is proven in Case 2.

We have now proven Statement 2 in both Cases 1 and 2. Hence, Statement 2 is
always proven.]

Now, we can prove that Lemma \ref{lem.ent.gcd.bezout.++} holds whenever
$a+b=n$:

\begin{statement}
\textit{Statement 3:} Let $a\in\mathbb{N}$ and $b\in\mathbb{N}$ be such that
$a+b=n$. Then, there exist integers $x\in\mathbb{Z}$ and $y\in\mathbb{Z}$ such
that $\gcd\left(  a,b\right)  =xa+yb$.
\end{statement}

[\textit{Proof of Statement 3:} We are in one of the following two cases:

\textit{Case 1:} We have $b\geq a$.

\textit{Case 2:} We have $b<a$.

Let us first consider Case 1. In this case, we have $b\geq a$. Hence,
Statement 2 shows that there exist integers $x\in\mathbb{Z}$ and
$y\in\mathbb{Z}$ such that $\gcd\left(  a,b\right)  =xa+yb$. Thus, Statement 3
is proven in Case 1.

Let us next consider Case 2. In this case, we have $b<a$. Hence, $a>b$, so
that $a\geq b$. This shows that we can apply Statement 2 \textbf{to }%
$b$\textbf{ and }$a$ \textbf{instead of }$a$ \textbf{and }$b$. Thus we obtain
that there exist integers $x\in\mathbb{Z}$ and $y\in\mathbb{Z}$ such that
$\gcd\left(  b,a\right)  =xb+ya$. Fix two such integers $x$ and $y$, and
denote them by $x_{0}$ and $y_{0}$. Thus, $x_{0}$ and $y_{0}$ are two integers
such that $\gcd\left(  b,a\right)  =x_{0}b+y_{0}a$. Now, Proposition
\ref{prop.ent.gcd.props1} \textbf{(b)} yields $\gcd\left(  a,b\right)
=\gcd\left(  b,a\right)  =x_{0}b+y_{0}a=y_{0}a+x_{0}b$. Hence, there exist
integers $x\in\mathbb{Z}$ and $y\in\mathbb{Z}$ such that $\gcd\left(
a,b\right)  =xa+yb$ (namely, $x=y_{0}$ and $y=x_{0}$). Thus, Statement 3 is
proven in Case 2.

We have now proven Statement 3 in both Cases 1 and 2. Hence, Statement 3 is
always proven.]

By proving Statement 3, we have shown that Lemma \ref{lem.ent.gcd.bezout.++}
holds whenever $a+b=n$. This completes the induction step. Thus, Lemma
\ref{lem.ent.gcd.bezout.++} is proven by strong induction.
\end{proof}

Next, we shall prove Theorem \ref{thm.ent.gcd.bezout} when $a\in\mathbb{N}$
but $b$ may be negative:

\begin{lemma}
\label{lem.ent.gcd.bezout.+}Let $a\in\mathbb{N}$ and $b\in\mathbb{Z}$. Then,
there exist integers $x\in\mathbb{Z}$ and $y\in\mathbb{Z}$ such that%
\[
\gcd\left(  a,b\right)  =xa+yb.
\]

\end{lemma}

\begin{proof}
[Proof of Lemma \ref{lem.ent.gcd.bezout.+}.]We are in one of the following two cases:

\textit{Case 1:} We have $b\geq0$.

\textit{Case 2:} We have $b<0$.

Let us first consider Case 1. In this case, we have $b\geq0$. Thus,
$b\in\mathbb{N}$ (since $b\in\mathbb{Z}$). Therefore, Lemma
\ref{lem.ent.gcd.bezout.++} shows that there exist integers $x\in\mathbb{Z}$
and $y\in\mathbb{Z}$ such that $\gcd\left(  a,b\right)  =xa+yb$. Thus, Lemma
\ref{lem.ent.gcd.bezout.+} is proven in Case 1.

Let us now consider Case 2. In this case, we have $b<0$. Hence, $-b>0$, so
that $-b\in\mathbb{N}$ (since $-b\in\mathbb{Z}$). Therefore, Lemma
\ref{lem.ent.gcd.bezout.++} (applied to $-b$ instead of $b$) shows that there
exist integers $x\in\mathbb{Z}$ and $y\in\mathbb{Z}$ such that $\gcd\left(
a,-b\right)  =xa+y\left(  -b\right)  $. Fix such integers, and denote them by
$x_{0}$ and $y_{0}$. Thus, $x_{0}\in\mathbb{Z}$ and $y_{0}\in\mathbb{Z}$ are
integers such that $\gcd\left(  a,-b\right)  =x_{0}a+y_{0}\left(  -b\right)  $.

Now, Proposition \ref{prop.ent.gcd.props1} \textbf{(h)} yields $\gcd\left(
a,-b\right)  =\gcd\left(  a,b\right)  $. Hence,%
\[
\gcd\left(  a,b\right)  =\gcd\left(  a,-b\right)  =x_{0}a+y_{0}\left(
-b\right)  =x_{0}a+\left(  -y_{0}\right)  b.
\]
Hence, there exist integers $x\in\mathbb{Z}$ and $y\in\mathbb{Z}$ such that
$\gcd\left(  a,b\right)  =xa+yb$ (namely, $x=x_{0}$ and $y=-y_{0}$). Thus,
Lemma \ref{lem.ent.gcd.bezout.+} is proven in Case 2.

We have now proven Lemma \ref{lem.ent.gcd.bezout.+} in both Cases 1 and 2.
Hence, Lemma \ref{lem.ent.gcd.bezout.+} is proven.
\end{proof}

Now, we can prove the whole Theorem \ref{thm.ent.gcd.bezout}:

\begin{proof}
[Proof of Theorem \ref{thm.ent.gcd.bezout}.]Theorem \ref{thm.ent.gcd.bezout}
can be derived from Lemma \ref{lem.ent.gcd.bezout.+} in the same way as Lemma
\ref{lem.ent.gcd.bezout.+} was derived from Lemma \ref{lem.ent.gcd.bezout.++}
(except that this time, we have to distinguish between the cases $a\geq0$ and
$a<0$, and we have to use Proposition \ref{prop.ent.gcd.props1} \textbf{(g)}
instead of Proposition \ref{prop.ent.gcd.props1} \textbf{(h)}). Again, let us
give the detailed argument for the sake of completeness:

\begin{fineprint}
We are in one of the following two cases:

\textit{Case 1:} We have $a\geq0$.

\textit{Case 2:} We have $a<0$.

Let us first consider Case 1. In this case, we have $a\geq0$. Thus,
$a\in\mathbb{N}$ (since $a\in\mathbb{Z}$). Therefore, Lemma
\ref{lem.ent.gcd.bezout.+} shows that there exist integers $x\in\mathbb{Z}$
and $y\in\mathbb{Z}$ such that $\gcd\left(  a,b\right)  =xa+yb$. Thus, Theorem
\ref{thm.ent.gcd.bezout} is proven in Case 1.

Let us now consider Case 2. In this case, we have $a<0$. Hence, $-a>0$, so
that $-a\in\mathbb{N}$ (since $-a\in\mathbb{Z}$). Therefore, Lemma
\ref{lem.ent.gcd.bezout.+} (applied to $-a$ instead of $a$) shows that there
exist integers $x\in\mathbb{Z}$ and $y\in\mathbb{Z}$ such that $\gcd\left(
-a,b\right)  =x\left(  -a\right)  +yb$. Fix such integers, and denote them by
$x_{0}$ and $y_{0}$. Thus, $x_{0}\in\mathbb{Z}$ and $y_{0}\in\mathbb{Z}$ are
integers such that $\gcd\left(  -a,b\right)  =x_{0}\left(  -a\right)  +y_{0}b$.

Now, Proposition \ref{prop.ent.gcd.props1} \textbf{(g)} yields $\gcd\left(
-a,b\right)  =\gcd\left(  a,b\right)  $. Hence,%
\[
\gcd\left(  a,b\right)  =\gcd\left(  -a,b\right)  =x_{0}\left(  -a\right)
+y_{0}b=\left(  -x_{0}\right)  a+y_{0}b.
\]
Hence, there exist integers $x\in\mathbb{Z}$ and $y\in\mathbb{Z}$ such that
$\gcd\left(  a,b\right)  =xa+yb$ (namely, $x=-x_{0}$ and $y=y_{0}$). Thus,
Theorem \ref{thm.ent.gcd.bezout} is proven in Case 2.

We have now proven Theorem \ref{thm.ent.gcd.bezout} in both Cases 1 and 2.
Hence, Theorem \ref{thm.ent.gcd.bezout} is proven.
\end{fineprint}
\end{proof}

\subsubsection{First applications of Bezout's theorem}

An important corollary of Theorem \ref{thm.ent.gcd.bezout} is the following fact:

\begin{theorem}
\label{thm.ent.gcd.uniprop}Let $a,b\in\mathbb{Z}$. Then:

\textbf{(a)} For each $m\in\mathbb{Z}$, we have the following logical
equivalence:%
\begin{equation}
\left(  m\mid a\ \text{and }m\mid b\right)  \ \Longleftrightarrow\ \left(
m\mid\gcd\left(  a,b\right)  \right)  . \label{eq.thm.ent.gcd.uniprop.equiv}%
\end{equation}


\textbf{(b)} The common divisors of $a$ and $b$ are precisely the divisors of
$\gcd\left(  a,b\right)  $.

\textbf{(c)} We have $\operatorname*{Div}\left(  a,b\right)
=\operatorname*{Div}\left(  \gcd\left(  a,b\right)  \right)  $.
\end{theorem}

The three parts of this theorem are saying the same thing from slightly
different perspectives; the importance of the theorem nevertheless justifies
this repetition. To prove the theorem, we first show the following:

\begin{lemma}
\label{lem.ent.gcd.uniprop}Let $m,a,b\in\mathbb{Z}$ be such that $m\mid a$ and
$m\mid b$. Then, $m\mid\gcd\left(  a,b\right)  $.
\end{lemma}

\begin{proof}
[Proof of Lemma \ref{lem.ent.gcd.uniprop}.]Theorem \ref{thm.ent.gcd.bezout}
shows that there exist integers $x\in\mathbb{Z}$ and $y\in\mathbb{Z}$ such
that%
\begin{equation}
\gcd\left(  a,b\right)  =xa+yb. \label{pf.lem.ent.gcd.uniprop.1}%
\end{equation}
Consider these $x$ and $y$. Now, $m\mid a\mid xa$, so that $xa\equiv
0\operatorname{mod}m$. Also, $m\mid b\mid yb$, thus $yb\equiv
0\operatorname{mod}m$. Adding the congruences $xa\equiv0\operatorname{mod}m$
and $yb\equiv0\operatorname{mod}m$ together, we find $xa+yb\equiv
0+0=0\operatorname{mod}m$; in other words, $m\mid xa+yb$. In view of
(\ref{pf.lem.ent.gcd.uniprop.1}), this rewrites as $m\mid\gcd\left(
a,b\right)  $. This proves Lemma \ref{lem.ent.gcd.uniprop}.
\end{proof}

\begin{proof}
[Proof of Theorem \ref{thm.ent.gcd.uniprop}.]\textbf{(a)} Let $m\in\mathbb{Z}%
$. In order to prove (\ref{eq.thm.ent.gcd.uniprop.equiv}), we need to prove
the \textquotedblleft$\Longrightarrow$\textquotedblright\ and
\textquotedblleft$\Longleftarrow$\textquotedblright\ directions of the
equivalence (\ref{eq.thm.ent.gcd.uniprop.equiv}). But this is easy: The
\textquotedblleft$\Longrightarrow$\textquotedblright\ direction is just the
statement of Lemma \ref{lem.ent.gcd.uniprop}, whereas the \textquotedblleft%
$\Longleftarrow$\textquotedblright\ direction is trivial (to wit: if
$m\mid\gcd\left(  a,b\right)  $, then
\[
m\mid\gcd\left(  a,b\right)  \mid a\ \ \ \ \ \ \ \ \ \ \left(  \text{by
Proposition \ref{prop.ent.gcd.props1} \textbf{(e)}}\right)
\]
and%
\[
m\mid\gcd\left(  a,b\right)  \mid b\ \ \ \ \ \ \ \ \ \ \left(  \text{by
Proposition \ref{prop.ent.gcd.props1} \textbf{(e)}}\right)  ,
\]
and thus $\left(  m\mid a\ \text{and }m\mid b\right)  $). Hence, the
equivalence (\ref{eq.thm.ent.gcd.uniprop.equiv}) is proven. This proves
Theorem \ref{thm.ent.gcd.uniprop} \textbf{(a)}.

\textbf{(b)} The common divisors of $a$ and $b$ are precisely the integers $m$
that satisfy $\left(  m\mid a\text{ and }m\mid b\right)  $ (by the definition
of \textquotedblleft common divisor\textquotedblright). In view of the
equivalence (\ref{eq.thm.ent.gcd.uniprop.equiv}), this rewrites as follows:
The common divisors of $a$ and $b$ are precisely the integers $m$ that satisfy
$m\mid\gcd\left(  a,b\right)  $. In other words, the common divisors of $a$
and $b$ are precisely the divisors of $\gcd\left(  a,b\right)  $. This proves
Theorem \ref{thm.ent.gcd.uniprop} \textbf{(b)}.

\textbf{(c)} Recall that each $c\in\mathbb{Z}$ satisfies%
\begin{align*}
\operatorname*{Div}\left(  c\right)   &  =\left\{  \text{the common divisors
of }c\right\}  \ \ \ \ \ \ \ \ \ \ \left(  \text{by the definition of
}\operatorname*{Div}\left(  c\right)  \right) \\
&  =\left\{  \text{the integers }x\text{ such that }x\mid c\right\} \\
&  \ \ \ \ \ \ \ \ \ \ \left(  \text{by the definition of \textquotedblleft
common divisors\textquotedblright}\right) \\
&  =\left\{  \text{the divisors of }c\right\}  .
\end{align*}
Applying this to $c=\gcd\left(  a,b\right)  $, we obtain%
\begin{equation}
\operatorname*{Div}\left(  \gcd\left(  a,b\right)  \right)  =\left\{
\text{the divisors of }\gcd\left(  a,b\right)  \right\}  .
\label{pf.thm.ent.gcd.uniprop.c.1a}%
\end{equation}


The definition of $\operatorname*{Div}\left(  a,b\right)  $ yields%
\begin{align*}
\operatorname*{Div}\left(  a,b\right)   &  =\left\{  \text{the common divisors
of }a\text{ and }b\right\} \\
&  =\left\{  \text{the divisors of }\gcd\left(  a,b\right)  \right\}
\ \ \ \ \ \ \ \ \ \ \left(  \text{by Theorem \ref{thm.ent.gcd.uniprop}
\textbf{(b)}}\right) \\
&  =\operatorname*{Div}\left(  \gcd\left(  a,b\right)  \right)
\ \ \ \ \ \ \ \ \ \ \left(  \text{by (\ref{pf.thm.ent.gcd.uniprop.c.1a}%
)}\right)  .
\end{align*}
This proves Theorem \ref{thm.ent.gcd.uniprop} \textbf{(c)}.
\end{proof}

The following corollary of Theorem \ref{thm.ent.gcd.bezout} let us
\textquotedblleft combine\textquotedblright\ two divisibilities $a\mid c$ and
$b\mid c$. In fact, Proposition \ref{prop.ent.div.2} \textbf{(c)} would
already allow us to \textquotedblleft combine\textquotedblright\ them to form
$ab\mid cc=c^{2}$; but we can also \textquotedblleft combine\textquotedblright%
\ them to $ab\mid\gcd\left(  a,b\right)  \cdot c$ using the following fact:

\begin{theorem}
\label{thm.ent.gcd.combine}Let $a,b,c\in\mathbb{Z}$ satisfy $a\mid c$ and
$b\mid c$. Then, $ab\mid\gcd\left(  a,b\right)  \cdot c$.
\end{theorem}

\begin{example}
Let $a=6$ and $b=10$ and $c=30$. Then, $a=6\mid30=c$ and $b=10\mid30=c$. Thus,
Theorem \ref{thm.ent.gcd.combine} yields $ab\mid\gcd\left(  a,b\right)  \cdot
c$. And indeed, this is true, since $ab=6\cdot10\mid2\cdot30=\gcd\left(
a,b\right)  \cdot c$ (because $\gcd\left(  a,b\right)  =\gcd\left(
6,10\right)  =2$). Note that this latter divisibility is actually an equality:
we have $6\cdot10=2\cdot30$. Note also that we do \textbf{not} obtain $ab\mid
c$ (and indeed, this does not hold).
\end{example}

\begin{proof}
[Proof of Theorem \ref{thm.ent.gcd.combine}.]Theorem \ref{thm.ent.gcd.bezout}
yields that there exist integers $x\in\mathbb{Z}$ and $y\in\mathbb{Z}$ such
that $\gcd\left(  a,b\right)  =xa+yb$. Consider these $x$ and $y$.

We have $a\mid c$. In other words, there exists an integer $u$ such that
$c=au$. Consider this $u$.

We have $b\mid c$. In other words, there exists an integer $v$ such that
$c=bv$. Consider this $v$.

Now,%
\[
\underbrace{\gcd\left(  a,b\right)  }_{=xa+yb}\cdot c=\left(  xa+yb\right)
c=xa\underbrace{c}_{=bv}+yb\underbrace{c}_{=au}=xabv+ybau=ab\left(
xv+yu\right)  .
\]
Thus, there exists an integer $d$ such that $\gcd\left(  a,b\right)  \cdot
c=abd$ (namely, $d=xv+yu$). In other words, $ab\mid\gcd\left(  a,b\right)
\cdot c$. This proves Theorem \ref{thm.ent.gcd.combine}.
\end{proof}

Here is another corollary of Theorem \ref{thm.ent.gcd.bezout} whose usefulness
will become clearer later on:

\begin{theorem}
\label{thm.ent.gcd.cancel}Let $a,b,c\in\mathbb{Z}$ satisfy $a\mid bc$. Then,
$a\mid\gcd\left(  a,b\right)  \cdot c$.
\end{theorem}

At this point, you should see that Theorem \ref{thm.ent.gcd.cancel} allows
\textquotedblleft strengthening\textquotedblright\ divisibilities: You give it
a \textquotedblleft weak\textquotedblright\ divisibility $a\mid bc$, and
obtain a \textquotedblleft stronger\textquotedblright\ divisibility $a\mid
\gcd\left(  a,b\right)  \cdot c$ from it (stronger because $\gcd\left(
a,b\right)  $ is usually smaller than $b$).

\begin{proof}
[Proof of Theorem \ref{thm.ent.gcd.cancel}.]Theorem \ref{thm.ent.gcd.bezout}
yields that there exist integers $x\in\mathbb{Z}$ and $y\in\mathbb{Z}$ such
that $\gcd\left(  a,b\right)  =xa+yb$. Consider these $x$ and $y$.

We have $a\mid bc\mid ybc$; in other words, $ybc\equiv0\operatorname{mod}a$.
Also, $a\mid axc$, so that $axc\equiv0\operatorname{mod}a$. Adding the two
congruences $axc\equiv0\operatorname{mod}a$ and $ybc\equiv0\operatorname{mod}%
a$ together, we obtain $axc+ybc\equiv0+0=0\operatorname{mod}a$. In view of
$axc+ybc=\underbrace{\left(  xa+yb\right)  }_{=\gcd\left(  a,b\right)  }%
c=\gcd\left(  a,b\right)  \cdot c$, this rewrites as $\gcd\left(  a,b\right)
\cdot c\equiv0\operatorname{mod}a$. In other words, $a\mid\gcd\left(
a,b\right)  \cdot c$. This proves Theorem \ref{thm.ent.gcd.cancel}.
\end{proof}

\begin{corollary}
\label{cor.ent.gcd.sa,sb}Let $s,a,b\in\mathbb{Z}$. Then,
\[
\gcd\left(  sa,sb\right)  =\left\vert s\right\vert \gcd\left(  a,b\right)  .
\]

\end{corollary}

\begin{proof}
[Proof of Corollary \ref{cor.ent.gcd.sa,sb}.]We shall prove that the two
integers $\gcd\left(  sa,sb\right)  $ and $s\gcd\left(  a,b\right)  $ mutually
divide each other (i.e., they satisfy $\gcd\left(  sa,sb\right)  \mid
s\gcd\left(  a,b\right)  $ and $s\gcd\left(  a,b\right)  \mid\gcd\left(
sa,sb\right)  $). Then, Exercise \ref{exe.ent.div.abba} will let us conclude
that $\left\vert \gcd\left(  sa,sb\right)  \right\vert =\left\vert
s\gcd\left(  a,b\right)  \right\vert $. This will then rewrite as $\gcd\left(
sa,sb\right)  =\left\vert s\right\vert \gcd\left(  a,b\right)  $, and we will
be done. (This trick is actually a common strategy for proving equalities
between gcds.)

For the sake of brevity, let us set $g=\gcd\left(  sa,sb\right)  $ and
$h=s\gcd\left(  a,b\right)  $. So our first goal is to prove that $g\mid h$
and $h\mid g$.

\textit{Proof of }$g\mid h$\textit{:} Theorem \ref{thm.ent.gcd.bezout} yields
that there exist integers $x\in\mathbb{Z}$ and $y\in\mathbb{Z}$ such that
$\gcd\left(  a,b\right)  =xa+yb$. Consider these $x$ and $y$.

Proposition \ref{prop.ent.gcd.props1} \textbf{(f)} (applied to $sa$ and $sb$
instead of $a$ and $b$) yields that $\gcd\left(  sa,sb\right)  \mid sa$ and
$\gcd\left(  sa,sb\right)  \mid sb$. From $g=\gcd\left(  sa,sb\right)  \mid
sa$, we obtain $g\mid sa\mid xsa$, thus $xsa\equiv0\operatorname{mod}g$.
Similarly, $ysb\equiv0\operatorname{mod}g$. Adding these two congruences
together, we find $xsa+ysb\equiv0\operatorname{mod}g$. Now,%
\[
h=s\underbrace{\gcd\left(  a,b\right)  }_{=xa+yb}=s\left(  xa+yb\right)
=xsa+ysb\equiv0\operatorname{mod}g.
\]
In other words, $g\mid h$. Thus, we have proven $g\mid h$.

\textit{Proof of }$h\mid g$\textit{:} Proposition \ref{prop.ent.gcd.props1}
\textbf{(f)} yields $\gcd\left(  a,b\right)  \mid a$ and $\gcd\left(
a,b\right)  \mid b$. Also, $s\mid s$. Hence, Proposition \ref{prop.ent.div.2}
\textbf{(c)} (applied to $s,\gcd\left(  a,b\right)  ,s,a$ instead of
$a_{1},a_{2},b_{1},b_{2}$) yields $s\gcd\left(  a,b\right)  \mid sa$.
Similarly, $s\gcd\left(  a,b\right)  \mid sb$. Hence, Lemma
\ref{lem.ent.gcd.uniprop} (applied to $s\gcd\left(  a,b\right)  $, $sa$ and
$sb$ instead of $m$, $a$ and $b$) yields $s\gcd\left(  a,b\right)  \mid
\gcd\left(  sa,sb\right)  $. In view of $g=\gcd\left(  sa,sb\right)  $ and
$h=s\gcd\left(  a,b\right)  $, this rewrites as $h\mid g$. So we have proven
$h\mid g$.

Now, Exercise \ref{exe.ent.div.abba} (applied to $g$ and $h$ instead of $a$
and $b$) yields $\left\vert g\right\vert =\left\vert h\right\vert $.

But recall that a gcd of any finitely many integers is nonnegative (by
Definition \ref{def.ent.gcd.gcd}). Hence, in particular, $\gcd\left(
a,b\right)  $ and $\gcd\left(  sa,sb\right)  $ are nonnegative. From
$g=\gcd\left(  sa,sb\right)  $, we obtain%
\[
\left\vert g\right\vert =\left\vert \gcd\left(  sa,sb\right)  \right\vert
=\gcd\left(  sa,sb\right)
\]
(since $\gcd\left(  sa,sb\right)  $ is nonnegative). Also, from $h=s\gcd
\left(  a,b\right)  $, we obtain%
\begin{align*}
\left\vert h\right\vert  &  =\left\vert s\gcd\left(  a,b\right)  \right\vert
=\left\vert s\right\vert \cdot\underbrace{\left\vert \gcd\left(  a,b\right)
\right\vert }_{\substack{=\gcd\left(  a,b\right)  \\\text{(since }\gcd\left(
a,b\right)  \\\text{is nonnegative)}}}\ \ \ \ \ \ \ \ \ \ \left(  \text{by
(\ref{eq.ent.div.abs(xy)})}\right) \\
&  =\left\vert s\right\vert \gcd\left(  a,b\right)  .
\end{align*}
Hence,
\[
\gcd\left(  sa,sb\right)  =\left\vert g\right\vert =\left\vert h\right\vert
=\left\vert s\right\vert \gcd\left(  a,b\right)  .
\]
This proves Corollary \ref{cor.ent.gcd.sa,sb}.
\end{proof}

\begin{exercise}
\label{exe.ent.gcd.div}Let $a_{1},a_{2},b_{1},b_{2}\in\mathbb{Z}$ satisfy
$a_{1}\mid b_{1}$ and $a_{2}\mid b_{2}$. Prove that%
\[
\gcd\left(  a_{1},a_{2}\right)  \mid\gcd\left(  b_{1},b_{2}\right)  .
\]

\end{exercise}

\begin{fineprint}
\begin{proof}
[Solution to Exercise \ref{exe.ent.gcd.div}.]Proposition
\ref{prop.ent.gcd.props1} \textbf{(f)} (applied to $a=a_{1}$ and $b=a_{2}$)
yields that we have $\gcd\left(  a_{1},a_{2}\right)  \mid a_{1}$ and
$\gcd\left(  a_{1},a_{2}\right)  \mid a_{2}$. Thus, $\gcd\left(  a_{1}%
,a_{2}\right)  \mid a_{1}\mid b_{1}$ and $\gcd\left(  a_{1},a_{2}\right)  \mid
a_{2}\mid b_{2}$.

So we know that $\gcd\left(  a_{1},a_{2}\right)  \mid b_{1}$ and $\gcd\left(
a_{1},a_{2}\right)  \mid b_{2}$. Hence, Lemma \ref{lem.ent.gcd.uniprop}
(applied to $m=\gcd\left(  a_{1},a_{2}\right)  $, $a=b_{1}$ and $b=b_{2}$)
yields $\gcd\left(  a_{1},a_{2}\right)  \mid\gcd\left(  b_{1},b_{2}\right)  $.
This solves Exercise \ref{exe.ent.gcd.div}.
\end{proof}
\end{fineprint}

\begin{exercise}
\label{exe.ent.gcd.abs}Let $a,b\in\mathbb{Z}$.

\textbf{(a)} Prove that $\gcd\left(  a,\left\vert b\right\vert \right)
=\gcd\left(  a,b\right)  $.

\textbf{(b)} Prove that $\gcd\left(  \left\vert a\right\vert ,b\right)
=\gcd\left(  a,b\right)  $.

\textbf{(c)} Prove that $\gcd\left(  \left\vert a\right\vert ,\left\vert
b\right\vert \right)  =\gcd\left(  a,b\right)  $.
\end{exercise}

\begin{fineprint}
\begin{proof}
[Solution to Exercise \ref{exe.ent.gcd.abs}.]\textbf{(a)} If $b\geq0$, then
$\left\vert b\right\vert =b$. Hence, if $b\geq0$, then Exercise
\ref{exe.ent.gcd.abs} \textbf{(a)} holds (since $\gcd\left(
a,\underbrace{\left\vert b\right\vert }_{=b}\right)  =\gcd\left(  a,b\right)
$). Thus, for the rest of this solution to Exercise \ref{exe.ent.gcd.abs}
\textbf{(a)}, we WLOG assume that we don't have $b\geq0$. Hence, we have
$b<0$. Thus, $\left\vert b\right\vert =-b$ and therefore $\gcd\left(
a,\underbrace{\left\vert b\right\vert }_{=-b}\right)  =\gcd\left(
a,-b\right)  =\gcd\left(  a,b\right)  $ (by Proposition
\ref{prop.ent.gcd.props1} \textbf{(h)}). This solves Exercise
\ref{exe.ent.gcd.abs} \textbf{(a)}.

\textbf{(b)} If $a\geq0$, then $\left\vert a\right\vert =a$. Hence, if
$a\geq0$, then Exercise \ref{exe.ent.gcd.abs} \textbf{(b)} holds (since
$\gcd\left(  \underbrace{\left\vert a\right\vert }_{=a},b\right)  =\gcd\left(
a,b\right)  $). Thus, for the rest of this solution to Exercise
\ref{exe.ent.gcd.abs} \textbf{(b)}, we WLOG assume that we don't have $a\geq
0$. Hence, we have $a<0$. Thus, $\left\vert a\right\vert =-a$ and therefore
$\gcd\left(  \underbrace{\left\vert a\right\vert }_{=-a},b\right)
=\gcd\left(  -a,b\right)  =\gcd\left(  a,b\right)  $ (by Proposition
\ref{prop.ent.gcd.props1} \textbf{(g)}). This solves Exercise
\ref{exe.ent.gcd.abs} \textbf{(b)}.

\textbf{(c)} Exercise \ref{exe.ent.gcd.abs} \textbf{(b)} (applied to
$\left\vert b\right\vert $ instead of $b$) yields $\gcd\left(  \left\vert
a\right\vert ,\left\vert b\right\vert \right)  =\gcd\left(  a,\left\vert
b\right\vert \right)  =\gcd\left(  a,b\right)  $ (by Exercise
\ref{exe.ent.gcd.abs} \textbf{(a)}). This solves Exercise
\ref{exe.ent.gcd.abs} \textbf{(c)}.
\end{proof}
\end{fineprint}

\subsubsection{gcds of multiple numbers}

The following theorem generalizes some of the previous facts to gcds of
multiple integers:

\begin{theorem}
\label{thm.ent.gcd.uniprop-mul}Let $b_{1},b_{2},\ldots,b_{k}$ be integers.

\textbf{(a)} For each $m\in\mathbb{Z}$, we have the following logical
equivalence:%
\[
\left(  m\mid b_{i}\text{ for all }i\in\left\{  1,2,\ldots,k\right\}  \right)
\ \Longleftrightarrow\ \left(  m\mid\gcd\left(  b_{1},b_{2},\ldots
,b_{k}\right)  \right)  .
\]


\textbf{(b)} The common divisors of $b_{1},b_{2},\ldots,b_{k}$ are precisely
the divisors of $\gcd\left(  b_{1},b_{2},\ldots,b_{k}\right)  $.

\textbf{(c)} We have $\operatorname*{Div}\left(  b_{1},b_{2},\ldots
,b_{k}\right)  =\operatorname*{Div}\left(  \gcd\left(  b_{1},b_{2}%
,\ldots,b_{k}\right)  \right)  $.

\textbf{(d)} If $k>0$, then%
\[
\gcd\left(  b_{1},b_{2},\ldots,b_{k}\right)  =\gcd\left(  \gcd\left(
b_{1},b_{2},\ldots,b_{k-1}\right)  ,b_{k}\right)  .
\]

\end{theorem}

\begin{proof}
[Proof of Theorem \ref{thm.ent.gcd.uniprop-mul}.]Forget that we fixed
$b_{1},b_{2},\ldots,b_{k}$. Rather than prove the four parts of Theorem
\ref{thm.ent.gcd.uniprop-mul} separately, we shall prove them together as a package.

We shall proceed by induction on $k$:

\textit{Induction base:} Theorem \ref{thm.ent.gcd.uniprop-mul} holds for $k=0$.

\begin{fineprint}
[\textit{Proof:} This is a straightforward exercise in dealing with empty
sets, $0$-tuples and vacuous truths. For the sake of completeness, here is the
full argument:

Assume that $k=0$. We must prove that Theorem \ref{thm.ent.gcd.uniprop-mul} holds.

Let $b_{1},b_{2},\ldots,b_{k}$ be integers. Of course, these are $0$ integers,
since $k=0$.

We don't have $k>0$ (since $k=0$). Hence, Theorem
\ref{thm.ent.gcd.uniprop-mul} \textbf{(d)} is vacuously true.

All of $b_{1},b_{2},\ldots,b_{k}$ are $0$ (indeed, this is vacuously true).
Thus, $\gcd\left(  b_{1},b_{2},\ldots,b_{k}\right)  =0$ (by Definition
\ref{def.ent.gcd.gcd}).

For each $m\in\mathbb{Z}$, we have the logical equivalence%
\begin{align*}
&  \ \left(  m\mid b_{i}\text{ for all }i\in\left\{  1,2,\ldots,k\right\}
\right) \\
&  \Longleftrightarrow\ \left(  \text{truth}\right)
\ \ \ \ \ \ \ \ \ \ \left(  \text{since there exists no }i\in\left\{
1,2,\ldots,k\right\}  \right) \\
&  \Longleftrightarrow\ \left(  m\mid0\right)  \ \ \ \ \ \ \ \ \ \ \left(
\text{since }m\mid0\text{ is always true}\right) \\
&  \Longleftrightarrow\ \left(  m\mid\gcd\left(  b_{1},b_{2},\ldots
,b_{k}\right)  \right)  \ \ \ \ \ \ \ \ \ \ \left(  \text{since }0=\gcd\left(
b_{1},b_{2},\ldots,b_{k}\right)  \right)  .
\end{align*}
This proves Theorem \ref{thm.ent.gcd.uniprop-mul} \textbf{(a)} (in the case
$k=0$, that is). Parts \textbf{(b)} and \textbf{(c)} of Theorem
\ref{thm.ent.gcd.uniprop-mul} are restatements of Theorem
\ref{thm.ent.gcd.uniprop-mul} \textbf{(a)} and can be derived from it in the
same way as we derived parts \textbf{(b)} and \textbf{(c)} of Theorem
\ref{thm.ent.gcd.uniprop} from Theorem \ref{thm.ent.gcd.uniprop} \textbf{(a)}.

Thus, all four parts of Theorem \ref{thm.ent.gcd.uniprop-mul} are proven for
$k=0$. This completes the induction base.]
\end{fineprint}

\textit{Induction step:} Let $\ell$ be a positive integer. Assume that Theorem
\ref{thm.ent.gcd.uniprop-mul} holds for $k=\ell-1$. We must prove that Theorem
\ref{thm.ent.gcd.uniprop-mul} holds for $k=\ell$.

We have assumed that Theorem \ref{thm.ent.gcd.uniprop-mul} holds for
$k=\ell-1$. In other words, the following statement holds:

\begin{statement}
\textit{Statement 1:} Let $b_{1},b_{2},\ldots,b_{\ell-1}$ be integers.

\textbf{(a)} For each $m\in\mathbb{Z}$, we have the following logical
equivalence:%
\[
\left(  m\mid b_{i}\text{ for all }i\in\left\{  1,2,\ldots,\ell-1\right\}
\right)  \ \Longleftrightarrow\ \left(  m\mid\gcd\left(  b_{1},b_{2}%
,\ldots,b_{\ell-1}\right)  \right)  .
\]


\textbf{(b)} The common divisors of $b_{1},b_{2},\ldots,b_{\ell-1}$ are
precisely the divisors of $\gcd\left(  b_{1},b_{2},\ldots,b_{\ell-1}\right)  $.

\textbf{(c)} We have $\operatorname*{Div}\left(  b_{1},b_{2},\ldots,b_{\ell
-1}\right)  =\operatorname*{Div}\left(  \gcd\left(  b_{1},b_{2},\ldots
,b_{\ell-1}\right)  \right)  $.

\textbf{(d)} If $\ell-1>0$, then%
\[
\gcd\left(  b_{1},b_{2},\ldots,b_{\ell-1}\right)  =\gcd\left(  \gcd\left(
b_{1},b_{2},\ldots,b_{\left(  \ell-1\right)  -1}\right)  ,b_{\ell-1}\right)
.
\]

\end{statement}

Recall that we must prove that Theorem \ref{thm.ent.gcd.uniprop-mul} holds for
$k=\ell$. In other words, we must prove the following statement:

\begin{statement}
\textit{Statement 2:} Let $b_{1},b_{2},\ldots,b_{\ell}$ be integers.

\textbf{(a)} For each $m\in\mathbb{Z}$, we have the following logical
equivalence:%
\[
\left(  m\mid b_{i}\text{ for all }i\in\left\{  1,2,\ldots,\ell\right\}
\right)  \ \Longleftrightarrow\ \left(  m\mid\gcd\left(  b_{1},b_{2}%
,\ldots,b_{\ell}\right)  \right)  .
\]


\textbf{(b)} The common divisors of $b_{1},b_{2},\ldots,b_{\ell}$ are
precisely the divisors of $\gcd\left(  b_{1},b_{2},\ldots,b_{\ell}\right)  $.

\textbf{(c)} We have $\operatorname*{Div}\left(  b_{1},b_{2},\ldots,b_{\ell
}\right)  =\operatorname*{Div}\left(  \gcd\left(  b_{1},b_{2},\ldots,b_{\ell
}\right)  \right)  $.

\textbf{(d)} If $\ell>0$, then%
\[
\gcd\left(  b_{1},b_{2},\ldots,b_{\ell}\right)  =\gcd\left(  \gcd\left(
b_{1},b_{2},\ldots,b_{\ell-1}\right)  ,b_{\ell}\right)  .
\]

\end{statement}

\textit{Proof of Statement 2:} \textbf{(d)} Let us begin with part
\textbf{(d)}. Assume that $\ell>0$ (though we already know that this is true).

Let $g=\gcd\left(  b_{1},b_{2},\ldots,b_{\ell}\right)  $ and $h=\gcd\left(
\gcd\left(  b_{1},b_{2},\ldots,b_{\ell-1}\right)  ,b_{\ell}\right)  $.

If the integers $b_{1},b_{2},\ldots,b_{\ell}$ are all $0$, then Statement 2
\textbf{(d)} holds\footnote{\textit{Proof.} Assume that $b_{1},b_{2}%
,\ldots,b_{\ell}$ are all $0$. Then, $\gcd\left(  b_{1},b_{2},\ldots,b_{\ell
}\right)  =0$ (by Definition \ref{def.ent.gcd.gcd}). Moreover, $b_{1}%
,b_{2},\ldots,b_{\ell-1}$ are all $0$ (since $b_{1},b_{2},\ldots,b_{\ell}$ are
all $0$), and thus $\gcd\left(  b_{1},b_{2},\ldots,b_{\ell-1}\right)  =0$.
Finally, $b_{\ell}=0$ (since $b_{1},b_{2},\ldots,b_{\ell}$ are all $0$).
Comparing $\gcd\left(  b_{1},b_{2},\ldots,b_{\ell}\right)  =0$ with
$\gcd\left(  \underbrace{\gcd\left(  b_{1},b_{2},\ldots,b_{\ell-1}\right)
}_{=0},\underbrace{b_{\ell}}_{=0}\right)  =\gcd\left(  0,0\right)  =0$, we
obtain $\gcd\left(  b_{1},b_{2},\ldots,b_{\ell}\right)  =\gcd\left(
\gcd\left(  b_{1},b_{2},\ldots,b_{\ell-1}\right)  ,b_{\ell}\right)  $. In
other words, Statement 2 \textbf{(d)} holds.}. Hence, for the rest of this
proof, we WLOG assume that the integers $b_{1},b_{2},\ldots,b_{\ell}$ are not
all $0$. Therefore, $\gcd\left(  b_{1},b_{2},\ldots,b_{\ell}\right)  $ is the
largest element of the set $\operatorname*{Div}\left(  b_{1},b_{2}%
,\ldots,b_{\ell}\right)  $ (by Definition \ref{def.ent.gcd.gcd}). In other
words, $g$ is the largest element of the set $\operatorname*{Div}\left(
b_{1},b_{2},\ldots,b_{\ell}\right)  $ (since $g=\gcd\left(  b_{1},b_{2}%
,\ldots,b_{\ell}\right)  $).

Furthermore, the two integers $\gcd\left(  b_{1},b_{2},\ldots,b_{\ell
-1}\right)  $ and $b_{\ell}$ are not all $0$\ \ \ \ \footnote{\textit{Proof.}
Assume the contrary. Thus, both $\gcd\left(  b_{1},b_{2},\ldots,b_{\ell
-1}\right)  $ and $b_{\ell}$ are $0$. Thus, in particular, $b_{\ell}=0$. If
the $\ell-1$ integers $b_{1},b_{2},\ldots,b_{\ell-1}$ were all $0$, then the
$\ell$ integers $b_{1},b_{2},\ldots,b_{\ell}$ would be all $0$ (since
$b_{\ell}=0$), which would contradict the fact that the integers $b_{1}%
,b_{2},\ldots,b_{\ell}$ are not all $0$. Hence, the $\ell-1$ integers
$b_{1},b_{2},\ldots,b_{\ell-1}$ are not all $0$. Thus, $\gcd\left(
b_{1},b_{2},\ldots,b_{\ell-1}\right)  $ is a positive integer (by Definition
\ref{def.ent.gcd.gcd}). Thus, $\gcd\left(  b_{1},b_{2},\ldots,b_{\ell
-1}\right)  >0$, which contradicts the fact that $\gcd\left(  b_{1}%
,b_{2},\ldots,b_{\ell-1}\right)  $ is $0$. This contradiction shows that our
assumption was false, qed.}. Hence, $\gcd\left(  \gcd\left(  b_{1}%
,b_{2},\ldots,b_{\ell-1}\right)  ,b_{\ell}\right)  $ is the largest element of
the set \newline$\operatorname*{Div}\left(  \gcd\left(  b_{1},b_{2}%
,\ldots,b_{\ell-1}\right)  ,b_{\ell}\right)  $ (by Definition
\ref{def.ent.gcd.gcd}). In other words, $h$ is the largest element of the set
$\operatorname*{Div}\left(  \gcd\left(  b_{1},b_{2},\ldots,b_{\ell-1}\right)
,b_{\ell}\right)  $ (since $h=\gcd\left(  \gcd\left(  b_{1},b_{2}%
,\ldots,b_{\ell-1}\right)  ,b_{\ell}\right)  $).

We intend to show that $g=h$. For that, it suffices to prove $g\leq h$ and
$h\leq g$.

\textit{Proof of }$g\leq h$\textit{:} Recall that $g$ is the largest element
of the set $\operatorname*{Div}\left(  b_{1},b_{2},\ldots,b_{\ell}\right)  $.
Therefore, $g\in\operatorname*{Div}\left(  b_{1},b_{2},\ldots,b_{\ell}\right)
$. In other words, $g$ is a common divisor of $b_{1},b_{2},\ldots,b_{\ell}$.
Hence, $g\mid b_{i}$ for each $i\in\left\{  1,2,\ldots,\ell\right\}  $. Thus,
in particular, $g\mid b_{i}$ for each $i\in\left\{  1,2,\ldots,\ell-1\right\}
$. But Statement 1 \textbf{(a)} (applied to $m=g$) shows that we have the
equivalence%
\[
\left(  g\mid b_{i}\text{ for all }i\in\left\{  1,2,\ldots,\ell-1\right\}
\right)  \ \Longleftrightarrow\ \left(  g\mid\gcd\left(  b_{1},b_{2}%
,\ldots,b_{\ell-1}\right)  \right)  .
\]
Hence, we have $g\mid\gcd\left(  b_{1},b_{2},\ldots,b_{\ell-1}\right)  $
(since we know that $g\mid b_{i}$ for all $i\in\left\{  1,2,\ldots
,\ell-1\right\}  $). Combining this with $g\mid b_{\ell}$, we conclude that
$g$ is a common divisor of $\gcd\left(  b_{1},b_{2},\ldots,b_{\ell-1}\right)
$ and $b_{\ell}$. In other words, $g\in\operatorname*{Div}\left(  \gcd\left(
b_{1},b_{2},\ldots,b_{\ell-1}\right)  ,b_{\ell}\right)  $. Therefore, $g\leq
h$ (since $h$ is the largest element of the set $\operatorname*{Div}\left(
\gcd\left(  b_{1},b_{2},\ldots,b_{\ell-1}\right)  ,b_{\ell}\right)  $).

\textit{Proof of }$h\leq g$\textit{:} We have%
\[
h=\gcd\left(  \gcd\left(  b_{1},b_{2},\ldots,b_{\ell-1}\right)  ,b_{\ell
}\right)  \mid\gcd\left(  b_{1},b_{2},\ldots,b_{\ell-1}\right)
\]
(by Proposition \ref{prop.ent.gcd.props1} \textbf{(f)}, applied to
$a=\gcd\left(  b_{1},b_{2},\ldots,b_{\ell-1}\right)  $ and $b=b_{\ell}$).
Also,%
\[
h=\gcd\left(  \gcd\left(  b_{1},b_{2},\ldots,b_{\ell-1}\right)  ,b_{\ell
}\right)  \mid b_{\ell}%
\]
(by Proposition \ref{prop.ent.gcd.props1} \textbf{(f)}, applied to
$a=\gcd\left(  b_{1},b_{2},\ldots,b_{\ell-1}\right)  $ and $b=b_{\ell}$).

But Statement 1 \textbf{(a)} (applied to $m=h$) shows that we have the
equivalence%
\[
\left(  h\mid b_{i}\text{ for all }i\in\left\{  1,2,\ldots,\ell-1\right\}
\right)  \ \Longleftrightarrow\ \left(  h\mid\gcd\left(  b_{1},b_{2}%
,\ldots,b_{\ell-1}\right)  \right)  .
\]
Thus, we have $\left(  h\mid b_{i}\text{ for all }i\in\left\{  1,2,\ldots
,\ell-1\right\}  \right)  $ (since we have $h\mid\gcd\left(  b_{1}%
,b_{2},\ldots,b_{\ell-1}\right)  $).

This divisibility $h\mid b_{i}$ holds not only for all $i\in\left\{
1,2,\ldots,\ell-1\right\}  $, but also for $i=\ell$ (because $h\mid b_{\ell}%
$). Thus, we conclude that $h\mid b_{i}$ for all $i\in\left\{  1,2,\ldots
,\ell\right\}  $. In other words, $h$ is a common divisor of $b_{1}%
,b_{2},\ldots,b_{\ell}$. In other words, $h\in\operatorname*{Div}\left(
b_{1},b_{2},\ldots,b_{\ell}\right)  $. Thus, $h\leq g$ (since $g$ is the
largest element of the set $\operatorname*{Div}\left(  b_{1},b_{2}%
,\ldots,b_{\ell}\right)  $).

Combining $h\leq g$ with $g\leq h$, we obtain $g=h$. In other words,
\[
\gcd\left(  b_{1},b_{2},\ldots,b_{\ell}\right)  =\gcd\left(  \gcd\left(
b_{1},b_{2},\ldots,b_{\ell-1}\right)  ,b_{\ell}\right)
\]
(since $g=\gcd\left(  b_{1},b_{2},\ldots,b_{\ell}\right)  $ and $h=\gcd\left(
\gcd\left(  b_{1},b_{2},\ldots,b_{\ell-1}\right)  ,b_{\ell}\right)  $). Hence,
Statement 2 \textbf{(d)} is proven.

\textbf{(a)} Let $m\in\mathbb{Z}$. Then, we have the equivalence%
\begin{align*}
&  \ \left(  m\mid b_{i}\text{ for all }i\in\left\{  1,2,\ldots,\ell\right\}
\right) \\
&  \Longleftrightarrow\ \left(  \underbrace{\left(  m\mid b_{i}\text{ for all
}i\in\left\{  1,2,\ldots,\ell-1\right\}  \right)  }%
_{\substack{\Longleftrightarrow\ \left(  m\mid\gcd\left(  b_{1},b_{2}%
,\ldots,b_{\ell-1}\right)  \right)  \\\text{(by Statement 1 \textbf{(a)})}%
}}\text{ and }m\mid b_{\ell}\right) \\
&  \Longleftrightarrow\ \left(  m\mid\gcd\left(  b_{1},b_{2},\ldots,b_{\ell
-1}\right)  \text{ and }m\mid b_{\ell}\right) \\
&  \Longleftrightarrow\ \left(  m\mid\underbrace{\gcd\left(  \gcd\left(
b_{1},b_{2},\ldots,b_{\ell-1}\right)  ,b_{\ell}\right)  }_{\substack{=\gcd
\left(  b_{1},b_{2},\ldots,b_{\ell}\right)  \\\text{(by Statement 2
\textbf{(d)}, which we have just proved)}}}\right) \\
&  \ \ \ \ \ \ \ \ \ \ \left(  \text{by Theorem \ref{thm.ent.gcd.uniprop}
\textbf{(a)}, applied to }a=\gcd\left(  b_{1},b_{2},\ldots,b_{\ell-1}\right)
\text{ and }b=b_{\ell}\right) \\
&  \Longleftrightarrow\ \left(  m\mid\gcd\left(  b_{1},b_{2},\ldots,b_{\ell
}\right)  \right)  .
\end{align*}
Thus, Statement 2 \textbf{(a)} follows.

Statement 2 \textbf{(b)} is a restatement of Statement 2 \textbf{(a)} (in the
same way that Theorem \ref{thm.ent.gcd.uniprop} \textbf{(b)} is a restatement
of Theorem \ref{thm.ent.gcd.uniprop} \textbf{(a)}).

Statement 2 \textbf{(c)} is a restatement of Statement 2 \textbf{(b)} (in the
same way that Theorem \ref{thm.ent.gcd.uniprop} \textbf{(c)} is a restatement
of Theorem \ref{thm.ent.gcd.uniprop} \textbf{(b)}).

We are thus done proving Statement 2.

In other words, we have proven that Theorem \ref{thm.ent.gcd.uniprop-mul}
holds for $k=\ell$. This completes the induction step. Thus, Theorem
\ref{thm.ent.gcd.uniprop-mul} is proven by induction.
\end{proof}

Theorem \ref{thm.ent.gcd.uniprop-mul} \textbf{(d)} is the reason why most
properties of gcds of multiple numbers can be derived from corresponding
properties of gcds of two numbers. For example, we can easily prove the
following analogue of Corollary \ref{cor.ent.gcd.sa,sb} for gcds of three numbers:

\begin{exercise}
\label{exe.ent.gcd.sa,sb,sc}Let $s,a,b,c\in\mathbb{Z}$. Prove that
$\gcd\left(  sa,sb,sc\right)  =\left\vert s\right\vert \gcd\left(
a,b,c\right)  $.
\end{exercise}

\begin{fineprint}
\begin{proof}
[Solution to Exercise \ref{exe.ent.gcd.sa,sb,sc}.]Corollary
\ref{cor.ent.gcd.sa,sb} yields
\begin{equation}
\gcd\left(  sa,sb\right)  =\left\vert s\right\vert \gcd\left(  a,b\right)  .
\label{sol.ent.gcd.sa,sb,sc.1}%
\end{equation}
But $\gcd\left(  a,b\right)  $ is a nonnegative integer (by the definition of
$\gcd\left(  a,b\right)  $). The equality (\ref{eq.ent.div.abs(xy)}) (applied
to $x=s$ and $y=\gcd\left(  a,b\right)  $) yields%
\begin{align}
\left\vert s\gcd\left(  a,b\right)  \right\vert  &  =\left\vert s\right\vert
\cdot\underbrace{\left\vert \gcd\left(  a,b\right)  \right\vert }%
_{\substack{=\gcd\left(  a,b\right)  \\\text{(since }\gcd\left(  a,b\right)
\text{ is nonnegative)}}}=\left\vert s\right\vert \gcd\left(  a,b\right)
\nonumber\\
&  =\gcd\left(  sa,sb\right)  \ \ \ \ \ \ \ \ \ \ \left(  \text{by
(\ref{sol.ent.gcd.sa,sb,sc.1})}\right)  . \label{sol.ent.gcd.sa,sb,sc.2}%
\end{align}


Now, Theorem \ref{thm.ent.gcd.uniprop-mul} \textbf{(d)} (applied to $3$ and
$\left(  a,b,c\right)  $ instead of $k$ and $\left(  b_{1},b_{2},\ldots
,b_{k}\right)  $) yields%
\begin{equation}
\gcd\left(  a,b,c\right)  =\gcd\left(  \gcd\left(  a,b\right)  ,c\right)  .
\label{sol.ent.gcd.sa,sb,sc.3}%
\end{equation}
The same argument (applied to $sa,sb,sc$ instead of $a,b,c$) yields%
\begin{align*}
&  \gcd\left(  sa,sb,sc\right) \\
&  =\gcd\left(  \underbrace{\gcd\left(  sa,sb\right)  }_{\substack{=\left\vert
s\gcd\left(  a,b\right)  \right\vert \\\text{(by (\ref{sol.ent.gcd.sa,sb,sc.2}%
))}}},sc\right)  =\gcd\left(  \left\vert s\gcd\left(  a,b\right)  \right\vert
,sc\right) \\
&  =\gcd\left(  s\gcd\left(  a,b\right)  ,sc\right)
\ \ \ \ \ \ \ \ \ \ \left(
\begin{array}
[c]{c}%
\text{by Exercise \ref{exe.ent.gcd.abs} \textbf{(b)}, applied to }s\gcd\left(
a,b\right) \\
\text{and }sc\text{ instead of }a\text{ and }b
\end{array}
\right) \\
&  =\left\vert s\right\vert \underbrace{\gcd\left(  \gcd\left(  a,b\right)
,c\right)  }_{\substack{=\gcd\left(  a,b,c\right)  \\\text{(by
(\ref{sol.ent.gcd.sa,sb,sc.3}))}}}\ \ \ \ \ \ \ \ \ \ \left(
\begin{array}
[c]{c}%
\text{by Corollary \ref{cor.ent.gcd.sa,sb}, applied to }\gcd\left(  a,b\right)
\\
\text{and }c\text{ instead of }a\text{ and }b
\end{array}
\right) \\
&  =\left\vert s\right\vert \gcd\left(  a,b,c\right)  .
\end{align*}
This solves Exercise \ref{exe.ent.gcd.sa,sb,sc}.
\end{proof}
\end{fineprint}

Bezout's theorem (Theorem \ref{thm.ent.gcd.bezout}) also holds for any finite
number of integers:

\begin{theorem}
\label{thm.ent.gcd.bezout-mul}Let $b_{1},b_{2},\ldots,b_{k}$ be integers.
Then, there exist integers $x_{1},x_{2},\ldots,x_{k}$ such that%
\[
\gcd\left(  b_{1},b_{2},\ldots,b_{k}\right)  =x_{1}b_{1}+x_{2}b_{2}%
+\cdots+x_{k}b_{k}.
\]

\end{theorem}

Once again, we can restate Theorem \ref{thm.ent.gcd.bezout-mul} by using the
concept of a $\mathbb{Z}$-linear combination. Let us define this concept finally:

\begin{definition}
Let $b_{1},b_{2},\ldots,b_{k}$ be numbers. A $\mathbb{Z}$\textit{-linear
combination} of $b_{1},b_{2},\ldots,b_{k}$ shall mean a number of the form
$x_{1}b_{1}+x_{2}b_{2}+\cdots+x_{k}b_{k}$, where $x_{1},x_{2},\ldots,x_{k}$
are integers.
\end{definition}

Thus, Theorem \ref{thm.ent.gcd.bezout-mul} can be restated as follows:

\begin{theorem}
\label{thm.ent.gcd.bezout-mul'}Let $b_{1},b_{2},\ldots,b_{k}$ be integers.
Then, $\gcd\left(  b_{1},b_{2},\ldots,b_{k}\right)  $ is a $\mathbb{Z}$-linear
combination of $b_{1},b_{2},\ldots,b_{k}$.
\end{theorem}

\begin{proof}
[Proof of Theorem \ref{thm.ent.gcd.bezout-mul'}.]We shall prove this by
induction on $k$:

\textit{Induction base:} Recall that the empty list $\left(  {}\right)  $
satisfies $\gcd\left(  {}\right)  =0$ (by Definition \ref{def.ent.gcd.gcd},
since all entries of the empty list are $0$). But $0$ is a $\mathbb{Z}$-linear
combination of an empty list of numbers, because $0=\left(  \text{empty
sum}\right)  $. Combining these facts, we conclude that $\gcd\left(
{}\right)  $ is a $\mathbb{Z}$-linear combination of an empty list of numbers.
But this is precisely the claim of Theorem \ref{thm.ent.gcd.bezout-mul'} for
$k=0$. Thus, Theorem \ref{thm.ent.gcd.bezout-mul'} holds for $k=0$. This
completes the induction base.

\textit{Induction step:} Let $\ell$ be a positive integer. Assume that Theorem
\ref{thm.ent.gcd.bezout-mul'} holds for $k=\ell-1$. We must prove that Theorem
\ref{thm.ent.gcd.bezout-mul} holds for $k=\ell$.

We have assumed that Theorem \ref{thm.ent.gcd.bezout-mul'} holds for
$k=\ell-1$. In other words, the following statement holds:

\begin{statement}
\textit{Statement 1:} Let $b_{1},b_{2},\ldots,b_{\ell-1}$ be integers. Then,
$\gcd\left(  b_{1},b_{2},\ldots,b_{\ell-1}\right)  $ is a $\mathbb{Z}$-linear
combination of $b_{1},b_{2},\ldots,b_{\ell-1}$.
\end{statement}

Our goal is to prove that Theorem \ref{thm.ent.gcd.bezout-mul'} holds for
$k=\ell$. In other words, we must prove the following statement:

\begin{statement}
\textit{Statement 2:} Let $b_{1},b_{2},\ldots,b_{\ell}$ be integers. Then,
$\gcd\left(  b_{1},b_{2},\ldots,b_{\ell}\right)  $ is a $\mathbb{Z}$-linear
combination of $b_{1},b_{2},\ldots,b_{\ell}$.
\end{statement}

\textit{Proof of Statement 2:} Statement 1 shows that $\gcd\left(  b_{1}%
,b_{2},\ldots,b_{\ell-1}\right)  $ is a $\mathbb{Z}$-linear combination of
$b_{1},b_{2},\ldots,b_{\ell-1}$. In other words, there exist $\ell-1$ integers
$y_{1},y_{2},\ldots,y_{\ell-1}$ such that%
\[
\gcd\left(  b_{1},b_{2},\ldots,b_{\ell-1}\right)  =y_{1}b_{1}+y_{2}%
b_{2}+\cdots+y_{\ell-1}b_{\ell-1}.
\]
Consider these $y_{1},y_{2},\ldots,y_{\ell-1}$.

Furthermore, Theorem \ref{thm.ent.gcd.bezout} (applied to $a=\gcd\left(
b_{1},b_{2},\ldots,b_{\ell-1}\right)  $ and $b=b_{\ell}$) yields that there
exist two integers $x$ and $y$ such that%
\[
\gcd\left(  \gcd\left(  b_{1},b_{2},\ldots,b_{\ell-1}\right)  ,b_{\ell
}\right)  =x\gcd\left(  b_{1},b_{2},\ldots,b_{\ell-1}\right)  +yb_{\ell}.
\]
Consider these $x$ and $y$.

Now, $\ell>0$; thus, Theorem \ref{thm.ent.gcd.uniprop-mul} \textbf{(d)}
(applied to $k=\ell$) yields%
\begin{align*}
\gcd\left(  b_{1},b_{2},\ldots,b_{\ell}\right)   &  =\gcd\left(  \gcd\left(
b_{1},b_{2},\ldots,b_{\ell-1}\right)  ,b_{\ell}\right) \\
&  =x\underbrace{\gcd\left(  b_{1},b_{2},\ldots,b_{\ell-1}\right)  }%
_{=y_{1}b_{1}+y_{2}b_{2}+\cdots+y_{\ell-1}b_{\ell-1}}+yb_{\ell}\\
&  =x\left(  y_{1}b_{1}+y_{2}b_{2}+\cdots+y_{\ell-1}b_{\ell-1}\right)
+yb_{\ell}\\
&  =xy_{1}b_{1}+xy_{2}b_{2}+\cdots+xy_{\ell-1}b_{\ell-1}+yb_{\ell}.
\end{align*}
This is clearly a $\mathbb{Z}$-linear combination of the $b_{1},b_{2}%
,\ldots,b_{\ell}$. Thus, $\gcd\left(  b_{1},b_{2},\ldots,b_{\ell}\right)  $ is
a $\mathbb{Z}$-linear combination of $b_{1},b_{2},\ldots,b_{\ell}$. So
Statement 2 is proven.

In other words, we have proven that Theorem \ref{thm.ent.gcd.bezout-mul'}
holds for $k=\ell$. This completes the induction step. Thus, Theorem
\ref{thm.ent.gcd.bezout-mul'} is proven by induction.
\end{proof}

\begin{proof}
[Proof of Theorem \ref{thm.ent.gcd.bezout-mul}.]We have just proven Theorem
\ref{thm.ent.gcd.bezout-mul'}, which is a restatement of Theorem
\ref{thm.ent.gcd.bezout-mul}. Thus, Theorem \ref{thm.ent.gcd.bezout-mul} is
also proven.
\end{proof}

\begin{theorem}
\label{thm.ent.gcd.split}Let $b_{1},b_{2},\ldots,b_{k}$ be integers, and let
$c_{1},c_{2},\ldots,c_{\ell}$ be integers. Then,%
\begin{align*}
&  \gcd\left(  b_{1},b_{2},\ldots,b_{k},c_{1},c_{2},\ldots,c_{\ell}\right) \\
&  =\gcd\left(  \gcd\left(  b_{1},b_{2},\ldots,b_{k}\right)  ,\gcd\left(
c_{1},c_{2},\ldots,c_{\ell}\right)  \right)  .
\end{align*}

\end{theorem}

Our proof of this theorem will rely on a simple trick, which we state as a lemma:

\begin{lemma}
\label{lem.ent.gcd.yoneda}Let $a$ and $b$ be two integers.

\textbf{(a)} If each $m\in\mathbb{Z}$ satisfies the implication $\left(  m\mid
a\right)  \Longrightarrow\left(  m\mid b\right)  $, then $a\mid b$.

\textbf{(b)} If each $m\in\mathbb{Z}$ satisfies the equivalence $\left(  m\mid
a\right)  \Longleftrightarrow\left(  m\mid b\right)  $, then $\left\vert
a\right\vert =\left\vert b\right\vert $.
\end{lemma}

Lemma \ref{lem.ent.gcd.yoneda} \textbf{(b)} says that the divisors of an
integer $a$ uniquely determine $\left\vert a\right\vert $ (that is, they
uniquely determine $a$ up to sign). Thus, when you want to prove that two
integers have the same absolute values, it suffices to prove that they have
the same divisors. If you know that your two integers are nonnegative, then
you can prove this way that they are equal (since their absolute values are
just themselves). This is exactly how we will prove that the left and right
hand sides in Theorem \ref{thm.ent.gcd.split} are equal.

\begin{proof}
[Proof of Lemma \ref{lem.ent.gcd.yoneda}.]\textbf{(a)} Assume that each
$m\in\mathbb{Z}$ satisfies the implication $\left(  m\mid a\right)
\Longrightarrow\left(  m\mid b\right)  $. Then, applying this to $m=a$, we
obtain the implication $\left(  a\mid a\right)  \Longrightarrow\left(  a\mid
b\right)  $. Since $a\mid a$ holds, we thus obtain $a\mid b$. This proves
Lemma \ref{lem.ent.gcd.yoneda} \textbf{(a)}.

\textbf{(b)} Assume that each $m\in\mathbb{Z}$ satisfies the equivalence
$\left(  m\mid a\right)  \Longleftrightarrow\left(  m\mid b\right)  $. Thus,
each $m\in\mathbb{Z}$ satisfies the implication $\left(  m\mid a\right)
\Longrightarrow\left(  m\mid b\right)  $ (since this implication is part of
the equivalence we just assumed). Thus, Lemma \ref{lem.ent.gcd.yoneda}
\textbf{(a)} yields $a\mid b$.

Recall again that each $m\in\mathbb{Z}$ satisfies the equivalence $\left(
m\mid a\right)  \Longleftrightarrow\left(  m\mid b\right)  $. Thus, each
$m\in\mathbb{Z}$ satisfies the implication $\left(  m\mid b\right)
\Longrightarrow\left(  m\mid a\right)  $ (since this implication is also part
of the equivalence). Hence, Lemma \ref{lem.ent.gcd.yoneda} \textbf{(b)} yields
$b\mid a$.

Hence, Exercise \ref{exe.ent.div.abba} yields $\left\vert a\right\vert
=\left\vert b\right\vert $. This proves Lemma \ref{lem.ent.gcd.yoneda}.
\end{proof}

Lemma \ref{lem.ent.gcd.yoneda} is a simple case of what is known in category
theory as the \textit{Yoneda lemma}.

\begin{proof}
[Proof of Theorem \ref{thm.ent.gcd.split}.]Let $m\in\mathbb{Z}$. Theorem
\ref{thm.ent.gcd.uniprop-mul} \textbf{(a)} (applied to $\left(  b_{1}%
,b_{2},\ldots,b_{k},c_{1},c_{2},\ldots,c_{\ell}\right)  $ instead of $\left(
b_{1},b_{2},\ldots,b_{k}\right)  $) shows that we have the following
equivalence:%
\begin{align*}
&  \ \left(  \left(  m\mid b_{i}\text{ for all }i\in\left\{  1,2,\ldots
,k\right\}  \right)  \text{ and }\left(  m\mid c_{i}\text{ for all }%
i\in\left\{  1,2,\ldots,\ell\right\}  \right)  \right) \\
&  \Longleftrightarrow\ \left(  m\mid\gcd\left(  b_{1},b_{2},\ldots
,b_{k},c_{1},c_{2},\ldots,c_{\ell}\right)  \right)  .
\end{align*}
Hence, we have the following chain of equivalences:%
\begin{align*}
&  \ \left(  m\mid\gcd\left(  b_{1},b_{2},\ldots,b_{k},c_{1},c_{2}%
,\ldots,c_{\ell}\right)  \right) \\
&  \Longleftrightarrow\ \left(  \underbrace{\left(  m\mid b_{i}\text{ for all
}i\in\left\{  1,2,\ldots,k\right\}  \right)  }_{\substack{\Longleftrightarrow
\ \left(  m\mid\gcd\left(  b_{1},b_{2},\ldots,b_{k}\right)  \right)
\\\text{(by Theorem \ref{thm.ent.gcd.uniprop-mul} \textbf{(a)})}}}\text{ and
}\underbrace{\left(  m\mid c_{i}\text{ for all }i\in\left\{  1,2,\ldots
,\ell\right\}  \right)  }_{\substack{\Longleftrightarrow\ \left(  m\mid
\gcd\left(  c_{1},c_{2},\ldots,c_{\ell}\right)  \right)  \\\text{(by Theorem
\ref{thm.ent.gcd.uniprop-mul} \textbf{(a)},}\\\text{applied to }\left(
c_{1},c_{2},\ldots,c_{\ell}\right)  \text{ instead of }\left(  b_{1}%
,b_{2},\ldots,b_{k}\right)  \text{)}}}\right) \\
&  \Longleftrightarrow\ \left(  m\mid\gcd\left(  b_{1},b_{2},\ldots
,b_{k}\right)  \text{ and }m\mid\gcd\left(  c_{1},c_{2},\ldots,c_{\ell
}\right)  \right) \\
&  \Longleftrightarrow\ \left(  m\mid\gcd\left(  \gcd\left(  b_{1}%
,b_{2},\ldots,b_{k}\right)  ,\gcd\left(  c_{1},c_{2},\ldots,c_{\ell}\right)
\right)  \right) \\
&  \ \ \ \ \ \ \ \ \ \ \left(
\begin{array}
[c]{c}%
\text{by Theorem \ref{thm.ent.gcd.uniprop} \textbf{(a)},}\\
\text{applied to }a=\gcd\left(  b_{1},b_{2},\ldots,b_{k}\right)  \text{ and
}b=\gcd\left(  c_{1},c_{2},\ldots,c_{\ell}\right)
\end{array}
\right)  .
\end{align*}


Now, forget that we fixed $m$. We thus have shown that each $m\in\mathbb{Z}$
satisfies the equivalence%
\begin{align*}
&  \ \left(  m\mid\gcd\left(  b_{1},b_{2},\ldots,b_{k},c_{1},c_{2}%
,\ldots,c_{\ell}\right)  \right) \\
&  \Longleftrightarrow\ \left(  m\mid\gcd\left(  \gcd\left(  b_{1}%
,b_{2},\ldots,b_{k}\right)  ,\gcd\left(  c_{1},c_{2},\ldots,c_{\ell}\right)
\right)  \right)  .
\end{align*}
Hence, Lemma \ref{lem.ent.gcd.yoneda} (applied to $a=\gcd\left(  b_{1}%
,b_{2},\ldots,b_{k},c_{1},c_{2},\ldots,c_{\ell}\right)  $ and \newline%
$b=\gcd\left(  \gcd\left(  b_{1},b_{2},\ldots,b_{k}\right)  ,\gcd\left(
c_{1},c_{2},\ldots,c_{\ell}\right)  \right)  $) yields%
\begin{align}
&  \left\vert \gcd\left(  b_{1},b_{2},\ldots,b_{k},c_{1},c_{2},\ldots,c_{\ell
}\right)  \right\vert \nonumber\\
&  =\left\vert \gcd\left(  \gcd\left(  b_{1},b_{2},\ldots,b_{k}\right)
,\gcd\left(  c_{1},c_{2},\ldots,c_{\ell}\right)  \right)  \right\vert .
\label{pf.thm.ent.gcd.split.abs-equal}%
\end{align}


But a gcd of integers is always nonnegative (by Definition
\ref{def.ent.gcd.gcd}); thus, the absolute value of a gcd is always this gcd
itself. Therefore, we can remove the absolute value signs on both sides of
(\ref{pf.thm.ent.gcd.split.abs-equal}). We thus obtain%
\[
\gcd\left(  b_{1},b_{2},\ldots,b_{k},c_{1},c_{2},\ldots,c_{\ell}\right)
=\gcd\left(  \gcd\left(  b_{1},b_{2},\ldots,b_{k}\right)  ,\gcd\left(
c_{1},c_{2},\ldots,c_{\ell}\right)  \right)  .
\]
This proves Theorem \ref{thm.ent.gcd.split}.
\end{proof}

\begin{center}
\textbf{2019-02-06 lecture}
\end{center}

\subsection{Coprime integers}

\subsubsection{Definition}

The concept of a gcd leads to one of the most important notions of number theory:

\begin{definition}
\label{def.ent.coprime.coprime}Let $a$ and $b$ be two integers. We say that
$a$ is \textit{coprime} to $b$ if and only if $\gcd\left(  a,b\right)  =1$.
\end{definition}

Instead of \textquotedblleft coprime\textquotedblright, some authors say
\textquotedblleft relatively prime\textquotedblright\ or even
\textquotedblleft prime\textquotedblright\ (but the latter language risks
confusion with a more standard notion of \textquotedblleft
prime\textquotedblright\ that we will see later on.)

\begin{example}
\label{exa.ent.coprime.1}\textbf{(a)} The number $2$ is coprime to $3$, since
$\gcd\left(  2,3\right)  =1$.

\textbf{(b)} The number $6$ is not coprime to $15$, since $\gcd\left(
6,15\right)  =3\neq1$.

\textbf{(c)} Let $a$ be an integer. We claim (as a generalization of part
\textbf{(a)}) that the number $a$ is coprime to $a+1$. To prove this, we note
that%
\begin{align*}
\gcd\left(  a,\underbrace{a}_{=1a}+1\right)   &  =\gcd\left(  a,1a+1\right)
=\gcd\left(  a,1\right) \\
&  \ \ \ \ \ \ \ \ \ \ \left(  \text{by Proposition \ref{prop.ent.gcd.props1}
\textbf{(c)}, applied to }u=1\text{ and }b=1\right) \\
&  \mid1\ \ \ \ \ \ \ \ \ \ \left(  \text{by Proposition
\ref{prop.ent.gcd.props1} \textbf{(e)}, applied to }b=1\right)  ,
\end{align*}
and thus $\gcd\left(  a,a+1\right)  =1$ (since $\gcd\left(  a,a+1\right)  $ is
a nonnegative integer), which means that $a$ is coprime to $a+1$.

\textbf{(d)} Let $a$ be an integer. When is $a$ coprime to $a+2$? If we try to
compute $\gcd\left(  a,a+2\right)  $, we find%
\begin{align*}
\gcd\left(  a,\underbrace{a}_{=1a}+2\right)   &  =\gcd\left(  a,1a+2\right)
=\gcd\left(  a,2\right) \\
&  \ \ \ \ \ \ \ \ \ \ \left(  \text{by Proposition \ref{prop.ent.gcd.props1}
\textbf{(c)}, applied to }u=1\text{ and }b=2\right)  .
\end{align*}
It remains to find $\gcd\left(  a,2\right)  $. Proposition
\ref{prop.ent.gcd.props1} \textbf{(e)} (applied to $b=2$) yields $\gcd\left(
a,2\right)  \mid a$ and $\gcd\left(  a,2\right)  \mid2$. Since $\gcd\left(
a,2\right)  $ is a nonnegative integer and is a divisor of $2$ (because
$\gcd\left(  a,2\right)  \mid2$), we see that $\gcd\left(  a,2\right)  $ must
be either $1$ or $2$ (since the only nonnegative divisors of $2$ are $1$ and
$2$). If $a$ is even, then $2$ is a common divisor of $a$ and $2$, and thus
must be the greatest common divisor of $a$ and $2$ (because a common divisor
of $a$ and $2$ cannot be greater than $2$); in other words, we have
$\gcd\left(  a,2\right)  =2$ in this case. On the other hand, if $a$ is odd,
then $2$ is not a common divisor of $a$ and $2$ (since $2$ does not divide
$a$), and thus cannot be the greatest common divisor of $a$ and $2$; hence, in
this case, we have $\gcd\left(  a,2\right)  \neq2$ and thus $\gcd\left(
a,2\right)  =1$. Summarizing, we conclude that%
\[
\gcd\left(  a,2\right)  =%
\begin{cases}
2, & \text{if }a\text{ is even};\\
1, & \text{if }a\text{ is odd.}%
\end{cases}
\]
Now, recall that $\gcd\left(  a,a+2\right)  =\gcd\left(  a,2\right)  =%
\begin{cases}
2, & \text{if }a\text{ is even};\\
1, & \text{if }a\text{ is odd.}%
\end{cases}
$ Hence, $a$ is coprime to $a+2$ if and only if $a$ is odd.
\end{example}

Following the book \cite{GKP}, we introduce a slightly quaint notation:

\begin{definition}
\label{def.ent.coprime.perp}Let $a$ and $b$ be two integers. We write
\textquotedblleft$a\perp b$\textquotedblright\ to signify that $a$ is coprime
to $b$.
\end{definition}

Note that the \textquotedblleft$\perp$\textquotedblright\ relation is symmetric:

\begin{proposition}
\label{prop.ent.coprime.perp-symm}Let $a$ and $b$ be two integers. Then,
$a\perp b$ if and only if $b\perp a$.
\end{proposition}

\begin{proof}
[Proof of Proposition \ref{prop.ent.coprime.perp-symm}.]We have the following
chain of equivalences:%
\begin{align*}
\left(  a\perp b\right)  \  &  \Longleftrightarrow\ \left(  a\text{ is coprime
to }b\right)  \ \ \ \ \ \ \ \ \ \ \left(  \text{by the definition of
\textquotedblleft}\perp\text{\textquotedblright}\right) \\
&  \Longleftrightarrow\ \left(  \gcd\left(  a,b\right)  =1\right)
\ \ \ \ \ \ \ \ \ \ \left(  \text{by the definition of \textquotedblleft
coprime\textquotedblright}\right) \\
&  \Longleftrightarrow\ \left(  \gcd\left(  b,a\right)  =1\right)
\ \ \ \ \ \ \ \ \ \ \left(
\begin{array}
[c]{c}%
\text{since Proposition \ref{prop.ent.gcd.props1} \textbf{(b)}}\\
\text{yields }\gcd\left(  a,b\right)  =\gcd\left(  b,a\right)
\end{array}
\right) \\
&  \Longleftrightarrow\ \left(  b\text{ is coprime to }a\right)
\ \ \ \ \ \ \ \ \ \ \left(  \text{by the definition of \textquotedblleft
coprime\textquotedblright}\right) \\
&  \Longleftrightarrow\ \left(  b\perp a\right)  \ \ \ \ \ \ \ \ \ \ \left(
\text{by the definition of \textquotedblleft}\perp\text{\textquotedblright%
}\right)  .
\end{align*}
This proves Proposition \ref{prop.ent.coprime.perp-symm}.
\end{proof}

\begin{definition}
Let $a$ and $b$ be two integers. Proposition \ref{prop.ent.coprime.perp-symm}
shows that $a$ is coprime to $b$ if and only if $b$ is coprime to $a$. Hence,
we shall sometimes use a more symmetric terminology for this situation: We
shall say that \textquotedblleft$a$ and $b$ \textit{are coprime}%
\textquotedblright\ to mean that $a$ is coprime to $b$ (or, equivalently, that
$b$ is coprime to $a$).
\end{definition}

\subsubsection{Properties of coprime integers}

We can now state multiple theorems about coprime numbers. The first one states
that we can \textquotedblleft cancel\textquotedblright\ a factor $b$ from a
divisibility $a\mid bc$ as long as this factor is coprime to $a$:

\begin{theorem}
\label{thm.ent.coprime.cancel}Let $a,b,c\in\mathbb{Z}$ satisfy $a\mid bc$ and
$a\perp b$. Then, $a\mid c$.
\end{theorem}

\begin{proof}
[Proof of Theorem \ref{thm.ent.coprime.cancel}.]We have $a\perp b$; in other
words, $a$ is coprime to $b$ (by Definition \ref{def.ent.coprime.perp}). In
other words, $\gcd\left(  a,b\right)  =1$ (by the definition of
\textquotedblleft coprime\textquotedblright). Now, Theorem
\ref{thm.ent.gcd.cancel} yields $a\mid\underbrace{\gcd\left(  a,b\right)
}_{=1}\cdot c=c$. This proves Theorem \ref{thm.ent.coprime.cancel}.
\end{proof}

I like to think of Theorem \ref{thm.ent.coprime.combine} as a way of removing
\textquotedblleft unsolicited guests\textquotedblright\ from divisibilities.
Indeed, it says that we can remove the factor $b$ from $a\mid bc$ if we know
that $b$ is \textquotedblleft unrelated\textquotedblright\ (i.e., coprime) to
$a$.

The next theorem lets us \textquotedblleft combine\textquotedblright\ two
divisibilities $a\mid c$ and $b\mid c$ to $ab\mid c$ as long as $a$ and $b$
are coprime:

\begin{theorem}
\label{thm.ent.coprime.combine}Let $a,b,c\in\mathbb{Z}$ satisfy $a\mid c$ and
$b\mid c$ and $a\perp b$. Then, $ab\mid c$.
\end{theorem}

\begin{proof}
[Proof of Theorem \ref{thm.ent.coprime.combine}.]We have $a\perp b$; in other
words, $a$ is coprime to $b$ (by Definition \ref{def.ent.coprime.perp}). In
other words, $\gcd\left(  a,b\right)  =1$ (by the definition of
\textquotedblleft coprime\textquotedblright). Now, Theorem
\ref{thm.ent.gcd.combine} yields $ab\mid\underbrace{\gcd\left(  a,b\right)
}_{=1}\cdot c=c$. This proves Theorem \ref{thm.ent.coprime.combine}.
\end{proof}

Theorem \ref{thm.ent.coprime.combine} can be restated as follows: If $a$ and
$b$ are two coprime divisors of an integer $c$, then $ab$ is also a divisor of
$c$. This is often helpful when proving divisibilities where the left hand
side (i.e., the number in front of the \textquotedblleft$\mid$%
\textquotedblright\ sign) can be split into a product of two mutually coprime
factors. Similar reasoning works with several coprime factors (see Exercise
\ref{exe.ent.coprime.combinek} below).

The next theorem (still part of the fallout of Bezout's theorem) is important,
but we will not truly appreciate it until later:

\begin{theorem}
\label{thm.ent.coprime.modinv}Let $a,n\in\mathbb{Z}$.

\textbf{(a)} There exists a $b\in\mathbb{Z}$ such that $ab\equiv\gcd\left(
a,n\right)  \operatorname{mod}n$.

\textbf{(b)} If $a\perp n$, then there exists an $a^{\prime}\in\mathbb{Z}$
such that $aa^{\prime}\equiv1\operatorname{mod}n$.

\textbf{(c)} If there exists an $a^{\prime}\in\mathbb{Z}$ such that
$aa^{\prime}\equiv1\operatorname{mod}n$, then $a\perp n$.
\end{theorem}

If $a,n\in\mathbb{Z}$, then an integer $a^{\prime}\in\mathbb{Z}$ satisfying
$aa^{\prime}\equiv1\operatorname{mod}n$ is called a \textit{modular inverse}
of $a$ modulo $n$. The word \textquotedblleft modular
inverse\textquotedblright\ is chosen in analogy to the usual concept of an
\textquotedblleft inverse\textquotedblright\ (which stands for an integer
$a^{\prime}\in\mathbb{Z}$ satisfying $aa^{\prime}=1$; this exists if and only
if $a$ equals $1$ or $-1$). Theorem \ref{thm.ent.coprime.modinv} \textbf{(b)}
shows that such a modular inverse always exists when $a\perp n$; Theorem
\ref{thm.ent.coprime.modinv} \textbf{(c)} is the converse of this statement
(i.e., it says that if a modular inverse of $a$ modulo $n$ exists, then
$a\perp n$).

\begin{proof}
[Proof of Theorem \ref{thm.ent.coprime.modinv}.]\textbf{(a)} Theorem
\ref{thm.ent.gcd.bezout} (applied to $b=n$) yields that there exist integers
$x\in\mathbb{Z}$ and $y\in\mathbb{Z}$ such that $\gcd\left(  a,n\right)
=xa+yn$. Consider these $x$ and $y$. We have $ax=xa\equiv
xa+yn\operatorname{mod}n$ (since $xa-\left(  xa+yn\right)  =-yn=n\left(
-y\right)  $ is clearly divisible by $n$). Thus, $ax\equiv xa+yn=\gcd\left(
a,n\right)  \operatorname{mod}n$. Thus, there exists a $a^{\prime}%
\in\mathbb{Z}$ such that $aa^{\prime}\equiv\left(  a,n\right)
\operatorname{mod}n$ (namely, $a^{\prime}=x$). This proves Theorem
\ref{thm.ent.coprime.modinv} \textbf{(a)}.

\textbf{(b)} Assume that $a\perp n$. In other words, $a$ is coprime to $n$ (by
Definition \ref{def.ent.coprime.perp}). In other words, $\gcd\left(
a,n\right)  =1$ (by the definition of \textquotedblleft
coprime\textquotedblright). Now, Theorem \ref{thm.ent.coprime.modinv}
\textbf{(a)} yields that there exists a $a^{\prime}\in\mathbb{Z}$ such that
$aa^{\prime}\equiv\gcd\left(  a,n\right)  \operatorname{mod}n$. In view of
$\gcd\left(  a,n\right)  =1$, this rewrites as follows: There exists an
$a^{\prime}\in\mathbb{Z}$ such that $aa^{\prime}\equiv1\operatorname{mod}n$.
This proves Theorem \ref{thm.ent.coprime.modinv} \textbf{(b)}.

\textbf{(c)} Assume that there exists an $a^{\prime}\in\mathbb{Z}$ such that
$aa^{\prime}\equiv1\operatorname{mod}n$. Consider this $a^{\prime}$.

Proposition \ref{prop.ent.gcd.props1} \textbf{(f)} yields $\gcd\left(
a,n\right)  \mid a$ and $\gcd\left(  a,n\right)  \mid n$. Set $g=\gcd\left(
a,n\right)  $. Then, $g$ is a nonnegative integer.

Now, $g=\gcd\left(  a,n\right)  \mid a\mid aa^{\prime}$, so that $aa^{\prime
}\equiv0\operatorname{mod}g$. But also $g=\gcd\left(  a,n\right)  \mid n$.
Hence, from $aa^{\prime}\equiv1\operatorname{mod}n$, we obtain $aa^{\prime
}\equiv1\operatorname{mod}g$ (by Proposition \ref{prop.ent.mod.basics}
\textbf{(e)}, applied to $g$, $aa^{\prime}$ and $1$ instead of $m$, $a$ and
$b$). Hence, $1\equiv aa^{\prime}\equiv0\operatorname{mod}g$. Equivalently,
$g\mid1-0=1$. Thus, $g$ is a divisor of $1$, and hence must be $1$ itself
(since $g$ is a nonnegative integer). Thus, $\gcd\left(  a,n\right)  =g=1$. In
other words, $a$ is coprime to $n$. In other words, $a\perp n$. This proves
Theorem \ref{thm.ent.coprime.modinv} \textbf{(c)}.
\end{proof}

\begin{theorem}
\label{thm.ent.coprime.ab-to-c}Let $a,b,c\in\mathbb{Z}$ such that $a\perp c$
and $b\perp c$. Then, $ab\perp c$.
\end{theorem}

\begin{proof}
[Proof of Theorem \ref{thm.ent.coprime.ab-to-c}.]Theorem
\ref{thm.ent.coprime.modinv} \textbf{(b)} (applied to $n=c$) yields that there
exists an $a^{\prime}\in\mathbb{Z}$ such that $aa^{\prime}\equiv
1\operatorname{mod}c$. Consider this $a^{\prime}$.

Theorem \ref{thm.ent.coprime.modinv} \textbf{(b)} (applied to $b$ and $c$
instead of $a$ and $n$) yields that there exists a $b^{\prime}\in\mathbb{Z}$
such that $bb^{\prime}\equiv1\operatorname{mod}c$. Consider this $b^{\prime}$.

Multiplying the two congruences $aa^{\prime}\equiv1\operatorname{mod}c$ and
$bb^{\prime}\equiv1\operatorname{mod}c$, we obtain $\left(  aa^{\prime
}\right)  \left(  bb^{\prime}\right)  \equiv1\cdot1=1\operatorname{mod}c$.

Now, define the integers $r=ab$ and $s=a^{\prime}b^{\prime}$. Then,
$\underbrace{r}_{=ab}\underbrace{s}_{=a^{\prime}b^{\prime}}=\left(  ab\right)
\left(  a^{\prime}b^{\prime}\right)  =\left(  aa^{\prime}\right)  \left(
bb^{\prime}\right)  \equiv1\operatorname{mod}c$. Hence, there exists an
$r^{\prime}\in\mathbb{Z}$ such that $rr^{\prime}\equiv1\operatorname{mod}c$
(namely, $r^{\prime}=s$). Thus, Theorem \ref{thm.ent.coprime.modinv}
\textbf{(c)} (applied to $r$ and $c$ instead of $a$ and $n$) yields that
$r\perp c$. In view of $r=ab$, this rewrites as $ab\perp c$. This proves
Theorem \ref{thm.ent.coprime.ab-to-c}.
\end{proof}

Let us generalize Theorem \ref{thm.ent.coprime.ab-to-c} to products of several
numbers instead of just the two numbers $a$ and $b$:

\begin{exercise}
\label{exe.ent.coprime.ab-to-ck}Let $c\in\mathbb{Z}$. Let $a_{1},a_{2}%
,\ldots,a_{k}$ be integers such that each $i\in\left\{  1,2,\ldots,k\right\}
$ satisfies $a_{i}\perp c$. Prove that $a_{1}a_{2}\cdots a_{k}\perp c$.
\end{exercise}

\begin{fineprint}
\begin{proof}
[Solution to Exercise \ref{exe.ent.coprime.combinek}.]Let us prove that%
\begin{equation}
a_{1}a_{2}\cdots a_{i}\perp c\ \ \ \ \ \ \ \ \ \ \text{for each }i\in\left\{
0,1,\ldots,k\right\}  . \label{sol.ent.coprime.ab-to-ck.goal}%
\end{equation}


\textit{Proof of (\ref{sol.ent.coprime.ab-to-ck.goal}):} We shall prove
(\ref{sol.ent.coprime.ab-to-ck.goal}) by induction on $i$:

\textit{Induction base:} Proposition \ref{prop.ent.gcd.props1} \textbf{(f)}
yields $\gcd\left(  1,c\right)  \mid1$ and thus $\gcd\left(  1,c\right)  =1$
(since $\gcd\left(  1,c\right)  $ is a nonnegative integer). In other words,
$1$ is coprime to $c$. In other words, $1\perp c$. Now, $a_{1}a_{2}\cdots
a_{0}=\left(  \text{empty product}\right)  =1\perp c$. Hence,
(\ref{sol.ent.coprime.ab-to-ck.goal}) holds for $i=0$. This completes the
induction base.

\textit{Induction step:} Let $j\in\left\{  1,2,\ldots,k\right\}  $. Assume
that (\ref{sol.ent.coprime.ab-to-ck.goal}) holds for $i=j-1$. We must now
prove that (\ref{sol.ent.coprime.ab-to-ck.goal}) holds for $i=j$.

We have assumed that (\ref{sol.ent.coprime.ab-to-ck.goal}) holds for $i=j-1$.
In other words, $a_{1}a_{2}\cdots a_{j-1}\perp c$.

We have assumed that each $i\in\left\{  1,2,\ldots,k\right\}  $ satisfies
$a_{i}\perp c$. Applying this to $i=j$, we find $a_{j}\perp c$.

Now we know that $a_{1}a_{2}\cdots a_{j-1}\perp c$ and $a_{j}\perp c$. Hence,
Theorem \ref{thm.ent.coprime.ab-to-c} (applied to $a=a_{1}a_{2}\cdots a_{j-1}$
and $b=a_{j}$) yields $\left(  a_{1}a_{2}\cdots a_{j-1}\right)  a_{j}\perp c$.
In other words, $a_{1}a_{2}\cdots a_{j}\perp c$ (since $a_{1}a_{2}\cdots
a_{j}=\left(  a_{1}a_{2}\cdots a_{j-1}\right)  a_{j}$). In other words,
(\ref{sol.ent.coprime.ab-to-ck.goal}) holds for $i=j$. This completes the
induction step. Thus, (\ref{sol.ent.coprime.ab-to-ck.goal}) is proven by induction.

Now, we can apply (\ref{sol.ent.coprime.ab-to-ck.goal}) to $i=k$. We thus
obtain $a_{1}a_{2}\cdots a_{k}\perp c$. This proves Exercise
\ref{exe.ent.coprime.combinek}.
\end{proof}
\end{fineprint}

We can similarly generalize Theorem \ref{thm.ent.coprime.combine} to show that
the product of several mutually coprime divisors of an integer $c$ must again
be a divisor of $c$:

\begin{exercise}
\label{exe.ent.coprime.combinek}Let $c\in\mathbb{Z}$. Let $b_{1},b_{2}%
,\ldots,b_{k}$ be integers that are mutually coprime (i.e., they satisfy
$b_{i}\perp b_{j}$ for all $i\neq j$). Assume that $b_{i}\mid c$ for each
$i\in\left\{  1,2,\ldots,k\right\}  $. Prove that $b_{1}b_{2}\cdots b_{k}\mid
c$.
\end{exercise}

\begin{fineprint}
\begin{proof}
[Solution to Exercise \ref{exe.ent.coprime.combinek}.]We assumed that the
integers $b_{1},b_{2},\ldots,b_{k}$ are mutually coprime. In other words, we
have%
\begin{equation}
b_{i}\perp b_{j}\text{ for all }i,j\in\left\{  1,2,\ldots,k\right\}  \text{
satisfying }i\neq j. \label{sol.ent.coprime.combinek.copr}%
\end{equation}


Let us prove that%
\begin{equation}
b_{1}b_{2}\cdots b_{i}\mid c\ \ \ \ \ \ \ \ \ \ \text{for each }i\in\left\{
0,1,\ldots,k\right\}  . \label{sol.ent.coprime.combinek.goal}%
\end{equation}


\textit{Proof of (\ref{sol.ent.coprime.combinek.goal}):} We shall prove
(\ref{sol.ent.coprime.combinek.goal}) by induction on $i$:

\textit{Induction base:} We have $b_{1}b_{2}\cdots b_{0}=\left(  \text{empty
product}\right)  =1\mid c$. Hence, (\ref{sol.ent.coprime.combinek.goal}) holds
for $i=0$. This completes the induction base.

\textit{Induction step:} Let $j\in\left\{  1,2,\ldots,k\right\}  $. Assume
that (\ref{sol.ent.coprime.combinek.goal}) holds for $i=j-1$. We must now
prove that (\ref{sol.ent.coprime.combinek.goal}) holds for $i=j$.

We have assumed that (\ref{sol.ent.coprime.ab-to-ck.goal}) holds for $i=j-1$.
In other words, $b_{1}b_{2}\cdots b_{j-1}\mid c$.

We have assumed that each $i\in\left\{  1,2,\ldots,k\right\}  $ satisfies
$b_{i}\mid c$. Applying this to $i=j$, we find $b_{j}\mid c$.

For each $i\in\left\{  1,2,\ldots,j-1\right\}  $, we have $i\leq j-1<j$ and
thus $i\neq j$ and therefore $b_{i}\perp b_{j}$ (by
(\ref{sol.ent.coprime.combinek.copr})). Hence, Exercise
\ref{exe.ent.coprime.combinek} (applied to $j-1$, $b_{j}$ and $\left(
b_{1},b_{2},\ldots,b_{j-1}\right)  $ instead of $k$, $c$ and $\left(
a_{1},a_{2},\ldots,a_{k}\right)  $) yields $b_{1}b_{2}\cdots b_{j-1}\perp
b_{j}$.

Now we know that $b_{1}b_{2}\cdots b_{j-1}\mid c$ and $b_{j}\mid c$ and
$b_{1}b_{2}\cdots b_{j-1}\perp b_{j}$. Hence, Theorem
\ref{thm.ent.coprime.combine} (applied to $a=b_{1}b_{2}\cdots b_{j-1}$ and
$b=b_{j}$) yields $\left(  b_{1}b_{2}\cdots b_{j-1}\right)  b_{j}\mid c$. In
other words, $b_{1}b_{2}\cdots b_{j}\mid c$ (since $b_{1}b_{2}\cdots
b_{j}=\left(  b_{1}b_{2}\cdots b_{j-1}\right)  b_{j}$). In other words,
(\ref{sol.ent.coprime.combinek.goal}) holds for $i=j$. This completes the
induction step. Thus, (\ref{sol.ent.coprime.combinek.goal}) is proven by induction.

Now, we can apply (\ref{sol.ent.coprime.combinek.goal}) to $i=k$. We thus
obtain $b_{1}b_{2}\cdots b_{k}\mid c$. This proves Exercise
\ref{exe.ent.coprime.combinek}.
\end{proof}
\end{fineprint}

\begin{exercise}
\label{exe.ent.coprime.powers}Let $a,b\in\mathbb{Z}$ be such that $a\perp b$.
Let $n,m\in\mathbb{N}$. Prove that $a^{n}\perp b^{m}$.
\end{exercise}

\begin{fineprint}
\begin{proof}
[Solution to Exercise \ref{exe.ent.coprime.powers}.]We have $a\perp b$. Thus,
Exercise \ref{exe.ent.coprime.combinek} (applied to $n$, $b$ and $\left(
\underbrace{a,a,\ldots,a}_{n\text{ times}}\right)  $ instead of $k$, $c$ and
$\left(  a_{1},a_{2},\ldots,a_{k}\right)  $) yields that $\underbrace{aa\cdots
a}_{n\text{ times}}\perp b$. In other words, $a^{n}\perp b$.

According to Proposition \ref{prop.ent.coprime.perp-symm} (applied to $a^{n}$
instead of $a$), we have $a^{n}\perp b$ if and only if $b\perp a^{n}$. Thus,
$b\perp a^{n}$ (since $a^{n}\perp b$). Hence, Exercise
\ref{exe.ent.coprime.combinek} (applied to $m$, $a^{n}$ and $\left(
\underbrace{b,b,\ldots,b}_{m\text{ times}}\right)  $ instead of $k$, $c$ and
$\left(  a_{1},a_{2},\ldots,a_{k}\right)  $) yields that $\underbrace{bb\cdots
b}_{m\text{ times}}\perp a^{n}$. In other words, $b^{m}\perp a^{n}$.

According to Proposition \ref{prop.ent.coprime.perp-symm} (applied to $a^{n}$
and $b^{m}$ instead of $a$ and $b$), we have $a^{n}\perp b^{m}$ if and only if
$b^{m}\perp a^{n}$. Hence, $a^{n}\perp b^{m}$ (since $b^{m}\perp a^{n}$). This
solves Exercise \ref{exe.ent.coprime.powers}.
\end{proof}
\end{fineprint}

The above results have one important application to congruences. Recall that
if $a,b,c$ are integers satisfying $ab=ac$, then we can \textquotedblleft
cancel\textquotedblright\ $a$ from the equality $ab=ac$ to obtain $b=c$ as
long as $a$ is nonzero. Something similar is true for congruences modulo $n$,
but the condition \textquotedblleft$a$ is nonzero\textquotedblright\ has to be
replaced by \textquotedblleft$a$ is coprime to $n$\textquotedblright:

\begin{lemma}
\label{lem.ent.coprime.cancel}Let $a,b,c,n$ be integers such that $a\perp n$
and $ab\equiv ac\operatorname{mod}n$. Then, $b\equiv c\operatorname{mod}n$.
\end{lemma}

Lemma \ref{lem.ent.coprime.cancel} says that we can cancel an integer $a$ from
a congruence $ab\equiv ac\operatorname{mod}n$ as long as $a$ is coprime to
$n$. Let us give two proofs of this lemma, to illustrate the uses of some of
the previous results:

\begin{proof}
[First proof of Lemma \ref{lem.ent.coprime.cancel}.]We have $ab\equiv
ac\operatorname{mod}n$. In other words, $n\mid ab-ac=a\left(  b-c\right)  $.
But Proposition \ref{prop.ent.coprime.perp-symm} (applied to $n$ instead of
$b$) shows that $a\perp n$ if and only if $n\perp a$. Thus, we have $n\perp a$
(since $a\perp a$).

Thus, we know that $n\mid a\left(  b-c\right)  $ and $n\perp a$. Hence,
Theorem \ref{thm.ent.coprime.cancel} (applied to $n$, $a$ and $b-c$ instead of
$a$, $b$ and $c$) yields $n\mid b-c$. In other words, $b\equiv
c\operatorname{mod}n$. This proves Lemma \ref{lem.ent.coprime.cancel}.
\end{proof}

\begin{proof}
[Second proof of Lemma \ref{lem.ent.coprime.cancel}.]Theorem
\ref{thm.ent.coprime.modinv} \textbf{(b)} yields that there exists an
$a^{\prime}\in\mathbb{Z}$ such that $aa^{\prime}\equiv1\operatorname{mod}n$
(since $a\perp n$). Consider this $a^{\prime}$. Now, let us multiply the
(trivial) congruence $a^{\prime}\equiv a^{\prime}\operatorname{mod}n$ with the
congruence $ab\equiv ac\operatorname{mod}n$. We thus find%
\[
a^{\prime}ab\equiv\underbrace{a^{\prime}a}_{\equiv1\operatorname{mod}n}%
c\equiv1c=c\operatorname{mod}n.
\]
Hence,
\[
c\equiv\underbrace{a^{\prime}a}_{\equiv1\operatorname{mod}n}b\equiv
1b=b\operatorname{mod}n.
\]
In other words, $b\equiv c\operatorname{mod}n$. This proves Lemma
\ref{lem.ent.coprime.cancel}.
\end{proof}

\subsubsection{An application to sums of powers}

Let us show an application of Theorem \ref{thm.ent.coprime.combine}. First, we
shall prove a simple lemma:

\begin{lemma}
\label{lem.ent.xd-yd}Let $d\in\mathbb{N}$. Let $x$ and $y$ be integers.

\textbf{(a)} We have $x-y\mid x^{d}-y^{d}$.

\textbf{(b)} We have $x+y\mid x^{d}+y^{d}$ if $d$ is odd.
\end{lemma}

\begin{proof}
[Proof of Lemma \ref{lem.ent.xd-yd}.]\textbf{(a)} Here are two ways of proving this:

\textit{First proof of Lemma \ref{lem.ent.xd-yd} \textbf{(a)}:} We have
$x\equiv y\operatorname{mod}x-y$ (since $x-y\mid x-y$). Thus, Exercise
\ref{exe.ent.mod.basics.k-power} (applied to $n=x-y$, $a=x$, $b=y$ and $k=d$)
yields $x^{d}\equiv y^{d}\operatorname{mod}x-y$. In other words, $x-y\mid
x^{d}-y^{d}$. This proves Lemma \ref{lem.ent.xd-yd} \textbf{(a)}.

\textit{Second proof of Lemma \ref{lem.ent.xd-yd} \textbf{(a)}:} Recall that%
\begin{equation}
\left(  a-b\right)  \left(  a^{k-1}+a^{k-2}b+a^{k-3}b^{2}+\cdots
+ab^{k-2}+b^{k-1}\right)  =a^{k}-b^{k} \label{pf.lem.ent.xd-yd.1}%
\end{equation}
for every $a,b\in\mathbb{Q}$ and $k\in\mathbb{N}$. (This is a well-known
identity, and it appears (with $k$ renamed as $n$) as the first half of
Exercise 1 on
\href{http://www-users.math.umn.edu/~dgrinber/19s/hw0s.pdf}{homework set
\#0}.) Applying this identity to $a=x$, $b=y$ and $k=d$, we obtain%
\[
\left(  x-y\right)  \left(  x^{d-1}+x^{d-2}y+x^{d-3}y^{2}+\cdots
+xy^{d-2}+y^{d-1}\right)  =x^{d}-y^{d}.
\]
Thus, $x-y\mid x^{d}-y^{d}$ (since $x^{d-1}+x^{d-2}y+x^{d-3}y^{2}%
+\cdots+xy^{d-2}+y^{d-1}$ is an integer). This proves Lemma
\ref{lem.ent.xd-yd} \textbf{(a)}.

\textbf{(b)} Assume that $d$ is odd. Thus, $\left(  -1\right)  ^{d}=-1$. Now,
Lemma \ref{lem.ent.xd-yd} \textbf{(a)} (applied to $-y$ instead of $y$) yields
$x-\left(  -y\right)  \mid x^{d}-\left(  -y\right)  ^{d}$. Since $x-\left(
-y\right)  =x+y$ and $x^{d}-\underbrace{\left(  -y\right)  ^{d}}_{=\left(
-1\right)  ^{d}y^{d}}=x^{d}-\underbrace{\left(  -1\right)  ^{d}}_{=-1}%
y^{d}=x^{d}-\left(  -1\right)  y^{d}=x^{d}+y^{d}$, this rewrites as $x+y\mid
x^{d}+y^{d}$. This proves Lemma \ref{lem.ent.xd-yd} \textbf{(b)}.
\end{proof}

Next, let us recall a basic fact from combinatorics (the \textquotedblleft
Little Gauss\textquotedblright\ sum):

\begin{proposition}
\label{prop.ent.1+2+...+n}Let $n\in\mathbb{N}$. Then,
\[
1+2+\cdots+n=\dfrac{n\left(  n+1\right)  }{2}.
\]

\end{proposition}

\begin{proof}
[Proof of Proposition \ref{prop.ent.1+2+...+n}.]Here is one of several equally
valid arguments:%
\begin{align*}
2\cdot\left(  1+2+\cdots+n\right)   &  =\left(  1+2+\cdots+n\right)
+\underbrace{\left(  1+2+\cdots+n\right)  }_{\substack{=n+\left(  n-1\right)
+\cdots+1\\\text{(here, we have reversed}\\\text{the order of the addends)}%
}}\\
&  =\underbrace{\left(  1+2+\cdots+n\right)  }_{=\sum_{k=1}^{n}k}%
+\underbrace{\left(  n+\left(  n-1\right)  +\cdots+1\right)  }_{=\sum
_{k=1}^{n}\left(  n+1-k\right)  }\\
&  =\sum_{k=1}^{n}k+\sum_{k=1}^{n}\left(  n+1-k\right)  =\sum_{k=1}%
^{n}\underbrace{\left(  k+\left(  n+1-k\right)  \right)  }_{=n+1}\\
&  =\sum_{k=1}^{n}\left(  n+1\right)  =n\left(  n+1\right)  .
\end{align*}
Thus, $1+2+\cdots+n=\dfrac{n\left(  n+1\right)  }{2}$, so that Proposition
\ref{prop.ent.1+2+...+n} is proven.
\end{proof}

Proposition \ref{prop.ent.1+2+...+n} tells us what the sum $1+2+\cdots+n$ of
the first $n$ positive integers is. One might also ask what the sum
$1^{2}+2^{2}+\cdots+n^{2}$ of their squares is, and similarly for higher
powers. While this is tangential to our course, let us collect some formulas
for this:

\begin{proposition}
\label{prop.ent.1d+2d+...+nd-for-5}Let $n\in\mathbb{N}$. Then:

\textbf{(a)} We have $1+2+\cdots+n=\dfrac{1}{2}n\left(  n+1\right)  $.

\textbf{(b)} We have $1^{2}+2^{2}+\cdots+n^{2}=\dfrac{1}{6}n\left(
n+1\right)  \left(  2n+1\right)  $.

\textbf{(c)} We have $1^{3}+2^{3}+\cdots+n^{3}=\dfrac{1}{4}n^{2}\left(
n+1\right)  ^{2}$.

\textbf{(d)} We have $1^{4}+2^{4}+\cdots+n^{4}=\dfrac{1}{30}n\left(
2n+1\right)  \left(  n+1\right)  \left(  3n+3n^{2}-1\right)  $.

\textbf{(e)} We have $1^{5}+2^{5}+\cdots+n^{5}=\dfrac{1}{12}n^{2}\left(
n+1\right)  ^{2}\left(  2n+2n^{2}-1\right)  $.
\end{proposition}

Each part of Proposition \ref{prop.ent.1d+2d+...+nd-for-5} can be
straightforwardly proven by induction on $n$; we don't need ingenious
arguments like the one we gave above for Proposition \ref{prop.ent.1+2+...+n}
(and in fact, such arguments cannot always be found).

\begin{fineprint}
You probably see a pattern in Proposition \ref{prop.ent.1d+2d+...+nd-for-5}:
It appears that for each positive integer $d$, there exists some polynomial
$p_{d}\left(  x\right)  $ of degree $d+1$ with rational coefficients such that
each $n\in\mathbb{N}$ satisfies $1^{d}+2^{d}+\cdots+n^{d}=p_{d}\left(
n\right)  $. This is indeed the case. Indeed, this is proven (e.g.) in
\cite[Proposition 23.2]{Galvin} and in \cite[Theorem 3.7]{lucas}. The
polynomial $p_{d}\left(  x\right)  $ is uniquely determined for each $d$, and
can be explicitly computed via the formula%
\[
p_{d}\left(  x\right)  =\sum_{k=1}^{d}k!%
%TCIMACRO{\QDATOPD{\{}{\}}{d}{k}}%
%BeginExpansion
\genfrac{\{}{\}}{0pt}{0}{d}{k}%
%EndExpansion
\dbinom{x+1}{k+1},
\]
where $\dbinom{x+1}{k+1}=\dfrac{\left(  x+1\right)  x\left(  x-1\right)
\cdots\left(  x-k+1\right)  }{\left(  k+1\right)  !}$ and where $%
%TCIMACRO{\QDATOPD{\{}{\}}{d}{k}}%
%BeginExpansion
\genfrac{\{}{\}}{0pt}{0}{d}{k}%
%EndExpansion
$ is a \textit{Stirling number of the 2nd kind}. Without going into the
details of what Stirling numbers of the 2nd kind are, let me say that $k!%
%TCIMACRO{\QDATOPD{\{}{\}}{d}{k}}%
%BeginExpansion
\genfrac{\{}{\}}{0pt}{0}{d}{k}%
%EndExpansion
$ is the number of surjective maps from $\left\{  1,2,\ldots,d\right\}  $ to
$\left\{  1,2,\ldots,k\right\}  $. For example,%
\begin{align*}
p_{2}\left(  x\right)   &  =\sum_{k=1}^{2}k!%
%TCIMACRO{\QDATOPD{\{}{\}}{2}{k}}%
%BeginExpansion
\genfrac{\{}{\}}{0pt}{0}{2}{k}%
%EndExpansion
\dbinom{x+1}{k+1}=\underbrace{1!%
%TCIMACRO{\QDATOPD{\{}{\}}{2}{1}}%
%BeginExpansion
\genfrac{\{}{\}}{0pt}{0}{2}{1}%
%EndExpansion
}_{=1}\dbinom{x+1}{2}+\underbrace{2!%
%TCIMACRO{\QDATOPD{\{}{\}}{2}{2}}%
%BeginExpansion
\genfrac{\{}{\}}{0pt}{0}{2}{2}%
%EndExpansion
}_{=2}\dbinom{x+1}{3}\\
&  =\dbinom{x+1}{2}+2\dbinom{x+1}{3}=\dfrac{\left(  x+1\right)  x}{2}%
+2\cdot\dfrac{\left(  x+1\right)  x\left(  x-1\right)  }{6}\\
&  =\dfrac{1}{6}x\left(  x+1\right)  \left(  2x+1\right)  ,
\end{align*}
and thus%
\[
1^{2}+2^{2}+\cdots+n^{2}=p_{2}\left(  n\right)  =\dfrac{1}{6}n\left(
n+1\right)  \left(  2n+1\right)  \ \ \ \ \ \ \ \ \ \ \text{for each }%
n\in\mathbb{N}.
\]
This recovers the claim of Proposition \ref{prop.ent.1d+2d+...+nd-for-5}
\textbf{(b)}. The combinatorial proof presented in \cite[Proposition
23.2]{Galvin} is highly recommended reading for anyone interested in this kind
of formulas.

Let us note that the polynomials $p_{d}\left(  x\right)  $ do \textbf{not}
have integer coefficients, but nevertheless all their values $p_{d}\left(
n\right)  $ for $n\in\mathbb{N}$ are integers.
\end{fineprint}

Let us now show the power of Theorem \ref{thm.ent.coprime.combine} on the
following exercise:

\begin{exercise}
\label{exe.ent.coprime.1+2+...+n}Let $n\in\mathbb{N}$. Let $d$ be an odd
positive integer. Prove that%
\[
1+2+\cdots+n\mid1^{d}+2^{d}+\cdots+n^{d}.
\]

\end{exercise}

\begin{fineprint}
\begin{proof}
[Solution to Exercise \ref{exe.ent.coprime.1+2+...+n}.]Proposition
\ref{prop.ent.1+2+...+n} yields $1+2+\cdots+n=\dfrac{n\left(  n+1\right)  }%
{2}$. Thus, we need to prove that%
\[
\dfrac{n\left(  n+1\right)  }{2}\mid1^{d}+2^{d}+\cdots+n^{d}.
\]
This is equivalent to%
\begin{equation}
n\left(  n+1\right)  \mid2\left(  1^{d}+2^{d}+\cdots+n^{d}\right)
\label{sol.ent.coprime.1+2+...+n.g}%
\end{equation}
(by Exercise \ref{exe.ent.div.acbc}, applied to $a=\dfrac{n\left(  n+1\right)
}{2}$, $b=1^{d}+2^{d}+\cdots+n^{d}$ and $c=2$). Hence, it suffices to prove
(\ref{sol.ent.coprime.1+2+...+n.g}).

In order to prove (\ref{sol.ent.coprime.1+2+...+n.g}), it suffices to show
that%
\begin{align}
n  &  \mid2\left(  1^{d}+2^{d}+\cdots+n^{d}\right)
\ \ \ \ \ \ \ \ \ \ \text{and}\label{sol.ent.coprime.1+2+...+n.g1}\\
n+1  &  \mid2\left(  1^{d}+2^{d}+\cdots+n^{d}\right)  .
\label{sol.ent.coprime.1+2+...+n.g2}%
\end{align}
Indeed, the integers $n$ and $n+1$ are coprime (by Example
\ref{exa.ent.coprime.1} \textbf{(c)}, applied to $a=n$); in other words,
$n\perp n+1$. Hence, if we can prove (\ref{sol.ent.coprime.1+2+...+n.g1}) and
(\ref{sol.ent.coprime.1+2+...+n.g2}), then Theorem
\ref{thm.ent.coprime.combine} (applied to $a=n$, $b=n+1$ and $c=2\left(
1^{d}+2^{d}+\cdots+n^{d}\right)  $) will yield $n\left(  n+1\right)
\mid2\left(  1^{d}+2^{d}+\cdots+n^{d}\right)  $; this will prove
(\ref{sol.ent.coprime.1+2+...+n.g}) and therefore complete our solution.

We shall prove (\ref{sol.ent.coprime.1+2+...+n.g2}) first:

\textit{Proof of (\ref{sol.ent.coprime.1+2+...+n.g2}):} We have%
\begin{align}
2\left(  1^{d}+2^{d}+\cdots+n^{d}\right)   &  =\left(  1^{d}+2^{d}%
+\cdots+n^{d}\right)  +\left(  1^{d}+2^{d}+\cdots+n^{d}\right) \nonumber\\
&  =\left(  1^{d}+2^{d}+\cdots+n^{d}\right)  +\left(  n^{d}+\left(
n-1\right)  ^{d}+\cdots+1^{d}\right) \nonumber\\
&  =\sum_{k=1}^{n}k^{d}+\sum_{k=1}^{n}\left(  n+1-k\right)  ^{d}\nonumber\\
&  =\sum_{k=1}^{n}\left(  k^{d}+\left(  n+1-k\right)  ^{d}\right)  .
\label{sol.ent.coprime.1+2+...+n.5}%
\end{align}


But if $k\in\mathbb{Z}$, then Lemma \ref{lem.ent.xd-yd} \textbf{(b)} (applied
to $x=k$ and $y=n+1-k$) shows that $k^{d}+\left(  n+1-k\right)  ^{d}$ is
divisible by $k+\left(  n+1-k\right)  =n+1$. Hence, each addend in the sum on
the right hand side of (\ref{sol.ent.coprime.1+2+...+n.5}) is divisible by
$n+1$. Therefore, the whole sum is divisible by $n+1$ as well. Thus, the left
hand side is divisible by $n+1$, too. In other words, $n+1\mid2\left(
1^{d}+2^{d}+\cdots+n^{d}\right)  $. Thus, (\ref{sol.ent.coprime.1+2+...+n.g2})
is proven.

\textit{Proof of} \textit{(\ref{sol.ent.coprime.1+2+...+n.g1}):} If $n=0$,
then (\ref{sol.ent.coprime.1+2+...+n.g1}) boils down to $0\mid2\cdot0$ (since
empty sums are $0$); this is obvious. Thus, for the rest of this proof, we
WLOG assume that $n\neq0$. Hence, $n$ is a positive integer, and thus
$n-1\in\mathbb{N}$. Therefore, we can apply
(\ref{sol.ent.coprime.1+2+...+n.g2}) to $n-1$ instead of $n$ (since we have
already proven (\ref{sol.ent.coprime.1+2+...+n.g2}) for each $n\in\mathbb{N}%
$). We thus obtain%
\[
n\mid2\left(  1^{d}+2^{d}+\cdots+\left(  n-1\right)  ^{d}\right)  .
\]
In other words, $2\left(  1^{d}+2^{d}+\cdots+\left(  n-1\right)  ^{d}\right)
\equiv0\operatorname{mod}n$. Now,%
\begin{align*}
&  2\left(  1^{d}+2^{d}+\cdots+n^{d}\right)  -2\left(  1^{d}+2^{d}%
+\cdots+\left(  n-1\right)  ^{d}\right) \\
&  =2\cdot\underbrace{\left(  \left(  1^{d}+2^{d}+\cdots+n^{d}\right)
-\left(  1^{d}+2^{d}+\cdots+\left(  n-1\right)  ^{d}\right)  \right)
}_{=n^{d}}\\
&  =2n^{d}=n\cdot2n^{d-1}\ \ \ \ \ \ \ \ \ \ \left(  \text{since }d\geq1\text{
(because }d\text{ is odd)}\right)
\end{align*}
is clearly divisible by $n$. In other words,%
\[
2\left(  1^{d}+2^{d}+\cdots+n^{d}\right)  \equiv2\left(  1^{d}+2^{d}%
+\cdots+\left(  n-1\right)  ^{d}\right)  \operatorname{mod}n.
\]
Hence,
\[
2\left(  1^{d}+2^{d}+\cdots+n^{d}\right)  \equiv2\left(  1^{d}+2^{d}%
+\cdots+\left(  n-1\right)  ^{d}\right)  \equiv0\operatorname{mod}n.
\]
That is, $n\mid2\left(  1^{d}+2^{d}+\cdots+n^{d}\right)  $. This proves
(\ref{sol.ent.coprime.1+2+...+n.g1}).

We have now proven both (\ref{sol.ent.coprime.1+2+...+n.g1}) and
(\ref{sol.ent.coprime.1+2+...+n.g2}). As we have explained, this yields
(\ref{sol.ent.coprime.1+2+...+n.g}), which in turn solves the problem.
\end{proof}
\end{fineprint}

\begin{exercise}
\label{exe.ent.coprime.gcd*gcd1}Let $u,v,x,y\in\mathbb{Z}$. Prove that
$\gcd\left(  u,v\right)  \cdot\gcd\left(  x,y\right)  =\gcd\left(
ux,uy,vx,vy\right)  $.
\end{exercise}

\begin{fineprint}
\begin{proof}
[Solution to Exercise \ref{exe.ent.coprime.gcd*gcd1}.]Let $g=\gcd\left(
x,y\right)  $. Then, $g$ is a nonnegative integer (since any gcd is a
nonnegative integer); thus, $\left\vert g\right\vert =g$.

Theorem \ref{thm.ent.gcd.split} (applied to $2$, $\left(  ux,uy\right)  $, $2$
and $\left(  vx,vy\right)  $ instead of $k$, $\left(  b_{1},b_{2},\ldots
,b_{k}\right)  $, $k$ and $\left(  c_{1},c_{2},\ldots,c_{\ell}\right)  $)
yields
\begin{equation}
\gcd\left(  ux,uy,vx,vy\right)  =\gcd\left(  \gcd\left(  ux,uy\right)
,\gcd\left(  vx,vy\right)  \right)  . \label{sol.ent.coprime.gcd*gcd1.3}%
\end{equation}


Corollary \ref{cor.ent.gcd.sa,sb} (applied to $s=u$, $a=x$ and $b=y$) yields%
\[
\gcd\left(  ux,uy\right)  =\left\vert u\right\vert \underbrace{\gcd\left(
x,y\right)  }_{=g}=\left\vert u\right\vert g=g\left\vert u\right\vert .
\]
The same argument (applied to $v$ instead of $u$) yields%
\[
\gcd\left(  vx,vy\right)  =g\left\vert v\right\vert .
\]
Now, (\ref{sol.ent.coprime.gcd*gcd1.3}) becomes%
\begin{align*}
\gcd\left(  ux,uy,vx,vy\right)   &  =\gcd\left(  \underbrace{\gcd\left(
ux,uy\right)  }_{=g\left\vert u\right\vert },\underbrace{\gcd\left(
vx,vy\right)  }_{=g\left\vert v\right\vert }\right)  =\gcd\left(  g\left\vert
u\right\vert ,g\left\vert v\right\vert \right) \\
&  =\underbrace{\left\vert g\right\vert }_{\substack{=g\\=\gcd\left(
x,y\right)  }}\underbrace{\gcd\left(  \left\vert u\right\vert ,\left\vert
v\right\vert \right)  }_{\substack{=\gcd\left(  u,v\right)  \\\text{(by
Exercise \ref{exe.ent.gcd.abs} \textbf{(c)},}\\\text{applied to }a=u\text{ and
}b=v\text{)}}}\\
&  \ \ \ \ \ \ \ \ \ \ \left(  \text{by Corollary \ref{cor.ent.gcd.sa,sb},
applied to }s=g\text{, }a=\left\vert u\right\vert \text{ and }b=\left\vert
v\right\vert \right) \\
&  =\gcd\left(  x,y\right)  \cdot\gcd\left(  u,v\right)  =\gcd\left(
u,v\right)  \cdot\gcd\left(  x,y\right)  .
\end{align*}
This solves Exercise \ref{exe.ent.coprime.gcd*gcd1}.
\end{proof}
\end{fineprint}

\begin{exercise}
\label{exe.ent.coprime.gcd*gcd2}Let $a,b,c\in\mathbb{Z}$.

\textbf{(a)} Prove that $\gcd\left(  a,b\right)  \cdot\gcd\left(  a,c\right)
=\gcd\left(  ag,bc\right)  $, where $g=\gcd\left(  a,b,c\right)  $.

\textbf{(b)} Prove that $\gcd\left(  a,b\right)  \cdot\gcd\left(  a,c\right)
=\gcd\left(  a,bc\right)  $ if $b\perp c$.
\end{exercise}

\begin{proof}
[Solution to Exercise \ref{exe.ent.coprime.gcd*gcd2}.]TODO.
\end{proof}

\begin{exercise}
\label{exe.ent.coprime.combine2}Let $a,b,c\in\mathbb{Z}$ satisfy $a\mid c$ and
$b\mid c$. Prove that $ab\mid\gcd\left(  a,b,c\right)  \cdot c$.
\end{exercise}

\begin{proof}
[Solution to Exercise \ref{exe.ent.coprime.combine2}.]TODO.
\end{proof}

\begin{center}
\textbf{2019-02-08 lecture}
\end{center}

\subsection{Lowest common multiples}

Common multiples are, in a sense, a \textquotedblleft mirror
version\textquotedblright\ of common divisors. Here is their definition:

\begin{definition}
\label{def.ent.Mul}Let $b_{1},b_{2},\ldots,b_{k}$ be integers. Then, the
\textit{common multiples} of $b_{1},b_{2},\ldots,b_{k}$ are defined to be the
integers $a$ that satisfy%
\[
\left(  b_{i}\mid a\text{ for all }i\in\left\{  1,2,\ldots,k\right\}  \right)
.
\]
(In other words, a \textit{common multiple} of $b_{1},b_{2},\ldots,b_{k}$ is
an integer that is a multiple of each of $b_{1},b_{2},\ldots,b_{k}$.) We let
$\operatorname*{Mul}\left(  b_{1},b_{2},\ldots,b_{k}\right)  $ denote the set
of these common multiples.
\end{definition}

\begin{example}
The common multiples of $4,6$ are $\ldots,-36,-24,-12,0,12,24,36,\ldots$, that
is, all multiples of $12$.

The common multiples of $1,2,3$ are all multiples of $6$.
\end{example}

Note that the common multiples of a single integer $b$ are simply the
multiples of $b$. (Also, the common multiples of an empty list of integers are
all the integers; in other words, $\operatorname*{Mul}\left(  {}\right)
=\mathbb{Z}$.)

Note that the definition of common multiples of $b_{1},b_{2},\ldots,b_{k}$
(Definition \ref{def.ent.Mul}) is the same as the definition of common
divisors of $b_{1},b_{2},\ldots,b_{k}$ except that the divisibility has been
flipped (i.e., it says \textquotedblleft$b_{i}\mid a$\textquotedblright%
\ instead of \textquotedblleft$a\mid b_{i}$\textquotedblright). This is why
common multiples are a \textquotedblleft mirror version\textquotedblright\ of
common divisors. This analogy is not perfect -- in particular, (for example)
two nonzero integers have infinitely many common multiples but only finitely
many common divisors. We shall now introduce lowest common multiples, which
correspond to greatest common divisors in this analogy. However, we have to
prove a simple proposition first:

\begin{proposition}
\label{prop.ent.Mul.exi}Let $b_{1},b_{2},\ldots,b_{k}$ be finitely many
nonzero integers. Then, the set $\operatorname*{Mul}\left(  b_{1},b_{2}%
,\ldots,b_{k}\right)  $ has a smallest positive element.
\end{proposition}

Proposition \ref{prop.ent.Mul.exi} is similar to Proposition
\ref{prop.ent.Div.fin} (and will play a similar role), but note the
differences: It requires \textbf{all} of $b_{1},b_{2},\ldots,b_{k}$ to be
nonzero (unlike Proposition \ref{prop.ent.Div.fin}, which needed only one of
them to be nonzero), and it does not claim finiteness of any set.

\begin{proof}
[Proof of Proposition \ref{prop.ent.Mul.exi}.]We claim that%
\begin{equation}
\left\vert b_{1}b_{2}\cdots b_{k}\right\vert \in\operatorname*{Mul}\left(
b_{1},b_{2},\ldots,b_{k}\right)  . \label{pf.prop.ent.Mul.exi.1}%
\end{equation}


\textit{Proof of (\ref{pf.prop.ent.Mul.exi.1}):} Let $i\in\left\{
1,2,\ldots,k\right\}  $. Then, the product $b_{1}b_{2}\cdots b_{k}$ can be
written as%
\[
b_{1}b_{2}\cdots b_{k}=b_{i}\cdot\left(  b_{1}b_{2}\cdots b_{i-1}%
b_{i+1}b_{i+2}\cdots b_{k}\right)  ,
\]
and thus is divisible by $b_{i}$. In other words, $b_{i}\mid b_{1}b_{2}\cdots
b_{k}$. But Exercise \ref{exe.ent.div.aabs} \textbf{(a)} (applied to
$a=b_{1}b_{2}\cdots b_{k}$) yields $b_{1}b_{2}\cdots b_{k}\mid\left\vert
b_{1}b_{2}\cdots b_{k}\right\vert $. Altogether, $b_{i}\mid b_{1}b_{2}\cdots
b_{k}\mid\left\vert b_{1}b_{2}\cdots b_{k}\right\vert $.

Now forget that we fixed $i$. We thus have proven that $b_{i}\mid\left\vert
b_{1}b_{2}\cdots b_{k}\right\vert $ for all $i\in\left\{  1,2,\ldots
,k\right\}  $. In other words, $\left\vert b_{1}b_{2}\cdots b_{k}\right\vert $
is a common multiple of $b_{1},b_{2},\ldots,b_{k}$ (by the definition of a
\textquotedblleft common multiple\textquotedblright). In other words,
$\left\vert b_{1}b_{2}\cdots b_{k}\right\vert \in\operatorname*{Mul}\left(
b_{1},b_{2},\ldots,b_{k}\right)  $. This proves (\ref{pf.prop.ent.Mul.exi.1}).]

We know that $b_{1},b_{2},\ldots,b_{k}$ are nonzero integers. Hence, their
product $b_{1}b_{2}\cdots b_{k}$ is a nonzero integer as well. Thus, its
absolute value $\left\vert b_{1}b_{2}\cdots b_{k}\right\vert $ is a positive
integer. Hence, $\left\vert b_{1}b_{2}\cdots b_{k}\right\vert $ is a positive
element of $\operatorname*{Mul}\left(  b_{1},b_{2},\ldots,b_{k}\right)  $
(since (\ref{pf.prop.ent.Mul.exi.1}) shows that it is an element of
$\operatorname*{Mul}\left(  b_{1},b_{2},\ldots,b_{k}\right)  $). Thus, the set
$\operatorname*{Mul}\left(  b_{1},b_{2},\ldots,b_{k}\right)  $ has a positive
element. Therefore, this set $\operatorname*{Mul}\left(  b_{1},b_{2}%
,\ldots,b_{k}\right)  $ has a \textbf{smallest} positive element as
well\footnote{Here we are using the following basic fact: If a set of integers
$S$ has a positive element, then it has a \textbf{smallest} positive element
as well. (To prove this fact, you can fix a positive element $s\in S$, which
exists by assumption; then, the set $\left\{  1,2,\ldots,s\right\}  $ is
finite and nonempty, and thus clearly has a smallest element; now you can
easily check that its smallest element must also be the smallest positive
element of $S$.)}. This proves Proposition \ref{prop.ent.Mul.exi}.
\end{proof}

\begin{definition}
\label{def.ent.lcm.lcm}Let $b_{1},b_{2},\ldots,b_{k}$ be finitely many
integers. The \textit{lowest common multiple} of $b_{1},b_{2},\ldots,b_{k}$ is
defined as follows:

\begin{itemize}
\item If $b_{1},b_{2},\ldots,b_{k}$ are all nonzero, then it is defined as the
smallest positive element of the set $\operatorname*{Mul}\left(  b_{1}%
,b_{2},\ldots,b_{k}\right)  $. This smallest positive element is well-defined
(by Proposition \ref{prop.ent.Mul.exi}), and is a positive integer (obviously).

\item If $b_{1},b_{2},\ldots,b_{k}$ are not all nonzero (i.e., at least one of
$b_{1},b_{2},\ldots,b_{k}$ is zero), then it is defined to be $0$.
\end{itemize}

Thus, in either case, this lowest common multiple is a nonnegative integer. We
denote it by $\operatorname{lcm}\left(  b_{1},b_{2},\ldots,b_{k}\right)  $.

We shall also use the word \textquotedblleft\textit{lcm}\textquotedblright\ as
shorthand for \textquotedblleft lowest common multiple\textquotedblright.
\end{definition}

Some authors say \textquotedblleft\textit{least common multiple}%
\textquotedblright\ instead of \textquotedblleft lowest common
multiple\textquotedblright.

We are slightly abusing the word \textquotedblleft lowest common
multiple\textquotedblright, of course; it would be more precise to say
\textquotedblleft lowest \textbf{positive} common multiple\textquotedblright,
and even this would only hold for the case when $b_{1},b_{2},\ldots,b_{k}$ are
all nonzero. Taken literally, a \textquotedblleft lowest common
multiple\textquotedblright\ of $2$ and $3$ would not exist, since $2$ and $3$
have infinitely many negative common multiples.

Note that the lcm of a single number is the absolute value of this number:
i.e., we have $\operatorname{lcm}\left(  a\right)  =\left\vert a\right\vert $
for each $a\in\mathbb{Z}$. (This is easy to prove.) Also, the lcm of an empty
list of numbers is $1$: that is, $\operatorname{lcm}\left(  {}\right)  =1$.

We observe a trivial property of lcms, which (for the sake of brevity) we only
state for two integers $a$ and $b$ despite it holding for any number of
integers (with the same proof):

\begin{proposition}
\label{prop.ent.lcm.divides}Let $a,b\in\mathbb{Z}$.

\textbf{(a)} We have $0\in\operatorname*{Mul}\left(  a,b\right)  $.

\textbf{(b)} We have $\operatorname{lcm}\left(  a,b\right)  \in
\operatorname*{Mul}\left(  a,b\right)  $.

\textbf{(c)} We have $a\mid\operatorname{lcm}\left(  a,b\right)  $ and
$b\mid\operatorname{lcm}\left(  a,b\right)  $.
\end{proposition}

\begin{proof}
[Proof of Proposition \ref{prop.ent.lcm.divides}.]\textbf{(a)} The integer $0$
clearly satisfies $\left(  a\mid0\text{ and }b\mid0\right)  $. In other words,
$0$ is a common multiple of $a$ and $b$ (by the definition of a
\textquotedblleft common multiple\textquotedblright). In other words,
$0\in\operatorname*{Mul}\left(  a,b\right)  $ (by the definition of
$\operatorname*{Mul}\left(  a,b\right)  $). This proves Proposition
\ref{prop.ent.lcm.divides} \textbf{(a)}.

\textbf{(b)} If the two integers $a$ and $b$ are not all nonzero, then
Proposition \ref{prop.ent.lcm.divides} \textbf{(b)}
holds\footnote{\textit{Proof.} Assume that the two integers $a$ and $b$ are
not all nonzero. Hence, Definition \ref{def.ent.lcm.lcm} shows that
$\operatorname{lcm}\left(  a,b\right)  =0\in\operatorname*{Mul}\left(
a,b\right)  $ (by Proposition \ref{prop.ent.lcm.divides} \textbf{(a)}). Thus,
Proposition \ref{prop.ent.lcm.divides} \textbf{(b)} holds.}. Hence, for the
rest of this proof, we WLOG assume that the two integers $a$ and $b$ are all
nonzero. Thus, Definition \ref{def.ent.lcm.lcm} yields that
$\operatorname{lcm}\left(  a,b\right)  $ is the smallest positive element of
the set $\operatorname*{Mul}\left(  a,b\right)  $. Hence, $\operatorname{lcm}%
\left(  a,b\right)  \in\operatorname*{Mul}\left(  a,b\right)  $. This proves
Proposition \ref{prop.ent.lcm.divides} \textbf{(b)}.

\textbf{(c)} Proposition \ref{prop.ent.lcm.divides} \textbf{(b)} yields
$\operatorname{lcm}\left(  a,b\right)  \in\operatorname*{Mul}\left(
a,b\right)  $. In other words, $\operatorname{lcm}\left(  a,b\right)  $ is a
common multiple of $a$ and $b$ (by the definition of $\operatorname*{Mul}%
\left(  a,b\right)  $). In other words, we have $\left(  a\mid
\operatorname{lcm}\left(  a,b\right)  \text{ and }b\mid\operatorname{lcm}%
\left(  a,b\right)  \right)  $ (by the definition of \textquotedblleft common
multiple\textquotedblright). This proves Proposition
\ref{prop.ent.lcm.divides} \textbf{(c)}.
\end{proof}

The following theorem yields a good way of computing lcms of two numbers
(since we already know how to compute gcds via the Euclidean algorithm):

\begin{theorem}
\label{thm.ent.lcm.gcd*lcm}Let $a,b\in\mathbb{Z}$. Then, $\gcd\left(
a,b\right)  \cdot\operatorname{lcm}\left(  a,b\right)  =\left\vert
ab\right\vert $.
\end{theorem}

\begin{proof}
[Proof of Theorem \ref{thm.ent.lcm.gcd*lcm}.]If at least one of the two
numbers $a$ and $b$ is $0$, then Theorem \ref{thm.ent.lcm.gcd*lcm}
holds\footnote{\textit{Proof.} Assume that at least one of the two numbers $a$
and $b$ is $0$. Thus, the product $ab$ is $0$. Hence, $ab=0$, so that
$\left\vert ab\right\vert =0$.
\par
On the other hand, the two numbers $a,b$ are not all nonzero (since at least
one of the two numbers $a$ and $b$ is $0$). Hence, Definition
\ref{def.ent.lcm.lcm} shows that $\operatorname{lcm}\left(  a,b\right)  =0$.
Comparing $\gcd\left(  a,b\right)  \cdot\underbrace{\operatorname{lcm}\left(
a,b\right)  }_{=0}=0$ with $\left\vert ab\right\vert =0$, we obtain
$\gcd\left(  a,b\right)  \cdot\operatorname{lcm}\left(  a,b\right)
=\left\vert ab\right\vert $. In other words, Theorem \ref{thm.ent.lcm.gcd*lcm}
holds.}. Hence, for the rest of this proof, we WLOG assume that none of the
two numbers $a$ and $b$ is $0$. In other words, $a$ and $b$ are nonzero. Thus,
Definition \ref{def.ent.lcm.lcm} yields that $\operatorname{lcm}\left(
a,b\right)  $ is the smallest positive element of the set $\operatorname*{Mul}%
\left(  a,b\right)  $. Also, $\gcd\left(  a,b\right)  $ is a positive integer
(since $a$ and $b$ are nonzero) and thus nonzero. Hence, we can define
$c\in\mathbb{Q}$ by $c=\dfrac{ab}{\gcd\left(  a,b\right)  }$. Consider this
$c$. From $c=\dfrac{ab}{\gcd\left(  a,b\right)  }$, we obtain $ab=\gcd\left(
a,b\right)  \cdot c$.

Let $d=\left\vert c\right\vert $. The number $c=\dfrac{ab}{\gcd\left(
a,b\right)  }$ is nonzero (since $a$ and $b$ are nonzero). Hence, its absolute
value $\left\vert c\right\vert $ is positive. In other words, $d$ is positive
(since $d=\left\vert c\right\vert $). From $ab=\gcd\left(  a,b\right)  \cdot
c$, we obtain%
\begin{align}
\left\vert ab\right\vert  &  =\left\vert \gcd\left(  a,b\right)  \cdot
c\right\vert =\underbrace{\left\vert \gcd\left(  a,b\right)  \right\vert
}_{\substack{=\gcd\left(  a,b\right)  \\\text{(since }\gcd\left(  a,b\right)
\text{ is positive)}}}\cdot\underbrace{\left\vert c\right\vert }%
_{=d}\nonumber\\
&  \ \ \ \ \ \ \ \ \ \ \left(  \text{by (\ref{eq.ent.div.abs(xy)}), applied to
}\gcd\left(  a,b\right)  \text{ and }c\text{ instead of }x\text{ and }y\right)
\nonumber\\
&  =\gcd\left(  a,b\right)  \cdot d. \label{pf.thm.ent.lcm.gcd*lcm.1}%
\end{align}
Solving this for $d$, we find $d=\dfrac{\left\vert ab\right\vert }{\gcd\left(
a,b\right)  }$ (since $\gcd\left(  a,b\right)  $ is nonzero).

We have $\gcd\left(  a,b\right)  \mid b$ (by Proposition
\ref{prop.ent.gcd.props1} \textbf{(f)}). Thus, $\dfrac{b}{\gcd\left(
a,b\right)  }$ is an integer. Now, $c=\dfrac{ab}{\gcd\left(  a,b\right)
}=a\cdot\dfrac{b}{\gcd\left(  a,b\right)  }$ is the product of two integers
(since $a$ and $\dfrac{b}{\gcd\left(  a,b\right)  }$ are integers). Therefore,
$c$ itself is an integer. Thus, $d$ is an integer as well (since $d=\left\vert
c\right\vert $). Moreover, $c=a\cdot\dfrac{b}{\gcd\left(  a,b\right)  }$ shows
that $a\mid c$ (since $\dfrac{b}{\gcd\left(  a,b\right)  }$ is an integer).
But Exercise \ref{exe.ent.div.aabs} \textbf{(a)} (applied to $c$ instead of
$a$) yields $c\mid\left\vert c\right\vert $ (this means \textquotedblleft$c$
divides $\left\vert c\right\vert $\textquotedblright). In other words, $c\mid
d$ (since $d=\left\vert c\right\vert $). Hence, $a\mid c\mid d$.

So we have proven that $a\mid d$. Similarly, $b\mid d$. Thus, we know that
$\left(  a\mid d\text{ and }b\mid d\right)  $. In other words, $d$ is a common
multiple of $a$ and $b$ (by the definition of a \textquotedblleft common
multiple\textquotedblright). In other words, $d\in\operatorname*{Mul}\left(
a,b\right)  $ (by the definition of $\operatorname*{Mul}\left(  a,b\right)
$). Thus, $d$ is a positive element of the set $\operatorname*{Mul}\left(
a,b\right)  $ (since $d\in\operatorname*{Mul}\left(  a,b\right)  $).

We shall now show that $d$ is the smallest positive element of this set.
Indeed, let $x$ be any positive element of $\operatorname*{Mul}\left(
a,b\right)  $. We are going to prove that $x\geq d$.

In fact, $x\in\operatorname*{Mul}\left(  a,b\right)  $. In other words, $x$ is
a common multiple of $a$ and $b$. In other words, we have $\left(  a\mid
x\text{ and }b\mid x\right)  $. Hence, Theorem \ref{thm.ent.gcd.combine}
(applied to $x$ instead of $c$) yields $ab\mid\gcd\left(  a,b\right)  \cdot
x$. Both numbers $\gcd\left(  a,b\right)  $ and $x$ are positive; hence, their
product $\gcd\left(  a,b\right)  \cdot x$ is positive as well, and thus we
have $\gcd\left(  a,b\right)  \cdot x\neq0$. Hence, Proposition
\ref{prop.ent.div.1} \textbf{(b)} (applied to $ab$ and $\gcd\left(
a,b\right)  \cdot x$ instead of $a$ and $b$) yields $\left\vert ab\right\vert
\leq\left\vert \gcd\left(  a,b\right)  \cdot x\right\vert =\gcd\left(
a,b\right)  \cdot x$ (since $\gcd\left(  a,b\right)  \cdot x$ is positive).
Thus,%
\[
\gcd\left(  a,b\right)  \cdot x\geq\left\vert ab\right\vert =\gcd\left(
a,b\right)  \cdot d\ \ \ \ \ \ \ \ \ \ \left(  \text{by
(\ref{pf.thm.ent.lcm.gcd*lcm.1})}\right)  .
\]
We can divide this inequality by $\gcd\left(  a,b\right)  $ (since
$\gcd\left(  a,b\right)  $ is positive), and thus obtain $x\geq d$.

Now, forget that we fixed $x$. We thus have proven that each positive element
$x$ of the set $\operatorname*{Mul}\left(  a,b\right)  $ satisfies $x\geq d$.
Hence, $d$ is the \textbf{smallest} positive element of the set
$\operatorname*{Mul}\left(  a,b\right)  $ (since we already know that $d$ is a
positive element of the set $\operatorname*{Mul}\left(  a,b\right)  $). In
other words, $d$ is $\operatorname{lcm}\left(  a,b\right)  $ (since
$\operatorname{lcm}\left(  a,b\right)  $ is the smallest positive element of
the set $\operatorname*{Mul}\left(  a,b\right)  $). In other words,
$d=\operatorname{lcm}\left(  a,b\right)  $. Hence,
(\ref{pf.thm.ent.lcm.gcd*lcm.1}) becomes $\left\vert ab\right\vert
=\gcd\left(  a,b\right)  \cdot\underbrace{d}_{=\operatorname{lcm}\left(
a,b\right)  }=\gcd\left(  a,b\right)  \cdot\operatorname{lcm}\left(
a,b\right)  $. This proves Theorem \ref{thm.ent.lcm.gcd*lcm}.
\end{proof}

Next, we state an analogue of Theorem \ref{thm.ent.gcd.uniprop} (with all
divisibilities flipped):

\begin{theorem}
\label{thm.ent.lcm.uniprop}Let $a,b\in\mathbb{Z}$. Then:

\textbf{(a)} For each $m\in\mathbb{Z}$, we have the following logical
equivalence:%
\begin{equation}
\left(  a\mid m\ \text{and }b\mid m\right)  \ \Longleftrightarrow\ \left(
\operatorname{lcm}\left(  a,b\right)  \mid m\right)  .
\label{eq.thm.ent.lcm.uniprop.equiv}%
\end{equation}


\textbf{(b)} The common multiples of $a$ and $b$ are precisely the multiples
of $\operatorname{lcm}\left(  a,b\right)  $.

\textbf{(c)} We have $\operatorname*{Mul}\left(  a,b\right)
=\operatorname*{Mul}\left(  \gcd\left(  a,b\right)  \right)  $.
\end{theorem}

Again, the three parts of this theorem are saying the same thing from slightly
different perspectives. Our proof of Theorem \ref{thm.ent.lcm.uniprop} will
rely on the following lemma:

\begin{lemma}
\label{lem.ent.lcm.uniprop}Let $m,a,b\in\mathbb{Z}$ be such that $a\mid m$ and
$b\mid m$. Then, $\operatorname{lcm}\left(  a,b\right)  \mid m$.
\end{lemma}

Lemma \ref{lem.ent.lcm.uniprop} is similar to Lemma \ref{lem.ent.gcd.uniprop},
but its proof is not:

\begin{proof}
[Proof of Lemma \ref{lem.ent.lcm.uniprop}.]If at least one of the two numbers
$a$ and $b$ is $0$, then Lemma \ref{lem.ent.lcm.uniprop}
holds\footnote{\textit{Proof.} Assume that at least one of the two numbers $a$
and $b$ is $0$. In other words, $a=0$ or $b=0$. Let us WLOG assume that $a=0$
(since the proof in the case $b=0$ is analogous). We have $a\mid m$, thus
$0=a\mid m$.
\par
On the other hand, the two numbers $a,b$ are not all nonzero (since at least
one of the two numbers $a$ and $b$ is $0$). Hence, Definition
\ref{def.ent.lcm.lcm} shows that $\operatorname{lcm}\left(  a,b\right)
=0=a\mid m$. In other words, Lemma \ref{lem.ent.lcm.uniprop} holds.}. Hence,
for the rest of this proof, we WLOG assume that none of the two numbers $a$
and $b$ is $0$. In other words, $a$ and $b$ are nonzero. Thus, Definition
\ref{def.ent.lcm.lcm} yields that $\operatorname{lcm}\left(  a,b\right)  $ is
the smallest positive element of the set $\operatorname*{Mul}\left(
a,b\right)  $. Set $n=\operatorname{lcm}\left(  a,b\right)  $. Thus, $n$ is
the smallest positive element of the set $\operatorname*{Mul}\left(
a,b\right)  $ (since $\operatorname{lcm}\left(  a,b\right)  $ is the smallest
positive element of the set $\operatorname*{Mul}\left(  a,b\right)  $).
Therefore, $n$ is a positive integer and belongs to $\operatorname*{Mul}%
\left(  a,b\right)  $.

Now, $n$ is a common multiple of $a$ and $b$ (since $n$ belongs to
$\operatorname*{Mul}\left(  a,b\right)  $). In other words, we have $\left(
a\mid n\text{ and }b\mid n\right)  $.

Our goal is to prove that $\operatorname{lcm}\left(  a,b\right)  \mid m$. In
other words, our goal is to prove that $n\mid m$ (since $n=\operatorname{lcm}%
\left(  a,b\right)  $). Assume the contrary. Thus, we don't have $n\mid m$.
Hence, we don't have $m\%n=0$ (because Corollary \ref{cor.ent.quo-rem.remmod}
\textbf{(b)} (applied to $u=m$) shows that we have $n\mid m$ if and only if
$m\%n=0$). In other words, we have $m\%n\neq0$.

Corollary \ref{cor.ent.quo-rem.remmod} \textbf{(a)} (applied to $u=m$) yields
that $m\%n\in\left\{  0,1,\ldots,n-1\right\}  $ and $m\%n\equiv
m\operatorname{mod}n$. Combining $m\%n\in\left\{  0,1,\ldots,n-1\right\}  $
with $m\%n\neq0$, we obtain $m\%n\in\left\{  0,1,\ldots,n-1\right\}
\setminus\left\{  0\right\}  =\left\{  1,2,\ldots,n-1\right\}  $. Hence,
$m\%n$ is a positive integer and satisfies $m\%n\leq n-1<n$.

From $m\%n\equiv m\operatorname{mod}n$ and $a\mid n$, we obtain $m\%n\equiv
m\operatorname{mod}a$ (by Proposition \ref{prop.ent.mod.basics} \textbf{(e)},
applied to $a$, $m\%n$ and $m$ instead of $m$, $a$ and $b$). But
$m\equiv0\operatorname{mod}a$ (since $a\mid m$). Thus, $m\%n\equiv
m\equiv0\operatorname{mod}a$. In other words, $a\mid m\%n$. Similarly, $b\mid
m\%n$.

So we have proven that $\left(  a\mid m\%n\text{ and }b\mid m\%n\right)  $. In
other words, $m\%n$ is a common multiple of $a$ and $b$. In other words,
$m\%n\in\operatorname*{Mul}\left(  a,b\right)  $. Therefore, $m\%n$ is a
positive element of $\operatorname*{Mul}\left(  a,b\right)  $ (since $m\%n$ is
positive). Thus, $m\%n\geq n$ (since $n$ is the \textbf{smallest} positive
element of $\operatorname*{Mul}\left(  a,b\right)  $). This contradicts the
fact that $m\%n<n$. This contradiction shows that our assumption was false.
Hence, Lemma \ref{lem.ent.lcm.uniprop} is proven.
\end{proof}

\begin{proof}
[Proof of Theorem \ref{thm.ent.lcm.uniprop}.]\textbf{(a)} Let $m\in\mathbb{Z}%
$. In order to prove (\ref{eq.thm.ent.lcm.uniprop.equiv}), we need to prove
the \textquotedblleft$\Longrightarrow$\textquotedblright\ and
\textquotedblleft$\Longleftarrow$\textquotedblright\ directions of the
equivalence (\ref{eq.thm.ent.lcm.uniprop.equiv}). But this is easy: The
\textquotedblleft$\Longrightarrow$\textquotedblright\ direction is just the
statement of Lemma \ref{lem.ent.lcm.uniprop}, whereas the \textquotedblleft%
$\Longleftarrow$\textquotedblright\ direction is trivial (to wit: if
$\operatorname{lcm}\left(  a,b\right)  \mid m$, then%
\begin{align*}
a  &  \mid\operatorname{lcm}\left(  a,b\right)  \ \ \ \ \ \ \ \ \ \ \left(
\text{by Proposition \ref{prop.ent.lcm.divides} \textbf{(c)}}\right) \\
&  \mid m
\end{align*}
and%
\begin{align*}
b  &  \mid\operatorname{lcm}\left(  a,b\right)  \ \ \ \ \ \ \ \ \ \ \left(
\text{by Proposition \ref{prop.ent.lcm.divides} \textbf{(c)}}\right) \\
&  \mid m
\end{align*}
and thus $\left(  a\mid m\ \text{and }b\mid m\right)  $). Hence, the
equivalence (\ref{eq.thm.ent.lcm.uniprop.equiv}) is proven. This proves
Theorem \ref{thm.ent.lcm.uniprop} \textbf{(a)}.

\textbf{(b)} Theorem \ref{thm.ent.lcm.uniprop} \textbf{(b)} can be derived
from Theorem \ref{thm.ent.lcm.uniprop} \textbf{(a)} in the same way as Theorem
\ref{thm.ent.gcd.uniprop} \textbf{(b)} was derived from Theorem
\ref{thm.ent.gcd.uniprop} \textbf{(a)} (after the necessary changes are made
-- such as flipping all divisibility relations and replacing \textquotedblleft
divisor\textquotedblright\ by \textquotedblleft multiple\textquotedblright).

\textbf{(c)} Theorem \ref{thm.ent.lcm.uniprop} \textbf{(c)} can be derived
from Theorem \ref{thm.ent.lcm.uniprop} \textbf{(b)} in the same way as Theorem
\ref{thm.ent.gcd.uniprop} \textbf{(c)} was derived from Theorem
\ref{thm.ent.gcd.uniprop} \textbf{(b)} (after the necessary changes are made
-- such as flipping all divisibility relations and replacing \textquotedblleft
divisor\textquotedblright\ by \textquotedblleft multiple\textquotedblright).
\end{proof}

Our next claim is an analogue of Theorem \ref{thm.ent.gcd.uniprop-mul}:

\begin{theorem}
\label{thm.ent.lcm.uniprop-mul}Let $b_{1},b_{2},\ldots,b_{k}$ be integers.

\textbf{(a)} For each $m\in\mathbb{Z}$, we have the following logical
equivalence:%
\[
\left(  b_{i}\mid m\text{ for all }i\in\left\{  1,2,\ldots,k\right\}  \right)
\ \Longleftrightarrow\ \left(  \operatorname{lcm}\left(  b_{1},b_{2}%
,\ldots,b_{k}\right)  \mid m\right)  .
\]


\textbf{(b)} The common multiples of $b_{1},b_{2},\ldots,b_{k}$ are precisely
the multiples of $\operatorname{lcm}\left(  b_{1},b_{2},\ldots,b_{k}\right)  $.

\textbf{(c)} We have $\operatorname*{Mul}\left(  b_{1},b_{2},\ldots
,b_{k}\right)  =\operatorname*{Mul}\left(  \operatorname{lcm}\left(
b_{1},b_{2},\ldots,b_{k}\right)  \right)  $.

\textbf{(d)} If $k>0$, then%
\[
\operatorname{lcm}\left(  b_{1},b_{2},\ldots,b_{k}\right)  =\operatorname{lcm}%
\left(  \operatorname{lcm}\left(  b_{1},b_{2},\ldots,b_{k-1}\right)
,b_{k}\right)  .
\]

\end{theorem}

\begin{proof}
[Proof of Theorem \ref{thm.ent.lcm.uniprop-mul} (sketched).]It is not hard to
transform our above proof of Theorem \ref{thm.ent.gcd.uniprop-mul} into a
proof of Theorem \ref{thm.ent.lcm.uniprop-mul}. To do so, we need (of course)
to flip the divisibility relations and replace \textquotedblleft
divisor\textquotedblright\ by \textquotedblleft multiple\textquotedblright%
\ and \textquotedblleft$\gcd$\textquotedblright\ by \textquotedblleft%
$\operatorname{lcm}$\textquotedblright. (Some more changes need to be made as
well -- for example, the induction base needs to be handled differently, and
the WLOG assumption that \textquotedblleft the integers $b_{1},b_{2}%
,\ldots,b_{\ell}$ are not all $0$\textquotedblright\ needs to be replaced by a
WLOG assumption that \textquotedblleft the integers $b_{1},b_{2}%
,\ldots,b_{\ell}$ are all nonzero\textquotedblright. Also, \textquotedblleft
largest element\textquotedblright\ needs to be replaced by \textquotedblleft
smallest positive element\textquotedblright. But these are fairly
straightforward changes; the main thrust of the argument remains unchanged.)
\end{proof}

\begin{exercise}
\label{exe.ent.lcm.lcmabc}Let $a,b,c$ be three integers.

\textbf{(a)} Prove that $\gcd\left(  a,b,c\right)  \cdot\operatorname{lcm}%
\left(  bc,ca,ab\right)  =\left\vert abc\right\vert $.

\textbf{(b)} Prove that $\operatorname{lcm}\left(  a,b,c\right)  \cdot
\gcd\left(  bc,ca,ab\right)  =\left\vert abc\right\vert $.
\end{exercise}

\begin{fineprint}
\begin{proof}
[Solution to Exercise \ref{exe.ent.lcm.lcmabc}.]Let us prove a more general fact:

\begin{statement}
\textit{Claim 1:} Let $x,y,z,N$ be four integers such that $ax=by=cz=N$. Then,
$\gcd\left(  a,b,c\right)  \cdot\operatorname{lcm}\left(  x,y,z\right)
=\left\vert N\right\vert $.
\end{statement}

Once we have proven Claim 1, we will immediately obtain Exercise
\ref{exe.ent.lcm.lcmabc} \textbf{(a)} by applying Claim 1 to $x=bc$, $y=ca$,
$z=ab$ and $N=abc$; and we will obtain Exercise \ref{exe.ent.lcm.lcmabc}
\textbf{(b)} easily in a similar way (see below for the details). Thus, let us
focus on proving Claim 1.

\textit{Proof of Claim 1:} If the integers $x,y,z$ are not all nonzero, then
Claim 1 holds\footnote{\textit{Proof.} Assume that the integers $x,y,z$ are
not all nonzero. In other words, $x=0$ or $y=0$ or $z=0$. We thus WLOG assume
that $x=0$ (since the proofs in the two cases $y=0$ and $z=0$ are analogous).
\par
The integers $x,y,z$ are not all nonzero. Hence, Definition
\ref{def.ent.lcm.lcm} yields that their lowest common multiple is $0$. In
other words, $\operatorname{lcm}\left(  x,y,z\right)  =0$.
\par
But $ax=N$, thus $N=a\underbrace{x}_{=0}=0$. Hence, $\left\vert N\right\vert
=\left\vert 0\right\vert =0$. Comparing this with $\gcd\left(  a,b,c\right)
\cdot\underbrace{\operatorname{lcm}\left(  x,y,z\right)  }_{=0}=0$, we obtain
$\gcd\left(  a,b,c\right)  \cdot\operatorname{lcm}\left(  x,y,z\right)
=\left\vert N\right\vert $. Hence, Claim 1 holds, qed.}. Thus, for the rest of
this proof, we WLOG assume that the integers $x,y,z$ are all nonzero. Hence,
$\operatorname{lcm}\left(  x,y,z\right)  $ is the smallest positive element of
the set $\operatorname*{Mul}\left(  x,y,z\right)  $ (by Definition
\ref{def.ent.lcm.lcm}). Thus, $\operatorname{lcm}\left(  x,y,z\right)  $ is a
positive integer.

If the integers $a,b,c$ are all zero, then Claim 1
holds\footnote{\textit{Proof.} Assume that the integers $a,b,c$ are all zero.
Hence, $\gcd\left(  a,b,c\right)  =0$ (by Definition \ref{def.ent.gcd.gcd}).
Also, $a=0$ (since $a,b,c$ are all zero).
\par
But $ax=N$, thus $N=\underbrace{a}_{=0}x=0$. Hence, $\left\vert N\right\vert
=\left\vert 0\right\vert =0$. Comparing this with $\underbrace{\gcd\left(
a,b,c\right)  }_{=0}\cdot\operatorname{lcm}\left(  x,y,z\right)  =0$, we
obtain $\gcd\left(  a,b,c\right)  \cdot\operatorname{lcm}\left(  x,y,z\right)
=\left\vert N\right\vert $. Hence, Claim 1 holds, qed.}. Hence, for the rest
of this proof, we WLOG assume that the integers $a,b,c$ are not all zero.
Hence, $\gcd\left(  a,b,c\right)  $ is a positive integer (by Definition
\ref{def.ent.gcd.gcd}). Denote this positive integer by $g$. Hence,
$g=\gcd\left(  a,b,c\right)  $.

Definition \ref{def.ent.gcd.gcd} also shows that $\gcd\left(  a,b,c\right)  $
is the largest element of the set $\operatorname*{Div}\left(  a,b,c\right)  $
(since $a,b,c$ are not all zero). Hence, $\gcd\left(  a,b,c\right)
\in\operatorname*{Div}\left(  a,b,c\right)  $. In other words, $g\in
\operatorname*{Div}\left(  a,b,c\right)  $ (since $g=\gcd\left(  a,b,c\right)
$). In other words, $g$ is a common divisor of $a,b,c$ (by the definition of
$\operatorname*{Div}\left(  a,b,c\right)  $). In other words, $g$ is an
integer satisfying $\left(  g\mid a\text{ and }g\mid b\text{ and }g\mid
c\right)  $. Thus, $g\mid a\mid ax=N$. In other words, there exists an integer
$h$ such that $N=gh$. Consider this $h$.

It is easy to see that $N\neq0$\ \ \ \ \footnote{\textit{Proof.} Assume the
contrary. Thus, $N=0$. But $x\neq0$ (since $x,y,z$ are nonzero). Hence, from
$ax=N=0$, we obtain $a=0$. Similarly, $b=0$ and $c=0$. Thus, the integers
$a,b,c$ are all zero. This contradicts the fact that the integers $a,b,c$ are
not all zero. This contradiction shows that our assumption was wrong, qed.}.
Now, $gh=N\neq0$ and thus $h\neq0$. Hence, $\left\vert h\right\vert $ is a
positive integer (since $h$ is an integer). Denote this positive integer by
$m$. Thus, $m=\left\vert h\right\vert $.

Also, set $N^{\prime}=\left\vert N\right\vert $. Thus, $N^{\prime}$ is an
integer satisfying%
\begin{align}
N^{\prime}  &  =\left\vert \underbrace{N}_{=gh}\right\vert =\left\vert
gh\right\vert =\underbrace{\left\vert g\right\vert }%
_{\substack{=g\\\text{(since }g\text{ is positive)}}}\cdot
\underbrace{\left\vert h\right\vert }_{=m}\ \ \ \ \ \ \ \ \ \ \left(  \text{by
(\ref{eq.ent.div.abs(xy)})}\right) \nonumber\\
&  =gm. \label{sol.ent.lcm.lcmabc.c1.pf.3}%
\end{align}


Our next goal is to prove that $m=\operatorname{lcm}\left(  x,y,z\right)  $.
First, we shall prove that $m\in\operatorname*{Mul}\left(  x,y,z\right)  $.

Indeed, we have $h\neq0$. Hence, Exercise \ref{exe.ent.div.acbc} (applied to
$g$, $a$ and $h$ instead of $a$, $b$ and $c$) shows that $g\mid a$ holds if
and only if $gh\mid ah$. Hence, $gh\mid ah$ holds (since $g\mid a$ holds).
Now, $xa=ax=N=gh\mid ah=ha$. But $a\neq0$ (since $ax=N\neq0$). Thus, Exercise
\ref{exe.ent.div.acbc} (applied to $x$, $h$ and $a$ instead of $a$, $b$ and
$c$) shows that $x\mid h$ holds if and only if $xa\mid ha$. Hence, $x\mid h$
holds (since $xa\mid ha$ holds). But Exercise \ref{exe.ent.div.aabs}
\textbf{(a)} (applied to $h$ instead of $a$) yields $h\mid\left\vert
h\right\vert =m$. Thus, $x\mid h\mid m$. Similarly, $y\mid m$ and $z\mid m$.
Thus, we have $\left(  x\mid m\text{ and }y\mid m\text{ and }z\mid m\right)
$. In other words, $m$ is a common multiple of $x,y,z$. In other words,
$m\in\operatorname*{Mul}\left(  x,y,z\right)  $. So we know that $m$ is a
positive element of the set $\operatorname*{Mul}\left(  x,y,z\right)  $ (since
$m$ is positive).

We shall now show that $m$ is the smallest positive element of this set.
Indeed, let $w$ be any positive element of $\operatorname*{Mul}\left(
x,y,z\right)  $. We are going to prove that $w\geq m$.

In fact, $w\in\operatorname*{Mul}\left(  x,y,z\right)  $. In other words, $w$
is a common multiple of $x,y,z$. In other words, we have $\left(  x\mid
w\text{ and }y\mid w\text{ and }z\mid w\right)  $. Also, $w\neq0$ (since $w$
is positive).

We have $wa\neq0$ (since $w\neq0$ and $a\neq0$). Hence, the integers
$wa,wb,wc$ are not all zero. Thus, Definition \ref{def.ent.gcd.gcd} shows that
$\gcd\left(  wa,wb,wc\right)  $ is the largest element of the set
$\operatorname*{Div}\left(  wa,wb,wc\right)  $.

We have $a\neq0$. Hence, Exercise \ref{exe.ent.div.acbc} (applied to $x$, $w$
and $a$ instead of $a$, $b$ and $c$) shows that $x\mid w$ holds if and only if
$xa\mid wa$. Hence, $xa\mid wa$ holds (since $x\mid w$). Thus, $N=ax=xa\mid
wa$. But Exercise \ref{exe.ent.div.aabs} \textbf{(b)} (applied to $N$ instead
of $a$) yields $\left\vert N\right\vert \mid N$. In other words, $N^{\prime
}\mid N$ (since $N^{\prime}=\left\vert N\right\vert $). Hence, $N^{\prime}\mid
N\mid wa$. Similarly, $N^{\prime}\mid wb$ and $N^{\prime}\mid wc$. Thus,
$\left(  N^{\prime}\mid wa\text{ and }N^{\prime}\mid wb\text{ and }N^{\prime
}\mid wc\right)  $. In other words, $N^{\prime}$ is a common divisor of
$wa,wb,wc$. In other words, $N^{\prime}\in\operatorname*{Div}\left(
wa,wb,wc\right)  $. Hence, $N^{\prime}\leq\gcd\left(  wa,wb,wc\right)  $
(since $\gcd\left(  wa,wb,wc\right)  $ is the \textbf{largest} element of the
set $\operatorname*{Div}\left(  wa,wb,wc\right)  $). Now,
(\ref{sol.ent.lcm.lcmabc.c1.pf.3}) yields%
\begin{align*}
gm  &  =N^{\prime}\leq\gcd\left(  wa,wb,wc\right)  =\underbrace{\left\vert
w\right\vert }_{\substack{=w\\\text{(since }w\text{ is positive)}%
}}\underbrace{\gcd\left(  a,b,c\right)  }_{=g}\\
&  \ \ \ \ \ \ \ \ \ \ \left(  \text{by Exercise \ref{exe.ent.gcd.sa,sb,sc}
(applied to }s=w\text{)}\right) \\
&  =wg=gw.
\end{align*}
We can divide both sides of this inequality by $g$ (since $g$ is positive),
and thus obtain $m\leq w$. In other words, $w\geq m$.

Now, forget that we fixed $w$. We thus have proven that each positive element
$w$ of the set $\operatorname*{Mul}\left(  x,y,z\right)  $ satisfies $w\geq
m$. Hence, $m$ is the \textbf{smallest} positive element of the set
$\operatorname*{Mul}\left(  x,y,z\right)  $ (since we already know that $m$ is
a positive element of the set $\operatorname*{Mul}\left(  x,y,z\right)  $). In
other words, $m$ is $\operatorname{lcm}\left(  x,y,z\right)  $ (since
$\operatorname{lcm}\left(  x,y,z\right)  $ is the smallest positive element of
the set $\operatorname*{Mul}\left(  x,y,z\right)  $). In other words,
$m=\operatorname{lcm}\left(  x,y,z\right)  $. Hence,
(\ref{sol.ent.lcm.lcmabc.c1.pf.3}) becomes%
\[
N^{\prime}=\underbrace{g}_{=\gcd\left(  a,b,c\right)  }\underbrace{m}%
_{=\operatorname{lcm}\left(  x,y,z\right)  }=\gcd\left(  a,b,c\right)
\cdot\operatorname{lcm}\left(  x,y,z\right)  .
\]
Thus, $\gcd\left(  a,b,c\right)  \cdot\operatorname{lcm}\left(  x,y,z\right)
=N^{\prime}=\left\vert N\right\vert $. This proves Claim 1.

We can now solve the actual exercise:

\textbf{(a)} We have $a\left(  bc\right)  =b\left(  ca\right)  =c\left(
ab\right)  =abc$. Hence, Claim 1 (applied to $x=bc$, $y=ca$, $z=ab$ and
$N=abc$) yields $\gcd\left(  a,b,c\right)  \cdot\operatorname{lcm}\left(
bc,ca,ab\right)  =\left\vert abc\right\vert $. This solves Exercise
\ref{exe.ent.lcm.lcmabc} \textbf{(a)}.

\textbf{(b)} We have $\left(  bc\right)  a=\left(  ca\right)  b=\left(
ab\right)  c=abc$. Hence, Claim 1 (applied to $bc$, $ca$, $ab$, $a$, $b$, $c$
and $abc$ instead of $a$, $b$, $c$, $x$, $y$, $z$ and $N$) yields $\gcd\left(
bc,ca,ab\right)  \cdot\operatorname{lcm}\left(  a,b,c\right)  =\left\vert
abc\right\vert $. Thus, $\operatorname{lcm}\left(  a,b,c\right)  \cdot
\gcd\left(  bc,ca,ab\right)  =\gcd\left(  bc,ca,ab\right)  \cdot
\operatorname{lcm}\left(  a,b,c\right)  =\left\vert abc\right\vert $. This
solves Exercise \ref{exe.ent.lcm.lcmabc} \textbf{(b)}.
\end{proof}
\end{fineprint}

\subsection{The Chinese remainder theorem (elementary form)}

\begin{theorem}
\label{thm.ent.crt1}Let $m$ and $n$ be two coprime integers. Let
$a,b\in\mathbb{Z}$.

\textbf{(a)} There exists an integer $x\in\mathbb{Z}$ such that%
\[
\left(  x\equiv a\operatorname{mod}m\text{ and }x\equiv b\operatorname{mod}%
n\right)  .
\]


\textbf{(b)} If $x_{1}$ and $x_{2}$ are two such integers $x$, then
$x_{1}\equiv x_{2}\operatorname{mod}mn$.
\end{theorem}

Theorem \ref{thm.ent.crt1} is known as the \textit{Chinese remainder theorem}.
More precisely, there is a sizeable cloud of results that share this name;
Theorem \ref{thm.ent.crt1} is one of the most elementary and basic of these
results. A more general result is Theorem \ref{thm.ent.crt1k} further below.
However, the strongest and most general \textquotedblleft Chinese remainder
theorems\textquotedblright\ rely on concept from abstract algebra such as
rings and ideals; it will take us a while to get to them.

Theorem \ref{thm.ent.crt1} has gotten its name from the fact that
\href{https://en.wikipedia.org/wiki/Chinese_remainder_theorem#History}{a first
glimpse of it appears in \textquotedblleft Master Sun's Mathematical
Manual\textquotedblright\ from the 3rd century AD}; it took centuries until it
become a theorem with proof and precise statement.

The claim of Theorem \ref{thm.ent.crt1} \textbf{(b)} is often restated as
\textquotedblleft This integer $x$ (i.e., the integer $x$ satisfying $\left(
x\equiv a\operatorname{mod}m\text{ and }x\equiv b\operatorname{mod}n\right)
$) is unique modulo $mn$\textquotedblright. The \textquotedblleft modulo
$mn$\textquotedblright\ here signifies that what we are not claiming literal
uniqueness (which would mean that if $x_{1}$ and $x_{2}$ are two such integers
$x$, then $x_{1}=x_{2}$), but merely claiming a weaker form (namely, that if
$x_{1}$ and $x_{2}$ are two such integers $x$, then $x_{1}\equiv
x_{2}\operatorname{mod}mn$).

\begin{example}
Theorem \ref{thm.ent.crt1} \textbf{(a)} (applied to $m=5$, $n=6$ and $a=3$ and
$b=2$) shows that there exists an integer $x\in\mathbb{Z}$ such that%
\[
\left(  x\equiv3\operatorname{mod}5\text{ and }x\equiv2\operatorname{mod}%
6\right)  .
\]
We will soon find such an integer, after we have proved Theorem
\ref{thm.ent.crt1}.
\end{example}

\begin{proof}
[Proof of Theorem \ref{thm.ent.crt1}.]The integers $m$ and $n$ are coprime. In
other words, $m\perp n$, so that $n\perp m$ (by Proposition
\ref{prop.ent.coprime.perp-symm}).

\textbf{(a)} Theorem \ref{thm.ent.coprime.modinv} \textbf{(b)} (applied to $m$
instead of $a$) shows that there exists a $m^{\prime}\in\mathbb{Z}$ such that
$mm^{\prime}\equiv1\operatorname{mod}n$.

Similarly, there exists an $n^{\prime}\in\mathbb{Z}$ such that $nn^{\prime
}\equiv1\operatorname{mod}m$ (since $m$ and $n$ play symmetric roles in
Theorem \ref{thm.ent.crt1}).

Now, set $x_{0}=nn^{\prime}a+mm^{\prime}b$. Then,%
\[
x_{0}=nn^{\prime}a+\underbrace{mm^{\prime}b}_{\equiv0\operatorname{mod}%
m}\equiv\underbrace{nn^{\prime}}_{\equiv1\operatorname{mod}m}a\equiv
a\operatorname{mod}m
\]
(here, we have used the Principle of substitutivity for congruences, which we
described in Section \ref{sect.ent.subst-mod}) and similarly $x_{0}\equiv
b\operatorname{mod}n$. Thus, there exists an integer $x\in\mathbb{Z}$ such
that \newline$\left(  x\equiv a\operatorname{mod}m\text{ and }x\equiv
b\operatorname{mod}n\right)  $ (namely, $x=x_{0}$). This proves Theorem
\ref{thm.ent.crt1} \textbf{(a)}.

\textbf{(b)} Let $x_{1}$ and $x_{2}$ be two such integers $x$. We want to
prove that $x_{1}\equiv x_{2}\operatorname{mod}mn$.

We know that $x_{1}$ is an integer $x$ such that $\left(  x\equiv
a\operatorname{mod}m\text{ and }x\equiv b\operatorname{mod}n\right)  $. Thus,
$x_{1}\equiv a\operatorname{mod}m$ and $x_{1}\equiv b\operatorname{mod}n$.

In particular, $x_{1}\equiv a\operatorname{mod}m$, and similarly $x_{2}\equiv
a\operatorname{mod}m$. Thus, $x_{1}\equiv a\equiv x_{2}\operatorname{mod}m$,
so that $m\mid x_{1}-x_{2}$. Similarly, $n\mid x_{1}-x_{2}$. Since $m\perp n$,
we thus obtain $mn\mid x_{1}-x_{2}$ (by Theorem \ref{thm.ent.coprime.combine},
applied to $m$, $n$ and $x_{1}-x_{2}$ instead of $a$, $b$ and $c$). In other
words, $x_{1}\equiv x_{2}\operatorname{mod}mn$. This proves Theorem
\ref{thm.ent.crt1}.
\end{proof}

\begin{noncompile}
Theorem \ref{thm.ent.crt1} is (the simplest form of) the \textit{Chinese
Remainder Theorem} (Sunzi, 3rd century AD).
\end{noncompile}

\begin{example}
Assume that we want to find an $x\in\mathbb{Z}$ such that%
\[
\left(  x\equiv3\operatorname{mod}5\text{ and }x\equiv2\operatorname{mod}%
6\right)  .
\]
To compute such an $x$, let us follow the proof of Theorem \ref{thm.ent.crt1}
\textbf{(a)} above.

We need a modular inverse $5^{\prime}$ of $5$ modulo $6$. Such an inverse is
$5$, since $5\cdot5\equiv1\operatorname{mod}6$. (In this particular case,
finding this modular inverse was easy, because all we had to do is to test the
$6$ numbers $0,1,2,3,4,5$; it is clear that a modular inverse of $a$ modulo
$m$, if it exists, can be found within the set $\left\{  0,1,\ldots
,m-1\right\}  $. In general, there is
\href{https://en.wikipedia.org/wiki/Modular_multiplicative_inverse#Computation}{a
quick way to find a modular inverse of an integer $a$ modulo an integer $m$
using the \textquotedblleft Extended Euclidean algorithm\textquotedblright}.)

We need a modular inverse $6^{\prime}$ of $6$ modulo $5$. Such an inverse is
$1$, since $6\cdot1\equiv1\operatorname{mod}5$.

Now, the proof of Theorem \ref{thm.ent.crt1} \textbf{(a)} tells us that
$x_{0}=6\cdot6^{\prime}\cdot3+5\cdot5^{\prime}\cdot2$ is an integer
$x\in\mathbb{Z}$ such that $\left(  x\equiv3\operatorname{mod}5\text{ and
}x\equiv2\operatorname{mod}6\right)  $. This $x_{0}$ is%
\[
6\cdot6^{\prime}\cdot3+5\cdot5^{\prime}\cdot2=6\cdot1\cdot3+5\cdot5\cdot2=68.
\]
So we have found an $x\in\mathbb{Z}$ such that $\left(  x\equiv
3\operatorname{mod}5\text{ and }x\equiv2\operatorname{mod}6\right)  $, namely
$x=68$. (We can easily check this: $68\equiv3\operatorname{mod}5$ since
$68-3=5\cdot13$; and $68\equiv2\operatorname{mod}6$ since $68-2=6\cdot11$.)
\end{example}

There is also a version of Theorem \ref{thm.ent.crt1} for multiple integers:

\begin{theorem}
\label{thm.ent.crt1k}Let $m_{1},m_{2},\ldots,m_{k}$ be $k$ mutually coprime
integers. Let $a_{1},a_{2},\ldots,a_{k}\in\mathbb{Z}$.

\textbf{(a)} There exists an integer $x$ such that
\[
\left(  x\equiv a_{i}\operatorname{mod}m_{i}\text{ for all }i\in\left\{
1,2,\ldots,k\right\}  \right)  .
\]


\textbf{(b)} If $x_{1}$ and $x_{2}$ are two such integers $x$, then
$x_{1}\equiv x_{2}\operatorname{mod}m_{1}m_{2}\cdots m_{k}$.
\end{theorem}

Again, Theorem \ref{thm.ent.crt1k} \textbf{(b)} is often stated in the form
\textquotedblleft This integer $x$ is unique modulo $m_{1}m_{2}\cdots m_{k}%
$\textquotedblright.

\begin{proof}
[Proof of Theorem \ref{thm.ent.crt1k}.]TODO: Exercise on induction.
\end{proof}

\subsection{Primes}

\subsubsection{Definition and the Sieve of Eratosthenes}

\begin{definition}
Let $p$ be an integer greater than $1$. We say that $p$ is \textit{prime} if
the only positive divisors of $p$ are $1$ and $p$. A prime integer is often
just called \textit{a prime}.
\end{definition}

Note that we required $p$ to be greater than $1$ here. Thus, $1$ does not
count as prime even though its only positive divisor is $1$ itself.

\begin{example}
\label{exa.ent.primes.1}\textbf{(a)} The only positive divisors of $7$ are $1$
and $7$. Thus, $7$ is a prime.

\textbf{(b)} The positive divisors of $14$ are $1$, $2$, $7$ and $14$. These
are more than just $1$ and $14$. Thus, $14$ is not a prime.

\textbf{(c)} None of the numbers $4,6,8,10,12,14,16,\ldots$ (that is, the
multiples of $2$ that are larger than $2$) is a prime. Indeed, if $p$ is any
of the numbers, then $p$ has a positive divisor other than $1$ and $p$
(namely, $2$), and therefore does not meet the definition of \textquotedblleft
prime\textquotedblright.

\textbf{(d)} None of the numbers $6,9,12,15,18,\ldots$ (that is, the multiples
of $3$ that are larger than $3$) is a prime. Indeed, if $p$ is any of the
numbers, then $p$ has a positive divisor other than $1$ and $p$ (namely, $3$),
and therefore does not meet the definition of \textquotedblleft
prime\textquotedblright.
\end{example}

Parts \textbf{(c)} and \textbf{(d)} of Example \ref{exa.ent.primes.1} suggest
a method for finding all primes up to a given integer:

\begin{example}
Let us say we want to find all primes that are $\leq30$.

\textit{Step 1:} All such primes must lie in $\left\{  2,3,\ldots,30\right\}
$ (since a prime is always an integer greater than $1$); thus, let us first
write down all elements of $\left\{  2,3,\ldots,30\right\}  $:%
\[%
\begin{array}
[c]{cccccccccc}
& 2 & 3 & 4 & 5 & 6 & 7 & 8 & 9 & 10\\
11 & 12 & 13 & 14 & 15 & 16 & 17 & 18 & 19 & 20\\
21 & 22 & 23 & 24 & 25 & 26 & 27 & 28 & 29 & 30
\end{array}
.
\]
(We are using a table just in order to fit these elements on a page.)

We now plan to remove non-prime numbers from this table until only primes are left.

\textit{Step 2:} First, let us remove all multiples of $2$ that are larger
than $2$ from our table, because none of them is a prime (see Example
\ref{exa.ent.primes.1} \textbf{(c)}). We thus are left with%
\[%
\begin{array}
[c]{cccccccccc}
& 2 & 3 &  & 5 &  & 7 &  & 9 & \\
11 &  & 13 &  & 15 &  & 17 &  & 19 & \\
21 &  & 23 &  & 25 &  & 27 &  & 29 &
\end{array}
.
\]


\textit{Step 3:} Next, let us remove all multiples of $3$ that are larger than
$3$ from our table, because none of them is a prime (see Example
\ref{exa.ent.primes.1} \textbf{(d)}). We thus are left with%
\[%
\begin{array}
[c]{cccccccccc}
& 2 & 3 &  & 5 &  & 7 &  &  & \\
11 &  & 13 &  &  &  & 17 &  & 19 & \\
&  & 23 &  & 25 &  &  &  & 29 &
\end{array}
.
\]
(Note that some of these multiples have already been removed in Step 2.)

\textit{Step 4:} Next, let us remove all multiples of $4$ that are larger than
$4$ from our table, because none of them is a prime (for similar reasons). It
turns out that this does not change the table at all, because all such
multiples have already been removed in Step 2. This is not a coincidence:
Since $4$ itself has been removed, we know that $4$ was a multiple of some
number $d<4$ (in this case, $d=2$) whose multiples have been removed;
therefore, all multiples of $4$ are also multiples of $d$ and thus have been
removed along with $4$.

\textit{Step 5:} Next, let us remove all multiples of $5$ that are larger than
$5$ from our table, because none of them is a prime (for similar reasons). We
thus are left with%
\[%
\begin{array}
[c]{cccccccccc}
& 2 & 3 &  & 5 &  & 7 &  &  & \\
11 &  & 13 &  &  &  & 17 &  & 19 & \\
&  & 23 &  &  &  &  &  & 29 &
\end{array}
.
\]


\textit{Step 6:} Next, let us remove all multiples of $6$ that are larger than
$6$ from our table, because none of them is a prime. Just as Step 4, this does
not change the table, since all such multiples have already been removed in
Step 2.

\textit{Step 7:} Next, let us remove all multiples of $7$ that are larger than
$7$ from our table, because none of them is a prime. Again, this does not
change the table, since all such multiples have already been removed.

Proceed likewise until Step 30, at which point the table has become%
\[%
\begin{array}
[c]{cccccccccc}
& 2 & 3 &  & 5 &  & 7 &  &  & \\
11 &  & 13 &  &  &  & 17 &  & 19 & \\
&  & 23 &  &  &  &  &  & 29 &
\end{array}
.
\]
(You are reading it right: None of the steps from Step 6 to Step 30 causes any
changes to the table, since all multiples that these steps attempt to remove
have already been removed beforehand.)

The resulting table has the following property: If $p$ is an element of this
table, then $p$ cannot be a multiple of any $d\in\left\{  2,3,\ldots
,p-1\right\}  $ (because if it was such a multiple, then it would have been
removed from the table in Step $d$ or earlier). In other words, if $p$ is an
element of this table, then $p$ cannot have any divisor $d\in\left\{
2,3,\ldots,p-1\right\}  $. In other words, if $p$ is an element of this table,
then the only positive divisors of $p$ are $1$ and $p$. In other words, if $p$
is an element of this table, then $p$ is prime. Conversely, any prime $\leq30$
is in our table, since the only numbers we have removed from the table were
guaranteed to be non-prime. Thus, the table now contains all the primes
$\leq30$ and only them. So we conclude that the primes $\leq30$ are
$2,3,5,7,11,13,17,19,23,29$.

This method of finding primes is known as the \textbf{sieve of Eratosthenes}.
We could have made it more efficient using the following two tricks:

\begin{itemize}
\item If a number $d\in\left\{  2,3,\ldots,30\right\}  $ has been removed from
the table before Step $d$, then we know immediately that Step $d$ will not
change the table (because all multiples of $d$ have already been removed
before this step). Thus, we do not need to make this step.

\item If $d\in\left\{  2,3,\ldots,30\right\}  $ satisfies $d^{2}>30$, then
Step $d$ will not change the table (because if $m\in\left\{  2,3,\ldots
,30\right\}  $ is a multiple of $d$ that is larger than $d$, then $m$ is also
a multiple of the integer $m/d$ as well (since $d\mid m$ and thus
$m/d\in\mathbb{Z}$ and of course $m/d\mid m$), and therefore $m$ has already
been removed in Step $m/d$ (which has already happened before Step $d$
(because $d^{2}>30\geq m$ and therefore $d>m/d$)). Thus, we only need to take
the Steps $d$ with $d^{2}\leq30$.
\end{itemize}

Together, these tricks tell us that the only steps we need to take are the
Steps 2, 3 and 5.
\end{example}

\begin{center}
\textbf{2019-02-11 lecture}
\end{center}

\subsubsection{Basic properties of primes}

\begin{proposition}
\label{prop.ent.primes.each-i-coprime}Let $p$ be a prime. Then, each
$i\in\left\{  1,2,\ldots,p-1\right\}  $ is coprime to $p$.
\end{proposition}

\begin{proof}
[Proof of Proposition \ref{prop.ent.primes.each-i-coprime}.]Let $i\in\left\{
1,2,\ldots,p-1\right\}  $. We must prove that $i$ is coprime to $p$.

From $i\in\left\{  1,2,\ldots,p-1\right\}  $, we obtain $1\leq i\leq p-1$ and
thus $i\geq1>0$, so that $i\neq0$. Hence, $i$ and $p$ are not all zero. Also,
$\left\vert i\right\vert =i$ (since $i>0$).

Also, $\gcd\left(  i,p\right)  $ is a positive integer (since $i$ and $p$ are
not all zero). Thus, $\left\vert \gcd\left(  i,p\right)  \right\vert
=\gcd\left(  i,p\right)  $.

Proposition \ref{prop.ent.gcd.props1} \textbf{(f)} (applied to $a=i$ and
$b=p$) shows that $\gcd\left(  i,p\right)  \mid i$ and $\gcd\left(
i,p\right)  \mid p$. From $\gcd\left(  i,p\right)  \mid i$ and $i\neq0$, we
obtain $\left\vert \gcd\left(  i,p\right)  \right\vert \leq\left\vert
i\right\vert $ (by Exercise \ref{prop.ent.div.1} \textbf{(a)}, applied to
$a=\gcd\left(  i,p\right)  $ and $b=i$). In view of $\left\vert \gcd\left(
i,p\right)  \right\vert =\gcd\left(  i,p\right)  $ and $\left\vert
i\right\vert =i$, this rewrites as $\gcd\left(  i,p\right)  \leq i$. Hence,
$\gcd\left(  i,p\right)  \leq i\leq p-1<p$ and therefore $\gcd\left(
i,p\right)  \neq p$.

We know that $p$ is prime. In other words, the only positive divisors of $p$
are $1$ and $p$ (by the definition of \textquotedblleft
prime\textquotedblright).

The integer $\gcd\left(  i,p\right)  $ is a positive divisor of $p$ (since
$\gcd\left(  i,p\right)  $ is positive and satisfies $\gcd\left(  i,p\right)
\mid p$), and thus must be either $1$ or $p$ (since the only positive divisors
of $p$ are $1$ and $p$). Since we know that $\gcd\left(  i,p\right)  \neq p$,
we thus conclude that $\gcd\left(  i,p\right)  =1$. In other words, $i$ is
coprime to $p$ (by the definition of \textquotedblleft
coprime\textquotedblright). This proves Proposition
\ref{prop.ent.primes.each-i-coprime}.
\end{proof}

Note that this proposition characterizes primes: If $p>1$ is an integer such
that each $i\in\left\{  1,2,\ldots,p-1\right\}  $ is coprime to $p$, then $p$
is prime. (The proof of this is left as an easy exercise.)

\begin{proposition}
\label{prop.ent.primes.div-or-coprime}Let $p$ be a prime. Let $a\in\mathbb{Z}%
$. Then, either $p\mid a$ or $p\perp a$.
\end{proposition}

\begin{proof}
[Proof of Proposition \ref{prop.ent.primes.div-or-coprime}.]Assume the
contrary. Thus, neither $p\mid a$ nor $p\perp a$.

Corollary \ref{cor.ent.quo-rem.remmod} \textbf{(b)} (applied to $n=p$ and
$u=a$) yields that we have $p\mid a$ if and only if $a\%p=0$. Thus, we don't
have $a\%p=0$ (since we don't have $p\mid a$). In other words, $a\%p\neq0$.

Corollary \ref{cor.ent.quo-rem.remmod} \textbf{(a)} (applied to $n=p$ and
$u=a$) yields that $a\%p\in\left\{  0,1,\ldots,p-1\right\}  $ and $a\%p\equiv
a\operatorname{mod}p$. Combining $a\%p\in\left\{  0,1,\ldots,p-1\right\}  $
with $a\%p\neq0$, we obtain $a\%p\in\left\{  0,1,\ldots,p-1\right\}
\setminus\left\{  0\right\}  =\left\{  1,2,\ldots,p-1\right\}  $. Hence,
Proposition \ref{prop.ent.primes.each-i-coprime} (applied to $i=a\%p$) yields
that $a\%p$ is coprime to $p$. In other words, $\gcd\left(  a\%p,p\right)  =1$.

But $p$ is prime, and thus is a positive integer (by the definition of a
\textquotedblleft prime\textquotedblright). Proposition
\ref{prop.ent.gcd.props1} \textbf{(e)} (applied to $p$ and $a$ instead of $a$
and $b$) yields%
\begin{align*}
\gcd\left(  p,a\right)   &  =\gcd\left(  p,a\%p\right)  =\gcd\left(
a\%p,p\right) \\
&  \ \ \ \ \ \ \ \ \ \ \left(
\begin{array}
[c]{c}%
\text{by Proposition \ref{prop.ent.gcd.props1} \textbf{(b)},}\\
\text{applied to }p\text{ and }a\%p\text{ instead of }a\text{ and }b
\end{array}
\right) \\
&  =1.
\end{align*}
In other words, $p$ is coprime to $a$. In other words, $p\perp a$. This
contradicts the fact that we don't have $p\perp a$.

This contradiction shows that our assumption was false. Hence, Proposition
\ref{prop.ent.primes.div-or-coprime} is proven.
\end{proof}

We note that a converse of Proposition \ref{prop.ent.primes.div-or-coprime}
holds as well: If $p>1$ is an integer such that each $a\in\mathbb{Z}$
satisfies either $p\mid a$ or $p\perp a$, then $p$ is a prime. This is easy to
prove and left to the reader.

\begin{theorem}
\label{thm.ent.primes.pab}Let $p$ be a prime. Let $a,b\in\mathbb{Z}$ such that
$p\mid ab$. Then, $p\mid a$ or $p\mid b$.
\end{theorem}

\begin{proof}
[Proof of Theorem \ref{thm.ent.primes.pab}.]Assume the contrary. Thus, neither
$p\mid a$ nor $p\mid b$.

Proposition \ref{prop.ent.primes.div-or-coprime} yields that either $p\mid a$
or $p\perp a$. Hence, $p\perp a$ (since $p\mid a$ does not hold). But $p\mid
ab$. Hence, Theorem \ref{thm.ent.coprime.cancel} (applied to $p$, $a$ and $b$
instead of $a$, $b$ and $c$) yields $p\mid b$. This contradicts the fact that
we don't have $p\mid b$.

This contradiction shows that our assumption was false. Hence, Theorem
\ref{thm.ent.primes.pab} is proven.
\end{proof}

Again, Theorem \ref{thm.ent.primes.pab} has a converse:

\begin{exercise}
\label{exe.ent.primes.pab-conv}Let $p>1$ be an integer. Assume that for every
$a,b\in\mathbb{Z}$ satisfying $p\mid ab$, we must have $p\mid a$ or $p\mid b$.
Then, $p$ is prime.
\end{exercise}

\begin{fineprint}
\begin{proof}
[Solution to Exercise \ref{exe.ent.primes.pab-conv}.]The integer $p$ is
positive (since $p>1>0$). Thus, $\left\vert p\right\vert =p$.

Let $d$ be a positive divisor of $p$ other than $1$ and $p$. We shall derive a contradiction.

We know that $d$ is a divisor of $p$ \textbf{other than} $1$ and $p$. Hence,
$d\neq1$ and $d\neq p$.

Indeed, $d$ is a divisor of $p$. In other words, there exists an integer $c$
such that $p=dc$. Consider this $c$.

The integer $d$ is positive, therefore nonzero. Hence, we can solve the
equality $p=dc$ for $c$; thus we find $c=p/d>0$ (since both $p$ and $d$ are
positive). Thus, the integer $c$ is positive; hence, $c\geq1$. Also, $d\geq1$
(since $d$ is a positive integer). Combining this with $d\neq1$, we obtain
$d>1$.

Since $c>0$, we can multiply the inequality $d>1$ by $c$. We thus find
$cd>c\cdot1=c$. Hence, $c<cd=dc=p$. Since $c$is positive, we have $\left\vert
c\right\vert =c<p$. But $p$ is positive; thus, $\left\vert p\right\vert
=p>\left\vert c\right\vert $ (since $\left\vert c\right\vert <p$).

Since $d>0$, we can multiply the inequality $c\geq1$ by $d$. We thus find
$dc\geq d\cdot1=d$. Hence, $d\leq dc=p$. Combining this with $d\neq p$, we
obtain $d<p$. Since $d$ is positive, we have $\left\vert d\right\vert =d<p$.
But $p$ is positive; thus, $\left\vert p\right\vert =p>\left\vert d\right\vert
$ (since $\left\vert d\right\vert <p$).

We have $p\mid p=dc$. But let us recall that for every $a,b\in\mathbb{Z}$
satisfying $p\mid ab$, we must have $p\mid a$ or $p\mid b$. Applying this to
$a=d$ and $b=c$, we conclude that $p\mid d$ or $p\mid c$.

We have $d\neq0$ (since $d>0$). Hence, if we had $p\mid d$, then we would have
$\left\vert p\right\vert \leq\left\vert d\right\vert $ (by Proposition
\ref{prop.ent.div.1} \textbf{(b)}, applied to $a=p$ and $b=d$); but this would
contradict $\left\vert p\right\vert >\left\vert d\right\vert $. Hence, we
cannot have $p\mid d$.

We have $c\neq0$ (since $c>0$). Hence, if we had $p\mid c$, then we would have
$\left\vert p\right\vert \leq\left\vert c\right\vert $ (by Proposition
\ref{prop.ent.div.1} \textbf{(b)}, applied to $a=p$ and $b=c$); but this would
contradict $\left\vert p\right\vert >\left\vert c\right\vert $. Hence, we
cannot have $p\mid c$.

Thus, we have neither $p\mid d$ nor $p\mid c$. This contradicts the fact that
$p\mid d$ or $p\mid c$.

Now, forget that we have fixed $d$. We thus have found a contradiction for
each positive divisor $d$ of $p$ other than $1$ and $p$. Thus, there exists no
positive divisor $d$ of $p$ other than $1$ and $p$. In other words, each
positive divisor of $p$ is either $1$ or $p$. Thus, the only positive divisors
of $p$ are $1$ and $p$ (since $1$ and $p$ are indeed positive divisors of
$p$). In other words, $p$ is prime (by the definition of \textquotedblleft
prime\textquotedblright). This solves Exercise \ref{exe.ent.primes.pab-conv}.
\end{proof}
\end{fineprint}

There is also a version of Theorem \ref{thm.ent.primes.pab} for products of
multiple integers:

\begin{proposition}
\label{prop.ent.primes.pabk}Let $p$ be a prime. Let $a_{1},a_{2},\ldots,a_{k}$
be integers such that $p\mid a_{1}a_{2}\cdots a_{k}$. Then, $p\mid a_{i}$ for
some $i\in\left\{  1,2,\ldots,k\right\}  $.
\end{proposition}

We could prove Proposition \ref{prop.ent.primes.pabk} by induction on $k$. But
here is a more direct argument:

\begin{fineprint}
\begin{proof}
[Proof of Proposition \ref{prop.ent.primes.pabk}.]Assume the contrary. Thus,
there exists no $i\in\left\{  1,2,\ldots,k\right\}  $ such that $p\mid a_{i}$.
In other words, for each $i\in\left\{  1,2,\ldots,k\right\}  $, we have%
\begin{equation}
\left(  \text{not }p\mid a_{i}\right)  . \label{pf.prop.ent.primes.pabk.1}%
\end{equation}


Now, let $i\in\left\{  1,2,\ldots,k\right\}  $. Then, we don't have $p\mid
a_{i}$ (by (\ref{pf.prop.ent.primes.pabk.1})). But Proposition
\ref{prop.ent.primes.div-or-coprime} (applied to $a=a_{i}$) shows that either
$p\mid a_{i}$ or $p\perp a_{i}$. Hence, we have $p\perp a_{i}$ (since we don't
have $p\mid a_{i}$). In other words, $a_{i}\perp p$ (by Proposition
\ref{prop.ent.coprime.perp-symm}).

Now, forget that we fixed $i$. We thus have proven that each $i\in\left\{
1,2,\ldots,k\right\}  $ satisfies $a_{i}\perp p$. Hence, Exercise
\ref{exe.ent.coprime.ab-to-ck} (applied to $c=p$) yields $a_{1}a_{2}\cdots
a_{k}\perp p$. In other words, $a_{1}a_{2}\cdots a_{k}$ is coprime to $p$. In
other words, $\gcd\left(  a_{1}a_{2}\cdots a_{k},p\right)  =1$. Hence,
Proposition \ref{prop.ent.gcd.props1} \textbf{(b)} yields $\gcd\left(
p,a_{1}a_{2}\cdots a_{k}\right)  =\gcd\left(  a_{1}a_{2}\cdots a_{k},p\right)
=1$.

But $p$ is prime; thus, $p>1$. Hence, $p$ is positive. Recall that $p\mid
a_{1}a_{2}\cdots a_{k}$; thus, Proposition \ref{prop.ent.gcd.props1}
\textbf{(i)} (applied to $a=p$ and $b=a_{1}a_{2}\cdots a_{k}$) yields
$\gcd\left(  p,a_{1}a_{2}\cdots a_{k}\right)  =\left\vert p\right\vert =p$
(since $p$ is positive). Comparing this with $\gcd\left(  p,a_{1}a_{2}\cdots
a_{k}\right)  =1$, we obtain $p=1$. This contradicts $p>1$. This contradiction
shows that our assumption was wrong. This proves Proposition
\ref{prop.ent.primes.pabk}.
\end{proof}
\end{fineprint}

\subsubsection{Prime factorization I}

The next simple proposition says that every integer $n>1$ is divisible by at
least one prime:

\begin{proposition}
\label{prop.ent.primes.ex-pri-div}Let $n>1$ be an integer. Then, there exists
at least one prime $p$ such that $p\mid n$.
\end{proposition}

\begin{proof}
[Proof of Proposition \ref{prop.ent.primes.ex-pri-div}.]Clearly, $n$ is a
divisor of $n$ such that $n>1$. Thus, there exists a divisor $q$ of $n$ such
that $q>1$ (namely, $q=n$). Let $d$ be the \textbf{smallest} such
divisor\footnote{This exists, because the set of possible candidates is
nonempty (by the previous sentence) and finite.}. Thus, $d$ is a divisor of
$n$ and satisfies $d>1$. The integer $d$ is positive (since $d>1>0$).

We claim that $d$ is a prime.

\begin{fineprint}
[\textit{Proof:} Let $e$ be any positive divisor of $d$. Assume (for the sake
of contradiction) that $e\notin\left\{  1,d\right\}  $. Thus, $e\neq1$ and
$e\neq d$. Now, $e$ is a divisor of $d$; thus, $e\mid d\mid n$ (since $d$ is a
divisor of $n$). In other words, $e$ is a divisor of $n$. Also, $e>1$ (because
$e$ is positive and $e\neq1$). Hence, $e$ is a divisor $q$ of $n$ such that
$q>1$.

But $d$ was defined as the \textbf{smallest} divisor $q$ of $n$ such that
$q>1$. Hence, any such divisor is $\geq d$. In other words, any divisor $q$ of
$n$ such that $q>1$ must satisfy $q\geq d$. Applying this to $q=e$, we
conclude that $e\geq d$ (since $e$ is a divisor $q$ of $n$ such that $q>1$).
Combined with $e\neq d$, this yields $e>d$.

But $e\mid d$ and $d\neq0$ (since $d>1>0$). Hence, $\left\vert e\right\vert
\leq\left\vert d\right\vert $ (by Exercise \ref{prop.ent.div.1} \textbf{(a)},
applied to $a=e$ and $b=d$). Since $e$ is positive, we have $e=\left\vert
e\right\vert \leq\left\vert d\right\vert =d$ (since $d$ is positive). This
contradicts $e>d$. This contradiction shows that our assumption (that
$e\notin\left\{  1,d\right\}  $) was false. Thus, we have proven that
$e\in\left\{  1,d\right\}  $. In other worde, $e$ is either $1$ or $d$.

Now, forget that we fixed $e$. We thus have proven that if $e$ is any positive
divisor of $d$, then $e\in\left\{  1,d\right\}  $. In other words, any
positive divisor of $d$ is either $1$ or $d$. Thus, the only positive divisors
of $d$ are $1$ and $d$ (since $1$ and $d$ clearly \textbf{are} positive
divisors of $d$). In other words, $d$ is prime (by the definition of
\textquotedblleft prime\textquotedblright).]
\end{fineprint}

So we know that $d\mid n$ (since $d$ is a divisor of $n$), and that $d$ is
prime. Hence, there exists at least one prime $p$ such that $p\mid n$ (namely,
$p=d$). This proves Proposition \ref{prop.ent.primes.ex-pri-div}.
\end{proof}

\begin{definition}
Let $n$ be an integer. A \textit{prime factor} of $n$ means a prime $p$ such
that $p\mid n$. Some say \textquotedblleft prime divisor\textquotedblright%
\ instead of \textquotedblleft prime factor\textquotedblright.
\end{definition}

Thus, Proposition \ref{prop.ent.primes.ex-pri-div} says that each integer
$n>1$ has at least one prime divisor.

\begin{proposition}
\label{prop.ent.primes.fac-ex}Let $n$ be a positive integer. Then, $n$ can be
written as a product of finitely many primes.
\end{proposition}

\begin{example}
\label{exa.ent.primes.fac-ex}\textbf{(a)} The integer $60$ can be written as a
product of four primes: namely, $60=2\cdot2\cdot3\cdot5$.

\textbf{(b)} The integer $1$ is the product of $0$ many primes (because a
product of $0$ many primes is the empty product, which is defined to be $1$).
\end{example}

\begin{proof}
[Proof of Proposition \ref{prop.ent.primes.fac-ex}.]We shall prove Proposition
\ref{prop.ent.primes.fac-ex} by strong induction on $n$. Thus, we fix a
positive integer $N$, and we assume (as the induction hypothesis) that
Proposition \ref{prop.ent.primes.fac-ex} holds whenever $n<N$. We must now
prove that Proposition \ref{prop.ent.primes.fac-ex} holds for $n=N$. In other
words, we must prove that $N$ can be written as a product of finitely many primes.

If $N=1$, then this is obvious (because $1$ is a product of $0$ many
primes\footnote{See Example \ref{exa.ent.primes.fac-ex} \textbf{(b)}.}). Thus,
for the rest of this proof, we WLOG assume that $N\neq1$. Hence, $N>1$ (since
$N$ is a positive integer). Therefore, Proposition
\ref{prop.ent.primes.ex-pri-div} (applied to $n=N$) shows that there exists at
least one prime $p$ such that $p\mid N$. Consider this $p$.

We have $p\mid N$. In other words, there exists an integer $c$ such that
$N=pc$. Consider this $c$. We have $p>1$ (since $p$ is prime); thus, $p$ is
positive. Hence, $p\neq0$. Thus, solving the equality $N=pc$ for $c$, we find
$c=N/\underbrace{p}_{>1}<N/1$ (since $N$ is positive), so that $c<N/1=N$. But
our induction hypothesis says that Proposition \ref{prop.ent.primes.fac-ex}
holds whenever $n<N$. Hence, we can apply Proposition
\ref{prop.ent.primes.fac-ex} to $n=c$ (since $c<N$). We thus conclude that $c$
can be written as a product of finitely many primes. In other words, there
exist primes $q_{1},q_{2},\ldots,q_{k}$ such that $c=q_{1}q_{2}\cdots q_{k}$.
Consider these $q_{1},q_{2},\ldots,q_{k}$.

But%
\[
N=p\underbrace{c}_{=q_{1}q_{2}\cdots q_{k}}=pq_{1}q_{2}\cdots q_{k}.
\]
Hence, $N$ can be written as a product of finitely many primes (namely, of the
primes $p,q_{1},q_{2},\ldots,q_{k}$). In other words, Proposition
\ref{prop.ent.primes.fac-ex} holds for $n=N$. This completes the induction
step. Hence, Proposition \ref{prop.ent.primes.fac-ex} is proven by strong induction.
\end{proof}

Proposition \ref{prop.ent.primes.fac-ex} shows that every positive integer $n$
can be represented as a product of finitely many primes. Such a representation
-- or, more precisely, the list of the primes it contains -- will be called
the \textit{prime factorization} of $n$. Rigorously speaking, this means that
we make the following definition:

\begin{definition}
Let $n$ be a positive integer. A \textit{prime factorization} of $n$ means a
tuple $\left(  p_{1},p_{2},\ldots,p_{k}\right)  $ of primes such that
$n=p_{1}p_{2}\cdots p_{k}$.
\end{definition}

\begin{example}
\textbf{(a)} The prime factorizations of $12$ are%
\[
\left(  2,2,3\right)  ,\ \ \ \ \ \ \ \ \ \ \left(  2,3,2\right)
,\ \ \ \ \ \ \ \ \ \ \left(  3,2,2\right)  .
\]
Indeed, these three $3$-tuples are prime factorizations of $12$ because
$12=2\cdot2\cdot3=2\cdot3\cdot2=3\cdot2\cdot2$. It is not hard to check that
they are the only prime factorizations of $12$.

\textbf{(b)} If $p$ is a prime, then the only prime factorization of $p$ is
the $1$-tuple $\left(  p\right)  $.

\textbf{(c)} If $p$ is a prime and $i\in\mathbb{N}$, then the only prime
factorization of $p^{i}$ is the $i$-tuple $\left(  \underbrace{p,p,\ldots
,p}_{i\text{ times}}\right)  $. This is not quite obvious at this point
(though it is not hard to derive from Proposition \ref{prop.ent.primes.pabk}).

\textbf{(d)} The only prime factorization of $1$ is the $0$-tuple $\left(
{}\right)  $.
\end{example}

This example suggests that all prime factorizations of a given positive
integer $n$ are equal to each other up to the order of their entries (i.e.,
are permutations of each other). This is indeed true, and we are going to
prove this soon (in Theorem \ref{thm.ent.primes.fac-uni} below).

\subsubsection{$p$-valuations}

\begin{lemma}
\label{lem.ent.prime.vp-wd}Let $p$ be a prime. Let $n$ be a nonzero integer.
Then, there exists a largest $m\in\mathbb{N}$ such that $p^{m}\mid n$.
\end{lemma}

\begin{proof}
[Proof of Lemma \ref{lem.ent.prime.vp-wd}.]We know that $p$ is a prime. Thus,
$p$ is an integer and $p>1$ (by the definition of a \textquotedblleft
prime\textquotedblright). This is all we shall need from our assumption that
$p$ is prime.

Let $W$ be the set of all $m\in\mathbb{N}$ satisfying $p^{m}\mid n$. Then, $W$
is a set of integers. Moreover, $0$ is an $m\in\mathbb{N}$ satisfying
$p^{m}\mid n$ (since $p^{0}=1\mid n$); in other words, $0\in W$ (by the
definition of $W$). Hence, the set $W$ is nonempty.

Let $u=\left\vert n\right\vert $. Thus, $u\in\mathbb{N}$.

It is easy to see that $p^{k}>k$ for each $k\in\mathbb{N}$%
\ \ \ \ \footnote{\textit{Proof.} This can easily be proven by induction on
$k$; but here is a more artful proof:
\par
Let $k\in\mathbb{N}$. We have $p\geq2$ (since $p$ is an integer and satisfies
$p>1$). Thus, $p-1\geq1$.
\par
Recall the identity (\ref{pf.lem.ent.xd-yd.1}), which holds for every
$a,b\in\mathbb{Q}$. Let us apply this identity to $a=p$ and $b=1$. We thus
obtain%
\[
\left(  p-1\right)  \left(  p^{k-1}+p^{k-2}\cdot1+p^{k-3}\cdot1^{2}%
+\cdots+p\cdot1^{k-2}+1^{k-1}\right)  =p^{k}-\underbrace{1^{k}}_{=1}=p^{k}-1.
\]
Thus,%
\begin{align*}
p^{k}-1  &  =\left(  p-1\right)  \underbrace{\left(  p^{k-1}+p^{k-2}%
\cdot1+p^{k-3}\cdot1^{2}+\cdots+p\cdot1^{k-2}+1^{k-1}\right)  }_{=\sum
_{i=0}^{k-1}p^{i}1^{k-i}}\\
&  =\underbrace{\left(  p-1\right)  }_{\geq1}\sum_{i=0}^{k-1}p^{i}1^{k-i}%
\geq1\sum_{i=0}^{k-1}p^{i}1^{k-i}\ \ \ \ \ \ \ \ \ \ \left(  \text{since }%
\sum_{i=0}^{k-1}p^{i}1^{k-i}\text{ is clearly }\geq0\right) \\
&  =\sum_{i=0}^{k-1}\underbrace{p^{i}1^{k-i}}_{\substack{=p^{i}\geq
1^{i}\\\text{(since }p\geq1\text{)}}}=\sum_{i=0}^{k-1}\underbrace{1^{i}}%
_{=1}=\sum_{i=0}^{k-1}1=k.
\end{align*}
Hence, $p^{k}\geq k+1>k$, qed.}. Thus, each $g\in W$ satisfies $g\in\left\{
0,1,\ldots,u-1\right\}  $\ \ \ \ \footnote{\textit{Proof.} Let $g\in W$. Thus,
$g$ is an $m\in\mathbb{N}$ satisfying $p^{m}\mid n$ (by the definition of
$W$). In other words, $g\in\mathbb{N}$ and $p^{g}\mid n$. Also, $n\neq0$
(since $n$ is nonzero). Hence, Proposition \ref{prop.ent.div.1} \textbf{(b)}
(applied to $a=p^{g}$ and $b=n$) yields $\left\vert p^{g}\right\vert
\leq\left\vert n\right\vert =u$. But $p$ is positive (since $p>1>0$); thus,
$p^{g}$ is positive. Hence, $\left\vert p^{g}\right\vert =p^{g}$. Thus,
$p^{g}=\left\vert p^{g}\right\vert \leq u$. But recall that $p^{k}>k$ for each
$k\in\mathbb{N}$. Applying this to $k=g$, we find $p^{g}>g$. Hence,
$g<p^{g}\leq u$, so that $g\in\left\{  0,1,\ldots,u-1\right\}  $ (since
$g\in\mathbb{N}$). Qed.}. In other words, $W\subseteq\left\{  0,1,\ldots
,u-1\right\}  $. Hence, the set $W$ is finite (since the set $\left\{
0,1,\ldots,u-1\right\}  $ is finite). Thus, $W$ is a finite nonempty set of
integers. Therefore, the set $W$ has a largest element. In view of how $W$ was
defined, this can be restated as follows: There exists a largest
$m\in\mathbb{N}$ such that $p^{m}\mid n$. This proves Lemma
\ref{lem.ent.prime.vp-wd}.
\end{proof}

\begin{definition}
\label{def.ent.prime.vp}Let $p$ be a prime.

\textbf{(a)} Let $n$ be a nonzero integer. Then, $v_{p}\left(  n\right)  $
shall denote the largest $m\in\mathbb{N}$ such that $p^{m}\mid n$. This is
well-defined (by Lemma \ref{lem.ent.prime.vp-wd}). This nonnegative integer
$v_{p}\left(  n\right)  $ will be called the $p$\textit{-valuation} (or the
$p$\textit{-adic valuation}) of $n$.

\textbf{(b)} We extend this definition of $v_{p}\left(  n\right)  $ to the
case of $n=0$ as follows: Set $v_{p}\left(  0\right)  =\infty$, where $\infty$
is a new symbol. This symbol $\infty$ is supposed to model \textquotedblleft
positive infinity\textquotedblright; in particular, we take it to satisfy the
following rules:

\begin{itemize}
\item We have $k+\infty=\infty+k=\infty$ for all integers $k$.

\item We have $\infty+\infty=\infty$.

\item Each integer $k$ satisfies $k<\infty$ and $\infty>k$ (and thus
$k\leq\infty$ and $\infty\geq k$).

\item No integer $k$ satisfies $k\geq\infty$ or $\infty\leq k$ (or $k>\infty$
or $\infty<k$).

\item If $S$ is a nonempty set of integers, then $\min\left(  S\cup\left\{
\infty\right\}  \right)  =\min S$.

\item If $S$ is any set of integers, then $\max\left(  S\cup\left\{
\infty\right\}  \right)  =\infty$.
\end{itemize}

(Note, however, that $\infty$ is not supposed to be a \textquotedblleft first
class citizen\textquotedblright\ of the number system. In particular,
$\infty-\infty$ is not defined. More generally, $k-\infty$ is never defined,
whatever $k$ is. Indeed, any definition of $k-\infty$ would break some of the
familiar rules of arithmetic. The only operations that we shall subject
$\infty$ to are addition, minimum and maximum.)
\end{definition}

Note that the rules for the symbol $\infty$ yield that%
\[
k+\infty=\infty+k=\max\left\{  k,\infty\right\}  =\infty
\]
and%
\[
\min\left\{  k,\infty\right\}  =k
\]
for each $k\in\mathbb{Z}\cup\left\{  \infty\right\}  $. It is not hard to see
that basic properties of inequalities (such as \textquotedblleft if $a\leq b$
and $b\leq c$, then $a\leq c$\textquotedblright) and of addition (such as
\textquotedblleft$\left(  a+b\right)  +c=a+\left(  b+c\right)  $%
\textquotedblright) and of the interplay between inequalities and addition
(such as \textquotedblleft if $a\leq b$, then $a+c\leq b+c$\textquotedblright)
are still valid in $\mathbb{Z}\cup\left\{  \infty\right\}  $ (that is, they
still hold if we plug $\infty$ for one or more of the variables). However, of
course, we cannot \textquotedblleft cancel\textquotedblright\ $\infty$ from
equalities (i.e., we cannot cancel $\infty$ from $a+\infty=b+\infty$ to obtain
$a=b$) or inequalities.

\begin{example}
\label{exa.ent.prime.vp}\textbf{(a)} We have $v_{5}\left(  50\right)  =2$.
Indeed, $2$ is the largest $m\in\mathbb{N}$ such that $5^{m}\mid50$ (because
$5^{2}=25\mid50$ but $5^{3}=125\nmid50$).

\textbf{(b)} We have $v_{5}\left(  51\right)  =0$. Indeed, $0$ is the largest
$m\in\mathbb{N}$ such that $5^{m}\mid51$ (because $5^{0}=1\mid51$ but
$5^{1}=5\nmid51$).

\textbf{(c)} We have $v_{5}\left(  55\right)  =1$. Indeed, $1$ is the largest
$m\in\mathbb{N}$ such that $5^{m}\mid55$ (because $5^{1}=5\mid55$ but
$5^{2}=25\nmid55$).

\textbf{(d)} We have $v_{5}\left(  0\right)  =\infty$ (by Definition
\ref{def.ent.prime.vp} \textbf{(b)}).
\end{example}

Definition \ref{def.ent.prime.vp} \textbf{(a)} can be restated in the
following more intuitive way: Given a prime $p$ and a nonzero integer $n$, we
let $v_{p}\left(  n\right)  $ be the number of times we can divide $n$ by $p$
without leaving $\mathbb{Z}$. Definition \ref{def.ent.prime.vp} \textbf{(b)}
is consistent with this picture, because we can clearly divide $0$ by $p$
infinitely often without leaving $\mathbb{Z}$. From this point of view, the
following lemma should be obvious:

\begin{lemma}
\label{lem.ent.prime.vp-def}Let $p$ be a prime. Let $i\in\mathbb{N}$. Let
$n\in\mathbb{Z}$. Then, $p^{i}\mid n$ if and only if $v_{p}\left(  n\right)
\geq i$.
\end{lemma}

\begin{proof}
[Proof of Lemma \ref{lem.ent.prime.vp-def}.]First, let us notice that
$p^{i}\mid0$. Also, Definition \ref{def.ent.prime.vp} \textbf{(b)} yields
$v_{p}\left(  0\right)  =\infty\geq i$ (according to our rules for the symbol
$\infty$). Hence, both statements $\left(  p^{i}\mid0\right)  $ and $\left(
v_{p}\left(  0\right)  \geq i\right)  $ hold. Thus, $p^{i}\mid0$ if and only
if $v_{p}\left(  0\right)  \geq i$. In other words, Lemma
\ref{lem.ent.prime.vp-def} holds if $n=0$. Thus, for the rest of this proof,
we WLOG assume that $n\neq0$. Hence, $n$ is nonzero. Thus, $v_{p}\left(
n\right)  $ is the largest $m\in\mathbb{N}$ such that $p^{m}\mid n$ (by
Definition \ref{def.ent.prime.vp} \textbf{(a)}). Hence, $v_{p}\left(
n\right)  $ itself is an $m\in\mathbb{N}$ such that $p^{m}\mid n$. In other
words, $v_{p}\left(  n\right)  \in\mathbb{N}$ and $p^{v_{p}\left(  n\right)
}\mid n$.

We must prove that $p^{i}\mid n$ if and only if $v_{p}\left(  n\right)  \geq
i$. Let us prove the \textquotedblleft$\Longrightarrow$\textquotedblright\ and
\textquotedblleft$\Longleftarrow$\textquotedblright\ directions of this
\textquotedblleft if and only if\textquotedblright\ statement separately:

$\Longrightarrow:$ Assume that $p^{i}\mid n$. We must prove that $v_{p}\left(
n\right)  \geq i$.

The integer $i$ is an $m\in\mathbb{N}$ such that $p^{m}\mid n$ (since
$p^{i}\mid n$). But $v_{p}\left(  n\right)  $ is the \textbf{largest} such $m$
(by Definition \ref{def.ent.prime.vp} \textbf{(a)}). Hence, $v_{p}\left(
n\right)  \geq i$. This proves the \textquotedblleft$\Longrightarrow
$\textquotedblright\ direction of Lemma \ref{lem.ent.prime.vp-def}.

$\Longleftarrow:$ Assume that $v_{p}\left(  n\right)  \geq i$. We must prove
that $p^{i}\mid n$.

We have $v_{p}\left(  n\right)  \geq i$, thus $i\leq v_{p}\left(  n\right)  $.
Hence, Exercise \ref{exe.ent.div.powers} (applied to $p$, $i$ and
$v_{p}\left(  n\right)  $ instead of $n$, $a$ and $b$) yields $p^{i}\mid
p^{v_{p}\left(  n\right)  }$. Thus, $p^{i}\mid p^{v_{p}\left(  n\right)  }\mid
n$.

Hence, we have proven $p^{i}\mid n$. This proves the \textquotedblleft%
$\Longleftarrow$\textquotedblright\ direction of Lemma
\ref{lem.ent.prime.vp-def}.
\end{proof}

\begin{corollary}
\label{cor.ent.prime.vp-0}Let $p$ be a prime. Let $n\in\mathbb{Z}$. Then,
$v_{p}\left(  n\right)  =0$ if and only if $p\nmid n$.
\end{corollary}

\begin{proof}
[Proof of Corollary \ref{cor.ent.prime.vp-0}.]$\Longrightarrow:$ Assume that
$v_{p}\left(  n\right)  =0$. We must prove that $p\nmid n$.

We don't have $v_{p}\left(  n\right)  \geq1$ (since $v_{p}\left(  n\right)
=0<1$). But Lemma \ref{lem.ent.prime.vp-def} (applied to $i=1$) shows that
$p^{1}\mid n$ if and only if $v_{p}\left(  n\right)  \geq1$. Hence, we don't
have $p^{1}\mid n$ (since we don't have $v_{p}\left(  n\right)  \geq1$). In
other words, we have $p^{1}\nmid n$. In other words, $p\nmid n$ (since
$p=p^{1}$). This proves the \textquotedblleft$\Longrightarrow$%
\textquotedblright\ direction of Corollary \ref{cor.ent.prime.vp-0}.

$\Longleftarrow:$ Assume that $p\nmid n$. We must prove that $v_{p}\left(
n\right)  =0$.

We don't have $p\mid n$ (since $p\nmid n$). In other words, we don't have
$p^{1}\mid n$ (since $p^{1}=p$). But Lemma \ref{lem.ent.prime.vp-def} (applied
to $i=1$) shows that $p^{1}\mid n$ if and only if $v_{p}\left(  n\right)
\geq1$. Hence, we don't have $v_{p}\left(  n\right)  \geq1$ (since we don't
have $p^{1}\mid n$). In other words, $v_{p}\left(  n\right)  <1$.

If we had $n=0$, then we would have $p\mid0=n$, which would contradict $p\nmid
n$. Hence, we don't have $n=0$. Thus, $p$ is nonzero. Hence, Definition
\ref{def.ent.prime.vp} \textbf{(a)} shows that $v_{p}\left(  n\right)
\in\mathbb{N}$. In light of this, we can conclude $v_{p}\left(  n\right)  =0$
from $v_{p}\left(  n\right)  <1$. This proves the \textquotedblleft%
$\Longleftarrow$\textquotedblright\ direction of Corollary
\ref{cor.ent.prime.vp-0}.
\end{proof}

Here is another property of $p$-valuations that is useful in their study:

\begin{lemma}
\label{lem.ent.prime.vp-copr}Let $p$ be a prime. Let $n\in\mathbb{Z}$ be
nonzero. Then:

\textbf{(a)} There exists a nonzero integer $u$ such that $u\perp p$ and
$n=up^{v_{p}\left(  n\right)  }$.

\textbf{(b)} If $i\in\mathbb{N}$ and $w\in\mathbb{Z}$ are such that $w\perp p$
and $n=wp^{i}$, then $v_{p}\left(  n\right)  =i$.
\end{lemma}

Before we prove this formally, let us show the idea behind this lemma. Recall
that, given a prime $p$ and a nonzero integer $n$, the number $v_{p}\left(
n\right)  $ counts how often we can divide $n$ by $p$ without leaving
$\mathbb{Z}$. What happens after we have divided $n$ by $p$ this many times?
We get a number $u$ that is still an integer, but is no longer divisible by
$p$, and thus must be coprime to $p$ (by Proposition
\ref{prop.ent.primes.div-or-coprime}). This is what Lemma
\ref{lem.ent.prime.vp-copr} \textbf{(a)} says. Lemma
\ref{lem.ent.prime.vp-copr} \textbf{(b)} is a converse statement: It says that
if we divide $n$ by $p$ some number of times (say, $i$ times) and obtain an
integer coprime to $p$, then $i$ must be $v_{p}\left(  n\right)  $.

\begin{proof}
[Proof of Lemma \ref{lem.ent.prime.vp-copr}.]Definition \ref{def.ent.prime.vp}
\textbf{(a)} shows that $v_{p}\left(  n\right)  $ is the largest
$m\in\mathbb{N}$ such that $p^{m}\mid n$. Hence, $v_{p}\left(  n\right)  $
itself is an $m\in\mathbb{N}$ such that $p^{m}\mid n$. In other words,
$v_{p}\left(  n\right)  \in\mathbb{N}$ and $p^{v_{p}\left(  n\right)  }\mid n$.

Thus, in particular, $p^{v_{p}\left(  n\right)  }\mid n$. In other words,
there exists an integer $c$ such that $n=p^{v_{p}\left(  n\right)  }c$.
Consider this $c$. We have $n=p^{v_{p}\left(  n\right)  }c=cp^{v_{p}\left(
n\right)  }$.

Assume (for the sake of contradiction) that $p\mid c$. Thus, there exists an
integer $d$ such that $c=pd$. Consider this $d$. Now,%
\[
n=p^{v_{p}\left(  n\right)  }\underbrace{c}_{=pd}=\underbrace{p^{v_{p}\left(
n\right)  }p}_{=p^{v_{p}\left(  n\right)  +1}}d=p^{v_{p}\left(  n\right)
+1}d.
\]
Hence, $p^{v_{p}\left(  n\right)  +1}\mid n$ (since $d$ is an integer). In
other words, $v_{p}\left(  n\right)  +1$ is an $m\in\mathbb{N}$ such that
$p^{m}\mid n$. But we know that $v_{p}\left(  n\right)  $ is the
\textbf{largest} such $m$ (by Definition \ref{def.ent.prime.vp} \textbf{(a)}).
Hence, we conclude that $v_{p}\left(  n\right)  \geq v_{p}\left(  n\right)
+1$. But this is clearly absurd. This contradiction shows that our assumption
(that $p\mid c$) was wrong. Hence, we do not have $p\mid c$.

But Proposition \ref{prop.ent.primes.div-or-coprime} (applied to $a=c$) shows
that either $p\mid c$ or $p\perp c$. Hence, $p\perp c$ (since we do not have
$p\mid c$). In other words, $c\perp p$ (because of Proposition
\ref{prop.ent.coprime.perp-symm}).

If we had $c=0$, then we would have $n=p^{v_{p}\left(  n\right)
}\underbrace{c}_{=0}=0$, which would contradict the fact that $n$ is nonzero.
Hence, we cannot have $c=0$. Thus, $c$ is nonzero.

Now, we know that $c$ is a nonzero integer satisfying $c\perp p$ and
$n=cp^{v_{p}\left(  n\right)  }$. Hence, there exists a nonzero integer $u$
such that $u\perp p$ and $n=up^{v_{p}\left(  n\right)  }$ (namely, $u=c$).
This proves Lemma \ref{lem.ent.prime.vp-copr} \textbf{(a)}.

\textbf{(b)} Let $i\in\mathbb{N}$ and $w\in\mathbb{Z}$ be such that $w\perp p$
and $n=wp^{i}$. We must prove that $v_{p}\left(  n\right)  =i$.

From $w\perp p$, we obtain $p\perp w$ (by Proposition
\ref{prop.ent.coprime.perp-symm}). In other words, $\gcd\left(  p,w\right)
=1$.

We have $n=wp^{i}=p^{i}w$ and thus $p^{i}\mid n$ (since $w$ is an integer).
But Lemma \ref{lem.ent.prime.vp-def} yields that $p^{i}\mid n$ if and only if
$v_{p}\left(  n\right)  \geq i$. Hence, we have $v_{p}\left(  n\right)  \geq
i$ (since we have $p^{i}\mid n$).

Now, we shall prove that $v_{p}\left(  n\right)  \leq i$. Indeed, assume the
contrary. Thus, $v_{p}\left(  n\right)  >i$, so that $v_{p}\left(  n\right)
\geq i+1$ (since $v_{p}\left(  n\right)  $ and $i$ are integers). But Lemma
\ref{lem.ent.prime.vp-def} (applied to $i+1$ instead of $i$) shows that
$p^{i+1}\mid n$ if and only if $v_{p}\left(  n\right)  \geq i+1$. Thus, we
have $p^{i+1}\mid n$ (since we have $v_{p}\left(  n\right)  \geq i+1$). In
other words, $pp^{i}\mid wp^{i}$ (since $p^{i+1}=pp^{i}$ and $n=wp^{i}$). But
$p$ is a prime; thus, $p>1>0$ and therefore $p\neq0$. Hence, $p^{i}\neq0$.
Thus, Exercise \ref{exe.ent.div.acbc} (applied to $p$, $w$ and $p^{i}$ instead
of $a$, $b$ and $c$) shows that $p\mid w$ holds if and only if $pp^{i}\mid
wp^{i}$. Hence, $p\mid w$ holds (since $pp^{i}\mid wp^{i}$ holds). Thus,
Proposition \ref{prop.ent.gcd.props1} \textbf{(i)} (applied to $p$ and $w$
instead of $a$ and $b$) yields $\gcd\left(  p,w\right)  =\left\vert
p\right\vert =p$ (since $p>0$). Comparing this with $\gcd\left(  p,w\right)
=1$, we find $p=1$. This contradicts $p>1$.

This contradiction shows that our assumption was false. Hence, $v_{p}\left(
n\right)  \leq i$ is proven. Combining this with $v_{p}\left(  n\right)  \geq
i$, we obtain $v_{p}\left(  n\right)  =i$. This proves Lemma
\ref{lem.ent.prime.vp-copr} \textbf{(b)}.
\end{proof}

The next property of $p$-adic valuations is crucial, as it reveals how they
can be computed and bounded:

\begin{theorem}
\label{thm.ent.prime.vp-ring}Let $p$ be a prime.

\textbf{(a)} We have $v_{p}\left(  ab\right)  =v_{p}\left(  a\right)
+v_{p}\left(  b\right)  $ for any two integers $a$ and $b$.

\textbf{(b)} We have $v_{p}\left(  a+b\right)  \geq\min\left\{  v_{p}\left(
a\right)  ,v_{p}\left(  b\right)  \right\}  $ for any two integers $a$ and $b$.

\textbf{(c)} We have $v_{p}\left(  1\right)  =0$.

\textbf{(d)} We have $v_{p}\left(  q\right)  =%
\begin{cases}
1, & \text{if }q=p;\\
0, & \text{if }q\neq p
\end{cases}
$ for any prime $q$.
\end{theorem}

Note that Theorem \ref{thm.ent.prime.vp-ring} \textbf{(a)} gives a formula for
$v_{p}\left(  ab\right)  $ in terms of $v_{p}\left(  a\right)  $ and
$v_{p}\left(  b\right)  $, but there is no such formula for $v_{p}\left(
a+b\right)  $ (since $v_{p}\left(  a\right)  $ and $v_{p}\left(  b\right)  $
do not uniquely determine $v_{p}\left(  a+b\right)  $). Thus, Theorem
\ref{thm.ent.prime.vp-ring} \textbf{(b)} only gives a bound.

\begin{proof}
[Proof of Theorem \ref{thm.ent.prime.vp-ring}.]\textbf{(a)} Let $a$ and $b$ be
two integers. We must prove that $v_{p}\left(  ab\right)  =v_{p}\left(
a\right)  +v_{p}\left(  b\right)  $.

If $a=0$, then this is true\footnote{\textit{Proof.} Assume that $a=0$. Then,
$\underbrace{a}_{=0}b=0$ and thus $v_{p}\left(  ab\right)  =v_{p}\left(
0\right)  =\infty$ (by Definition \ref{def.ent.prime.vp} \textbf{(b)}). Also,
from $a=0$, we obtain $v_{p}\left(  a\right)  =v_{p}\left(  0\right)  =\infty
$. Hence, $\underbrace{v_{p}\left(  a\right)  }_{=\infty}+v_{p}\left(
b\right)  =\infty+v_{p}\left(  b\right)  =\infty$ (since $\infty+k=\infty$ for
each $k\in\mathbb{Z}\cup\left\{  \infty\right\}  $). Comparing this with
$v_{p}\left(  ab\right)  =\infty$, we obtain $v_{p}\left(  ab\right)
=v_{p}\left(  a\right)  +v_{p}\left(  b\right)  $. This is exactly what we
wanted to prove.}. Thus, for the rest of the proof of Theorem
\ref{thm.ent.prime.vp-ring} \textbf{(a)}, we WLOG assume that $a\neq0$. For
similar reasons, we WLOG assume that $b\neq0$.

The integer $a$ is nonzero (since $a\neq0$). Thus, Lemma
\ref{lem.ent.prime.vp-copr} \textbf{(b)} (applied to $n=a$) shows that there
exists a nonzero integer $u$ such that $u\perp p$ and $a=up^{v_{p}\left(
a\right)  }$. Consider this $u$, and denote it by $x$. Thus, $x$ is a nonzero
integer such that $x\perp p$ and $a=xp^{v_{p}\left(  a\right)  }$.

The integer $b$ is nonzero (since $b\neq0$). Thus, Lemma
\ref{lem.ent.prime.vp-copr} \textbf{(a)} (applied to $n=b$) shows that there
exists a nonzero integer $u$ such that $u\perp p$ and $b=up^{v_{p}\left(
b\right)  }$. Consider this $u$, and denote it by $y$. Thus, $y$ is a nonzero
integer such that $y\perp p$ and $b=yp^{v_{p}\left(  b\right)  }$.

We have $x\perp p$ and $y\perp p$. Thus, Theorem \ref{thm.ent.coprime.ab-to-c}
(applied to $x$, $y$ and $p$ instead of $a$, $b$ and $c$) shows that $xy\perp
p$.

The integer $ab$ is nonzero (since $a\neq0$ and $b\neq0$).

Furthermore, multiplying the equalities $a=xp^{v_{p}\left(  a\right)  }$ and
$b=yp^{v_{p}\left(  b\right)  }$, we obtain%
\[
ab=\left(  xp^{v_{p}\left(  a\right)  }\right)  \left(  yp^{v_{p}\left(
b\right)  }\right)  =\left(  xy\right)  \underbrace{\left(  p^{v_{p}\left(
a\right)  }p^{v_{p}\left(  b\right)  }\right)  }_{=p^{v_{p}\left(  a\right)
+v_{p}\left(  b\right)  }}=\left(  xy\right)  p^{v_{p}\left(  a\right)
+v_{p}\left(  b\right)  }.
\]
Thus, Lemma \ref{lem.ent.prime.vp-copr} \textbf{(b)} (applied to $n=ab$,
$i=v_{p}\left(  a\right)  +v_{p}\left(  b\right)  $ and $w=xy$) shows that
$v_{p}\left(  ab\right)  =v_{p}\left(  a\right)  +v_{p}\left(  b\right)  $
(since $v_{p}\left(  a\right)  +v_{p}\left(  b\right)  \in\mathbb{N}$ and
$xy\in\mathbb{Z}$ and $xy\perp p$). This proves Theorem
\ref{thm.ent.prime.vp-ring} \textbf{(a)}.

\textbf{(b)} Let $a$ and $b$ be two integers. We must prove that $v_{p}\left(
a+b\right)  \geq\min\left\{  v_{p}\left(  a\right)  ,v_{p}\left(  b\right)
\right\}  $.

If $a=0$, then this is true\footnote{\textit{Proof.} Assume that $a=0$. Then,
$v_{p}\left(  \underbrace{a}_{=0}+b\right)  =v_{p}\left(  b\right)  \geq
\min\left\{  v_{p}\left(  a\right)  ,v_{p}\left(  b\right)  \right\}  $ (since
any element of a set is $\geq$ to the minimum of this set). This is exactly
what we wanted to prove.}. Thus, for the rest of the proof of Theorem
\ref{thm.ent.prime.vp-ring} \textbf{(b)}, we WLOG assume that $a\neq0$. For
similar reasons, we WLOG assume that $b\neq0$.

The integer $a$ is nonzero (since $a\neq0$). Thus, $v_{p}\left(  a\right)
\in\mathbb{N}$ (by Definition \ref{def.ent.prime.vp} \textbf{(a)}). Similarly,
$v_{p}\left(  b\right)  \in\mathbb{N}$.

Let $m=\min\left\{  v_{p}\left(  a\right)  ,v_{p}\left(  b\right)  \right\}
$. Thus, $m\in\mathbb{N}$ (since $v_{p}\left(  a\right)  \in\mathbb{N}$ and
$v_{p}\left(  b\right)  \in\mathbb{N}$).

We have $m=\min\left\{  v_{p}\left(  a\right)  ,v_{p}\left(  b\right)
\right\}  \leq v_{p}\left(  a\right)  $; in other words, $v_{p}\left(
a\right)  \geq m$. But Lemma \ref{lem.ent.prime.vp-def} (applied to $n=a$ and
$i=m$) shows that $p^{m}\mid a$ if and only if $v_{p}\left(  a\right)  \geq
m$. Hence, we have $p^{m}\mid a$ (since $v_{p}\left(  a\right)  \geq m$). In
other words, $a\equiv0\operatorname{mod}p^{m}$. Similarly, $b\equiv
0\operatorname{mod}p^{m}$. Adding these two congruences together, we obtain
$a+b\equiv0+0=0\operatorname{mod}p^{m}$. In other words, $p^{m}\mid a+b$.

But Lemma \ref{lem.ent.prime.vp-def} (applied to $n=a+b$ and $i=m$) shows that
$p^{m}\mid a+b$ if and only if $v_{p}\left(  a+b\right)  \geq m$. Hence, we
have $v_{p}\left(  a+b\right)  \geq m$ (since $p^{m}\mid a+b$). Thus,
$v_{p}\left(  a+b\right)  \geq m=\min\left\{  v_{p}\left(  a\right)
,v_{p}\left(  b\right)  \right\}  $. This proves Theorem
\ref{thm.ent.prime.vp-ring} \textbf{(b)}.

\textbf{(c)} We have $1\mid p$. Thus, Proposition \ref{prop.ent.gcd.props1}
\textbf{(i)} (applied to $1$ and $p$ instead of $a$ and $b$) yields
$\gcd\left(  1,p\right)  =\left\vert 1\right\vert =1$. In other words, $1\perp
p$. Also, $1=1\cdot p^{0}$. Thus, Lemma \ref{lem.ent.prime.vp-copr}
\textbf{(b)} (applied to $n=1$, $i=0$ and $w=1$) yields $v_{p}\left(
1\right)  =0$. This proves Theorem \ref{thm.ent.prime.vp-ring} \textbf{(c)}.

\textbf{(d)} Let $q$ be a prime. We must prove that $v_{p}\left(  q\right)  =%
\begin{cases}
1, & \text{if }q=p;\\
0, & \text{if }q\neq p
\end{cases}
$.

We are in one of the following two cases:

\textit{Case 1:} We have $q=p$.

\textit{Case 2:} We have $q\neq p$.

Let us first consider Case 1. In this case, we have $q=p$. But $1\perp p$ (as
we just saw when proving Theorem \ref{thm.ent.prime.vp-ring} \textbf{(c)}).
Also, $p=1\cdot p^{1}$. Thus, Lemma \ref{lem.ent.prime.vp-copr} \textbf{(b)}
(applied to $n=p$, $i=1$ and $w=1$) yields $v_{p}\left(  p\right)  =1$. From
$q=p$, we obtain $v_{p}\left(  q\right)  =v_{p}\left(  p\right)  =1$.
Comparing this with
\[%
\begin{cases}
1, & \text{if }q=p;\\
0, & \text{if }q\neq p
\end{cases}
=1\ \ \ \ \ \ \ \ \ \ \left(  \text{since }q=p\right)  ,
\]
we obtain $v_{p}\left(  q\right)  =%
\begin{cases}
1, & \text{if }q=p;\\
0, & \text{if }q\neq p
\end{cases}
$. Hence, Theorem \ref{thm.ent.prime.vp-ring} \textbf{(d)} is proven in Case 1.

Let us now consider Case 2. In this case, we have $q\neq p$. Hence, $q\nmid
p$\ \ \ \ \footnote{\textit{Proof.} Assume the contrary. Thus, $q\mid p$. In
other words, $q$ is a divisor of $p$.
\par
But $p$ is a prime. According to the definition of a prime, this means that
$p>1$ and that the only positive divisors of $p$ are $1$ and $p$.
\par
Also, $q$ is a prime; thus, $q>1$ (by the definition of a prime); hence,
$q>1>0$. Thus, $q$ is a positive divisor of $p$. Hence, $q$ must be either $1$
or $p$ (since the only positive divisors of $p$ are $1$ and $p$). Since $q$
cannot be $1$ (because $q>1$), we thus conclude that $q$ must be $p$. In other
words, $q=p$. This contradicts $q\neq p$. This contradiction shows that our
assumption was false, qed.}.

But Proposition \ref{prop.ent.primes.div-or-coprime} (applied to $q$ and $p$
instead of $p$ and $a$) shows that either $q\mid p$ or $q\perp p$. Since
$q\mid p$ cannot hold (because we have $q\nmid p$), we thus conclude that
$q\perp p$. Also, $q=q\cdot p^{0}$ (since $p^{0}=1$). Thus, Lemma
\ref{lem.ent.prime.vp-copr} \textbf{(b)} (applied to $n=q$, $i=0$ and $w=q$)
yields $v_{p}\left(  q\right)  =0$. Comparing this with
\[%
\begin{cases}
1, & \text{if }q=p;\\
0, & \text{if }q\neq p
\end{cases}
=0\ \ \ \ \ \ \ \ \ \ \left(  \text{since }q\neq p\right)  ,
\]
we obtain $v_{p}\left(  q\right)  =%
\begin{cases}
1, & \text{if }q=p;\\
0, & \text{if }q\neq p
\end{cases}
$. Hence, Theorem \ref{thm.ent.prime.vp-ring} \textbf{(d)} is proven in Case 2.

We have now proven Theorem \ref{thm.ent.prime.vp-ring} \textbf{(d)} in each of
the two Cases 1 and 2. Thus, Theorem \ref{thm.ent.prime.vp-ring} \textbf{(d)}
is always proven.
\end{proof}

\begin{corollary}
\label{cor.ent.prime.vp-ringk}Let $p$ be a prime. Let $a_{1},a_{2}%
,\ldots,a_{k}$ be $k$ integers. Then, $v_{p}\left(  a_{1}a_{2}\cdots
a_{k}\right)  =v_{p}\left(  a_{1}\right)  +v_{p}\left(  a_{2}\right)
+\cdots+v_{p}\left(  a_{k}\right)  $.
\end{corollary}

\begin{proof}
[Proof of Corollary \ref{cor.ent.prime.vp-ringk}.]This follows
straightforwardly by induction on $k$, using Theorem
\ref{thm.ent.prime.vp-ring} \textbf{(a)} (as well as Theorem
\ref{thm.ent.prime.vp-ring} \textbf{(c)} for the induction base). We leave the
details to the reader, who has seen this sort of proof several times already.
\end{proof}

\begin{exercise}
\label{exe.ent.prime.vp-abs}Let $p$ be a prime. Let $n\in\mathbb{Z}$. Then,
$v_{p}\left(  \left\vert n\right\vert \right)  =v_{p}\left(  n\right)  $.
\end{exercise}

\begin{fineprint}
\begin{proof}
[Solution to Exercise \ref{exe.ent.prime.vp-abs}.]If $n\geq0$, then we have
$\left\vert n\right\vert =n$ and thus $v_{p}\left(  \left\vert n\right\vert
\right)  =v_{p}\left(  n\right)  $. Hence, if $n\geq0$, then Exercise
\ref{exe.ent.prime.vp-abs} holds. Thus, for the rest of this solution, we WLOG
assume that $n<0$. Hence, $\left\vert n\right\vert =-n$.

We have $-1\mid p$ (since $p=\left(  -1\right)  \cdot\left(  -p\right)  $).
Thus, Proposition \ref{prop.ent.gcd.props1} \textbf{(i)} (applied to $-1$ and
$p$ instead of $a$ and $b$) yields $\gcd\left(  -1,p\right)  =\left\vert
-1\right\vert =1$. In other words, $-1\perp p$. Also, $-1=\left(  -1\right)
\cdot p^{0}$. Thus, Lemma \ref{lem.ent.prime.vp-copr} \textbf{(b)} (applied to
$-1$, $0$ and $1$ instead of $n$, $i$ and $w$) yields $v_{p}\left(  -1\right)
=0$. Now, Theorem \ref{thm.ent.prime.vp-ring} \textbf{(a)} (applied to $a=-1$
and $b=n$) yields $v_{p}\left(  \left(  -1\right)  n\right)
=\underbrace{v_{p}\left(  -1\right)  }_{=0}+v_{p}\left(  n\right)  $. In view
of $\left(  -1\right)  n=-n=\left\vert n\right\vert $, this rewrites as
$v_{p}\left(  \left\vert n\right\vert \right)  =v_{p}\left(  n\right)  $. This
solves Exercise \ref{exe.ent.prime.vp-abs}.
\end{proof}
\end{fineprint}

\subsubsection{Prime factorization II}

\begin{proposition}
\label{prop.ent.prime.mult-in-pf}Let $n$ be a positive integer. Let $\left(
a_{1},a_{2},\ldots,a_{k}\right)  $ be a prime factorization of $n$. Let $p$ be
a prime. Then,%
\begin{align*}
&  \left(  \text{the number of times }p\text{ appears in the tuple }\left(
a_{1},a_{2},\ldots,a_{k}\right)  \right) \\
&  =\left(  \text{the number of }i\in\left\{  1,2,\ldots,k\right\}  \text{
such that }a_{i}=p\right) \\
&  =v_{p}\left(  n\right)  .
\end{align*}

\end{proposition}

\begin{proof}
[Proof of Proposition \ref{prop.ent.prime.mult-in-pf}.]We have assumed that
$\left(  a_{1},a_{2},\ldots,a_{k}\right)  $ is a prime factorization of $n$.
Thus, $a_{1},a_{2},\ldots,a_{k}$ are primes satisfying $n=a_{1}a_{2}\cdots
a_{k}$. Hence, for each $i\in\left\{  1,2,\ldots,k\right\}  $, the integer
$a_{i}$ is prime and thus satisfies%
\begin{equation}
v_{p}\left(  a_{i}\right)  =%
\begin{cases}
1, & \text{if }a_{i}=p;\\
0, & \text{if }a_{i}\neq p
\end{cases}
\label{pf.prop.ent.prime.mult-in-pf.1}%
\end{equation}
(by Theorem \ref{thm.ent.prime.vp-ring} \textbf{(d)}, applied to $q=a_{i}$).

From $n=a_{1}a_{2}\cdots a_{k}$, we obtain%
\begin{align*}
v_{p}\left(  n\right)   &  =v_{p}\left(  a_{1}a_{2}\cdots a_{k}\right) \\
&  =v_{p}\left(  a_{1}\right)  +v_{p}\left(  a_{2}\right)  +\cdots
+v_{p}\left(  a_{k}\right)  \ \ \ \ \ \ \ \ \ \ \left(  \text{by Corollary
\ref{cor.ent.prime.vp-ringk}}\right) \\
&  =\underbrace{\sum_{i=1}^{k}}_{=\sum_{i\in\left\{  1,2,\ldots,k\right\}  }%
}\underbrace{v_{p}\left(  a_{i}\right)  }_{\substack{=%
\begin{cases}
1, & \text{if }a_{i}=p;\\
0, & \text{if }a_{i}\neq p
\end{cases}
\\\text{(by (\ref{pf.prop.ent.prime.mult-in-pf.1}))}}}=\sum_{i\in\left\{
1,2,\ldots,k\right\}  }%
\begin{cases}
1, & \text{if }a_{i}=p;\\
0, & \text{if }a_{i}\neq p
\end{cases}
\\
&  =\sum_{\substack{i\in\left\{  1,2,\ldots,k\right\}  ;\\a_{i}=p}%
}\underbrace{%
\begin{cases}
1, & \text{if }a_{i}=p;\\
0, & \text{if }a_{i}\neq p
\end{cases}
}_{\substack{=1\\\text{(since }a_{i}=p\text{)}}}+\sum_{\substack{i\in\left\{
1,2,\ldots,k\right\}  ;\\a_{i}\neq p}}\underbrace{%
\begin{cases}
1, & \text{if }a_{i}=p;\\
0, & \text{if }a_{i}\neq p
\end{cases}
}_{\substack{=0\\\text{(since }a_{i}\neq p\text{)}}}\\
&  \ \ \ \ \ \ \ \ \ \ \left(
\begin{array}
[c]{c}%
\text{since each }i\in\left\{  1,2,\ldots,k\right\}  \text{ satisfies either
}a_{i}=p\text{ or }a_{i}\neq p\\
\text{(but not both)}%
\end{array}
\right) \\
&  =\sum_{\substack{i\in\left\{  1,2,\ldots,k\right\}  ;\\a_{i}=p}%
}1+\underbrace{\sum_{\substack{i\in\left\{  1,2,\ldots,k\right\}  ;\\a_{i}\neq
p}}0}_{=0}=\sum_{\substack{i\in\left\{  1,2,\ldots,k\right\}  ;\\a_{i}=p}}1\\
&  =\left(  \text{the number of }i\in\left\{  1,2,\ldots,k\right\}  \text{
such that }a_{i}=p\right)  \cdot1\\
&  =\left(  \text{the number of }i\in\left\{  1,2,\ldots,k\right\}  \text{
such that }a_{i}=p\right) \\
&  =\left(  \text{the number of times }p\text{ appears in }\left(  a_{1}%
,a_{2},\ldots,a_{k}\right)  \right)  .
\end{align*}
This proves Proposition \ref{prop.ent.prime.mult-in-pf}.
\end{proof}

The next lemma that we shall use is a basic fact from elementary combinatorics:

\begin{lemma}
\label{lem.comb.tuples.mult=perm}Let $P$ be a set. Let $\left(  a_{1}%
,a_{2},\ldots,a_{k}\right)  $ and $\left(  b_{1},b_{2},\ldots,b_{\ell}\right)
$ be two tuples of elements of $P$. Assume that for each $p\in P$, we have%
\begin{align}
&  \left(  \text{the number of times }p\text{ appears in }\left(  a_{1}%
,a_{2},\ldots,a_{k}\right)  \right) \nonumber\\
&  =\left(  \text{the number of times }p\text{ appears in }\left(  b_{1}%
,b_{2},\ldots,b_{\ell}\right)  \right)  .
\label{eq.lem.comb.tuples.mult=perm.ass}%
\end{align}
Then, the two tuples $\left(  a_{1},a_{2},\ldots,a_{k}\right)  $ and $\left(
b_{1},b_{2},\ldots,b_{\ell}\right)  $ differ only in the order of their
entries (i.e., are permutations of each other). (In other words, we have
$k=\ell$, and there exists a permutation $\sigma$ of $\left\{  1,2,\ldots
,\ell\right\}  $ such that $\left(  a_{1},a_{2},\ldots,a_{k}\right)  =\left(
b_{\sigma\left(  1\right)  },b_{\sigma\left(  2\right)  },\ldots
,b_{\sigma\left(  \ell\right)  }\right)  $.)
\end{lemma}

Lemma \ref{lem.comb.tuples.mult=perm} is an intuitively obvious fact: It says
that if two tuples (of any objects -- e.g., numbers) have the property that
any object occurs as often in the first tuple as it does in the second tuple,
then the two tuples differ only in the order of their entries. From the formal
point of view, though, it is a statement that needs proof. Let us merely
sketch how such a proof can be obtained, without going into the details:

\begin{fineprint}
\begin{proof}
[Proof of Lemma \ref{lem.comb.tuples.mult=perm} (sketched).]We can WLOG assume
that the set $P$ is finite (since otherwise, we can replace $P$ by the finite
subset $\left\{  a_{1},a_{2},\ldots,a_{k},b_{1},b_{2},\ldots,b_{\ell}\right\}
$, without breaking the assumption that $\left(  a_{1},a_{2},\ldots
,a_{k}\right)  $ and $\left(  b_{1},b_{2},\ldots,b_{\ell}\right)  $ are two
tuples of elements of $P$). Assume this (at least if you don't want to use the
Axiom of Choice\footnote{I don't.}).

For each $p\in P$, define two sets%
\begin{align*}
A_{p}  &  =\left\{  i\in\left\{  1,2,\ldots,k\right\}  \ \mid\ a_{i}%
=p\right\}  ;\\
B_{p}  &  =\left\{  j\in\left\{  1,2,\ldots,\ell\right\}  \ \mid
\ b_{j}=p\right\}  .
\end{align*}
The equation (\ref{eq.lem.comb.tuples.mult=perm.ass}) then says that
$\left\vert A_{p}\right\vert =\left\vert B_{p}\right\vert $ for each $p\in P$.
Hence, for each $p\in P$, there exists a bijection $\phi_{p}:A_{p}\rightarrow
B_{p}$ (because if two sets have the same size, then there exists a bijection
between them). Pick such a bijection $\phi_{p}$ for each $p\in P$. (This does
not require the Axiom of Choice, since $P$ is finite.)

Now, define a map $\sigma:\left\{  1,2,\ldots,k\right\}  \rightarrow\left\{
1,2,\ldots,\ell\right\}  $ as follows: For each $i\in\left\{  1,2,\ldots
,k\right\}  $, set $\sigma\left(  i\right)  =\phi_{p}\left(  i\right)  $,
where $p=a_{i}$. It is not hard to see that this map $\sigma$ is a bijection.
(Its inverse map sends each $j\in\left\{  1,2,\ldots,\ell\right\}  $ to
$\phi_{p}^{-1}\left(  j\right)  $, where $p=b_{j}$.) Thus, we have found a
bijection from $\left\{  1,2,\ldots,k\right\}  $ to $\left\{  1,2,\ldots
,\ell\right\}  $. This shows that the sets $\left\{  1,2,\ldots,k\right\}  $
and $\left\{  1,2,\ldots,\ell\right\}  $ have the same size; in other words,
$k=\ell$. Thus, the bijection $\sigma$ is actually a bijection from $\left\{
1,2,\ldots,\ell\right\}  $ to $\left\{  1,2,\ldots,\ell\right\}  $. In other
words, $\sigma$ is a permutation of $\left\{  1,2,\ldots,\ell\right\}  $.
Finally, it is easy to see that $\left(  a_{1},a_{2},\ldots,a_{k}\right)
=\left(  b_{\sigma\left(  1\right)  },b_{\sigma\left(  2\right)  }%
,\ldots,b_{\sigma\left(  \ell\right)  }\right)  $. (Indeed, let $i\in\left\{
1,2,\ldots,k\right\}  $, and set $p=a_{i}$; then, the definition of $\sigma$
yields $\sigma\left(  i\right)  =\phi_{p}\left(  i\right)  \in B_{p}$ and
therefore $b_{\sigma\left(  i\right)  }=a_{i}$. Since this holds for each $i$,
we thus conclude that $\left(  b_{\sigma\left(  1\right)  },b_{\sigma\left(
2\right)  },\ldots,b_{\sigma\left(  k\right)  }\right)  =\left(  a_{1}%
,a_{2},\ldots,a_{k}\right)  $. Thus, $\left(  a_{1},a_{2},\ldots,a_{k}\right)
=\left(  b_{\sigma\left(  1\right)  },b_{\sigma\left(  2\right)  }%
,\ldots,b_{\sigma\left(  k\right)  }\right)  =\left(  b_{\sigma\left(
1\right)  },b_{\sigma\left(  2\right)  },\ldots,b_{\sigma\left(  \ell\right)
}\right)  $ (since $k=\ell$).) Thus, we have found a permutation $\sigma$ of
$\left\{  1,2,\ldots,\ell\right\}  $ such that $\left(  a_{1},a_{2}%
,\ldots,a_{k}\right)  =\left(  b_{\sigma\left(  1\right)  },b_{\sigma\left(
2\right)  },\ldots,b_{\sigma\left(  \ell\right)  }\right)  $. In other words,
the two tuples $\left(  a_{1},a_{2},\ldots,a_{k}\right)  $ and $\left(
b_{1},b_{2},\ldots,b_{\ell}\right)  $ are permutations of each other. This
proves Lemma \ref{lem.comb.tuples.mult=perm}.
\end{proof}
\end{fineprint}

We are finally ready to prove the so-called \textit{Fundamental Theorem of
Arithmetic}:

\begin{theorem}
\label{thm.ent.primes.fac-uni}Let $n$ be a positive integer.

\textbf{(a)} There exists a prime factorization of $n$.

\textbf{(b)} Any two such factorizations differ only in the order of their
entries (i.e., are permutations of each other).
\end{theorem}

\begin{proof}
[Proof of Theorem \ref{thm.ent.primes.fac-uni}.]\textbf{(a)} Proposition
\ref{prop.ent.primes.fac-ex} shows that $n$ can be written as a product of
finitely many primes. In other words, there exist finitely many primes
$p_{1},p_{2},\ldots,p_{k}$ such that $n=p_{1}p_{2}\cdots p_{k}$. Consider
these primes. Thus, $\left(  p_{1},p_{2},\ldots,p_{k}\right)  $ is a prime
factorization of $n$ (by the definition of \textquotedblleft prime
factorization\textquotedblright). Hence, there exists a prime factorization of
$n$. This proves Theorem \ref{thm.ent.primes.fac-uni} \textbf{(a)}.

\textbf{(b)} Let $\left(  a_{1},a_{2},\ldots,a_{k}\right)  $ and $\left(
b_{1},b_{2},\ldots,b_{\ell}\right)  $ be two prime factorizations of $n$. We
must prove that $\left(  a_{1},a_{2},\ldots,a_{k}\right)  $ and $\left(
b_{1},b_{2},\ldots,b_{\ell}\right)  $ differ only in the order of their
entries (i.e., are permutations of each other).

Let $P$ be the set of all primes. Note that $\left(  a_{1},a_{2},\ldots
,a_{k}\right)  $ and $\left(  b_{1},b_{2},\ldots,b_{\ell}\right)  $ are prime
factorizations of $n$. Hence, $\left(  a_{1},a_{2},\ldots,a_{k}\right)  $ and
$\left(  b_{1},b_{2},\ldots,b_{\ell}\right)  $ are tuples of primes, i.e.,
tuples of elements of $P$.

Let $p\in P$. Thus, $p$ is a prime (by the definition of $P$). Hence,
Proposition \ref{prop.ent.prime.mult-in-pf} shows that%
\begin{align*}
&  \left(  \text{the number of times }p\text{ appears in the tuple }\left(
a_{1},a_{2},\ldots,a_{k}\right)  \right) \\
&  =\left(  \text{the number of }i\in\left\{  1,2,\ldots,k\right\}  \text{
such that }a_{i}=p\right) \\
&  =v_{p}\left(  n\right)  .
\end{align*}
Similarly,%
\begin{align*}
&  \left(  \text{the number of times }p\text{ appears in the tuple }\left(
b_{1},b_{2},\ldots,b_{\ell}\right)  \right) \\
&  =\left(  \text{the number of }i\in\left\{  1,2,\ldots,\ell\right\}  \text{
such that }b_{i}=p\right) \\
&  =v_{p}\left(  n\right)  .
\end{align*}
Comparing these two equalities, we conclude that
\begin{align}
&  \left(  \text{the number of times }p\text{ appears in }\left(  a_{1}%
,a_{2},\ldots,a_{k}\right)  \right) \nonumber\\
&  =\left(  \text{the number of times }p\text{ appears in }\left(  b_{1}%
,b_{2},\ldots,b_{\ell}\right)  \right)  .
\label{pf.thm.ent.primes.fac-uni.b.1}%
\end{align}


Now, forget that we fixed $p$. We thus have proven
(\ref{pf.thm.ent.primes.fac-uni.b.1}) for each $p\in P$. Hence, Lemma
\ref{lem.comb.tuples.mult=perm} shows that the tuples $\left(  a_{1}%
,a_{2},\ldots,a_{k}\right)  $ and $\left(  b_{1},b_{2},\ldots,b_{\ell}\right)
$ differ only in the order of their entries (i.e., are permutations of each
other). This completes our proof of Theorem \ref{thm.ent.primes.fac-uni}
\textbf{(b)}.
\end{proof}

\subsubsection{The canonical factorization}

You have seen finite products such as
\begin{align*}
\prod_{i\in\left\{  1,2,3,4,5\right\}  }i  &  =1\cdot2\cdot3\cdot
4\cdot5=5!=120\ \ \ \ \ \ \ \ \ \ \text{and}\\
\prod_{i\in\left\{  3,5,7\right\}  }\left(  i^{2}+1\right)   &  =\left(
3^{2}+1\right)  \cdot\left(  5^{2}+1\right)  \cdot\left(  7^{2}+1\right)
=13000.
\end{align*}
Sometimes, infinite products (i.e., products ranging over infinite sets) also
make sense. Many examples of well-defined infinite products arise from
analysis and have to do with convergence. Here, we are doing algebra and thus
shall only consider a very elementary, non-analytic meaning of convergence.
Namely, we will consider infinite products that have only finitely many
factors different from $1$. For example, the product $2\cdot7\cdot
4\cdot\underbrace{1\cdot1\cdot1\cdot1\cdot\cdots}_{\text{infinitely many
}1\text{'s}}$ is of such form. It is easy to give a meaning to such products:
Just throw away all the $1$'s (since multiplying by $1$ does not change a
number) and take the product of the remaining (finitely many) numbers. So, for
example, our product $2\cdot7\cdot4\cdot\underbrace{1\cdot1\cdot1\cdot
1\cdot\cdots}_{\text{infinitely many }1\text{'s}}$ should evaluate to
$2\cdot7\cdot4=56$.

This is indeed a meaningful and useful definition. For example, the set of all
prime numbers is infinite (by Theorem \ref{thm.ent.primes.infin} below), but
nevertheless, for each nonzero integer $n$, the product $\prod_{p\text{
prime}}p^{v_{p}\left(  n\right)  }$ (where the \textquotedblleft%
$\prod_{p\text{ prime}}$\textquotedblright\ symbol means a product ranging
over all primes $p$) is well-defined due to having only finitely many factors
different from $1$:

\begin{lemma}
\label{lem.ent.primes.vpn=0}Let $n$ be a nonzero integer.

\textbf{(a)} We have $v_{p}\left(  n\right)  =0$ for every prime $p>\left\vert
n\right\vert $. (Note that \textquotedblleft for every prime $p>\left\vert
n\right\vert $\textquotedblright\ is shorthand for \textquotedblleft for every
prime $p$ satisfying $p>\left\vert n\right\vert $\textquotedblright.)

\textbf{(b)} The product $\prod_{p\text{ prime}}p^{v_{p}\left(  n\right)  }$
has only finitely many factors different from $1$. (Here and in the following,
the \textquotedblleft$\prod_{p\text{ prime}}$\textquotedblright\ symbol means
a product ranging over all primes $p$.)
\end{lemma}

\begin{proof}
[Proof of Lemma \ref{lem.ent.primes.vpn=0}.]\textbf{(a)} Let $p$ be a prime
such that $p>\left\vert n\right\vert $. We must prove that $v_{p}\left(
n\right)  =0$.

We have $p>1$ (since $p$ is prime); thus, $p>1>0$ and therefore $\left\vert
p\right\vert =p>\left\vert n\right\vert $.

We have $n\neq0$ (since $n$ is nonzero). Thus, if we had $p\mid n$, then we
would have $\left\vert p\right\vert \leq\left\vert n\right\vert $ (by
Proposition \ref{prop.ent.div.1} \textbf{(b)}, applied to $a=p$ and $b=n$),
which would contradict $\left\vert p\right\vert >\left\vert n\right\vert $.
Thus, we cannot have $p\mid n$. But Proposition
\ref{prop.ent.primes.div-or-coprime} (applied to $a=n$) shows that either
$p\mid n$ or $p\perp n$. Hence, $p\perp n$ (since we cannot have $p\mid n$).
In other words, $n\perp p$ (by Proposition \ref{prop.ent.coprime.perp-symm}).
Also, $n=n\cdot p^{0}$ (since $p^{0}=1$). Thus, Lemma
\ref{lem.ent.prime.vp-copr} \textbf{(b)} (applied to $i=0$ and $w=n$) yields
$v_{p}\left(  n\right)  =0$. This proves Lemma \ref{lem.ent.primes.vpn=0}
\textbf{(a)}.

\textbf{(b)} For every prime $p>\left\vert n\right\vert $, we have
$v_{p}\left(  n\right)  =0$ (by Lemma \ref{lem.ent.primes.vpn=0} \textbf{(a)})
and thus $p^{v_{p}\left(  n\right)  }=p^{0}=1$. Thus, all but finitely many
primes $p$ satisfy $p^{v_{p}\left(  n\right)  }=1$ (since all but finitely
many primes $p$ satisfy $p>\left\vert n\right\vert $). Therefore, all but
finitely many factors of the product $\prod_{p\text{ prime}}p^{v_{p}\left(
n\right)  }$ are $1$. In other words, the product $\prod_{p\text{ prime}%
}p^{v_{p}\left(  n\right)  }$ has only finitely many factors different from
$1$. This proves Lemma \ref{lem.ent.primes.vpn=0} \textbf{(b)}.
\end{proof}

\begin{corollary}
\label{cor.ent.primes.can-fac}Let $n$ be a positive integer. Then,%
\[
n=\prod_{p\text{ prime}}p^{v_{p}\left(  n\right)  }.
\]


Here, the infinite product $\prod_{p\text{ prime}}p^{v_{p}\left(  n\right)  }$
is well-defined (according to Lemma \ref{lem.ent.primes.vpn=0} \textbf{(b)}).
\end{corollary}

This expression $n=\prod_{p\text{ prime}}p^{v_{p}\left(  n\right)  }$ is
called the \textit{canonical factorization} of $n$.

\begin{proof}
[Proof of Corollary \ref{cor.ent.primes.can-fac}.]Theorem
\ref{thm.ent.primes.fac-uni} \textbf{(a)} shows that there exists a prime
factorization of $n$. Consider such a factorization, and denote it by $\left(
a_{1},a_{2},\ldots,a_{k}\right)  $. Thus, $\left(  a_{1},a_{2},\ldots
,a_{k}\right)  $ is a prime factorization of $n$; in other words, $a_{1}%
,a_{2},\ldots,a_{k}$ are primes satisfying $n=a_{1}a_{2}\cdots a_{k}$. For
each prime $p$, we have%
\begin{equation}
\left(  \text{the number of }i\in\left\{  1,2,\ldots,k\right\}  \text{ such
that }a_{i}=p\right)  =v_{p}\left(  n\right)
\label{pf.cor.ent.primes.can-fac.1}%
\end{equation}
(by Proposition \ref{prop.ent.prime.mult-in-pf}). Now,%
\begin{align*}
n  &  =a_{1}a_{2}\cdots a_{k}=\prod_{i\in\left\{  1,2,\ldots,k\right\}  }%
a_{i}\\
&  =\prod_{p\text{ prime}}\prod_{\substack{i\in\left\{  1,2,\ldots,k\right\}
;\\a_{i}=p}}\underbrace{a_{i}}_{=p}\\
&  \ \ \ \ \ \ \ \ \ \ \left(
\begin{array}
[c]{c}%
\text{here, we have split our product into smaller}\\
\text{products, according to the value of }a_{i}\text{;}\\
\text{this is allowed, since each }a_{i}\text{ is a prime}%
\end{array}
\right) \\
&  =\prod_{p\text{ prime}}\underbrace{\prod_{\substack{i\in\left\{
1,2,\ldots,k\right\}  ;\\a_{i}=p}}p}_{=p^{\left(  \text{the number of }%
i\in\left\{  1,2,\ldots,k\right\}  \text{ such that }a_{i}=p\right)  }}\\
&  =\prod_{p\text{ prime}}\underbrace{p^{\left(  \text{the number of }%
i\in\left\{  1,2,\ldots,k\right\}  \text{ such that }a_{i}=p\right)  }%
}_{\substack{=p^{v_{p}\left(  n\right)  }\\\text{(by
(\ref{pf.cor.ent.primes.can-fac.1}))}}}\\
&  =\prod_{p\text{ prime}}p^{v_{p}\left(  n\right)  }.
\end{align*}
This proves Corollary \ref{cor.ent.primes.can-fac}.
\end{proof}

\begin{corollary}
\label{cor.ent.primes.can-facZ}Let $n$ be a nonzero integer. Then,%
\[
\left\vert n\right\vert =\prod_{p\text{ prime}}p^{v_{p}\left(  n\right)  }.
\]


Here, the infinite product $\prod_{p\text{ prime}}p^{v_{p}\left(  n\right)  }$
is well-defined (according to Lemma \ref{lem.ent.primes.vpn=0} \textbf{(b)}).
\end{corollary}

\begin{proof}
[Proof of Corollary \ref{cor.ent.primes.can-facZ}.]The integer $\left\vert
n\right\vert $ is positive (since $n$ is nonzero). Hence, Corollary
\ref{cor.ent.primes.can-fac} (applied to $\left\vert n\right\vert $ instead of
$n$) yields%
\[
\left\vert n\right\vert =\prod_{p\text{ prime}}\underbrace{p^{v_{p}\left(
\left\vert n\right\vert \right)  }}_{\substack{=p^{v_{p}\left(  n\right)
}\\\text{(since Exercise \ref{exe.ent.prime.vp-abs}}\\\text{yields }%
v_{p}\left(  \left\vert n\right\vert \right)  =v_{p}\left(  n\right)
\text{)}}}=\prod_{p\text{ prime}}p^{v_{p}\left(  n\right)  }.
\]
This proves Corollary \ref{cor.ent.primes.can-facZ}.
\end{proof}

We can furthermore use $p$-adic valuations to check divisibility of integers:

\begin{proposition}
\label{prop.ent.primes.n|m}Let $n$ and $m$ be integers. Then, $n\mid m$ if and
only if each prime $p$ satisfies $v_{p}\left(  n\right)  \leq v_{p}\left(
m\right)  $.
\end{proposition}

\begin{proof}
[Proof of Proposition \ref{prop.ent.primes.n|m}.]If $m=0$, then Proposition
\ref{prop.ent.primes.n|m} is true\footnote{\textit{Proof.} Assume that $m=0$.
Thus, each prime $p$ satisfies $v_{p}\left(  \underbrace{m}_{=0}\right)
=v_{p}\left(  0\right)  =\infty$ (by Definition \ref{def.ent.prime.vp}
\textbf{(b)}) and thus $v_{p}\left(  m\right)  =\infty\geq v_{p}\left(
n\right)  $, so that $v_{p}\left(  n\right)  \leq v_{p}\left(  m\right)  $.
Also, $n\mid0=m$. Thus, the statements \textquotedblleft$n\mid m$%
\textquotedblright\ and \textquotedblleft each prime $p$ satisfies
$v_{p}\left(  n\right)  \leq v_{p}\left(  m\right)  $\textquotedblright\ are
both true. Hence, $n\mid m$ if and only if each prime $p$ satisfies
$v_{p}\left(  n\right)  \leq v_{p}\left(  m\right)  $. In other words,
Proposition \ref{prop.ent.primes.n|m} is true. Qed.}. Hence, for the rest of
this proof, we WLOG assume that $m\neq0$. Therefore, $m$ is nonzero. Hence,
$v_{p}\left(  m\right)  \in\mathbb{N}$ (by Definition \ref{def.ent.prime.vp}
\textbf{(a)}), so that $v_{p}\left(  m\right)  <\infty$.

If $n=0$, then Proposition \ref{prop.ent.primes.n|m} is
true\footnote{\textit{Proof.} Assume that $n=0$. Thus, each prime $p$
satisfies $v_{p}\left(  \underbrace{n}_{=0}\right)  =v_{p}\left(  0\right)
=\infty$ (by Definition \ref{def.ent.prime.vp} \textbf{(b)}) and thus
$v_{p}\left(  m\right)  <\infty=v_{p}\left(  n\right)  $. Applying this to
$p=2$, we obtain $v_{2}\left(  m\right)  <v_{2}\left(  n\right)  $ (since $2$
is a prime). Hence, we don't have $v_{2}\left(  n\right)  \leq v_{2}\left(
m\right)  $. Thus, the statement \textquotedblleft each prime $p$ satisfies
$v_{p}\left(  n\right)  \leq v_{p}\left(  m\right)  $\textquotedblright\ is
false (since $p=2$ is a counterexample).
\par
If we had $n\mid m$, then there would be an integer $c$ such that $m=nc$. This
would then lead to $m=\underbrace{n}_{=0}c=0$, which would contradict $m\neq
0$. Hence, we cannot have $n\mid m$. Thus, the statements \textquotedblleft%
$n\mid m$\textquotedblright\ and \textquotedblleft each prime $p$ satisfies
$v_{p}\left(  n\right)  \leq v_{p}\left(  m\right)  $\textquotedblright\ are
both false. Hence, $n\mid m$ if and only if each prime $p$ satisfies
$v_{p}\left(  n\right)  \leq v_{p}\left(  m\right)  $. In other words,
Proposition \ref{prop.ent.primes.n|m} is true. Qed.}. Hence, for the rest of
this proof, we WLOG assume that $n\neq0$. Therefore, $n$ is nonzero.

The statement of Proposition \ref{prop.ent.primes.n|m} does not change if we
replace $n$ and $m$ by $\left\vert n\right\vert $ and $\left\vert m\right\vert
$, respectively\footnote{Indeed, the statement \textquotedblleft$n\mid
m$\textquotedblright\ does not change (since Proposition \ref{prop.ent.div.1}
\textbf{(a)} yields that we have $n\mid m$ if and only if $\left\vert
n\right\vert \mid\left\vert m\right\vert $), and the statement
\textquotedblleft each prime $p$ satisfies $v_{p}\left(  n\right)  \leq
v_{p}\left(  m\right)  $\textquotedblright\ does not change either (because
Exercise \ref{exe.ent.prime.vp-abs} shows that $v_{p}\left(  \left\vert
n\right\vert \right)  =v_{p}\left(  n\right)  $ and $v_{p}\left(  \left\vert
m\right\vert \right)  =v_{p}\left(  m\right)  $).}. Hence, we can WLOG assume
that $n$ and $m$ are nonnegative. Assume this. Then, $n\geq0$, so that $n>0$
(since $n$ is nonzero). Hence, $n$ is a positive integer. Thus, Corollary
\ref{cor.ent.primes.can-fac} yields%
\begin{equation}
n=\prod_{p\text{ prime}}p^{v_{p}\left(  n\right)  }.
\label{pf.prop.ent.primes.n|m.n=}%
\end{equation}
Similarly,%
\begin{equation}
m=\prod_{p\text{ prime}}p^{v_{p}\left(  m\right)  }.
\label{pf.prop.ent.primes.n|m.m=}%
\end{equation}


Our goal is to prove that $n\mid m$ if and only if each prime $p$ satisfies
$v_{p}\left(  n\right)  \leq v_{p}\left(  m\right)  $. We shall now prove the
\textquotedblleft$\Longleftarrow$\textquotedblright\ and \textquotedblleft%
$\Longrightarrow$\textquotedblright\ directions of this \textquotedblleft if
and only if\textquotedblright\ statement separately.

$\Longleftarrow:$ Assume that each prime $p$ satisfies $v_{p}\left(  n\right)
\leq v_{p}\left(  m\right)  $. We must prove that $n\mid m$.

The product $\prod_{p\text{ prime}}p^{v_{p}\left(  m\right)  -v_{p}\left(
n\right)  }$ is well-defined\footnote{\textit{Proof. }Let $p$ be a prime such
that $p>\left\vert m\right\vert $. Then, $v_{p}\left(  m\right)  =0$ (by Lemma
\ref{lem.ent.primes.vpn=0} \textbf{(a)}, applied to $m$ instead of $n$), so
that $v_{p}\left(  m\right)  -\underbrace{v_{p}\left(  n\right)  }_{\geq0}\leq
v_{p}\left(  m\right)  =0$. On the other hand, $v_{p}\left(  n\right)  \leq
v_{p}\left(  m\right)  $ (since we assumed that each prime $p$ satisfies
$v_{p}\left(  n\right)  \leq v_{p}\left(  m\right)  $); thus, $v_{p}\left(
m\right)  -v_{p}\left(  n\right)  \geq0$. Combining this with $v_{p}\left(
m\right)  -v_{p}\left(  n\right)  \leq0$, we obtain $v_{p}\left(  m\right)
-v_{p}\left(  n\right)  =0$. Hence, $p^{v_{p}\left(  m\right)  -v_{p}\left(
n\right)  }=p^{0}=1$.
\par
Now, forget that we fixed $p$. We thus have proven that every prime
$p>\left\vert m\right\vert $ satisfies $p^{v_{p}\left(  m\right)
-v_{p}\left(  n\right)  }=1$. Hence, all but finitely many primes $p$ satisfy
$p^{v_{p}\left(  m\right)  -v_{p}\left(  n\right)  }=1$ (since all but
finitely many primes $p$ satisfy $p>\left\vert m\right\vert $). In other
words, the product $\prod_{p\text{ prime}}p^{v_{p}\left(  m\right)
-v_{p}\left(  n\right)  }$ has only finitely many factors different from $1$.
Hence, this product is well-defined.}.

We have assumed that each prime $p$ satisfies $v_{p}\left(  n\right)  \leq
v_{p}\left(  m\right)  $. In other words, each prime $p$ satisfies
$v_{p}\left(  m\right)  -v_{p}\left(  n\right)  \geq0$ and therefore
$p^{v_{p}\left(  m\right)  -v_{p}\left(  n\right)  }\in\mathbb{Z}$. Hence, the
product $\prod_{p\text{ prime}}p^{v_{p}\left(  m\right)  -v_{p}\left(
n\right)  }$ is a product of integers, and thus itself an integer. Let us
denote this product by $c$. Thus,%
\begin{equation}
c=\prod_{p\text{ prime}}p^{v_{p}\left(  m\right)  -v_{p}\left(  n\right)  }.
\label{pf.prop.ent.primes.n|m.c=}%
\end{equation}
Thus, $c$ is an integer (since we have just shown that $\prod_{p\text{ prime}%
}p^{v_{p}\left(  m\right)  -v_{p}\left(  n\right)  }$ is an integer).
Multiplying the equalities (\ref{pf.prop.ent.primes.n|m.n=}) and
(\ref{pf.prop.ent.primes.n|m.c=}), we obtain%
\begin{align*}
nc  &  =\left(  \prod_{p\text{ prime}}p^{v_{p}\left(  n\right)  }\right)
\left(  \prod_{p\text{ prime}}p^{v_{p}\left(  m\right)  -v_{p}\left(
n\right)  }\right)  =\prod_{p\text{ prime}}\underbrace{\left(  p^{v_{p}\left(
n\right)  }p^{v_{p}\left(  m\right)  -v_{p}\left(  n\right)  }\right)
}_{\substack{=p^{v_{p}\left(  n\right)  +\left(  v_{p}\left(  m\right)
-v_{p}\left(  n\right)  \right)  }=p^{v_{p}\left(  m\right)  }\\\text{(since
}v_{p}\left(  n\right)  +\left(  v_{p}\left(  m\right)  -v_{p}\left(
n\right)  \right)  =v_{p}\left(  m\right)  \text{)}}}\\
&  =\prod_{p\text{ prime}}p^{v_{p}\left(  m\right)  }%
=m\ \ \ \ \ \ \ \ \ \ \left(  \text{by (\ref{pf.prop.ent.primes.n|m.m=}%
)}\right)  .
\end{align*}
In other words, $m=nc$. Hence, $n\mid m$. This completes the proof of the
\textquotedblleft$\Longleftarrow$\textquotedblright\ direction of Proposition
\ref{prop.ent.primes.n|m}.

$\Longrightarrow:$ Assume that $n\mid m$. We must prove that each prime $p$
satisfies $v_{p}\left(  n\right)  \leq v_{p}\left(  m\right)  $.

So let $p$ be a prime. Recall that $n\mid m$. In other words, there exists
some integer $b$ such that $m=nb$. Consider this $b$. Now,%
\begin{align*}
v_{p}\left(  \underbrace{m}_{=nb}\right)   &  =v_{p}\left(  nb\right)
=v_{p}\left(  n\right)  +\underbrace{v_{p}\left(  b\right)  }_{\geq
0}\ \ \ \ \ \ \ \ \ \ \left(  \text{by Theorem \ref{thm.ent.prime.vp-ring}
\textbf{(a)}, applied to }a=n\right) \\
&  \geq v_{p}\left(  n\right)  ,
\end{align*}
so that $v_{p}\left(  n\right)  \leq v_{p}\left(  m\right)  $. Now, forget
that we fixed $p$. We thus have proven that each prime $p$ satisfies
$v_{p}\left(  n\right)  \leq v_{p}\left(  m\right)  $. This completes the
proof of the \textquotedblleft$\Longrightarrow$\textquotedblright\ direction
of Proposition \ref{prop.ent.primes.n|m}.
\end{proof}

Let us extract one of the steps of our above proof into a separate lemma,
since we shall use the same reasoning later on:

\begin{lemma}
\label{lem.ent.primes.prod|prod}For each prime $p$, let $a_{p}$ and $b_{p}$ be
nonnegative integers such that
\begin{equation}
a_{p}\leq b_{p}. \label{eq.lem.ent.primes.prod|prod.apbp}%
\end{equation}
Assume that all but finitely many primes $p$ satisfy $b_{p}=0$. Then, the
products $\prod_{p\text{ prime}}p^{a_{p}}$ and $\prod_{p\text{ prime}}%
p^{b_{p}}$ are both well-defined and satisfy $\prod_{p\text{ prime}}p^{a_{p}%
}\mid\prod_{p\text{ prime}}p^{b_{p}}$.
\end{lemma}

\begin{proof}
[Proof of Lemma \ref{lem.ent.primes.prod|prod}.]This is going to be really
boring: The well-definedness part is all about bookkeeping finiteness
information, whereas the $\prod_{p\text{ prime}}p^{a_{p}}\mid\prod_{p\text{
prime}}p^{b_{p}}$ claim is proven just as we proved the \textquotedblleft%
$\Longleftarrow$\textquotedblright\ direction of Proposition
\ref{prop.ent.primes.n|m}. For the sake of completeness, let us nevertheless
give the complete proof:

\begin{fineprint}
All but finitely many primes $p$ satisfy $b_{p}=0$. In other words, there
exists some finite set $S$ of primes such that every prime $p\notin S$
satisfies
\begin{equation}
b_{p}=0. \label{pf.lem.ent.primes.prod|prod.1}%
\end{equation}
Consider this $S$. Clearly, all but finitely many primes $p$ satisfy $p\notin
S$ (since $S$ is finite).

Now, every prime $p\notin S$ satisfies
\begin{equation}
a_{p}=0 \label{pf.lem.ent.primes.prod|prod.2}%
\end{equation}
\ \footnote{\textit{Proof.} Let $p\notin S$ be a prime. Then,
(\ref{eq.lem.ent.primes.prod|prod.apbp}) yields $a_{p}\leq b_{p}=0$ (by
(\ref{pf.lem.ent.primes.prod|prod.1})). Thus, $a_{p}=0$ (since $a_{p}$ is a
nonnegative integer), qed.}. Hence, all but finitely many primes $p$ satisfy
$a_{p}=0$ (since all but finitely many primes $p$ satisfy $p\notin S$). Thus,
all but finitely many primes $p$ satisfy $p^{a_{p}}=p^{0}=1$. In other words,
only finitely many primes $p$ satisfy $p^{a_{p}}\neq1$. In other words, only
finitely many factors of the product $\prod_{p\text{ prime}}p^{a_{p}}$ are
different from $1$. Hence, this product $\prod_{p\text{ prime}}p^{a_{p}}$ is well-defined.

Also, all but finitely many primes $p$ satisfy $b_{p}=0$. Therefore, all but
finitely many primes $p$ satisfy $p^{b_{p}}=p^{0}=1$. In other words, only
finitely many primes $p$ satisfy $p^{b_{p}}\neq1$. In other words, only
finitely many factors of the product $\prod_{p\text{ prime}}p^{b_{p}}$ are
different from $1$. Hence, this product $\prod_{p\text{ prime}}p^{b_{p}}$ is well-defined.

The product $\prod_{p\text{ prime}}p^{b_{p}-a_{p}}$ is
well-defined\footnote{\textit{Proof. }Every prime $p\notin S$ satisfies
$\underbrace{b_{p}}_{\substack{=0\\\text{(by
(\ref{pf.lem.ent.primes.prod|prod.1}))}}}-\underbrace{a_{p}}%
_{\substack{=0\\\text{(by (\ref{pf.lem.ent.primes.prod|prod.2}))}}}=0-0=0$ and
therefore $p^{b_{p}-a_{p}}=p^{0}=1$. Thus, all but finitely many primes $p$
satisfy $p^{b_{p}-a_{p}}=1$ (since all but finitely many primes $p$ satisfy
$p\notin S$). In other words, only finitely many primes $p$ satisfy
$p^{b_{p}-a_{p}}\neq1$. In other words, only finitely many factors of the
product $\prod_{p\text{ prime}}p^{b_{p}-a_{p}}$ are different from $1$. Hence,
this product $\prod_{p\text{ prime}}p^{b_{p}-a_{p}}$ is well-defined.}. Denote
this product by $c$.

For each prime $p$, we have $b_{p}-a_{p}\geq0$ (by
(\ref{eq.lem.ent.primes.prod|prod.apbp})) and thus $b_{p}-a_{p}\in\mathbb{N}$.
Hence, for each prime $p$, the number $p^{b_{p}-a_{p}}$ is an integer.
Therefore, $\prod_{p\text{ prime}}p^{b_{p}-a_{p}}$ is a product of integers,
and thus itself an integer. In other words, $c$ is an integer (since
$c=\prod_{p\text{ prime}}p^{b_{p}-a_{p}}$).

But from $c=\prod_{p\text{ prime}}p^{b_{p}-a_{p}}$, we obtain%
\[
\left(  \prod_{p\text{ prime}}p^{a_{p}}\right)  c=\left(  \prod_{p\text{
prime}}p^{a_{p}}\right)  \left(  \prod_{p\text{ prime}}p^{b_{p}-a_{p}}\right)
=\prod_{p\text{ prime}}\underbrace{\left(  p^{a_{p}}p^{b_{p}-a_{p}}\right)
}_{=p^{a_{p}+\left(  b_{p}-a_{p}\right)  }=p^{b_{p}}}=\prod_{p\text{ prime}%
}p^{b_{p}}.
\]
Thus, $\prod_{p\text{ prime}}p^{a_{p}}\mid\prod_{p\text{ prime}}p^{b_{p}}$
(since $c$ is an integer). This completes the proof of Lemma
\ref{lem.ent.primes.prod|prod}.
\end{fineprint}
\end{proof}

\begin{corollary}
\label{cor.ent.primes.vp-of-can}For each prime $p$, let $b_{p}$ be a
nonnegative integer. Assume that all but finitely many primes $p$ satisfy
$b_{p}=0$. Let $n=\prod_{p\text{ prime}}p^{b_{p}}$. Then,%
\[
v_{q}\left(  n\right)  =b_{q}\ \ \ \ \ \ \ \ \ \ \text{for each prime }q.
\]

\end{corollary}

\begin{proof}
[Proof of Corollary \ref{cor.ent.primes.vp-of-can}.]The product $\prod
_{p\text{ prime}}p^{b_{p}}$ is well-defined. (This can be shown just as in the
proof of Lemma \ref{lem.ent.primes.prod|prod}.) Now, choose a list $\left(
a_{1},a_{2},\ldots,a_{k}\right)  $ of primes that contains each prime $p$
exactly $b_{p}$ times. (Such a list clearly exists: For example, we can pick%
\[
\left(  \underbrace{2,2,\ldots,2}_{b_{2}\text{ times}},\underbrace{3,3,\ldots
,3}_{b_{3}\text{ times}},\underbrace{5,5,\ldots,5}_{b_{5}\text{ times}}%
,\ldots\right)  .
\]
This is indeed a finite list, since all but finitely many primes $p$ satisfy
$b_{p}=0$.)

Now, the list $\left(  a_{1},a_{2},\ldots,a_{k}\right)  $ contains each prime
$p$ exactly $b_{p}$ times (and no other entries). Hence, the product
$a_{1}a_{2}\cdots a_{k}$ of the entries of this list contains each prime $p$
exactly $b_{p}$ times as a factor (and no other factors). Thus, this product
equals $\prod_{p\text{ prime}}p^{b_{p}}$. In other words, $a_{1}a_{2}\cdots
a_{k}=\prod_{p\text{ prime}}p^{b_{p}}$. Hence,%
\[
n=\prod_{p\text{ prime}}p^{b_{p}}=a_{1}a_{2}\cdots a_{k}.
\]
Thus, $\left(  a_{1},a_{2},\ldots,a_{k}\right)  $ is a prime factorization of
$n$ (since $\left(  a_{1},a_{2},\ldots,a_{k}\right)  $ is a tuple of primes).

Let $q$ be a prime. Proposition \ref{prop.ent.prime.mult-in-pf} (applied to
$p=q$) yields%
\begin{align*}
&  \left(  \text{the number of times }q\text{ appears in the tuple }\left(
a_{1},a_{2},\ldots,a_{k}\right)  \right) \\
&  =\left(  \text{the number of }i\in\left\{  1,2,\ldots,k\right\}  \text{
such that }a_{i}=q\right) \\
&  =v_{q}\left(  n\right)  .
\end{align*}
Thus,%
\[
v_{q}\left(  n\right)  =\left(  \text{the number of times }q\text{ appears in
the tuple }\left(  a_{1},a_{2},\ldots,a_{k}\right)  \right)  =b_{q}%
\]
(since the list $\left(  a_{1},a_{2},\ldots,a_{k}\right)  $ contains each
prime $p$ exactly $b_{p}$ times, and thus contains the prime $q$ exactly
$b_{q}$ times). This proves Corollary \ref{cor.ent.primes.vp-of-can}.
\end{proof}

\begin{center}
\textbf{2019-02-13 lecture}
\end{center}

Canonical factorizations can also be used to describe gcds and lcms:

\begin{proposition}
\label{prop.ent.primes.gcd}Let $n$ and $m$ be two nonzero integers. Then,%
\begin{equation}
\gcd\left(  n,m\right)  =\prod_{p\text{ prime}}p^{\min\left\{  v_{p}\left(
n\right)  ,v_{p}\left(  m\right)  \right\}  }
\label{eq.prop.ent.primes.gcd.gcd}%
\end{equation}
an%
\begin{equation}
\operatorname{lcm}\left(  n,m\right)  =\prod_{p\text{ prime}}p^{\max\left\{
v_{p}\left(  n\right)  ,v_{p}\left(  m\right)  \right\}  }.
\label{eq.prop.ent.primes.gcd.lcm}%
\end{equation}

\end{proposition}

\begin{proof}
[Proof of Proposition \ref{prop.ent.primes.gcd}.]If $p$ is any prime, then
$v_{p}\left(  n\right)  $ and $v_{p}\left(  m\right)  $ are nonnegative
integers (since $n$ and $m$ are nonzero), and thus so are $\min\left\{
v_{p}\left(  n\right)  ,v_{p}\left(  m\right)  \right\}  $ and $\max\left\{
v_{p}\left(  n\right)  ,v_{p}\left(  m\right)  \right\}  $.

It is easy to see that the infinite products $\prod_{p\text{ prime}}%
p^{\min\left\{  v_{p}\left(  n\right)  ,v_{p}\left(  m\right)  \right\}  }$
and \newline$\prod_{p\text{ prime}}p^{\max\left\{  v_{p}\left(  n\right)
,v_{p}\left(  m\right)  \right\}  }$ are well-defined\footnote{\textit{Proof.}
Let $p$ be a prime such that $p>\max\left\{  \left\vert m\right\vert
,\left\vert n\right\vert \right\}  $. Thus, $p>\max\left\{  \left\vert
m\right\vert ,\left\vert n\right\vert \right\}  \geq\left\vert m\right\vert $
and therefore $v_{p}\left(  m\right)  =0$ (by Lemma \ref{lem.ent.primes.vpn=0}
\textbf{(a)}, applied to $m$ instead of $n$). Similarly, $v_{p}\left(
n\right)  =0$. Hence, $\max\left\{  \underbrace{v_{p}\left(  n\right)  }%
_{=0},\underbrace{v_{p}\left(  m\right)  }_{=0}\right\}  =\max\left\{
0,0\right\}  =0$ and therefore $p^{\max\left\{  v_{p}\left(  n\right)
,v_{p}\left(  m\right)  \right\}  }=p^{0}=1$.
\par
Now, forget that we fixed $p$. We thus have proven that every prime
$p>\max\left\{  \left\vert m\right\vert ,\left\vert n\right\vert \right\}  $
satisfies $p^{\max\left\{  v_{p}\left(  n\right)  ,v_{p}\left(  m\right)
\right\}  }=1$. Hence, all but finitely many primes $p$ satisfy $p^{\max
\left\{  v_{p}\left(  n\right)  ,v_{p}\left(  m\right)  \right\}  }=1$ (since
all but finitely many primes $p$ satisfy $p>\max\left\{  \left\vert
m\right\vert ,\left\vert n\right\vert \right\}  $). In other words, the
product $\prod_{p\text{ prime}}p^{\max\left\{  v_{p}\left(  n\right)
,v_{p}\left(  m\right)  \right\}  }$ has only finitely many factors different
from $1$. Hence, this product is well-defined. Similarly, we can show that the
product $\prod_{p\text{ prime}}p^{\min\left\{  v_{p}\left(  n\right)
,v_{p}\left(  m\right)  \right\}  }$ is well-defined.}.

Define two nonnegative integers%
\begin{equation}
g=\prod\limits_{p\text{ prime}}p^{\min\left\{  v_{p}\left(  n\right)
,v_{p}\left(  m\right)  \right\}  }\ \ \ \ \ \ \ \ \ \ \text{and}%
\ \ \ \ \ \ \ \ \ \ h=\gcd\left(  n,m\right)  .
\label{pf.prop.ent.primes.gcd.g=h=}%
\end{equation}


Note that $h=\gcd\left(  n,m\right)  $ is a positive integer (since $n$ and
$m$ are nonzero) and thus nonzero. Thus, $v_{p}\left(  h\right)  $ is a
nonnegative integer for each prime $p$.

Corollary \ref{cor.ent.primes.can-facZ} yields $\left\vert n\right\vert
=\prod\limits_{p\text{ prime}}p^{v_{p}\left(  n\right)  }$. But each prime $p$
satisfies \newline$\min\left\{  v_{p}\left(  n\right)  ,v_{p}\left(  m\right)
\right\}  \leq v_{p}\left(  n\right)  $ (since the minimum of a set is $\leq$
to any element of the set). Hence, Lemma \ref{lem.ent.primes.prod|prod}
(applied to $a_{p}=\min\left\{  v_{p}\left(  n\right)  ,v_{p}\left(  m\right)
\right\}  $ and $b_{p}=v_{p}\left(  n\right)  $) yields $\prod\limits_{p\text{
prime}}p^{\min\left\{  v_{p}\left(  n\right)  ,v_{p}\left(  m\right)
\right\}  }\mid\prod\limits_{p\text{ prime}}p^{v_{p}\left(  n\right)  }$. This
rewrites as $g\mid\left\vert n\right\vert $ (since $g=\prod\limits_{p\text{
prime}}p^{\min\left\{  v_{p}\left(  n\right)  ,v_{p}\left(  m\right)
\right\}  }$ and $\left\vert n\right\vert =\prod\limits_{p\text{ prime}%
}p^{v_{p}\left(  n\right)  }$). Hence, $g\mid\left\vert n\right\vert \mid n$
(by Exercise \ref{exe.ent.div.aabs} \textbf{(b)}). Similarly, $g\mid m$. Thus,
$\left(  g\mid n\text{ and }g\mid m\right)  $. Hence, Lemma
\ref{lem.ent.gcd.uniprop} (applied to $g$, $n$ and $m$ instead of $m$, $a$ and
$b$) yields $g\mid\gcd\left(  n,m\right)  =h$.

On the other hand, Proposition \ref{prop.ent.primes.n|m} (applied to $h$ and
$n$ instead of $n$ and $m$) shows that $h\mid n$ if and only if each prime $p$
satisfies $v_{p}\left(  h\right)  \leq v_{p}\left(  n\right)  $. Thus, each
prime $p$ satisfies $v_{p}\left(  h\right)  \leq v_{p}\left(  n\right)  $.

Now, fix any prime $p$. Then, $v_{p}\left(  h\right)  \leq v_{p}\left(
n\right)  $ (as we have just seen) and $v_{p}\left(  h\right)  \leq
v_{p}\left(  m\right)  $ (similarly). Combining these two inequalities, we
obtain%
\[
v_{p}\left(  h\right)  \leq\min\left\{  v_{p}\left(  n\right)  ,v_{p}\left(
m\right)  \right\}
\]
(since $\min\left\{  v_{p}\left(  n\right)  ,v_{p}\left(  m\right)  \right\}
$ must be one of the two numbers $v_{p}\left(  n\right)  $ and $v_{p}\left(
m\right)  $, but we have just seen that $v_{p}\left(  h\right)  $ is $\leq$ to
each of these two numbers).

Now, forget that we fixed $p$. We thus have show that each prime $p$ satisfies
$v_{p}\left(  h\right)  \leq\min\left\{  v_{p}\left(  n\right)  ,v_{p}\left(
m\right)  \right\}  $. Hence, Lemma \ref{lem.ent.primes.prod|prod} (applied to
$a_{p}=v_{p}\left(  h\right)  $ and $b_{p}=\min\left\{  v_{p}\left(  n\right)
,v_{p}\left(  m\right)  \right\}  $) yields $\prod\limits_{p\text{ prime}%
}p^{v_{p}\left(  h\right)  }\mid\prod\limits_{p\text{ prime}}p^{\min\left\{
v_{p}\left(  n\right)  ,v_{p}\left(  m\right)  \right\}  }$. But $h$ is
positive; hence, Corollary \ref{cor.ent.primes.can-fac} (applied to $h$
instead of $n$) yields%
\[
h=\prod\limits_{p\text{ prime}}p^{v_{p}\left(  h\right)  }\mid\prod
\limits_{p\text{ prime}}p^{\min\left\{  v_{p}\left(  n\right)  ,v_{p}\left(
m\right)  \right\}  }=g.
\]


Thus, we know that $g\mid h$ and $h\mid g$. Hence, Exercise
\ref{exe.ent.div.abba} (applied to $a=g$ and $b=h$) yields $\left\vert
g\right\vert =\left\vert h\right\vert $. But $g$ is nonnegative; thus,
$\left\vert g\right\vert =g$. Hence, $g=\left\vert g\right\vert =\left\vert
h\right\vert =h$ (since $h$ is positive). In view of
(\ref{pf.prop.ent.primes.gcd.g=h=}), this rewrites as $\prod\limits_{p\text{
prime}}p^{\min\left\{  v_{p}\left(  n\right)  ,v_{p}\left(  m\right)
\right\}  }=\gcd\left(  n,m\right)  $. This proves
(\ref{eq.prop.ent.primes.gcd.gcd}).

The proof of (\ref{eq.prop.ent.primes.gcd.lcm}) is entirely analogous to the
proof of (\ref{eq.prop.ent.primes.gcd.gcd}) we just gave: We merely need to
flip all divisibilities and inequalities and replace \textquotedblleft$\min
$\textquotedblright\ by \textquotedblleft$\max$\textquotedblright\ everywhere,
and use Lemma \ref{lem.ent.lcm.uniprop} instead of Lemma
\ref{lem.ent.gcd.uniprop}
\end{proof}

\begin{example}
For this example, set $n=3^{2}\cdot5\cdot7^{8}$ and $m=2\cdot3^{3}\cdot7^{2}$.
Let us compute $\gcd\left(  n,m\right)  $ and $\operatorname{lcm}\left(
n,m\right)  $ using Proposition \ref{prop.ent.primes.gcd}.

From $n=3^{2}\cdot5\cdot7^{8}$, we obtain (using Corollary
\ref{cor.ent.primes.vp-of-can}) that%
\begin{align*}
v_{3}\left(  n\right)   &  =2,\ \ \ \ \ \ \ \ \ \ v_{5}\left(  n\right)
=1,\ \ \ \ \ \ \ \ \ \ v_{7}\left(  n\right)
=8,\ \ \ \ \ \ \ \ \ \ \text{and}\\
v_{p}\left(  n\right)   &  =0\text{ for each prime }p\notin\left\{
3,5,7\right\}  .
\end{align*}
Similarly, from $m=2\cdot3^{3}\cdot7^{2}$, we obtain%
\begin{align*}
v_{2}\left(  m\right)   &  =1,\ \ \ \ \ \ \ \ \ \ v_{3}\left(  m\right)
=3,\ \ \ \ \ \ \ \ \ \ v_{7}\left(  m\right)
=2,\ \ \ \ \ \ \ \ \ \ \text{and}\\
v_{p}\left(  n\right)   &  =0\text{ for each prime }p\notin\left\{
2,3,7\right\}  .
\end{align*}
Now, (\ref{eq.prop.ent.primes.gcd.gcd}) yields%
\begin{align*}
&  \gcd\left(  n,m\right) \\
&  =\prod_{p\text{ prime}}p^{\min\left\{  v_{p}\left(  n\right)  ,v_{p}\left(
m\right)  \right\}  }\\
&  =\underbrace{2^{\min\left\{  v_{2}\left(  n\right)  ,v_{2}\left(  m\right)
\right\}  }}_{=2^{\min\left\{  0,1\right\}  }=2^{0}}\cdot\underbrace{3^{\min
\left\{  v_{3}\left(  n\right)  ,v_{3}\left(  m\right)  \right\}  }}%
_{=3^{\min\left\{  2,3\right\}  }=3^{2}}\cdot\underbrace{5^{\min\left\{
v_{5}\left(  n\right)  ,v_{5}\left(  m\right)  \right\}  }}_{=5^{\min\left\{
1,0\right\}  }=5^{0}}\\
&  \ \ \ \ \ \ \ \ \ \ \cdot\underbrace{7^{\min\left\{  v_{7}\left(  n\right)
,v_{7}\left(  m\right)  \right\}  }}_{=7^{\min\left\{  8,2\right\}  }=7^{2}%
}\cdot\prod_{\substack{p\text{ prime;}\\p\notin\left\{  2,3,5,7\right\}
}}\underbrace{p^{\min\left\{  v_{p}\left(  n\right)  ,v_{p}\left(  m\right)
\right\}  }}_{\substack{=1\\\text{(since }v_{p}\left(  n\right)  =0\text{ and
}v_{p}\left(  m\right)  =0\\\text{and thus }\min\left\{  v_{p}\left(
n\right)  ,v_{p}\left(  m\right)  \right\}  =\min\left\{  0,0\right\}
=0\text{)}}}\\
&  =2^{0}\cdot3^{2}\cdot5^{0}\cdot7^{2}=3^{2}\cdot7^{2}.
\end{align*}
Likewise, (\ref{eq.prop.ent.primes.gcd.lcm}) yields%
\begin{align*}
&  \operatorname{lcm}\left(  n,m\right) \\
&  =\prod_{p\text{ prime}}p^{\max\left\{  v_{p}\left(  n\right)  ,v_{p}\left(
m\right)  \right\}  }\\
&  =\underbrace{2^{\max\left\{  v_{2}\left(  n\right)  ,v_{2}\left(  m\right)
\right\}  }}_{=2^{\max\left\{  0,1\right\}  }=2^{1}}\cdot\underbrace{3^{\max
\left\{  v_{3}\left(  n\right)  ,v_{3}\left(  m\right)  \right\}  }}%
_{=3^{\max\left\{  2,3\right\}  }=3^{3}}\cdot\underbrace{5^{\max\left\{
v_{5}\left(  n\right)  ,v_{5}\left(  m\right)  \right\}  }}_{=5^{\max\left\{
1,0\right\}  }=5^{1}}\\
&  \ \ \ \ \ \ \ \ \ \ \cdot\underbrace{7^{\max\left\{  v_{7}\left(  n\right)
,v_{7}\left(  m\right)  \right\}  }}_{=7^{\max\left\{  8,2\right\}  }=7^{8}%
}\cdot\prod_{\substack{p\text{ prime;}\\p\notin\left\{  2,3,5,7\right\}
}}\underbrace{p^{\max\left\{  v_{p}\left(  n\right)  ,v_{p}\left(  m\right)
\right\}  }}_{\substack{=1\\\text{(since }v_{p}\left(  n\right)  =0\text{ and
}v_{p}\left(  m\right)  =0\\\text{and thus }\max\left\{  v_{p}\left(
n\right)  ,v_{p}\left(  m\right)  \right\}  =\max\left\{  0,0\right\}
=0\text{)}}}\\
&  =2^{1}\cdot3^{3}\cdot5^{1}\cdot7^{8}.
\end{align*}

\end{example}

TODO: Write from here.

We can use Proposition \ref{prop.ent.primes.gcd} to reprove certain facts
about lcms and gcds. For example:

\begin{proof}
[Second proof of Theorem \ref{thm.ent.lcm.gcd*lcm}.]We must prove that%
\[
\gcd\left(  a,b\right)  \cdot\operatorname{lcm}\left(  a,b\right)  =\left\vert
ab\right\vert
\]
for all $a,b\in\mathbb{Z}$. WLOG assume that $a,b\neq0$. Then, the canonical
factorization of $\left\vert a\right\vert $ is%
\[
\left\vert a\right\vert =\prod_{p\text{ prime}}p^{v_{p}\left(  \left\vert
a\right\vert \right)  }=\prod_{p\text{ prime}}p^{v_{p}\left(  a\right)
}\ \ \ \ \ \ \ \ \ \ \left(  \text{since }v_{p}\left(  \left\vert a\right\vert
\right)  =v_{p}\left(  a\right)  \right)  .
\]
Similarly,%
\[
\left\vert b\right\vert =\prod_{p\text{ prime}}p^{v_{p}\left(  b\right)  }.
\]
Now, Proposition \ref{prop.ent.primes.gcd} yields%
\begin{align*}
\gcd\left(  a,b\right)   &  =\prod_{p\text{ prime}}p^{\min\left\{
v_{p}\left(  a\right)  ,v_{p}\left(  b\right)  \right\}  }%
\ \ \ \ \ \ \ \ \ \ \text{and}\\
\operatorname{lcm}\left(  a,b\right)   &  =\prod_{p\text{ prime}}%
p^{\max\left\{  v_{p}\left(  a\right)  ,v_{p}\left(  b\right)  \right\}  }.
\end{align*}
Multiplying these, you get%
\begin{align*}
&  \gcd\left(  a,b\right)  \cdot\operatorname{lcm}\left(  a,b\right) \\
&  =\left(  \prod_{p\text{ prime}}p^{\min\left\{  v_{p}\left(  a\right)
,v_{p}\left(  b\right)  \right\}  }\right)  \cdot\left(  \prod_{p\text{
prime}}p^{\max\left\{  v_{p}\left(  a\right)  ,v_{p}\left(  b\right)
\right\}  }\right) \\
&  =\prod_{p\text{ prime}}\underbrace{\left(  p^{\min\left\{  v_{p}\left(
a\right)  ,v_{p}\left(  b\right)  \right\}  }p^{\max\left\{  v_{p}\left(
a\right)  ,v_{p}\left(  b\right)  \right\}  }\right)  }_{\substack{=p^{\min
\left\{  v_{p}\left(  a\right)  ,v_{p}\left(  b\right)  \right\}
+\max\left\{  v_{p}\left(  a\right)  ,v_{p}\left(  b\right)  \right\}
}\\=p^{v_{p}\left(  a\right)  +v_{p}\left(  b\right)  }\\\text{(since }%
\min\left\{  u,v\right\}  +\max\left\{  u,v\right\}  =u+v\text{ for any
}u,v\text{)}}}\\
&  =\prod_{p\text{ prime}}\underbrace{p^{v_{p}\left(  a\right)  +v_{p}\left(
b\right)  }}_{=p^{v_{p}\left(  a\right)  }p^{v_{p}\left(  b\right)  }}%
=\prod_{p\text{ prime}}\left(  p^{v_{p}\left(  a\right)  }p^{v_{p}\left(
b\right)  }\right) \\
&  =\underbrace{\left(  \prod_{p\text{ prime}}p^{v_{p}\left(  a\right)
}\right)  }_{\substack{=\left\vert a\right\vert \\\text{(by canonical
factorization)}}}\underbrace{\left(  \prod_{p\text{ prime}}p^{v_{p}\left(
b\right)  }\right)  }_{=\left\vert b\right\vert }=\left\vert a\right\vert
\cdot\left\vert b\right\vert =\left\vert ab\right\vert .
\end{align*}

\end{proof}

\begin{proof}
[Second solution to Exercise \ref{exe.ent.lcm.lcmabc}.]We have%
\begin{align*}
&  \underbrace{\gcd\left(  a,b,c\right)  }_{=\prod\limits_{p\text{ prime}%
}p^{\min\left\{  v_{p}\left(  a\right)  ,v_{p}\left(  b\right)  ,v_{p}\left(
c\right)  \right\}  }}\cdot\underbrace{\operatorname{lcm}\left(
bc,ca,ab\right)  }_{=\prod\limits_{p\text{ prime}}p^{\max\left\{  v_{p}\left(
bc\right)  ,v_{p}\left(  ca\right)  ,v_{p}\left(  ab\right)  \right\}  }}\\
&  =\left(  \prod\limits_{p\text{ prime}}p^{\min\left\{  v_{p}\left(
a\right)  ,v_{p}\left(  b\right)  ,v_{p}\left(  c\right)  \right\}  }\right)
\cdot\left(  \prod\limits_{p\text{ prime}}p^{\max\left\{  v_{p}\left(
bc\right)  ,v_{p}\left(  ca\right)  ,v_{p}\left(  ab\right)  \right\}
}\right) \\
&  =\prod\limits_{p\text{ prime}}\underbrace{p^{\min\left\{  v_{p}\left(
a\right)  ,v_{p}\left(  b\right)  ,v_{p}\left(  c\right)  \right\}
+\max\left\{  v_{p}\left(  bc\right)  ,v_{p}\left(  ca\right)  ,v_{p}\left(
ab\right)  \right\}  }}_{\substack{=p^{v_{p}\left(  a\right)  +v_{p}\left(
b\right)  +v_{p}\left(  c\right)  }\\\text{(this is because }v_{p}\left(
bc\right)  =v_{p}\left(  abc\right)  -v_{p}\left(  a\right)  \text{ and
likewise for the other two,}\\\text{so }\max\left\{  v_{p}\left(  bc\right)
,v_{p}\left(  ca\right)  ,v_{p}\left(  ab\right)  \right\}  \\=\max\left\{
v_{p}\left(  abc\right)  -v_{p}\left(  a\right)  ,v_{p}\left(  abc\right)
-v_{p}\left(  b\right)  ,v_{p}\left(  abc\right)  -v_{p}\left(  c\right)
\right\}  \\=v_{p}\left(  abc\right)  -\min\left\{  v_{p}\left(  a\right)
,v_{p}\left(  b\right)  ,v_{p}\left(  c\right)  \right\}  \text{,}\\\text{so
}\min\left\{  v_{p}\left(  a\right)  ,v_{p}\left(  b\right)  ,v_{p}\left(
c\right)  \right\}  +\max\left\{  v_{p}\left(  bc\right)  ,v_{p}\left(
ca\right)  ,v_{p}\left(  ab\right)  \right\}  \\=v_{p}\left(  abc\right)
=v_{p}\left(  a\right)  +v_{p}\left(  b\right)  +v_{p}\left(  c\right)
\text{)}}}\\
&  =\prod\limits_{p\text{ prime}}p^{v_{p}\left(  a\right)  +v_{p}\left(
b\right)  +v_{p}\left(  c\right)  }=\underbrace{\left(  \prod\limits_{p\text{
prime}}p^{v_{p}\left(  a\right)  }\right)  }_{=\left\vert a\right\vert
}\underbrace{\left(  \prod\limits_{p\text{ prime}}p^{v_{p}\left(  b\right)
}\right)  }_{=\left\vert b\right\vert }\underbrace{\left(  \prod
\limits_{p\text{ prime}}p^{v_{p}\left(  c\right)  }\right)  }_{=\left\vert
c\right\vert }\\
&  =\left\vert a\right\vert \cdot\left\vert b\right\vert \cdot\left\vert
c\right\vert =\left\vert abc\right\vert .
\end{align*}
This solves Exercise \ref{exe.ent.lcm.lcmabc} \textbf{(a)} again.

Similarly we can re-solve Exercise \ref{exe.ent.lcm.lcmabc} \textbf{(b)}.
\end{proof}

\begin{remark}
Similarly, we can show that
\begin{align*}
\gcd\left(  a,b,c,d\right)  \cdot\operatorname{lcm}\left(
bcd,cda,dab,abc\right)   &  =\left\vert abcd\right\vert ;\\
\operatorname{lcm}\left(  a,b,c,d\right)  \cdot\gcd\left(
bcd,cda,dab,abc\right)   &  =\left\vert abcd\right\vert
\end{align*}
for any four integers $a,b,c,d$. Indeed, the last equality holds since%
\begin{align*}
&  \min\left\{  v_{p}\left(  a\right)  ,v_{p}\left(  b\right)  ,v_{p}\left(
c\right)  ,v_{p}\left(  d\right)  \right\} \\
&  +\max\left\{  \underbrace{v_{p}\left(  bcd\right)  }_{=v_{p}\left(
abcd\right)  -v_{p}\left(  a\right)  },\underbrace{v_{p}\left(  cda\right)
,v_{p}\left(  dab\right)  ,v_{p}\left(  abc\right)  }_{\text{similarly}%
}\right\} \\
&  =\min\left\{  v_{p}\left(  a\right)  ,v_{p}\left(  b\right)  ,v_{p}\left(
c\right)  ,v_{p}\left(  d\right)  \right\} \\
&  +\underbrace{\max\left\{  v_{p}\left(  abcd\right)  -v_{p}\left(  a\right)
,v_{p}\left(  abcd\right)  -v_{p}\left(  b\right)  ,v_{p}\left(  abcd\right)
-v_{p}\left(  c\right)  ,v_{p}\left(  abcd\right)  -v_{p}\left(  d\right)
\right\}  }_{=v_{p}\left(  abcd\right)  -\min\left\{  v_{p}\left(  a\right)
,v_{p}\left(  b\right)  ,v_{p}\left(  c\right)  ,v_{p}\left(  d\right)
\right\}  }\\
&  =v_{p}\left(  abcd\right)  .
\end{align*}
Similarly the first equality holds. You can likewise prove generalizations to
$k$ integers.
\end{remark}

\subsubsection{Coprimality through prime factors}

\begin{proposition}
\label{prop.ent.coprime.via-primes}Let $n$ and $m$ be two integers. Then,
$n\perp m$ if and only if there exists no prime $p$ that divides both $n$ and
$m$.
\end{proposition}

\begin{proof}
[Proof of Proposition \ref{prop.ent.coprime.via-primes}.]$\Longrightarrow:$
Assume that $n\perp m$. Thus, $\gcd\left(  n,m\right)  =1$. Now, if a prime
$p$ would divide both $n$ and $m$, then it would also divide $\gcd\left(
n,m\right)  =1$; but a prime $p$ cannot divide $1$. So there exists no prime
$p$ that divides both $n$ and $m$.

$\Longleftarrow:$ Assume that there exists no prime $p$ that divides both $n$
and $m$. In other words, no prime divides $\gcd\left(  n,m\right)  $. But if
we had $\gcd\left(  n,m\right)  >1$, then there would be a prime dividing
$\gcd\left(  n,m\right)  $ (since any integer $>1$ has a prime factor). So we
have $\gcd\left(  n,m\right)  \leq1$, thus $\gcd\left(  n,m\right)  =1$. So
$n\perp m$.
\end{proof}

\begin{corollary}
Let $n$ and $m$ be two nonzero integers. Then, $n\perp m$ if and only if
\[
\sum_{p\text{ prime}}v_{p}\left(  n\right)  v_{p}\left(  m\right)  =0.
\]

\end{corollary}

\begin{proof}
We have the following equivalence:%
\begin{align*}
&  \left(  \sum_{p\text{ prime}}v_{p}\left(  n\right)  v_{p}\left(  m\right)
=0\right) \\
&  \Longleftrightarrow\ \left(  v_{p}\left(  n\right)  v_{p}\left(  m\right)
=0\text{ for all primes }p\right) \\
&  \ \ \ \ \ \ \ \ \ \ \left(  \text{since }v_{p}\left(  n\right)
v_{p}\left(  m\right)  \geq0\text{ for all }p\right) \\
&  \Longleftrightarrow\ \left(  \left(  v_{p}\left(  n\right)  =0\text{ or
}v_{p}\left(  m\right)  =0\right)  \text{ for all primes }p\right) \\
&  \Longleftrightarrow\ \left(  \left(  p\nmid n\text{ or }p\nmid m\right)
\text{ for all primes }p\right) \\
&  \Longleftrightarrow\ \left(  \text{there exists no prime }p\text{ dividing
both }n\text{ and }m\right) \\
&  \Longleftrightarrow\ \left(  n\perp m\right)  \ \ \ \ \ \ \ \ \ \ \left(
\text{by the previous proposition}\right)  .
\end{align*}

\end{proof}

\subsubsection{There are infinitely many primes}

\begin{theorem}
\label{thm.ent.primes.infin}There are infinitely many primes.
\end{theorem}

\begin{proof}
[Proof of Theorem \ref{thm.ent.primes.infin}.]The following proof is a
classic, appearing in Euclid's \textit{Elements}:

Let $\left(  p_{1},p_{2},\ldots,p_{k}\right)  $ be any finite list of primes.
We shall find a new prime distinct from each of $p_{1},p_{2},\ldots,p_{k}$.

Indeed, $p_{1},p_{2},\ldots,p_{k}$ are primes, and thus are integers $>1$ (by
the definition of a \textquotedblleft prime\textquotedblright). Hence, they
are positive integers; thus, their product $p_{1}p_{2}\cdots p_{k}$ is a
positive integer as well. Thus, $p_{1}p_{2}\cdots p_{k}>0$.

Now, let $n=p_{1}p_{2}\cdots p_{k}+1$. Then, $n=\underbrace{p_{1}p_{2}\cdots
p_{k}}_{>0}+1>1$. Hence, Proposition \ref{prop.ent.primes.ex-pri-div} shows
that there exists at least one prime $p$ such that $p\mid n$. Consider this
$p$.

We claim that $p$ is distinct from each of $p_{1},p_{2},\ldots,p_{k}$.

[\textit{Proof:} Assume the contrary. Thus, $p=p_{i}$ for some $i\in\left\{
1,2,\ldots,k\right\}  $. Consider this $i$.

We have $p_{1}p_{2}\cdots p_{k}=p_{i}\cdot\left(  p_{1}p_{2}\cdots
p_{i-1}p_{i+1}p_{i+2}\cdots p_{k}\right)  $. Thus, $p_{i}\mid p_{1}p_{2}\cdots
p_{k}$ (since $p_{1}p_{2}\cdots p_{i-1}p_{i+1}p_{i+2}\cdots p_{k}$ is an
integer). Hence, $p=p_{i}\mid p_{1}p_{2}\cdots p_{k}$. In other words,
$p_{1}p_{2}\cdots p_{k}\equiv0\operatorname{mod}p$. Now,%
\[
n=\underbrace{p_{1}p_{2}\cdots p_{k}}_{\equiv0\operatorname{mod}p_{i}}%
+1\equiv0+1=1\operatorname{mod}p_{i}.
\]
Hence, $1\equiv n\operatorname{mod}p$. But $p\mid n$ and thus $n\equiv
0\operatorname{mod}p$. Hence, $1\equiv n\equiv0\operatorname{mod}p$; in other
words, $p\mid1-0=1$. Thus, Proposition \ref{prop.ent.div.1} \textbf{(b)}
(applied to $a=p$ and $b=1$) yields $\left\vert p\right\vert \leq\left\vert
1\right\vert =1$. But $p$ is a prime; thus, $p>1>0$, so that $\left\vert
p\right\vert =p>1$. This contradicts $\left\vert p\right\vert \leq1$. This
contradiction shows that our assumption was wrong, qed.]

Thus, we have proven that $p$ is distinct from each of $p_{1},p_{2}%
,\ldots,p_{k}$. Hence, there exists a prime distinct from each of $p_{1}%
,p_{2},\ldots,p_{k}$ (namely, $p$).

Now, forget that we fixed $p_{1},p_{2},\ldots,p_{k}$. We thus have proven that
if $\left(  p_{1},p_{2},\ldots,p_{k}\right)  $ is any finite list of primes,
then there exists a prime distinct from each of $p_{1},p_{2},\ldots,p_{k}$. In
other words, given any finite list of primes, there exists at least one prime
that is not in this list. In other words, no finite list of primes can cover
all the primes. In other words, there are infinitely many primes. This proves
Theorem \ref{thm.ent.primes.infin}.
\end{proof}

Note that our proof of Theorem \ref{thm.ent.primes.infin} is constructive: It
gives an algorithm to construct arbitrarily many distinct primes. This
algorithm is not very efficient, since $p_{1}p_{2}\cdots p_{k}+1$ can be very
large even if $p_{1},p_{2},\ldots,p_{k}$ are fairly small. In practice, the
sieve of Eratosthenes is much better for generating primes.
\href{https://en.wikipedia.org/wiki/Generating_primes}{Much faster algorithms
are known}.

\subsection{Euler's totient function ($\phi$-function)}

Recall that $\mathbb{P}$ stands for the set of all positive integers.

\begin{definition}
\label{def.ent.phi.phi}We define a function $\phi:\mathbb{P}\rightarrow
\mathbb{N}$ as follows: For each $n\in\mathbb{P}$, we let $\phi\left(
n\right)  $ be the number of all $i\in\left\{  1,2,\ldots,n\right\}  $ that
are coprime to $n$. In other words,%
\begin{equation}
\phi\left(  n\right)  =\left\vert \left\{  i\in\left\{  1,2,\ldots,n\right\}
\ \mid\ i\perp n\right\}  \right\vert . \label{eq.def.ent.phi.phi.1}%
\end{equation}


This function $\phi$ is called \textit{Euler's totient function} or just
$\phi$\textit{-function}.
\end{definition}

\begin{example}
\textbf{(a)} We have $\phi\left(  12\right)  =4$, since the number of all
$i\in\left\{  1,2,\ldots,12\right\}  $ that are coprime to $12$ is $4$
(indeed, these $i$ are $1$, $5$, $7$ and $11$).

\textbf{(b)} We have $\phi\left(  13\right)  =12$, since the number of all
$i\in\left\{  1,2,\ldots,13\right\}  $ that are coprime to $13$ is $12$
(indeed, these $i$ are $1,2,\ldots,12$).

\textbf{(c)} We have $\phi\left(  14\right)  =6$, since the number of all
$i\in\left\{  1,2,\ldots,14\right\}  $ that are coprime to $14$ is $6$
(indeed, these $i$ are $1,3,5,9,11,13$).

\textbf{(d)} We have $\phi\left(  1\right)  =1$, since the number of all
$i\in\left\{  1,2,\ldots,1\right\}  $ that are coprime to $1$ is $1$ (indeed,
the only such $i$ is $1$).
\end{example}

The $\phi$-function $\phi$ is denoted by $\varphi$ by some authors.

\begin{proposition}
\label{prop.ent.phi.p}Let $p$ be a prime. Then, $\phi\left(  p\right)  =p-1$.
\end{proposition}

\begin{proof}
TODO.
\end{proof}

Proposition \ref{prop.ent.phi.p} can be generalized as follows:

\begin{exercise}
\label{exe.ent.phi.pk}Let $p$ be a prime. Let $k$ be a positive integer. Prove
that $\phi\left(  p^{k}\right)  =\left(  p-1\right)  p^{k-1}$.
\end{exercise}

\begin{proof}
[Solution to Exercise \ref{exe.ent.phi.pk}.]TODO (HW3).
\end{proof}

For example, for $n=12$, Theorem \ref{thm.ent.phi.sum-div} states that
\[
\phi\left(  1\right)  +\phi\left(  2\right)  +\phi\left(  3\right)
+\phi\left(  4\right)  +\phi\left(  6\right)  +\phi\left(  12\right)  =12.
\]


\begin{proof}
[Proof of Theorem \ref{thm.ent.phi.sum-div}.]TODO.
\end{proof}

\begin{theorem}
\label{thm.ent.phi.mult}Let $m$ and $n$ be two coprime integers. Then,
$\phi\left(  mn\right)  =\phi\left(  m\right)  \cdot\phi\left(  n\right)  $.
\end{theorem}

\begin{proof}
TODO.
\end{proof}

\begin{theorem}
\label{thm.ent.phi.explicit}Let $n$ be a positive integer. Then,%
\[
\phi\left(  n\right)  =\prod_{\substack{p\text{ prime;}\\p\mid n}}\left(
\left(  p-1\right)  p^{v_{p}\left(  n\right)  -1}\right)  =n\cdot
\prod_{\substack{p\text{ prime;}\\p\mid n}}\left(  1-\dfrac{1}{p}\right)  .
\]

\end{theorem}

We shall leave Theorem \ref{thm.ent.phi.explicit} unproven at this point: its
simplest proof relies on some abstract algebraic concepts (rings and their
direct products) that will be introduced later.

\begin{theorem}
\label{thm.ent.phi.sum-div}Let $n$ be a positive integer. Then,%
\[
\sum_{d\mid n}\phi\left(  d\right)  =n.
\]
Here and in the following, the symbol \textquotedblleft$\sum_{d\mid n}%
$\textquotedblright\ stands for \textquotedblleft sum over all
\textbf{positive} divisors $d$ of $n$\textquotedblright.
\end{theorem}

For example, for $n=12$, this theorem states that%
\[
\phi\left(  1\right)  +\phi\left(  2\right)  +\phi\left(  3\right)
+\phi\left(  4\right)  +\phi\left(  6\right)  +\phi\left(  12\right)  =12.
\]


\begin{proof}
[Proof of Theorem \ref{thm.ent.phi.sum-div}.]We have%
\begin{align*}
n  &  =\left(  \text{the number of }i\in\left\{  1,2,\ldots,n\right\}  \right)
\\
&  =\sum_{d\mid n}\left(  \text{the number of }i\in\left\{  1,2,\ldots
,n\right\}  \text{ such that }\gcd\left(  i,n\right)  =d\right)
\end{align*}
(because if $i\in\left\{  1,2,\ldots,n\right\}  $, then $\gcd\left(
i,n\right)  $ is a positive divisor of $n$). Now, we claim the following:

\begin{statement}
\textit{Claim:} For each positive divisor $d$ of $n$, we have%
\[
\left(  \text{the number of }i\in\left\{  1,2,\ldots,n\right\}  \text{ such
that }\gcd\left(  i,n\right)  =d\right)  =\phi\left(  n/d\right)  .
\]

\end{statement}

[\textit{Proof of Claim:} Recall that $\gcd\left(  sa,sb\right)  =\left\vert
s\right\vert \gcd\left(  a,b\right)  $ for all $a,b,s\in\mathbb{Z}$. We shall
use this twice below.

Let $d$ be a positive divisor of $n$. Then, there is a map%
\begin{align*}
&  \left\{  i\in\left\{  1,2,\ldots,n\right\}  \ \mid\ \gcd\left(  i,n\right)
=d\right\} \\
&  \rightarrow\left\{  j\in\left\{  1,2,\ldots,n/d\right\}  \ \mid\ j\perp
n/d\right\}
\end{align*}
given by $i\mapsto i/d$ (since $\gcd\left(  i,n\right)  =d$ implies that
$i/d\in\mathbb{Z}$ and $\gcd\left(  i/d,n/d\right)  =\underbrace{\gcd\left(
i,n\right)  }_{=d}/d=d/d=1$, so that $i/d\perp n/d$). This map is invertible
(in fact, its inverse sends each $j$ to $jd$), thus is a bijection. Hence,%
\begin{align*}
&  \left\vert \left\{  i\in\left\{  1,2,\ldots,n\right\}  \ \mid\ \gcd\left(
i,n\right)  =d\right\}  \right\vert \\
&  =\left\vert \left\{  j\in\left\{  1,2,\ldots,n/d\right\}  \ \mid\ j\perp
n/d\right\}  \right\vert \\
&  =\left\vert \left\{  i\in\left\{  1,2,\ldots,n/d\right\}  \ \mid\ i\perp
n/d\right\}  \right\vert \\
&  =\phi\left(  n/d\right)  \ \ \ \ \ \ \ \ \ \ \left(  \text{by the
definition of }\phi\right)  .
\end{align*}
This is precisely the Claim.]

Now,%
\begin{align*}
n  &  =\sum_{d\mid n}\underbrace{\left(  \text{the number of }i\in\left\{
1,2,\ldots,n\right\}  \text{ such that }\gcd\left(  i,n\right)  =d\right)
}_{\substack{=\phi\left(  n/d\right)  \\\text{(by the Claim)}}}\\
&  =\sum_{d\mid n}\phi\left(  n/d\right)  =\sum_{d\mid n}\phi\left(  d\right)
\end{align*}
(here, we have substituted $d$ for $n/d$ in the sum, since the map%
\begin{align*}
\left\{  \text{positive divisors of }n\right\}   &  \rightarrow\left\{
\text{positive divisors of }n\right\}  ,\\
d  &  \mapsto n/d
\end{align*}
is a bijection).
\end{proof}

\begin{center}
\textbf{2019-02-15 lecture}
\end{center}

\begin{exercise}
You have a corridor with $1000$ lamps, initially all off. Each lamp has a
lightswitch controlling its state.

Every night, a ghost glides through the corridor (always in the same
direction) and flips some of the switches:

On the $1$st night, the ghost flips every switch.

On the $2$nd night, the ghost flips switches $2,4,6,8,10,\ldots$.

On the $3$rd night, the ghost flips switches $3,6,9,12,15,\ldots$.

etc.

(That is: For each $k\in\left\{  1,2,\ldots,1000\right\}  $, the ghost spends
the $k$-th night flipping switches $k,2k,3k,\ldots$.)

Which lamp will be on after $1000$ nights?
\end{exercise}

\begin{proof}
[Solution.]We are asking which of the numbers $1,2,\ldots,1000$ have an odd
number of positive divisors. (Indeed, after $1000$ night, for each
$n\in\left\{  1,2,\ldots,1000\right\}  $, the $n$-th switch will have been
flipped exactly once for each positive divisor of $n$; thus, the $n$-th lamp
will be on if and only if $n$ has an odd number of positive divisors.)

\begin{statement}
\textit{Claim:} A positive integer $n$ has an odd number of positive divisors
if and only if $n$ is a perfect square.
\end{statement}

[\textit{Proof of claim:} Fix a positive integer $n$. Recall that the map%
\begin{align*}
\left\{  \text{positive divisors of }n\right\}   &  \rightarrow\left\{
\text{positive divisors of }n\right\}  ,\\
d  &  \mapsto n/d
\end{align*}
is a bijection. This groups the positive divisors $d$ of $n$ into pairs,
except for those that satisfy $d=n/d$. When $n$ is not a perfect square, there
are no such \textquotedblleft exceptional\textquotedblright\ divisors, since
$d=n/d$ means $n=d^{2}$. When $n$ is a perfect square, there is exactly one
such \textquotedblleft exceptional\textquotedblright\ divisor, namely
$\sqrt{n}$. So the number of positive divisors of $n$ is even if $n$ is not a
perfect square, and odd otherwise.]

So the lamps $1^{2},2^{2},\ldots,31^{2}$ (and no others) will be on after the
$1000$ nights.
\end{proof}

\subsection{Euler, Fermat, Wilson}

The following theorem is known as \textit{Fermat's Little Theorem} (often
abbreviated as \textquotedblleft FLT\textquotedblright):

\begin{theorem}
\label{thm.ent.fermat}Let $p$ be a prime. Let $a\in\mathbb{Z}$.

\textbf{(a)} If $p\nmid a$, then $a^{p-1}\equiv1\operatorname{mod}p$.

\textbf{(b)} We always have $a^{p}\equiv a\operatorname{mod}p$.
\end{theorem}

The word \textquotedblleft little\textquotedblright\ in the name of Theorem
\ref{thm.ent.fermat} is meant to distinguish the theorem from
\textquotedblleft Fermat's Last Theorem\textquotedblright, a much more
difficult result only proven in the 1990s. (Unfortunately, the latter result
is also abbreviated as \textquotedblleft FLT\textquotedblright.)

We will prove Theorem \ref{thm.ent.fermat} soon, by showing a more general
result (Theorem \ref{thm.ent.euler}). But before we do so, we shall convince
ourselves the parts \textbf{(a)} and \textbf{(b)} of Theorem
\ref{thm.ent.fermat} are equivalent:

\begin{remark}
Theorem \ref{thm.ent.fermat} \textbf{(b)} follows from Theorem
\ref{thm.ent.fermat} \textbf{(a)}, because:

\begin{itemize}
\item If $p\nmid a$, then Theorem \ref{thm.ent.fermat} \textbf{(a)} yields
$a^{p-1}\equiv1\operatorname{mod}p$, thus $a^{p}=a\underbrace{a^{p-1}}%
_{\equiv1\operatorname{mod}p}\equiv a1=a\operatorname{mod}p$.

\item If $p\mid a$, then both $a^{p}$ and $a$ are $\equiv0\operatorname{mod}%
p$, and therefore $a^{p}\equiv a\operatorname{mod}p$ follows.
\end{itemize}

Conversely, Theorem \ref{thm.ent.fermat} \textbf{(a)} follows from Theorem
\ref{thm.ent.fermat} \textbf{(b)}, because: Assume $p\nmid a$. Then, $p\perp
a$, so $a\perp p$. Thus, we can \textquotedblleft cancel\textquotedblright%
\ $a$ from any congruence modulo $p$ (by Lemma \ref{lem.ent.coprime.cancel}).
Doing this to the congruence $a^{p}\equiv a\operatorname{mod}p$ (which follows
from Theorem \ref{thm.ent.fermat} \textbf{(b)}), we obtain $a^{p-1}%
\equiv1\operatorname{mod}p$.
\end{remark}

The next result is known as \textit{Euler's theorem}:

\begin{theorem}
\label{thm.ent.euler}Let $n$ be a positive integer. Let $a\in\mathbb{Z}$ be
coprime to $n$.

Then, $a^{\phi\left(  n\right)  }\equiv1\operatorname{mod}n$.
\end{theorem}

Theorem \ref{thm.ent.euler} yields Theorem \ref{thm.ent.fermat} \textbf{(a)},
since $\phi\left(  p\right)  =p-1$ when $p$ is prime. Since we also know that
Theorem \ref{thm.ent.fermat} \textbf{(b)} follows from Theorem
\ref{thm.ent.fermat} \textbf{(a)}, we see that a proof of Theorem
\ref{thm.ent.euler} will immediately yield the whol\"{o}e Theorem
\ref{thm.ent.fermat}. Before we give said proof, let us show an example of how
Theorem \ref{thm.ent.euler} can be used:

\begin{exercise}
What is the last digit of $3^{4^{5}}$?

Notational remark: An expression of the form $a^{b^{c}}$ always means
$a^{\left(  b^{c}\right)  }$, not $\left(  a^{b}\right)  ^{c}=a^{bc}$.
\end{exercise}

\begin{proof}
[Solution.]The last digit of a positive integer $n$ is $n\%10$ (that is, the
remainder of $n$ upon division by $10$). So we need to work modulo $10$.

Since $3$ is coprime to $10$, we can apply Theorem \ref{thm.ent.euler} to
$n=10$ and $a=3$. We thus get $3^{\phi\left(  10\right)  }\equiv
1\operatorname{mod}10$. Since $\phi\left(  10\right)  =4$, this rewrites as
$3^{4}\equiv1\operatorname{mod}10$. Now,%
\[
3^{4^{5}}=3^{4\cdot4^{4}}=\left(  \underbrace{3^{4}}_{\equiv
1\operatorname{mod}10}\right)  ^{4^{4}}\equiv1^{4^{4}}=1\operatorname{mod}10.
\]
So the last digit of $3^{4^{5}}$ is $1$.
\end{proof}

\begin{lemma}
\label{lem.ent.euler.phi0}Let $n$ be a positive integer. Then,%
\[
\phi\left(  n\right)  =\left\vert \left\{  i\in\left\{  0,1,\ldots
,n-1\right\}  \ \mid\ i\perp n\right\}  \right\vert .
\]

\end{lemma}

\begin{proof}
\textit{First proof:} If $n=1$, then this is easily done by hand. Thus, WLOG
assume that $n>1$. Hence, neither $0$ nor $n$ is coprime to $n$ (since
$\gcd\left(  0,n\right)  =n>1$ and $\gcd\left(  n,n\right)  =n>1$). Hence,%
\begin{align*}
&  \left\{  i\in\left\{  0,1,\ldots,n-1\right\}  \ \mid\ i\perp n\right\} \\
&  =\left\{  i\in\left\{  1,2,\ldots,n\right\}  \ \mid\ i\perp n\right\}
\end{align*}
(because these sets could only differ in the elements $0$ and $n$, but none of
these two elements belongs to any of these two sets), and therefore%
\begin{align*}
&  \left\vert \left\{  i\in\left\{  0,1,\ldots,n-1\right\}  \ \mid\ i\perp
n\right\}  \right\vert \\
&  =\left\vert \left\{  i\in\left\{  1,2,\ldots,n\right\}  \ \mid\ i\perp
n\right\}  \right\vert =\phi\left(  n\right)
\end{align*}
by the definition of $\phi\left(  n\right)  $.

\textit{Second proof:} The number $0$ is coprime to $n$ if and only if $n$ is
coprime to $n$ (since $0\equiv n\operatorname{mod}n$ and thus $\gcd\left(
0,n\right)  =\gcd\left(  n,n\right)  $). Hence, if we remove $n$ from the set
$\left\{  1,2,\ldots,n\right\}  $ and add $0$ instead, the number of elements
coprime to $n$ in that set does not change. Hence,%
\begin{align*}
&  \left\vert \left\{  i\in\left\{  0,1,\ldots,n-1\right\}  \ \mid\ i\perp
n\right\}  \right\vert \\
&  =\left\vert \left\{  i\in\left\{  1,2,\ldots,n\right\}  \ \mid\ i\perp
n\right\}  \right\vert =\phi\left(  n\right)
\end{align*}
by the definition of $\phi\left(  n\right)  $.
\end{proof}

\begin{proof}
[Proof of Euler's theorem.]Let
\[
C=\left\{  i\in\left\{  0,1,\ldots,n-1\right\}  \ \mid\ i\perp n\right\}  .
\]
Then, Lemma \ref{lem.ent.euler.phi0} yields $\left\vert C\right\vert
=\phi\left(  n\right)  $.

Now, $z=\prod_{c\in C}c$. Then, $z\perp n$ (by Exercise
\ref{exe.ent.coprime.ab-to-ck}).

We have $\left(  ai\right)  \%n\in C$ for each $i\in C$.

[\textit{Proof:} Let $i\in C$. Clearly, $\left(  ai\right)  \%n\in\left\{
0,1,\ldots,n-1\right\}  $ (because it is a remainder). Also, $ai\perp n$
(since $a\perp n$ and $i\perp n$), thus $\gcd\left(  n,ai\right)  =1$, thus
$\gcd\left(  n,\left(  ai\right)  \%n\right)  =\gcd\left(  n,ai\right)  =1$,
thus $\left(  ai\right)  \%n\perp n$. Hence, $\left(  ai\right)  \%n\in C$
(since $\left(  ai\right)  \%n\in\left\{  0,1,\ldots,n-1\right\}  $), qed.]

Thus, we can define a map%
\begin{align*}
f:C  &  \rightarrow C,\\
i  &  \mapsto\left(  ai\right)  \%n.
\end{align*}


The map $f$ is injective.

[\textit{Proof:} Let $i$ and $j$ be two elements of $C$ such that $f\left(
i\right)  =f\left(  j\right)  $. We must prove that $i=j$.

We have $f\left(  i\right)  =f\left(  j\right)  $; in other words, $\left(
ai\right)  \%n=\left(  aj\right)  \%n$. Thus, $ai\equiv aj\operatorname{mod}n$
(by Exercise \ref{exe.ent.quo-rem.mod=rem}). By Lemma
\ref{lem.ent.coprime.cancel}, we can cancel $a$ from this congruence, and get
$i\equiv j\operatorname{mod}n$. Since both $i$ and $j$ lie in $\left\{
0,1,\ldots,n-1\right\}  $, this entails $i=j$. So $f$ is injective.]

Now, $C$ is finite, the map $f:C\rightarrow C$ is injective, and we have
$\left\vert C\right\vert \geq\left\vert C\right\vert $. This yields that $f$
is bijective, by the \textit{Pigeonhole Principle for Injections}:

\begin{statement}
\textit{Pigeonhole Principle for Injections}: Let $A$ and $B$ be finite sets
such that $\left\vert A\right\vert \geq\left\vert B\right\vert $. Let
$f:A\rightarrow B$ be an injective map. Then, $f$ is bijective.
\end{statement}

This Pigeonhole Principle is one of the fundamental facts of combinatorics (or
even of set theory); I will not prove it here. But since I have mentioned it,
let me also the (equally important) counterpart for surjective maps:

\begin{statement}
\textit{Pigeonhole Principle for Surjections}: Let $A$ and $B$ be finite sets
such that $\left\vert A\right\vert \leq\left\vert B\right\vert $. Let
$f:A\rightarrow B$ be an surjective map. Then, $f$ is bijective.
\end{statement}

Note that we didn't strictly \textbf{have} to apply the Pigeonhole Principle
here; we could have also proven that $f$ is surjective (using the existence of
modular inverses). I leave this alternative proof to the interested reader.

So we know that the map $f$ is bijective. Thus, we can substitute $f\left(
c\right)  $ for $c$ in the product $\prod\limits_{c\in C}c$. So%
\begin{align*}
\prod_{c\in C}c  &  =\prod_{c\in C}\underbrace{f\left(  c\right)
}_{\substack{=\left(  ac\right)  \%n\\\equiv ac\operatorname{mod}n}%
}\equiv\prod_{c\in C}\left(  ac\right)  \ \ \ \ \ \ \ \ \ \ \left(  \text{by
(\ref{eq.exe.ent.mod.k-sum.b})}\right) \\
&  =a^{\left\vert C\right\vert }\prod_{c\in C}c=a^{\phi\left(  n\right)
}\prod_{c\in C}c\operatorname{mod}n\ \ \ \ \ \ \ \ \ \ \left(  \text{since
}\left\vert C\right\vert =\phi\left(  n\right)  \right)  .
\end{align*}
In other words, $z\equiv a^{\phi\left(  n\right)  }z\operatorname{mod}n$
(since $z=\prod\limits_{c\in C}c$). Lemma \ref{lem.ent.coprime.cancel} lets us
cancel $z$ from this congruence (since $z\perp n$). We thus obtain $1\equiv
a^{\phi\left(  n\right)  }\operatorname{mod}n$. This proves Theorem
\ref{thm.ent.euler}.
\end{proof}

\begin{proof}
[Proof of Theorem \ref{thm.ent.fermat}.]As we have explained above, Theorem
\ref{thm.ent.fermat} follows from Theorem \ref{thm.ent.euler}.
\end{proof}

\begin{remark}
Let us take a look at some fractions in decimal notation:

$\dfrac{1}{7}=\allowbreak0.142857142857142857\cdots=0.\overline{142857}$
(where the $\overline{}$ bar means \textquotedblleft repeat this part over and
over\textquotedblright).

$\dfrac{1}{6}=\allowbreak0.1666666\cdots=0.1\overline{6}$.

$\dfrac{1}{2}=\allowbreak0.5000000\cdots=0.5\overline{0}$.

Any fraction $\dfrac{a}{b}$ with $a,b\in\mathbb{Z}$ has such a decimal
representation with a period. (A \textit{period} means a part that gets
repeated over and over.)

A fraction $\dfrac{a}{b}$ is called \textit{purely periodic} if its period (in
decimal notation) begins straight after the decimal point.

So $\dfrac{1}{7}$ is purely periodic, but $\dfrac{1}{6}$ and $\dfrac{1}{2}$
are not.

Which fractions are purely periodic?

Answer: A fraction $\dfrac{a}{b}$ (with $a\perp b$) is purely periodic if and
only if $b\perp10$ (in other words, $2\nmid b$ and $5\nmid b$).

This can be proven using Euler's theorem: see \cite{Conrad-Euler}.
\end{remark}

A teaser:

\begin{theorem}
(Wilson's theorem)

Let $p$ be a prime. Then, $\left(  p-1\right)  !\equiv-1\operatorname{mod}p$.
\end{theorem}

We will prove this later.

\subsection{Binomial coefficients}

Recall:

\begin{definition}
For any $n\in\mathbb{Q}$ and $k\in\mathbb{N}$, we define the \textit{binomial
coefficient} $\dbinom{n}{k}$ by
\[
\dbinom{n}{k}=\dfrac{n\left(  n-1\right)  \left(  n-2\right)  \cdots\left(
n-k+1\right)  }{k!}=\dfrac{\prod_{i=0}^{k-1}\left(  n-i\right)  }{k!}.
\]
We furthermore set $\dbinom{n}{k}=0$ for all rational $k\notin\mathbb{N}$.
\end{definition}

Recall the following properties of binomial coefficients (that you have seen
on HW0):

\begin{enumerate}
\item[\textbf{(a)}] If $n \in\mathbb{N}$ and $k \in\mathbb{N}$ are such that
$n \geq k$, then
\[
\dbinom{n}{k} = \dfrac{n!}{k! \left(  n-k \right)  !} .
\]
(This is often used as a definition of the binomial coefficients, but it is a
lousy definition, as it only covers the case when $n, k \in\mathbb{N}$ and $n
\geq k$.)

\item[\textbf{(b)}] If $n \in\mathbb{N}$ and $k \in\mathbb{Q}$ are such that
$k > n$, then
\[
\dbinom{n}{k} = 0.
\]


\item[\textbf{(c)}] If $n \in\mathbb{N}$ and $k \in\mathbb{Q}$, then
\begin{align}
\dbinom{n}{k} = \dbinom{n}{n-k} . \label{eq.exe.binom.101.c}%
\end{align}
(This is known as the \textit{symmetry of binomial coefficients}. Note that it
fails if $n \notin\mathbb{N}$.)

\item[\textbf{(d)}] Any $n \in\mathbb{Q}$ and $k \in\mathbb{Q}$ satisfy
\begin{align}
\dbinom{-n}{k} = \left(  -1 \right)  ^{k} \dbinom{k + n - 1}{k} .
\label{eq.exe.binom.101.d}%
\end{align}
(This is one of the versions of the \textit{upper negation formula}.)

\item[\textbf{(e)}] Any $n \in\mathbb{Q}$ and $k \in\mathbb{Q}$ satisfy
\begin{align}
\dbinom{n}{k} = \dbinom{n-1}{k} + \dbinom{n-1}{k-1} .
\label{eq.exe.binom.101.e}%
\end{align}
(This is the \textit{recurrence of the binomial coefficients}, and is the
reason why each entry of
\href{https://en.wikipedia.org/wiki/Pascal%27s_triangle}{Pascal's triangle} is
the sum of the two entries above it.)

\item[\textbf{(f)}] Any $n\in\mathbb{Q}$ and $k\in\mathbb{Q}$ satisfy
\[
k\dbinom{n}{k}=n\dbinom{n-1}{k-1}.
\]

\end{enumerate}

\begin{theorem}
(Combinatorial interpretation of binomial coefficients)

Let $n\in\mathbb{N}$ and $k\in\mathbb{Q}$. Let $N$ be an $n$-element set.
Then, $\dbinom{n}{k}$ is the number of $k$-element subsets of $N$.
\end{theorem}

\begin{proof}
(Proof sketch.) If $k\notin\mathbb{N}$, then this is just $0=0$. So WLOG
assume that $k\in\mathbb{N}$. Prove the theorem by induction on $n$:

\textit{Base case:} $n=0$. So $N$ is empty. Then, the number of $k$-element
subsets of $N$ is $%
\begin{cases}
1, & \text{if }k=0;\\
0, & \text{if }k\neq0.
\end{cases}
$ We have to prove that $\dbinom{n}{k}=%
\begin{cases}
1, & \text{if }k=0;\\
0, & \text{if }k\neq0.
\end{cases}
$. This is easy.

\textit{Induction step:} $n>0$. Assume we know the result for sets of size
$n-1$.

Now, let $N$ be an $n$-element set. Since $n>0$, we can pick some $a\in N$.
Now, the $k$-element subsets of $N$ can be classified into two types:

\begin{itemize}
\item the ones that don't contain $a$;

\item the ones that contain $a$.
\end{itemize}

The number of subsets of the first type is $\dbinom{n-1}{k}$ (since they are
just the $k$-element subsets of the $\left(  n-1\right)  $-element set
$N\setminus\left\{  a\right\}  $).

The number of subsets of the second type is $\dbinom{n-1}{k-1}$ (since there
is a bijection%
\begin{align*}
& \left\{  \left(  k-1\right)  \text{-element subsets of }N\setminus\left\{
a\right\}  \right\}  \\
& \rightarrow\left\{  k\text{-element subsets of }N\text{ of the second
type}\right\}
\end{align*}
sending each $S$ to $S\cup\left\{  a\right\}  $).

So the total number of $k$-element subsets of $N$ is%
\[
\dbinom{n-1}{k}+\dbinom{n-1}{k-1}=\dbinom{n}{k}\ \ \ \ \ \ \ \ \ \ \left(
\text{by recurrence relation}\right)  .
\]
So the induction step is complete.
\end{proof}

\begin{corollary}
Let $n\in\mathbb{N}$ and $k\in\mathbb{Q}$. Then, $\dbinom{n}{k}$ is a
nonnegative integer.
\end{corollary}

\begin{proposition}
Let $n\in\mathbb{Z}$ and $k\in\mathbb{Q}$. Then, $\dbinom{n}{k}$ is a integer.
\end{proposition}

\begin{proof}
We WLOG assume that $n<0$, since otherwise this follows from the Corollary we
just proved.

We WLOG assume that $k\in\mathbb{N}$, since otherwise $\dbinom{n}{k}=0$.

Now, the upper negation formula (applied to $-n$ instead of $n$) yields%
\[
\dbinom{n}{k}=\left(  -1\right)  ^{k}\underbrace{\dbinom{k-n-1}{k}%
}_{\substack{\text{a nonnegative integer}\\\text{(by Corollary, since }%
k\in\mathbb{N}\text{ and }-n-1\in\mathbb{N}\text{)}}},
\]
which is clearly an integer.
\end{proof}

The Proposition allows us to play congruence games with binomial coefficients
(and you have seen these on HW1).The study of binomial coefficients properly
belongs into Combinatorics. One of the main identities for binomial
coefficients is:

\begin{theorem}
(Vandermonde convolution) For any $x,y\in\mathbb{Q}$ and $n\in\mathbb{N}$, we
have%
\[
\dbinom{x+y}{n}=\sum_{k=0}^{n}\dbinom{x}{k}\dbinom{y}{n-k}.
\]

\end{theorem}

\begin{proof}
[Proof sketch.]\textit{Step 1:} First, let us prove the Theorem for
$x,y\in\mathbb{N}$.

Indeed, let $x,y\in\mathbb{N}$. Let $A$ be the $\left(  x+y\right)  $-element
set%
\[
\left\{  1,2,\ldots,x\right\}  \cup\left\{  -1,-2,\ldots,-y\right\}  .
\]
How many $n$-element subsets does $A$ have?

\begin{itemize}
\item On the one hand, clearly $\dbinom{x+y}{n}$ many (by the Combinatorial interpretation).

\item On the other hand, we can classify these subsets according to how many
positive elements they contain. For each $k\in\left\{  0,1,\ldots,n\right\}
$, the number of $n$-element subsets of $A$ \textbf{having exactly }%
$k$\textbf{ positive elements} is%
\[
\underbrace{\dbinom{x}{k}}_{\substack{\text{number of ways}\\\text{to pick
}k\text{ positive elements}}}\underbrace{\dbinom{y}{n-k}}%
_{\substack{\text{number of ways to}\\\text{pick }n-k\text{ negative
elements}}}.
\]
Thus, in total, the number of $n$-element subsets of $A$ is $\sum_{k=0}%
^{n}\dbinom{x}{k}\dbinom{y}{n-k}$.
\end{itemize}

Comparing these two results, we get%
\[
\dbinom{x+y}{n}=\sum_{k=0}^{n}\dbinom{x}{k}\dbinom{y}{n-k}.
\]
Thus, the Theorem is proven for $x,y\in\mathbb{N}$.

\textit{Step 2:} The \textquotedblleft polynomial identity
trick\textquotedblright\ (for now somewhat informal):

Fix $y\in\mathbb{N}$, but consider $x$ as a variable. Then, $\dbinom{x+y}{n}$
is a polynomial of degree $n$ in $x$, since%
\[
\dbinom{x+y}{n}=\dfrac{\left(  x+y\right)  \left(  x+y-1\right)  \left(
x+y-2\right)  \cdots\left(  x+y-n+1\right)  }{n!}.
\]
Similarly, $\sum_{k=0}^{n}\dbinom{x}{k}\dbinom{y}{n-k}$ is a polynomial of
degree $\leq n$ in $x$.

So we are proving an equality between two polynomials in $x$.

\begin{statement}
\textit{Fact (to be proven later):} If $P$ and $Q$ are two polynomials in $x$
(say, with real coefficients), and if infinitely many values of $x$ satisfy
$P\left(  x\right)  =Q\left(  x\right)  $, then $P=Q$ as polynomials.
\end{statement}

Set $P\left(  x\right)  =\dbinom{x+y}{n}$ and $Q\left(  x\right)  =\sum
_{k=0}^{n}\dbinom{x}{k}\dbinom{y}{n-k}$. Then, Step 1 shows that $P\left(
x\right)  =Q\left(  x\right)  $ for all $x\in\mathbb{N}$. This means that
infinitely many values of $x$ satisfy $P\left(  x\right)  =Q\left(  x\right)
$. Thus, the Fact we just stated yields $P=Q$ as polynomials. Thus, $P\left(
x\right)  =Q\left(  x\right)  $ for all $x\in\mathbb{Q}$. But this is
precisely saying that the Theorem holds for all $x\in\mathbb{Q}$ and
$y\in\mathbb{N}$.

\textit{Step 3:} How do we extend this to $x,y\in\mathbb{Q}$ ?

Proceed as in step 2, but with the roles of $x$ and $y$ swapped.

So the Theorem is proven (up to proving the Fact that we used).
\end{proof}

\begin{theorem}
Let $p$ be a prime. Let $k\in\left\{  1,2,\ldots,p-1\right\}  $. Then,
$p\mid\dbinom{p}{k}$.
\end{theorem}

\begin{proof}
[First proof.] The absorption identity (for $n=p$) yields $k\dbinom{p}%
{k}=p\underbrace{\dbinom{p-1}{k-1}}_{\text{an integer}}$. Thus, $p\mid
k\dbinom{p}{k}$. Since $p\perp k$ (because $p$ is prime and $k\in\left\{
1,2,\ldots,p-1\right\}  $), this yields $p\mid\dbinom{p}{k}$.

[Second proof.] We know that $\dbinom{p}{k}$ counts $k$-element subsets of
$\left\{  1,2,\ldots,p\right\}  $. We can restate this using $p$%
\textit{-bitstrings}.

A $p$\textit{-bitstring} is a $p$-tuple $\left(  i_{1},i_{2},\ldots
,i_{p}\right)  \in\left\{  0,1\right\}  ^{p}$. For example, the $2$-bitstrings
are $\left(  0,0\right)  ,\ \left(  0,1\right)  ,\ \left(  1,0\right)
,\ \left(  1,1\right)  $.

I claim that there is a bijection%
\begin{align*}
\left\{  k\text{-element subsets of }\left\{  1,2,\ldots,p\right\}  \right\}
& \rightarrow\left\{  p\text{-bitstrings with }k\text{ 1's}\right\}  ,\\
S  & \mapsto\left\{  \left(  i_{1},i_{2},\ldots,i_{p}\right)  \right\}  ,
\end{align*}
where $i_{x}=%
\begin{cases}
1, & \text{if }x\in S;\\
0, & \text{if }x\notin S
\end{cases}
$.

(Example: If $p=3$ and $k=2$, then

\begin{itemize}
\item the $k$-element subsets of $\left\{  1,2,\ldots,p\right\}  $ are
$\left\{  1,2\right\}  ,\ \left\{  1,3\right\}  ,\ \left\{  2,3\right\}  $;

\item the $p$-bitstrings with $k$ 1's are $\left(  1,1,0\right)  ,\ \left(
1,0,1\right)  ,\ \left(  0,1,1\right)  $.
\end{itemize}

These are listed in a way that the bijection sends them to each other in order.)

Thus,%
\begin{align*}
& \left(  \text{the number of }p\text{-bitstrings with }k\text{ 1's}\right)
\\
& =\left(  \text{the number of }k\text{-element subsets of }\left\{
1,2,\ldots,p\right\}  \right)  \\
& =\dbinom{p}{k}.
\end{align*}


A \textit{necklace} is (basically) a bitstring up to cyclic rotation.

\textit{Cyclic rotation} is the map $\mathbf{c}$ sending a $p$-bitstring
$\left(  i_{1},i_{2},\ldots,i_{p}\right)  $ to $\left(  i_{2},i_{3}%
,\ldots,i_{p},i_{1}\right)  $.

If $\left(  i_{1},i_{2},\ldots,i_{p}\right)  $ is any $p$-bitstring, then
\begin{align*}
& \left(  i_{1},i_{2},\ldots,i_{p}\right)  ,\ \ \ \ \ \ \ \ \ \ \mathbf{c}%
\left(  i_{1},i_{2},\ldots,i_{p}\right)  ,\ \ \ \ \ \ \ \ \ \ \mathbf{c}%
^{2}\left(  i_{1},i_{2},\ldots,i_{p}\right)  ,\\
& \ldots,\ \ \ \ \ \ \ \ \ \ \mathbf{c}^{p-1}\left(  i_{1},i_{2},\ldots
,i_{p}\right)
\end{align*}
are all distinct \textbf{unless} it is either $\left(  0,0,\ldots,0\right)  $
or $\left(  1,1,\ldots,1\right)  $.

(Example: $\left(  1,1,0\right)  ,\ \left(  1,0,1\right)  ,\ \left(
0,1,1\right)  $ are distinct. 

$\left(  1,0,1,0\right)  ,\ \left(  0,1,0,1\right)  ,\ \left(  1,0,1,0\right)
,\ \left(  0,1,0,1\right)  $ are \textbf{not} distinct, but $4$ is not prime.)

We don't claim that this is mathematically obvious! It will be a consequence
of things we do later (group actions on sets).

Thus, the $p$-bitstrings with $k$ 1's can be grouped into \textquotedblleft
necklaces\textquotedblright: A necklace is always a group formed by a
$p$-bitstring $\left(  i_{1},i_{2},\ldots,i_{p}\right)  $ and all its images
$\mathbf{c}^{\ell}\left(  i_{1},i_{2},\ldots,i_{p}\right)  $ under cyclic
rotation (applied several times). Each necklace contains exactly $p$ many
$p$-bitstrings. So the number of $p$-bitstrings with $k$ 1's equals $p$ times
the number of necklaces.
\end{proof}

\begin{theorem}
Let $a,b\in\mathbb{Z}$.

\textbf{(a)} We have $\dbinom{2a}{2b}\equiv\dbinom{a}{b}\operatorname{mod}2$.

\textbf{(b)} We have $\dbinom{2a+1}{2b}\equiv\dbinom{a}{b}\operatorname{mod}2$.

\textbf{(c)} We have $\dbinom{2a}{2b+1}\equiv0\operatorname{mod}2$.

\textbf{(d)} We have $\dbinom{2a+1}{2b+1}\equiv\dbinom{a}{b}\operatorname{mod}%
2$.

\textbf{(e)} (Lucas's theorem, version 1) Let $p$ be any prime. Let
$c,d\in\left\{  0,1,\ldots,p-1\right\}  $. Then,%
\[
\dbinom{pa+c}{pb+d}\equiv\dbinom{a}{b}\dbinom{c}{d}\operatorname{mod}p.
\]

\end{theorem}

\begin{theorem}
(Lucas's theorem, version 2) If $a\in\mathbb{N}$ and $b\in\mathbb{N}$ are
written in base $p$ (a prime) as follows:%
\[
a=\overline{a_{k}a_{k-1}\cdots a_{0}},\ \ \ \ \ \ \ \ \ \ b=\overline
{b_{k}b_{k-1}\cdots b_{0}},
\]
then%
\[
\dbinom{a}{b}\equiv\dbinom{a_{k}}{b_{k}}\dbinom{a_{k-1}}{b_{k-1}}\cdots
\dbinom{a_{0}}{b_{0}}\operatorname{mod}p.
\]
(Will be proven below using formal power series.)
\end{theorem}


\end{document}