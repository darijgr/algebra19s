% Like most advanced LaTeX files, this one begins with a lot of
% boilerplate. You don't need to understand (or even read) most of it.
% All you need to do is fill in your name, UMN ID, email address,
% and the number of the pset. (Search for "METADATA" to find the place
% for this.) Then, you can go straight to the "EXERCISE 1"
% section and start writing your solutions.
% The "VARIOUS USEFUL COMMANDS" section is probably worth taking a
% look at at some point.

%----------------------------------------------------------------------------------------
%	PACKAGES AND OTHER DOCUMENT CONFIGURATIONS
%----------------------------------------------------------------------------------------
\documentclass[paper=a4, fontsize=12pt]{scrartcl} % A4 paper and 12pt font size
\usepackage[T1]{fontenc} % Use 8-bit encoding that has 256 glyphs
\usepackage[english]{babel} % English language/hyphenation
\usepackage{amsmath,amsfonts,amsthm,amssymb} % Math packages
\usepackage{mathrsfs}    % More math packages
\usepackage{sectsty}  % Allows customizing section commands
\allsectionsfont{\centering \normalfont\scshape} % Make all section titles centered, the default font and small caps %remove this to left align section tites
\usepackage{hyperref} % Turns cross-references into hyperlinks,
                      % and defines \url and \href commands.
\usepackage{graphicx} % For embedding graphics files.
\usepackage{framed}   % For the "leftbar" environment used below.
\usepackage{ifthen}   % Used for the \powset command below.
\usepackage{lastpage} % for counting the number of pages
\usepackage[headsepline,footsepline,manualmark]{scrlayer-scrpage}
\usepackage[height=10in,a4paper,hmargin={1in,0.8in}]{geometry}
\usepackage[usenames,dvipsnames]{xcolor}
\usepackage{tikz}     % This is a powerful tool to draw vector
                      % graphics inside LaTeX. In particular, you can
                      % use it to draw graphs.
\usepackage{verbatim} % For the "verbatim" environment, in which
                      % special symbols can be used freely without
                      % confusing the compiler. (And it's typeset in
                      % a constant-width font.)
                      % Useful, e.g., for quoting code (or ASCII art).

%\numberwithin{table}{section} % Number tables within sections (i.e. 1.1, 1.2, 2.1, 2.2 instead of 1, 2, 3, 4)

\setlength\parindent{20pt} % Makes indentation for paragraphs longer.
                           % This makes paragraphs stand out more.

%----------------------------------------------------------------------------------------
%	VARIOUS USEFUL COMMANDS
%----------------------------------------------------------------------------------------
% The commands below might be convenient. For example, you probably
% prefer to write $\powset[2]{V}$ for the set of $2$-element subsets
% of $V$, rather than writing $\mathcal{P}_2(V)$.
% Notice that you can easily define your own commands like this.
% Caveat: Some of these commands need to be properly "guarded" when
% they occur in subscripts or superscripts. So you should not write
% $K_\CC$, but rather $K_{\CC}$.
\newcommand{\CC}{\mathbb{C}} % complex numbers
\newcommand{\RR}{\mathbb{R}} % real numbers
\newcommand{\QQ}{\mathbb{Q}} % rational numbers
\newcommand{\NN}{\mathbb{N}} % nonnegative integers
\newcommand{\PP}{\mathbb{P}} % positive integers
\newcommand{\Z}[1]{\mathbb{Z}/#1\mathbb{Z}} % integers modulo k
                                            % (syntax: "\Z{k}")
\newcommand{\ZZ}{\mathbb{Z}} % integers
\newcommand{\id}{\operatorname{id}} % identity map
\newcommand{\lcm}{\operatorname{lcm}}
% Lowest common multiple. For historical reasons, LaTeX has a \gcd
% command built in, but not an \lcm command. The preceding line
% rectifies that.
\newcommand{\set}[1]{\left\{ #1 \right\}}
% $\set{...}$ compiles to {...} (set-brackets).
\newcommand{\abs}[1]{\left| #1 \right|}
% $\abs{...}$ compiles to |...| (absolute value, or size of a set).
\newcommand{\tup}[1]{\left( #1 \right)}
% $\tup{...}$ compiles to (...) (parentheses, or tuple-brackets).
\newcommand{\ive}[1]{\left[ #1 \right]}
% $\ive{...}$ compiles to [...] (Iverson bracket, aka truth value; also, set of first n integers).
\newcommand{\floor}[1]{\left\lfloor #1 \right\rfloor}
% $\floor{...}$ compiles to |_..._| (floor function).
\newcommand{\underbrack}[2]{\underbrace{#1}_{\substack{#2}}}
% $\underbrack{...1}{...2}$ yields
% $\underbrace{...1}_{\substack{...2}}$. This is useful for doing
% local rewriting transformations on mathematical expressions with
% justifications. For example, try this out:
% $ \underbrack{(a+b)^2}{= a^2 + 2ab + b^2 \\ \text{(by the binomial formula)}} $
\newcommand{\powset}[2][]{\ifthenelse{\equal{#2}{}}{\mathcal{P}\left(#1\right)}{\mathcal{P}_{#1}\left(#2\right)}}
% $\powset[k]{S}$ stands for the set of all $k$-element subsets of
% $S$. The argument $k$ is optional, and if not provided, the result
% is the whole powerset of $S$.
\newcommand{\horrule}[1]{\rule{\linewidth}{#1}} % Create horizontal rule command with 1 argument of height
\newcommand{\nnn}{\nonumber\\} % Don't number this line in an "align" environment, and move on to the next line.

%----------------------------------------------------------------------------------------
%	MAKING SUMMATION SIGNS ALWAYS PUT THEIR BOUNDS ABOVE AND BELOW
%	THE SIGN
%----------------------------------------------------------------------------------------
% The following are hacks to ensure that sums (such as
% $\sum_{k=1}^n k$) always put their bounds (i.e., the $k=1$ and the
% $n$) underneath and above the sign, as opposed to on its right.
% Same for products (\prod), set unions (\bigcup) and set
% intersections (\bigcap). Remove the 8 lines below if you do not want
% this behavior.
\let\sumnonlimits\sum
\let\prodnonlimits\prod
\let\cupnonlimits\bigcup
\let\capnonlimits\bigcap
\renewcommand{\sum}{\sumnonlimits\limits}
\renewcommand{\prod}{\prodnonlimits\limits}
\renewcommand{\bigcup}{\cupnonlimits\limits}
\renewcommand{\bigcap}{\capnonlimits\limits}

%----------------------------------------------------------------------------------------
%	ENVIRONMENTS
%----------------------------------------------------------------------------------------
% The incantations below define how theorem environments
% (\begin{theorem} ... \end{theorem}) and their likes will look like.
\newtheoremstyle{plainsl}% <name>
  {8pt plus 2pt minus 4pt}% <Space above>
  {8pt plus 2pt minus 4pt}% <Space below>
  {\slshape}% <Body font>
  {0pt}% <Indent amount>
  {\bfseries}% <Theorem head font>
  {.}% <Punctuation after theorem head>
  {5pt plus 1pt minus 1pt}% <Space after theorem headi>
  {}% <Theorem head spec (can be left empty, meaning `normal')>

% Environments which make the text inside them slanted:
\theoremstyle{plainsl}
  \newtheorem{theorem}{Theorem}[section]
  \newtheorem{proposition}[theorem]{Proposition}
  \newtheorem{lemma}[theorem]{Lemma}
  \newtheorem{corollary}[theorem]{Corollary}
  \newtheorem{conjecture}[theorem]{Conjecture}
% Environments that don't:
\theoremstyle{definition}
  \newtheorem{definition}[theorem]{Definition}
  \newtheorem{example}[theorem]{Example}
  \newtheorem{exercise}[theorem]{Exercise}
  \newtheorem{examples}[theorem]{Examples}
  \newtheorem{algorithm}[theorem]{Algorithm}
  \newtheorem{question}[theorem]{Question}
 \theoremstyle{remark}
  \newtheorem{remark}[theorem]{Remark}
\newenvironment{statement}{\begin{quote}}{\end{quote}}
\newenvironment{fineprint}{\begin{small}}{\end{small}}

%----------------------------------------------------------------------------------------
%	METADATA
%----------------------------------------------------------------------------------------
\newcommand{\myname}{Darij Grinberg} % ENTER YOUR NAME HERE
\newcommand{\myid}{00000000} % ENTER YOUR UMN ID HERE
\newcommand{\mymail}{dgrinber@umn.edu} % ENTER YOUR EMAIL HERE
\newcommand{\psetnumber}{3} % ENTER THE NUMBER OF THIS PSET HERE

%----------------------------------------------------------------------------------------
%	HEADER AND FOOTER
%----------------------------------------------------------------------------------------
\ihead{Solutions to homework set \#\psetnumber} % Page header left
\ohead{page \thepage\ of \pageref{LastPage}} % Page header right
\ifoot{\myname, \myid} % left footer
\ofoot{\mymail} % right footer

%----------------------------------------------------------------------------------------
%	TITLE SECTION
%----------------------------------------------------------------------------------------
\title{	
\normalfont \normalsize 
\textsc{University of Minnesota, School of Mathematics} \\ [25pt] % Your university, school and/or department name(s)
\horrule{0.5pt} \\[0.4cm] % Thin top horizontal rule
\huge Math 4281: Introduction to Modern Algebra, \\
Spring 2019:
Homework \psetnumber\\% The assignment title
\horrule{2pt} \\[0.5cm] % Thick bottom horizontal rule
}
\author{\myname}

\begin{document}

\maketitle % Print the title

\begin{center} % Delete this if you want to save space!
{\large due date: \textbf{Wednesday, 20 February 2019} at the beginning of class, \\
or before that by email or canvas.

Please solve \textbf{at most 3 of the 6 exercises}!}
\end{center}

%----------------------------------------------------------------------------------------
%	EXERCISE 1
%----------------------------------------------------------------------------------------
\horrule{0.3pt} \\[0.4cm]

\section{Exercise 1: The Chinese remainder theorem for $k$ moduli}

\subsection{Problem}

Let $m_1, m_2, \ldots, m_k$ be $k$ mutually coprime integers.
Let $a_1, a_2, \ldots, a_k \in \ZZ$.

Prove the following:

\begin{enumerate}

\item[\textbf{(a)}]
There exists an integer $x$ such that
\[
\tup{ x \equiv a_i \mod m_i \qquad \text{for all } i \in \set{1, 2, \ldots, k} } .
\]

\item[\textbf{(b)}]
If $x_1$ and $x_2$ are two such integers $x$, then
$x_1 \equiv x_2 \mod m_1 m_2 \cdots m_k$.

\end{enumerate}

[\textbf{Note:} This is stated without proof in the lecture notes;
you cannot just cite that statement.]

\subsection{Solution}

[...]

%----------------------------------------------------------------------------------------
%	EXERCISE 2
%----------------------------------------------------------------------------------------
\horrule{0.3pt} \\[0.4cm]

\section{Exercise 2: More products of gcds}

\subsection{Problem}

Let $a, b, c$ be three integers.

\begin{enumerate}

\item[\textbf{(a)}]
Prove that
$\gcd\tup{a, b} \gcd\tup{a, c}
= \gcd\tup{ag, bc}$,
where $g = \gcd\tup{a, b, c}$.

\item[\textbf{(b)}]
Assume that $b \perp c$.
Prove that
$\gcd\tup{a, b} \gcd\tup{a, c} = \gcd\tup{a, bc}$.

\end{enumerate}

\subsection{Solution}

[...]

%----------------------------------------------------------------------------------------
%	EXERCISE 3
%----------------------------------------------------------------------------------------
\horrule{0.3pt} \\[0.4cm]

\section{Exercise 3: gcds and roots}

\subsection{Problem}

Prove the following:

\begin{enumerate}

\item[\textbf{(a)}]
If two integers $a$ and $b$ are not both zero, and
if $g = \gcd\tup{a, b}$, then $a/g \perp b/g$.

\item[\textbf{(b)}]
If $a$ and $b$ are two integers, then
$\gcd\tup{a^k, b^k} = \gcd\tup{a, b}^k$ for each
$k \in \NN$.

\item[\textbf{(c)}]
If $r \in \QQ$, then there exist two \textbf{coprime}
integers $a$ and $b$ satisfying $r = a/b$.

\item[\textbf{(d)}]
If a positive integer $u$ is not a perfect
square\footnote{A \textit{perfect square} means a square
of an integer.},
then $\sqrt{u}$ is irrational.

\item[\textbf{(e)}]
If $u$ and $v$ are two positive integers,
then $\sqrt{u} + \sqrt{v}$ is irrational,
unless both $u$ and $v$ are perfect squares.

\end{enumerate}

\subsection{Solution}

[...]

%----------------------------------------------------------------------------------------
%	EXERCISE 4
%----------------------------------------------------------------------------------------
\horrule{0.3pt} \\[0.4cm]

\section{Exercise 4: Basic binomial congruences}

\subsection{Problem}

Let $p$ be a prime.
Let $k \in \set{0, 1, \ldots, p-1}$.
Prove the following:

\begin{enumerate}

\item[\textbf{(a)}]
We have $k! \perp p$.

\item[\textbf{(b)}]
If $u$ and $v$ are two integers such that $u \equiv v \mod p$,
then
$\dbinom{u}{k} \equiv \dbinom{v}{k} \mod p$.

\item[\textbf{(c)}]
We have $\dbinom{p-1}{k} \equiv \tup{-1}^k \mod p$.

\end{enumerate}

\subsection{Solution}

[...]

%----------------------------------------------------------------------------------------
%	EXERCISE 5
%----------------------------------------------------------------------------------------
\horrule{0.3pt} \\[0.4cm]

\section{Exercise 5: $\phi\tup{n}$ is even}

\subsection{Problem}

Let $n \in \NN$ satisfy $n > 2$.
Recall that $\phi$ denotes the Euler totient function.
Prove that $\phi\tup{n}$ is even.

[\textbf{Hint:} Is there a way to pair up the numbers
$i \in \set{1, 2, \ldots, n}$ coprime to $n$?]

\subsection{Solution}

[...]

%----------------------------------------------------------------------------------------
%	EXERCISE 6
%----------------------------------------------------------------------------------------
\horrule{0.3pt} \\[0.4cm]

\section{Exercise 6: $\phi\tup{p^k}$}

\subsection{Problem}

Let $p$ be a prime.
Let $k$ be a positive integer.
Prove that $\phi\tup{p^k} = \tup{p-1} p^{k-1}$.

\subsection{Solution}

[...]

\begin{thebibliography}{99999999}                                                                                         %

% Feel free to add your sources -- or copy some from the source code
% of the class notes ( http://www.cip.ifi.lmu.de/~grinberg/t/19s/notes.tex ).

% This is the bibliography: The list of papers/books/articles/blogs/...
% cited. The syntax is: "\bibitem[name]{tag}Reference",
% where "name" is the name that will appear in the compiled
% bibliography, and "tag" is the tag by which you will refer to
% the source in the TeX file. For example, the following source
% has name "GrKnPa94" (so you will see it referenced as
% "[GrKnPa94]" in the compiled PDF) and tag "GKP" (so you
% can cite it by writing "\cite{GKP}").

\bibitem[GrKnPa94]{GKP}Ronald L. Graham, Donald E. Knuth, Oren Patashnik,
\textit{Concrete Mathematics, Second Edition}, Addison-Wesley 1994.\\
See \url{https://www-cs-faculty.stanford.edu/~knuth/gkp.html} for errata.

\bibitem[Grinbe19]{detnotes}Darij Grinberg,
\textit{Notes on the combinatorial fundamentals of algebra},
10 January 2019. \\
\url{http://www.cip.ifi.lmu.de/~grinberg/primes2015/sols.pdf}
\\
The numbering of theorems and formulas in this link might shift
when the project gets updated; for a ``frozen'' version whose
numbering is guaranteed to match that in the citations above, see
\url{https://github.com/darijgr/detnotes/releases/tag/2019-01-10} .

\end{thebibliography}

\end{document}

