% LaTeX solution template for Math 4281 (Owen Levin, Darij Grinberg)
% ------------------------------------------------------------------

% Like most advanced LaTeX files, this one begins with a lot of
% boilerplate. You don't need to understand (or even read) most of it.
% All you need to do is fill in your name, UMN ID, email address,
% and the number of the pset. (Search for "METADATA" to find the place
% for this.) Then, you can go straight to the "EXERCISE 1"
% section and start writing your solutions.
% The "VARIOUS USEFUL COMMANDS" section is probably worth taking a
% look at at some point.

%----------------------------------------------------------------------------------------
%	PACKAGES AND OTHER DOCUMENT CONFIGURATIONS
%----------------------------------------------------------------------------------------
\documentclass[paper=a4, fontsize=12pt]{scrartcl} % A4 paper and 12pt font size
\usepackage[T1]{fontenc} % Use 8-bit encoding that has 256 glyphs
\usepackage[english]{babel} % English language/hyphenation
\usepackage{amsmath,amsfonts,amsthm,amssymb} % Math packages
\usepackage{mathrsfs}    % More math packages
\usepackage{sectsty}  % Allows customizing section commands
\allsectionsfont{\centering \normalfont\scshape} % Make all section titles centered, the default font and small caps %remove this to left align section tites
\usepackage{hyperref} % Turns cross-references into hyperlinks,
                      % and defines \url and \href commands.
\usepackage{graphicx} % For embedding graphics files.
\usepackage{framed}   % For the "leftbar" environment used below.
\usepackage{ifthen}   % Used for the \powset command below.
\usepackage{lastpage} % for counting the number of pages
\usepackage[headsepline,footsepline,manualmark]{scrlayer-scrpage}
\usepackage[height=10in,a4paper,hmargin={1in,0.8in}]{geometry}
\usepackage[usenames,dvipsnames]{xcolor}
\usepackage{tikz}     % This is a powerful tool to draw vector
                      % graphics inside LaTeX. In particular, you can
                      % use it to draw graphs.
\usepackage{verbatim} % For the "verbatim" environment, in which
                      % special symbols can be used freely without
                      % confusing the compiler. (And it's typeset in
                      % a constant-width font.)
                      % Useful, e.g., for quoting code (or ASCII art).

%\numberwithin{table}{section} % Number tables within sections (i.e. 1.1, 1.2, 2.1, 2.2 instead of 1, 2, 3, 4)

\setlength\parindent{20pt} % Makes indentation for paragraphs longer.
                           % This makes paragraphs stand out more.

%----------------------------------------------------------------------------------------
%	VARIOUS USEFUL COMMANDS
%----------------------------------------------------------------------------------------
% The commands below might be convenient. For example, you probably
% prefer to write $\powset[2]{V}$ for the set of $2$-element subsets
% of $V$, rather than writing $\mathcal{P}_2(V)$.
% Notice that you can easily define your own commands like this.
% Caveat: Some of these commands need to be properly "guarded" when
% they occur in subscripts or superscripts. So you should not write
% $K_\CC$, but rather $K_{\CC}$.
\newcommand{\CC}{\mathbb{C}} % complex numbers
\newcommand{\RR}{\mathbb{R}} % real numbers
\newcommand{\QQ}{\mathbb{Q}} % rational numbers
\newcommand{\NN}{\mathbb{N}} % nonnegative integers
\newcommand{\PP}{\mathbb{P}} % positive integers
\newcommand{\Z}[1]{\mathbb{Z}/#1\mathbb{Z}} % integers modulo k
                                            % (syntax: "\Z{k}")
\newcommand{\ZZ}{\mathbb{Z}} % integers
\newcommand{\id}{\operatorname{id}} % identity map
\newcommand{\lcm}{\operatorname{lcm}}
% Lowest common multiple. For historical reasons, LaTeX has a \gcd
% command built in, but not an \lcm command. The preceding line
% rectifies that.
\newcommand{\set}[1]{\left\{ #1 \right\}}
% $\set{...}$ compiles to {...} (set-brackets).
\newcommand{\abs}[1]{\left| #1 \right|}
% $\abs{...}$ compiles to |...| (absolute value, or size of a set).
\newcommand{\tup}[1]{\left( #1 \right)}
% $\tup{...}$ compiles to (...) (parentheses, or tuple-brackets).
\newcommand{\ive}[1]{\left[ #1 \right]}
% $\ive{...}$ compiles to [...] (Iverson bracket, aka truth value; also, set of first n integers).
\newcommand{\floor}[1]{\left\lfloor #1 \right\rfloor}
% $\floor{...}$ compiles to |_..._| (floor function).
\newcommand{\underbrack}[2]{\underbrace{#1}_{\substack{#2}}}
% $\underbrack{...1}{...2}$ yields
% $\underbrace{...1}_{\substack{...2}}$. This is useful for doing
% local rewriting transformations on mathematical expressions with
% justifications. For example, try this out:
% $ \underbrack{(a+b)^2}{= a^2 + 2ab + b^2 \\ \text{(by the binomial formula)}} $
\newcommand{\powset}[2][]{\ifthenelse{\equal{#2}{}}{\mathcal{P}\left(#1\right)}{\mathcal{P}_{#1}\left(#2\right)}}
% $\powset[k]{S}$ stands for the set of all $k$-element subsets of
% $S$. The argument $k$ is optional, and if not provided, the result
% is the whole powerset of $S$.
\newcommand{\horrule}[1]{\rule{\linewidth}{#1}} % Create horizontal rule command with 1 argument of height
\newcommand{\nnn}{\nonumber\\} % Don't number this line in an "align" environment, and move on to the next line.

%----------------------------------------------------------------------------------------
%	MAKING SUMMATION SIGNS ALWAYS PUT THEIR BOUNDS ABOVE AND BELOW
%	THE SIGN
%----------------------------------------------------------------------------------------
% The following are hacks to ensure that sums (such as
% $\sum_{k=1}^n k$) always put their bounds (i.e., the $k=1$ and the
% $n$) underneath and above the sign, as opposed to on its right.
% Same for products (\prod), set unions (\bigcup) and set
% intersections (\bigcap). Remove the 8 lines below if you do not want
% this behavior.
\let\sumnonlimits\sum
\let\prodnonlimits\prod
\let\cupnonlimits\bigcup
\let\capnonlimits\bigcap
\renewcommand{\sum}{\sumnonlimits\limits}
\renewcommand{\prod}{\prodnonlimits\limits}
\renewcommand{\bigcup}{\cupnonlimits\limits}
\renewcommand{\bigcap}{\capnonlimits\limits}

%----------------------------------------------------------------------------------------
%	ENVIRONMENTS
%----------------------------------------------------------------------------------------
% The incantations below define how theorem environments
% (\begin{theorem} ... \end{theorem}) and their likes will look like.
\newtheoremstyle{plainsl}% <name>
  {8pt plus 2pt minus 4pt}% <Space above>
  {8pt plus 2pt minus 4pt}% <Space below>
  {\slshape}% <Body font>
  {0pt}% <Indent amount>
  {\bfseries}% <Theorem head font>
  {.}% <Punctuation after theorem head>
  {5pt plus 1pt minus 1pt}% <Space after theorem headi>
  {}% <Theorem head spec (can be left empty, meaning `normal')>

% Environments which make the text inside them slanted:
\theoremstyle{plainsl}
  \newtheorem{theorem}{Theorem}[section]
  \newtheorem{proposition}[theorem]{Proposition}
  \newtheorem{lemma}[theorem]{Lemma}
  \newtheorem{corollary}[theorem]{Corollary}
  \newtheorem{conjecture}[theorem]{Conjecture}
% Environments that don't:
\theoremstyle{definition}
  \newtheorem{definition}[theorem]{Definition}
  \newtheorem{example}[theorem]{Example}
  \newtheorem{exercise}[theorem]{Exercise}
  \newtheorem{examples}[theorem]{Examples}
  \newtheorem{algorithm}[theorem]{Algorithm}
  \newtheorem{question}[theorem]{Question}
 \theoremstyle{remark}
  \newtheorem{remark}[theorem]{Remark}
\newenvironment{statement}{\begin{quote}}{\end{quote}}
\newenvironment{fineprint}{\begin{small}}{\end{small}}

%----------------------------------------------------------------------------------------
%	METADATA
%----------------------------------------------------------------------------------------
\newcommand{\myname}{Darij Grinberg} % ENTER YOUR NAME HERE
\newcommand{\myid}{00000000} % ENTER YOUR UMN ID HERE
\newcommand{\mymail}{dgrinber@umn.edu} % ENTER YOUR EMAIL HERE
\newcommand{\psetnumber}{0} % ENTER THE NUMBER OF THIS PSET HERE

%----------------------------------------------------------------------------------------
%	HEADER AND FOOTER
%----------------------------------------------------------------------------------------
\ihead{Solutions to homework set \#\psetnumber} % Page header left
\ohead{page \thepage\ of \pageref{LastPage}} % Page header right
\ifoot{\myname, \myid} % left footer
\ofoot{\mymail} % right footer

%----------------------------------------------------------------------------------------
%	TITLE SECTION
%----------------------------------------------------------------------------------------
\title{	
\normalfont \normalsize 
\textsc{University of Minnesota, School of Mathematics} \\ [25pt] % Your university, school and/or department name(s)
\horrule{0.5pt} \\[0.4cm] % Thin top horizontal rule
\huge Math 4281: Introduction to Modern Algebra, \\
Spring 2019:
Homework \psetnumber\\% The assignment title
\horrule{2pt} \\[0.5cm] % Thick bottom horizontal rule
}
\author{\myname}

\begin{document}

\maketitle % Print the title

%----------------------------------------------------------------------------------------
%	EXERCISE 1
%----------------------------------------------------------------------------------------
\horrule{0.3pt} \\[0.4cm]

\section{Exercise 1: Geometric series and a bit more}

\subsection{Problem}

Let $x \in \QQ$.
% "\QQ" is an abbreviation for "\mathbb{Q}", and stands for the rational numbers.
Prove that the equalities
\begin{align}
\tup{1-x} \sum_{k=0}^n x^k &= 1 - x^{n+1}
% Several things are happening here.
% 1. The "\begin{align}" and "\end{align}" commands delimit an "align"
%    environment, which is a way to write several equations one under
%    the other while aligning them at the equality sign (or at any
%    other places).
% 2. The "\tup{1-x}" yields "(1-x)" when compiled. You could achieve
%    the same effect by just writing "(1-x)". The difference is mostly
%    aesthetical: If you want something bigger than "1-x" inside the
%    parentheses -- say, a fraction or a summation sign --, then
%    writing "\tup{...}" will automatically stretch the parentheses to
%    the size of whatever you put inside them, while "(...)" will
%    just generate two normal-size parentheses. You don't have toß
%    follow my stylistic choice.
% 3. The "\sum_{k=0}^n" gives a summation sign, representing a sum
%    from k=0 to n. As usual, what comes after the "_" is a subscript
%    (so it is put under the summation sign), and what comes after the
%    "^" is a superscript (so it is put above the summation sign).
%    You have to use {} braces if your subscript or superscript is
%    longer than one symbol; this is why the "k=0" is inside such
%    braces while the "n" is not. (Of course, you *can* put the "n" in
%    such braces too, if you wish; it just doesn't change anything.
%    But if you forget the braces around the "k=0", you get a mess:
%    Only the first letter "k" ends up in the subscript.)
% 4. The "&" symbol means "align the equations here".
\label{eq.exe.geo-series.1}
% The "\label" command lets you give an equation a label by which
% you can later refer to it. So this equation is now labelled
% "eq.exe.geo-series.1", and you can refer to it using
% "\eqref{eq.exe.geo-series.1}". Note that the labels do *not* get
% printed in the compiled PDF; they just become "(1)", "(2)" etc.
\\
% An "\\" means "new line". You can do this in text, too, not just
% in align environments.
\tup{1-x}^2 \sum_{k=0}^n k x^k
&= x \tup{ n x^{n+1} - \tup{n+1} x^n + 1 }
% LaTeX ignores single linebreaks in the sourcecode (or, rather,
% treats them just as whitespaces). So you can write 
\label{eq.exe.geo-series.2}
\end{align}
hold for each $n \in \NN$.
% "\NN" is an abbreviation for "\mathbb{N}", and stands for the nonnegative integers.

(Here and in the following, $\NN$ stands for the set $\set{0, 1, 2, \ldots}$.)
% The "\set{...}" command puts the "..." in set-braces.
% The "\ldots" command gives an ellipsis that looks a bit better than what you
% would get by just writing "...". It's mostly a matter of taste,
% and I don't mind if you just write "...".

\subsection{Remark}

It is more common to see the formulas
\eqref{eq.exe.geo-series.1} and \eqref{eq.exe.geo-series.2}
restated as
\begin{align*}
% The "\begin{align*}" and "\end{align*}" commands do the same
% as "\begin{align}" and "\end{align}", except that they don't
% number the equations.
% Also, I'm using them here for just a single line (so I'm not
% aligning anything). This is perfectly fine.
\sum_{k=0}^n x^k = \dfrac{1 - x^{n+1}}{1 - x}
% "\dfrac{a}{b}" gives the "a over b" fraction.
% Actually, "\frac{a}{b}" does the same thing; you will only
% see the difference if you use them in text (in which case
% the "\frac{a}{b}" fraction will be a lot smaller than the
% "\dfrac{a}{b}" one). Again, it is a matter of taste.
\qquad
% "\qquad" produces an amount of white space, similar to a
% "tab" in text editors. I usually put it between different
% equations on a single line.
\text{and} \qquad
% "\text{...}" allows you to write regular (non-italicized) text
% inside a maths environment. For example, "\text{and}" will
% make the "and" look like regular text.
\sum_{k=0}^n k x^k
= \dfrac{x \tup{ n x^{n+1} - \tup{n+1} x^n + 1 }}{\tup{1 - x}^2} .
\end{align*}
But this restatement only makes sense when $x \neq 1$.
More generally, the formulas
\eqref{eq.exe.geo-series.1} and \eqref{eq.exe.geo-series.2}
remain true when $x$ is an element of an arbitrary ring
(we will later learn what this means; for now, let us
just say that, e.g., we could let $x$ be a square matrix
instead of a rational numbers), whereas the
restatements only make sense when $1 - x$ is invertible.
So the formulas
\eqref{eq.exe.geo-series.1} and \eqref{eq.exe.geo-series.2}
are more general.

\subsection{Solution}

% The following solution is wordy and tries to be pedagogical.
% You don't need all that detail and all that exposition.
% See the solutions to the other two exercises for examples
% of more concise writing.

XXX

%----------------------------------------------------------------------------------------
%	EXERCISE 2
%----------------------------------------------------------------------------------------
\horrule{0.3pt} \\[0.4cm]

\section{Exercise 2}

\subsection{Problem}

\begin{enumerate} % This creates a list.

\item[\textbf{(a)}]
% "\item[\textbf{(a)}]" starts the first item of the list, and names
% (or numbers) it by a boldfaced "(a)".
XXX

\item[\textbf{(b)}]
A \textit{composition} of $n$ shall mean a list
\footnote{``List'' means the same as ``tuple'';
          lists are always ordered and finite.}
% "\footnote" does what you would expect it to do.
% Caveat: You can't put a \footnote inside a mathematical
% expression or formula. (But you shouldn't anyway -- it would
% look like an exponent.)
$\tup{i_1, i_2, \ldots, i_k}$
of positive integers such that $i_1 + i_2 + \cdots + i_k = n$.
% The "\tup{...}" command puts the "..." in parentheses.
% The "\cdots" is the "right" way to get a vertically centered ellipsis.
(For example, the compositions of $3$ are
$\tup{3}$, $\tup{1, 2}$, $\tup{2, 1}$ and $\tup{1, 1, 1}$.)

Show that the number of all compositions of $n$ is
\[  % These "\[" and "\]" brackets are one way to delimit a formula.
\begin{cases}
2^{n-1}, & \text{ if } n > 0 ; \\
1,       & \text{ if } n = 0 .
\end{cases}
\]
% Note that we had to write "2^{n-1}", not "2^n-1", in order to get
% the whole of "n-1" into the exponents, rather than just the "n".
% For example, 2^{3-1} = 2^2 = 4 whereas 2^3-1 = 8-1 = 7.
% 
% "\text{...}" is a way to write text inside a mathematical expression.
% 
% The "&" character vertically aligns symbols on different rows.
% (Here, we have used it in the "cases" environment; it works
% similarly in "align" environments. In tables/matrices, it
% separates columns from each other.

\end{enumerate} % This ends the list.
XXX

\subsection{Remark}

XXX

\subsection{Solution}

XXX

XXX
% This is a "proof" environment. In general
% "\begin{proof}[...]" begins a proof; the "..." is written
% in italics at the beginning of the proof.
% If you just write "\begin{proof}" without the "[...]",
% then it just says "Proof.".
XXX
% The "\end{proof}" ends the "proof" environment and automatically
% puts a black square at its end.

XXX

%----------------------------------------------------------------------------------------
%	EXERCISE 3
%----------------------------------------------------------------------------------------
\horrule{0.3pt} \\[0.4cm]

\section{Exercise 3}

\subsection{Problem}

XXX

\subsection{Solution}

XXX

\begin{thebibliography}{99999999}                                                                                         %

% This is the bibliography: The list of papers/books/articles/blogs/...
% cited. The syntax is: "\bibitem[name]{tag}Reference",
% where "name" is the name that will appear in the compiled
% bibliography, and "tag" is the tag by which you will refer to
% the source in the TeX file. For example, the following source
% has name "Grinbe16" (so you will see it referenced as
% "[Grinbe16]" in the compiled PDF) and tag "detnotes" (so you
% can cite it by writing "\cite{detnotes}").

\bibitem[Grinbe16]{detnotes}Darij Grinberg,
\textit{Notes on the combinatorial fundamentals of algebra},
10 January 2019. \\
\url{http://www.cip.ifi.lmu.de/~grinberg/primes2015/sols.pdf}
\\
The numbering of theorems and formulas in this link might shift
when the project gets updated; for a ``frozen'' version whose
numbering is guaranteed to match that in the citations above, see
\url{https://github.com/darijgr/detnotes/releases/tag/2019-01-10} .

\end{thebibliography}

\end{document}

% You can use the space after "\end{document}" as scratch paper --
% LaTeX stops compiling after it sees the "\end{document}"
% instruction, so everything that comes after it is ignored.
% For example, the following nonsense doesn't appear anywhere
% in the PDF file:

$aaaaaaaa$
