% Like most advanced LaTeX files, this one begins with a lot of
% boilerplate. You don't need to understand (or even read) most of it.
% All you need to do is fill in your name, UMN ID, email address,
% and the number of the pset. (Search for "METADATA" to find the place
% for this.) Then, you can go straight to the "EXERCISE 1"
% section and start writing your solutions.
% The "VARIOUS USEFUL COMMANDS" section is probably worth taking a
% look at at some point.

%----------------------------------------------------------------------------------------
%	PACKAGES AND OTHER DOCUMENT CONFIGURATIONS
%----------------------------------------------------------------------------------------
\documentclass[paper=a4, fontsize=12pt]{scrartcl} % A4 paper and 12pt font size
\usepackage[T1]{fontenc} % Use 8-bit encoding that has 256 glyphs
\usepackage[english]{babel} % English language/hyphenation
\usepackage{amsmath,amsfonts,amsthm,amssymb} % Math packages
\usepackage{mathrsfs}    % More math packages
\usepackage{sectsty}  % Allows customizing section commands
\allsectionsfont{\centering \normalfont\scshape} % Make all section titles centered, the default font and small caps %remove this to left align section tites
\usepackage{hyperref} % Turns cross-references into hyperlinks,
                      % and defines \url and \href commands.
\usepackage{graphicx} % For embedding graphics files.
\usepackage{framed}   % For the "leftbar" environment used below.
\usepackage{ifthen}   % Used for the \powset command below.
\usepackage{lastpage} % for counting the number of pages
\usepackage[headsepline,footsepline,manualmark]{scrlayer-scrpage}
\usepackage[height=10in,a4paper,hmargin={1in,0.8in}]{geometry}
\usepackage[usenames,dvipsnames]{xcolor}
\usepackage{tikz}     % This is a powerful tool to draw vector
                      % graphics inside LaTeX. In particular, you can
                      % use it to draw graphs.
\usepackage{verbatim} % For the "verbatim" environment, in which
                      % special symbols can be used freely without
                      % confusing the compiler. (And it's typeset in
                      % a constant-width font.)
                      % Useful, e.g., for quoting code (or ASCII art).

%\numberwithin{table}{section} % Number tables within sections (i.e. 1.1, 1.2, 2.1, 2.2 instead of 1, 2, 3, 4)

\setlength\parindent{20pt} % Makes indentation for paragraphs longer.
                           % This makes paragraphs stand out more.

%----------------------------------------------------------------------------------------
%	VARIOUS USEFUL COMMANDS
%----------------------------------------------------------------------------------------
% The commands below might be convenient. For example, you probably
% prefer to write $\powset[2]{V}$ for the set of $2$-element subsets
% of $V$, rather than writing $\mathcal{P}_2(V)$.
% Notice that you can easily define your own commands like this.
% Caveat: Some of these commands need to be properly "guarded" when
% they occur in subscripts or superscripts. So you should not write
% $K_\CC$, but rather $K_{\CC}$.
\newcommand{\CC}{\mathbb{C}} % complex numbers
\newcommand{\RR}{\mathbb{R}} % real numbers
\newcommand{\QQ}{\mathbb{Q}} % rational numbers
\newcommand{\NN}{\mathbb{N}} % nonnegative integers
\newcommand{\DD}{{\mathbb{D}}} % dual numbers
\newcommand{\PP}{\mathbb{P}} % positive integers
\newcommand{\KK}{\mathbb{K}} % notation for a ring
\newcommand{\LL}{\mathbb{L}} % notation for a ring
\newcommand{\MM}{\mathbb{M}} % notation for a ring
\newcommand{\FF}{\mathbb{F}} % notation for a field
\newcommand{\Z}[1]{\mathbb{Z}/#1\mathbb{Z}} % integers modulo k
                                            % (syntax: "\Z{k}")
\newcommand{\ZZ}{\mathbb{Z}} % integers
\newcommand{\id}{\operatorname{id}} % identity map
\newcommand{\op}{\operatorname{op}} % opposite ring
\newcommand{\End}{\operatorname{End}} % endomorphism ring
\newcommand{\lcm}{\operatorname{lcm}}
% Lowest common multiple. For historical reasons, LaTeX has a \gcd
% command built in, but not an \lcm command. The preceding line
% rectifies that.
\newcommand{\Int}{\operatorname{Int}} % integer-valued polynomials
\newcommand{\set}[1]{\left\{ #1 \right\}}
% $\set{...}$ compiles to {...} (set-brackets).
\newcommand{\abs}[1]{\left| #1 \right|}
% $\abs{...}$ compiles to |...| (absolute value, or size of a set).
\newcommand{\tup}[1]{\left( #1 \right)}
% $\tup{...}$ compiles to (...) (parentheses, or tuple-brackets).
\newcommand{\ive}[1]{\left[ #1 \right]}
% $\ive{...}$ compiles to [...] (Iverson bracket, aka truth value; also, set of first n integers; also, polynomial).
\newcommand{\ivee}[1]{\left[ \left[ #1 \right] \right]}
% $\ivee{...}$ compiles to [[...]] (formal power series).
\newcommand{\floor}[1]{\left\lfloor #1 \right\rfloor}
% $\floor{...}$ compiles to |_..._| (floor function).
\newcommand{\underbrack}[2]{\underbrace{#1}_{\substack{#2}}}
% $\underbrack{...1}{...2}$ yields
% $\underbrace{...1}_{\substack{...2}}$. This is useful for doing
% local rewriting transformations on mathematical expressions with
% justifications. For example, try this out:
% $ \underbrack{(a+b)^2}{= a^2 + 2ab + b^2 \\ \text{(by the binomial formula)}} $
\newcommand{\powset}[2][]{\ifthenelse{\equal{#2}{}}{\mathcal{P}\left(#1\right)}{\mathcal{P}_{#1}\left(#2\right)}}
% $\powset[k]{S}$ stands for the set of all $k$-element subsets of
% $S$. The argument $k$ is optional, and if not provided, the result
% is the whole powerset of $S$.
\newcommand{\calF}{\mathcal{F}}
\newcommand{\horrule}[1]{\rule{\linewidth}{#1}} % Create horizontal rule command with 1 argument of height
\newcommand{\nnn}{\nonumber\\} % Don't number this line in an "align" environment, and move on to the next line.

%----------------------------------------------------------------------------------------
%	MAKING SUMMATION SIGNS ALWAYS PUT THEIR BOUNDS ABOVE AND BELOW
%	THE SIGN
%----------------------------------------------------------------------------------------
% The following are hacks to ensure that sums (such as
% $\sum_{k=1}^n k$) always put their bounds (i.e., the $k=1$ and the
% $n$) underneath and above the sign, as opposed to on its right.
% Same for products (\prod), set unions (\bigcup) and set
% intersections (\bigcap). Remove the 8 lines below if you do not want
% this behavior.
\let\sumnonlimits\sum
\let\prodnonlimits\prod
\let\cupnonlimits\bigcup
\let\capnonlimits\bigcap
\renewcommand{\sum}{\sumnonlimits\limits}
\renewcommand{\prod}{\prodnonlimits\limits}
\renewcommand{\bigcup}{\cupnonlimits\limits}
\renewcommand{\bigcap}{\capnonlimits\limits}

%----------------------------------------------------------------------------------------
%	ENVIRONMENTS
%----------------------------------------------------------------------------------------
% The incantations below define how theorem environments
% (\begin{theorem} ... \end{theorem}) and their likes will look like.
\newtheoremstyle{plainsl}% <name>
  {8pt plus 2pt minus 4pt}% <Space above>
  {8pt plus 2pt minus 4pt}% <Space below>
  {\slshape}% <Body font>
  {0pt}% <Indent amount>
  {\bfseries}% <Theorem head font>
  {.}% <Punctuation after theorem head>
  {5pt plus 1pt minus 1pt}% <Space after theorem headi>
  {}% <Theorem head spec (can be left empty, meaning `normal')>

% Environments which make the text inside them slanted:
\theoremstyle{plainsl}
  \newtheorem{theorem}{Theorem}[section]
  \newtheorem{proposition}[theorem]{Proposition}
  \newtheorem{lemma}[theorem]{Lemma}
  \newtheorem{corollary}[theorem]{Corollary}
  \newtheorem{conjecture}[theorem]{Conjecture}
% Environments that don't:
\theoremstyle{definition}
  \newtheorem{definition}[theorem]{Definition}
  \newtheorem{example}[theorem]{Example}
  \newtheorem{exercise}[theorem]{Exercise}
  \newtheorem{examples}[theorem]{Examples}
  \newtheorem{algorithm}[theorem]{Algorithm}
  \newtheorem{question}[theorem]{Question}
 \theoremstyle{remark}
  \newtheorem{remark}[theorem]{Remark}
\newenvironment{statement}{\begin{quote}}{\end{quote}}
\newenvironment{fineprint}{\begin{small}}{\end{small}}

%----------------------------------------------------------------------------------------
%	METADATA
%----------------------------------------------------------------------------------------
\newcommand{\myname}{Darij Grinberg} % ENTER YOUR NAME HERE
\newcommand{\myid}{00000000} % ENTER YOUR UMN ID HERE
\newcommand{\mymail}{dgrinber@umn.edu} % ENTER YOUR EMAIL HERE
\newcommand{\psetnumber}{3} % ENTER THE NUMBER OF THIS PSET HERE

%----------------------------------------------------------------------------------------
%	HEADER AND FOOTER
%----------------------------------------------------------------------------------------
\ihead{Solutions to midterm \#\psetnumber} % Page header left
\ohead{page \thepage\ of \pageref{LastPage}} % Page header right
\ifoot{\myname, \myid} % left footer
\ofoot{\mymail} % right footer

%----------------------------------------------------------------------------------------
%	TITLE SECTION
%----------------------------------------------------------------------------------------
\title{	
\normalfont \normalsize 
\textsc{University of Minnesota, School of Mathematics} \\ [25pt] % Your university, school and/or department name(s)
\horrule{0.5pt} \\[0.4cm] % Thin top horizontal rule
\huge Math 4281: Introduction to Modern Algebra, \\
Spring 2019:
Midterm \psetnumber\\% The assignment title
\horrule{2pt} \\[0.5cm] % Thick bottom horizontal rule
}
\author{\myname}

\begin{document}

\maketitle % Print the title

\begin{center} % Delete this if you want to save space!
{\large due date: \textbf{Monday, 6 May 2019 at 20:00} on Canvas or by email. \\

\textbf{No collaboration allowed} -- this is a midterm.

Please solve \textbf{at most 3 of the 6 exercises}!}
\end{center}

%----------------------------------------------------------------------------------------
%	EXERCISE 1
%----------------------------------------------------------------------------------------
\horrule{0.3pt} \\[0.4cm]

\section{Exercise 1: Nonunital rings and local unities}

\subsection{Problem}

A \textit{nonunital ring} is defined in the same way as we defined a ring,
except that we don't require it to be endowed with an element $1$ (and,
correspondingly, we omit the ``Neutrality of one'' axiom).
This does not mean that a nonunital ring must not contain an element $1$
that would satisfy the ``Neutrality of one'' axiom; it simply means that
such an element is not required (and not considered part of the ring
structure).
So, formally speaking, a nonunital ring is a
$4$-tuple $\tup{\KK, +, \cdot, 0}$
(while a ring in the usual sense is a $5$-tuple
$\tup{\KK, +, \cdot, 0, 1}$)
that satisfies all the ring axioms except for ``Neutrality of one''.

Thus, every ring becomes a nonunital ring if we forget its unity (i.e., if
$\tup{\KK, +, \cdot, 0, 1}$ is a ring, then $\tup{\KK, +, \cdot, 0}$
is a nonunital ring).
But there are other examples as well:
For instance, if $n \in \ZZ$ is arbitrary, then
$n \ZZ := \set{ nz \mid z \in \ZZ }
= \set{ \text{all multiples of } n }$ is a nonunital ring
(when endowed with the usual $+$, $\cdot$ and $0$).

An element $z$ of a nonunital ring $\KK$ is said to be a \textit{unity}
of $\KK$ if every $a \in \KK$ satisfies $az = za = a$.
In other words, an element $z$ of a nonunital ring $\KK$ is said
to be a \textit{unity} of $\KK$ if equipping $\KK$ with the unity
$z$ results in a ring (in the usual sense of this word).

Prove the following:

\begin{enumerate}

\item[\textbf{(a)}] If $n \in \ZZ$, then the nonunital ring $n \ZZ$
has a unity if and only if $n \in \set{1, 0, -1}$.

\item[\textbf{(b)}] Any nonunital ring has \textbf{at most one} unity.

\end{enumerate}

Now, let $\KK$ be a nonunital ring.
As usual, we write $+$ and $\cdot$ for its two operations,
and $0$ for its zero.

Let $z \in \KK$. Define a subset $U_z$ of $\KK$ by
\[
U_z = \set{ r \in \KK \mid rz = zr = r }.
\]

\begin{enumerate}

\item[\textbf{(c)}] Prove that $0 \in U_z$, and that every
$a, b \in U_z$ satisfy $a + b \in U_z$ and $ab \in U_z$.

\end{enumerate}

Thus, we can turn $U_z$ into a nonunital ring by endowing $U_z$
with the binary operations $+$ and $\cdot$ (inherited from $\KK$) and
the element $0$. Consider this nonunital ring $U_z$.

\begin{enumerate}

\item[\textbf{(d)}]
Assume that $z^2 = z$.
Prove that $z$ is a unity of the nonunital ring $U_z$.

\end{enumerate}

[\textbf{Hint:} In \textbf{(b)}, what would the product of two unities be?]

\subsection{Solution}

[...]

%----------------------------------------------------------------------------------------
%	EXERCISE 2
%----------------------------------------------------------------------------------------
\horrule{0.3pt} \\[0.4cm]

\section{Exercise 2: Rings from nonunital rings}

\subsection{Problem}

Let $\KK$ be a nonunital ring.
(See Exercise 1 for the definition of this notion.)
Let $\LL$ be the Cartesian product $\ZZ \times \KK$ (so far, just a set).
Define a binary operation $+$ on $\LL$ by setting
\[
\tup{n, a} + \tup{m, b} = \tup{n+m, a+b}
\qquad
\text{for all } \tup{n, a}, \tup{m, b} \in \LL .
\]
(This is an entrywise addition.)
Define a binary operation $\cdot$ on $\LL$ by
\[
\tup{n, a} \tup{m, b} = \tup{nm, nb + ma + ab}
\qquad
\text{for all } \tup{n, a}, \tup{m, b} \in \LL .
\]
(Here, $nb$ and $ma$ are defined in the usual way:
If $n \in \ZZ$ and $a \in \KK$, then $na \in \KK$ is defined by
\[
na=
\begin{cases}
\underbrace{a+a+\cdots+a}_{n \text{ times}},           & \text{if } n \geq 0;\\
- \tup{\underbrace{a+a+\cdots+a}_{-n \text{ times}}} , & \text{if } n < 0
\end{cases}
.
\]
This does not require $\KK$ to have a unity.)

Prove that $\LL$, endowed with these two operations $+$ and $\cdot$ and
the zero $\tup{0, 0}$ and the unity $\tup{1, 0}$, is a ring
(in the usual sense of this word).

[\textbf{Hint:} You can use rules like $n \tup{a + b} = na + nb$ and
$\tup{n + m} a = na + ma$ and $\tup{nm} a = n \tup{ma}$
(for $n, m \in \ZZ$ and $a, b \in \KK$) without proof; they can be
proven just as for usual rings.
You can also use the fact that finite sums of elements of $\KK$ are
well-defined and behave as we would expect them to
(we already tacitly used that in writing
``$\underbrace{a+a+\cdots+a}_{n \text{ times}}$'' without parentheses).

You don't need to check the ``additive'' axioms (associativity of
addition, commutativity of addition, neutrality of zero, and
existence of additive inverses); as far as addition and zero are
concerned, $\LL$ is just a Cartesian product.]

\subsection{Remark}

This exercise gives a way to ``embed'' any nonunital ring $\KK$
into a ring $\LL$.
This helps proving properties of nonunital rings, assuming that
you can prove them for rings.

There is also a much simpler notion of a Cartesian product of two nonunital
rings (in which both addition and multiplication are defined entrywise). This
lets us define a nonunital ring $\ZZ \times \KK$. But this is
\textbf{not} the ring $\LL$; it does not generally have a unity.

\subsection{Solution}

[...]

%----------------------------------------------------------------------------------------
%	EXERCISE 3
%----------------------------------------------------------------------------------------
\horrule{0.3pt} \\[0.4cm]

\section{Exercise 3: More sums from number theory}

\subsection{Problem}

\begin{enumerate}

\item[\textbf{(a)}]
Let $n$ be a positive integer.
Prove that
\[
\sum_{j=1}^{n} \gcd \tup{j, n}
= \sum_{d \mid n} d \phi\tup{ \dfrac{n}{d} }.
\]
More generally, if $\tup{a_1, a_2, a_3, \ldots}$ is a sequence
of reals, then prove that
\[
\sum_{j=1}^{n} a_{\gcd \tup{j, n}}
= \sum_{d \mid n} a_d \phi\tup{ \dfrac{n}{d} }.
\]

\item[\textbf{(b)}]
Let $n \in \NN$. Prove that
\begin{align*}
& \tup{ \text{the number of $\tup{x, y} \in \ZZ^2$ satisfying
              $x^2 + y^2 \leq n$} } \\
& = 1 + 4\sum_{k \in \NN} \tup{-1}^k \floor{ \dfrac{n}{2k+1} } \\
& = 1 + 4 \tup{   \floor{\dfrac{n}{1}} - \floor{\dfrac{n}{3}} 
                + \floor{\dfrac{n}{5}} - \floor{\dfrac{n}{7}}
                + \floor{\dfrac{n}{9}} - \floor{\dfrac{n}{11}}
                \pm \cdots } .
\end{align*}
(The infinite sums in this equality have only finitely many nonzero addends,
and thus are well-defined.)

\end{enumerate}

[\textbf{Hint:} Parts \textbf{(a)} and \textbf{(b)} have nothing to do with
each other.

This is a good place for a reminder that results proven in the notes, as well
as problems from previous homework sets and midterms, can be freely used. Both
parts have rather short solutions if you remember the right results to use!]

\subsection{Remark}

Part \textbf{(b)} of this exercise is a ``discrete'' version of the
famous Madhava--Gregory--Leibniz series
\[
\dfrac{\pi}{4}
= \dfrac{1}{1} - \dfrac{1}{3} + \dfrac{1}{5} - \dfrac{1}{7}
  + \dfrac{1}{9} - \dfrac{1}{11} \pm \cdots
\]
(where $\pi$, at last, does denote the area of the unit circle).
Indeed, if we divide the number of $\tup{x, y} \in \ZZ^2$ satisfying
$x^2 + y^2 \leq n$ by $n$, then we obtain an approximation to
the area of the unit circle that gets better as $n$
grows\footnote{Just observe that the pairs
$\tup{x, y} \in \ZZ^2$ satisfying $x^2 + y^2 \leq n$, regarded
as points in the Euclidean plane, are precisely the lattice points
inside the circle with center $0$ and radius $\sqrt{n}$.
Thus, by counting these pairs, we are approximating the area of this
circle. See \cite[Theorem 12.1]{Clark18} for a rigorous proof.}.
On the other hand, it appears reasonable that dividing
\[
1 + 4 \tup{   \floor{\dfrac{n}{1}} - \floor{\dfrac{n}{3}} 
                + \floor{\dfrac{n}{5}} - \floor{\dfrac{n}{7}}
                + \floor{\dfrac{n}{9}} - \floor{\dfrac{n}{11}}
                \pm \cdots }
\]
by $n$, we obtain an approximation to
$4 \tup{\dfrac{1}{1} - \dfrac{1}{3} + \dfrac{1}{5} - \dfrac{1}{7}
        + \dfrac{1}{9} - \dfrac{1}{11} \pm \cdots}$.
I am not sure whether this can be rigorously proven,
however.\footnote{Of course, for any given $k \in \NN$,
the number
$\dfrac{1}{n}\tup{\floor{\dfrac{n}{2k+1}} - \dfrac{n}{2k+1}}$
does converge to $0$ when $n \to \infty$.
But here we are taking an alternating sum of infinitely many
such numbers; we can ignore all but the first $n$, but even
the first $n$ may no longer converge to $0$ when summed
together.}

\subsection{Solution}

[...]

%----------------------------------------------------------------------------------------
%	EXERCISE 4
%----------------------------------------------------------------------------------------
\horrule{0.3pt} \\[0.4cm]

\section{Exercise 4: Squares in finite fields II}

\subsection{Problem}

Let $\FF$ be a finite field such that
$2 \cdot 1_{\FF} \neq 0_{\FF}$.
In Exercise 5 of
\href{http://www.cip.ifi.lmu.de/~grinberg/t/19s/hw6s.pdf}{homework set \#6},
we have seen that
$\abs{\FF}$ is odd, and that the number of squares in $\FF$ is
$\dfrac{1}{2} \tup{ \abs{\FF} + 1 }$.

In the following, the word ``square'' shall always mean
``square in $\FF$''.

A \textit{nonsquare} shall mean an element of $\FF$ that is not
a square.

Prove the following:

\begin{enumerate}

\item[\textbf{(a)}]
The product of two squares is always a square.

\item[\textbf{(b)}]
The product of a nonzero square with a nonsquare is always
a nonsquare.

\item[\textbf{(c)}]
The product of two nonsquares is always a square.

\end{enumerate}

[\textbf{Hint:} It is easiest to solve the three parts in
this exact order.
For \textbf{(c)}, recall that if a subset
$Y$ of a finite set $X$ satisfies $\abs{Y} \geq \abs{X}$,
then $Y = X$.]

\subsection{Solution}

[...]

%----------------------------------------------------------------------------------------
%	EXERCISE 5
%----------------------------------------------------------------------------------------
\horrule{0.3pt} \\[0.4cm]

\section{Exercise 5: Formal differential calculus}

\subsection{Problem}

Let $\KK$ be a commutative ring.
For each FPS\footnote{Just as in class, the abbreviation ``FPS''
stands for ``formal power series''.
All FPSs and polynomials in this exercise are in $1$ indeterminate
over $\KK$; the indeterminate is called $x$.}
\[
f = \sum_{k \in \NN} a_k x^k = a_0 x^0 + a_1 x^1 + a_2 x^2 + \cdots \in \KK\ivee{x}
\qquad
\text{(where $a_i \in \KK$),}
\]
we define the \textit{derivative} $f'$ of $f$ to be the FPS
\[
\sum_{k > 0} k a_k x^{k-1}
= 1 a_1 x^0 + 2 a_2 x^1 + 3 a_3 x^2 + \cdots \in \KK\ivee{x} .
\]
(This definition imitates the standard procedure for differentiating
power series in analysis, but it does not require any analysis or
topology itself. In particular, $\KK$ may be any commutative ring
-- e.g., a finite field.)

Let $D : \KK\ivee{x} \to \KK\ivee{x}$ be the map sending
each FPS $f$ to its derivative $f'$.
We refer to $D$ as \textit{(formal) differentiation}.
As usual, for any $n \in \NN$, we let $D^n$ denote
$\underbrace{D \circ D \circ \cdots \circ D}_{n \text{ times}}$
(which means $\id$ if $n = 0$).

Prove the following:

\begin{enumerate}

\item[\textbf{(a)}]
If $f \in \KK\ive{x}$, then $f' \in \KK\ive{x}$ and
$\deg \tup{f'} \leq \deg f - 1$.
(In other words, the derivative of a polynomial is again
a polynomial of degree at least $1$ less.)

\item[\textbf{(b)}]
The map $D : \KK\ivee{x} \to \KK\ivee{x}$ is $\KK$-linear
(with respect to the $\KK$-module structure on $\KK\ivee{x}$
defined in class -- i.e., both addition and scaling of FPSs
are defined entrywise).

\item[\textbf{(c)}]
We have $\tup{fg}' = f' g + f g'$ for any two FPSs
$f$ and $g$.
(This is called the \textit{Leibniz rule}.)

\item[\textbf{(d)}]
We have $D^n \tup{x^k} = n! \dbinom{k}{n} x^{k-n}$
for all $n \in \NN$ and $k \in \NN$.
Here, the expression ``$\dbinom{k}{n} x^{k-n}$'' is
to be understood as $0$ when $k < n$.

\item[\textbf{(e)}]
If $\QQ$ is a subring of $\KK$, then
every polynomial $f \in \KK\ive{x}$ satisfies%
\footnote{Just as in class, I am using the notation
``$f \ive{u}$'' for the evaluation of $f$ at $u$.
The more common notation for this is $f \tup{u}$,
but is too easily mistaken for a product. \\
Note also that we need to require $f$ to be a
polynomial here, since $f \ive{x+a}$ would not
be defined if $f$ was merely an FPS.}
\[
f \ive{x+a} = \sum_{n \in \NN} \dfrac{1}{n!} \tup{D^n \tup{f}} \ive{a} \cdot x^n
\qquad \text{for all $a \in \KK$} .
\]
(The infinite sum on the right hand side has only
finitely many nonzero addends.)

\item[\textbf{(f)}]
If $p$ is a prime such that $p \cdot 1_\KK = 0$
(for example, this happens if $\KK = \ZZ / p$),
then $D^p \tup{f} = 0$ for each $f \in \KK\ivee{x}$.

\end{enumerate}

Now, assume that $\QQ$ is a subring of $\KK$.
For each FPS
\[
f = \sum_{k \in \NN} a_k x^k = a_0 x^0 + a_1 x^1 + a_2 x^2 + \cdots \in \KK\ivee{x}
\qquad
\text{(where $a_i \in \KK$),}
\]
we define the \textit{integral} $\int f$ of $f$ to be the FPS
\[
\sum_{k \geq 0} \dfrac{1}{k+1} a_k x^{k+1}
= \dfrac{1}{1} a_0 x^1 + \dfrac{1}{2} a_1 x^2 + \dfrac{1}{3} a_2 x^3 + \cdots \in \KK\ivee{x} .
\]
(This definition imitates the standard procedure for integrating
power series in analysis, but again works for any commutative
ring $\KK$ that contains $\QQ$ as subring.)

Let $J : \KK\ivee{x} \to \KK\ivee{x}$ be the map sending
each FPS $f$ to its integral $\int f$.
Prove the following:

\begin{enumerate}

\item[\textbf{(g)}]
The map $J : \KK\ivee{x} \to \KK\ivee{x}$ is $\KK$-linear.

\item[\textbf{(h)}]
We have $D \circ J = \id$.

\item[\textbf{(i)}]
We have $J \circ D \neq \id$.

\end{enumerate}

[\textbf{Hint:} Don't give too much detail; workable outlines
are sufficient.
Feel free to interchange summation signs without justification.
For part \textbf{(c)}, it is easiest to first prove it
in the particular case when $f = x^a$ and $g = x^b$ for some
$f$ and $g$, and then obtain the general case by interchanging
summations.]

\subsection{Remark}

This exercise is just the beginning of ``algebraic calculus''.
A lot more can be done: Differentiation can be extended to
rational functions; partial derivatives can be defined for
multivariate polynomials and FPSs; differential equations can
be solved formally in FPSs (rather than functions); even a
purely algebraic analogue of the classical
$f'\tup{x} = \lim\limits_{\varepsilon\to 0} \dfrac{f\tup{x+\varepsilon} - f\tup{x}}{\varepsilon}$
definition exists\footnote{See Theorem 5 in
\url{https://math.stackexchange.com/a/2974977/} .}.
These algebraic derivatives play crucial roles in the
study of fields (including finite fields!), in algebraic
geometry (where they help define what a
``singularity'' of an algebraic variety is) and in
enumerative combinatorics (where they aid in computing
generating functions).

Part \textbf{(e)} is perhaps the easiest instance of
the well-known Taylor formula (no error terms, no
smoothness requirements, no convergence issues).

The ``integral'' $\int f$ we defined above is, of course,
only one possible choice of an FPS $g$ satisfying $g' = f$.
Just as in calculus, you can add any constant to it, and
you get another.
Part \textbf{(h)} is an algebraic version of one half of
the Fundamental Theorem of Calculus.
You can easily prove the other half: For each FPS $f$,
the FPS $\tup{J \circ D} \tup{f}$ differs from $f$ only
in its constant term.

If $\KK$ contains $\QQ$ as a subring, then both $J$ and $D$ are
elements of the $\KK$-algebra $\End \tup{\KK\ivee{x}}$
(by parts \textbf{(b)} and \textbf{(g)} of this exercise).
Part \textbf{(h)} of this exercise shows that $J$ is a right
inverse of $D$; but part \textbf{(i)} shows that $J$ is not a
left inverse (and thus not an inverse) of $D$.
This yields an example of a left inverse that is not a right
inverse.

\subsection{Solution}

[...]

%----------------------------------------------------------------------------------------
%	EXERCISE 6
%----------------------------------------------------------------------------------------
\horrule{0.3pt} \\[0.4cm]

\section{Exercise 6: Formal difference calculus and integer-valued polynomials}

\subsection{Problem}

Let $\KK$ be a commutative ring.

For any polynomial $f \in \KK\ive{x}$,
we define the \textit{first finite difference} $f^\Delta$ of $f$
to be the polynomial
\[
f \ive{x+1} - f \ive{x} \in \KK\ive{x} .
\]
(This is a ``discrete analogue'' of the derivative,
in case the analysis-free derivative from Exercise 5
was not discrete enough for you.
It cannot be extended to FPSs, however, since you cannot
substitute $x+1$ for $x$ in an FPS.)

Let $\Delta : \KK\ive{x} \to \KK\ive{x}$ be the map sending
each polynomial $f$ to $f^\Delta$.
As usual, for any $n \in \NN$, we let $\Delta^n$ denote
$\underbrace{\Delta \circ \Delta \circ \cdots \circ \Delta}_{n \text{ times}}$
(which means $\id$ if $n = 0$).

Prove the following:

\begin{enumerate}

\item[\textbf{(a)}]
The map $\Delta : \KK\ive{x} \to \KK\ive{x}$ is $\KK$-linear
(with respect to the $\KK$-module structure on $\KK\ive{x}$
defined in class -- i.e., both addition and scaling of
polynomials are defined entrywise).

\item[\textbf{(b)}]
We have $\tup{fg}^\Delta = f^\Delta g + f\ive{x+1} g^\Delta$
for any two polynomials $f$ and $g$.

\end{enumerate}

Now, assume that $\QQ$ is a subring of $\KK$.

For any $n \in \NN$, we define a polynomial%
\footnote{Note that we are within our rights to divide
by $n!$ here, since $\QQ$ is a subring of $\KK$.}
\[
\dbinom{x}{n} := \dfrac{x\tup{x-1}\tup{x-2}\cdots\tup{x-n+1}}{n!}
\in \KK\ive{x} .
\]
We also set $\dbinom{x}{n} := 0$ for every negative $n$.

Prove the following:

\begin{enumerate}

\item[\textbf{(c)}]
We have $\Delta^n \tup{\dbinom{x}{k}} = \dbinom{x}{k-n}$
for all $n \in \NN$ and $k \in \ZZ$.

\item[\textbf{(d)}]
If $m \in \NN$, and if $f \in \KK\ive{x}$ is a polynomial
of degree $\leq m$,
then there exist elements $a_0, a_1, \ldots, a_m \in \KK$
such that $f = \sum_{i=0}^m a_i \dbinom{x}{i}$.

\item[\textbf{(e)}]
Every polynomial $f \in \KK\ive{x}$ satisfies
\[
f \ive{x+a} = \sum_{n \in \NN} \tup{\Delta^n \tup{f}} \ive{a} \cdot \dbinom{x}{n}
\qquad \text{for all $a \in \KK$} .
\]
(The infinite sum on the right hand side has only
finitely many nonzero addends.)

\item[\textbf{(f)}]
Let $m \in \NN$, and let $f \in \KK\ive{x}$ be a polynomial
of degree $\leq m$.
Assume that $f\ive{k} \in \ZZ$ for each $k \in \set{0, 1, \ldots, m}$.
Then, there exist integers $a_0, a_1, \ldots, a_m$
such that $f = \sum_{i=0}^m a_i \dbinom{x}{i}$.

\end{enumerate}

[\textbf{Hint:} Part \textbf{(d)} is easiest to prove by
induction on $m$.
You can then prove part \textbf{(e)} for $f = \dbinom{x}{i}$
first (where $i \in \NN$), and then extend it to arbitrary $f$
by means of part \textbf{(d)}.
Part \textbf{(f)}, in turn, can be derived from part
\textbf{(e)} through a strategic choice of $a$.]

\subsection{Remark}

Just as $\Delta$ is an analogue of the differentiation
operator $D$ from Exercise 5,
we can define an analogue of the integration operator
$J$ from Exercise 5.
This will be a $\KK$-linear map $\Sigma : \KK\ive{x} \to \KK\ive{x}$
that sends each polynomial
$\sum_{i=0}^m a_i \dbinom{x}{i}$ to
$\sum_{i=0}^m a_i \dbinom{x}{i+1}$
(this definition makes sense, because part \textbf{(d)}
of this exercise
shows that each polynomial can be written in the form
$\sum_{i=0}^m a_i \dbinom{x}{i}$, uniquely except for
``leading zeroes'').
Again, we have $\Delta \circ \Sigma = \id$ but
$\Sigma \circ \Delta \neq \id$.

Moreover, if $f \in \KK\ive{x}$ is a polynomial, then
$\tup{\Sigma\tup{f}} \ive{0} = 0$
and
$\tup{\Sigma\tup{f}} \ive{n}
 = \tup{\Sigma\tup{f}} \ive{n-1} + f \ive{n-1}$
for each $n \in \ZZ$ (indeed, the former equality
follows easily from the definition of $\Sigma$,
while the latter follows from $\Delta \circ \Sigma = \id$).
Hence, by induction, we can see that
\[
\tup{\Sigma\tup{f}} \ive{n}
= f\ive{0} + f\ive{1} + \cdots + f\ive{n-1}
\qquad \text{for each polynomial $f \in \KK\ive{x}$
             and each $n \in \NN$}.
\]
In other words, the value of $\Sigma\tup{f}$ at an
$n \in \NN$ is the sum of the first $n$ values of
$f$ on nonnegative integers! (Whence the notation $\Sigma$.)
For example, if we set $f = x^2$, then it is easy to see
that $\Sigma\tup{f} = 2 \dbinom{x}{3} + \dbinom{x}{2}$
(to see this, just expand $f$ in the form
$\sum_{i=0}^m a_i \dbinom{x}{i}$ -- namely,
$f = x^2 = 2 \dbinom{x}{2} + \dbinom{x}{1}$ --, and
then apply the definition of $\Sigma$); thus we obtain
\[
2 \dbinom{n}{3} + \dbinom{n}{2}
= 0^2 + 1^2 + \cdots + \tup{n-1}^2 .
\]
Similarly you can find a formula for
$0^k + 1^k + \cdots + \tup{n-1}^k$ whenever $k \in \NN$.    

Part \textbf{(e)} of this exercise is a result of Newton.

\subsection{Solution}

[...]

\begin{thebibliography}{99999999}                                                                                         %

% Feel free to add your sources -- or copy some from the source code
% of the class notes ( http://www.cip.ifi.lmu.de/~grinberg/t/19s/notes.tex ).

% \bibitem[Clark07]{Clark07}
% Pete Clark, \textit{Gauss's circle problem},
% \url{http://math.uga.edu/~pete/4400gausscircle.pdf} .

\bibitem[Clark18]{Clark18}
Pete L. Clark, \textit{Number Theory: A Contemporary Introduction},
8 January 2018. \\
\url{http://math.uga.edu/~pete/4400FULL.pdf}

\end{thebibliography}

\end{document}

