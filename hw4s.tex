% The LaTeX below is mostly computer-generated (for reasons of speed); don't expect it to be very readable. Sorry.

\documentclass[paper=a4, fontsize=12pt]{scrartcl}%
\usepackage[T1]{fontenc}
\usepackage[english]{babel}
\usepackage{amsmath,amsfonts,amsthm,amssymb}
\usepackage{mathrsfs}
\usepackage{sectsty}
\usepackage{hyperref}
\usepackage{graphicx}
\usepackage{framed}
\usepackage{ifthen}
\usepackage{lastpage}
\usepackage[headsepline,footsepline,manualmark]{scrlayer-scrpage}
\usepackage[height=10in,a4paper,hmargin={1in,0.8in}]{geometry}
\usepackage[usenames,dvipsnames]{xcolor}
\usepackage{tikz}
\usepackage{verbatim}
\usepackage{amsmath}
\usepackage{amsfonts}
\usepackage{amssymb}%
\setcounter{MaxMatrixCols}{30}
%TCIDATA{OutputFilter=latex2.dll}
%TCIDATA{Version=5.50.0.2960}
%TCIDATA{LastRevised=Sunday, March 24, 2019 19:24:41}
%TCIDATA{<META NAME="GraphicsSave" CONTENT="32">}
%TCIDATA{<META NAME="SaveForMode" CONTENT="1">}
%TCIDATA{BibliographyScheme=Manual}
%BeginMSIPreambleData
\providecommand{\U}[1]{\protect\rule{.1in}{.1in}}
%EndMSIPreambleData
\allsectionsfont{\centering \normalfont\scshape}
\setlength\parindent{20pt}
\newcommand{\CC}{\mathbb{C}}
\newcommand{\RR}{\mathbb{R}}
\newcommand{\QQ}{\mathbb{Q}}
\newcommand{\NN}{\mathbb{N}}
\newcommand{\DD}{{\mathbb{D}}}
\newcommand{\PP}{\mathbb{P}}
\newcommand{\Z}[1]{\mathbb{Z}/#1\mathbb{Z}}
\newcommand{\ZZ}{\mathbb{Z}}
\newcommand{\id}{\operatorname{id}}
\newcommand{\lcm}{\operatorname{lcm}}
\newcommand{\set}[1]{\left\{ #1 \right\}}
\newcommand{\abs}[1]{\left| #1 \right|}
\newcommand{\tup}[1]{\left( #1 \right)}
\newcommand{\ive}[1]{\left[ #1 \right]}
\newcommand{\floor}[1]{\left\lfloor #1 \right\rfloor}
\newcommand{\underbrack}[2]{\underbrace{#1}_{\substack{#2}}}
\newcommand{\powset}[2][]{\ifthenelse{\equal{#2}{}}{\mathcal{P}\left(#1\right)}{\mathcal{P}_{#1}\left(#2\right)}}
\newcommand{\mapeq}[1]{\underset{#1}{\equiv}}
\newcommand{\eps}{\varepsilon}
\newcommand{\horrule}[1]{\rule{\linewidth}{#1}}
\newcommand{\nnn}{\nonumber\\}
\let\sumnonlimits\sum
\let\prodnonlimits\prod
\let\cupnonlimits\bigcup
\let\capnonlimits\bigcap
\renewcommand{\sum}{\sumnonlimits\limits}
\renewcommand{\prod}{\prodnonlimits\limits}
\renewcommand{\bigcup}{\cupnonlimits\limits}
\renewcommand{\bigcap}{\capnonlimits\limits}
\newtheoremstyle{plainsl}
{8pt plus 2pt minus 4pt}
{8pt plus 2pt minus 4pt}
{\slshape}
{0pt}
{\bfseries}
{.}
{5pt plus 1pt minus 1pt}
{}
\theoremstyle{plainsl}
\newtheorem{theorem}{Theorem}[section]
\newtheorem{proposition}[theorem]{Proposition}
\newtheorem{lemma}[theorem]{Lemma}
\newtheorem{corollary}[theorem]{Corollary}
\newtheorem{conjecture}[theorem]{Conjecture}
\theoremstyle{definition}
\newtheorem{definition}[theorem]{Definition}
\newtheorem{example}[theorem]{Example}
\newtheorem{exercise}[theorem]{Exercise}
\newtheorem{examples}[theorem]{Examples}
\newtheorem{algorithm}[theorem]{Algorithm}
\newtheorem{question}[theorem]{Question}
\theoremstyle{remark}
\newtheorem{remark}[theorem]{Remark}
\newenvironment{statement}{\begin{quote}}{\end{quote}}
\newenvironment{fineprint}{\begin{small}}{\end{small}}
\newcommand{\myname}{Darij Grinberg}
\newcommand{\myid}{00000000}
\newcommand{\mymail}{dgrinber@umn.edu}
\newcommand{\psetnumber}{4}
\ihead{Solutions to homework set \#\psetnumber}
\ohead{page \thepage\ of \pageref{LastPage}}
\ifoot{\myname, \myid}
\ofoot{\mymail}
\begin{document}

\title{ \normalfont {\normalsize \textsc{University of Minnesota, School of
Mathematics} }\\[25pt] \rule{\linewidth}{0.5pt} \\[0.4cm] {\huge Math 4281: Introduction to Modern Algebra, }\\Spring 2019: Homework 4\\\rule{\linewidth}{2pt} \\[0.5cm] }
\author{Darij Grinberg}
\maketitle

%----------------------------------------------------------------------------------------
%	EXERCISE 1
%----------------------------------------------------------------------------------------
\rule{0pt}{0.3pt} \\[0.4cm]

\section{Exercise 1: Equivalence relations always come from maps}

\subsection{Problem}

Let $S$ be a set.

Recall that if $T$ is a further set, and if $f:S\rightarrow T$ is a map, then
$\underset{f}{\equiv}$ denotes the relation on $S$ defined by
\[
\left(  a\underset{f}{\equiv}b\right)  \iff\left(  f\left(  a\right)
=f\left(  b\right)  \right)  .
\]
This is an equivalence relation, called \textquotedblleft equality upon
applying $f$\textquotedblright\ or \textquotedblleft equality under
$f$\textquotedblright.

Now, let $\sim$ be \textbf{any} equivalence relation on $S$. Prove that $\sim$
has the form $\underset{f}{\equiv}$ for a properly chosen set $T$ and a
properly chosen $f : S \to T$.

More precisely, prove that $\sim$ equals $\underset{f}{\equiv}$, where $T$ is
the quotient set $S / \sim$ and where $f : S \to T$ is the projection map
$\pi_{\sim}: S \to S / \sim$.

[\textbf{Hint:} To prove that two relations $R_{1}$ and $R_{2}$ on $S$ are
equal, you need to check that every pair $\left(  a, b \right)  $ of elements
of $S$ satisfies the equivalence $\left(  a R_{1} b \right)  \iff\left(  a
R_{2} b \right)  $.]

\subsection{Solution}

See \href{http://www-users.math.umn.edu/~dgrinber/19s/notes.pdf}{the class
notes}, where this is Exercise 3.3.3. (The numbering may shift; it is one of
the exercises in the ``Equivalence classes'' section.)

%----------------------------------------------------------------------------------------
%	EXERCISE 2
%----------------------------------------------------------------------------------------
\rule{\linewidth}{0.3pt} \\[0.4cm]

\section{Exercise 2: Totient-related sum}

\subsection{Problem}

Let $n > 1$ be an integer. Prove that
\[
\sum_{\substack{i \in\left\{  1, 2, \ldots, n \right\}  ; \\i \perp n}} i = n
\phi\left(  n \right)  / 2 .
\]


\subsection{Solution}

See \href{http://www-users.math.umn.edu/~dgrinber/19s/notes.pdf}{the class
notes}, where this is Exercise 2.14.5. (The numbering may shift; it is one of
the exercises in the \textquotedblleft Euler's totient function ($\phi
$-function)\textquotedblright\ section.)

%This exercise is also more or less clearly
%equivalent to §2.3, problem 40 in Niven/Zuckerman/Montgomery
%(see kim-NT/sol3.pdf §13).


%----------------------------------------------------------------------------------------
%	EXERCISE 3
%----------------------------------------------------------------------------------------
\rule{\linewidth}{0.3pt} \\[0.4cm]

\section{Exercise 3: Dual numbers}

\subsection{Problem}

Recall that complex numbers were defined as pairs $\left(  a, b \right)  $ of
real numbers, with entrywise addition and subtraction and a certain
weird-looking multiplication.

Let me define a different kind of ``numbers'': the \textit{dual numbers}. (The
word ``numbers'' may appear a bit inappropriate for them, but it is not
exactly a trademark...)

We define a \textit{dual number} to be a pair $\left(  a, b \right)  $ of two
real numbers $a$ and $b$.

We let ${\mathbb{D}}$ be the set of all dual numbers.

For each real number $r$, we denote the dual number $\left(  r, 0 \right)  $
by $r_{\mathbb{D}}$.

We let $\varepsilon$ denote the dual number $\left(  0, 1 \right)  $.

Define three binary operations $+$, $-$ and $\cdot$ on ${\mathbb{D}}$ by
setting
\begin{align}
\left(  a, b \right)  + \left(  c, d \right)   &  = \left(  a+c, b+d \right)
,\\
\left(  a, b \right)  - \left(  c, d \right)   &  = \left(  a-c, b-d \right)
,\\
\left(  a, b \right)  \cdot\left(  c, d \right)   &  = \left(  ac, ad+bc
\right)
\end{align}
for all $\left(  a, b \right)  \in{\mathbb{D}}$ and $\left(  c, d \right)
\in{\mathbb{D}}$.

(Note that the only difference to complex numbers is the definition of $\cdot
$, which is lacking a $-bd$ term.)

Again, we are following standard PEMDAS
rules\footnote{\url{https://en.wikipedia.org/wiki/Order_of_operations}} for
the order of operations, and we abbreviate $\alpha\cdot\beta$ as $\alpha\beta$.

\begin{enumerate}
\item[\textbf{(a)}] Prove that $\alpha\cdot\left(  \beta\cdot\gamma\right)  =
\left(  \alpha\cdot\beta\right)  \cdot\gamma$ for any $\alpha, \beta,
\gamma\in{\mathbb{D}}$.
\end{enumerate}

You can use the following properties of dual numbers without proof (they are
all essentially obvious):

\begin{itemize}
\item We have $\alpha+ \beta= \beta+ \alpha$ for any $\alpha, \beta
\in{\mathbb{D}}$.

\item We have $\alpha+ \left(  \beta+ \gamma\right)  = \left(  \alpha+
\beta\right)  + \gamma$ for any $\alpha, \beta, \gamma\in{\mathbb{D}}$.

\item We have $\alpha+ 0_{\mathbb{D}} = \alpha$ for any $\alpha\in{\mathbb{D}%
}$.

\item We have $\alpha\cdot1_{\mathbb{D}} = \alpha$ for any $\alpha
\in{\mathbb{D}}$.

\item We have $\alpha\cdot\beta= \beta\cdot\alpha$ for any $\alpha, \beta
\in{\mathbb{D}}$.

\item We have $\alpha\cdot\left(  \beta+ \gamma\right)  = \alpha\beta+
\alpha\gamma$ and $\left(  \alpha+ \beta\right)  \cdot\gamma= \alpha\gamma+
\beta\gamma$ for any $\alpha, \beta, \gamma\in{\mathbb{D}}$.

\item We have $\alpha\cdot0_{\mathbb{D}} = 0_{\mathbb{D}}$ for any $\alpha
\in{\mathbb{D}}$.

\item If $\alpha, \beta, \gamma\in{\mathbb{D}}$, then we have the equivalence
$\left(  \alpha- \beta= \gamma\right)  \iff\left(  \alpha= \beta+
\gamma\right)  $.
\end{itemize}

We shall identify each real number $r$ with the dual number $r_{\mathbb{D}} =
\left(  r, 0 \right)  $.

\begin{enumerate}
\item[\textbf{(b)}] Prove that $a + b\varepsilon= \left(  a, b \right)  $ for
any $a, b \in\mathbb{R}$.
\end{enumerate}

An \textit{inverse} of a dual number $\alpha\in{\mathbb{D}}$ means a dual
number $\beta$ such that $\alpha\beta= 1_{\mathbb{D}}$. This inverse is
unique, and is called $\alpha^{-1}$.

\begin{enumerate}
\item[\textbf{(c)}] Prove that a dual number $\alpha= a + b\varepsilon$ (with
$a, b \in\mathbb{R}$) has an inverse if and only if $a \neq0$.

\item[\textbf{(d)}] If $a, b \in\mathbb{R}$ satisfy $a \neq0$, prove that the
inverse of the dual number $a + b\varepsilon$ is $\dfrac{1}{a} - \dfrac
{b}{a^{2}} \varepsilon$.
\end{enumerate}

We define finite sums and products of dual numbers in the usual way (i.e.,
just as finite sums and products of real numbers were defined). Here, an empty
sum of dual numbers is always understood to be $0_{\mathbb{D}}$, whereas an
empty product of dual numbers is always understood to be $1_{\mathbb{D}}$.

If $\alpha$ is a dual number and $k\in\mathbb{N}$, then the $k$\textit{-th
power of }$\alpha$ is defined to be the dual number $\underbrace{\alpha
\alpha\cdots\alpha}_{k\text{ factors}}$. This $k$-th power is denoted by
$\alpha^{k}$. Thus, in particular, $\alpha^{0}=\underbrace{\alpha\alpha
\cdots\alpha}_{0\text{ factors}}=\left(  \text{empty product}\right)
=1_{{\mathbb{D}}}$.

\begin{enumerate}
\item[\textbf{(e)}] Let $P\left(  x \right)  = a_{k} x^{k} + a_{k-1} x^{k-1} +
\cdots+ a_{0}$ be a polynomial with real coefficients. Prove that
\[
P \left(  a + b \varepsilon\right)  = P\left(  a \right)  + b P^{\prime
}\left(  a \right)  \varepsilon\qquad\text{for any } a, b \in\mathbb{R} .
\]
Here, $P^{\prime}$ denotes the derivative of $P$, which is defined by
\[
P^{\prime}\left(  x \right)  = k a_{k} x^{k-1} + \left(  k-1 \right)  a_{k-1}
x^{k-2} + \cdots+ 1 a_{1} x^{0} .
\]

\end{enumerate}

\subsection{Remark}

The dual number $\varepsilon$ is one of the simplest ``rigorous
infinitesimals'' that appear in mathematics. Part \textbf{(e)} of the exercise
shows that we can literally write $P \left(  a + \varepsilon\right)  =
P\left(  a \right)  + P^{\prime}\left(  a \right)  \varepsilon$ when $P$ is a
polynomial, without having to compute any limits. It is tempting to ``solve''
this equation for $P^{\prime}\left(  a \right)  $, thus obtaining something
like $P^{\prime}\left(  a \right)  = \dfrac{P \left(  a + \varepsilon\right)
- P\left(  a \right)  }{\varepsilon}$. However, this needs to be taken with a
grain of salt, since $\varepsilon$ has no inverse and the fraction $\dfrac{P
\left(  a + \varepsilon\right)  - P\left(  a \right)  }{\varepsilon}$ is not
uniquely determined. There are other, subtler ways to put infinitesimals on a
firm algebraic footing, but dual numbers are already useful in some situations.

Note that dual numbers have \textit{zero-divisors}: i.e., there exist nonzero
dual numbers $a$ and $b$ such that $ab = 0$. The simplest example is probably
$\varepsilon^{2} = 0$ (despite $\varepsilon\neq0$).

\subsection{Solution}

\textbf{(a)} Let $\alpha,\beta,\gamma\in{\mathbb{D}}$. We must prove that
$\alpha\cdot\left(  \beta\cdot\gamma\right)  =\left(  \alpha\cdot\beta\right)
\cdot\gamma$.

We have $\alpha\in\mathbb{D}$; in other words, $\alpha$ is a dual number.
Thus, $\alpha$ is a pair $\left(  a,b\right)  $ of two real numbers $a$ and $b
$. Consider these $a$ and $b$.

We have $\beta\in\mathbb{D}$; in other words, $\beta$ is a dual number. Thus,
$\beta$ is a pair $\left(  c,d\right)  $ of two real numbers $c$ and $d$.
Consider these $c$ and $d$.

We have $\gamma\in\mathbb{D}$; in other words, $\gamma$ is a dual number.
Thus, $\gamma$ is a pair $\left(  e,f\right)  $ of two real numbers $e$ and $f
$. Consider these $e$ and $f$.

Comparing the equalities
\begin{align*}
\underbrace{\alpha}_{=\left(  a,b\right)  }\cdot\left(  \underbrace{\beta
}_{=\left(  c,d\right)  }\cdot\underbrace{\gamma}_{=\left(  e,f\right)
}\right)   &  =\left(  a,b\right)  \cdot\underbrace{\left(  \left(
c,d\right)  \cdot\left(  e,f\right)  \right)  }_{\substack{=\left(
ce,cf+de\right)  \\\text{(by the definition of} \\\text{the operation }%
\cdot\text{ on }\mathbb{D}\text{)}}}=\left(  a,b\right)  \cdot\left(
ce,cf+de\right) \\
&  =\left(  \underbrace{a\left(  ce\right)  }_{=ace},\underbrace{a\left(
cf+de\right)  +b\left(  ce\right)  }_{=acf+ade+bce}\right) \\
&  \ \ \ \ \ \ \ \ \ \ \left(  \text{by the definition of the operation }%
\cdot\text{ on }\mathbb{D}\right) \\
&  =\left(  ace,acf+ade+bce\right)
\end{align*}
and%
\begin{align*}
\left(  \underbrace{\alpha}_{=\left(  a,b\right)  }\cdot\underbrace{\beta
}_{=\left(  c,d\right)  }\right)  \cdot\underbrace{\gamma}_{=\left(
e,f\right)  }  &  =\underbrace{\left(  \left(  a,b\right)  \cdot\left(
c,d\right)  \right)  } _{\substack{=\left(  ac,ad+bc\right)  \\\text{(by the
definition of} \\\text{the operation }\cdot\text{ on }\mathbb{D}\text{)}%
}}\cdot\left(  e,f\right)  =\left(  ac,ad+bc\right)  \cdot\left(  e,f\right)
\\
&  =\left(  \underbrace{\left(  ac\right)  e}_{=ace},\underbrace{\left(
ac\right)  f+\left(  ad+bc\right)  e}_{=acf+ade+bce}\right) \\
&  \ \ \ \ \ \ \ \ \ \ \left(  \text{by the definition of the operation }%
\cdot\text{ on }\mathbb{D}\right) \\
&  =\left(  ace,acf+ade+bce\right)  ,
\end{align*}
we obtain $\alpha\cdot\left(  \beta\cdot\gamma\right)  =\left(  \alpha
\cdot\beta\right)  \cdot\gamma$. This solves part \textbf{(a)} of the
exercise.\\[0.4cm]

\textbf{(b)} Let $a,b\in\mathbb{R}$. Then, the definition of $a_{\mathbb{D}} $
yields $a_{\mathbb{D}}=\left(  a,0\right)  $, whereas the definition of
$b_{\mathbb{D}}$ yields $b_{\mathbb{D}}=\left(  b,0\right)  $. The definition
of $\varepsilon$ yields $\varepsilon=\left(  0,1\right)  $. Thus,%
\begin{align*}
\underbrace{a_{\mathbb{D}}}_{=\left(  a,0\right)  }+\underbrace{b_{\mathbb{D}%
}}_{=\left(  b,0\right)  }\underbrace{\varepsilon}_{=\left(  0,1\right)  }  &
=\left(  a,0\right)  +\underbrace{\left(  b,0\right)  \cdot\left(  0,1\right)
} _{\substack{=\left(  b\cdot0,b\cdot1+0\cdot0\right)  \\\text{(by the
definition of} \\\text{the operation }\cdot\text{ on }\mathbb{D}\text{)}%
}}=\left(  a,0\right)  +\left(  \underbrace{b\cdot0}_{=0},\underbrace{b\cdot
1+0\cdot0}_{=b}\right) \\
&  =\left(  a,0\right)  +\left(  0,b\right)  =\left(  \underbrace{a+0}%
_{=a},\underbrace{0+b}_{=b}\right)  \ \ \ \ \ \ \ \ \ \ \left(
\begin{array}
[c]{c}%
\text{by the definition of}\\
\text{the operation }+\text{ on }\mathbb{D}%
\end{array}
\right) \\
&  =\left(  a,b\right)  .
\end{align*}
Since we identify the real numbers $a$ and $b$ with the dual numbers
$a_{\mathbb{D}}$ and $b_{\mathbb{D}}$, we can rewrite this equality as
$a+b\varepsilon=\left(  a,b\right)  $. This solves part \textbf{(b)} of the
exercise.\\[0.4cm]

\textbf{(c)} Let $\alpha=a+b\varepsilon$ be a dual number with $a,b\in
\mathbb{R}$. We must prove that $\alpha$ has an inverse if and only if
$a\neq0$.

In other words, we must prove the logical equivalence%
\begin{equation}
\left(  \alpha\text{ has an inverse}\right)  \ \Longleftrightarrow\ \left(
a\neq0\right)  . \label{sol.dualnums.basics.c.goal}%
\end{equation}


We shall prove the \textquotedblleft$\Longleftarrow$\textquotedblright\ and
\textquotedblleft$\Longrightarrow$\textquotedblright\ parts of this
equivalence separately:

[\textit{Proof of the \textquotedblleft}$\Longleftarrow$%
\textit{\textquotedblright\ direction of (\ref{sol.dualnums.basics.c.goal}):}
Assume that $a\neq0$. We must prove that $\alpha$ has an inverse.

The real numbers $\dfrac{1}{a}$ and $\dfrac{b}{a^{2}}$ are well-defined (since
$a\neq0$). Hence, the dual number $\left(  \dfrac{1}{a},-\dfrac{b}{a^{2}%
}\right)  $ is well-defined.

Part \textbf{(b)} of this exercise yields $a+b\varepsilon=\left(  a,b\right)
$. Hence, $\alpha=a+b\varepsilon=\left(  a,b\right)  $. Thus,%
\begin{align*}
\underbrace{\alpha}_{=\left(  a,b\right)  }\cdot\left(  \dfrac{1}{a}%
,-\dfrac{b}{a^{2}}\right)   &  =\left(  a,b\right)  \cdot\left(  \dfrac{1}%
{a},-\dfrac{b}{a^{2}}\right)  =\left(  \underbrace{a\cdot\dfrac{1}{a}}%
_{=1},\underbrace{a\cdot\left(  -\dfrac{b}{a^{2}}\right)  +b\cdot\dfrac{1}{a}%
}_{=0}\right) \\
&  =\left(  1,0\right)  =1_{\mathbb{D}}\ \ \ \ \ \ \ \ \ \ \left(  \text{since
the definition of }1_{\mathbb{D}}\text{ yields }1_{\mathbb{D}}=\left(
1,0\right)  \right)  .
\end{align*}
In other words, the dual number $\left(  \dfrac{1}{a},-\dfrac{b}{a^{2}%
}\right)  $ is an inverse of $\alpha$ (by the definition of \textquotedblleft
inverse\textquotedblright). Hence, the dual number $\alpha$ has an inverse
(namely, $\left(  \dfrac{1}{a},-\dfrac{b}{a^{2}}\right)  $). This proves the
\textquotedblleft$\Longleftarrow$\textquotedblright\ direction of
(\ref{sol.dualnums.basics.c.goal}).]

[\textit{Proof of the \textquotedblleft}$\Longrightarrow$%
\textit{\textquotedblright\ direction of (\ref{sol.dualnums.basics.c.goal}):}
Assume that $\alpha$ has an inverse. We must prove that $a\neq0$.

We have assumed that $\alpha$ has an inverse. Let $\beta$ be such an inverse.
Thus, $\beta$ is an inverse of $\alpha$. In other words, $\beta$ is a dual
number such that $\alpha\beta=1_{\mathbb{D}}$ (by the definition of
\textquotedblleft inverse\textquotedblright).

Part \textbf{(b)} of this exercise yields $a+b\varepsilon=\left(  a,b\right)
$. Hence, $\alpha=a+b\varepsilon=\left(  a,b\right)  $.

But $\beta$ is a dual number. Thus, $\beta$ is a pair $\left(  c,d\right)  $
of two real numbers $c$ and $d$. Consider these $c$ and $d$.

We have $\alpha\beta=1_{\mathbb{D}}=\left(  1,0\right)  $ (by the definition
of $1_{\mathbb{D}}$). Thus,
\[
\left(  1,0\right)  =\underbrace{\alpha}_{=\left(  a,b\right)  }%
\underbrace{\beta}_{=\left(  c,d\right)  }=\left(  a,b\right)  \cdot\left(
c,d\right)  =\left(  ac,ad+bc\right)  \ \ \ \ \ \ \ \ \ \ \left(
\begin{array}
[c]{c}%
\text{by the definition of the}\\
\text{operation }\cdot\text{ on }\mathbb{D}%
\end{array}
\right)  .
\]


Thus, $1=ac$ and $0=ad+bc$. From $1=ac$, we conclude that $ac=1\neq0$ and
therefore $a\neq0$. This proves the \textquotedblleft$\Longrightarrow
$\textquotedblright\ direction of (\ref{sol.dualnums.basics.c.goal}).]

We thus have proven both directions of the equivalence
(\ref{sol.dualnums.basics.c.goal}). Thus, (\ref{sol.dualnums.basics.c.goal})
is proven, and part \textbf{(c)} of the exercise is solved.\\[0.4cm]

\textbf{(d)} Let $a,b\in\mathbb{R}$ satisfy $a\neq0$. We must prove that the
inverse of the dual number $a+b\varepsilon$ is $\dfrac{1}{a}-\dfrac{b}{a^{2}%
}\varepsilon$.

The real numbers $\dfrac{1}{a}$ and $\dfrac{b}{a^{2}}$ are well-defined (since
$a\neq0$). Hence, the dual number $\left(  \dfrac{1}{a},-\dfrac{b}{a^{2}%
}\right)  $ is well-defined. Part \textbf{(b)} of this exercise (applied to
$\dfrac{1}{a}$ and $-\dfrac{b}{a^{2}}$ instead of $a$ and $b$) yields
$\dfrac{1}{a}+\left(  -\dfrac{b}{a^{2}}\right)  \varepsilon=\left(  \dfrac
{1}{a},-\dfrac{b}{a^{2}}\right)  $.

Part \textbf{(b)} of this exercise yields $a+b\varepsilon=\left(  a,b\right)
$. Hence,%
\begin{align*}
&  \underbrace{\left(  a+b\varepsilon\right)  }_{=\left(  a,b\right)  }%
\cdot\underbrace{\left(  \dfrac{1}{a}-\dfrac{b}{a^{2}}\varepsilon\right)
}_{=\dfrac{1}{a}+\left(  -\dfrac{b}{a^{2}}\right)  \varepsilon=\left(
\dfrac{1}{a},-\dfrac{b}{a^{2}}\right)  }\\
&  =\left(  a,b\right)  \cdot\left(  \dfrac{1}{a},-\dfrac{b}{a^{2}}\right)
=\left(  \underbrace{a\cdot\dfrac{1}{a}}_{=1},\underbrace{a\cdot\left(
-\dfrac{b}{a^{2}}\right)  +b\cdot\dfrac{1}{a}}_{=0}\right) \\
&  =\left(  1,0\right)  =1_{\mathbb{D}}\ \ \ \ \ \ \ \ \ \ \left(  \text{since
the definition of }1_{\mathbb{D}}\text{ yields }1_{\mathbb{D}}=\left(
1,0\right)  \right)  .
\end{align*}
In other words, the dual number $\dfrac{1}{a}-\dfrac{b}{a^{2}}\varepsilon$ is
an inverse of $a+b\varepsilon$ (by the definition of \textquotedblleft
inverse\textquotedblright). Hence, the dual number $a+b\varepsilon$ has a
unique inverse (because any dual number that has an inverse must have a unique
inverse), and this inverse is $\dfrac{1}{a}-\dfrac{b}{a^{2}}\varepsilon$. This
solves part \textbf{(d)} of the exercise.\\[0.4cm]

\textbf{(e)} We will use two auxiliary claims:

\begin{statement}
\textit{Claim 1:} Let $J$ be a finite set. For each $j\in J$, let $a_{j}$ and
$b_{j}$ be two real numbers. Then,%
\[
\sum_{j\in J}\left(  a_{j},b_{j}\right)  =\left(  \sum_{j\in J}a_{j}%
,\sum_{j\in J}b_{j}\right)
\]
(as dual numbers). (Here, the \textquotedblleft$\sum_{j\in J}$%
\textquotedblright\ sign on the left hand side stands for a sum of finitely
many dual numbers; this is defined just as we defined sums of finitely many
integers or real numbers or complex numbers.)
\end{statement}

[\textit{Proof of Claim 1:} This claim can be proven by a straightforward
induction on $\left\vert J\right\vert $, using the definition of the operation
$+$ on $\mathbb{D}$. (We leave the details to the reader.).]

\begin{statement}
\textit{Claim 2:} Let $a,b\in\mathbb{R}$. Then,%
\[
\left(  a,b\right)  ^{k}=\left(  a^{k},ka^{k-1}b\right)
\ \ \ \ \ \ \ \ \ \ \text{for every }k\in\mathbb{N}.
\]
(Here, we agree to understand the expression \textquotedblleft$ka^{k-1}%
$\textquotedblright\ to mean $0$ when $k=0$, even if its sub-expression
\textquotedblleft$a^{k-1}$\textquotedblright\ may be meaningless\footnote{The
sub-expression \textquotedblleft$a^{k-1}$\textquotedblright\ is indeed
meaningless when $a=0$ and $k=0$.}.)
\end{statement}

[\textit{First proof of Claim 2:} We shall prove Claim 2 by induction on $k$:

\textit{Induction base:} We have $\left(  a,b\right)  ^{0}=1_{\mathbb{D}%
}=\left(  1,0\right)  $ (by the definition of $1_{\mathbb{D}}$). Comparing
this with $\left(  \underbrace{a^{0}}_{=1},\underbrace{0a^{0-1}b}_{=0}\right)
=\left(  1,0\right)  $, we obtain $\left(  a,b\right)  ^{0}=\left(
a^{0},0a^{0-1}b\right)  $. In other words, Claim 2 holds for $k=0$. This
completes the induction base.

\textit{Induction step:} Let $m\in\mathbb{N}$. Assume that Claim 2 holds for
$k=m$. We must prove that Claim 2 holds for $k=m+1$.

We have assumed that Claim 2 holds for $k=m$. In other words, we have $\left(
a,b\right)  ^{m}=\left(  a^{m},ma^{m-1}b\right)  $. Now,%
\[
\left(  a,b\right)  ^{m+1}=\left(  a,b\right)  \cdot\underbrace{\left(
a,b\right)  ^{m}}_{=\left(  a^{m},ma^{m-1}b\right)  }=\left(  a,b\right)
\cdot\left(  a^{m},ma^{m-1}b\right)  =\left(  aa^{m},ama^{m-1}b+ba^{m}\right)
\]
(by the definition of the operation $\cdot$ on $\mathbb{D}$). But we have
$ama^{m-1}b=m\underbrace{aa^{m-1}}_{=a^{m}}b=ma^{m}b$. (Strictly speaking,
this computation is only justified when $m\neq0$, because we agreed to give
the expression \textquotedblleft$ka^{k-1}$\textquotedblright\ special
treatment when $k=0$. But it is clear that the equality $ama^{m-1}b=ma^{m}b$
also holds when $m=0$.) Thus,%
\begin{align*}
\left(  a,b\right)  ^{m+1}  &  =\left(  \underbrace{aa^{m}}_{=a^{m+1}%
},\underbrace{ama^{m-1}b}_{=ma^{m}b}+\underbrace{ba^{m}}_{=a^{m}b}\right)
=\left(  a^{m+1},\underbrace{ma^{m}b+a^{m}b}_{\substack{=\left(  m+1\right)
a^{m}b=\left(  m+1\right)  a^{\left(  m+1\right)  -1}b\\\text{(since
}m=\left(  m+1\right)  -1\text{)}}}\right) \\
&  =\left(  a^{m+1},\left(  m+1\right)  a^{\left(  m+1\right)  -1}b\right)  .
\end{align*}
In other words, Claim 2 holds for $k=m+1$. This completes the induction step.
Thus, Claim 2 is proven by induction.]

The proof we just gave for Claim 2 was straightforward and completely
elementary; for the sake of instructivity, let us next outline a different
proof of Claim 2, which relies on the binomial formula. First, we recall that
the binomial formula (see, e.g., Theorem 2.17.13 in
\href{http://www-users.math.umn.edu/~dgrinber/19s/notes.pdf}{the class notes})
says that any real numbers $x$ and $y$ and any $n\in\mathbb{N}$ satisfy%
\begin{equation}
\left(  x+y\right)  ^{n}=\sum_{k=0}^{n}\dbinom{n}{k}x^{k}y^{n-k}.
\label{sol.dualnums.basics.e.binomf}%
\end{equation}
We can replace \textquotedblleft real numbers\textquotedblright\ by
\textquotedblleft dual numbers\textquotedblright\ in this statement (i.e., we
can let $x$ and $y$ be dual numbers instead of being real numbers) without
sacrificing its correctness; indeed, the very same argument that proves
(\ref{sol.dualnums.basics.e.binomf}) for arbitrary real numbers $x$ and $y$
(by induction on $n$) will also prove (\ref{sol.dualnums.basics.e.binomf}) for
dual numbers $x$ and $y$. This is because the basic rules of addition,
multiplication and taking powers that hold for real numbers all hold for dual
numbers as well\footnote{For example, part \textbf{(a)} of this exercise shows
that the associativity of multiplication holds for dual numbers; likewise, all
the other basic rules can be proven.}. So we know that
(\ref{sol.dualnums.basics.e.binomf}) holds whenever $x$ and $y$ are dual
numbers. In other words, any dual numbers $x$ and $y$ and any $n\in\mathbb{N}$
satisfy%
\begin{equation}
\left(  x+y\right)  ^{n}=\sum_{k=0}^{n}\dbinom{n}{k}x^{k}y^{n-k}=\sum
_{m=0}^{n}\dbinom{n}{m}x^{m}y^{n-m} \label{sol.dualnums.basics.e.binomf2}%
\end{equation}
(here, we have renamed the summation index $k$ as $m$).

Let us also observe that $\varepsilon^{2}=0$. (Indeed,
\begin{align*}
\varepsilon^{2}  &  =\underbrace{\varepsilon}_{=\left(  0,1\right)  }%
\cdot\underbrace{\varepsilon}_{=\left(  0,1\right)  }=\left(  0,1\right)
\cdot\left(  0,1\right)  =\left(  \underbrace{0\cdot0}_{=0},\underbrace{0\cdot
1+1\cdot0}_{=0}\right) \\
&  \ \ \ \ \ \ \ \ \ \ \left(  \text{by the definition of the operation }%
\cdot\text{ on }\mathbb{D}\right) \\
&  =\left(  0,0\right)  =0.
\end{align*}
)

We are now ready to give our second proof of Claim 2:

[\textit{Second proof of Claim 2 (sketched):} It is easy to see that Claim 2
holds for $k=0$. Thus, for the rest of this proof, we WLOG assume that
$k\neq0$. Hence, $k\geq1$ (since $k\in\mathbb{N}$).

Part \textbf{(b)} of this exercise yields $a+b\varepsilon=\left(  a,b\right)
$. Hence, $\left(  a,b\right)  =a+b\varepsilon$. Thus,%
\begin{align*}
\left(  a,b\right)  ^{k}  &  =\left(  a+b\varepsilon\right)  ^{k}=\sum
_{m=0}^{k}\dbinom{k}{m}a^{m}\left(  b\varepsilon\right)  ^{k-m}%
\ \ \ \ \ \ \ \ \ \ \left(
\begin{array}
[c]{c}%
\text{by (\ref{sol.dualnums.basics.e.binomf2}), applied}\\
\text{to }x=a\text{ and }y=b\varepsilon
\end{array}
\right) \\
&  =\sum_{m=0}^{k-2}\dbinom{k}{m}a^{m}\underbrace{\left(  b\varepsilon\right)
^{k-m}}_{=b^{k-m}\varepsilon^{k-m}}+\underbrace{\dbinom{k}{k-1}}%
_{\substack{=k\\\text{(this is easy}\\\text{to check)}}}a^{k-1}%
\underbrace{\left(  b\varepsilon\right)  ^{k-\left(  k-1\right)  }}_{=\left(
b\varepsilon\right)  ^{1}=b\varepsilon}+\underbrace{\dbinom{k}{k}%
}_{\substack{=1\\\text{(this is easy}\\\text{to check)}}}a^{k}%
\underbrace{\left(  b\varepsilon\right)  ^{k-k}}_{=\left(  b\varepsilon
\right)  ^{0}=1}\\
&  \ \ \ \ \ \ \ \ \ \ \left(
\begin{array}
[c]{c}%
\text{here, we have split off the addends for }m=k-1\\
\text{and for }m=k\text{ from the sum (which is allowed,}\\
\text{because }k\geq1\text{ shows that both of these addends exist)}%
\end{array}
\right) \\
&  =\sum_{m=0}^{k-2}\dbinom{k}{m}a^{m}b^{k-m}\underbrace{\varepsilon^{k-m}%
}_{\substack{=\varepsilon^{2}\varepsilon^{\left(  k-m\right)  -2}%
\\\text{(since }k-m\geq2\\\text{(because }m\leq k-2\text{))}}}+ka^{k-1}%
b\varepsilon+a^{k}\\
&  =\sum_{m=0}^{k-2}\dbinom{k}{m}a^{m}b^{k-m}\underbrace{\varepsilon^{2}}%
_{=0}\varepsilon^{\left(  k-m\right)  -2}+ka^{k-1}b\varepsilon+a^{k}%
=ka^{k-1}b\varepsilon+a^{k}\\
&  =a^{k}+ka^{k-1}b\varepsilon=\left(  a^{k},ka^{k-1}b\right)
\end{align*}
(by part \textbf{(b)} of the exercise, applied to $a^{k}$ and $ka^{k-1}b$
instead of $a$ and $b$). Thus, Claim 2 is proven again.]

We can now finally solve part \textbf{(e)} of the exercise:

We have $P\left(  x\right)  =a_{k}x^{k}+a_{k-1}x^{k-1}+\cdots+a_{0}=\sum
_{j=0}^{k}a_{j}x^{j}$. Substituting $a$ for $x$ in this equation, we find%
\begin{equation}
P\left(  a\right)  =\sum_{j=0}^{k}a_{j}a^{j}.
\label{sol.dualnums.basics.e.Pa=}%
\end{equation}


Moreover, $P^{\prime}\left(  x\right)  =ka_{k}x^{k-1}+\left(  k-1\right)
a_{k-1}x^{k-2}+\cdots+1a_{1}x^{0}=\sum_{j=1}^{k}ja_{j}x^{j-1}=\sum_{j=1}%
^{k}a_{j}jx^{j-1}$. Substituting $a$ for $x$ in this equation, we find%
\begin{equation}
P^{\prime}\left(  a\right)  =\sum_{j=1}^{k}a_{j}ja^{j-1}.
\label{sol.dualnums.basics.e.P'a=}%
\end{equation}


Substitute the dual number $\left(  a,b\right)  $ for $x$ in the equation
$P\left(  x\right)  =\sum_{j=0}^{k}a_{j}x^{j}$. We thus obtain\footnote{Here,
we agree to understand the expression \textquotedblleft$ja^{j-1}%
$\textquotedblright\ to mean $0$ when $j=0$, even if its sub-expression
\textquotedblleft$a^{j-1}$\textquotedblright\ may be meaningless. This is the
same convention that we followed in Claim 2.}%
\begin{align*}
P\left(  \left(  a,b\right)  \right)   &  =\sum_{j=0}^{k}\underbrace{a_{j}%
}_{\substack{=\left(  a_{j}\right)  _{\mathbb{D}}=\left(  a_{j},0\right)
\\\text{(by the definition}\\\text{of }\left(  a_{j}\right)  _{\mathbb{D}%
}\text{)}}}\underbrace{\left(  a,b\right)  ^{j}}_{\substack{=\left(
a^{j},ja^{j-1}b\right)  \\\text{(by Claim 2,}\\\text{applied to }j\text{
instead of }k\text{)}}}=\sum_{j=0}^{k}\underbrace{\left(  a_{j},0\right)
\cdot\left(  a^{j},ja^{j-1}b\right)  }_{\substack{=\left(  a_{j}a^{j}%
,a_{j}ja^{j-1}b+0a^{j}\right)  \\\text{(by the definition}\\\text{of the
operation }\cdot\text{ on }\mathbb{D}\text{)}}}\\
&  =\sum_{j=0}^{k}\left(  a_{j}a^{j},\underbrace{a_{j}ja^{j-1}b+0a^{j}%
}_{=a_{j}ja^{j-1}b}\right)  =\sum_{j=0}^{k}\left(  a_{j}a^{j},ja_{j}%
a^{j-1}b\right)  =\left(  \sum_{j=0}^{k}a_{j}a^{j},\sum_{j=0}^{k}a_{j}%
ja^{j-1}b\right) \\
&  \ \ \ \ \ \ \ \ \ \ \left(
\begin{array}
[c]{c}%
\text{by Claim 1, applied to }\left\{  0,1,\ldots,k\right\}  \text{, }%
a_{j}a^{j}\text{ and }a_{j}ja^{j-1}b\\
\text{instead of }J\text{, }a_{j}\text{ and }b_{j}\text{ (since the
\textquotedblleft}\sum_{j=0}^{k}\text{\textquotedblright\ sign means
\textquotedblleft}\sum_{j\in\left\{  0,1,\ldots,k\right\}  }%
\text{\textquotedblright)}%
\end{array}
\right) \\
&  =\left(  \underbrace{\sum_{j=0}^{k}a_{j}a^{j}}_{\substack{=P\left(
a\right)  \\\text{(by (\ref{sol.dualnums.basics.e.Pa=}))}}},\underbrace{\sum
_{j=1}^{k}a_{j}ja^{j-1}}_{\substack{=P^{\prime}\left(  a\right)  \\\text{(by
(\ref{sol.dualnums.basics.e.P'a=}))}}}b\right) \\
&  \ \ \ \ \ \ \ \ \ \ \left(  \text{since }\sum_{j=0}^{k}a_{j}ja^{j-1}%
b=a_{0}\underbrace{0a^{0-1}}_{=0}b+\sum_{j=1}^{k}a_{j}ja^{j-1}b=\sum_{j=1}%
^{k}a_{j}ja^{j-1}b\right) \\
&  =\left(  P\left(  a\right)  ,\underbrace{P^{\prime}\left(  a\right)
b}_{=bP^{\prime}\left(  a\right)  }\right)  =\left(  P\left(  a\right)
,bP^{\prime}\left(  a\right)  \right)  .
\end{align*}
But part \textbf{(b)} of this exercise (applied to $P\left(  a\right)  $ and
$bP^{\prime}\left(  a\right)  $ instead of $a$ and $b$) yields
\begin{equation}
P\left(  a\right)  +bP^{\prime}\left(  a\right)  \varepsilon=\left(  P\left(
a\right)  ,bP^{\prime}\left(  a\right)  \right)  .
\label{sol.dualnums.basics.e.ur}%
\end{equation}


But part \textbf{(b)} of this exercise yields $a+b\varepsilon=\left(
a,b\right)  $. Hence,%
\[
P\left(  \underbrace{a+b\varepsilon}_{=\left(  a,b\right)  }\right)  =P\left(
\left(  a,b\right)  \right)  =\left(  P\left(  a\right)  ,bP^{\prime}\left(
a\right)  \right)  =P\left(  a\right)  +bP^{\prime}\left(  a\right)
\varepsilon
\]
(by (\ref{sol.dualnums.basics.e.ur})). This solves part \textbf{(e)} of the problem.

%----------------------------------------------------------------------------------------
%	EXERCISE 4
%----------------------------------------------------------------------------------------
\rule{\linewidth}{0.3pt} \\[0.4cm]

\section{Exercise 4: $\mathbb{Z}\left[  \sqrt2 \right]  $}

\subsection{Problem}

Let $\mathbb{Z}\left[  \sqrt2 \right]  $ denote the set of all reals of the
form $a + b \sqrt2$ with $a, b \in\mathbb{Z}$. We shall call such reals
\textit{$\sqrt2$-integers}.

\begin{enumerate}
\item[\textbf{(a)}] Prove that any $\alpha, \beta\in\mathbb{Z}\left[  \sqrt2
\right]  $ satisfy $\alpha+ \beta\in\mathbb{Z}\left[  \sqrt2 \right]  $ and
$\alpha- \beta\in\mathbb{Z}\left[  \sqrt2 \right]  $ and $\alpha\beta
\in\mathbb{Z}\left[  \sqrt2 \right]  $.

\item[\textbf{(b)}] Prove that every element of $\mathbb{Z}\left[  \sqrt2
\right]  $ can be written as $a + b \sqrt2$ for a \textbf{unique} pair
$\left(  a, b \right)  $ of integers. (In other words, if four integers $a, b,
c, d$ satisfy $a + b \sqrt2 = c + d \sqrt2$, then $a = c$ and $b = d$.)
\end{enumerate}

For any $\alpha\in\mathbb{Z}\left[  \sqrt{2}\right]  $, define the $\sqrt{2}%
$\textit{-norm} $N_{2}\left(  \alpha\right)  $ of $\alpha$ by $N_{2}\left(
\alpha\right)  =a^{2}-2b^{2}$, where $\alpha$ is written in the form
$\alpha=a+b\sqrt{2}$ with $a,b\in\mathbb{Z}$. This is well-defined by part
\textbf{(b)} of this exercise.

\begin{enumerate}
\item[\textbf{(c)}] Prove that $N_{2}\left(  \alpha\beta\right)  =
N_{2}\left(  \alpha\right)  N_{2}\left(  \beta\right)  $ for all $\alpha,
\beta\in\mathbb{Z}\left[  \sqrt2 \right]  $.
\end{enumerate}

Let $\left(  p_{0}, p_{1}, p_{2}, \ldots\right)  $ be the sequence of
nonnegative integers defined recursively by
\[
p_{0} = 0, \qquad p_{1} = 1, \qquad\text{and } \qquad p_{n} = 2p_{n-1} +
p_{n-2} \text{ for all } n \geq2.
\]
(Thus, $p_{2} = 2$ and $p_{3} = 5$ and $p_{4} = 12$ and so on.)

\begin{enumerate}
\item[\textbf{(d)}] Prove that $p_{n+1} p_{n-1} - p_{n}^{2} = \left(  -1
\right)  ^{n} $ for each $n \geq1$.

\item[\textbf{(e)}] Prove that $\left(  p_{n-1} + p_{n} + p_{n} \sqrt2
\right)  \cdot\left(  p_{n-1} + p_{n} - p_{n} \sqrt2 \right)  = \left(  -1
\right)  ^{n}$ for each $n \geq1$.
\end{enumerate}

[\textbf{Hint:} For \textbf{(d)}, use induction.]

\subsection{Remark}

The set $\mathbb{Z}\left[  \sqrt2 \right]  $ of $\sqrt2$-integers is rather
similar to the set $\mathbb{Z}\left[  i \right]  $ of Gaussian integers: the
former has elements of the form $a + b\sqrt2$ with $a, b \in\mathbb{Z}$, while
the latter has elements of the form $a + b\sqrt{-1}$ with $a, b \in\mathbb{Z}%
$. The $\sqrt2$-norm on $\mathbb{Z}\left[  \sqrt2 \right]  $ is an analogue of
the (usual) norm on $\mathbb{Z}\left[  i \right]  $. However, visually
speaking, the latter set is ``spread out'' in the Euclidean plane, while the
former is ``concentrated'' on the real line (and actually everywhere dense on
it -- i.e., every little interval on the real line has a $\sqrt2$-integer
inside it). The difference has algebraic consequences; in particular, there
are only four units ($1, -1, i, -i$) in $\mathbb{Z}\left[  i \right]  $,
whereas $\mathbb{Z}\left[  \sqrt2 \right]  $ has infinitely many units
(namely, part \textbf{(e)} of the exercise shows that $p_{n-1} + p_{n} + p_{n}
\sqrt2$ is a unit for each $n \geq1$).

\subsection{Solution}

\textbf{(a)} Let $\alpha,\beta\in\mathbb{Z}\left[  \sqrt{2}\right]  $. We must
prove that $\alpha+\beta\in\mathbb{Z}\left[  \sqrt{2}\right]  $ and
$\alpha-\beta\in\mathbb{Z}\left[  \sqrt{2}\right]  $ and $\alpha\beta
\in\mathbb{Z}\left[  \sqrt{2}\right]  $.

We have $\alpha\in\mathbb{Z}\left[  \sqrt{2}\right]  $. In other words,
$\alpha$ is a real of the form $a+b\sqrt{2}$ with $a,b\in\mathbb{Z}$ (by the
definition of $\mathbb{Z}\left[  \sqrt{2}\right]  $). In other words, there
exist two integers $x_{1},x_{2}\in\mathbb{Z}$ such that $\alpha=x_{1}%
+x_{2}\sqrt{2}$. Similarly, there exist two integers $y_{1},y_{2}\in
\mathbb{Z}$ such that $\beta=y_{1}+y_{2}\sqrt{2}$. Consider these four
integers $x_{1},x_{2},y_{1},y_{2}$.

We have%
\[
\underbrace{\alpha}_{=x_{1}+x_{2}\sqrt{2}}+\underbrace{\beta}_{=y_{1}%
+y_{2}\sqrt{2}}=\left(  x_{1}+x_{2}\sqrt{2}\right)  +\left(  y_{1}+y_{2}%
\sqrt{2}\right)  =\left(  x_{1}+y_{1}\right)  +\left(  x_{2}+y_{2}\right)
\sqrt{2}.
\]
Hence, $\alpha+\beta$ is a real of the form $a+b\sqrt{2}$ with $a,b\in
\mathbb{Z}$ (namely, with $a=x_{1}+y_{1}$ and $b=x_{2}+y_{2}$). In other
words, $\alpha+\beta\in\mathbb{Z}\left[  \sqrt{2}\right]  $ (by the definition
of $\mathbb{Z}\left[  \sqrt{2}\right]  $).

We have%
\[
\underbrace{\alpha}_{=x_{1}+x_{2}\sqrt{2}}-\underbrace{\beta}_{=y_{1}%
+y_{2}\sqrt{2}}=\left(  x_{1}+x_{2}\sqrt{2}\right)  -\left(  y_{1}+y_{2}%
\sqrt{2}\right)  =\left(  x_{1}-y_{1}\right)  +\left(  x_{2}-y_{2}\right)
\sqrt{2}.
\]
Hence, $\alpha-\beta$ is a real of the form $a+b\sqrt{2}$ with $a,b\in
\mathbb{Z}$ (namely, with $a=x_{1}-y_{1}$ and $b=x_{2}-y_{2}$). In other
words, $\alpha-\beta\in\mathbb{Z}\left[  \sqrt{2}\right]  $ (by the definition
of $\mathbb{Z}\left[  \sqrt{2}\right]  $).

We have%
\begin{align}
\underbrace{\alpha}_{=x_{1}+x_{2}\sqrt{2}}\underbrace{\beta}_{=y_{1}%
+y_{2}\sqrt{2}}  &  =\left(  x_{1}+x_{2}\sqrt{2}\right)  \left(  y_{1}%
+y_{2}\sqrt{2}\right)  =x_{1}y_{1}+x_{1}y_{2}\sqrt{2}+x_{2}\sqrt{2}y_{1}%
+x_{2}\sqrt{2}y_{2}\sqrt{2}\nonumber\\
&  =x_{1}y_{1}+x_{1}y_{2}\sqrt{2}+x_{2}\underbrace{\sqrt{2}y_{1}}_{=y_{1}%
\sqrt{2}}+x_{2}\underbrace{\sqrt{2}y_{2}}_{=y_{2}\sqrt{2}}\sqrt{2}\nonumber\\
&  =x_{1}y_{1}+x_{1}y_{2}\sqrt{2}+x_{2}y_{1}\sqrt{2}+x_{2}y_{2}%
\underbrace{\sqrt{2}\sqrt{2}}_{=\left(  \sqrt{2}\right)  ^{2}=2}\nonumber\\
&  =x_{1}y_{1}+x_{1}y_{2}\sqrt{2}+x_{2}y_{1}\sqrt{2}+x_{2}y_{2}2\nonumber\\
&  =\left(  x_{1}y_{1}+2x_{2}y_{2}\right)  +\left(  x_{1}y_{2}+x_{2}%
y_{1}\right)  \sqrt{2}. \label{sol.Zsqrt2.basics.a.ab}%
\end{align}
Hence, $\alpha\beta$ is a real of the form $a+b\sqrt{2}$ with $a,b\in
\mathbb{Z}$ (namely, with $a=x_{1}y_{1}+2x_{2}y_{2}$ and $b=x_{1}y_{2}%
+x_{2}y_{1}$). In other words, $\alpha\beta\in\mathbb{Z}\left[  \sqrt
{2}\right]  $ (by the definition of $\mathbb{Z}\left[  \sqrt{2}\right]  $).

We have now shown that $\alpha+\beta\in\mathbb{Z}\left[  \sqrt{2}\right]  $
and $\alpha-\beta\in\mathbb{Z}\left[  \sqrt{2}\right]  $ and $\alpha\beta
\in\mathbb{Z}\left[  \sqrt{2}\right]  $. This solves part \textbf{(a)} of the
exercise.\\[0.4cm]

\textbf{(b)} Let $\alpha$ be an element of $\mathbb{Z}\left[  \sqrt{2}\right]
$. We must prove that $\alpha$ can be written as $a+b\sqrt{2}$ for a
\textbf{unique} pair $\left(  a,b\right)  $ of integers.

Clearly, $\alpha$ can be written as $a+b\sqrt{2}$ for \textbf{at least one}
pair $\left(  a,b\right)  $ of integers (because this is what it means for
$\alpha$ to belong to $\mathbb{Z}\left[  \sqrt{2}\right]  $). Thus, it remains
to prove that $\alpha$ can be written as $a+b\sqrt{2}$ for \textbf{at most
one} pair $\left(  a,b\right)  $ of integers. In other words, we must prove
that if $\left(  a_{1},b_{1}\right)  $ and $\left(  a_{2},b_{2}\right)  $ are
two pairs $\left(  a,b\right)  $ of integers such that $\alpha=a+b\sqrt{2}$,
then $\left(  a_{1},b_{1}\right)  =\left(  a_{2},b_{2}\right)  $.

Let us prove this. Let $\left(  a_{1},b_{1}\right)  $ and $\left(  a_{2}%
,b_{2}\right)  $ be two pairs $\left(  a,b\right)  $ of integers such that
$\alpha=a+b\sqrt{2}$. We must show that $\left(  a_{1},b_{1}\right)  =\left(
a_{2},b_{2}\right)  $.

Assume the contrary. Thus, $\left(  a_{1},b_{1}\right)  \neq\left(
a_{2},b_{2}\right)  $.

It is easy to check that $2$ is not a perfect square. (We can show something
more general: Any integer that is congruent to $2$ or $3$ modulo $4$ is not a
perfect square. Indeed, Exercise 2.7.2 in
\href{http://www-users.math.umn.edu/~dgrinber/19s/notes.pdf}{the class notes}
shows that each integer $u$ satisfies either $u^{2}\equiv0\operatorname{mod}4$
(if $u$ is even) or $u^{2}\equiv1\operatorname{mod}4$ (if $u$ is odd). In
other words, each perfect square is congruent to either $0$ or $1$ modulo $4$.
Thus, any integer that is congruent to $2$ or $3$ modulo $4$ is not a perfect square.)

Exercise 2.10.15 in
\href{http://www-users.math.umn.edu/~dgrinber/19s/notes.pdf}{the class notes}
shows that if a positive integer $u$ is not a perfect square, then $\sqrt{u}$
is irrational. Applying this to $u=2$, we conclude that $\sqrt{2}$ is
irrational (since $2$ is not a perfect square).

But $\left(  a_{1},b_{1}\right)  $ is a pair $\left(  a,b\right)  $ of
integers such that $\alpha=a+b\sqrt{2}$. In other words, $\left(  a_{1}%
,b_{1}\right)  $ is a pair of integers and satisfies $\alpha=a_{1}+b_{1}%
\sqrt{2}$. Similarly, $\left(  a_{2},b_{2}\right)  $ is a pair of integers and
satisfies $\alpha=a_{2}+b_{2}\sqrt{2}$. Hence, $a_{2}+b_{2}\sqrt{2}%
=\alpha=a_{1}+b_{1}\sqrt{2}$, so that%
\begin{equation}
a_{2}-a_{1}=b_{1}\sqrt{2}-b_{2}\sqrt{2}=\left(  b_{1}-b_{2}\right)  \sqrt{2}.
\label{sol.Zsqrt2.basics.b.2}%
\end{equation}
If we had $b_{1}=b_{2}$, then this would yield $a_{2}-a_{1}%
=\underbrace{\left(  b_{1}-b_{2}\right)  }_{\substack{=0\\\text{(since }%
b_{1}=b_{2}\text{)}}}\sqrt{2}=0$, which would lead to $a_{1}=a_{2}$ and
therefore $\left(  \underbrace{a_{1}}_{=a_{2}},\underbrace{b_{1}}_{=b_{2}%
}\right)  =\left(  a_{2},b_{2}\right)  $; but this would contradict $\left(
a_{1},b_{1}\right)  \neq\left(  a_{2},b_{2}\right)  $. Hence, we cannot have
$b_{1}=b_{2}$. Thus, we have $b_{1}\neq b_{2}$. In other words, $b_{1}%
-b_{2}\neq0$. Hence, we can divide both sides of the equality
(\ref{sol.Zsqrt2.basics.b.2}) by $b_{1}-b_{2}$. We thus obtain $\dfrac
{a_{2}-a_{1}}{b_{1}-b_{2}}=\sqrt{2}$. Hence, the number $\dfrac{a_{2}-a_{1}%
}{b_{1}-b_{2}}$ is irrational (since $\sqrt{2}$ is irrational). But this
contradicts the fact that $\dfrac{a_{2}-a_{1}}{b_{1}-b_{2}}$ is rational
(which is clear, since $a_{1},a_{2},b_{1},b_{2}$ are integers). This
contradiction shows that our assumption was wrong. Hence, $\left(  a_{1}%
,b_{1}\right)  =\left(  a_{2},b_{2}\right)  $ is proven. This completes our
solution of part \textbf{(b)} of the exercise.\\[0.4cm]

\textbf{(c)} Let $\alpha,\beta\in\mathbb{Z}\left[  \sqrt{2}\right]  $. We must
prove that $\alpha+\beta\in\mathbb{Z}\left[  \sqrt{2}\right]  $ and
$\alpha-\beta\in\mathbb{Z}\left[  \sqrt{2}\right]  $ and $\alpha\beta
\in\mathbb{Z}\left[  \sqrt{2}\right]  $.

We have $\alpha\in\mathbb{Z}\left[  \sqrt{2}\right]  $. In other words,
$\alpha$ is a real of the form $a+b\sqrt{2}$ with $a,b\in\mathbb{Z}$ (by the
definition of $\mathbb{Z}\left[  \sqrt{2}\right]  $). In other words, there
exist two integers $x_{1},x_{2}\in\mathbb{Z}$ such that $\alpha=x_{1}%
+x_{2}\sqrt{2}$. Similarly, there exist two integers $y_{1},y_{2}\in
\mathbb{Z}$ such that $\beta=y_{1}+y_{2}\sqrt{2}$. Consider these four
integers $x_{1},x_{2},y_{1},y_{2}$.

We have $\alpha=x_{1}+x_{2}\sqrt{2}$ with $x_{1},x_{2}\in\mathbb{Z}$. Thus,
the definition of $N_{2}\left(  \alpha\right)  $ yields $N_{2}\left(
\alpha\right)  =x_{1}^{2}-2x_{2}^{2}$. Similarly, $N_{2}\left(  \beta\right)
=y_{1}^{2}-2y_{2}^{2}$. But (\ref{sol.Zsqrt2.basics.a.ab}) shows that%
\[
\alpha\beta=\left(  x_{1}y_{1}+2x_{2}y_{2}\right)  +\left(  x_{1}y_{2}%
+x_{2}y_{1}\right)  \sqrt{2}\ \ \ \ \ \ \ \ \ \ \text{with }x_{1}y_{1}%
+2x_{2}y_{2},x_{1}y_{2}+x_{2}y_{1}\in\mathbb{Z}.
\]
Hence, the definition of $N_{2}\left(  \alpha\beta\right)  $ yields%
\[
N_{2}\left(  \alpha\beta\right)  =\left(  x_{1}y_{1}+2x_{2}y_{2}\right)
^{2}-2\left(  x_{1}y_{2}+x_{2}y_{1}\right)  ^{2}=x_{1}^{2}y_{1}^{2}-2x_{1}%
^{2}y_{2}^{2}-2x_{2}^{2}y_{1}^{2}+4x_{2}^{2}y_{2}^{2}%
\]
(after some straightforward computation). Comparing this with%
\[
\underbrace{N_{2}\left(  \alpha\right)  }_{=x_{1}^{2}-2x_{2}^{2}%
}\underbrace{N_{2}\left(  \beta\right)  }_{=y_{1}^{2}-2y_{2}^{2}}=\left(
x_{1}^{2}-2x_{2}^{2}\right)  \left(  y_{1}^{2}-2y_{2}^{2}\right)  =x_{1}%
^{2}y_{1}^{2}-2x_{1}^{2}y_{2}^{2}-2x_{2}^{2}y_{1}^{2}+4x_{2}^{2}y_{2}^{2},
\]
we obtain $N_{2}\left(  \alpha\beta\right)  =N_{2}\left(  \alpha\right)
N_{2}\left(  \beta\right)  $. This solves part \textbf{(c)} of the exercise.

[\textit{Remark:} An alternative solution to part \textbf{(c)} relies on the
concept of a $\sqrt{2}$-conjugate of an $\alpha\in\mathbb{Z}\left[  \sqrt
{2}\right]  $. Namely, the $\sqrt{2}$\textit{-conjugate} of an $\alpha
\in\mathbb{Z}\left[  \sqrt{2}\right]  $ is defined to be the number
$a-b\sqrt{2}$, where $\alpha$ is written in the form $\alpha=a+b\sqrt{2}$ with
$a,b\in\mathbb{Z}$. We denote this $\sqrt{2}$-conjugate by $\overline{\alpha}%
$, but keep in mind that this notation clashes with the notation
$\overline{\alpha}$ for complex numbers $\alpha$. (Fortunately, we will not
talk about complex numbers in this solution, so this clash does not matter.)
It is easy to see that $N_{2}\left(  \alpha\right)  =\alpha\overline{\alpha}$
for each $\alpha\in\mathbb{Z}\left[  \sqrt{2}\right]  $, and it is also easy
to see that $\overline{\alpha\cdot\beta}=\overline{\alpha}\cdot\overline
{\beta}$ for any $\alpha,\beta\in\mathbb{Z}\left[  \sqrt{2}\right]  $. Armed
with these two equalities, we can now observe that any $\alpha,\beta
\in\mathbb{Z}\left[  \sqrt{2}\right]  $ satisfy%
\[
N_{2}\left(  \alpha\beta\right)  =\alpha\beta\cdot\underbrace{\overline
{\alpha\beta}}_{=\overline{\alpha\cdot\beta}=\overline{\alpha}\cdot
\overline{\beta}}=\alpha\beta\cdot\overline{\alpha}\cdot\overline{\beta
}=\underbrace{\alpha\overline{\alpha}}_{=N_{2}\left(  \alpha\right)
}\underbrace{\beta\overline{\beta}}_{=N_{2}\left(  \beta\right)  }%
=N_{2}\left(  \alpha\right)  N_{2}\left(  \beta\right)  .
\]
\newline This solves part \textbf{(c)} of the exercise again.]\\[0.4cm]

\textbf{(d)} We shall solve part \textbf{(d)} of the exercise by induction on
$n$:

\textit{Induction base:} We have $p_{2}\underbrace{p_{0}}_{=0}-p_{1}^{2}%
=p_{2}0-p_{1}^{2}=-p_{1}^{2}=-1^{2}$ (since $p_{1}=1$). Thus, $p_{2}%
p_{0}-p_{1}^{2}=-1^{2}=-1=\left(  -1\right)  ^{1}$. In other words, part
\textbf{(d)} of the exercise holds for $n=1$. This completes the induction base.

\textit{Induction step:} Let $m\geq1$ be an integer. Assume that part
\textbf{(d)} of the exercise holds for $n=m$. We must prove that part
\textbf{(d)} of the exercise holds for $n=m+1$.

We have assumed that part \textbf{(d)} of the exercise holds for $n=m$. In
other words, $p_{m+1}p_{m-1}-p_{m}^{2}=\left(  -1\right)  ^{m}$.

The recursive definition of the sequence $\left(  p_{0},p_{1},p_{2}%
,\ldots\right)  $ yields $p_{m+2}=2p_{m+1}+p_{m}$ and $p_{m+1}=2p_{m}+p_{m-1}%
$. From $p_{m+1}=2p_{m}+p_{m-1}$, we obtain $2p_{m}-\underbrace{p_{m+1}%
}_{=2p_{m}+p_{m-1}}=2p_{m}-\left(  2p_{m}+p_{m-1}\right)  =-p_{m-1}$. Now,%
\begin{align*}
\underbrace{p_{m+2}}_{=2p_{m+1}+p_{m}}p_{m}-p_{m+1}^{2}  &  =\left(
2p_{m+1}+p_{m}\right)  p_{m}-p_{m+1}^{2}=2p_{m+1}p_{m}+p_{m}^{2}-p_{m+1}^{2}\\
&  =p_{m+1}\underbrace{\left(  2p_{m}-p_{m+1}\right)  }_{=-p_{m-1}}+p_{m}%
^{2}=p_{m+1}\left(  -p_{m-1}\right)  +p_{m}^{2}\\
&  =-\underbrace{\left(  p_{m+1}p_{m-1}-p_{m}^{2}\right)  }_{=\left(
-1\right)  ^{m}}=-\left(  -1\right)  ^{m}=\left(  -1\right)  ^{m+1}.
\end{align*}
In other words, part \textbf{(d)} of the exercise holds for $n=m+1$. This
completes the induction step. Thus, part \textbf{(d)} of the exercise is
proven by induction.\\[0.4cm]

\textbf{(e)} Let $n\geq1$. The recursive definition of the sequence $\left(
p_{0},p_{1},p_{2},\ldots\right)  $ yields $p_{n+1}=2p_{n}+p_{n-1}$.

Recall the classical identity $\left(  a+b\right)  \left(  a-b\right)
=a^{2}-b^{2}$, which holds for any reals $a$ and $b$. Applying this to
$a=p_{n-1}+p_{n}$ and $b=p_{n}\sqrt{2}$, we obtain%
\begin{align*}
&  \left(  p_{n-1}+p_{n}+p_{n}\sqrt{2}\right)  \left(  p_{n-1}+p_{n}%
-p_{n}\sqrt{2}\right) \\
&  =\underbrace{\left(  p_{n-1}+p_{n}\right)  ^{2}}_{=p_{n-1}^{2}%
+2p_{n-1}p_{n}+p_{n}^{2}}-\underbrace{\left(  p_{n}\sqrt{2}\right)  ^{2}%
}_{=p_{n}^{2}\cdot2=2p_{n}^{2}}=p_{n-1}^{2}+2p_{n-1}p_{n}+p_{n}^{2}-2p_{n}%
^{2}=2p_{n-1}p_{n}+p_{n-1}^{2}-p_{n}^{2}\\
&  =p_{n-1}\underbrace{\left(  2p_{n}+p_{n-1}\right)  }_{=p_{n+1}}-p_{n}%
^{2}=p_{n-1}p_{n+1}-p_{n}^{2}=p_{n+1}p_{n-1}-p_{n}^{2}=\left(  -1\right)  ^{n}%
\end{align*}
(by part \textbf{(d)} of the exercise). This solves part \textbf{(e)} of the exercise.

\subsection{Remark}

Let $r$ be any nonnegative real number.

Part \textbf{(a)} of this exercise remains valid if we replace each appearance
of \textquotedblleft$2$\textquotedblright\ by \textquotedblleft$r$%
\textquotedblright. (We could even allow $r$ to be negative, if we also
replace \textquotedblleft reals\textquotedblright\ by \textquotedblleft
complex numbers\textquotedblright.)

Parts \textbf{(b)} and \textbf{(c)} of this exercise remain valid if we
replace each appearance of \textquotedblleft$2$\textquotedblright\ by
\textquotedblleft$r$\textquotedblright, provided that $r$ is a positive
integer that is not a perfect square. (Of course, the $\sqrt{2}$-norm
$N_{2}\left(  \alpha\right)  $ needs to be replaced by the $\sqrt{r}$-norm
$N_{r}\left(  \alpha\right)  $, defined by setting $N_{r}\left(
\alpha\right)  =a^{2}-rb^{2}$ for $\alpha=a+b\sqrt{r}$.)

All three parts \textbf{(a)}, \textbf{(b)} and \textbf{(c)} of this exercise
remain valid if we replace integers by rational numbers throughout (i.e., we
consider the set of reals of the form $a+b\sqrt{2}$ with $a,b\in\mathbb{Q}$
instead of $a,b\in\mathbb{Z}$).

Part \textbf{(d)} of this exercise remains valid if we replace each appearance
of \textquotedblleft$2$\textquotedblright\ by \textquotedblleft$r$%
\textquotedblright. (Again, we could even allow $r$ to be negative.) Note that
if we set $r=1$, then the sequence $\left(  p_{0},p_{1},p_{2},\ldots\right)  $
becomes the famous
\href{https://en.wikipedia.org/wiki/Fibonacci_number}{Fibonacci sequence}
$\left(  0,1,1,2,3,5,8,13,21,34,\ldots\right)  $.

Part \textbf{(e)}, on the other hand, hinges on the specific properties of
$\sqrt{2}$.

%----------------------------------------------------------------------------------------
%	EXERCISE 5
%----------------------------------------------------------------------------------------
\rule{\linewidth}{0.3pt} \\[0.4cm]

\section{Exercise 5: Euler's theorem for non-coprime integers}

\subsection{Problem}

Let $a$ be an integer, and let $n$ be a positive integer. Prove that $a^{n}
\equiv a^{n - \phi\left(  n \right)  } \mod n$.

\subsection{Solution}

See \href{http://www-users.math.umn.edu/~dgrinber/19s/notes.pdf}{the class
notes}, where this is Exercise 2.16.3. (The numbering may shift; it is one of
the exercises in the ``The Chinese Remainder Theorem as a bijection'' section.)

%This exercise is also
%§2.3, problem 44 in Niven/Zuckerman/Montgomery
%(see kim-NT/sol3.pdf §14).


%----------------------------------------------------------------------------------------
%	EXERCISE 6
%----------------------------------------------------------------------------------------
\rule{\linewidth}{0.3pt} \\[0.4cm]

\section{Exercise 6: Wilson strikes again}

\subsection{Problem}

Let $p$ be an odd prime. Write $p$ in the form $p = 2k+1$ for some $k
\in\mathbb{N}$. Prove that $k!^{2} \equiv- \left(  -1 \right)  ^{k} \mod p$.

[\textbf{Hint:} Each $j \in\mathbb{Z}$ satisfies $j \left(  p-j \right)
\equiv-j^{2} \mod p$.]

\subsection{Solution}

See \href{http://www-users.math.umn.edu/~dgrinber/19s/notes.pdf}{the class
notes}, where this is Exercise 2.15.5. (The numbering may shift; it is one of
the exercises in the ``Fermat, Euler, Wilson'' section.)


\end{document}