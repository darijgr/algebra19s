% Like most advanced LaTeX files, this one begins with a lot of
% boilerplate. You don't need to understand (or even read) most of it.
% All you need to do is fill in your name, UMN ID, email address,
% and the number of the pset. (Search for "METADATA" to find the place
% for this.) Then, you can go straight to the "EXERCISE 1"
% section and start writing your solutions.
% The "VARIOUS USEFUL COMMANDS" section is probably worth taking a
% look at at some point.

%----------------------------------------------------------------------------------------
%	PACKAGES AND OTHER DOCUMENT CONFIGURATIONS
%----------------------------------------------------------------------------------------
\documentclass[paper=a4, fontsize=12pt]{scrartcl} % A4 paper and 12pt font size
\usepackage[T1]{fontenc} % Use 8-bit encoding that has 256 glyphs
\usepackage[english]{babel} % English language/hyphenation
\usepackage{amsmath,amsfonts,amsthm,amssymb} % Math packages
\usepackage{mathrsfs}    % More math packages
\usepackage{sectsty}  % Allows customizing section commands
\allsectionsfont{\centering \normalfont\scshape} % Make all section titles centered, the default font and small caps %remove this to left align section tites
\usepackage{hyperref} % Turns cross-references into hyperlinks,
                      % and defines \url and \href commands.
\usepackage{graphicx} % For embedding graphics files.
\usepackage{framed}   % For the "leftbar" environment used below.
\usepackage{ifthen}   % Used for the \powset command below.
\usepackage{lastpage} % for counting the number of pages
\usepackage{mathdots} % for \iddots
\usepackage[headsepline,footsepline,manualmark]{scrlayer-scrpage}
\usepackage[height=10in,a4paper,hmargin={1in,0.8in}]{geometry}
\usepackage[usenames,dvipsnames]{xcolor}
\usepackage{tikz}     % This is a powerful tool to draw vector
                      % graphics inside LaTeX. In particular, you can
                      % use it to draw graphs.
\usepackage{verbatim} % For the "verbatim" environment, in which
                      % special symbols can be used freely without
                      % confusing the compiler. (And it's typeset in
                      % a constant-width font.)
                      % Useful, e.g., for quoting code (or ASCII art).

%\numberwithin{table}{section} % Number tables within sections (i.e. 1.1, 1.2, 2.1, 2.2 instead of 1, 2, 3, 4)

\setlength\parindent{20pt} % Makes indentation for paragraphs longer.
                           % This makes paragraphs stand out more.

%----------------------------------------------------------------------------------------
%	VARIOUS USEFUL COMMANDS
%----------------------------------------------------------------------------------------
% The commands below might be convenient. For example, you probably
% prefer to write $\powset[2]{V}$ for the set of $2$-element subsets
% of $V$, rather than writing $\mathcal{P}_2(V)$.
% Notice that you can easily define your own commands like this.
% Caveat: Some of these commands need to be properly "guarded" when
% they occur in subscripts or superscripts. So you should not write
% $K_\CC$, but rather $K_{\CC}$.
\newcommand{\CC}{\mathbb{C}} % complex numbers
\newcommand{\RR}{\mathbb{R}} % real numbers
\newcommand{\QQ}{\mathbb{Q}} % rational numbers
\newcommand{\NN}{\mathbb{N}} % nonnegative integers
\newcommand{\DD}{{\mathbb{D}}} % dual numbers
\newcommand{\PP}{\mathbb{P}} % positive integers
\newcommand{\KK}{\mathbb{K}} % notation for a ring
\newcommand{\LL}{\mathbb{L}} % notation for a ring
\newcommand{\MM}{\mathbb{M}} % notation for a ring
\newcommand{\FF}{\mathbb{F}} % notation for a field
\newcommand{\Z}[1]{\mathbb{Z}/#1\mathbb{Z}} % integers modulo k
                                            % (syntax: "\Z{k}")
\newcommand{\ZZ}{\mathbb{Z}} % integers
\newcommand{\id}{\operatorname{id}} % identity map
\newcommand{\op}{\operatorname{op}} % opposite ring
\newcommand{\tildot}{\left. \widetilde{\cdot} \right.}
% I use \tildot for an alternative multiplication in one of the
% exercises below.
\newcommand{\lcm}{\operatorname{lcm}}
% Lowest common multiple. For historical reasons, LaTeX has a \gcd
% command built in, but not an \lcm command. The preceding line
% rectifies that.
\newcommand{\set}[1]{\left\{ #1 \right\}}
% $\set{...}$ compiles to {...} (set-brackets).
\newcommand{\abs}[1]{\left| #1 \right|}
% $\abs{...}$ compiles to |...| (absolute value, or size of a set).
\newcommand{\tup}[1]{\left( #1 \right)}
% $\tup{...}$ compiles to (...) (parentheses, or tuple-brackets).
\newcommand{\ive}[1]{\left[ #1 \right]}
% $\ive{...}$ compiles to [...] (Iverson bracket, aka truth value; also, set of first n integers).
\newcommand{\floor}[1]{\left\lfloor #1 \right\rfloor}
% $\floor{...}$ compiles to |_..._| (floor function).
\newcommand{\underbrack}[2]{\underbrace{#1}_{\substack{#2}}}
% $\underbrack{...1}{...2}$ yields
% $\underbrace{...1}_{\substack{...2}}$. This is useful for doing
% local rewriting transformations on mathematical expressions with
% justifications. For example, try this out:
% $ \underbrack{(a+b)^2}{= a^2 + 2ab + b^2 \\ \text{(by the binomial formula)}} $
\newcommand{\powset}[2][]{\ifthenelse{\equal{#2}{}}{\mathcal{P}\left(#1\right)}{\mathcal{P}_{#1}\left(#2\right)}}
% $\powset[k]{S}$ stands for the set of all $k$-element subsets of
% $S$. The argument $k$ is optional, and if not provided, the result
% is the whole powerset of $S$.
\newcommand{\mapeq}[1]{\underset{#1}{\equiv}}
% $\mapeq{f}$ yields $\underset{f}{\equiv}$.
% This is the "equality upon applying $f$" relation.
\newcommand{\eps}{\varepsilon}
% Just a shorthand for a commonly-used symbol.
\newcommand{\N}{\operatorname{N}} % norm of a complex number
\newcommand{\horrule}[1]{\rule{\linewidth}{#1}} % Create horizontal rule command with 1 argument of height
\newcommand{\nnn}{\nonumber\\} % Don't number this line in an "align" environment, and move on to the next line.

%----------------------------------------------------------------------------------------
%	MAKING SUMMATION SIGNS ALWAYS PUT THEIR BOUNDS ABOVE AND BELOW
%	THE SIGN
%----------------------------------------------------------------------------------------
% The following are hacks to ensure that sums (such as
% $\sum_{k=1}^n k$) always put their bounds (i.e., the $k=1$ and the
% $n$) underneath and above the sign, as opposed to on its right.
% Same for products (\prod), set unions (\bigcup) and set
% intersections (\bigcap). Remove the 8 lines below if you do not want
% this behavior.
\let\sumnonlimits\sum
\let\prodnonlimits\prod
\let\cupnonlimits\bigcup
\let\capnonlimits\bigcap
\renewcommand{\sum}{\sumnonlimits\limits}
\renewcommand{\prod}{\prodnonlimits\limits}
\renewcommand{\bigcup}{\cupnonlimits\limits}
\renewcommand{\bigcap}{\capnonlimits\limits}

%----------------------------------------------------------------------------------------
%	ENVIRONMENTS
%----------------------------------------------------------------------------------------
% The incantations below define how theorem environments
% (\begin{theorem} ... \end{theorem}) and their likes will look like.
\newtheoremstyle{plainsl}% <name>
  {8pt plus 2pt minus 4pt}% <Space above>
  {8pt plus 2pt minus 4pt}% <Space below>
  {\slshape}% <Body font>
  {0pt}% <Indent amount>
  {\bfseries}% <Theorem head font>
  {.}% <Punctuation after theorem head>
  {5pt plus 1pt minus 1pt}% <Space after theorem headi>
  {}% <Theorem head spec (can be left empty, meaning `normal')>

% Environments which make the text inside them slanted:
\theoremstyle{plainsl}
  \newtheorem{theorem}{Theorem}[section]
  \newtheorem{proposition}[theorem]{Proposition}
  \newtheorem{lemma}[theorem]{Lemma}
  \newtheorem{corollary}[theorem]{Corollary}
  \newtheorem{conjecture}[theorem]{Conjecture}
% Environments that don't:
\theoremstyle{definition}
  \newtheorem{definition}[theorem]{Definition}
  \newtheorem{example}[theorem]{Example}
  \newtheorem{exercise}[theorem]{Exercise}
  \newtheorem{examples}[theorem]{Examples}
  \newtheorem{algorithm}[theorem]{Algorithm}
  \newtheorem{question}[theorem]{Question}
 \theoremstyle{remark}
  \newtheorem{remark}[theorem]{Remark}
\newenvironment{statement}{\begin{quote}}{\end{quote}}
\newenvironment{fineprint}{\begin{small}}{\end{small}}

%----------------------------------------------------------------------------------------
%	METADATA
%----------------------------------------------------------------------------------------
\newcommand{\myname}{Darij Grinberg} % ENTER YOUR NAME HERE
\newcommand{\myid}{00000000} % ENTER YOUR UMN ID HERE
\newcommand{\mymail}{dgrinber@umn.edu} % ENTER YOUR EMAIL HERE
\newcommand{\psetnumber}{6} % ENTER THE NUMBER OF THIS PSET HERE

%----------------------------------------------------------------------------------------
%	HEADER AND FOOTER
%----------------------------------------------------------------------------------------
\ihead{Solutions to homework set \#\psetnumber} % Page header left
\ohead{page \thepage\ of \pageref{LastPage}} % Page header right
\ifoot{\myname, \myid} % left footer
\ofoot{\mymail} % right footer

%----------------------------------------------------------------------------------------
%	TITLE SECTION
%----------------------------------------------------------------------------------------
\title{	
\normalfont \normalsize 
\textsc{University of Minnesota, School of Mathematics} \\ [25pt] % Your university, school and/or department name(s)
\horrule{0.5pt} \\[0.4cm] % Thin top horizontal rule
\huge Math 4281: Introduction to Modern Algebra, \\
Spring 2019:
Homework \psetnumber\\% The assignment title
\horrule{2pt} \\[0.5cm] % Thick bottom horizontal rule
}
\author{\myname}

\begin{document}

\maketitle % Print the title

\begin{center} % Delete this if you want to save space!
{\large due date: \textbf{by Canvas or email till the beginning of class on Monday, 22 April 2019}.

Please solve \textbf{at most 3 of the 6 exercises}!}
\end{center}

%----------------------------------------------------------------------------------------
%	EXERCISE 1
%----------------------------------------------------------------------------------------
\horrule{0.3pt} \\[0.4cm]

\section{Exercise 1: The opposite ring}

Let $\KK$ be a ring.
We define a new binary operation $\tildot$ on $\KK$ by setting
\[
a \tildot b = ba \qquad \text{for all } a, b \in \KK .
\]
(Thus, $\tildot$ is the multiplication of $\KK$, but with the
arguments switched.)

\begin{enumerate}

\item[\textbf{(a)}] Prove that the set $\KK$, equipped with the
addition $+$, the multiplication $\tildot$, the zero $0_{\KK}$
and the unity $1_{\KK}$, is a ring.

\end{enumerate}

\noindent This new ring is called the \textit{opposite ring} of $\KK$,
and is denoted by $\KK^{\op}$.

Note that the \textbf{sets} $\KK$ and $\KK^{\op}$ are identical
(so a map from $\KK$ to $\KK$ is the same as a map from
$\KK$ to $\KK^{\op}$);
but the \textbf{rings} $\KK$ and $\KK^{\op}$ are generally
not the same
(so a ring homomorphism from $\KK$ to $\KK$ is not the same as a
ring homomorphism from $\KK$ to $\KK^{\op}$).

\begin{enumerate}

\item[\textbf{(b)}] Prove that the identity map
$\id : \KK \to \KK$ is a ring isomorphism from $\KK$ to
$\KK^{\op}$ if and only if $\KK$ is commutative.

\item[\textbf{(c)}] Now, assume that $\KK$ is the matrix ring
$\LL^{n \times n}$ for some commutative ring $\LL$ and some
$n \in \NN$.
Prove that the map
\[
\KK \to \KK^{\op}, \qquad A \mapsto A^T
\]
(where $A^T$, as usual, denotes the transpose of a matrix $A$)
is a ring isomorphism.

\end{enumerate}

[\textbf{Hint:} In \textbf{(a)}, you only have to check the ring axioms that
have to do with multiplication.
Similarly, in \textbf{(b)}, you are free to check the one axiom relating
to multiplication only.
In \textbf{(c)}, you can use \cite[Exercise 6.5]{detnotes} without proof.]

\subsection{Remark}

This exercise gives some examples of rings $\KK$ that are isomorphic
to their opposite rings $\KK^{\op}$. See
\url{https://mathoverflow.net/questions/64370/} for examples of rings
that are not.

\subsection{Solution}

[...]

%----------------------------------------------------------------------------------------
%	EXERCISE 2
%----------------------------------------------------------------------------------------
\horrule{0.3pt} \\[0.4cm]

\section{Exercise 2: More ring isomorphisms}

\subsection{Problem}

\begin{enumerate}

\item[\textbf{(a)}]
Let $\LL$ be a ring.
Let $w \in \LL$ be an invertible element.
Prove that the map
\[
\LL \to \LL , \qquad a \mapsto waw^{-1}
\]
is a ring isomorphism.

\item[\textbf{(b)}] Let $\KK$ be a ring.
Let $W$ be the $n \times n$-matrix
\[
\tup{ \ive{ i + j = n + 1 } } _{1\leq i\leq n,\ 1\leq j\leq n}
=
\begin{pmatrix}
0 & \cdots & 0 & 0 & 1\\
0 & \cdots & 0 & 1 & 0\\
0 & \cdots & 1 & 0 & 0\\
\vdots & \iddots & \vdots & \vdots & \vdots\\
1 & \cdots & 0 & 0 & 0
\end{pmatrix}
\in \KK^{n \times n}
\]
(where we are using
\href{https://en.wikipedia.org/wiki/Iverson_bracket}{the
Iverson bracket notation} again).

Prove that $W = W^{-1}$.

\item[\textbf{(c)}] Let
$A = \tup{ a_{i,j} }_{1\leq i\leq n,\ 1\leq j\leq n}
\in \KK^{n \times n}$ be any $n \times n$-matrix.
Prove that
\[
WAW^{-1}
= \tup{ a_{n+1-i,n+1-j} }_{1\leq i\leq n,\ 1\leq j\leq n}.
\]
(In other words, $WAW^{-1}$ is the $n \times n$-matrix obtained from $A$ by
reversing the order of the rows and also reversing the order of the columns.)

\end{enumerate}

\subsection{Remark}

The map
\[
\LL \to \LL , \qquad a \mapsto waw^{-1}
\]
in part \textbf{(a)} of this exercise is called \textit{conjugation by $w$}.
It is best known in the case of a matrix ring, where it corresponds to a
change of basis for an endomorphism of a vector space. When $\KK$ is a
field, the \textbf{only} ring isomorphisms $\KK^{n \times n}
\to \KK^{n \times n}$ are conjugations by invertible matrices;
this is the Noether--Skolem theorem (in one of its less general variants).

\subsection{Solution}

[...]

%----------------------------------------------------------------------------------------
%	EXERCISE 3
%----------------------------------------------------------------------------------------
\horrule{0.3pt} \\[0.4cm]

\section{Exercise 3: Entangled inverses}

Let $\KK$ be a ring.

A \textit{left inverse} of an element $x \in \KK$
is defined to be a $y \in \KK$ such that $yx = 1$.

A \textit{right inverse} of an element $x \in \KK$
is defined to be a $y \in \KK$ such that $xy = 1$.

Let $a$ and $b$ be two elements of $\KK$.
Prove the following:

\begin{enumerate}

\item[\textbf{(a)}] If $c$ is a left inverse of $1 - ab$,
then $1 + bca$ is a left inverse of $1 - ba$.

\item[\textbf{(b)}] If $c$ is a right inverse of $1 - ab$,
then $1 + bca$ is a right inverse of $1 - ba$.

\item[\textbf{(c)}] If $c$ is an inverse of $1 - ab$,
then $1 + bca$ is an inverse of $1 - ba$.

\end{enumerate}

Here and in the following, the word ``\textit{inverse}''
(unless qualified with an adjective) means
``multiplicative inverse''.

\subsection{Solution}

[...]

%----------------------------------------------------------------------------------------
%	EXERCISE 4
%----------------------------------------------------------------------------------------
\horrule{0.3pt} \\[0.4cm]

\section{Exercise 4: Composition of ring homomorphisms}

\subsection{Problem}

Let $\KK$, $\LL$ and $\MM$ be three rings.
Prove the following:

\begin{enumerate}
\item[\textbf{(a)}] If $f : \KK \to \LL$ and $g : \LL \to \MM$
are two ring homomorphisms, then $g \circ f : \KK \to \MM$
is a ring homomorphism.

\item[\textbf{(b)}] If $f : \KK \to \LL$ and $g : \LL \to \MM$
are two ring isomorphisms, then $g \circ f : \KK \to \MM$
is a ring isomorphism.
\end{enumerate}

\subsection{Solution}

[...]

%----------------------------------------------------------------------------------------
%	EXERCISE 5
%----------------------------------------------------------------------------------------
\horrule{0.3pt} \\[0.4cm]

\section{Exercise 5: Squares in finite fields I}

\subsection{Problem}

Let $\FF$ be a field.

\begin{enumerate}

\item[\textbf{(a)}] Prove that if $a, b \in \FF$ satisfy
$ab = 0$, then $a = 0$ or $b = 0$.

\item[\textbf{(b)}] Prove that if $a, b \in \FF$ satisfy
$a^2 = b^2$, then $a = b$ or $a = -b$.

\end{enumerate}

Recall that an element $\eta \in \FF$ is called a
\textit{square} if there exists some $\alpha \in \FF$
such that $\eta = \alpha^2$.

From now on, assume that $2 \cdot 1_{\FF} \neq 0_{\FF}$ (that is,
$1_{\FF} + 1_{\FF} \neq 0_{\FF}$). Note that this is
satisfied whenever $\FF = \ZZ / p$ for a prime $p > 2$ (but also for
various other finite fields), but fails when $\FF = \ZZ / 2$.

\begin{enumerate}

\item[\textbf{(c)}] Prove that $a \neq -a$ for every
nonzero $a \in \FF$.

\end{enumerate}

From now on, assume that $\FF$ is finite.

\begin{enumerate}

\item[\textbf{(d)}] Prove that the number of squares in $\FF$ is
$\dfrac{1}{2} \tup{ \abs{\FF} + 1 }$.

\item[\textbf{(e)}] Conclude that $\abs{\FF}$ is odd.

\end{enumerate}

[\textbf{Hint:} For part \textbf{(d)}, argue that each nonzero square in
$\FF$ can be written as $\alpha^2$ for exactly two $\alpha \in \FF$.]

\subsection{Solution}

[...]

%----------------------------------------------------------------------------------------
%	EXERCISE 6
%----------------------------------------------------------------------------------------
\horrule{0.3pt} \\[0.4cm]

\section{Exercise 6: The characteristic of a field}

\subsection{Problem}

Let $\FF$ be a field.
Recall that we have defined $na$ to mean
$\underbrace{a + a + \cdots + a}_{n \text{ times}}$ whenever
$n \in \NN$ and $a \in \FF$.

Assume that there exists a positive integer $n$ such that
$n \cdot 1_{\FF} = 0$.
Let $p$ be the \textbf{smallest} such $n$.

Prove that $p$ is prime.

[\textbf{Hint:}
$\tup{a \cdot 1_{\FF}} \cdot \tup{b \cdot 1_{\FF}}
= ab \cdot 1_{\FF}$
for all $a, b \in \NN$.]

\subsection{Remark}

The $p$ we just defined is called the \textit{characteristic} of the field
$\FF$ when it exists. (Otherwise, the characteristic of the field
$\FF$ is defined to be $0$.)

Thus, for each prime $p$, the finite field $\ZZ / p$, as well as the
finite field of size $p^2$ that we constructed in class,
have characteristic $p$.

\subsection{Solution}

[...]

\begin{thebibliography}{99999999}                                                                                         %

% Feel free to add your sources -- or copy some from the source code
% of the class notes ( http://www.cip.ifi.lmu.de/~grinberg/t/19s/notes.tex ).

% This is the bibliography: The list of papers/books/articles/blogs/...
% cited. The syntax is: "\bibitem[name]{tag}Reference",
% where "name" is the name that will appear in the compiled
% bibliography, and "tag" is the tag by which you will refer to
% the source in the TeX file. For example, the following source
% has name "GrKnPa94" (so you will see it referenced as
% "[GrKnPa94]" in the compiled PDF) and tag "GKP" (so you
% can cite it by writing "\cite{GKP}").

% I've commented the below references out, since I'm not citing
% them above.

% \bibitem[GrKnPa94]{GKP}Ronald L. Graham, Donald E. Knuth, Oren Patashnik,
% \textit{Concrete Mathematics, Second Edition}, Addison-Wesley 1994.\\
% See \url{https://www-cs-faculty.stanford.edu/~knuth/gkp.html} for errata.

\bibitem[Grinbe19]{detnotes}Darij Grinberg,
\textit{Notes on the combinatorial fundamentals of algebra},
10 January 2019. \\
\url{http://www.cip.ifi.lmu.de/~grinberg/primes2015/sols.pdf}
\\
The numbering of theorems and formulas in this link might shift
when the project gets updated; for a ``frozen'' version whose
numbering is guaranteed to match that in the citations above, see
\url{https://github.com/darijgr/detnotes/releases/tag/2019-01-10} .

\end{thebibliography}

\end{document}
