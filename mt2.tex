% Like most advanced LaTeX files, this one begins with a lot of
% boilerplate. You don't need to understand (or even read) most of it.
% All you need to do is fill in your name, UMN ID, email address,
% and the number of the pset. (Search for "METADATA" to find the place
% for this.) Then, you can go straight to the "EXERCISE 1"
% section and start writing your solutions.
% The "VARIOUS USEFUL COMMANDS" section is probably worth taking a
% look at at some point.

%----------------------------------------------------------------------------------------
%	PACKAGES AND OTHER DOCUMENT CONFIGURATIONS
%----------------------------------------------------------------------------------------
\documentclass[paper=a4, fontsize=12pt]{scrartcl} % A4 paper and 12pt font size
\usepackage[T1]{fontenc} % Use 8-bit encoding that has 256 glyphs
\usepackage[english]{babel} % English language/hyphenation
\usepackage{amsmath,amsfonts,amsthm,amssymb} % Math packages
\usepackage{mathrsfs}    % More math packages
\usepackage{sectsty}  % Allows customizing section commands
\allsectionsfont{\centering \normalfont\scshape} % Make all section titles centered, the default font and small caps %remove this to left align section tites
\usepackage{hyperref} % Turns cross-references into hyperlinks,
                      % and defines \url and \href commands.
\usepackage{graphicx} % For embedding graphics files.
\usepackage{framed}   % For the "leftbar" environment used below.
\usepackage{ifthen}   % Used for the \powset command below.
\usepackage{lastpage} % for counting the number of pages
\usepackage[headsepline,footsepline,manualmark]{scrlayer-scrpage}
\usepackage[height=10in,a4paper,hmargin={1in,0.8in}]{geometry}
\usepackage[usenames,dvipsnames]{xcolor}
\usepackage{tikz}     % This is a powerful tool to draw vector
                      % graphics inside LaTeX. In particular, you can
                      % use it to draw graphs.
\usepackage{verbatim} % For the "verbatim" environment, in which
                      % special symbols can be used freely without
                      % confusing the compiler. (And it's typeset in
                      % a constant-width font.)
                      % Useful, e.g., for quoting code (or ASCII art).

%\numberwithin{table}{section} % Number tables within sections (i.e. 1.1, 1.2, 2.1, 2.2 instead of 1, 2, 3, 4)

\setlength\parindent{20pt} % Makes indentation for paragraphs longer.
                           % This makes paragraphs stand out more.

%----------------------------------------------------------------------------------------
%	VARIOUS USEFUL COMMANDS
%----------------------------------------------------------------------------------------
% The commands below might be convenient. For example, you probably
% prefer to write $\powset[2]{V}$ for the set of $2$-element subsets
% of $V$, rather than writing $\mathcal{P}_2(V)$.
% Notice that you can easily define your own commands like this.
% Caveat: Some of these commands need to be properly "guarded" when
% they occur in subscripts or superscripts. So you should not write
% $K_\CC$, but rather $K_{\CC}$.
\newcommand{\CC}{\mathbb{C}} % complex numbers
\newcommand{\RR}{\mathbb{R}} % real numbers
\newcommand{\QQ}{\mathbb{Q}} % rational numbers
\newcommand{\NN}{\mathbb{N}} % nonnegative integers
\newcommand{\PP}{\mathbb{P}} % positive integers
\newcommand{\KK}{\mathbb{K}} % my favorite notation for a commutative ring
\newcommand{\Z}[1]{\mathbb{Z}/#1\mathbb{Z}} % integers modulo k
                                            % (syntax: "\Z{k}")
\newcommand{\ZZ}{\mathbb{Z}} % integers
\newcommand{\id}{\operatorname{id}} % identity map
\newcommand{\lcm}{\operatorname{lcm}}
% Lowest common multiple. For historical reasons, LaTeX has a \gcd
% command built in, but not an \lcm command. The preceding line
% rectifies that.
\newcommand{\set}[1]{\left\{ #1 \right\}}
% $\set{...}$ compiles to {...} (set-brackets).
\newcommand{\abs}[1]{\left| #1 \right|}
% $\abs{...}$ compiles to |...| (absolute value, or size of a set).
\newcommand{\tup}[1]{\left( #1 \right)}
% $\tup{...}$ compiles to (...) (parentheses, or tuple-brackets).
\newcommand{\ive}[1]{\left[ #1 \right]}
% $\ive{...}$ compiles to [...] (Iverson bracket, aka truth value; also, set of first n integers).
\newcommand{\floor}[1]{\left\lfloor #1 \right\rfloor}
% $\floor{...}$ compiles to |_..._| (floor function).
\newcommand{\underbrack}[2]{\underbrace{#1}_{\substack{#2}}}
% $\underbrack{...1}{...2}$ yields
% $\underbrace{...1}_{\substack{...2}}$. This is useful for doing
% local rewriting transformations on mathematical expressions with
% justifications. For example, try this out:
% $ \underbrack{(a+b)^2}{= a^2 + 2ab + b^2 \\ \text{(by the binomial formula)}} $
\newcommand{\powset}[2][]{\ifthenelse{\equal{#2}{}}{\mathcal{P}\left(#1\right)}{\mathcal{P}_{#1}\left(#2\right)}}
% $\powset[k]{S}$ stands for the set of all $k$-element subsets of
% $S$. The argument $k$ is optional, and if not provided, the result
% is the whole powerset of $S$.
\newcommand{\calF}{\mathcal{F}}
\newcommand{\horrule}[1]{\rule{\linewidth}{#1}} % Create horizontal rule command with 1 argument of height
\newcommand{\nnn}{\nonumber\\} % Don't number this line in an "align" environment, and move on to the next line.

%----------------------------------------------------------------------------------------
%	MAKING SUMMATION SIGNS ALWAYS PUT THEIR BOUNDS ABOVE AND BELOW
%	THE SIGN
%----------------------------------------------------------------------------------------
% The following are hacks to ensure that sums (such as
% $\sum_{k=1}^n k$) always put their bounds (i.e., the $k=1$ and the
% $n$) underneath and above the sign, as opposed to on its right.
% Same for products (\prod), set unions (\bigcup) and set
% intersections (\bigcap). Remove the 8 lines below if you do not want
% this behavior.
\let\sumnonlimits\sum
\let\prodnonlimits\prod
\let\cupnonlimits\bigcup
\let\capnonlimits\bigcap
\renewcommand{\sum}{\sumnonlimits\limits}
\renewcommand{\prod}{\prodnonlimits\limits}
\renewcommand{\bigcup}{\cupnonlimits\limits}
\renewcommand{\bigcap}{\capnonlimits\limits}

%----------------------------------------------------------------------------------------
%	ENVIRONMENTS
%----------------------------------------------------------------------------------------
% The incantations below define how theorem environments
% (\begin{theorem} ... \end{theorem}) and their likes will look like.
\newtheoremstyle{plainsl}% <name>
  {8pt plus 2pt minus 4pt}% <Space above>
  {8pt plus 2pt minus 4pt}% <Space below>
  {\slshape}% <Body font>
  {0pt}% <Indent amount>
  {\bfseries}% <Theorem head font>
  {.}% <Punctuation after theorem head>
  {5pt plus 1pt minus 1pt}% <Space after theorem headi>
  {}% <Theorem head spec (can be left empty, meaning `normal')>

% Environments which make the text inside them slanted:
\theoremstyle{plainsl}
  \newtheorem{theorem}{Theorem}[section]
  \newtheorem{proposition}[theorem]{Proposition}
  \newtheorem{lemma}[theorem]{Lemma}
  \newtheorem{corollary}[theorem]{Corollary}
  \newtheorem{conjecture}[theorem]{Conjecture}
% Environments that don't:
\theoremstyle{definition}
  \newtheorem{definition}[theorem]{Definition}
  \newtheorem{example}[theorem]{Example}
  \newtheorem{exercise}[theorem]{Exercise}
  \newtheorem{examples}[theorem]{Examples}
  \newtheorem{algorithm}[theorem]{Algorithm}
  \newtheorem{question}[theorem]{Question}
 \theoremstyle{remark}
  \newtheorem{remark}[theorem]{Remark}
\newenvironment{statement}{\begin{quote}}{\end{quote}}
\newenvironment{fineprint}{\begin{small}}{\end{small}}

%----------------------------------------------------------------------------------------
%	METADATA
%----------------------------------------------------------------------------------------
\newcommand{\myname}{Darij Grinberg} % ENTER YOUR NAME HERE
\newcommand{\myid}{00000000} % ENTER YOUR UMN ID HERE
\newcommand{\mymail}{dgrinber@umn.edu} % ENTER YOUR EMAIL HERE
\newcommand{\psetnumber}{2} % ENTER THE NUMBER OF THIS PSET HERE

%----------------------------------------------------------------------------------------
%	HEADER AND FOOTER
%----------------------------------------------------------------------------------------
\ihead{Solutions to midterm \#\psetnumber} % Page header left
\ohead{page \thepage\ of \pageref{LastPage}} % Page header right
\ifoot{\myname, \myid} % left footer
\ofoot{\mymail} % right footer

%----------------------------------------------------------------------------------------
%	TITLE SECTION
%----------------------------------------------------------------------------------------
\title{	
\normalfont \normalsize 
\textsc{University of Minnesota, School of Mathematics} \\ [25pt] % Your university, school and/or department name(s)
\horrule{0.5pt} \\[0.4cm] % Thin top horizontal rule
\huge Math 4281: Introduction to Modern Algebra, \\
Spring 2019:
Midterm \psetnumber\\% The assignment title
\horrule{2pt} \\[0.5cm] % Thick bottom horizontal rule
}
\author{\myname}

\begin{document}

\maketitle % Print the title

\begin{center} % Delete this if you want to save space!
{\large due date: \textbf{Friday, 12 April 2019} at the beginning of class, \\
or before that through Canvas.

\textbf{No collaboration allowed} -- this is a midterm.

Please solve \textbf{at most 3 of the 6 exercises}!}
\end{center}

%----------------------------------------------------------------------------------------
%	EXERCISE 1
%----------------------------------------------------------------------------------------
\horrule{0.3pt} \\[0.4cm]

\section{Exercise 1: Not-quite-all-rationals}

\subsection{Problem}

Fix an integer $m$.
An \textit{$m$-integer} shall mean a rational number
$r$ such that there exists a $k \in \NN$ satisfying
$m^k r \in \ZZ$.

For example\footnote{You don't need to prove these.}:
\begin{itemize}
 \item Each integer $r$ is an $m$-integer (since $m^k r \in \ZZ$
       for $k = 0$).
 \item The rational number $\dfrac{5}{12}$
       is a $6$-integer (since $6^k \cdot \dfrac{5}{12} \in \ZZ$
       for $k = 2$), but neither a $2$-integer nor a $3$-integer
       (since multiplying it by a power of $2$ will not ``get
       rid of'' the prime factor $3$ in the denominator, and
       vice versa\footnote{You would have to be more rigorous
       than this in your solution, if you were to make an
       argument like this.}).
 \item The $1$-integers are the integers (since $1^k r = r$
       for all $r$).
 \item Every rational number $r$ is a $0$-integer (since
       $0^k r \in \ZZ$ for $k = 1$).
\end{itemize}

Let $R_m$ denote the set of all $m$-integers.
Prove the following:

\begin{enumerate}

\item[\textbf{(a)}] The set $R_m$ (endowed with the usual
addition, the usual multiplication, the usual integer $0$
as zero, and the usual integer $1$ as unity) is a
commutative ring. \\
(You don't need to prove axioms like commutativity of
multiplication, since these follow from the corresponding
facts about rational numbers, which are well-known.
You only need to check that $R_m$ is
closed under addition and multiplication\footnote{This
means that every $a, b \in R_m$ satisfy $a + b \in R_m$
and $a b \in R_m$.}, and contains
additive inverses of all its elements.)

\item[\textbf{(b)}] Let $x \in \QQ$ be nonzero.
Then, $x \in R_m$ if and only if every prime $p$ satisfying
$w_p\tup{x} < 0$ satisfies $p \mid m$.
Here, we are using the notation $w_p\tup{r}$ defined
in Exercise 3.4.1 of the
\href{http://www.cip.ifi.lmu.de/~grinberg/t/19s/notes.pdf}{class notes}.

\end{enumerate}

\subsection{Remark}

The ring $R_m$ is an example of a ring ``between $\ZZ$ and $\QQ$''.
Note that $R_1 = \ZZ$ and $R_0 = \QQ$, whereas $R_2 = R_4 = R_8 = \cdots$
is the ring of all rational numbers that can be written in the form
$a / 2^k$ with $a \in \ZZ$ and $k \in \NN$.

\subsection{Solution}

[...]

%----------------------------------------------------------------------------------------
%	EXERCISE 2
%----------------------------------------------------------------------------------------
\horrule{0.3pt} \\[0.4cm]

\section{Exercise 2: Rings with $x^2 = x$}

\subsection{Problem}

Let $\KK$ be a ring with the property that
\begin{align}
 u^2 = u \qquad \text{ for all } u \in \KK .
 \label{eq.exe.ring.xx=x.cond}
\end{align}

(Examples of such rings are $\ZZ / 2$ as well as the
``power set'' ring $\tup{\powset{S}, \triangle, \cap, \varnothing, S}$
constructed from any given set $S$.)

Prove the following:

\begin{enumerate}

\item[\textbf{(a)}]
We have $2x = 0$ for each $x \in \KK$.

\item[\textbf{(b)}]
We have $-x = x$ for each $x \in \KK$.

\item[\textbf{(c)}]
We have $xy = yx$ for all $x, y \in \KK$.
(In other words, the ring $\KK$ is commutative.)

\end{enumerate}

(As usual, ``$0$'' stands for the zero of the ring $\KK$.)

[\textbf{Hint:}
For part \textbf{(a)},
apply \eqref{eq.exe.ring.xx=x.cond} to $u = x$
but also to $u = 2x = x + x$, and see what comes out.
For part \textbf{(c)},
apply \eqref{eq.exe.ring.xx=x.cond} to $u = x + y$.]

\subsection{Remark}

You might wonder what happens if we replace
\eqref{eq.exe.ring.xx=x.cond} by
\begin{align}
 u^3 = u \qquad \text{ for all } u \in \KK .
 \label{eq.exe.ring.xx=x.cube}
\end{align}
This no longer leads to $2x = 0$ (nor to $3x = 0$ as you
might perhaps expect).
Instead, it can be shown that $6x = 0$ for all $x \in \KK$.
It can also be shown that it leads to $xy = yx$.

\subsection{Solution}

[...]

%----------------------------------------------------------------------------------------
%	EXERCISE 3
%----------------------------------------------------------------------------------------
\horrule{0.3pt} \\[0.4cm]

\section{Exercise 3: A matrix of gcds}

\subsection{Problem}

In this exercise, we shall again use
\href{https://en.wikipedia.org/wiki/Iverson_bracket}{the \textit{Iverson bracket notation}}:

Let $n \in \NN$.
Let $G$ be the $n \times n$-matrix
\[
 \tup{ \gcd\tup{i, j} }_{1 \leq i \leq n, \ 1 \leq j \leq n }
 =
 \begin{pmatrix}   % "pmatrix" stands for "parenthesis-bounded matrix".
                   % That's how we write matrices in this class.
  \gcd\tup{1, 1} & \gcd\tup{1, 2} & \cdots & \gcd\tup{1, n} \\
  \gcd\tup{2, 1} & \gcd\tup{2, 2} & \cdots & \gcd\tup{2, n} \\
  \vdots & \vdots & \ddots & \vdots \\
  \gcd\tup{n, 1} & \gcd\tup{n, 2} & \cdots & \gcd\tup{n, n}
 \end{pmatrix} .
\]

Let $L$ be the $n \times n$-matrix
\[
 \tup{ \ive{j \mid i} }_{1 \leq i \leq n, \ 1 \leq j \leq n }
 =
 \begin{pmatrix}
  \ive{1 \mid 1} & \ive{2 \mid 1} & \cdots & \ive{n \mid 1} \\
  \ive{1 \mid 2} & \ive{2 \mid 2} & \cdots & \ive{n \mid 2} \\
  \vdots & \vdots & \ddots & \vdots \\
  \ive{1 \mid n} & \ive{2 \mid n} & \cdots & \ive{n \mid n}
 \end{pmatrix} .
\]

Let $D$ be the $n \times n$-matrix
\[
 \tup{ \ive{i = j} \phi\tup{i} }_{1 \leq i \leq n, \ 1 \leq j \leq n }
 =
 \begin{pmatrix}
  \phi\tup{1} & 0 & 0 & \cdots & 0 \\
  0 & \phi\tup{2} & 0 & \cdots & 0 \\
  0 & 0 & \phi\tup{3} & \cdots & 0 \\
  \vdots & \vdots & \vdots & \ddots & \vdots \\
  0 & 0 & 0 & \cdots & \phi\tup{n}
 \end{pmatrix} .
\]

Prove that\footnote{We are using the standard notation $A^T$ for
the \textit{transpose} of a matrix $A$. This transpose is defined
as follows:
If $A = \tup{ a_{i, j} }_{1 \leq i \leq n, \ 1 \leq j \leq m }$,
then
$A^T = \tup{ a_{j, i} }_{1 \leq i \leq m, \ 1 \leq j \leq n }$.}
$G = LDL^T$.

[\textbf{Hint:} Diagonal matrices are particularly easy
to multiply with other matrices. For example, given a
diagonal matrix
$\mathbf{D} = \tup{ \ive{i = j} d_i }_{1 \leq i \leq n, \ 1 \leq j \leq n }$
(where $d_1, d_2, \ldots, d_n$ are $n$ elements of a ring $\KK$)
and an arbitrary matrix
$\mathbf{A} = \tup{ a_{i, j} }_{1 \leq i \leq n, \ 1 \leq j \leq m }$
(over the same ring $\KK$),
the product $\mathbf{D} \mathbf{A}$ is simply given by
\begin{align}
 \mathbf{D} \mathbf{A} =
 \tup{ d_i a_{i, j} }_{1 \leq i \leq n, \ 1 \leq j \leq m } .
 \label{eq.matrix.gcd-LDU.row-mult}
\end{align}
(That is, multiplying by a diagonal matrix on the left is tantamount
to rescaling each row by the corresponding entry of the diagonal
matrix.)
You can use the formula \eqref{eq.matrix.gcd-LDU.row-mult} without
proof (though its easy proof is instructive).]

\subsection{Remark}

As the names suggest, the matrix $L$ is lower-triangular
(so that the matrix $L^T$ is upper-triangular),
and the matrix $D$ is diagonal.
Thus, $G = LDL^T$ is an instance of an
\href{https://en.wikipedia.org/wiki/LU_decomposition#LDU_decomposition}{\textit{LDU decomposition}}.

\subsection{Solution}

[...]

%----------------------------------------------------------------------------------------
%	EXERCISE 4
%----------------------------------------------------------------------------------------
\horrule{0.3pt} \\[0.4cm]

\section{Exercise 4: Idempotent and involutive elements}

\subsection{Problem}

Let $\KK$ be a ring.

An element $a$ of $\KK$ is said to be \textit{idempotent}
if it satisfies $a^2 = a$.

An element $a$ of $\KK$ is said to be \textit{involutive}
if it satisfies $a^2 = 1$.

\begin{enumerate}
 \item[\textbf{(a)}] Let $a \in \KK$.
 Prove that if $a$ is idempotent, then $1 - 2a$ is
 involutive.
 
 \item[\textbf{(b)}] Now, assume that $2$ is
 \textit{cancellable} in $\KK$; this means that if
 $u$ and $v$ are two elements of $\KK$ satisfying
 $2u = 2v$, then $u = v$.
 Prove that the converse of the claim of part
 \textbf{(a)} holds:
 If $a \in \KK$ is such that $1 - 2a$ is involutive,
 then $a$ is idempotent.
 
 \item[\textbf{(c)}] Now, let $\KK = \ZZ / 4$.
 Find an element $a \in \KK$ such that $1 - 2a$
 is involutive, but $a$ is not idempotent.
\end{enumerate}

\subsection{Remark}

The idempotent elements of $\RR$ are $0$ and $1$.
The involutive elements of $\RR$ are $1$ and $-1$.
A matrix ring like $\RR^{n \times n}$ usually has
infinitely many idempotent elements (viz., all
projection matrices on subspaces of $\RR^n$) and
infinitely many involutive elements (viz., all
matrices $A$ satisfying $A^2 = I_n$; for instance,
all reflections across hyperplanes are represented
by such matrices).

Part \textbf{(a)} of this exercise assigns an involutive
element to each idempotent element of $\KK$.
If $2$ is invertible in $\KK$ (that is, if the element
$2 \cdot 1_{\KK}$ has a multiplicative inverse), then this
assignment is a bijection (as can be easily derived from
part \textbf{(b)}).
Part \textbf{(c)} shows that we cannot drop the
``$2$ is cancellable'' condition in part \textbf{(b)}.

\subsection{Solution}

[...]

%----------------------------------------------------------------------------------------
%	EXERCISE 5
%----------------------------------------------------------------------------------------
\horrule{0.3pt} \\[0.4cm]

\section{Exercise 5: The matrix approach to Fibonacci numbers}

\subsection{Problem}

Let $A$ be the $2 \times 2$-matrix
$\begin{pmatrix}
  0 & 1 \\
  1 & 1
 \end{pmatrix}$
over $\ZZ$.
Consider also the identity matrix $I_2$.

Let $\calF$ % I have defined \calF to mean \mathcal{F} for this problem.
be the subset
\[
 \set{ aA + bI_2 \mid a, b \in \ZZ }
 =
 \set{ \begin{pmatrix} b & a \\ a & a+b \end{pmatrix} \mid a, b \in \ZZ }
\]
of the matrix ring $\ZZ^{2 \times 2}$.

\begin{enumerate}

\item[\textbf{(a)}]
Prove that $A^2 = A + I_2$.

\item[\textbf{(b)}]
Prove that the set $\calF$ (equipped with the addition of matrices,
the multiplication of matrices, the zero $0_{2 \times 2}$ and
the unity $I_2$) is a commutative ring. \\
(Again, you don't need to check the ring axioms, as we already
know that they hold for arbitrary matrices and thus all the more
for matrices in $\calF$.
But you do need to check commutativity of multiplication in
$\calF$, since it does not hold for arbitrary matrices.
You also need to check that $\calF$ is closed under addition
and multiplication and has additive inverses.)

\end{enumerate}

Let $\tup{f_0, f_1, f_2, \ldots}$ be the Fibonacci sequence (which
we have already encountered on
\href{http://www.cip.ifi.lmu.de/~grinberg/t/19s/hw5s.pdf}{homework set \#5}).
Recall that it is defined recursively by
\[
f_0 = 0, \qquad
f_1 = 1, \qquad \text{and} \qquad
f_n = f_{n-1} + f_{n-2} \text{ for all } n \geq 2 .
\]

\begin{enumerate}

\item[\textbf{(c)}]
Prove that $A^n = f_n A + f_{n-1} I_2$ for all positive
integers $n$.

\item[\textbf{(d)}]
Prove that
$f_{n+m} = f_n f_{m+1} + f_{n-1} f_m$ for all positive
integers $n$ and all $m \in \NN$.

\end{enumerate}

Now, define a further matrix $B \in \calF$ by
$B = \tup{-1}A + 1I_2 = I_2 - A$.

\begin{enumerate}

\item[\textbf{(e)}]
Prove that $B^2 = B + I_2$ and
$B^n = f_n B + f_{n-1} I_2$ for all positive
integers $n$.

\item[\textbf{(f)}]
Prove that
$A^n - B^n = f_n \tup{A - B}$ for all $n \in \NN$.

\item[\textbf{(g)}]
Prove (again!) that $f_d \mid f_{dn}$ for any nonnegative
integers $d$ and $n$.

% Prove that
% \[
%  A^n \begin{pmatrix} 0 \\ 1 \end{pmatrix}
%  = \begin{pmatrix} f_n \\ f_{n+1} \end{pmatrix}
%  \qquad \text{ for all } n \in \NN .
% \]
\end{enumerate}

[\textbf{Hint:} One way to prove \textbf{(d)} is by
comparing the $\tup{1, 1}$-th entries of the two
(equal) matrices $A^n A^{m+1}$ and $A^{n+m+1}$, after
first using part \textbf{(c)} to expand these matrices.

For part \textbf{(g)}, compare the $\tup{1, 1}$-th
entries of the matrices $A^d - B^d$ and $A^{dn} - B^{dn}$,
after first proving that $A^d - B^d \mid A^{dn} - B^{dn}$ in
the commutative ring $\calF$.
Note that divisibility is a tricky concept in general
rings, but $\calF$ is a commutative ring, which lets
many arguments from the integer setting go
through unchanged.]

\subsection{Remark}

Contrast the ring $\calF$ with the ring $\ZZ\ive{\phi}$
from Exercise 5 on
\href{http://www.cip.ifi.lmu.de/~grinberg/t/19s/hw5s.pdf}{homework set \#5}.
Both of these rings, as we see, can be used to prove that $f_d \mid f_{dn}$
for any nonnegative integers $d$ and $n$.
What else do these rings have in common?

\subsection{Solution}

[...]

%----------------------------------------------------------------------------------------
%	EXERCISE 6
%----------------------------------------------------------------------------------------
\horrule{0.3pt} \\[0.4cm]

\section{Exercise 6: ISBNs vs. fat fingers}

\subsection{Problem}

An \textit{ISBN} shall mean a $10$-tuple
$\tup{a_1, a_2, \ldots, a_{10}} \in \set{0, 1, \ldots, 10}^{10}$
such that
\[
1 a_1 + 2 a_2 + \cdots + 10 a_{10} \equiv 0 \mod 11 .
\]

\noindent
(For example, the $10$-tuple $\tup{1, 1, \ldots, 1}$ is an ISBN.)

Prove the following:

\begin{enumerate}

\item[\textbf{(a)}]
If $\mathbf{a} = \tup{a_1, a_2, \ldots, a_{10}}$ and
$\mathbf{b} = \tup{b_1, b_2, \ldots, b_{10}}$ are two ISBNs
that are equal in all but one entry (i.e.,
there exists some $k \in \set{1, 2, \ldots, 10}$
such that $a_i = b_i$ for all $i \neq k$),
then $\mathbf{a} = \mathbf{b}$.

\item[\textbf{(b)}]
If an ISBN $\mathbf{a} = \tup{a_1, a_2, \ldots, a_{10}}$ is
obtained from an ISBN
$\mathbf{b} = \tup{b_1, b_2, \ldots, b_{10}}$ by swapping
two entries (i.e., there exist $k, \ell \in \set{1, 2, \ldots, 10}$
such that $a_k = b_\ell$ and $a_\ell = b_k$ and
$a_i = b_i$ for all $i \notin \set{k, \ell}$),
then $\mathbf{a} = \mathbf{b}$.

\end{enumerate}

\subsection{Remark}

What we called ISBN here is essentially the definition
of an \href{https://en.wikipedia.org/wiki/International_Standard_Book_Number#ISBN-10_check_digits}{ISBN-10}
-- an international standard for book identifiers used from the
1970s until 2007.
For example, the ISBN-10 of the Graham/Knuth/Patashnik
book ``Concrete Mathematics'' is ``0-201-55802-5'', which corresponds
to $\tup{0, 2, 0, 1, 5, 5, 8, 0, 2, 5}$; you can check
that this is indeed an ISBN according to our definition.

(An ``X'' in a real-life ISBN stands for an entry that
is $10$.)

As this exercise shows, ISBNs have an error-detection
property:
If you make a typo in a single digit or accidentally swap
two digits, the result will not be an ISBN, so you will
know that something has gone wrong.
This helps you avoid ordering the wrong book from a bookstore
or library.
\href{https://en.wikipedia.org/wiki/Luhn_algorithm}{Credit card numbers have a similar error-detection feature}.

This is one of the simplest examples of an
\href{https://en.wikipedia.org/wiki/Error_correction_code}{error correction code}.
We may or may not see more of them in class.
For now, you can think about how to define ``ISBNs''
\begin{itemize}
\item in $\set{0, 1, \ldots, 4}^4$;
\item in $\set{0, 1, \ldots, 6}^6$;
\item in $\set{0, 1, \ldots, 8}^8$ (this is harder!).
\end{itemize}

\subsection{Solution}

[...]

\begin{thebibliography}{99999999}                                                                                         %

% Feel free to add your sources -- or copy some from the source code
% of the class notes ( http://www.cip.ifi.lmu.de/~grinberg/t/19s/notes.tex ).

\end{thebibliography}

\end{document}

