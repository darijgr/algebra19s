\documentclass[numbers=enddot,12pt,final,onecolumn,notitlepage]{scrartcl}%
\usepackage[headsepline,footsepline,manualmark]{scrlayer-scrpage}
\usepackage[all,cmtip]{xy}
\usepackage{amssymb}
\usepackage{amsmath}
\usepackage{amsthm}
\usepackage{framed}
\usepackage{comment}
\usepackage{color}
\usepackage{hyperref}
\usepackage[sc]{mathpazo}
\usepackage[T1]{fontenc}
\usepackage{needspace}
\usepackage{tabls}
%TCIDATA{OutputFilter=latex2.dll}
%TCIDATA{Version=5.50.0.2960}
%TCIDATA{LastRevised=Thursday, June 27, 2019 15:43:56}
%TCIDATA{SuppressPackageManagement}
%TCIDATA{<META NAME="GraphicsSave" CONTENT="32">}
%TCIDATA{<META NAME="SaveForMode" CONTENT="1">}
%TCIDATA{BibliographyScheme=Manual}
%TCIDATA{Language=American English}
%BeginMSIPreambleData
\providecommand{\U}[1]{\protect\rule{.1in}{.1in}}
%EndMSIPreambleData
\newcounter{exer}
\newcounter{exera}
\theoremstyle{definition}
\newtheorem{theo}{Theorem}[subsection]
\newenvironment{theorem}[1][]
{\begin{theo}[#1]\begin{leftbar}}
{\end{leftbar}\end{theo}}
\newtheorem{lem}[theo]{Lemma}
\newenvironment{lemma}[1][]
{\begin{lem}[#1]\begin{leftbar}}
{\end{leftbar}\end{lem}}
\newtheorem{prop}[theo]{Proposition}
\newenvironment{proposition}[1][]
{\begin{prop}[#1]\begin{leftbar}}
{\end{leftbar}\end{prop}}
\newtheorem{defi}[theo]{Definition}
\newenvironment{definition}[1][]
{\begin{defi}[#1]\begin{leftbar}}
{\end{leftbar}\end{defi}}
\newtheorem{remk}[theo]{Remark}
\newenvironment{remark}[1][]
{\begin{remk}[#1]\begin{leftbar}}
{\end{leftbar}\end{remk}}
\newtheorem{coro}[theo]{Corollary}
\newenvironment{corollary}[1][]
{\begin{coro}[#1]\begin{leftbar}}
{\end{leftbar}\end{coro}}
\newtheorem{conv}[theo]{Convention}
\newenvironment{convention}[1][]
{\begin{conv}[#1]\begin{leftbar}}
{\end{leftbar}\end{conv}}
\newtheorem{quest}[theo]{Question}
\newenvironment{question}[1][]
{\begin{quest}[#1]\begin{leftbar}}
{\end{leftbar}\end{quest}}
\newtheorem{warn}[theo]{Warning}
\newenvironment{conclusion}[1][]
{\begin{warn}[#1]\begin{leftbar}}
{\end{leftbar}\end{warn}}
\newtheorem{conj}[theo]{Conjecture}
\newenvironment{conjecture}[1][]
{\begin{conj}[#1]\begin{leftbar}}
{\end{leftbar}\end{conj}}
\newtheorem{exam}[theo]{Example}
\newenvironment{example}[1][]
{\begin{exam}[#1]\begin{leftbar}}
{\end{leftbar}\end{exam}}
\newtheorem{exmp}[exer]{Exercise}
\newenvironment{exercise}[1][]
{\begin{exmp}[#1]\begin{leftbar}}
{\end{leftbar}\end{exmp}}
\newenvironment{statement}{\begin{quote}}{\end{quote}}
\newenvironment{fineprint}{\begin{small}}{\end{small}}
\iffalse
\newenvironment{proof}[1][Proof]{\noindent\textbf{#1.} }{\ \rule{0.5em}{0.5em}}
\newenvironment{question}[1][Question]{\noindent\textbf{#1.} }{\ \rule{0.5em}{0.5em}}
\newenvironment{teachingnote}[1][Teaching note]{\noindent\textbf{#1.} }{\ \rule{0.5em}{0.5em}}
\fi
\let\sumnonlimits\sum
\let\prodnonlimits\prod
\let\cupnonlimits\bigcup
\let\capnonlimits\bigcap
\renewcommand{\sum}{\sumnonlimits\limits}
\renewcommand{\prod}{\prodnonlimits\limits}
\renewcommand{\bigcup}{\cupnonlimits\limits}
\renewcommand{\bigcap}{\capnonlimits\limits}
\setlength\tablinesep{3pt}
\setlength\arraylinesep{3pt}
\setlength\extrarulesep{3pt}
\voffset=0cm
\hoffset=0cm
\setlength\textheight{22.5cm}
\setlength\textwidth{15.5cm}
\newenvironment{verlong}{}{}
\newenvironment{vershort}{}{}
\newenvironment{noncompile}{}{}
\newenvironment{teachingnote}{}{}
\excludecomment{verlong}
\includecomment{vershort}
\excludecomment{noncompile}
\excludecomment{teachingnote}
\newcommand{\id}{\operatorname{id}}
\ihead{The existence of finite fields, again}
\ohead{page \thepage}
\cfoot{}
\begin{document}

\title{The existence of finite fields, again}
\author{Darij Grinberg}
\date{
%TCIMACRO{\TeXButton{today}{\today}}%
%BeginExpansion
\today
%EndExpansion
}
\maketitle
\tableofcontents

\section{Finite fields exist: the theorem}

Fix a prime $p$. We are going to prove the following classical fact from
abstract algebra:

\begin{theorem}
\label{thm.fpnexists}Let $n$ be a positive integer. Then, there exists a
finite field of size $p^{n}$.
\end{theorem}

Here, we are using standard notations from abstract algebra (see, e.g.,
\cite{19s}). In particular, fields are always commutative.

Theorem \ref{thm.fpnexists} is well-known; various proofs can be found in
\cite[Theorem 2.5]{LidNie97}, \cite[Theorem 9.14]{Knapp1}, \cite[Exercise
12.126]{Loehr-BC}, \cite[Theorem 2.2]{Conrad-FF}, \cite[Corollary
11.26]{Hungerford}, \cite[Chapter V, Proposition 5.6]{Hungerford-03},
\cite[Theorem 19.3]{Stewar15}, \cite[14.5.1]{Escofi01} and \cite[Theorem
6.2.11]{Walker87}. Most of these proofs use either Galois theory or the
M\"{o}bius function from number theory. In this note, we will give a proof
that uses only relatively basic properties of rings and fields.

We first strengthen Theorem \ref{thm.fpnexists} a little bit, for the
convenience of our proof.

We let $\mathbb{F}_{p}$ denote the ring of all residue classes of integers
modulo $p$. This is also known as $\mathbb{Z}/p\mathbb{Z}$ (in most of the
literature) or as $\mathbb{Z}/\left(  p\right)  $ or as $\mathbb{Z}/p$ (in
\cite{19s}) or as $\mathbb{Z}_{p}$ (sometimes). It is well-known that
$\mathbb{F}_{p}$ is a field\footnote{since $p$ is a prime} of size $p$.

\begin{definition}
An $\mathbb{F}_{p}$\textit{-field} will mean an $\mathbb{F}_{p}$-algebra that
is a field (with the same addition, multiplication, zero and unity).
\end{definition}

It is not hard to see that an $\mathbb{F}_{p}$-field is the same as a field of
characteristic $p$; but we will not need this in what follows.

Now, we can strengthen Theorem \ref{thm.fpnexists} as follows:

\begin{theorem}
\label{thm.fpnexists2}Let $n$ be a positive integer. Then, there exists a
finite $\mathbb{F}_{p}$-field of size $p^{n}$.
\end{theorem}

Of course, Theorem \ref{thm.fpnexists2} is not that much stronger than Theorem
\ref{thm.fpnexists}. In fact, any finite field of size $p^{n}$ (for $n$ a
positive integer) is an $\mathbb{F}_{p}$-field; this can easily be seen using
some linear algebra or group theory. But since Theorem \ref{thm.fpnexists2}
will fall into our hands in this exact form, we will have no need for such arguments.

\section{Ingredients of the proof}

Next we shall prepare for the proof of Theorem \ref{thm.fpnexists2} by stating
several results that will end up useful.

All polynomials that appear in this note are polynomials in a single variable
$x$. If $\mathbb{K}$ is a commutative ring, and if $\mathbf{a}=\sum_{k=0}%
^{n}a_{k}x^{k}\in\mathbb{K}\left[  x\right]  $ is a polynomial of degree
$n\geq0$ (with $a_{k}\in\mathbb{K}$), then:

\begin{itemize}
\item the \textit{leading term} of $\mathbf{a}$ is defined to be the
polynomial $a_{n}x^{n}\in\mathbb{K}\left[  x\right]  $;

\item the \textit{leading coefficient} of $\mathbf{a}$ is defined to be the
element $a_{n}\in\mathbb{K}$;

\item the polynomial $\mathbf{a}$ is said to be \textit{monic} if $a_{n}=1$
(that is, its leading term is $1$).
\end{itemize}

\subsection{Adjoining a root of a polynomial}

We shall use quotients of commutative rings, but we will only need the
simplest case of such quotients (namely, the case when we are quotienting by a
principal ideal). Let us quickly introduce shorthand notations for this kind
of quotients:

\begin{convention}
\label{conv.quotient}Let $\mathbb{K}$ be a commutative ring, and
$a\in\mathbb{K}$ be any element.

For each $u\in\mathbb{K}$, we let $\left[  u\right]  _{a}$ denote the residue
class of $u$ modulo $a$. (This is commonly denoted by $u+a\mathbb{K}$.)

We let $\mathbb{K}/a$ denote the set of all residue classes of elements of
$\mathbb{K}$ modulo $a$. (This is commonly denoted by $\mathbb{K}/a\mathbb{K}$
or by $\mathbb{K}/\left(  a\right)  $.) The set $\mathbb{K}/a$ is known to be
a commutative $\mathbb{K}$-algebra (with addition defined by $\left[
u\right]  _{a}+\left[  v\right]  _{a}=\left[  u+v\right]  _{a}$, and all other
operations defined similarly).
\end{convention}

Thus, for example, $\mathbb{Z}/p$ is the field $\mathbb{F}_{p}$, and its
elements are $\left[  0\right]  _{p},\left[  1\right]  _{p},\ldots,\left[
p-1\right]  _{p}$.

The following theorem is the only way by which we are going to extend our
fields in this note:\footnote{We shall tend to denote polynomials with
boldface letters, for the sake of readability.}

\begin{theorem}
\label{thm.field-ext}Let $\mathbb{F}$ be a field. Let $n\in\mathbb{N}$. Let
$\mathbf{a}\in\mathbb{F}\left[  x\right]  $ be a polynomial of degree $n$.

Consider the $\mathbb{F}$-algebra $\mathbb{F}\left[  x\right]  /\mathbf{a}$.

\textbf{(a)} Each element of $\mathbb{F}\left[  x\right]  /\mathbf{a}$ can be
uniquely written in the form%
\[
\lambda_{0}\left[  x^{0}\right]  _{\mathbf{a}}+\lambda_{1}\left[
x^{1}\right]  _{\mathbf{a}}+\cdots+\lambda_{n-1}\left[  x^{n-1}\right]
_{\mathbf{a}}\ \ \ \ \ \ \ \ \ \ \text{with }\lambda_{0},\lambda_{1}%
,\ldots,\lambda_{n-1}\in\mathbb{F}.
\]


\textbf{(b)} If $\mathbb{F}$ is finite, then $\left\vert \mathbb{F}\left[
x\right]  /\mathbf{a}\right\vert =\left\vert \mathbb{F}\right\vert ^{n}$.

\textbf{(c)} If the polynomial $\mathbf{a}$ is irreducible, then
$\mathbb{F}\left[  x\right]  /\mathbf{a}$ is a field.
\end{theorem}

\begin{proof}
[Proof of Theorem \ref{thm.field-ext}.]\textbf{(a)} The polynomial
$\mathbf{a}$ has degree $n$. Thus, the coefficient of $x^{n}$ in $\mathbf{a}$
is nonzero, and therefore invertible (since every nonzero element of
$\mathbb{F}$ is invertible\footnote{since $\mathbb{F}$ is a field}). Thus, the
claim of Theorem \ref{thm.field-ext} \textbf{(a)} follows from \cite[Theorem
8.1.9 \textbf{(a)}]{19s} (applied to $\mathbb{F}$, $n$ and $\mathbf{a}$
instead of $\mathbb{K}$, $m$ and $\mathbf{b}$).

Alternatively, it is easy to derive Theorem \ref{thm.field-ext} \textbf{(a)}
from the familiar fact that division with remainder works for polynomials in
$\mathbb{F}\left[  x\right]  $.

\textbf{(b)} Assume that $\mathbb{F}$ is finite. Then, Theorem
\ref{thm.field-ext} \textbf{(a)} shows that each element of $\mathbb{F}\left[
x\right]  /\mathbf{a}$ can be uniquely written in the form%
\[
\lambda_{0}\left[  x^{0}\right]  _{\mathbf{a}}+\lambda_{1}\left[
x^{1}\right]  _{\mathbf{a}}+\cdots+\lambda_{n-1}\left[  x^{n-1}\right]
_{\mathbf{a}}\ \ \ \ \ \ \ \ \ \ \text{with }\lambda_{0},\lambda_{1}%
,\ldots,\lambda_{n-1}\in\mathbb{F}.
\]
In other words, the map%
\begin{align*}
\mathbb{F}^{n}  &  \rightarrow\mathbb{F}\left[  x\right]  /\mathbf{a},\\
\left(  \lambda_{0},\lambda_{1},\ldots,\lambda_{n-1}\right)   &
\mapsto\lambda_{0}\left[  x^{0}\right]  _{\mathbf{a}}+\lambda_{1}\left[
x^{1}\right]  _{\mathbf{a}}+\cdots+\lambda_{n-1}\left[  x^{n-1}\right]
_{\mathbf{a}}%
\end{align*}
is a bijection. Hence, there exists a bijection from $\mathbb{F}^{n}$ to
$\mathbb{F}\left[  x\right]  /\mathbf{a}$ (namely, this map). Thus,
$\left\vert \mathbb{F}\left[  x\right]  /\mathbf{a}\right\vert =\left\vert
\mathbb{F}^{n}\right\vert =\left\vert \mathbb{F}\right\vert ^{n}$. This proves
Theorem \ref{thm.field-ext} \textbf{(b)}.

\textbf{(c)} Theorem \ref{thm.field-ext} \textbf{(c)} follows from
\cite[Theorem 8.1.17]{19s} or from \cite[4.7.2]{Escofi01} (applied to
$K=\mathbb{F}$ and $P=\mathbf{a}$).

Alternatively, it can be found in the literature under some fairly transparent
guises. For instance, several sources (e.g., \cite[Theorem 17.2]{Stewar15} or
\cite[proof of Theorem 9.10]{Knapp1} or \cite[1.25]{Milne-FT}) prove Theorem
\ref{thm.field-ext} \textbf{(c)} in the case when the polynomial $\mathbf{a}$
is monic. But the general case can easily be reduced to this case (because we
can always make a nonzero polynomial $\mathbf{a}\in\mathbb{F}\left[  x\right]
$ monic by scaling it with the multiplicative inverse of its leading coefficient).
\end{proof}

\subsection{Factoring polynomials into irreducibles}

How do we find irreducible polynomials to apply Theorem \ref{thm.field-ext}
\textbf{(c)} to? The following lemma shows a simple way:

\begin{lemma}
\label{lem.exist-irred-div}Let $\mathbb{F}$ be a field. Let $\mathbf{a}%
\in\mathbb{F}\left[  x\right]  $ be a non-constant polynomial. Then, there
exists a monic irreducible polynomial $\mathbf{u}$ such that $\mathbf{u}%
\mid\mathbf{a}$.
\end{lemma}

\begin{proof}
[Proof of Lemma \ref{lem.exist-irred-div}.]Clearly, there exists a
non-constant polynomial $\mathbf{v}\in\mathbb{F}\left[  x\right]  $ that
satisfies $\mathbf{v}\mid\mathbf{a}$ (for example, $\mathbf{a}$ itself is such
a polynomial). Choose such a $\mathbf{v}$ of the smallest possible degree, and
denote it by $\mathbf{b}$. Thus, $\mathbf{b}$ is a non-constant polynomial and
satisfies $\mathbf{b}\mid\mathbf{a}$.

Let $\kappa$ be the leading coefficient of $\mathbf{b}$ (that is, the
coefficient of $x^{\deg\mathbf{b}}$ in $\mathbf{b}$). Then, $\kappa$ is a
nonzero element of $\mathbb{F}$, and thus is invertible (since every nonzero
element of $\mathbb{F}$ is invertible\footnote{because $\mathbb{F}$ is a
field}). Therefore, $\kappa^{-1}\in\mathbb{F}$ is well-defined and nonzero.
Thus, if we regard $\kappa^{-1}$ as a polynomial in $\mathbb{F}\left[
x\right]  $, then $\kappa^{-1}$ is a nonzero constant; hence, $\deg\left(
\kappa^{-1}\right)  =0$. Now,
\[
\deg\left(  \kappa^{-1}\mathbf{b}\right)  =\underbrace{\deg\left(  \kappa
^{-1}\right)  }_{=0}+\deg\mathbf{b}=\deg\mathbf{b}>0
\]
(since $\mathbf{b}$ is non-constant); thus, the polynomial $\kappa
^{-1}\mathbf{b}$ is non-constant. Moreover, the polynomial $\kappa
^{-1}\mathbf{b}$ satisfies $\kappa^{-1}\mathbf{b}\mid\mathbf{b}$ (since
$\mathbf{b}=\underbrace{1}_{=\kappa\kappa^{-1}}\mathbf{b}=\kappa\kappa
^{-1}\mathbf{b}=\left(  \kappa^{-1}\mathbf{b}\right)  \cdot\kappa$).

It is easy to see that the polynomial $\kappa^{-1}\mathbf{b}$ is irreducible.

[\textit{Proof:} Assume the contrary. Then, we can write $\kappa
^{-1}\mathbf{b}$ as $\kappa^{-1}\mathbf{b}=\mathbf{cd}$ with two polynomials
$\mathbf{c},\mathbf{d}\in\mathbb{F}\left[  x\right]  $ satisfying
$\deg\mathbf{c}>0$ and $\deg\mathbf{d}>0$ (since $\kappa^{-1}\mathbf{b}$ is
non-constant but not irreducible). Consider these $\mathbf{c}$ and
$\mathbf{d}$. The polynomial $\mathbf{c}$ is a non-constant
polynomial\footnote{since $\deg\mathbf{c}>0$} and satisfies $\mathbf{c}%
\mid\mathbf{a}$ (since $\mathbf{c}\mid\mathbf{cd}=\kappa^{-1}\mathbf{b}%
\mid\mathbf{b}\mid\mathbf{a}$). In other words, $\mathbf{c}$ is a non-constant
polynomial $\mathbf{v}\in\mathbb{F}\left[  x\right]  $ that satisfies
$\mathbf{v}\mid\mathbf{a}$. Therefore, $\deg\mathbf{c}\geq\deg\mathbf{b}$
(because we defined $\mathbf{b}$ to be such a $\mathbf{v}$ of the smallest
possible degree). However, $\deg\underbrace{\left(  \kappa^{-1}\mathbf{b}%
\right)  }_{=\mathbf{cd}}=\deg\left(  \mathbf{cd}\right)  =\deg\mathbf{c}%
+\underbrace{\deg\mathbf{d}}_{>0}>\deg\mathbf{c}$ and thus $\deg
\mathbf{c}<\deg\left(  \kappa^{-1}\mathbf{b}\right)  =\deg\mathbf{b}$. This
contradicts $\deg\mathbf{c}\geq\deg\mathbf{b}$. This contradiction shows that
our assumption was false. Hence, we have shown that $\kappa^{-1}\mathbf{b}$ is irreducible.]

The leading coefficient of the polynomial $\mathbf{b}$ is $\kappa$. Hence, the
leading coefficient of the polynomial $\kappa^{-1}\mathbf{b}$ is $\kappa
^{-1}\kappa=1$. In other words, the polynomial $\kappa^{-1}\mathbf{b}$ is
monic. It further satisfies $\kappa^{-1}\mathbf{b}\mid\mathbf{a}$ (since
$\kappa^{-1}\mathbf{b}\mid\mathbf{b}\mid\mathbf{a}$). Hence, there exists a
monic irreducible polynomial $\mathbf{u}$ such that $\mathbf{u}\mid\mathbf{a}$
(namely, $\mathbf{u}=\kappa^{-1}\mathbf{b}$). This proves Lemma
\ref{lem.exist-irred-div}.
\end{proof}

\begin{corollary}
\label{cor.irred-factor}Let $\mathbb{F}$ be a field. Let $\mathbf{a}%
\in\mathbb{F}\left[  x\right]  $ be a nonzero polynomial. Then, $\mathbf{a}$
can be written in the form $\mathbf{a}=\lambda\mathbf{u}_{1}\mathbf{u}%
_{2}\cdots\mathbf{u}_{k}$, where $\lambda\in\mathbb{F}$ is a nonzero constant,
and where $\mathbf{u}_{1},\mathbf{u}_{2},\ldots,\mathbf{u}_{k}\in
\mathbb{F}\left[  x\right]  $ are monic irreducible polynomials.
\end{corollary}

\begin{proof}
[Proof of Corollary \ref{cor.irred-factor} (sketched).]This is an analogue of
the fact that every nonzero integer is a product of finitely many primes (up
to sign). The proof is analogous, too: Proceed by strong induction on
$\deg\mathbf{a}$, using Lemma \ref{lem.exist-irred-div} in the induction step.
\end{proof}

\begin{corollary}
\label{cor.irred-factor-monic}Let $\mathbb{F}$ be a field. Let $\mathbf{a}%
\in\mathbb{F}\left[  x\right]  $ be a monic polynomial. Then, $\mathbf{a}$ can
be written in the form $\mathbf{a}=\mathbf{u}_{1}\mathbf{u}_{2}\cdots
\mathbf{u}_{k}$, where $\mathbf{u}_{1},\mathbf{u}_{2},\ldots,\mathbf{u}_{k}%
\in\mathbb{F}\left[  x\right]  $ are monic irreducible polynomials.
\end{corollary}

\begin{proof}
[Proof of Corollary \ref{cor.irred-factor-monic}.]The polynomial $\mathbf{a}$
is monic and thus nonzero. Hence, Corollary \ref{cor.irred-factor} shows that
$\mathbf{a}$ can be written in the form $\mathbf{a}=\lambda\mathbf{u}%
_{1}\mathbf{u}_{2}\cdots\mathbf{u}_{k}$, where $\lambda\in\mathbb{F}$ is a
nonzero constant, and where $\mathbf{u}_{1},\mathbf{u}_{2},\ldots
,\mathbf{u}_{k}\in\mathbb{F}\left[  x\right]  $ are monic irreducible
polynomials. Consider this $\lambda$ and these $\mathbf{u}_{1},\mathbf{u}%
_{2},\ldots,\mathbf{u}_{k}$.

The polynomial $\mathbf{a}$ is monic, and thus has leading coefficient $1$.
The polynomial $\mathbf{u}_{1}\mathbf{u}_{2}\cdots\mathbf{u}_{k}$ is monic
(since it is the product of the monic polynomials $\mathbf{u}_{1}%
,\mathbf{u}_{2},\ldots,\mathbf{u}_{k}$), and thus has leading coefficient $1$.
Hence, the polynomial $\lambda\mathbf{u}_{1}\mathbf{u}_{2}\cdots\mathbf{u}%
_{k}$ has leading coefficient $\lambda$ (since $\lambda$ is nonzero). Now,
recall the equality $\mathbf{a}=\lambda\mathbf{u}_{1}\mathbf{u}_{2}%
\cdots\mathbf{u}_{k}$. Comparing the leading coefficients on both sides of
this equality, we obtain $1=\lambda$ (because the polynomial $\mathbf{a}$ has
leading coefficient $1$, while the polynomial $\lambda\mathbf{u}_{1}%
\mathbf{u}_{2}\cdots\mathbf{u}_{k}$ has leading coefficient $\lambda$). Hence,
$\lambda=1$. Thus,%
\[
\mathbf{a}=\underbrace{\lambda}_{=1}\mathbf{u}_{1}\mathbf{u}_{2}%
\cdots\mathbf{u}_{k}=\mathbf{u}_{1}\mathbf{u}_{2}\cdots\mathbf{u}_{k}.
\]


We thus have found monic irreducible polynomials $\mathbf{u}_{1}%
,\mathbf{u}_{2},\ldots,\mathbf{u}_{k}\in\mathbb{F}\left[  x\right]  $ such
that $\mathbf{a}=\mathbf{u}_{1}\mathbf{u}_{2}\cdots\mathbf{u}_{k}$. In other
words, we have written $\mathbf{a}$ in the form $\mathbf{a}=\mathbf{u}%
_{1}\mathbf{u}_{2}\cdots\mathbf{u}_{k}$, where $\mathbf{u}_{1},\mathbf{u}%
_{2},\ldots,\mathbf{u}_{k}\in\mathbb{F}\left[  x\right]  $ are monic
irreducible polynomials. Thus, $\mathbf{a}$ can be written in this form. This
proves Corollary \ref{cor.irred-factor-monic}.
\end{proof}

\subsection{A lemma on maps}

We shall furthermore use the following simple lemma about maps on a set:

\begin{lemma}
\label{lem.fgcd}Let $S$ be a set. Let $f:S\rightarrow S$ be a map. Let $a$ and
$b$ be two positive integers such that $f^{a}=\operatorname*{id}$ and
$f^{b}=\operatorname*{id}$. Then, $f^{\gcd\left(  a,b\right)  }%
=\operatorname*{id}$.
\end{lemma}

Here, of course, $f^{n}$ (where $n\in\mathbb{N}$) stands for the map
$\underbrace{f\circ f\circ\cdots\circ f}_{n\text{ times}}$.

Lemma \ref{lem.fgcd} is almost obvious if you know a bit of group theory
(specifically, the notion of the order of an element in a group). But in order
to keep this note self-contained, I shall give an elementary proof:

\begin{proof}
[Proof of Lemma \ref{lem.fgcd}.]We have $f^{a}=f\circ f^{a-1}$, so that
$f\circ f^{a-1}=f^{a}=\operatorname*{id}$. Also, $f^{a}=f^{a-1}\circ f$, so
that $f^{a-1}\circ f=f^{a}=\operatorname*{id}$. The equalities $f\circ
f^{a-1}=\operatorname*{id}$ and $f^{a-1}\circ f=\operatorname*{id}$ show that
the maps $f$ and $f^{a-1}$ are mutually inverse. Hence, the map $f$ is
invertible (with inverse $f^{a-1}$). Thus, the powers $f^{n}$ of this map $f$
are well-defined not only for $n\in\mathbb{N}$, but also for $n\in\mathbb{Z}$.
Furthermore, it is well-known that these powers satisfy the following
\textquotedblleft laws of exponents\textquotedblright:

\begin{itemize}
\item We have%
\begin{equation}
f^{n+m}=f^{n}\circ f^{m}\ \ \ \ \ \ \ \ \ \ \text{for all }n,m\in\mathbb{Z}.
\label{pf.lem.fgcd.fu+v}%
\end{equation}
(This can be proven similarly to the proof of \cite[Proposition 4.1.20
\textbf{(h)}]{19s}.)

\item We have%
\begin{equation}
f^{nm}=\left(  f^{n}\right)  ^{m}\ \ \ \ \ \ \ \ \ \ \text{for all }%
n,m\in\mathbb{Z}. \label{pf.lem.fgcd.fuv}%
\end{equation}
(This can be proven similarly to the proof of \cite[Proposition 4.1.20
\textbf{(l)}]{19s}.)
\end{itemize}

But Bezout's theorem (see, e.g., \cite[Theorem 2.9.12]{19s}) shows that there
exist integers $x\in\mathbb{Z}$ and $y\in\mathbb{Z}$ such that $\gcd\left(
a,b\right)  =xa+yb$. Consider these $x$ and $y$. From $\gcd\left(  a,b\right)
=xa+yb=ax+by$, we obtain%
\begin{align*}
f^{\gcd\left(  a,b\right)  }  &  =f^{ax+by}=\underbrace{f^{ax}}%
_{\substack{=\left(  f^{a}\right)  ^{x}\\\text{(by (\ref{pf.lem.fgcd.fuv}%
),}\\\text{applied to }n=a\text{ and }m=x\text{)}}}\circ\underbrace{f^{by}%
}_{\substack{=\left(  f^{b}\right)  ^{y}\\\text{(by (\ref{pf.lem.fgcd.fuv}%
),}\\\text{applied to }n=b\text{ and }m=y\text{)}}}\\
&  \ \ \ \ \ \ \ \ \ \ \left(  \text{by (\ref{pf.lem.fgcd.fu+v}), applied to
}n=ax\text{ and }m=by\right) \\
&  =\left(  \underbrace{f^{a}}_{=\operatorname*{id}}\right)  ^{x}\circ\left(
\underbrace{f^{b}}_{=\operatorname*{id}}\right)  ^{y}%
=\underbrace{\operatorname*{id}\nolimits^{x}}_{=\operatorname*{id}}%
\circ\underbrace{\operatorname*{id}\nolimits^{y}}_{=\operatorname*{id}%
}=\operatorname*{id}\circ\operatorname*{id}=\operatorname*{id}.
\end{align*}
This proves Lemma \ref{lem.fgcd}.
\end{proof}

\subsection{Restriction of modules and algebras}

In linear algebra, you may have learned that each $\mathbb{C}$-vector space
$V$ is (or, more precisely, becomes in a natural way) an $\mathbb{R}$-vector
space: Just restrict its scaling map $\cdot:\mathbb{C}\times V\rightarrow V$
to $\mathbb{R}\times V$, and you obtain a scaling map $\cdot:\mathbb{R}\times
V\rightarrow V$ that makes it into an $\mathbb{R}$-vector space. (However, the
dimension of this $\mathbb{R}$-vector space $V$ will be \textbf{double} the
dimension of the original $\mathbb{C}$-vector space $V$.)

The simple reason why this works is that $\mathbb{R}$ is a subring of
$\mathbb{C}$. More generally, if $\mathbb{K}$ is a subring of a commutative
ring $\mathbb{L}$, then any $\mathbb{L}$-module $V$ becomes a $\mathbb{K}%
$-module in the same way (i.e., by restricting the scaling map to
$\mathbb{K}\times V$).

With a little tweak, the same construction works even more generally: We don't
need $\mathbb{K}$ to be a subring of $\mathbb{L}$; we only need $\mathbb{L}$
to be a commutative $\mathbb{K}$-algebra\footnote{This case is indeed more
general, because if $\mathbb{K}$ is a subring of a commutative ring
$\mathbb{L}$, then $\mathbb{L}$ is clearly a commutative $\mathbb{K}%
$-algebra.}. The thing we need to do in order to turn an $\mathbb{L}$-module
$V$ into a $\mathbb{K}$-module is no longer literally restricting the scaling
map $\cdot:\mathbb{L}\times V\rightarrow V$ to $\mathbb{K}\times V$, but
rather a simple tweak:

\begin{proposition}
\label{prop.restrict-mod}Let $\mathbb{K}$ be a commutative ring. Let
$\mathbb{L}$ be a commutative $\mathbb{K}$-algebra. Let $V$ be an $\mathbb{L}%
$-module. Consider its addition $+$, its scaling $\cdot:\mathbb{L}\times
V\rightarrow V$ and its zero vector $0_{V}$.

Define a scaling $\cdot:\mathbb{K}\times V\rightarrow V$ by setting%
\begin{equation}
\lambda\cdot v=\left(  \lambda\cdot1_{\mathbb{L}}\right)  \cdot
v\ \ \ \ \ \ \ \ \ \ \text{for all }\lambda\in\mathbb{K}\text{ and }v\in V.
\label{eq.prop.restrict-mod.fml}%
\end{equation}
Here:

\begin{itemize}
\item the \textquotedblleft$\cdot$\textquotedblright\ on the left hand side
means the scaling $\cdot:\mathbb{K}\times V\rightarrow V$ that we are defining
(written infix);

\item the first \textquotedblleft$\cdot$\textquotedblright\ on the right hand
side means the scaling $\cdot:\mathbb{K}\times\mathbb{L}\rightarrow\mathbb{L}$
of the $\mathbb{K}$-algebra $\mathbb{L}$ (since $\mathbb{L}$ is a $\mathbb{K}%
$-algebra and thus is a $\mathbb{K}$-module, which means that it has a scaling);

\item the second \textquotedblleft$\cdot$\textquotedblright\ on the right hand
side means the scaling $\cdot:\mathbb{L}\times V\rightarrow V$ of the
$\mathbb{L}$-module $V$.
\end{itemize}

Then, the set $V$ (equipped with its addition $+$, its zero vector $0_{V}$ and
the scaling $\cdot:\mathbb{K}\times V\rightarrow V$ we just defined) is a
$\mathbb{K}$-module.
\end{proposition}

Roughly speaking, the definition of the scaling $\cdot:\mathbb{K}\times
V\rightarrow V$ in Proposition \ref{prop.restrict-mod} can be restated as
follows: In order to scale an element $v\in V$ by an element $\lambda
\in\mathbb{K}$, we first scale $1_{\mathbb{L}}$ by $\lambda$, thus obtaining
some sort of \textquotedblleft proxy element\textquotedblright\ of $\lambda$
in $\mathbb{L}$, and then we scale $v$ by this \textquotedblleft proxy
element\textquotedblright\ (which we know how to do, because $V$ is an
$\mathbb{L}$-module).

We shall prove Proposition \ref{prop.restrict-mod} in the Appendix (Section
\ref{sect.app-restrict}).

Next, let us state an analogue of Proposition \ref{prop.restrict-mod} for
algebras instead of modules:

\begin{proposition}
\label{prop.restrict-alg}Let $\mathbb{K}$ be a commutative ring. Let
$\mathbb{L}$ be a commutative $\mathbb{K}$-algebra. Let $V$ be an $\mathbb{L}%
$-algebra. Thus, $V$ is a ring and an $\mathbb{L}$-module at the same time.
Consider its addition $+$, its multiplication $\cdot$, its scaling $\cdot$,
its zero $0_{V}$ and its unity $1_{V}$.

Define a scaling $\cdot:\mathbb{K}\times V\rightarrow V$ as in Proposition
\ref{prop.restrict-mod}.

Then, the set $V$ (equipped with its addition $+$, its multiplication $\cdot$,
its zero $0_{V}$, its unity $1_{V}$, and the scaling $\cdot:\mathbb{K}\times
V\rightarrow V$ we just defined) is a $\mathbb{K}$-algebra.
\end{proposition}

Again, we refer to the Appendix (Section \ref{sect.app-restrict}) for a proof
of Proposition \ref{prop.restrict-alg}.

We can shorten Proposition \ref{prop.restrict-alg} significantly if we omit
the precise definition of the scaling $\cdot:\mathbb{K}\times V\rightarrow V$
and simply claim that such a scaling can be defined:

\begin{proposition}
\label{prop.restrict-alg-isol}Let $\mathbb{K}$ be a commutative ring. Let
$\mathbb{L}$ be a commutative $\mathbb{K}$-algebra. Let $V$ be an $\mathbb{L}%
$-algebra. Then, $V$ becomes a $\mathbb{K}$-algebra in a natural way.
\end{proposition}

Here, \textquotedblleft in a natural way\textquotedblright\ means
\textquotedblleft by equipping it with a scaling map $\cdot:\mathbb{K}\times
V\rightarrow V$ that is defined uniquely in terms of the existing
structures\textquotedblright\ (specifically, in terms of the unity
$1_{\mathbb{L}}$ of $\mathbb{L}$ and the scaling maps of the $\mathbb{K}%
$-algebra $\mathbb{L}$ and of the $\mathbb{L}$-algebra $V$).

\begin{proof}
[Proof of Proposition \ref{prop.restrict-alg-isol}.]Define a scaling
$\cdot:\mathbb{K}\times V\rightarrow V$ as in Proposition
\ref{prop.restrict-alg}. Then, Proposition \ref{prop.restrict-alg} shows that
the set $V$ (equipped with its addition $+$, its multiplication $\cdot$, its
zero $0_{V}$, its unity $1_{V}$, and the scaling $\cdot:\mathbb{K}\times
V\rightarrow V$ we just defined) is a $\mathbb{K}$-algebra. Thus, $V$ becomes
a $\mathbb{K}$-algebra in a natural way. This proves Proposition
\ref{prop.restrict-alg-isol}.
\end{proof}

\subsection{Fermat's little theorem for finite fields}

Fermat's little theorem, in one of its forms (e.g., \cite[Theorem 2.15.2
\textbf{(b)}]{19s}), says that $a^{p}\equiv a\operatorname{mod}p$ for every
integer $a$. We can transform this congruence into an equality in
$\mathbb{F}_{p}$; then, it becomes the statement that $\alpha^{p}=\alpha$ for
each $\alpha\in\mathbb{F}_{p}$. This can be generalized to finite fields other
than $\mathbb{F}_{p}$; namely, we have the following:

\begin{theorem}
\label{thm.field-FLT}Let $\mathbb{F}$ be a finite field. Then, $\alpha
^{\left\vert \mathbb{F}\right\vert }=\alpha$ for each $\alpha\in\mathbb{F}$.
\end{theorem}

The following proof of Theorem \ref{thm.field-FLT} uses the same idea as
\cite[proof of Theorem 2.15.3]{19s} (one of the standard proofs of Euler's theorem):

\begin{proof}
[Proof of Theorem \ref{thm.field-FLT}.]We know that $\mathbb{F}$ is a field.
In other words, $\mathbb{F}$ is a commutative skew field.

The field $\mathbb{F}$ contains at least $1$ element (since it contains
$0_{\mathbb{F}}$). Thus, $\left\vert \mathbb{F}\right\vert \geq1$, so that
$0^{\left\vert \mathbb{F}\right\vert }=0$. (Here and throughout this proof,
\textquotedblleft$0$\textquotedblright\ means the zero of $\mathbb{F}$.)

We know that $\mathbb{F}$ is skew a field. Thus, every nonzero element of
$\mathbb{F}$ is invertible, and we have $1\neq0$ in $\mathbb{F}$.

It is well-known that the product of any two invertible elements of
$\mathbb{F}$ is invertible\footnote{\textit{Proof.} Let $\alpha$ and $\beta$
be two invertible elements of $\mathbb{F}$. We must prove that $\alpha\beta$
is invertible.
\par
The multiplicative inverses $\alpha^{-1}$ and $\beta^{-1}$ of $\alpha$ and
$\beta$ are well-defined (since $\alpha$ and $\beta$ are invertible). Now, the
two elements $\beta^{-1}\alpha^{-1}$ and $\alpha\beta$ of $\mathbb{F}$ satisfy%
\begin{align*}
\left(  \beta^{-1}\alpha^{-1}\right)  \cdot\left(  \alpha\beta\right)   &
=\beta^{-1}\underbrace{\alpha^{-1}\alpha}_{=1}\beta=\beta^{-1}1\beta
=\beta^{-1}\beta=1\ \ \ \ \ \ \ \ \ \ \text{and}\\
\left(  \alpha\beta\right)  \cdot\left(  \beta^{-1}\alpha^{-1}\right)   &
=\alpha\underbrace{\beta\beta^{-1}}_{=1}\alpha^{-1}=\alpha1\alpha^{-1}%
=\alpha\alpha^{-1}=1.
\end{align*}
In other words, $\beta^{-1}\alpha^{-1}$ is a multiplicative inverse of
$\alpha\beta$. Thus, $\alpha\beta$ has a multiplicative inverse. In other
words, $\alpha\beta$ is invertible, qed.}. Thus, by induction, it is easy to
see that any product of finitely many invertible elements of $\mathbb{F}$ is
invertible\footnote{The induction base hinges on the fact that $1\in
\mathbb{F}$ is invertible (since the empty product equals $1$).}.

Each $\beta\in\mathbb{F}\setminus\left\{  0\right\}  $ is
nonzero\footnote{since $\beta\in\mathbb{F}\setminus\left\{  0\right\}  $ and
thus $\beta\neq0$} and thus invertible (since every nonzero element of
$\mathbb{F}$ is invertible). Hence, $\prod_{\beta\in\mathbb{F}\setminus
\left\{  0\right\}  }\beta$ is a product of finitely many invertible elements
of $\mathbb{F}$, and thus is invertible (since any product of finitely many
invertible elements of $\mathbb{F}$ is invertible).

Fix $\alpha\in\mathbb{F}$. We must prove that $\alpha^{\left\vert
\mathbb{F}\right\vert }=\alpha$. If $\alpha=0$, then this is true (since
$0^{\left\vert \mathbb{F}\right\vert }=0$). Hence, we WLOG assume that
$\alpha\neq0$ for the rest of this proof. Thus, the element $\alpha$ of
$\mathbb{F}$ is nonzero, and thus is invertible (since every nonzero element
of $\mathbb{F}$ is invertible). Hence, its multiplicative inverse $\alpha
^{-1}$ is well-defined. Moreover, $\alpha^{-1}\neq0$\ \ \ \ \footnote{because
otherwise, we would have $\alpha^{-1}=0$, and thus $1=\alpha\underbrace{\alpha
^{-1}}_{=0}=\alpha0=0$, which would contradict the fact that $1\neq0$ in
$\mathbb{F}$}.

Now, each $\beta\in\mathbb{F}\setminus\left\{  0\right\}  $ satisfies
$\alpha\beta\in\mathbb{F}\setminus\left\{  0\right\}  $%
\ \ \ \footnote{\textit{Proof.} Let $\beta\in\mathbb{F}\setminus\left\{
0\right\}  $. We must prove that $\alpha\beta\in\mathbb{F}\setminus\left\{
0\right\}  $. In other words, we must prove that $\alpha\beta\neq0$ (since
$\alpha\beta$ is clearly an element of $\mathbb{F}$).
\par
Assume the contrary. Thus, $\alpha\beta=0$. Hence, $\alpha^{-1}%
\underbrace{\alpha\beta}_{=0}=\alpha^{-1}0=0$, so that $0=\underbrace{\alpha
^{-1}\alpha}_{=1}\beta=\beta\in\mathbb{F}\setminus\left\{  0\right\}  $. But
this entails $0\notin\left\{  0\right\}  $, which is absurd. Hence, we have
obtained a contradiction. This contradiction shows that our assumption was
wrong. Thus, we have shown that $\alpha\beta\neq0$. Hence, $\alpha\beta
\in\mathbb{F}\setminus\left\{  0\right\}  $, qed.}. The same argument (applied
to $\alpha^{-1}$ instead of $\alpha$) shows that each $\beta\in\mathbb{F}%
\setminus\left\{  0\right\}  $ satisfies $\alpha^{-1}\beta\in\mathbb{F}%
\setminus\left\{  0\right\}  $ (since $\alpha^{-1}\neq0$).

Consider the map%
\begin{align*}
X:\mathbb{F}\setminus\left\{  0\right\}   &  \rightarrow\mathbb{F}%
\setminus\left\{  0\right\}  ,\\
\beta &  \mapsto\alpha\beta
\end{align*}
(this is well-defined, because each $\beta\in\mathbb{F}\setminus\left\{
0\right\}  $ satisfies $\alpha\beta\in\mathbb{F}\setminus\left\{  0\right\}
$) and the map%
\begin{align*}
Y:\mathbb{F}\setminus\left\{  0\right\}   &  \rightarrow\mathbb{F}%
\setminus\left\{  0\right\}  ,\\
\beta &  \mapsto\alpha^{-1}\beta
\end{align*}
(this is well-defined, because each $\beta\in\mathbb{F}\setminus\left\{
0\right\}  $ satisfies $\alpha^{-1}\beta\in\mathbb{F}\setminus\left\{
0\right\}  $).

It is easy to see that these maps $X$ and $Y$ are mutually
inverse\footnote{For example, $X\circ Y=\operatorname*{id}$ follows from the
following computation: For each $\beta\in\mathbb{F}\setminus\left\{
0\right\}  $, we have%
\begin{align*}
\left(  X\circ Y\right)  \left(  \beta\right)   &  =X\left(  Y\left(
\beta\right)  \right)  =\alpha\cdot\underbrace{Y\left(  \beta\right)
}_{\substack{=\alpha^{-1}\beta\\\text{(by the definition}\\\text{of }%
Y\text{)}}}\ \ \ \ \ \ \ \ \ \ \left(  \text{by the definition of }X\right) \\
&  =\underbrace{\alpha\cdot\alpha^{-1}}_{=1}\beta=1\beta=\beta
=\operatorname*{id}\left(  \beta\right)  .
\end{align*}
}. Thus, the map $X:\mathbb{F}\setminus\left\{  0\right\}  \rightarrow
\mathbb{F}\setminus\left\{  0\right\}  $ is invertible, i.e., is bijection.
Hence, we can substitute $X\left(  \beta\right)  $ for $\beta$ in the product
$\prod_{\beta\in\mathbb{F}\setminus\left\{  0\right\}  }\beta$. We thus find%
\[
\prod_{\beta\in\mathbb{F}\setminus\left\{  0\right\}  }\beta=\prod_{\beta
\in\mathbb{F}\setminus\left\{  0\right\}  }\underbrace{X\left(  \beta\right)
}_{\substack{=\alpha\beta\\\text{(by the definition}\\\text{of }X\text{)}%
}}=\prod_{\beta\in\mathbb{F}\setminus\left\{  0\right\}  }\left(  \alpha
\beta\right)  =\alpha^{\left\vert \mathbb{F}\setminus\left\{  0\right\}
\right\vert }\prod_{\beta\in\mathbb{F}\setminus\left\{  0\right\}  }\beta
\]
(since $\mathbb{F}$ is commutative). We can divide both sides of this equality
by $\prod_{\beta\in\mathbb{F}\setminus\left\{  0\right\}  }\beta$ (since
$\prod_{\beta\in\mathbb{F}\setminus\left\{  0\right\}  }\beta$ is invertible).
Thus we obtain%
\[
1=\alpha^{\left\vert \mathbb{F}\setminus\left\{  0\right\}  \right\vert
}=\alpha^{\left\vert \mathbb{F}\right\vert -1}\ \ \ \ \ \ \ \ \ \ \left(
\text{since }\left\vert \mathbb{F}\setminus\left\{  0\right\}  \right\vert
=\left\vert \mathbb{F}\right\vert -1\right)  .
\]
Multiplying both sides of this equality by $\alpha$, we find $\alpha
=\alpha^{\left\vert \mathbb{F}\right\vert -1}\alpha=\alpha^{\left\vert
\mathbb{F}\right\vert }$. In other words, $\alpha^{\left\vert \mathbb{F}%
\right\vert }=\alpha$. This proves Theorem \ref{thm.field-FLT}.
\end{proof}

\subsection{Frobenius endomorphisms}

We now introduce a very special map defined on any commutative $\mathbb{F}%
_{p}$-algebra:

\begin{definition}
\label{def.Frob}Let $\mathbb{K}$ be a commutative $\mathbb{F}_{p}$-algebra.
The map%
\[
\mathbb{K}\rightarrow\mathbb{K},\ \ \ \ \ \ \ \ \ \ a\mapsto a^{p}%
\]
will be called the \textit{Frobenius endomorphism} of $\mathbb{K}$ and will be
denoted by $F_{\mathbb{K}}$.
\end{definition}

For example, the Frobenius endomorphism of $\mathbb{F}_{p}$ is the identity
map (since every $a\in\mathbb{F}_{p}$ satisfies $a^{p}=a$; this is a
consequence of Fermat's Little Theorem). But $\mathbb{F}_{p}$-algebras can be
larger than $\mathbb{F}_{p}$, and usually their Frobenius endomorphisms will
not be the identity map.

The word \textquotedblleft endomorphism\textquotedblright\ means
\textquotedblleft homomorphism from an object (in this case, an $\mathbb{F}%
_{p}$-algebra) to itself\textquotedblright. Thus, the name \textquotedblleft
Frobenius endomorphism\textquotedblright\ suggests that $F_{\mathbb{K}}$ is an
$\mathbb{F}_{p}$-algebra homomorphism from $\mathbb{K}$ to $\mathbb{K}$. And
this is indeed the case:

\begin{theorem}
\label{thm.Frob.end}Let $\mathbb{K}$ be a commutative $\mathbb{F}_{p}%
$-algebra. Then, its Frobenius endomorphism $F_{\mathbb{K}}$ is an
$\mathbb{F}_{p}$-algebra homomorphism.
\end{theorem}

Before we prove this, let us show a simple proposition that will come useful
here and later on as well:

\begin{proposition}
\label{prop.Fp-alg.p=0}Let $\mathbb{K}$ be an $\mathbb{F}_{p}$-algebra.

\textbf{(a)} We have $pa=0$ for each $a\in\mathbb{K}$.

\textbf{(b)} Assume that $\mathbb{K}$ is commutative. Then, $p\mathbf{a}=0$
for each $\mathbf{a}\in\mathbb{K}\left[  x\right]  $.
\end{proposition}

\begin{proof}
[Proof of Proposition \ref{prop.Fp-alg.p=0}.]\textbf{(a)} Let $a\in\mathbb{K}%
$. Recall that $\mathbb{K}$ is an $\mathbb{F}_{p}$-algebra and thus satisfies
the module axioms; hence, $1_{\mathbb{F}_{p}}a=a$ and $0_{\mathbb{F}_{p}}a=0$.

The definition of $\mathbb{F}_{p}$ readily yields $p\cdot1_{\mathbb{F}_{p}%
}=0_{\mathbb{F}_{p}}$\ \ \ \ \footnote{\textit{Proof.} Recall that
$\mathbb{F}_{p}=\mathbb{Z}/p$ (where we are using the notations from
Convention \ref{conv.quotient}). Thus, the elements of $\mathbb{F}_{p}$ are
residue classes $\left[  u\right]  _{p}$ of integers $u$ modulo $p$. In
particular, $1_{\mathbb{F}_{p}}=\left[  1\right]  _{p}$. Thus,
\begin{align*}
p\cdot\underbrace{1_{\mathbb{F}_{p}}}_{=\left[  1\right]  _{p}}  &
=p\cdot\left[  1\right]  _{p}=\left[  p\cdot1\right]  _{p}=\left[  0\right]
_{p}\ \ \ \ \ \ \ \ \ \ \left(  \text{since }p\cdot1=p\equiv
0\operatorname{mod}p\right) \\
&  =0_{\mathbb{F}_{p}},
\end{align*}
qed.}. Now, using $1_{\mathbb{F}_{p}}a=a$, we find%
\[
p\underbrace{a}_{=1_{\mathbb{F}_{p}}a}=\underbrace{p\cdot1_{\mathbb{F}_{p}}%
}_{=0_{\mathbb{F}_{p}}}a=0_{\mathbb{F}_{p}}a=0.
\]
This proves Proposition \ref{prop.Fp-alg.p=0} \textbf{(a)}.

\textbf{(b)} There are two ways of proving this. One is the \textquotedblleft
right\textquotedblright\ way (in a philosophical sense), while another is the
short way.

Here is the \textquotedblleft right\textquotedblright\ way: We know that
$\mathbb{K}$ is a commutative $\mathbb{F}_{p}$-algebra, and we know that
$\mathbb{K}\left[  x\right]  $ is a $\mathbb{K}$-algebra. Hence, Proposition
\ref{prop.restrict-alg-isol} (applied to $\mathbb{F}_{p}$, $\mathbb{K}$ and
$\mathbb{K}\left[  x\right]  $ instead of $\mathbb{K}$, $\mathbb{L}$ and $V$)
shows that $\mathbb{K}\left[  x\right]  $ becomes an $\mathbb{F}_{p}$-algebra
in a natural way. Thus, Proposition \ref{prop.Fp-alg.p=0} \textbf{(a)}
(applied to $\mathbb{K}\left[  x\right]  $ and $\mathbf{a}$ instead of
$\mathbb{K}$ and $a$) shows that $p\mathbf{a}=0$ for each $\mathbf{a}%
\in\mathbb{K}\left[  x\right]  $. This proves Proposition
\ref{prop.Fp-alg.p=0} \textbf{(b)}.

Here is the short way: Let $\mathbf{a}\in\mathbb{K}\left[  x\right]  $. Write
the polynomial $\mathbf{a}$ in the form $\mathbf{a}=\sum_{i=0}^{n}a_{i}x^{i}$
with $n\in\mathbb{N}$ and $a_{0},a_{1},\ldots,a_{n}\in\mathbb{K}$. Then,
\[
p\underbrace{\mathbf{a}}_{=\sum_{i=0}^{n}a_{i}x^{i}}=p\sum_{i=0}^{n}a_{i}%
x^{i}=\sum_{i=0}^{n}\underbrace{pa_{i}}_{\substack{=0\\\text{(by Proposition
\ref{prop.Fp-alg.p=0} \textbf{(a)},}\\\text{applied to }a=a_{i}\text{)}}%
}x^{i}=\sum_{i=0}^{n}0x^{i}=0.
\]
This proves Proposition \ref{prop.Fp-alg.p=0} \textbf{(b)} again.
\end{proof}

\begin{proof}
[Proof of Theorem \ref{thm.Frob.end}.]Proposition \ref{prop.Fp-alg.p=0}
\textbf{(a)} (applied to $a=1_{\mathbb{K}}$) yields $p\cdot1_{\mathbb{K}}=0$.
Hence, $\mathbb{K}$ is a commutative ring such that $p\cdot1_{\mathbb{K}}=0$.
Also, $F_{\mathbb{K}}$ is the map
\[
\mathbb{K}\rightarrow\mathbb{K},\ \ \ \ \ \ \ \ \ \ a\mapsto a^{p}.
\]
Hence, \cite[Corollary 5.11.3]{19s} (applied to $F=F_{\mathbb{K}}$) shows that
$F_{\mathbb{K}}$ is a ring homomorphism.\footnote{Don't be fooled by the
reference to \cite{19s}; this is not a difficult result. Here is an outline of
the proof: It clearly suffices to show that $F_{\mathbb{K}}\left(  0\right)
=0$ and $F_{\mathbb{K}}\left(  1\right)  =1$ and $F_{\mathbb{K}}\left(
a+b\right)  =F_{\mathbb{K}}\left(  a\right)  +F_{\mathbb{K}}\left(  b\right)
$ and $F_{\mathbb{K}}\left(  ab\right)  =F_{\mathbb{K}}\left(  a\right)  \cdot
F_{\mathbb{K}}\left(  b\right)  $ for all $a,b\in\mathbb{K}$. In other words,
it suffices to show that $0^{p}=0$ and $1^{p}=1$ and $\left(  a+b\right)
^{p}=a^{p}+b^{p}$ and $\left(  ab\right)  ^{p}=a^{p}b^{p}$ for all
$a,b\in\mathbb{K}$ (because $F_{\mathbb{K}}\left(  u\right)  =u^{p}$ for each
$u\in\mathbb{K}$). But the first two of these four equalities are obvious,
whereas the fourth one follows from the commutativity of $\mathbb{K}$. It thus
remains to prove the third equality, i.e., to prove that $\left(  a+b\right)
^{p}=a^{p}+b^{p}$ for all $a,b\in\mathbb{K}$. But this is the famous
\textquotedblleft Freshman's Dream\textquotedblright, and can be shown by
expanding $\left(  a+b\right)  ^{p}$ using the binomial theorem and recalling
that all binomial coefficients $\dbinom{p}{k}$ for $k\in\left\{
1,2,\ldots,p-1\right\}  $ are divisible by $p$ (which means that they vanish
when they are used to scale elements of $\mathbb{K}$, by Proposition
\ref{prop.Fp-alg.p=0} \textbf{(a)}). Thus, all four equalities are proven, so
that $F_{\mathbb{K}}$ is a ring homomorphism.} Hence, $F_{\mathbb{K}}$ sends
$0$ to $0$ and respects addition. Furthermore, we have $F_{\mathbb{K}}\left(
\lambda a\right)  =\lambda F_{\mathbb{K}}\left(  a\right)  $ for each
$\lambda\in\mathbb{F}_{p}$ and $a\in\mathbb{K}$%
\ \ \ \ \footnote{\textit{Proof.} Let $\lambda\in\mathbb{F}_{p}$ and
$a\in\mathbb{K}$. Then, Theorem \ref{thm.field-FLT} (applied to $\mathbb{F}%
=\mathbb{F}_{p}$ and $\alpha=\lambda$) yields $\lambda^{\left\vert
\mathbb{F}_{p}\right\vert }=\lambda$. In view of $\left\vert \mathbb{F}%
_{p}\right\vert =p$, this rewrites as $\lambda^{p}=\lambda$. But the
definition of $F_{\mathbb{K}}$ yields $F_{\mathbb{K}}\left(  a\right)  =a^{p}$
and $F_{\mathbb{K}}\left(  \lambda a\right)  =\left(  \lambda a\right)
^{p}=\lambda^{p}a^{p}$ (by \cite[Proposition 6.9.7 \textbf{(b)}]{19s}, applied
to $\mathbb{F}_{p}$, $\mathbb{K}$ and $p$ instead of $\mathbb{K}$, $A$ and
$k$). Hence, $F_{\mathbb{K}}\left(  \lambda a\right)  =\underbrace{\lambda
^{p}}_{=\lambda}\underbrace{a^{p}}_{=F_{\mathbb{K}}\left(  a\right)  }=\lambda
F_{\mathbb{K}}\left(  a\right)  $, qed.}. Thus, $F_{\mathbb{K}}$ respects
scaling (where we consider $\mathbb{K}$ as an $\mathbb{F}_{p}$-module). Hence,
the map $F_{\mathbb{K}}$ is an $\mathbb{F}_{p}$-module homomorphism (since it
sends $0$ to $0$ and respects addition and respects scaling). Thus, this map
$F_{\mathbb{K}}$ is an $\mathbb{F}_{p}$-algebra homomorphism (since it is a
ring homomorphism and an $\mathbb{F}_{p}$-module homomorphism). This proves
Theorem \ref{thm.Frob.end}.
\end{proof}

Professional algebraists occasionally get really lazy and shorten
\textquotedblleft the Frobenius endomorphism\textquotedblright\ to
\textquotedblleft the Frobenius\textquotedblright.

The following property of the Frobenius endomorphism is an easy induction exercise:

\begin{proposition}
\label{prop.Frob.power}Let $\mathbb{K}$ be a commutative $\mathbb{F}_{p}%
$-algebra. Let $a\in\mathbb{K}$. Let $F$ be the map $F_{\mathbb{K}}%
:\mathbb{K}\rightarrow\mathbb{K}$. Then,%
\begin{equation}
F^{i}\left(  a\right)  =a^{p^{i}}\ \ \ \ \ \ \ \ \ \ \text{for all }%
i\in\mathbb{N}. \label{eq.prop.Frob.power.claim}%
\end{equation}

\end{proposition}

\begin{proof}
[Proof of Proposition \ref{prop.Frob.power}.]We know that $F$ is the map
$F_{\mathbb{K}}$. Thus, for each $u\in\mathbb{K}$, we have%
\begin{equation}
F\left(  u\right)  =F_{\mathbb{K}}\left(  u\right)  =u^{p}
\label{pf.prop.Frob.power.Fu=}%
\end{equation}
(by the definition of $F_{\mathbb{K}}$).

We shall prove (\ref{eq.prop.Frob.power.claim}) by induction on $i$:

\textit{Induction base:} Comparing $\underbrace{F^{0}}_{=\operatorname*{id}%
}\left(  a\right)  =\operatorname*{id}\left(  a\right)  =a$ with $a^{p^{0}%
}=a^{1}=a$, we obtain $F^{0}\left(  a\right)  =a^{p^{0}}$. In other words,
(\ref{eq.prop.Frob.power.claim}) holds for $i=0$. This completes the induction base.

\textit{Induction step:} Fix $j\in\mathbb{N}$. Assume that
(\ref{eq.prop.Frob.power.claim}) holds for $i=j$. We must prove that
(\ref{eq.prop.Frob.power.claim}) holds for $i=j+1$.

We have assumed that (\ref{eq.prop.Frob.power.claim}) holds for $i=j$. In
other words, we have $F^{j}\left(  a\right)  =a^{p^{j}}$. Now,%
\begin{align*}
\underbrace{F^{j+1}}_{=F\circ F^{j}}\left(  a\right)   &  =\left(  F\circ
F^{j}\right)  \left(  a\right)  =F\left(  \underbrace{F^{j}\left(  a\right)
}_{=a^{p^{j}}}\right)  =F\left(  a^{p^{j}}\right) \\
&  =\left(  a^{p^{j}}\right)  ^{p}\ \ \ \ \ \ \ \ \ \ \left(  \text{by
(\ref{pf.prop.Frob.power.Fu=}), applied to }u=a^{p^{j}}\right) \\
&  =a^{p^{j}p}=a^{p^{j+1}}\ \ \ \ \ \ \ \ \ \ \left(  \text{since }%
p^{j}p=p^{j+1}\right)  .
\end{align*}
In other words, (\ref{eq.prop.Frob.power.claim}) holds for $i=j+1$. This
completes the induction step. Hence, (\ref{eq.prop.Frob.power.claim}) is proven.

Thus, the proof of Proposition \ref{prop.Frob.power} is done.
\end{proof}

Combining Proposition \ref{prop.Frob.power} with Theorem \ref{thm.field-FLT},
we obtain the following:

\begin{corollary}
\label{cor.Frob.Fnid}Let $n\in\mathbb{N}$. Let $\mathbb{F}$ be a finite
$\mathbb{F}_{p}$-field of size $p^{n}$. Let $F$ be the map $F_{\mathbb{F}%
}:\mathbb{F}\rightarrow\mathbb{F}$. Then, $F^{n}=\operatorname*{id}$.
\end{corollary}

\begin{proof}
[Proof of Corollary \ref{cor.Frob.Fnid}.]We have $\left\vert \mathbb{F}%
\right\vert =p^{n}$ (since $\mathbb{F}$ has size $p^{n}$). Moreover,
$\mathbb{F}$ is an $\mathbb{F}_{p}$-field; in other words, $\mathbb{F}$ is an
$\mathbb{F}_{p}$-algebra that is a field. Hence, $\mathbb{F}$ is commutative
(since $\mathbb{F}$ is a field).

Let $a\in\mathbb{F}$. Then, Theorem \ref{thm.field-FLT} (applied to $\alpha
=a$) yields $a^{\left\vert \mathbb{F}\right\vert }=a$. In view of $\left\vert
\mathbb{F}\right\vert =p^{n}$, this rewrites as $a^{p^{n}}=a$. But Proposition
\ref{prop.Frob.power} (applied to $\mathbb{K}=\mathbb{F}$ and $i=n$) yields
$F^{n}\left(  a\right)  =a^{p^{n}}=a=\operatorname*{id}\left(  a\right)  $.

Forget that we fixed $a$. We thus have shown that $F^{n}\left(  a\right)
=\operatorname*{id}\left(  a\right)  $ for each $a\in\mathbb{F}$. In other
words, $F^{n}=\operatorname*{id}$. This proves Corollary \ref{cor.Frob.Fnid}.
\end{proof}

\begin{corollary}
\label{cor.Frob.Fnidr}Let $n$ be a positive integer. Let $\mathbb{F}$ be a
finite $\mathbb{F}_{p}$-field of size $p^{n}$. Let $a\in\mathbb{F}$ and
$r\in\mathbb{N}$. Then, $a^{p^{nr}}=a$.
\end{corollary}

\begin{proof}
[Proof of Corollary \ref{cor.Frob.Fnidr}.]We know that $\mathbb{F}$ is an
$\mathbb{F}_{p}$-field; in other words, $\mathbb{F}$ is an $\mathbb{F}_{p}%
$-algebra that is a field. Hence, $\mathbb{F}$ is commutative (since
$\mathbb{F}$ is a field).

Let $F$ be the map $F_{\mathbb{F}}:\mathbb{F}\rightarrow\mathbb{F}$. Then,
Corollary \ref{cor.Frob.Fnid} yields $F^{n}=\operatorname*{id}$. Now,
$F^{nr}=\left(  \underbrace{F^{n}}_{=\operatorname*{id}}\right)
^{r}=\operatorname*{id}\nolimits^{r}=\operatorname*{id}$. But Proposition
\ref{prop.Frob.power} (applied to $\mathbb{K}=\mathbb{F}$ and $i=nr$) yields
$F^{nr}\left(  a\right)  =a^{p^{nr}}$. Hence, $a^{p^{nr}}=\underbrace{F^{nr}%
}_{=\operatorname*{id}}\left(  a\right)  =\operatorname*{id}\left(  a\right)
=a$. This proves Corollary \ref{cor.Frob.Fnidr}.
\end{proof}

\subsection{Polynomials over fields have only so many roots}

Our next ingredient is a basic property of polynomials over fields. First we
define a slightly nonstandard notation:

\begin{convention}
Let $\mathbb{K}$ be a commutative ring. Let $\mathbf{f}\in\mathbb{K}\left[
x\right]  $ be a polynomial. Let $U$ be a $\mathbb{K}$-algebra. Let $u\in U$.
Then, $\mathbf{f}\left[  u\right]  $ will denote the value of the polynomial
$\mathbf{f}$ at $u$. (See \cite[Definition 7.6.1]{19s} for the definition of
this value. Roughly speaking, this value is obtained by substituting $u$ for
$x$ in $\mathbf{f}$.)
\end{convention}

I am using this notation $\mathbf{f}\left[  u\right]  $ in lieu of the more
usual notation $\mathbf{f}\left(  u\right)  $, since the latter can too easily
be mistaken for a product.

\begin{theorem}
\label{thm.FTAeasy}Let $\mathbb{K}$ be a field. Let $n\in\mathbb{N}$. Then,
any nonzero polynomial $\mathbf{a}\in\mathbb{K}\left[  x\right]  $ of degree
$\leq n$ has at most $n$ roots in $\mathbb{K}$. (We are not counting the roots
with multiplicity here.)
\end{theorem}

\begin{proof}
[Proof of Theorem \ref{thm.FTAeasy}.]This is \cite[Theorem 7.6.11]{19s}.
\end{proof}

For us, the use of Theorem \ref{thm.FTAeasy} is through the following corollary:

\begin{corollary}
\label{cor.inverse-field-FLT}Let $u$ be an integer such that $u>1$. Let
$\mathbb{F}$ be a field. Assume that all $a\in\mathbb{F}$ satisfy $a^{u}=a$.
Then, $\left\vert \mathbb{F}\right\vert \leq u$. (This means, in particular,
that the field $\mathbb{F}$ is finite.)
\end{corollary}

\begin{proof}
[Proof of Corollary \ref{cor.inverse-field-FLT}.]Define a polynomial
$\mathbf{a}\in\mathbb{F}\left[  x\right]  $ by $\mathbf{a}=x^{u}-x$. Then, the
leading term of $\mathbf{a}$ is $x^{u}$ (since $u>1$). Thus, the polynomial
$\mathbf{a}$ is monic of degree $u$; hence, $\mathbf{a}$ is nonzero. Hence,
Theorem \ref{thm.FTAeasy} (applied to $\mathbb{K}=\mathbb{F}$ and $n=u$) shows
that the polynomial $\mathbf{a}$ has at most $u$ roots in $\mathbb{F}$ (since
$\mathbf{a}$ has degree $u$ and thus has degree $\leq u$). In other words, the
number of roots of $\mathbf{a}$ in $\mathbb{F}$ is $\leq u$. In other words,
$\left\vert \left\{  \text{roots of }\mathbf{a}\text{ in }\mathbb{F}\right\}
\right\vert \leq u$.

But\textit{ }$\mathbb{F}\subseteq\left\{  \text{roots of }\mathbf{a}\text{ in
}\mathbb{F}\right\}  $\ \ \ \ \footnote{\textit{Proof.} Let $v\in\mathbb{F}$.
Then, $\mathbf{a}\left[  v\right]  =v^{u}-v$ (since $\mathbf{a}=x^{u}-x$). But
we assumed that all $a\in\mathbb{F}$ satisfy $a^{u}=a$. Applying this to
$a=v$, we find $v^{u}=v$. Hence, $\mathbf{a}\left[  v\right]  =v^{u}-v=0$
(since $v^{u}=v$). In other words, $v$ is a root of $\mathbf{a}$ in
$\mathbb{F}$. In other words, $v\in\left\{  \text{roots of }\mathbf{a}\text{
in }\mathbb{F}\right\}  $.
\par
Now, forget that we fixed $v$. We have thus shown that $v\in\left\{
\text{roots of }\mathbf{a}\text{ in }\mathbb{F}\right\}  $ for each
$v\in\mathbb{F}$. In other words, $\mathbb{F}\subseteq\left\{  \text{roots of
}\mathbf{a}\text{ in }\mathbb{F}\right\}  $.}. Hence, $\left\vert
\mathbb{F}\right\vert \leq\left\vert \left\{  \text{roots of }\mathbf{a}\text{
in }\mathbb{F}\right\}  \right\vert \leq u$. This proves Corollary
\ref{cor.inverse-field-FLT}.
\end{proof}

\subsection{Factorization into distinct factors and derivatives}

Another piece of our puzzle is the notion of the derivative of a polynomial.
We recall its definition:

\begin{definition}
\label{def.f'}Let $\mathbb{K}$ be a commutative ring.

\textbf{(a)} For each polynomial
\[
\mathbf{f}=\sum_{k\in\mathbb{N}}a_{k}x^{k}=a_{0}x^{0}+a_{1}x^{1}+a_{2}%
x^{2}+\cdots\in\mathbb{K}\left[  x\right]  \qquad\text{(where $a_{i}%
\in\mathbb{K}$),}%
\]
we define the \textit{derivative} $\mathbf{f}^{\prime}$ of $\mathbf{f}$ to be
the polynomial
\[
\sum_{k>0}ka_{k}x^{k-1}=1a_{1}x^{0}+2a_{2}x^{1}+3a_{3}x^{2}+\cdots
\in\mathbb{K}\left[  x\right]  .
\]


\textbf{(b)} Let $D:\mathbb{K}\left[  x\right]  \rightarrow\mathbb{K}\left[
x\right]  $ be the map sending each polynomial $\mathbf{f}$ to its derivative
$\mathbf{f}^{\prime}$.
\end{definition}

Definition \ref{def.f'} \textbf{(a)} is a particular case of the definition of
$\mathbf{f}^{\prime}$ in \cite[Exercise 5]{19s-mt3s}\footnote{Note that
\cite[Exercise 5]{19s-mt3s} uses the letter "$f$" instead of our "$\mathbf{f}%
$".}; more precisely, the latter definition defines $\mathbf{f}^{\prime}$ for
every formal power series $\mathbf{f}$, whereas here we restrict ourselves to
the case when $\mathbf{f}$ is a polynomial. It is almost obvious that
Definition \ref{def.f'} \textbf{(a)} is well-defined (i.e., the infinite sum
$\sum_{k>0}ka_{k}x^{k-1}$ in this definition actually is a polynomial); see
\cite[Exercise 5 \textbf{(a)}]{19s-mt3s} for the detailed proof of this fact.

The map $D$ in Definition \ref{def.f'} \textbf{(b)} is a restriction of the
map $D$ in \cite[Exercise 5]{19s-mt3s}. (Indeed, the latter map is defined on
formal power series, while the former map is defined on polynomials; but the
two maps are defined by the same rule.) Thus, any formulas for values of $D$
proven in \cite{19s-mt3s} are still valid for our map $D$, as long as they are
being applied to polynomials.

We shall need the following basic properties of derivatives:

\begin{proposition}
\label{prop.f'.leib}Let $\mathbb{K}$ be a commutative ring. Let $\mathbf{f}$
and $\mathbf{g}$ be two polynomials in $\mathbb{K}\left[  x\right]  $.

\textbf{(a)} We have $\left(  \mathbf{f}+\mathbf{g}\right)  ^{\prime
}=\mathbf{f}^{\prime}+\mathbf{g}^{\prime}$.

\textbf{(b)} We have $\left(  \mathbf{f}-\mathbf{g}\right)  ^{\prime
}=\mathbf{f}^{\prime}-\mathbf{g}^{\prime}$.

\textbf{(c)} We have $\left(  \mathbf{fg}\right)  ^{\prime}=\mathbf{f}%
^{\prime}\mathbf{g}+\mathbf{fg}^{\prime}$.
\end{proposition}

\begin{proof}
[Proof of Proposition \ref{prop.f'.leib}.]In \cite[Exercise 5 \textbf{(b)}%
]{19s-mt3s}, it is shown that the map $D$ is $\mathbb{K}$-linear\footnote{More
precisely: In \cite[Exercise 5 \textbf{(b)}]{19s-mt3s}, it is shown that the
map $D$ from \cite[Exercise 5]{19s-mt3s} is $\mathbb{K}$-linear. This is not
exactly our map $D$, but our map $D$ is a restriction of this map; thus, it
follows that our map $D$ is $\mathbb{K}$-linear as well.}. This quickly yields
parts \textbf{(a)} and \textbf{(b)} of Proposition \ref{prop.f'.leib}%
\footnote{In more details: The map $D$ is $\mathbb{K}$-linear. Thus, $D\left(
\mathbf{f}+\mathbf{g}\right)  =\underbrace{D\left(  \mathbf{f}\right)
}_{\substack{=\mathbf{f}^{\prime}\\\text{(by the definition of }D\text{)}%
}}+\underbrace{D\left(  \mathbf{g}\right)  }_{\substack{=\mathbf{g}^{\prime
}\\\text{(by the definition of }D\text{)}}}=\mathbf{f}^{\prime}+\mathbf{g}%
^{\prime}$. Comparing this with%
\[
D\left(  \mathbf{f}+\mathbf{g}\right)  =\left(  \mathbf{f}+\mathbf{g}\right)
^{\prime}\ \ \ \ \ \ \ \ \ \ \left(  \text{by the definition of }D\right)  ,
\]
we obtain $\left(  \mathbf{f}+\mathbf{g}\right)  ^{\prime}=\mathbf{f}^{\prime
}+\mathbf{g}^{\prime}$. This proves Proposition \ref{prop.f'.leib}
\textbf{(a)}. A similar argument proves Proposition \ref{prop.f'.leib}
\textbf{(b)}.}. Proposition \ref{prop.f'.leib} \textbf{(c)} follows from
\cite[Exercise 5 \textbf{(c)}]{19s-mt3s} (applied to $f=\mathbf{f}$ and
$g=\mathbf{g}$) or from \cite[Proposition 0.2 \textbf{(c)}]{Grinbe18}%
\footnote{In \cite{Grinbe18}, I denote the indeterminate by $t$ rather than
$x$, and I use the notation $\dfrac{d}{dt}\mathbf{h}$ for the derivative
$\mathbf{h}^{\prime}$ of a polynomial $\mathbf{h}$.} (applied to
$f=\mathbf{f}$ and $g=\mathbf{g}$).
\end{proof}

Proposition \ref{prop.f'.leib} \textbf{(c)} is known as the \textit{Leibniz
law} (or \textit{Leibniz identity}) for derivatives of polynomials. We will
need the following consequence of Proposition \ref{prop.f'.leib}:

\begin{corollary}
\label{cor.f'.leib2}Let $\mathbb{K}$ be a commutative ring. Let $\mathbf{f}$
and $\mathbf{g}$ be two polynomials in $\mathbb{K}\left[  x\right]  $. Then,%
\[
\left(  \mathbf{f}^{2}\mathbf{g}\right)  ^{\prime}=\mathbf{f}\cdot\left(
2\mathbf{f}^{\prime}\mathbf{g}+\mathbf{fg}^{\prime}\right)  .
\]

\end{corollary}

\begin{proof}
[Proof of Corollary \ref{cor.f'.leib2}.]Proposition \ref{prop.f'.leib}
\textbf{(c)} (applied to $\mathbf{f}$ instead of $\mathbf{g}$) shows that%
\[
\left(  \mathbf{ff}\right)  ^{\prime}=\underbrace{\mathbf{f}^{\prime
}\mathbf{f}}_{=\mathbf{ff}^{\prime}}+\mathbf{ff}^{\prime}=\mathbf{ff}^{\prime
}+\mathbf{ff}^{\prime}=2\mathbf{ff}^{\prime}.
\]
In view of $\mathbf{ff}=\mathbf{f}^{2}$, this rewrites as $\left(
\mathbf{f}^{2}\right)  ^{\prime}=2\mathbf{ff}^{\prime}$. Now, Proposition
\ref{prop.f'.leib} \textbf{(c)} (applied to $\mathbf{f}^{2}$ instead of
$\mathbf{f}$) shows that%
\[
\left(  \mathbf{f}^{2}\mathbf{g}\right)  ^{\prime}=\underbrace{\left(
\mathbf{f}^{2}\right)  ^{\prime}}_{=2\mathbf{ff}^{\prime}}\mathbf{g}%
+\mathbf{f}^{2}\mathbf{g}^{\prime}=2\mathbf{ff}^{\prime}\mathbf{g}%
+\mathbf{f}^{2}\mathbf{g}^{\prime}=\mathbf{f}\cdot\left(  2\mathbf{f}^{\prime
}\mathbf{g}+\mathbf{fg}^{\prime}\right)  .
\]
This proves Corollary \ref{cor.f'.leib2}.
\end{proof}

\begin{corollary}
\label{cor.irred-factor-distinct}Let $\mathbb{K}$ be a field. Let
$\mathbf{a}\in\mathbb{K}\left[  x\right]  $ be a monic polynomial such that
$\deg\left(  \mathbf{a}^{\prime}\right)  =0$ (that is, the polynomial
$\mathbf{a}^{\prime}$ is constant and nonzero). Then, $\mathbf{a}$ can be
written in the form $\mathbf{a}=\mathbf{u}_{1}\mathbf{u}_{2}\cdots
\mathbf{u}_{k}$, where $\mathbf{u}_{1},\mathbf{u}_{2},\ldots,\mathbf{u}_{k}%
\in\mathbb{K}\left[  x\right]  $ are \textbf{distinct} monic irreducible polynomials.
\end{corollary}

\begin{proof}
[Proof of Corollary \ref{cor.irred-factor-distinct}.]Corollary
\ref{cor.irred-factor-monic} (applied to $\mathbb{F}=\mathbb{K}$) shows that
$\mathbf{a}$ can be written in the form $\mathbf{a}=\mathbf{u}_{1}%
\mathbf{u}_{2}\cdots\mathbf{u}_{k}$, where $\mathbf{u}_{1},\mathbf{u}%
_{2},\ldots,\mathbf{u}_{k}\in\mathbb{K}\left[  x\right]  $ are monic
irreducible polynomials. Consider these $\mathbf{u}_{1},\mathbf{u}_{2}%
,\ldots,\mathbf{u}_{k}$.

Next, we shall show that the polynomials $\mathbf{u}_{1},\mathbf{u}_{2}%
,\ldots,\mathbf{u}_{k}$ are distinct.

Indeed, assume the contrary. Thus, there exist two elements $i$ and $j$ of
$\left\{  1,2,\ldots,k\right\}  $ such that $i<j$ and $\mathbf{u}%
_{i}=\mathbf{u}_{j}$. Consider these $i$ and $j$. Let $\mathbf{g}$ denote the
polynomial $\prod_{\substack{h\in\left\{  1,2,\ldots,k\right\}  ;\\h\neq
i\text{ and }h\neq j}}\mathbf{u}_{h}\in\mathbb{K}\left[  x\right]  $. (This
may be an empty product, i.e., the constant polynomial $1$.) Let
$\mathbf{f}\in\mathbb{K}\left[  x\right]  $ be the polynomial $\mathbf{u}_{j}%
$. Then, $\mathbf{f}=\mathbf{u}_{j}$, and thus $\mathbf{f}$ is irreducible
(since $\mathbf{u}_{j}$ is irreducible). Hence, $\deg\mathbf{f}>0$.

Now, $\mathbf{u}_{i}$ and $\mathbf{u}_{j}$ are two distinct factors of the
product $\mathbf{u}_{1}\mathbf{u}_{2}\cdots\mathbf{u}_{k}$ (since $i<j$).
Splitting off these two factors, we obtain%
\[
\mathbf{u}_{1}\mathbf{u}_{2}\cdots\mathbf{u}_{k}=\underbrace{\mathbf{u}_{i}%
}_{=\mathbf{u}_{j}=\mathbf{f}}\underbrace{\mathbf{u}_{j}}_{=\mathbf{f}%
}\underbrace{\prod_{\substack{h\in\left\{  1,2,\ldots,k\right\}  ;\\h\neq
i\text{ and }h\neq j}}\mathbf{u}_{h}}_{\substack{=\mathbf{g}\\\text{(by the
definition of }\mathbf{g}\text{)}}}=\mathbf{ffg}=\mathbf{f}^{2}\mathbf{g}.
\]
Therefore,%
\[
\mathbf{a}=\mathbf{u}_{1}\mathbf{u}_{2}\cdots\mathbf{u}_{k}=\mathbf{f}%
^{2}\mathbf{g}.
\]
Taking derivatives on both sides of this equality, we find%
\[
\mathbf{a}^{\prime}=\left(  \mathbf{f}^{2}\mathbf{g}\right)  ^{\prime
}=\mathbf{f}\cdot\left(  2\mathbf{f}^{\prime}\mathbf{g}+\mathbf{fg}^{\prime
}\right)
\]
(by Corollary \ref{cor.f'.leib2}). Thus, $\mathbf{f}\cdot\left(
2\mathbf{f}^{\prime}\mathbf{g}+\mathbf{fg}^{\prime}\right)  =\mathbf{a}%
^{\prime}\neq0$ (since $\deg\left(  \mathbf{a}^{\prime}\right)  =0\neq-\infty
$). Hence, both polynomials $\mathbf{f}$ and $2\mathbf{f}^{\prime}%
\mathbf{g}+\mathbf{fg}^{\prime}$ are nonzero. Thus,%
\[
\deg\left(  \mathbf{f}\cdot\left(  2\mathbf{f}^{\prime}\mathbf{g}%
+\mathbf{fg}^{\prime}\right)  \right)  =\underbrace{\deg\mathbf{f}}%
_{>0}+\underbrace{\deg\left(  2\mathbf{f}^{\prime}\mathbf{g}+\mathbf{fg}%
^{\prime}\right)  }_{\substack{\geq0\\\text{(since }2\mathbf{f}^{\prime
}\mathbf{g}+\mathbf{fg}^{\prime}\text{ is nonzero)}}}>0.
\]
This contradicts $\deg\underbrace{\left(  \mathbf{f}\cdot\left(
2\mathbf{f}^{\prime}\mathbf{g}+\mathbf{fg}^{\prime}\right)  \right)
}_{=\mathbf{a}^{\prime}}=\deg\left(  \mathbf{a}^{\prime}\right)  =0$. This
contradiction shows that our assumption was false. Hence, we have proven that
the polynomials $\mathbf{u}_{1},\mathbf{u}_{2},\ldots,\mathbf{u}_{k}$ are distinct.

We thus have found distinct monic irreducible polynomials $\mathbf{u}%
_{1},\mathbf{u}_{2},\ldots,\mathbf{u}_{k}\in\mathbb{K}\left[  x\right]  $ such
that $\mathbf{a}=\mathbf{u}_{1}\mathbf{u}_{2}\cdots\mathbf{u}_{k}$. In other
words, we have written $\mathbf{a}$ in the form $\mathbf{a}=\mathbf{u}%
_{1}\mathbf{u}_{2}\cdots\mathbf{u}_{k}$, where $\mathbf{u}_{1},\mathbf{u}%
_{2},\ldots,\mathbf{u}_{k}\in\mathbb{K}\left[  x\right]  $ are distinct monic
irreducible polynomials. Thus, $\mathbf{a}$ can be written in this form. This
proves Corollary \ref{cor.irred-factor-distinct}.
\end{proof}

We furthermore need a formula for derivatives of monomials:

\begin{proposition}
\label{prop.Dxm}Let $\mathbb{K}$ be a commutative ring. Let $m\in\mathbb{N}$.
Then, in $\mathbb{K}\left[  x\right]  $, we have $D\left(  x^{m}\right)
=mx^{m-1}$. (Here, the expression \textquotedblleft$mx^{m-1}$%
\textquotedblright\ is to be understood as $0$ when $m=0$.)
\end{proposition}

\begin{proof}
[Proof of Proposition \ref{prop.Dxm}.]This is proven in \cite[Statement 8 in
the solution to Exercise 5]{19s-mt3s}.
\end{proof}

We can restate Proposition \ref{prop.Dxm} as follows:

\begin{proposition}
\label{prop.xm'}Let $\mathbb{K}$ be a commutative ring. Let $m\in\mathbb{N}$.
Then, in $\mathbb{K}\left[  x\right]  $, we have $\left(  x^{m}\right)
^{\prime}=mx^{m-1}$. (Here, the expression \textquotedblleft$mx^{m-1}%
$\textquotedblright\ is to be understood as $0$ when $m=0$.)
\end{proposition}

\begin{proof}
[Proof of Proposition \ref{prop.xm'}.]Proposition \ref{prop.Dxm} yields
$D\left(  x^{m}\right)  =mx^{m-1}$. But the definition of $D$ yields $D\left(
x^{m}\right)  =\left(  x^{m}\right)  ^{\prime}$. Comparing these two
equalities, we find $\left(  x^{m}\right)  ^{\prime}=mx^{m-1}$. This proves
Proposition \ref{prop.xm'}.
\end{proof}

\begin{corollary}
\label{cor.pxm'}Let $\mathbb{K}$ be a commutative $\mathbb{F}_{p}$-algebra.
Let $m$ be a positive integer satisfying $p\mid m$. Then, in $\mathbb{K}%
\left[  x\right]  $, we have $\left(  x^{m}\right)  ^{\prime}=0$.
\end{corollary}

\begin{proof}
[Proof of Corollary \ref{cor.pxm'}.]We have $p\mid m$. Thus, there exists some
integer $c$ such that $m=pc$. Consider this $c$.

Proposition \ref{prop.Fp-alg.p=0} \textbf{(b)} (applied to $\mathbf{a}%
=cx^{m-1}$) shows that $pcx^{m-1}=0$. But Proposition \ref{prop.xm'} yields%
\[
\left(  x^{m}\right)  ^{\prime}=\underbrace{m}_{=pc}x^{m-1}=pcx^{m-1}=0.
\]
This proves Corollary \ref{cor.pxm'}.
\end{proof}

\subsection{Factoring $x^{p^{g}}-x$, part I}

\begin{lemma}
\label{lem.xpg-x.1}Let $g$ be a positive integer. Let $\mathbb{K}$ be an
$\mathbb{F}_{p}$-field. Then, the polynomial $x^{p^{g}}-x\in\mathbb{K}\left[
x\right]  $ can be written in the form $x^{p^{g}}-x=\mathbf{u}_{1}%
\mathbf{u}_{2}\cdots\mathbf{u}_{k}$, where $\mathbf{u}_{1},\mathbf{u}%
_{2},\ldots,\mathbf{u}_{k}\in\mathbb{K}\left[  x\right]  $ are
\textbf{distinct} monic irreducible polynomials.
\end{lemma}

\begin{proof}
[Proof of Lemma \ref{lem.xpg-x.1}.]We know that $\mathbb{K}$ is an
$\mathbb{F}_{p}$-field. In other words, $\mathbb{K}$ is an $\mathbb{F}_{p}%
$-algebra that is a field. Hence, $\mathbb{K}$ is commutative (since
$\mathbb{K}$ is a field). Also, $\mathbb{K}$ is an $\mathbb{F}_{p}$-algebra;
thus, $\mathbb{K}\left[  x\right]  $ is an $\mathbb{F}_{p}$-algebra as well.

We have $g>0$ (since $g$ is a positive integer), hence $p^{g}>p^{0}=1$. Thus,
the leading term of the polynomial $x^{p^{g}}-x$ is $x^{p^{g}}$. This shows
that the polynomial $x^{p^{g}}-x$ is monic of degree $p^{g}$.

Proposition \ref{prop.xm'} (applied to $m=1$) yields $\left(  x^{1}\right)
^{\prime}=1\underbrace{x^{1-1}}_{=x^{0}=1}=1$. In view of $x^{1}=x$, this
rewrites as $x^{\prime}=1$.

Also, $g$ is a positive integer; hence, $g\geq1$, so that $g-1\in\mathbb{N}$.
Hence, $p^{g-1}$ is an integer. Thus, $p\mid p^{g}$ (since $p^{g}=p\cdot
p^{g-1}$). Hence, Corollary \ref{cor.pxm'} (applied to $m=p^{g}$) yields
$\left(  x^{p^{g}}\right)  ^{\prime}=0$.

Now, Proposition \ref{prop.f'.leib} \textbf{(b)} (applied to $x^{p^{g}}$ and
$x$ instead of $\mathbf{f}$ and $\mathbf{g}$) shows that
\[
\left(  x^{p^{g}}-x\right)  ^{\prime}=\underbrace{\left(  x^{p^{g}}\right)
^{\prime}}_{=0}-\underbrace{x^{\prime}}_{=1}=0-1=-1.
\]
Thus, $\deg\left(  \underbrace{\left(  x^{p^{g}}-x\right)  ^{\prime}}%
_{=-1}\right)  =\deg\left(  -1\right)  =0$. So we know that $x^{p^{g}}-x$ is a
monic polynomial such that $\deg\left(  \left(  x^{p^{g}}-x\right)  ^{\prime
}\right)  =0$ (that is, the polynomial $\left(  x^{p^{g}}-x\right)  ^{\prime}$
is constant and nonzero). Hence, Corollary \ref{cor.irred-factor-distinct}
(applied to $\mathbf{a}=x^{p^{g}}-x$) shows that $x^{p^{g}}-x$ can be written
in the form $x^{p^{g}}-x=\mathbf{u}_{1}\mathbf{u}_{2}\cdots\mathbf{u}_{k}$,
where $\mathbf{u}_{1},\mathbf{u}_{2},\ldots,\mathbf{u}_{k}\in\mathbb{K}\left[
x\right]  $ are \textbf{distinct} monic irreducible polynomials. This proves
Lemma \ref{lem.xpg-x.1}.
\end{proof}

\begin{lemma}
\label{lem.xpg-x.2}Let $r$ and $m$ be positive integers. Let $\mathbb{K}$ be a
finite $\mathbb{F}_{p}$-field of size $p^{m}$. Let $\mathbf{a}\in
\mathbb{K}\left[  x\right]  $ be an irreducible polynomial such that
$\mathbf{a}\mid x^{p^{mr}}-x$ in $\mathbb{K}\left[  x\right]  $. Then,
$\deg\mathbf{a}\mid r$ in $\mathbb{Z}$.
\end{lemma}

\begin{proof}
[Proof of Lemma \ref{lem.xpg-x.2}.]The integers $m$ and $r$ are positive.
Hence, their product $mr$ is positive as well.

Let $n=\deg\mathbf{a}$. Thus, $\mathbf{a}$ is a polynomial of degree $n$. Note
that $\mathbf{a}$ is irreducible; thus, $\deg\mathbf{a}>0$. Hence,
$n=\deg\mathbf{a}>0$. Thus, $n$ is a positive integer. Hence, $mn$ is a
positive integer (since $m$ is a positive integer).

We know that $\mathbb{K}$ is an $\mathbb{F}_{p}$-field. In other words,
$\mathbb{K}$ is an $\mathbb{F}_{p}$-algebra that is a field. Thus,
$\mathbb{K}$ is a field, so that $\mathbb{K}$ is commutative. Moreover,
$\mathbb{K}$ has size $p^{m}$; thus, $\left\vert \mathbb{K}\right\vert =p^{m}%
$. Now, Theorem \ref{thm.field-ext} \textbf{(b)} (applied to $\mathbb{F}%
=\mathbb{K}$) yields that
\begin{align*}
\left\vert \mathbb{K}\left[  x\right]  /\mathbf{a}\right\vert  &  =\left\vert
\mathbb{K}\right\vert ^{n}=\left(  p^{m}\right)  ^{n}%
\ \ \ \ \ \ \ \ \ \ \left(  \text{since }\left\vert \mathbb{K}\right\vert
=p^{m}\right) \\
&  =p^{mn}.
\end{align*}
Furthermore, Theorem \ref{thm.field-ext} \textbf{(c)} (applied to
$\mathbb{F}=\mathbb{K}$) yields that $\mathbb{K}\left[  x\right]  /\mathbf{a}$
is a field. Let $\mathbb{F}$ denote this field. Thus,
\[
\mathbb{F}=\mathbb{K}\left[  x\right]  /\mathbf{a}%
,\ \ \ \ \ \ \ \ \ \ \text{so that}\ \ \ \ \ \ \ \ \ \ \left\vert
\mathbb{F}\right\vert =\left\vert \mathbb{K}\left[  x\right]  /\mathbf{a}%
\right\vert =p^{mn}.
\]


We know that $\mathbb{K}$ is a commutative $\mathbb{F}_{p}$-algebra. We also
know that $\mathbb{K}\left[  x\right]  $ is a $\mathbb{K}$-algebra. Thus,
Proposition \ref{prop.restrict-alg-isol} (applied to $\mathbb{F}_{p}$,
$\mathbb{K}$ and $\mathbb{K}\left[  x\right]  $ instead of $\mathbb{K}$,
$\mathbb{L}$ and $V$) shows that $\mathbb{K}\left[  x\right]  $ becomes an
$\mathbb{F}_{p}$-algebra in a natural way.

We thus know that $\mathbb{K}\left[  x\right]  $ is a commutative
$\mathbb{F}_{p}$-algebra\footnote{since $\mathbb{K}\left[  x\right]  $ is
commutative}. We also know that $\mathbb{F}$ is a $\mathbb{K}\left[  x\right]
$-algebra (since $\mathbb{F}=\mathbb{K}\left[  x\right]  /\mathbf{a}$). Thus,
Proposition \ref{prop.restrict-alg-isol} (applied to $\mathbb{F}_{p}$,
$\mathbb{K}\left[  x\right]  $ and $\mathbb{F}$ instead of $\mathbb{K}$,
$\mathbb{L}$ and $V$) shows that $\mathbb{F}$ becomes an $\mathbb{F}_{p}%
$-algebra in a natural way. Thus, $\mathbb{F}$ is a commutative $\mathbb{F}%
_{p}$-algebra (since $\mathbb{F}$ is commutative). Also, $\mathbb{F}$ is an
$\mathbb{F}_{p}$-field (since $\mathbb{F}$ is an $\mathbb{F}_{p}$-algebra that
is also a field). Recall that $\mathbb{F}$ has size $p^{mn}$ (since
$\left\vert \mathbb{F}\right\vert =p^{mn}$).

Let $F$ be the Frobenius endomorphism $F_{\mathbb{F}}:\mathbb{F}%
\rightarrow\mathbb{F}$ of $\mathbb{F}$. (See Definition \ref{def.Frob} for the
definition of a Frobenius endomorphism.)

Recall Convention \ref{conv.quotient}. We have
\begin{equation}
\left(  \left[  f\right]  _{\mathbf{a}}\right)  ^{k}=\left[  f^{k}\right]
_{\mathbf{a}}\ \ \ \ \ \ \ \ \ \ \text{for each }f\in\mathbb{K}\left[
x\right]  \text{ and each }k\in\mathbb{N}. \label{pf.lem.xpg-x.2.fak}%
\end{equation}
(Indeed, this can be proven by a straightforward induction on $k$, using the
definition of the multiplication on $\mathbb{K}\left[  x\right]  /\mathbf{a}$.)

The elements $x$ and $x^{p^{mr}}$ of $\mathbb{K}\left[  x\right]  $ satisfy
$\mathbf{a}\mid x^{p^{mr}}-x$. In other words, $x^{p^{mr}}\equiv
x\operatorname{mod}\mathbf{a}$. In other words,
\begin{equation}
\left[  x^{p^{mr}}\right]  _{\mathbf{a}}=\left[  x\right]  _{\mathbf{a}}.
\label{pf.lem.xpg-x.2.xpmr}%
\end{equation}
(Again, recall that we are using Convention \ref{conv.quotient}.)

Proposition \ref{prop.Frob.power} (applied to $\mathbb{F}$, $\left[  x\right]
_{\mathbf{a}}$ and $mr$ instead of $\mathbb{K}$, $a$ and $i$) yields that
\begin{align}
F^{mr}\left(  \left[  x\right]  _{\mathbf{a}}\right)   &  =\left(  \left[
x\right]  _{\mathbf{a}}\right)  ^{p^{mr}}=\left[  x^{p^{mr}}\right]
_{\mathbf{a}}\ \ \ \ \ \ \ \ \ \ \left(  \text{by (\ref{pf.lem.xpg-x.2.fak}),
applied to }f=x\text{ and }k=p^{mr}\right) \nonumber\\
&  =\left[  x\right]  _{\mathbf{a}}\ \ \ \ \ \ \ \ \ \ \left(  \text{by
(\ref{pf.lem.xpg-x.2.xpmr})}\right)  . \label{pf.lem.xpg-x.2.5}%
\end{align}
Also, the Frobenius endomorphism $F_{\mathbb{F}}$ is an $\mathbb{F}_{p}%
$-algebra homomorphism (by Theorem \ref{thm.Frob.end}, applied to $\mathbb{F}$
instead of $\mathbb{K}$). In other words, $F$ is an $\mathbb{F}_{p}$-algebra
homomorphism (since $F=F_{\mathbb{F}}$). Hence, $F^{mr}$ is an $\mathbb{F}%
_{p}$-algebra homomorphism as well (since any composition of $\mathbb{F}_{p}%
$-algebra homomorphisms is an $\mathbb{F}_{p}$-algebra homomorphism). In other
words, the map $F^{mr}$ is an $\mathbb{F}_{p}$-module homomorphism and a ring
homomorphism at the same time.

Next, we shall show that
\begin{equation}
F^{mr}\left(  u\right)  =u\ \ \ \ \ \ \ \ \ \ \text{for each }u\in\mathbb{F}.
\label{pf.lem.xpg-x.2.6}%
\end{equation}


[\textit{Proof of (\ref{pf.lem.xpg-x.2.6}):} Let $u\in\mathbb{F}$. Thus,
$u\in\mathbb{F}=\mathbb{K}\left[  x\right]  /\mathbf{a}$. But Theorem
\ref{thm.field-ext} \textbf{(a)} (applied to $\mathbb{K}$ instead of
$\mathbb{F}$) yields that each element of $\mathbb{K}\left[  x\right]
/\mathbf{a}$ can be uniquely written in the form%
\[
\lambda_{0}\left[  x^{0}\right]  _{\mathbf{a}}+\lambda_{1}\left[
x^{1}\right]  _{\mathbf{a}}+\cdots+\lambda_{n-1}\left[  x^{n-1}\right]
_{\mathbf{a}}\ \ \ \ \ \ \ \ \ \ \text{with }\lambda_{0},\lambda_{1}%
,\ldots,\lambda_{n-1}\in\mathbb{K}.
\]
Thus, in particular, $u$ can be written uniquely in this form (since
$u\in\mathbb{K}\left[  x\right]  /\mathbf{a}$). In other words, there exists a
unique $n$-tuple $\left(  \lambda_{0},\lambda_{1},\ldots,\lambda_{n-1}\right)
\in\mathbb{K}^{n}$ such that $u=\lambda_{0}\left[  x^{0}\right]  _{\mathbf{a}%
}+\lambda_{1}\left[  x^{1}\right]  _{\mathbf{a}}+\cdots+\lambda_{n-1}\left[
x^{n-1}\right]  _{\mathbf{a}}$. Consider this $n$-tuple. For each
$i\in\left\{  0,1,\ldots,n-1\right\}  $, we have $\lambda_{i}\in\mathbb{K}$
and therefore%
\begin{equation}
\left(  \lambda_{i}\right)  ^{p^{mr}}=\lambda_{i}
\label{pf.lem.xpg-x.2.6.pf.coeff-pow}%
\end{equation}
(by Corollary \ref{cor.Frob.Fnidr}, applied to $\mathbb{K}$ and $m$ instead of
$\mathbb{F}$ and $n$), since $\mathbb{K}$ is a finite $\mathbb{F}_{p}$-field
of size $p^{m}$.

But we have%
\begin{equation}
u=\lambda_{0}\left[  x^{0}\right]  _{\mathbf{a}}+\lambda_{1}\left[
x^{1}\right]  _{\mathbf{a}}+\cdots+\lambda_{n-1}\left[  x^{n-1}\right]
_{\mathbf{a}}=\sum_{i=0}^{n-1}\lambda_{i}\left[  x^{i}\right]  _{\mathbf{a}}.
\label{pf.lem.xpg-x.2.6.pf.1}%
\end{equation}
Applying the map $F^{mr}$ to both sides of this equality, we obtain%
\begin{align}
F^{mr}\left(  u\right)   &  =F^{mr}\left(  \sum_{i=0}^{n-1}\lambda_{i}\left[
x^{i}\right]  _{\mathbf{a}}\right)  =\sum_{i=0}^{n-1}\underbrace{F^{mr}\left(
\lambda_{i}\left[  x^{i}\right]  _{\mathbf{a}}\right)  }_{\substack{=\left(
\lambda_{i}\left[  x^{i}\right]  _{\mathbf{a}}\right)  ^{p^{mr}}\\\text{(by
Proposition \ref{prop.Frob.power},}\\\text{applied to }\mathbb{F}\text{,
$\lambda_{i}$}\left[  x^{i}\right]  _{\mathbf{a}}\text{ and }mr\\\text{instead
of }\mathbb{K}\text{, }a\text{ and }i\text{)}}}\nonumber\\
&  \ \ \ \ \ \ \ \ \ \ \left(  \text{since }F^{mr}\text{ is an }\mathbb{F}%
_{p}\text{-module homomorphism}\right) \nonumber\\
&  =\sum_{i=0}^{n-1}\underbrace{\left(  \lambda_{i}\left[  x^{i}\right]
_{\mathbf{a}}\right)  ^{p^{mr}}}_{\substack{=\left(  \lambda_{i}\right)
^{p^{mr}}\left(  \left[  x^{i}\right]  _{\mathbf{a}}\right)  ^{p^{mr}%
}\\\text{(since }\mathbb{F}\text{ is a }\mathbb{K}\text{-algebra)}}%
}=\sum_{i=0}^{n-1}\underbrace{\left(  \lambda_{i}\right)  ^{p^{mr}}%
}_{\substack{=\lambda_{i}\\\text{(by (\ref{pf.lem.xpg-x.2.6.pf.coeff-pow})) }%
}}\left(  \left[  x^{i}\right]  _{\mathbf{a}}\right)  ^{p^{mr}}\nonumber\\
&  =\sum_{i=0}^{n-1}\lambda_{i}\left(  \left[  x^{i}\right]  _{\mathbf{a}%
}\right)  ^{p^{mr}}. \label{pf.lem.xpg-x.2.6.pf.2}%
\end{align}
But for each $i\in\left\{  0,1,\ldots,n-1\right\}  $, we have%
\begin{align}
\left(  \left[  x^{i}\right]  _{\mathbf{a}}\right)  ^{p^{mr}}  &  =\left[
\left(  x^{i}\right)  ^{p^{mr}}\right]  _{\mathbf{a}}%
\ \ \ \ \ \ \ \ \ \ \left(  \text{by (\ref{pf.lem.xpg-x.2.fak}), applied to
}f=x^{i}\text{ and }k=p^{mr}\right) \nonumber\\
&  =\left[  \left(  x^{p^{mr}}\right)  ^{i}\right]  _{\mathbf{a}%
}\ \ \ \ \ \ \ \ \ \ \left(  \text{since }\left(  x^{i}\right)  ^{p^{mr}%
}=x^{ip^{mr}}=x^{p^{mr}i}=\left(  x^{p^{mr}}\right)  ^{i}\right) \nonumber\\
&  =\left(  \left[  x^{p^{mr}}\right]  _{\mathbf{a}}\right)  ^{i}%
\ \ \ \ \ \ \ \ \ \ \left(
\begin{array}
[c]{c}%
\text{since (\ref{pf.lem.xpg-x.2.fak}) (applied to }f=x^{p^{mr}}\text{ and
}k=i\text{)}\\
\text{yields }\left(  \left[  x^{p^{mr}}\right]  _{\mathbf{a}}\right)
^{i}=\left[  \left(  x^{p^{mr}}\right)  ^{i}\right]  _{\mathbf{a}}%
\end{array}
\right) \nonumber\\
&  =\left(  \left[  x\right]  _{\mathbf{a}}\right)  ^{i}%
\ \ \ \ \ \ \ \ \ \ \left(  \text{since }\left[  x^{p^{mr}}\right]
_{\mathbf{a}}=\left[  x\right]  _{\mathbf{a}}\right) \nonumber\\
&  =\left[  x^{i}\right]  _{\mathbf{a}}\ \ \ \ \ \ \ \ \ \ \left(  \text{by
(\ref{pf.lem.xpg-x.2.fak}), applied to }f=x\text{ and }k=i\right)  .
\label{pf.lem.xpg-x.2.6.pf.3}%
\end{align}
Hence, (\ref{pf.lem.xpg-x.2.6.pf.2}) becomes%
\[
F^{mr}\left(  u\right)  =\sum_{i=0}^{n-1}\lambda_{i}\underbrace{\left(
\left[  x^{i}\right]  _{\mathbf{a}}\right)  ^{p^{mr}}}_{\substack{=\left[
x^{i}\right]  _{\mathbf{a}}\\\text{(by (\ref{pf.lem.xpg-x.2.6.pf.3}))}}%
}=\sum_{i=0}^{n-1}\lambda_{i}\left[  x^{i}\right]  _{\mathbf{a}}=u
\]
(by (\ref{pf.lem.xpg-x.2.6.pf.1})). This proves (\ref{pf.lem.xpg-x.2.6}).]

Thus, we have shown that all $u\in\mathbb{F}$ satisfy $F^{mr}\left(  u\right)
=u$. Hence, all $u\in\mathbb{F}$ satisfy $F^{mr}\left(  u\right)
=u=\operatorname*{id}\left(  u\right)  $. In other words, $F^{mr}%
=\operatorname*{id}$.

On the other hand, $\mathbb{F}$ is a finite $\mathbb{F}_{p}$-field of size
$p^{mn}$ (since $\left\vert \mathbb{F}\right\vert =p^{mn}$). Thus, Corollary
\ref{cor.Frob.Fnid} (applied to $mn$ instead of $n$) yields $F^{mn}%
=\operatorname*{id}$.

Now we know that $F^{mr}=\operatorname*{id}$ and $F^{mn}=\operatorname*{id}$.
Hence, Lemma \ref{lem.fgcd} (applied to $\mathbb{F}$, $F$, $mr$ and $mn$
instead of $S$, $f$, $a$ and $b$) yields that $F^{\gcd\left(  mr,mn\right)
}=\operatorname*{id}$.

Let $i=\gcd\left(  mr,mn\right)  $. Thus, $i$ is a positive integer (since
$mr$ and $mn$ are positive integers), so that $p^{i}>1$. Furthermore, from
$i=\gcd\left(  mr,mn\right)  $, we obtain%
\[
F^{i}=F^{\gcd\left(  mr,mn\right)  }=\operatorname*{id}.
\]


Now, for each $a\in\mathbb{F}$, we have%
\[
F^{i}\left(  a\right)  =a^{p^{i}}\ \ \ \ \ \ \ \ \ \ \left(  \text{by
Proposition \ref{prop.Frob.power}, applied to }\mathbb{F}\text{ instead of
}\mathbb{K}\right)
\]
and thus%
\[
a^{p^{i}}=\underbrace{F^{i}}_{=\operatorname*{id}}\left(  a\right)
=\operatorname*{id}\left(  a\right)  =a.
\]
So we have shown that all $a\in\mathbb{F}$ satisfy $a^{p^{i}}=a$. Hence,
Corollary \ref{cor.inverse-field-FLT} (applied to $u=p^{i}$) shows that
$\left\vert \mathbb{F}\right\vert \leq p^{i}$ (since $p^{i}>1$). In view of
$\left\vert \mathbb{F}\right\vert =p^{mn}$, this rewrites as $p^{mn}\leq
p^{i}$. Hence, $mn\leq i$ (since $p>1$). But\footnote{From this place on, all
divisibilities are understood to mean divisibilities in $\mathbb{Z}$.}
$i=\gcd\left(  mr,mn\right)  \mid mn$ and thus $i\leq mn$ (since $i$ and $mn$
both are positive integers). Combining this with $mn\leq i$, we find $mn=i$.
Thus, $mn=i=\gcd\left(  mr,mn\right)  \mid mr$. We can cancel $m$ from this
divisibility (since $m$ is a nonzero integer), and thus find $n\mid r$. In
view of $n=\deg\mathbf{a}$, this rewrites as $\deg\mathbf{a}\mid r$. This
proves Lemma \ref{lem.xpg-x.2}.
\end{proof}

We will say more about the factorization of the polynomial $x^{p^{g}}-x$ (that
is, about the factors $\mathbf{a}_{1},\mathbf{a}_{2},\ldots,\mathbf{a}_{k}$ in
Lemma \ref{lem.xpg-x.1}) in Theorem \ref{thm.xpg-x.full} further below, but
for now let us draw the one consequence of Lemma \ref{lem.xpg-x.2} that we
will actually need for our proof of Theorem \ref{thm.fpnexists2}:

\begin{corollary}
\label{cor.xpg-x.prime}Let $r$ and $m$ be positive integers such that $r$ is
prime. Let $\mathbb{K}$ be a finite $\mathbb{F}_{p}$-field of size $p^{m}$.
Then, there exists a monic irreducible polynomial $\mathbf{a}\in
\mathbb{K}\left[  x\right]  $ of degree $r$.
\end{corollary}

\begin{proof}
[Proof of Corollary \ref{cor.xpg-x.prime}.]Clearly, $mr$ is a positive integer
(since $m$ and $r$ are positive integers). Thus, $p^{mr}>1$. Hence, the
polynomial $x^{p^{mr}}-x$ is a monic polynomial of degree $p^{mr}$. Thus,
$\deg\left(  x^{p^{mr}}-x\right)  =p^{mr}$.

Also, $r$ is prime. Hence, $r>1$. We can multiply this inequality by $m$
(since $m$ is positive), and thus $mr>m\cdot1=m$. Hence, $p^{mr}>p^{m}$.

Lemma \ref{lem.xpg-x.1} (applied to $g=mr$) shows that the polynomial
$x^{p^{mr}}-x\in\mathbb{K}\left[  x\right]  $ can be written in the form
$x^{p^{mr}}-x=\mathbf{u}_{1}\mathbf{u}_{2}\cdots\mathbf{u}_{k}$, where
$\mathbf{u}_{1},\mathbf{u}_{2},\ldots,\mathbf{u}_{k}\in\mathbb{K}\left[
x\right]  $ are \textbf{distinct} monic irreducible polynomials. Consider
these $\mathbf{u}_{1},\mathbf{u}_{2},\ldots,\mathbf{u}_{k}$.

Assume (for the sake of contradiction) that
\begin{equation}
\deg\left(  \mathbf{u}_{i}\right)  =1\ \ \ \ \ \ \ \ \ \ \text{for each }%
i\in\left\{  1,2,\ldots,k\right\}  . \label{pf.cor.xpg-x.prime.ass1}%
\end{equation}


Thus, the polynomials $\mathbf{u}_{1},\mathbf{u}_{2},\ldots,\mathbf{u}_{k}$
all have degree $1$. Hence, $\mathbf{u}_{1},\mathbf{u}_{2},\ldots
,\mathbf{u}_{k}$ are monic polynomials of degree $1$ (since we know that
$\mathbf{u}_{1},\mathbf{u}_{2},\ldots,\mathbf{u}_{k}$ are monic polynomials).
Furthermore, these $k$ monic polynomials of degree $1$ are distinct (as we
know). Thus, we have found $k$ distinct monic polynomials of degree $1$
(namely, $\mathbf{u}_{1},\mathbf{u}_{2},\ldots,\mathbf{u}_{k}$). Hence,%
\[
\left(  \text{the number of all monic polynomials of degree }1\text{ in
}\mathbb{K}\left[  x\right]  \right)  \geq k.
\]


On the other hand, each monic polynomial of degree $1$ (in $\mathbb{K}\left[
x\right]  $) has the form $x+c$ for some unique $c\in\mathbb{K}$. Hence,
\begin{align*}
&  \left(  \text{the number of all monic polynomials of degree }1\text{ in
}\mathbb{K}\left[  x\right]  \right) \\
&  =\left(  \text{the number of all }c\in\mathbb{K}\right)  =\left\vert
\mathbb{K}\right\vert =p^{m}\ \ \ \ \ \ \ \ \ \ \left(  \text{since
}\mathbb{K}\text{ has size }p^{m}\right)  .
\end{align*}
Thus,%
\[
p^{m}=\left(  \text{the number of all monic polynomials of degree }1\text{ in
}\mathbb{K}\left[  x\right]  \right)  \geq k.
\]


Recall that the degree of a product of nonzero polynomials over a field always
equals the sum of their degrees. We can apply this to the polynomials
$\mathbf{u}_{1},\mathbf{u}_{2},\ldots,\mathbf{u}_{k}$ (which are nonzero
because they are irreducible) and thus obtain%
\begin{align*}
\deg\left(  \mathbf{u}_{1}\mathbf{u}_{2}\cdots\mathbf{u}_{k}\right)   &
=\deg\left(  \mathbf{u}_{1}\right)  +\deg\left(  \mathbf{u}_{2}\right)
+\cdots+\deg\left(  \mathbf{u}_{k}\right)  =\sum_{i=1}^{k}\underbrace{\deg
\left(  \mathbf{u}_{i}\right)  }_{\substack{=1\\\text{(by
(\ref{pf.cor.xpg-x.prime.ass1}))}}}\\
&  =\sum_{i=1}^{k}1=k\cdot1=k.
\end{align*}
Hence,%
\[
k=\deg\underbrace{\left(  \mathbf{u}_{1}\mathbf{u}_{2}\cdots\mathbf{u}%
_{k}\right)  }_{\substack{=x^{p^{mr}}-x\\\text{(since }x^{p^{mr}}%
-x=\mathbf{u}_{1}\mathbf{u}_{2}\cdots\mathbf{u}_{k}\text{)}}}=\deg\left(
x^{p^{mr}}-x\right)  =p^{mr}>p^{m}\geq k.
\]
This is clearly absurd. This contradiction shows that our assumption (that is,
(\ref{pf.cor.xpg-x.prime.ass1})) is false. In other words, not every
$i\in\left\{  1,2,\ldots,k\right\}  $ satisfies $\deg\left(  \mathbf{u}%
_{i}\right)  =1$. In other words, there exists some $i\in\left\{
1,2,\ldots,k\right\}  $ such that $\deg\left(  \mathbf{u}_{i}\right)  \neq1$.
Consider this $i$. The polynomial $\mathbf{u}_{i}\in\mathbb{K}\left[
x\right]  $ is irreducible (since $\mathbf{u}_{1},\mathbf{u}_{2}%
,\ldots,\mathbf{u}_{k}$ are all irreducible). Thus, its degree $\deg\left(
\mathbf{u}_{i}\right)  $ is a positive integer.

We have $\mathbf{u}_{i}\mid\mathbf{u}_{1}\mathbf{u}_{2}\cdots\mathbf{u}_{k}$
(since $\mathbf{u}_{i}$ is a factor of the product $\mathbf{u}_{1}%
\mathbf{u}_{2}\cdots\mathbf{u}_{k}$). Hence, $\mathbf{u}_{i}\mid\mathbf{u}%
_{1}\mathbf{u}_{2}\cdots\mathbf{u}_{k}=x^{p^{mr}}-x$. Thus, Lemma
\ref{lem.xpg-x.2} (applied to $g=mr$ and $\mathbf{a}=\mathbf{u}_{i}$) shows
that $\deg\left(  \mathbf{u}_{i}\right)  \mid r$ in $\mathbb{Z}$. Hence,
$\deg\left(  \mathbf{u}_{i}\right)  $ is a divisor of $r$, and thus is a
positive divisor of $r$ (since $\deg\left(  \mathbf{u}_{i}\right)  $ is
positive). But the only positive divisors of $r$ are $1$ and $r$ (since $r$ is
prime). Hence, $\deg\left(  \mathbf{u}_{i}\right)  $ equals either $1$ or $r$
(since $\deg\left(  \mathbf{u}_{i}\right)  $ is a positive divisor of $r$).
Therefore, $\deg\left(  \mathbf{u}_{i}\right)  =r$ (since $\deg\left(
\mathbf{u}_{i}\right)  \neq1$). In other words, the polynomial $\mathbf{u}%
_{i}$ has degree $r$. Thus, there exists a monic irreducible polynomial
$\mathbf{a}\in\mathbb{K}\left[  x\right]  $ of degree $r$ (namely,
$\mathbf{a}=\mathbf{u}_{i}$). This proves Corollary \ref{cor.xpg-x.prime}.
\end{proof}

\section{The proof}

\subsection{The case of a prime exponent}

We are getting close to proving Theorem \ref{thm.fpnexists2}. The following is
one of our last steps:

\begin{lemma}
\label{lem.fpnexists-primestep}Let $r$ be a prime. Let $m$ be a positive
integer. Let $\mathbb{F}$ be a finite $\mathbb{F}_{p}$-field of size $p^{m}$.
Then, there exists a finite $\mathbb{F}_{p}$-field of size $p^{mr}$.
\end{lemma}

\begin{proof}
[Proof of Lemma \ref{lem.fpnexists-primestep}.]We know that $\mathbb{F}$ is an
$\mathbb{F}_{p}$-field. In other words, $\mathbb{F}$ is an $\mathbb{F}_{p}%
$-algebra that is a field. Thus, $\mathbb{F}$ is commutative (since
$\mathbb{F}$ is a field). Also, $\left\vert \mathbb{F}\right\vert =p^{m}$
(because $\mathbb{F}$ has size $p^{m}$).

Corollary \ref{cor.xpg-x.prime} (applied to $\mathbb{K}=\mathbb{F}$) shows
that there exists a monic irreducible polynomial $\mathbf{a}\in\mathbb{F}%
\left[  x\right]  $ of degree $r$. Consider this $\mathbf{a}$. Theorem
\ref{thm.field-ext} \textbf{(c)} (applied to $n=r$) yields that $\mathbb{F}%
\left[  x\right]  /\mathbf{a}$ is a field. Furthermore, Theorem
\ref{thm.field-ext} \textbf{(b)} (applied to $n=r$) yields that
\begin{align*}
\left\vert \mathbb{F}\left[  x\right]  /\mathbf{a}\right\vert  &  =\left\vert
\mathbb{F}\right\vert ^{r}=\left(  p^{m}\right)  ^{r}%
\ \ \ \ \ \ \ \ \ \ \left(  \text{since }\left\vert \mathbb{F}\right\vert
=p^{m}\right) \\
&  =p^{mr}.
\end{align*}
Thus, the field $\mathbb{F}\left[  x\right]  /\mathbf{a}$ is finite.

We know that $\mathbb{F}$ is a commutative $\mathbb{F}_{p}$-algebra. We also
know that $\mathbb{F}\left[  x\right]  $ is an $\mathbb{F}$-algebra. Thus,
Proposition \ref{prop.restrict-alg-isol} (applied to $\mathbb{F}_{p}$,
$\mathbb{F}$ and $\mathbb{F}\left[  x\right]  $ instead of $\mathbb{K}$,
$\mathbb{L}$ and $V$) shows that $\mathbb{F}\left[  x\right]  $ becomes an
$\mathbb{F}_{p}$-algebra in a natural way. Clearly, this $\mathbb{F}\left[
x\right]  $ is commutative (since $\mathbb{F}$ is commutative).

We thus know that $\mathbb{F}\left[  x\right]  $ is a commutative
$\mathbb{F}_{p}$-algebra. We also know that $\mathbb{F}\left[  x\right]
/\mathbf{a}$ is an $\mathbb{F}\left[  x\right]  $-algebra. Thus, Proposition
\ref{prop.restrict-alg-isol} (applied to $\mathbb{F}_{p}$, $\mathbb{F}\left[
x\right]  $ and $\mathbb{F}\left[  x\right]  /\mathbf{a}$ instead of
$\mathbb{K}$, $\mathbb{L}$ and $V$) shows that $\mathbb{F}\left[  x\right]
/\mathbf{a}$ becomes an $\mathbb{F}_{p}$-algebra in a natural way. Thus,
$\mathbb{F}\left[  x\right]  /\mathbf{a}$ is an $\mathbb{F}_{p}$-field (since
$\mathbb{F}\left[  x\right]  /\mathbf{a}$ is an $\mathbb{F}_{p}$-algebra that
is also a field).

Thus we have shown that $\mathbb{F}\left[  x\right]  /\mathbf{a}$ is a finite
$\mathbb{F}_{p}$-field of size $p^{mr}$ (since $\left\vert \mathbb{F}\left[
x\right]  /\mathbf{a}\right\vert =p^{mr}$). Hence, there exists a finite
$\mathbb{F}_{p}$-field of size $p^{mr}$ (namely, $\mathbb{F}\left[  x\right]
/\mathbf{a}$). This proves Lemma \ref{lem.fpnexists-primestep}.
\end{proof}

\subsection{Proof of Theorem \ref{thm.fpnexists2}}

Recall the following basic fact from number theory:

\begin{lemma}
\label{lem.prime-factor-exists}Let $N>1$ be an integer. Then, there exists at
least one prime $r$ such that $r\mid N$.
\end{lemma}

Lemma \ref{lem.prime-factor-exists} is, for example, \cite[Proposition
2.13.8]{19s} (with $n$ and $p$ renamed as $N$ and $r$).

At last, we can prove Theorem \ref{thm.fpnexists2}:

\begin{proof}
[Proof of Theorem \ref{thm.fpnexists2}.]We shall prove Theorem
\ref{thm.fpnexists2} by strong induction on $n$.

\textit{Induction step:} Let $N$ be a positive integer. Assume (as the
induction hypothesis) that Theorem \ref{thm.fpnexists2} holds for all $n<N$.
We must now prove that Theorem \ref{thm.fpnexists2} holds for $n=N$. In other
words, we must prove that there exists a finite $\mathbb{F}_{p}$-field of size
$p^{N}$.

If $N=1$, then this is obvious\footnote{\textit{Proof.} The finite field
$\mathbb{F}_{p}$ is clearly an $\mathbb{F}_{p}$-algebra, and thus is an
$\mathbb{F}_{p}$-field (by the definition of an $\mathbb{F}_{p}$-field).
Moreover, it has size $p^{1}$ (since $\left\vert \mathbb{F}_{p}\right\vert
=p=p^{1}$). Thus, $\mathbb{F}_{p}$ is a finite $\mathbb{F}_{p}$-field of size
$p^{1}$. Hence, there exists a finite $\mathbb{F}_{p}$-field of size $p^{1}$
(namely, $\mathbb{F}_{p}$). Hence, there exists a finite $\mathbb{F}_{p}%
$-field of size $p^{N}$ when $N=1$. Qed.}. Thus, for the rest of this proof,
we WLOG assume that $N\neq1$. Hence, $N>1$ (since $N$ is a positive integer).
Thus, Lemma \ref{lem.prime-factor-exists} shows that there exists at least one
prime $r$ such that $r\mid N$. Consider this $r$. We have $r>1$ (since $r$ is
prime), so that $r>1>0$. Also, there exists an integer $m$ such that $N=rm$
(since $r\mid N$). Consider this $m$. We have $N=rm$, thus $m=N/r>0$ (since
$N>0$ and $r>0$). Hence, we can multiply the inequality $r>1$ by $m$. We thus
find $rm>1m=m$, so that $m<rm=N$. Also, $m$ is a positive integer (since $m$
is an integer and since $m>0$).

Recall that Theorem \ref{thm.fpnexists2} holds for all $n<N$ (by our induction
hypothesis). Hence, Theorem \ref{thm.fpnexists2} holds for $n=m$ (since $m$ is
a positive integer satisfying $m<N$). In other words, there exists a finite
$\mathbb{F}_{p}$-field of size $p^{m}$. Consider such an $\mathbb{F}_{p}%
$-field, and denote it by $\mathbb{F}$. Thus, $\mathbb{F}$ is a finite
$\mathbb{F}_{p}$-field of size $p^{m}$. Hence, Lemma
\ref{lem.fpnexists-primestep} shows that there exists a finite $\mathbb{F}%
_{p}$-field of size $p^{mr}$. In other words, there exists a finite
$\mathbb{F}_{p}$-field of size $p^{N}$ (since $mr=rm=N$). In other words,
Theorem \ref{thm.fpnexists2} holds for $n=N$. This completes the induction
step. Thus, Theorem \ref{thm.fpnexists2} is proven by strong induction.
\end{proof}

\section{Appendices}

\subsection{\label{sect.app-restrict}Appendix 1: Proofs of Proposition
\ref{prop.restrict-mod} and Proposition \ref{prop.restrict-alg}}

We still have to prove two propositions that we used: Proposition
\ref{prop.restrict-mod} and Proposition \ref{prop.restrict-alg}. Both proofs
are straightforward, but require us to recall precisely how modules and
algebras were defined.

First, let us recall the definition of a $\mathbb{K}$-module. Several
equivalent definitions exist; we shall use the one from \cite[Definition
6.3.1]{19s}:

\begin{definition}
\label{def.module.module}Let $\mathbb{K}$ be a commutative ring.

A $\mathbb{K}$\textit{-module} means a set $M$ equipped with

\begin{itemize}
\item a binary operation $+$ on $M$ (called \textquotedblleft\textit{addition}%
\textquotedblright, and not to be confused with the addition $+_{\mathbb{K}}$
of $\mathbb{K}$),

\item a map $\cdot\ :\ \mathbb{K}\times M\rightarrow M$ (called
\textquotedblleft\textit{scaling\textquotedblright}, and not to be confused
with the multiplication $\cdot_{\mathbb{K}}$ of $\mathbb{K}$), and

\item an element $0_{M}\in M$ (called \textquotedblleft\textit{zero
vector}\textquotedblright\ or \textquotedblleft zero\textquotedblright, and
not to be confused with the zero of $\mathbb{K}$)
\end{itemize}

\noindent satisfying the following axioms:

\begin{itemize}
\item \textbf{(a)} We have $a+b=b+a$ for all $a,b\in M$.

\item \textbf{(b)} We have $a+\left(  b+c\right)  =\left(  a+b\right)  +c$ for
all $a,b,c\in M$.

\item \textbf{(c)} We have $a+0_{M}=0_{M}+a=a$ for all $a\in M$.

\item \textbf{(d)} Each $a\in M$ has an additive inverse (i.e., there is an
$a^{\prime}\in M$ such that $a+a^{\prime}=a^{\prime}+a=0_{M}$).

\item \textbf{(e)} We have $\lambda\left(  a+b\right)  =\lambda a+\lambda b$
for all $\lambda\in\mathbb{K}$ and $a,b\in M$. Here and in the following, we
use the notation \textquotedblleft$\lambda c$\textquotedblright\ (or,
equivalently, \textquotedblleft$\lambda\cdot c$\textquotedblright) for the
image of a pair $\left(  \lambda,c\right)  \in\mathbb{K}\times M$ under the
\textquotedblleft scaling\textquotedblright\ map $\cdot$ (similarly to how we
write $ab$ for the image of a pair $\left(  a,b\right)  \in\mathbb{K}%
\times\mathbb{K}$ under the \textquotedblleft multiplication\textquotedblright%
\ map $\cdot$).

\item \textbf{(f)} We have $\left(  \lambda+\mu\right)  a=\lambda a+\mu a$ for
all $\lambda,\mu\in\mathbb{K}$ and $a\in M$.

\item \textbf{(g)} We have $0a=0_{M}$ for all $a\in M$.

\item \textbf{(h)} We have $\left(  \lambda\mu\right)  a=\lambda\left(  \mu
a\right)  $ for all $\lambda,\mu\in\mathbb{K}$ and $a\in M$.

\item \textbf{(i)} We have $1a=a$ for all $a\in M$.

\item \textbf{(j)} We have $\lambda\cdot0_{M}=0_{M}$ for all $\lambda
\in\mathbb{K}$.
\end{itemize}

\noindent These ten axioms are called the \textit{module axioms}.
\end{definition}

Let us also recall the definition of a $\mathbb{K}$-algebra (\cite[Definition
6.9.1]{19s}):

\begin{definition}
\label{def.K-alg.K-alg.short-def}Let $\mathbb{K}$ be a commutative ring. A
$\mathbb{K}$\textit{-algebra} is a set $M$ endowed with two binary operations
$+$ and $\cdot$ as well as a scaling map $\cdot:\mathbb{K}\times M\rightarrow
M$ (not to be confused with the multiplication map, which is also denoted by
$\cdot$) and two elements $0,1\in M$ that satisfy all the ring axioms (with
$\mathbb{K}$ replaced by $M$) as well as all the module axioms (where the zero
vector $0_{M}$ is taken to be the element $0\in M$) and also the following axiom:

\begin{itemize}
\item \textbf{Scale-invariance of multiplication:} We have $\lambda\left(
ab\right)  =\left(  \lambda a\right)  \cdot b=a\cdot\left(  \lambda b\right)
$ for all $\lambda\in\mathbb{K}$ and $a,b\in M$.
\end{itemize}
\end{definition}

We can now prove Proposition \ref{prop.restrict-mod} by a fairly
straightforward verification of the module axioms:

\begin{fineprint}
\begin{proof}
[Proof of Proposition \ref{prop.restrict-mod}.]We have assumed that $V$ is an
$\mathbb{L}$-module. Thus, it satisfies the module axioms. In other words, the
following ten statements hold:\footnote{As usual, we let $0_{V}$ denote the
zero vector of the $\mathbb{L}$-module $V$.}

\begin{itemize}
\item \textbf{(a}$_{1}$\textbf{)} We have $a+b=b+a$ for all $a,b\in V$.

\item \textbf{(b}$_{1}$\textbf{)} We have $a+\left(  b+c\right)  =\left(
a+b\right)  +c$ for all $a,b,c\in V$.

\item \textbf{(c}$_{1}$\textbf{)} We have $a+0_{V}=0_{V}+a=a$ for all $a\in V$.

\item \textbf{(d}$_{1}$\textbf{)} Each $a\in V$ has an additive inverse (i.e.,
there is an $a^{\prime}\in V$ such that $a+a^{\prime}=a^{\prime}+a=0_{V}$).

\item \textbf{(e}$_{1}$\textbf{)} We have $\lambda\left(  a+b\right)  =\lambda
a+\lambda b$ for all $\lambda\in\mathbb{L}$ and $a,b\in V$.

\item \textbf{(f}$_{1}$\textbf{)} We have $\left(  \lambda+\mu\right)
a=\lambda a+\mu a$ for all $\lambda,\mu\in\mathbb{L}$ and $a\in V$.

\item \textbf{(g}$_{1}$\textbf{)} We have $0_{\mathbb{L}}a=0_{V}$ for all
$a\in V$.

\item \textbf{(h}$_{1}$\textbf{)} We have $\left(  \lambda\mu\right)
a=\lambda\left(  \mu a\right)  $ for all $\lambda,\mu\in\mathbb{L}$ and $a\in
V$.

\item \textbf{(i}$_{1}$\textbf{)} We have $1_{\mathbb{L}}a=a$ for all $a\in V$.

\item \textbf{(j}$_{1}$\textbf{)} We have $\lambda\cdot0_{V}=0_{V}$ for all
$\lambda\in\mathbb{L}$.
\end{itemize}

We have also assumed that $\mathbb{L}$ is a $\mathbb{K}$-algebra. Thus,
$\mathbb{L}$ satisfies the ring axioms, the module axioms and the
\textquotedblleft Scale-invariance of multiplication\textquotedblright\ axiom.
In other words, the following 15 statements hold:

\begin{itemize}
\item \textbf{(a}$_{2}$\textbf{)} We have $a+b=b+a$ for all $a,b\in\mathbb{L}$.

\item \textbf{(b}$_{2}$\textbf{)} We have $a+\left(  b+c\right)  =\left(
a+b\right)  +c$ for all $a,b,c\in\mathbb{L}$.

\item \textbf{(c}$_{2}$\textbf{)} We have $a+0_{\mathbb{L}}=0_{\mathbb{L}%
}+a=a$ for all $a\in\mathbb{L}$.

\item \textbf{(d}$_{2}$\textbf{)} Each $a\in\mathbb{L}$ has an additive
inverse (i.e., there is an $a^{\prime}\in\mathbb{L}$ such that $a+a^{\prime
}=a^{\prime}+a=0_{\mathbb{L}}$).

\item \textbf{(e}$_{2}$\textbf{)} We have $\lambda\left(  a+b\right)  =\lambda
a+\lambda b$ for all $\lambda\in\mathbb{K}$ and $a,b\in\mathbb{L}$.

\item \textbf{(f}$_{2}$\textbf{)} We have $\left(  \lambda+\mu\right)
a=\lambda a+\mu a$ for all $\lambda,\mu\in\mathbb{K}$ and $a\in\mathbb{L}$.

\item \textbf{(g}$_{2}$\textbf{)} We have $0_{\mathbb{K}}a=0_{\mathbb{L}}$ for
all $a\in\mathbb{L}$.

\item \textbf{(h}$_{2}$\textbf{)} We have $\left(  \lambda\mu\right)
a=\lambda\left(  \mu a\right)  $ for all $\lambda,\mu\in\mathbb{K}$ and
$a\in\mathbb{L}$.

\item \textbf{(i}$_{2}$\textbf{)} We have $1_{\mathbb{K}}a=a$ for all
$a\in\mathbb{L}$.

\item \textbf{(j}$_{2}$\textbf{)} We have $\lambda\cdot0_{\mathbb{L}%
}=0_{\mathbb{L}}$ for all $\lambda\in\mathbb{L}$.

\item \textbf{(k}$_{2}$\textbf{)} We have $a\left(  bc\right)  =\left(
ab\right)  c$ for all $a,b,c\in\mathbb{L}$.

\item \textbf{(l}$_{2}$\textbf{)} We have $a1_{\mathbb{L}}=1_{\mathbb{L}}a=a$
for all $a\in\mathbb{L}$.

\item \textbf{(m}$_{2}$\textbf{)} We have $a0_{\mathbb{L}}=0_{\mathbb{L}%
}a=0_{\mathbb{L}}$ for all $a\in\mathbb{L}$.

\item \textbf{(n}$_{2}$\textbf{)} We have $a\left(  b+c\right)  =ab+ac$ and
$\left(  a+b\right)  c=ac+bc$ for all $a,b,c\in\mathbb{L}$.

\item \textbf{(o}$_{2}$\textbf{)} We have $\lambda\left(  ab\right)  =\left(
\lambda a\right)  \cdot b=a\cdot\left(  \lambda b\right)  $ for all
$\lambda\in\mathbb{K}$ and $a,b\in\mathbb{L}$.
\end{itemize}

Now, our set $V$ is endowed with an addition $+$, a scaling map $\cdot
:\mathbb{K}\times V\rightarrow V$ (defined by (\ref{eq.prop.restrict-mod.fml}%
)) and a zero vector $0_{V}$. Our goal is to show that $V$ is a $\mathbb{K}%
$-module (when equipped with this addition, this scaling map and this zero
vector). In other words, our goal is to show that it satisfies the module
axioms. In other words, our goal is to show that the following ten statements hold:

\begin{itemize}
\item \textbf{(a}$_{3}$\textbf{)} We have $a+b=b+a$ for all $a,b\in V$.

\item \textbf{(b}$_{3}$\textbf{)} We have $a+\left(  b+c\right)  =\left(
a+b\right)  +c$ for all $a,b,c\in V$.

\item \textbf{(c}$_{3}$\textbf{)} We have $a+0_{V}=0_{V}+a=a$ for all $a\in V$.

\item \textbf{(d}$_{3}$\textbf{)} Each $a\in V$ has an additive inverse (i.e.,
there is an $a^{\prime}\in V$ such that $a+a^{\prime}=a^{\prime}+a=0_{V}$).

\item \textbf{(e}$_{3}$\textbf{)} We have $\lambda\left(  a+b\right)  =\lambda
a+\lambda b$ for all $\lambda\in\mathbb{K}$ and $a,b\in V$.

\item \textbf{(f}$_{3}$\textbf{)} We have $\left(  \lambda+\mu\right)
a=\lambda a+\mu a$ for all $\lambda,\mu\in\mathbb{K}$ and $a\in V$.

\item \textbf{(g}$_{3}$\textbf{)} We have $0_{\mathbb{K}}a=0_{V}$ for all
$a\in V$.

\item \textbf{(h}$_{3}$\textbf{)} We have $\left(  \lambda\mu\right)
a=\lambda\left(  \mu a\right)  $ for all $\lambda,\mu\in\mathbb{K}$ and $a\in
V$.

\item \textbf{(i}$_{3}$\textbf{)} We have $1_{\mathbb{K}}a=a$ for all $a\in V$.

\item \textbf{(j}$_{3}$\textbf{)} We have $\lambda\cdot0_{V}=0_{V}$ for all
$\lambda\in\mathbb{K}$.
\end{itemize}

So it remains to prove these ten statements \textbf{(a}$_{3}$\textbf{)},
\textbf{(b}$_{3}$\textbf{)}, $\ldots$, \textbf{(j}$_{3}$\textbf{)}. Let us do
so now.

The four statements \textbf{(a}$_{3}$\textbf{)}, \textbf{(b}$_{3}$\textbf{)},
\textbf{(c}$_{3}$\textbf{)} and \textbf{(d}$_{3}$\textbf{)} are literally
identical with the four statements \textbf{(a}$_{1}$\textbf{)}, \textbf{(b}%
$_{1}$\textbf{)}, \textbf{(c}$_{1}$\textbf{)} and \textbf{(d}$_{1}$\textbf{)},
and therefore hold (since we know that the latter four statements hold).
Hence, it remains to prove the other six statements.

[\textit{Proof of statement \textbf{(e}}$_{3}$\textit{\textbf{)}:} Let
$\lambda\in\mathbb{K}$ and $a,b\in V$. We must prove that $\lambda\left(
a+b\right)  =\lambda a+\lambda b$.

Applying (\ref{eq.prop.restrict-mod.fml}) to $v=a$, we obtain $\lambda\cdot
a=\left(  \lambda\cdot1_{\mathbb{L}}\right)  \cdot a$. Applying
(\ref{eq.prop.restrict-mod.fml}) to $v=b$, we obtain $\lambda\cdot b=\left(
\lambda\cdot1_{\mathbb{L}}\right)  \cdot b$. Applying
(\ref{eq.prop.restrict-mod.fml}) to $v=a+b$, we obtain
\[
\lambda\cdot\left(  a+b\right)  =\left(  \lambda\cdot1_{\mathbb{L}}\right)
\cdot\left(  a+b\right)  =\left(  \lambda\cdot1_{\mathbb{L}}\right)  \cdot
a+\left(  \lambda\cdot1_{\mathbb{L}}\right)  \cdot b
\]
(by statement \textbf{(e}$_{1}$\textbf{)}, applied to $\lambda\cdot
1_{\mathbb{L}}$ instead of $\lambda$). Comparing this with%
\[
\underbrace{\lambda a}_{\substack{=\lambda\cdot a\\=\left(  \lambda
\cdot1_{\mathbb{L}}\right)  \cdot a}}+\underbrace{\lambda b}%
_{\substack{=\lambda\cdot b\\=\left(  \lambda\cdot1_{\mathbb{L}}\right)  \cdot
b}}=\left(  \lambda\cdot1_{\mathbb{L}}\right)  \cdot a+\left(  \lambda
\cdot1_{\mathbb{L}}\right)  \cdot b,
\]
we obtain $\lambda\cdot\left(  a+b\right)  =\lambda a+\lambda b$. Thus,
$\lambda\left(  a+b\right)  =\lambda\cdot\left(  a+b\right)  =\lambda
a+\lambda b$. This proves statement \textbf{(e}$_{3}$\textbf{)}.]

[\textit{Proof of statement \textbf{(f}}$_{3}$\textit{\textbf{)}:} Let
$\lambda,\mu\in\mathbb{K}$ and $a\in V$. We must prove that $\left(
\lambda+\mu\right)  a=\lambda a+\mu a$.

Applying (\ref{eq.prop.restrict-mod.fml}) to $v=a$, we obtain $\lambda\cdot
a=\left(  \lambda\cdot1_{\mathbb{L}}\right)  \cdot a$. Applying
(\ref{eq.prop.restrict-mod.fml}) to $\mu$ and $a$ instead of $\lambda$ and
$v$, we obtain $\mu\cdot a=\left(  \mu\cdot1_{\mathbb{L}}\right)  \cdot a$.
Applying (\ref{eq.prop.restrict-mod.fml}) to $\lambda+\mu$ and $a$ instead of
$\lambda$ and $v$, we obtain $\left(  \lambda+\mu\right)  \cdot a=\left(
\left(  \lambda+\mu\right)  \cdot1_{\mathbb{L}}\right)  \cdot a$. But
statement \textbf{(f}$_{2}$\textbf{)} (applied to $1_{\mathbb{L}}$ instead of
$a$) yields $\left(  \lambda+\mu\right)  \cdot1_{\mathbb{L}}=\lambda
\cdot1_{\mathbb{L}}+\mu\cdot1_{\mathbb{L}}$. Hence,%
\begin{align*}
\left(  \lambda+\mu\right)  a  &  =\left(  \lambda+\mu\right)  \cdot
a=\underbrace{\left(  \left(  \lambda+\mu\right)  \cdot1_{\mathbb{L}}\right)
}_{=\lambda\cdot1_{\mathbb{L}}+\mu\cdot1_{\mathbb{L}}}\cdot a=\left(
\lambda\cdot1_{\mathbb{L}}+\mu\cdot1_{\mathbb{L}}\right)  \cdot a\\
&  =\left(  \lambda\cdot1_{\mathbb{L}}\right)  \cdot a+\left(  \mu
\cdot1_{\mathbb{L}}\right)  \cdot a
\end{align*}
(by statement \textbf{(f}$_{1}$\textbf{)}, applied to $\lambda\cdot
1_{\mathbb{L}}$ and $\mu\cdot1_{\mathbb{L}}$ instead of $\lambda$ and $\mu$).
Comparing this with%
\[
\underbrace{\lambda a}_{\substack{=\lambda\cdot a\\=\left(  \lambda
\cdot1_{\mathbb{L}}\right)  \cdot a}}+\underbrace{\mu a}_{\substack{=\mu\cdot
a\\=\left(  \mu\cdot1_{\mathbb{L}}\right)  \cdot a}}=\left(  \lambda
\cdot1_{\mathbb{L}}\right)  \cdot a+\left(  \mu\cdot1_{\mathbb{L}}\right)
\cdot a,
\]
we obtain $\left(  \lambda+\mu\right)  a=\lambda a+\mu a$. This proves
statement \textbf{(f}$_{3}$\textbf{)}.]

[\textit{Proof of statement \textbf{(g}}$_{3}$\textit{\textbf{)}:} Let $a\in
V$. We must prove that $0_{\mathbb{K}}a=0_{V}$.

Statement \textbf{(g}$_{2}$\textbf{)} (applied to $1_{\mathbb{L}}$ instead of
$a$) yields $0_{\mathbb{K}}\cdot1_{\mathbb{L}}=0_{\mathbb{L}}$. But
(\ref{eq.prop.restrict-mod.fml}) (applied to $\lambda=0_{\mathbb{K}}$ and
$v=a$) yields $0_{\mathbb{K}}\cdot a=\underbrace{\left(  0_{\mathbb{K}}%
\cdot1_{\mathbb{L}}\right)  }_{=0_{\mathbb{L}}}\cdot a=0_{\mathbb{L}}\cdot
a=0_{V}$ (by statement \textbf{(g}$_{1}$\textbf{)}). This proves statement
\textbf{(g}$_{3}$\textbf{)}.]

[\textit{Proof of statement \textbf{(h}}$_{3}$\textit{\textbf{)}:} Let
$\lambda,\mu\in\mathbb{K}$ and $a\in V$. We must prove that $\left(
\lambda\mu\right)  a=\lambda\left(  \mu a\right)  $.

First, we shall show that
\begin{equation}
\left(  \lambda\mu\right)  \cdot1_{\mathbb{L}}=\left(  \lambda\cdot
1_{\mathbb{L}}\right)  \cdot\left(  \mu\cdot1_{\mathbb{L}}\right)  .
\label{pf.prop.restrict-mod.h.1}%
\end{equation}
Indeed, statement \textbf{(o}$_{2}$\textbf{)} (applied to $\mu$, $\lambda
\cdot1_{\mathbb{L}}$ and $1_{\mathbb{L}}$ instead of $\lambda$, $a$ and $b$)
yields $\mu\left(  \left(  \lambda\cdot1_{\mathbb{L}}\right)  1_{\mathbb{L}%
}\right)  =\left(  \mu\left(  \lambda\cdot1_{\mathbb{L}}\right)  \right)
\cdot1_{\mathbb{L}}=\left(  \lambda\cdot1_{\mathbb{L}}\right)  \cdot\left(
\mu\cdot1_{\mathbb{L}}\right)  $. But statement \textbf{(l}$_{2}$\textbf{)}
(applied to $\lambda\cdot1_{\mathbb{L}}$ instead of $a$) yields $\left(
\lambda\cdot1_{\mathbb{L}}\right)  1_{\mathbb{L}}=1_{\mathbb{L}}\left(
\lambda\cdot1_{\mathbb{L}}\right)  =\lambda\cdot1_{\mathbb{L}}$. Hence,
$\mu\underbrace{\left(  \left(  \lambda\cdot1_{\mathbb{L}}\right)
1_{\mathbb{L}}\right)  }_{=\lambda\cdot1_{\mathbb{L}}}=\mu\left(  \lambda
\cdot1_{\mathbb{L}}\right)  $. Thus,%
\[
\mu\left(  \lambda\cdot1_{\mathbb{L}}\right)  =\mu\left(  \left(  \lambda
\cdot1_{\mathbb{L}}\right)  1_{\mathbb{L}}\right)  =\left(  \lambda
\cdot1_{\mathbb{L}}\right)  \cdot\left(  \mu\cdot1_{\mathbb{L}}\right)  .
\]
But statement \textbf{(h}$_{2}$\textbf{)} (applied to $\mu$, $\lambda$ and
$1_{\mathbb{L}}$ instead of $\lambda$, $\mu$ and $a$) yields $\left(
\mu\lambda\right)  \cdot1_{\mathbb{L}}=\mu\left(  \lambda\cdot1_{\mathbb{L}%
}\right)  =\left(  \lambda\cdot1_{\mathbb{L}}\right)  \cdot\left(  \mu
\cdot1_{\mathbb{L}}\right)  $. However, $\lambda\mu=\mu\lambda$ (since
$\mathbb{L}$ is commutative). Thus,%
\[
\underbrace{\left(  \lambda\mu\right)  }_{=\mu\lambda}\cdot1_{\mathbb{L}%
}=\left(  \mu\lambda\right)  \cdot1_{\mathbb{L}}=\left(  \lambda
\cdot1_{\mathbb{L}}\right)  \cdot\left(  \mu\cdot1_{\mathbb{L}}\right)  .
\]
Thus, (\ref{pf.prop.restrict-mod.h.1}) is proven.

Now, (\ref{eq.prop.restrict-mod.fml}) (applied to $\lambda\mu$ and $a$ instead
of $\lambda$ and $v$) yields
\begin{align}
\left(  \lambda\mu\right)  \cdot a  &  =\underbrace{\left(  \left(  \lambda
\mu\right)  \cdot1_{\mathbb{L}}\right)  }_{\substack{=\left(  \lambda
\cdot1_{\mathbb{L}}\right)  \cdot\left(  \mu\cdot1_{\mathbb{L}}\right)
\\\text{(by (\ref{pf.prop.restrict-mod.h.1}))}}}\cdot a=\left(  \left(
\lambda\cdot1_{\mathbb{L}}\right)  \cdot\left(  \mu\cdot1_{\mathbb{L}}\right)
\right)  \cdot a\nonumber\\
&  =\left(  \lambda\cdot1_{\mathbb{L}}\right)  \cdot\left(  \left(  \mu
\cdot1_{\mathbb{L}}\right)  \cdot a\right)  \label{pf.prop.restrict-mod.h.4}%
\end{align}
(by statement \textbf{(h}$_{1}$\textbf{)}, applied to $\lambda\cdot
1_{\mathbb{L}}$ and $\mu\cdot1_{\mathbb{L}}$ instead of $\lambda$ and $\mu$).

But (\ref{eq.prop.restrict-mod.fml}) (applied to $v=\mu a$) yields%
\[
\lambda\cdot\left(  \mu a\right)  =\left(  \lambda\cdot1_{\mathbb{L}}\right)
\cdot\underbrace{\left(  \mu a\right)  }_{\substack{=\mu\cdot a\\=\left(
\left(  \mu\cdot1_{\mathbb{L}}\right)  \cdot a\right)  \\\text{(by
(\ref{eq.prop.restrict-mod.fml}), applied to }\mu\\\text{instead of }%
\lambda\text{)}}}=\left(  \lambda\cdot1_{\mathbb{L}}\right)  \cdot\left(
\left(  \mu\cdot1_{\mathbb{L}}\right)  \cdot a\right)  .
\]
Comparing this with (\ref{pf.prop.restrict-mod.h.4}), we obtain $\left(
\lambda\mu\right)  \cdot a=\lambda\cdot\left(  \mu a\right)  =\lambda\left(
\mu a\right)  $. Thus, $\left(  \lambda\mu\right)  a=\left(  \lambda
\mu\right)  \cdot a=\lambda\left(  \mu a\right)  $. This proves statement
\textbf{(h}$_{3}$\textbf{)}.]

[\textit{Proof of statement \textbf{(i}}$_{3}$\textit{\textbf{)}:} Let $a\in
V$. We must prove that $1_{\mathbb{K}}a=a$.

Statement \textbf{(i}$_{2}$\textbf{)} (applied to $1_{\mathbb{L}}$ instead of
$a$) yields $1_{\mathbb{K}}\cdot1_{\mathbb{L}}=1_{\mathbb{L}}$. But
(\ref{eq.prop.restrict-mod.fml}) (applied to $\lambda=1_{\mathbb{K}}$) yields
$1_{\mathbb{K}}\cdot a=\underbrace{\left(  1_{\mathbb{K}}\cdot1_{\mathbb{L}%
}\right)  }_{=1_{\mathbb{L}}}\cdot a=1_{\mathbb{L}}\cdot a=a$ (by statement
\textbf{(i}$_{1}$\textbf{)}). This proves statement \textbf{(i}$_{3}%
$\textbf{)}.]

[\textit{Proof of statement \textbf{(j}}$_{3}$\textit{\textbf{)}:} Let
$\lambda\in\mathbb{K}$. We must prove that $\lambda\cdot0_{V}=0_{V}$.

But (\ref{eq.prop.restrict-mod.fml}) (applied to $v=0_{V}$) yields
$\lambda\cdot0_{V}=\left(  \lambda\cdot1_{\mathbb{L}}\right)  \cdot0_{V}%
=0_{V}$ (by statement \textbf{(j}$_{1}$\textbf{)}, applied to $\lambda
\cdot1_{\mathbb{L}}$ instead of $\lambda$). This proves statement
\textbf{(j}$_{3}$\textbf{)}.]

We thus have proven the ten statements \textbf{(a}$_{3}$\textbf{)},
\textbf{(b}$_{3}$\textbf{)}, $\ldots$, \textbf{(j}$_{3}$\textbf{)}. These ten
statements show that $V$ satisfies the module axioms that are required to
ensure that $V$ is a $\mathbb{K}$-module. Hence, $V$ is a $\mathbb{K}$-module.
This proves Proposition \ref{prop.restrict-mod}.
\end{proof}

\begin{proof}
[Proof of Proposition \ref{prop.restrict-alg}.]Proposition
\ref{prop.restrict-mod} shows that the set $V$ (equipped with its addition
$+$, its zero vector $0_{V}$ and the scaling $\cdot:\mathbb{K}\times
V\rightarrow V$ defined by (\ref{eq.prop.restrict-mod.fml})) is a $\mathbb{K}%
$-module. Furthermore, the set $V$ (equipped with its addition $+$, its
multiplication $\cdot$, its zero $0_{V}$ and its unity $1_{V}$) is a ring.

Now, our set $V$ is endowed with an addition $+$, a multiplication $\cdot$, a
scaling map $\cdot:\mathbb{K}\times V\rightarrow V$ (defined by
(\ref{eq.prop.restrict-mod.fml})), a zero $0_{V}$ and a unity $1_{V}$. Our
goal is to show that $V$ is a $\mathbb{K}$-algebra (when equipped with this
addition, this multiplication, this scaling map, this zero and this unity). In
other words, our goal is to show that it satisfies the ring axioms, the module
axioms and the \textquotedblleft Scale-invariance of
multiplication\textquotedblright\ axiom (because this is precisely what is
needed to ensure that $V$ is a $\mathbb{K}$-algebra, according to Definition
\ref{def.K-alg.K-alg.short-def}). But it clearly satisfies the ring axioms
(since $V$ is a ring) and the module axioms (since $V$ is a $\mathbb{K}%
$-module). Hence, it suffices to prove that it satisfies the \textquotedblleft
Scale-invariance of multiplication\textquotedblright\ axiom. In other words,
we must prove the following statement:

\begin{itemize}
\item \textbf{(k}$_{3}$\textbf{)} We have $\lambda\left(  ab\right)  =\left(
\lambda a\right)  \cdot b=a\cdot\left(  \lambda b\right)  $ for all
$\lambda\in\mathbb{K}$ and $a,b\in V$.
\end{itemize}

Before we prove this statement, let us recall something: We know that $V$ is
an $\mathbb{L}$-algebra, and therefore satisfies the \textquotedblleft
Scale-invariance of multiplication\textquotedblright\ axiom. In other words,
the following statement holds:

\begin{itemize}
\item \textbf{(k}$_{1}$\textbf{)} We have $\lambda\left(  ab\right)  =\left(
\lambda a\right)  \cdot b=a\cdot\left(  \lambda b\right)  $ for all
$\lambda\in\mathbb{L}$ and $a,b\in V$.
\end{itemize}

Now, we can prove the statement \textbf{(k}$_{3}$\textbf{)}:

[\textit{Proof of statement \textbf{(k}}$_{3}$\textit{\textbf{)}:} Let
$\lambda\in\mathbb{K}$ and $a,b\in V$. We must prove that $\lambda\left(
ab\right)  =\left(  \lambda a\right)  \cdot b=a\cdot\left(  \lambda b\right)
$.

We have $\lambda\cdot1_{\mathbb{L}}\in\mathbb{L}$. Thus, statement
\textbf{(k}$_{1}$\textbf{)} (applied to $\lambda\cdot1_{\mathbb{L}}$ instead
of $\lambda$) yields $\left(  \lambda\cdot1_{\mathbb{L}}\right)  \left(
ab\right)  =\left(  \left(  \lambda\cdot1_{\mathbb{L}}\right)  a\right)  \cdot
b=a\cdot\left(  \left(  \lambda\cdot1_{\mathbb{L}}\right)  b\right)  $.

Applying (\ref{eq.prop.restrict-mod.fml}) to $v=a$, we obtain $\lambda\cdot
a=\left(  \lambda\cdot1_{\mathbb{L}}\right)  \cdot a=\left(  \lambda
\cdot1_{\mathbb{L}}\right)  a$. Applying (\ref{eq.prop.restrict-mod.fml}) to
$v=b$, we obtain $\lambda\cdot b=\left(  \lambda\cdot1_{\mathbb{L}}\right)
\cdot b=\left(  \lambda\cdot1_{\mathbb{L}}\right)  b$. Applying
(\ref{eq.prop.restrict-mod.fml}) to $v=ab$, we obtain
\[
\lambda\cdot\left(  ab\right)  =\left(  \lambda\cdot1_{\mathbb{L}}\right)
\cdot\left(  ab\right)  =\underbrace{\left(  \left(  \lambda\cdot
1_{\mathbb{L}}\right)  a\right)  }_{\substack{=\lambda a\\\text{(since
}\lambda a=\lambda\cdot a=\left(  \lambda\cdot1_{\mathbb{L}}\right)
a\text{)}}}\cdot b=\left(  \lambda a\right)  \cdot b.
\]
Also,%
\[
\lambda\cdot\left(  ab\right)  =\left(  \lambda\cdot1_{\mathbb{L}}\right)
\cdot\left(  ab\right)  =a\cdot\underbrace{\left(  \left(  \lambda
\cdot1_{\mathbb{L}}\right)  b\right)  }_{\substack{=\lambda b\\\text{(since
}\lambda b=\lambda\cdot b=\left(  \lambda\cdot1_{\mathbb{L}}\right)
b\text{)}}}=a\cdot\left(  \lambda b\right)  .
\]
Combining these two equalities, we find $\lambda\cdot\left(  ab\right)
=\left(  \lambda a\right)  \cdot b=a\cdot\left(  \lambda b\right)  $. In other
words, $\lambda\left(  ab\right)  =\left(  \lambda a\right)  \cdot
b=a\cdot\left(  \lambda b\right)  $. This proves statement \textbf{(k}$_{3}%
$\textbf{)}.]

So we have proven that statement \textbf{(k}$_{3}$\textbf{)} holds. In other
words, $V$ satisfies the \textquotedblleft Scale-invariance of
multiplication\textquotedblright\ axiom. Thus, altogether, we have shown that
$V$ satisfies all the ring axioms as well as all the module axioms and also
the \textquotedblleft Scale-invariance of multiplication\textquotedblright%
\ axiom. Thus, $V$ is a $\mathbb{K}$-algebra (by Definition
\ref{def.K-alg.K-alg.short-def}). This proves Proposition
\ref{prop.restrict-alg}.
\end{proof}
\end{fineprint}

\subsection{Appendix 2: Factoring $x^{p^{g}}-x$, part II}

If $r$, $m$ and $\mathbb{K}$ are as in Lemma \ref{lem.xpg-x.2}, then every
irreducible divisor of the polynomial in $x^{p^{mr}}-x$ in $\mathbb{K}\left[
x\right]  $ satisfies $\deg\mathbf{a}\mid r$ in $\mathbb{Z}$; this is what
Lemma \ref{lem.xpg-x.2} stated. But a converse also holds:

\begin{lemma}
\label{lem.xpg-x.4}Let $r$ and $m$ be positive integers. Let $\mathbb{K}$ be a
finite $\mathbb{F}_{p}$-field of size $p^{m}$. Let $\mathbf{a}\in
\mathbb{K}\left[  x\right]  $ be an irreducible polynomial such that
$\deg\mathbf{a}\mid r$ in $\mathbb{Z}$. Then, $\mathbf{a}\mid x^{p^{mr}}-x$ in
$\mathbb{K}\left[  x\right]  $.
\end{lemma}

\begin{proof}
[Proof of Lemma \ref{lem.xpg-x.4}.]Let $n=\deg\mathbf{a}$. Thus, $\mathbf{a}$
is a polynomial of degree $n$. Theorem \ref{thm.field-ext} \textbf{(c)}
(applied to $\mathbb{F}=\mathbb{K}$) yields that $\mathbb{K}\left[  x\right]
/\mathbf{a}$ is a field. Let $\mathbb{F}$ denote this field.

We have $n=\deg\mathbf{a}\mid r$ in $\mathbb{Z}$. Hence, there exists an
integer $u$ such that $r=nu$. Consider this $u$. Thus, $nu=r$. Also, it is
easy to see that $u\in\mathbb{N}$\ \ \ \ \footnote{\textit{Proof.} We have
$n=\deg\mathbf{a}>0$ (since $\mathbf{a}$ is irreducible). But we have $nu=r>0$
(since $r$ is positive). We can divide this inequality by $n$ (since $n>0$),
and thus find $u>0$. Hence, $u\in\mathbb{N}$ (since $u$ is an integer).}.

The following observations can be proven exactly as they were proven in our
proof of Lemma \ref{lem.xpg-x.2} above:

\begin{itemize}
\item The number $mn$ is a positive integer.

\item The ring $\mathbb{K}$ is a field and is commutative.

\item We have
\begin{equation}
\left(  \left[  f\right]  _{\mathbf{a}}\right)  ^{k}=\left[  f^{k}\right]
_{\mathbf{a}}\ \ \ \ \ \ \ \ \ \ \text{for each }f\in\mathbb{K}\left[
x\right]  \text{ and each }k\in\mathbb{N}. \label{pf.lem.xpg-x.4.fak}%
\end{equation}
(Recall Convention \ref{conv.quotient} in order to make sense of this.)

\item The ring $\mathbb{F}$ is an $\mathbb{F}_{p}$-field and has size $p^{mn}$.
\end{itemize}

Thus, $\mathbb{F}$ is finite. Now, $\left[  x\right]  _{\mathbf{a}}%
\in\mathbb{K}\left[  x\right]  /\mathbf{a}=\mathbb{F}$. Hence, Corollary
\ref{cor.Frob.Fnidr} (applied to $mn$, $\left[  x\right]  _{\mathbf{a}}$ and
$u$ instead of $n$, $a$ and $r$) yields $\left(  \left[  x\right]
_{\mathbf{a}}\right)  ^{p^{mnu}}=\left[  x\right]  _{\mathbf{a}}$. In view of
$m\underbrace{nu}_{=r}=mr$, this rewrites as $\left(  \left[  x\right]
_{\mathbf{a}}\right)  ^{p^{mr}}=\left[  x\right]  _{\mathbf{a}}$. Comparing
this with%
\[
\left(  \left[  x\right]  _{\mathbf{a}}\right)  ^{p^{mr}}=\left[  x^{p^{mr}%
}\right]  _{\mathbf{a}}\ \ \ \ \ \ \ \ \ \ \left(  \text{by
(\ref{pf.lem.xpg-x.4.fak}), applied to }f=x\text{ and }k=p^{mr}\right)  ,
\]
we obtain $\left[  x^{p^{mr}}\right]  _{\mathbf{a}}=\left[  x\right]
_{\mathbf{a}}$. In other words, $x^{p^{mr}}\equiv x\operatorname{mod}%
\mathbf{a}$ in $\mathbb{K}\left[  x\right]  $. In other words, $\mathbf{a}\mid
x^{p^{mr}}-x$ in $\mathbb{K}\left[  x\right]  $. This proves Lemma
\ref{lem.xpg-x.4}.
\end{proof}

Lemma \ref{lem.xpg-x.2} with Lemma \ref{lem.xpg-x.4} can be merged into a
single theorem, which gives an \textquotedblleft explicit\textquotedblright%
\ factorization of $x^{p^{mr}}-x$ into irreducible
polynomials\footnote{\textquotedblleft Explicit\textquotedblright\ only in the
sense that the irreducible polynomials of any given degree over $\mathbb{K}$
can be found.}:

\begin{theorem}
\label{thm.xpg-x.full}Let $r$ and $m$ be positive integers. Let $\mathbb{K}$
be a finite $\mathbb{F}_{p}$-field of size $p^{m}$. Then,%
\[
x^{p^{mr}}-x=\prod_{\substack{\mathbf{a}\in\mathbb{K}\left[  x\right]  \text{
is}\\\text{irreducible}\\\text{and monic;}\\\deg\mathbf{a}\mid r}}\mathbf{a}.
\]

\end{theorem}

Our proof of Theorem \ref{thm.xpg-x.full} relies on the following fact about
irreducible polynomials:

\begin{proposition}
\label{prop.irredpol.pabk}Let $\mathbb{K}$ be a field. Let $\mathbf{p}%
\in\mathbb{K}\left[  x\right]  $ be an irreducible polynomial. Let
$\mathbf{a}_{1},\mathbf{a}_{2},\ldots,\mathbf{a}_{k}\in\mathbb{K}\left[
x\right]  $ be polynomials such that $\mathbf{p}\mid\mathbf{a}_{1}%
\mathbf{a}_{2}\mathbf{\cdots a}_{k}$. Then, $\mathbf{p}\mid\mathbf{a}_{i}$ for
some $i\in\left\{  1,2,\ldots,k\right\}  $.
\end{proposition}

\begin{proof}
[Proof of Proposition \ref{prop.irredpol.pabk}.]Proposition
\ref{prop.irredpol.pabk} is the analogue of \cite[Proposition 2.13.7]{19s} for
polynomials (in $\mathbb{K}\left[  x\right]  $) instead of integers. It can be
proven in the same way as the latter result was proven (but with the usual
changes that are required to turn an argument about integers into the
analogous argument about polynomials).

\begin{fineprint}
Alternatively, we can prove Proposition \ref{prop.irredpol.pabk} using Theorem
\ref{thm.field-ext} as follows: Assume the contrary. Thus, we don't have
$\left(  \mathbf{p}\mid\mathbf{a}_{i}\text{ for some }i\in\left\{
1,2,\ldots,k\right\}  \right)  $. In other words, we have%
\begin{equation}
\mathbf{p}\nmid\mathbf{a}_{i}\ \ \ \ \ \ \ \ \ \ \text{for each }i\in\left\{
1,2,\ldots,k\right\}  . \label{pf.prop.irredpol.pabk.atc}%
\end{equation}


The polynomial $\mathbf{p}$ is irreducible and thus non-constant. Hence,
$\deg\mathbf{p}>0$.

Let $n=\deg\mathbf{p}$. Thus, $\mathbf{p}$ is a polynomial of degree $n$.
Theorem \ref{thm.field-ext} \textbf{(c)} (applied to $\mathbb{F}=\mathbb{K}$
and $\mathbf{a}=\mathbf{p}$) shows that $\mathbb{K}\left[  x\right]
/\mathbf{p}$ is a field. Recall Convention \ref{conv.quotient}. The definition
of the multiplication on $\mathbb{K}\left[  x\right]  /\mathbf{p}$ shows that
$\left[  \mathbf{u}\right]  _{\mathbf{p}}\left[  \mathbf{v}\right]
_{\mathbf{p}}=\left[  \mathbf{uv}\right]  _{\mathbf{p}}$ for any
$\mathbf{u},\mathbf{v}\in\mathbb{K}\left[  x\right]  $. Hence, a
straightforward induction on $j$ shows that%
\[
\left[  \mathbf{u}_{1}\right]  _{\mathbf{p}}\left[  \mathbf{u}_{2}\right]
_{\mathbf{p}}\cdots\left[  \mathbf{u}_{j}\right]  _{\mathbf{p}}=\left[
\mathbf{u}_{1}\mathbf{u}_{2}\cdots\mathbf{u}_{j}\right]  _{\mathbf{p}}%
\]
for every $j\in\mathbb{N}$ and every $\mathbf{u}_{1},\mathbf{u}_{2}%
,\ldots,\mathbf{u}_{j}\in\mathbb{K}\left[  x\right]  $. Applying this to $j=k$
and $\mathbf{u}_{i}=\mathbf{a}_{i}$, we conclude that
\[
\left[  \mathbf{a}_{1}\right]  _{\mathbf{p}}\left[  \mathbf{a}_{2}\right]
_{\mathbf{p}}\cdots\left[  \mathbf{a}_{k}\right]  _{\mathbf{p}}=\left[
\mathbf{a}_{1}\mathbf{a}_{2}\cdots\mathbf{a}_{k}\right]  _{\mathbf{p}}=\left[
0\right]  _{\mathbf{p}}%
\]
(since $\mathbf{a}_{1}\mathbf{a}_{2}\cdots\mathbf{a}_{k}\equiv
0\operatorname{mod}\mathbf{p}$ (because $\mathbf{p}\mid\mathbf{a}%
_{1}\mathbf{a}_{2}\cdots\mathbf{a}_{k}$)).

But $\mathbb{K}\left[  x\right]  /\mathbf{p}$ is a field, and thus is a
commutative skew field. Hence, every nonzero element of $\mathbb{K}\left[
x\right]  /\mathbf{p}$ is invertible (since $\mathbb{K}\left[  x\right]
/\mathbf{p}$ is a skew field).

Now, let $i\in\left\{  1,2,\ldots,k\right\}  $. Then, $\mathbf{p}%
\nmid\mathbf{a}_{i}$ (by (\ref{pf.prop.irredpol.pabk.atc})). In other words,
$\mathbf{a}_{i}\not \equiv 0\operatorname{mod}\mathbf{p}$. In other words,
$\left[  \mathbf{a}_{i}\right]  _{\mathbf{p}}\neq\left[  0\right]
_{\mathbf{p}}$. The element $\left[  \mathbf{a}_{i}\right]  _{\mathbf{p}}$ of
$\mathbb{K}\left[  x\right]  /\mathbf{p}$ is nonzero (since $\left[
\mathbf{a}_{i}\right]  _{\mathbf{p}}\neq\left[  0\right]  _{\mathbf{p}%
}=0_{\mathbb{K}\left[  x\right]  /\mathbf{p}}$) and thus invertible (since
every nonzero element of $\mathbb{K}\left[  x\right]  /\mathbf{p}$ is
invertible). In other words, there exists a multiplicative inverse of $\left[
\mathbf{a}_{i}\right]  _{\mathbf{p}}$ in $\mathbb{K}\left[  x\right]
/\mathbf{p}$. In other words, there exists some $\beta_{i}\in\mathbb{K}\left[
x\right]  /\mathbf{p}$ such that $\left[  \mathbf{a}_{i}\right]  _{\mathbf{p}%
}\beta_{i}=\beta_{i}\left[  \mathbf{a}_{i}\right]  _{\mathbf{p}}%
=1_{\mathbb{K}\left[  x\right]  /\mathbf{p}}$. Consider this $\beta_{i}$.

Forget that we fixed $i$. Thus, for each $i\in\left\{  1,2,\ldots,k\right\}
$, we have constructed some $\beta_{i}\in\mathbb{K}\left[  x\right]
/\mathbf{p}$ such that
\begin{equation}
\left[  \mathbf{a}_{i}\right]  _{\mathbf{p}}\beta_{i}=\beta_{i}\left[
\mathbf{a}_{i}\right]  _{\mathbf{p}}=1_{\mathbb{K}\left[  x\right]
/\mathbf{p}}. \label{pf.prop.irredpol.pabk.ab}%
\end{equation}
But the ring $\mathbb{K}\left[  x\right]  /\mathbf{p}$ is commutative (since
$\mathbb{K}\left[  x\right]  $ is commutative). Thus,%
\[
\prod_{i=1}^{k}\left(  \left[  \mathbf{a}_{i}\right]  _{\mathbf{p}}\beta
_{i}\right)  =\underbrace{\left(  \prod_{i=1}^{k}\left[  \mathbf{a}%
_{i}\right]  _{\mathbf{p}}\right)  }_{\substack{=\left[  \mathbf{a}%
_{1}\right]  _{\mathbf{p}}\left[  \mathbf{a}_{2}\right]  _{\mathbf{p}}%
\cdots\left[  \mathbf{a}_{k}\right]  _{\mathbf{p}}\\=\left[  0\right]
_{\mathbf{p}}=0_{\mathbb{K}\left[  x\right]  /\mathbf{p}}}}\left(  \prod
_{i=1}^{k}\beta_{i}\right)  =0_{\mathbb{K}\left[  x\right]  /\mathbf{p}%
}\left(  \prod_{i=1}^{k}\beta_{i}\right)  =0_{\mathbb{K}\left[  x\right]
/\mathbf{p}}=\left[  0\right]  _{\mathbf{p}}.
\]
Hence,%
\[
\left[  0\right]  _{\mathbf{p}}=\prod_{i=1}^{k}\underbrace{\left(  \left[
\mathbf{a}_{i}\right]  _{\mathbf{p}}\beta_{i}\right)  }%
_{\substack{=1_{\mathbb{K}\left[  x\right]  /\mathbf{p}}\\\text{(by
(\ref{pf.prop.irredpol.pabk.ab}))}}}=\prod_{i=1}^{k}1_{\mathbb{K}\left[
x\right]  /\mathbf{p}}=1_{\mathbb{K}\left[  x\right]  /\mathbf{p}}=\left[
1\right]  _{\mathbf{p}}.
\]
Thus, $\left[  1\right]  _{\mathbf{p}}=\left[  0\right]  _{\mathbf{p}}$. In
other words, $1\equiv0\operatorname{mod}\mathbf{p}$. In other words,
$\mathbf{p}\mid1$ in $\mathbb{K}\left[  x\right]  $. Hence, there exists some
$\mathbf{c}\in\mathbb{K}\left[  x\right]  $ such that $1=\mathbf{pc}$.
Consider this $\mathbf{c}$. We have $\mathbf{c}\neq0$ (since $\mathbf{pc}%
=1\neq0$) and thus $\deg\mathbf{c}\geq0$. But $\deg1=0$, so that
$0=\deg\underbrace{1}_{=\mathbf{pc}}=\deg\left(  \mathbf{pc}\right)
=\underbrace{\deg\mathbf{p}}_{>0}+\underbrace{\deg\mathbf{c}}_{\geq0}>0$. This
is absurd. This contradiction shows that our assumption was false. Hence,
Proposition \ref{prop.irredpol.pabk} is proven.
\end{fineprint}
\end{proof}

\begin{corollary}
\label{cor.irredpol.pabk2}Let $\mathbb{K}$ be a field. Let $\mathbf{p}%
\in\mathbb{K}\left[  x\right]  $ be a monic irreducible polynomial. Let
$\mathbf{a}_{1},\mathbf{a}_{2},\ldots,\mathbf{a}_{k}\in\mathbb{K}\left[
x\right]  $ be monic irreducible polynomials such that $\mathbf{p}%
\mid\mathbf{a}_{1}\mathbf{a}_{2}\mathbf{\cdots a}_{k}$. Then, $\mathbf{p}%
=\mathbf{a}_{i}$ for some $i\in\left\{  1,2,\ldots,k\right\}  $.
\end{corollary}

\begin{proof}
[Proof of Corollary \ref{cor.irredpol.pabk2}.]Proposition
\ref{prop.irredpol.pabk} shows that $\mathbf{p}\mid\mathbf{a}_{i}$ for some
$i\in\left\{  1,2,\ldots,k\right\}  $. Consider this $i$, and denote it by
$j$. Thus, $j$ is an element of $\left\{  1,2,\ldots,k\right\}  $ and
satisfies $\mathbf{p}\mid\mathbf{a}_{j}$. From $\mathbf{p}\mid\mathbf{a}_{j}$,
we conclude that there exists a polynomial $\mathbf{u}\in\mathbb{K}\left[
x\right]  $ such that $\mathbf{a}_{j}=\mathbf{pu}$. Consider this $\mathbf{u}$.

The polynomial $\mathbf{p}$ is irreducible and thus non-constant. In other
words, $\deg\mathbf{p}>0$.

The polynomial $\mathbf{a}_{j}$ is monic (since all $k$ polynomials
$\mathbf{a}_{1},\mathbf{a}_{2},\ldots,\mathbf{a}_{k}$ are monic).

The polynomial $\mathbf{a}_{j}$ is irreducible (since all $k$ polynomials
$\mathbf{a}_{1},\mathbf{a}_{2},\ldots,\mathbf{a}_{k}$ are irreducible). In
other words, $\deg\left(  \mathbf{a}_{j}\right)  >0$ and there exist no two
polynomials $\mathbf{b},\mathbf{c}\in\mathbb{K}\left[  x\right]  $ with
$\mathbf{a}_{j}=\mathbf{bc}$ and $\deg\mathbf{b}>0$ and $\deg\mathbf{c}>0$
(because this is how \textquotedblleft irreducible\textquotedblright\ is defined).

Assume (for the sake of contradiction) that $\deg\mathbf{u}>0$. Then, the
polynomials $\mathbf{p},\mathbf{u}\in\mathbb{K}\left[  x\right]  $ satisfy
$\mathbf{a}_{j}=\mathbf{pu}$ and $\deg\mathbf{p}>0$ and $\deg\mathbf{u}>0$.
Hence, there exist two polynomials $\mathbf{b},\mathbf{c}\in\mathbb{K}\left[
x\right]  $ with $\mathbf{a}_{j}=\mathbf{bc}$ and $\deg\mathbf{b}>0$ and
$\deg\mathbf{c}>0$ (namely, $\mathbf{b}=\mathbf{p}$ and $\mathbf{c}%
=\mathbf{u}$). This contradicts the fact that there exist no two polynomials
$\mathbf{b},\mathbf{c}\in\mathbb{K}\left[  x\right]  $ with $\mathbf{a}%
_{j}=\mathbf{bc}$ and $\deg\mathbf{b}>0$ and $\deg\mathbf{c}>0$.

This contradiction shows that our assumption (that $\deg\mathbf{u}>0$) was
false. Hence, $\deg\mathbf{u}\leq0$. Thus, the polynomial $\mathbf{u}$ is
constant. In other words, $\mathbf{u}=\lambda$ for some $\lambda\in\mathbb{K}%
$. Consider this $\lambda$. Now, $\mathbf{a}_{j}=\mathbf{p}%
\underbrace{\mathbf{u}}_{=\lambda}=\mathbf{p}\lambda=\lambda\mathbf{p}$, so
that $\lambda\mathbf{p}=\mathbf{a}_{j}\neq0$ (because $\mathbf{a}_{j}$ is
irreducible) and thus $\lambda\neq0$.

The leading coefficient of the polynomial $\mathbf{p}$ is $1$ (since
$\mathbf{p}$ is monic). Hence, the leading coefficient of the polynomial
$\lambda\mathbf{p}$ is $\lambda\cdot1=\lambda$. In other words, the leading
coefficient of the polynomial $\mathbf{a}_{j}$ is $\lambda$ (since
$\mathbf{a}_{j}=\lambda\mathbf{p}$). Thus,%
\[
\lambda=\left(  \text{the leading term of the polynomial }\mathbf{a}%
_{j}\right)  =1
\]
(since the polynomial $\mathbf{a}_{j}$ is monic). Hence, $\mathbf{a}%
_{j}=\underbrace{\lambda}_{=1}\mathbf{p}=\mathbf{p}$, so that $\mathbf{p}%
=\mathbf{a}_{j}$. Thus, $\mathbf{p}=\mathbf{a}_{i}$ for some $i\in\left\{
1,2,\ldots,k\right\}  $ (namely, for $i=j$). This proves Corollary
\ref{cor.irredpol.pabk2}.
\end{proof}

\begin{proof}
[Proof of Theorem \ref{thm.xpg-x.full}.]Recall that $m$ and $r$ are positive
integers. Hence, their product $mr$ is a positive integer. Thus, Lemma
\ref{lem.xpg-x.1} (applied to $g=mr$) shows that the polynomial $x^{p^{mr}%
}-x\in\mathbb{K}\left[  x\right]  $ can be written in the form $x^{p^{mr}%
}-x=\mathbf{u}_{1}\mathbf{u}_{2}\cdots\mathbf{u}_{k}$, where $\mathbf{u}%
_{1},\mathbf{u}_{2},\ldots,\mathbf{u}_{k}\in\mathbb{K}\left[  x\right]  $ are
\textbf{distinct} monic irreducible polynomials. Consider these $\mathbf{u}%
_{1},\mathbf{u}_{2},\ldots,\mathbf{u}_{k}$.

Let $D$ be the set of all monic irreducible polynomials $\mathbf{a}%
\in\mathbb{K}\left[  x\right]  $ that satisfy $\deg\mathbf{a}\mid r$. For each
$i\in\left\{  1,2,\ldots,k\right\}  $, we have%
\begin{equation}
\mathbf{u}_{i}\in D. \label{pf.thm.xpg-x.full.uiiD}%
\end{equation}


[\textit{Proof of (\ref{pf.thm.xpg-x.full.uiiD}):} Let $i\in\left\{
1,2,\ldots,k\right\}  $. Then, $\mathbf{u}_{i}$ is a monic irreducible
polynomial (since $\mathbf{u}_{1},\mathbf{u}_{2},\ldots,\mathbf{u}_{k}$ are
monic irreducible polynomials). Moreover, $\mathbf{u}_{i}\mid\mathbf{u}%
_{1}\mathbf{u}_{2}\cdots\mathbf{u}_{k}$ (since $\mathbf{u}_{i}$ is a factor in
the product $\mathbf{u}_{1}\mathbf{u}_{2}\cdots\mathbf{u}_{k}$). This rewrites
as $\mathbf{u}_{i}\mid x^{p^{mr}}-x$ (since $x^{p^{mr}}-x=\mathbf{u}%
_{1}\mathbf{u}_{2}\cdots\mathbf{u}_{k}$). Hence, Lemma \ref{lem.xpg-x.2}
(applied to $\mathbf{a}=\mathbf{u}_{i}$) shows that $\deg\left(
\mathbf{u}_{i}\right)  \mid r$ in $\mathbb{Z}$. Thus, $\mathbf{u}_{i}$ is a
monic irreducible polynomial in $\mathbb{K}\left[  x\right]  $ that satisfies
$\deg\left(  \mathbf{u}_{i}\right)  \mid r$ in $\mathbb{Z}$. In other words,
$\mathbf{u}_{i}$ is a monic irreducible polynomial $\mathbf{a}\in
\mathbb{K}\left[  x\right]  $ that satisfies $\deg\mathbf{a}\mid r$. In other
words, $\mathbf{u}_{i}\in D$ (since $D$ is the set of all monic irreducible
polynomials $\mathbf{a}\in\mathbb{K}\left[  x\right]  $ that satisfy
$\deg\mathbf{a}\mid r$). This proves (\ref{pf.thm.xpg-x.full.uiiD}).]

Thus, we have shown that we have $\mathbf{u}_{i}\in D$ for each $i\in\left\{
1,2,\ldots,k\right\}  $. Hence, the map%
\begin{align*}
\left\{  1,2,\ldots,k\right\}   &  \rightarrow D,\\
i  &  \mapsto\mathbf{u}_{i}%
\end{align*}
is well-defined. Denote this map by $\alpha$. This map $\alpha$ is
injective\footnote{\textit{Proof.} Let $i$ and $j$ be two elements of
$\left\{  1,2,\ldots,k\right\}  $ such that $\alpha\left(  i\right)
=\alpha\left(  j\right)  $. We shall prove that $i=j$.
\par
We have $\alpha\left(  j\right)  =\mathbf{u}_{j}$ (by the definition of
$\alpha$) and $\alpha\left(  i\right)  =\mathbf{u}_{i}$ (similarly). Thus,
$\mathbf{u}_{i}=\alpha\left(  i\right)  =\alpha\left(  j\right)
=\mathbf{u}_{j}$. Hence, $i=j$ (since the polynomials $\mathbf{u}%
_{1},\mathbf{u}_{2},\ldots,\mathbf{u}_{k}$ are distinct).
\par
Now, forget that we fixed $i$ and $j$. We thus have shown that if $i$ and $j$
are two elements of $\left\{  1,2,\ldots,k\right\}  $ such that $\alpha\left(
i\right)  =\alpha\left(  j\right)  $, then $i=j$. In other words, the map
$\alpha$ is injective.} and surjective\footnote{\textit{Proof.} Let
$\mathbf{d}\in D$. Thus, $\mathbf{d}$ is a monic irreducible polynomial
$\mathbf{a}\in\mathbb{K}\left[  x\right]  $ that satisfies $\deg\mathbf{a}\mid
r$ (since $D$ is the set of all monic irreducible polynomials $\mathbf{a}%
\in\mathbb{K}\left[  x\right]  $ that satisfy $\deg\mathbf{a}\mid r$). In
other words, $\mathbf{d}$ is a monic irreducible polynomial in $\mathbb{K}%
\left[  x\right]  $ and satisfies $\deg\mathbf{d}\mid r$. Hence, Lemma
\ref{lem.xpg-x.4} (applied to $\mathbf{a}=\mathbf{d}$) shows that
$\mathbf{d}\mid x^{p^{mr}}-x$ in $\mathbb{K}\left[  x\right]  $. Thus,
$\mathbf{d}\mid x^{p^{mr}}-x=\mathbf{u}_{1}\mathbf{u}_{2}\cdots\mathbf{u}_{k}%
$. Hence, Corollary \ref{cor.irredpol.pabk2} (applied to $\mathbf{p}%
=\mathbf{d}$ and $\mathbf{a}_{i}=\mathbf{u}_{i}$) shows that $\mathbf{d}%
=\mathbf{u}_{i}$ for some $i\in\left\{  1,2,\ldots,k\right\}  $. Consider this
$i$. The definition of $\alpha$ yields $\alpha\left(  i\right)  =\mathbf{u}%
_{i}$. Comparing this with $\mathbf{d}=\mathbf{u}_{i}$, we find $\mathbf{d}%
=\alpha\left(  \underbrace{i}_{\in\left\{  1,2,\ldots,k\right\}  }\right)
\in\alpha\left(  \left\{  1,2,\ldots,k\right\}  \right)  $.
\par
Now, forget that we fixed $\mathbf{d}$. We thus have shown that $\mathbf{d}%
\in\alpha\left(  \left\{  1,2,\ldots,k\right\}  \right)  $ for each
$\mathbf{d}\in D$. In other words, $D\subseteq\alpha\left(  \left\{
1,2,\ldots,k\right\}  \right)  $. In other words, the map $\alpha$ is
surjective.}. Hence, this map $\alpha$ is bijective. Thus, $\alpha$ is a
bijection from $\left\{  1,2,\ldots,k\right\}  $ to $D$.

Now,%
\begin{align*}
x^{p^{mr}}-x  &  =\mathbf{u}_{1}\mathbf{u}_{2}\cdots\mathbf{u}_{k}=\prod
_{i\in\left\{  1,2,\ldots,k\right\}  }\underbrace{\mathbf{u}_{i}%
}_{\substack{=\alpha\left(  i\right)  \\\text{(since }\alpha\left(  i\right)
=\mathbf{u}_{i}\\\text{(by the definition of }\alpha\text{))}}}=\prod
_{i\in\left\{  1,2,\ldots,k\right\}  }\alpha\left(  i\right)  =\prod
_{\mathbf{a}\in D}\mathbf{a}\\
&  \ \ \ \ \ \ \ \ \ \ \left(
\begin{array}
[c]{c}%
\text{here, we have substituted }\mathbf{a}\text{ for }\alpha\left(  i\right)
\text{ in the product,}\\
\text{since the map }\alpha:\left\{  1,2,\ldots,k\right\}  \rightarrow D\text{
is a bijection}%
\end{array}
\right) \\
&  =\prod_{\substack{\mathbf{a}\text{ is a monic}\\\text{irreducible
polynomial in }\mathbb{K}\left[  x\right]  \\\text{that satisfies }%
\deg\mathbf{a}\mid r}}\mathbf{a}\\
&  \ \ \ \ \ \ \ \ \ \ \left(
\begin{array}
[c]{c}%
\text{since }D\text{ is the set of all monic irreducible}\\
\text{polynomials }\mathbf{a}\in\mathbb{K}\left[  x\right]  \text{ that
satisfy }\deg\mathbf{a}\mid r
\end{array}
\right) \\
&  =\prod_{\substack{\mathbf{a}\in\mathbb{K}\left[  x\right]  \text{
is}\\\text{monic irreducible;}\\\deg\mathbf{a}\mid r}}\mathbf{a.}%
\end{align*}
This proves Theorem \ref{thm.xpg-x.full}.
\end{proof}

\begin{thebibliography}{999999999}                                                                                        %


\bibitem[Bourba72]{Bourba72}Nicolas Bourbaki, \textit{Elements of Mathematics:
Commutative Algebra}, Hermann 1972.

\bibitem[Bourba74]{Bourba74}Nicolas Bourbaki, \textit{Algebra I, Chapters
1-3}, Hermann 1974.\newline\url{https://archive.org/details/ElementsOfMathematics-AlgebraPart1/page/n0}

\bibitem[ConradF]{Conrad-FF}Keith Conrad, \textit{Finite fields}, 4 February
2018.\newline\url{https://kconrad.math.uconn.edu/blurbs/galoistheory/finitefields.pdf}

\bibitem[Escofi01]{Escofi01}%
\href{https://dx.doi.org/10.1007/978-1-4613-0191-2}{Jean-Pierre Escofier,
\textit{Galois Theory}, translated by Leila Schneps, Springer 2001}.

\bibitem[Goodma16]{Goodman}Frederick M. Goodman, \textit{Algebra: Abstract and
Concrete}, edition 2.6, 12 October 2016.\newline\url{https://homepage.divms.uiowa.edu/~goodman/algebrabook.dir/algebrabook.html}

\bibitem[Grinbe18]{Grinbe18}Darij Grinberg, \textit{Why the log and exp series
are mutually inverse}, May 11, 2018.\newline\url{https://www.cip.ifi.lmu.de/~grinberg/t/17f/logexp.pdf}

\bibitem[Grinbe19a]{19s}Darij Grinberg, \textit{Introduction to Modern Algebra
(UMN Spring 2019 Math 4281 notes)}, 31 May 2019.\newline%
\url{https://www.cip.ifi.lmu.de/~grinberg/t/19s/notes.pdf}\newline This
document is still unfinished; numbering may change in the future. For a frozen
version whose numbering matches the references used above, see\newline\url{https://github.com/darijgr/algebra19s/releases/tag/2019-05-31}

\bibitem[Grinbe19b]{19s-mt3s}Darij Grinberg, \textit{UMN Spring 2019 Math 4281
midterm \#3 solutions}.\newline\url{https://www.cip.ifi.lmu.de/~grinberg/t/19s/mt3s.pdf}

\bibitem[Hunger03]{Hungerford-03}%
\href{http://link.springer.com/book/10.1007/978-1-4612-6101-8}{Thomas W.
Hungerford, \textit{Algebra}, 12th printing, Springer 2003.}

\bibitem[Hunger14]{Hungerford}Thomas W. Hungerford, \textit{Abstract Algebra:
An Introduction}, 3rd edition, Brooks/Cole 2014.

\bibitem[Knapp16a]{Knapp1}Anthony W. Knapp, \textit{Basic Algebra}, digital
2nd edition 2016. \newline\url{https://www.math.stonybrook.edu/~aknapp/download.html}

\bibitem[LidNie97]{LidNie97}Rudolf Lidl, Harald Niederreiter, \textit{Finite
fields}, 2nd edition, Cambridge University Press 1997.\newline\url{https://doi.org/10.1017/CBO9780511525926}

\bibitem[Loehr11]{Loehr-BC}%
\href{https://www.math.vt.edu/people/nloehr/bijbook.html}{Nicholas A. Loehr,
\textit{Bijective Combinatorics}, Chapman \& Hall/CRC 2011.}

\bibitem[Milne18]{Milne-FT}James S. Milne, \textit{Fields and Galois Theory},
version 4.60, September 2018.\newline\url{https://www.jmilne.org/math/CourseNotes/}

\bibitem[Stewar15]{Stewar15}Ian Stewart, \textit{Galois theory}, 4th edition,
CRC Press 2015.\newline\url{http://matematicaeducativa.com/foro/download/file.php?id=1647}

\bibitem[Walker87]{Walker87}%
\href{https://web.archive.org/web/20170809055317/https://www.math.nmsu.edu/~elbert/AbsAlgeb.pdf}{Elbert
A. Walker, \textit{Introduction to Abstract Algebra}, Random House/Birkhauser,
New York, 1987.}
\end{thebibliography}


\end{document}