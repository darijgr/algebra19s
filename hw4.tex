% Like most advanced LaTeX files, this one begins with a lot of
% boilerplate. You don't need to understand (or even read) most of it.
% All you need to do is fill in your name, UMN ID, email address,
% and the number of the pset. (Search for "METADATA" to find the place
% for this.) Then, you can go straight to the "EXERCISE 1"
% section and start writing your solutions.
% The "VARIOUS USEFUL COMMANDS" section is probably worth taking a
% look at at some point.

%----------------------------------------------------------------------------------------
%	PACKAGES AND OTHER DOCUMENT CONFIGURATIONS
%----------------------------------------------------------------------------------------
\documentclass[paper=a4, fontsize=12pt]{scrartcl} % A4 paper and 12pt font size
\usepackage[T1]{fontenc} % Use 8-bit encoding that has 256 glyphs
\usepackage[english]{babel} % English language/hyphenation
\usepackage{amsmath,amsfonts,amsthm,amssymb} % Math packages
\usepackage{mathrsfs}    % More math packages
\usepackage{sectsty}  % Allows customizing section commands
\allsectionsfont{\centering \normalfont\scshape} % Make all section titles centered, the default font and small caps %remove this to left align section tites
\usepackage{hyperref} % Turns cross-references into hyperlinks,
                      % and defines \url and \href commands.
\usepackage{graphicx} % For embedding graphics files.
\usepackage{framed}   % For the "leftbar" environment used below.
\usepackage{ifthen}   % Used for the \powset command below.
\usepackage{lastpage} % for counting the number of pages
\usepackage[headsepline,footsepline,manualmark]{scrlayer-scrpage}
\usepackage[height=10in,a4paper,hmargin={1in,0.8in}]{geometry}
\usepackage[usenames,dvipsnames]{xcolor}
\usepackage{tikz}     % This is a powerful tool to draw vector
                      % graphics inside LaTeX. In particular, you can
                      % use it to draw graphs.
\usepackage{verbatim} % For the "verbatim" environment, in which
                      % special symbols can be used freely without
                      % confusing the compiler. (And it's typeset in
                      % a constant-width font.)
                      % Useful, e.g., for quoting code (or ASCII art).

%\numberwithin{table}{section} % Number tables within sections (i.e. 1.1, 1.2, 2.1, 2.2 instead of 1, 2, 3, 4)

\setlength\parindent{20pt} % Makes indentation for paragraphs longer.
                           % This makes paragraphs stand out more.

%----------------------------------------------------------------------------------------
%	VARIOUS USEFUL COMMANDS
%----------------------------------------------------------------------------------------
% The commands below might be convenient. For example, you probably
% prefer to write $\powset[2]{V}$ for the set of $2$-element subsets
% of $V$, rather than writing $\mathcal{P}_2(V)$.
% Notice that you can easily define your own commands like this.
% Caveat: Some of these commands need to be properly "guarded" when
% they occur in subscripts or superscripts. So you should not write
% $K_\CC$, but rather $K_{\CC}$.
\newcommand{\CC}{\mathbb{C}} % complex numbers
\newcommand{\RR}{\mathbb{R}} % real numbers
\newcommand{\QQ}{\mathbb{Q}} % rational numbers
\newcommand{\NN}{\mathbb{N}} % nonnegative integers
\newcommand{\DD}{{\mathbb{D}}} % dual numbers
\newcommand{\PP}{\mathbb{P}} % positive integers
\newcommand{\Z}[1]{\mathbb{Z}/#1\mathbb{Z}} % integers modulo k
                                            % (syntax: "\Z{k}")
\newcommand{\ZZ}{\mathbb{Z}} % integers
\newcommand{\id}{\operatorname{id}} % identity map
\newcommand{\lcm}{\operatorname{lcm}}
% Lowest common multiple. For historical reasons, LaTeX has a \gcd
% command built in, but not an \lcm command. The preceding line
% rectifies that.
\newcommand{\set}[1]{\left\{ #1 \right\}}
% $\set{...}$ compiles to {...} (set-brackets).
\newcommand{\abs}[1]{\left| #1 \right|}
% $\abs{...}$ compiles to |...| (absolute value, or size of a set).
\newcommand{\tup}[1]{\left( #1 \right)}
% $\tup{...}$ compiles to (...) (parentheses, or tuple-brackets).
\newcommand{\ive}[1]{\left[ #1 \right]}
% $\ive{...}$ compiles to [...] (Iverson bracket, aka truth value; also, set of first n integers).
\newcommand{\floor}[1]{\left\lfloor #1 \right\rfloor}
% $\floor{...}$ compiles to |_..._| (floor function).
\newcommand{\underbrack}[2]{\underbrace{#1}_{\substack{#2}}}
% $\underbrack{...1}{...2}$ yields
% $\underbrace{...1}_{\substack{...2}}$. This is useful for doing
% local rewriting transformations on mathematical expressions with
% justifications. For example, try this out:
% $ \underbrack{(a+b)^2}{= a^2 + 2ab + b^2 \\ \text{(by the binomial formula)}} $
\newcommand{\powset}[2][]{\ifthenelse{\equal{#2}{}}{\mathcal{P}\left(#1\right)}{\mathcal{P}_{#1}\left(#2\right)}}
% $\powset[k]{S}$ stands for the set of all $k$-element subsets of
% $S$. The argument $k$ is optional, and if not provided, the result
% is the whole powerset of $S$.
\newcommand{\mapeq}[1]{\underset{#1}{\equiv}}
% $\mapeq{f}$ yields $\underset{f}{\equiv}$.
% This is the "equality upon applying $f$" relation.
\newcommand{\eps}{\varepsilon}
% Just a shorthand for a commonly-used symbol.
\newcommand{\horrule}[1]{\rule{\linewidth}{#1}} % Create horizontal rule command with 1 argument of height
\newcommand{\nnn}{\nonumber\\} % Don't number this line in an "align" environment, and move on to the next line.

%----------------------------------------------------------------------------------------
%	MAKING SUMMATION SIGNS ALWAYS PUT THEIR BOUNDS ABOVE AND BELOW
%	THE SIGN
%----------------------------------------------------------------------------------------
% The following are hacks to ensure that sums (such as
% $\sum_{k=1}^n k$) always put their bounds (i.e., the $k=1$ and the
% $n$) underneath and above the sign, as opposed to on its right.
% Same for products (\prod), set unions (\bigcup) and set
% intersections (\bigcap). Remove the 8 lines below if you do not want
% this behavior.
\let\sumnonlimits\sum
\let\prodnonlimits\prod
\let\cupnonlimits\bigcup
\let\capnonlimits\bigcap
\renewcommand{\sum}{\sumnonlimits\limits}
\renewcommand{\prod}{\prodnonlimits\limits}
\renewcommand{\bigcup}{\cupnonlimits\limits}
\renewcommand{\bigcap}{\capnonlimits\limits}

%----------------------------------------------------------------------------------------
%	ENVIRONMENTS
%----------------------------------------------------------------------------------------
% The incantations below define how theorem environments
% (\begin{theorem} ... \end{theorem}) and their likes will look like.
\newtheoremstyle{plainsl}% <name>
  {8pt plus 2pt minus 4pt}% <Space above>
  {8pt plus 2pt minus 4pt}% <Space below>
  {\slshape}% <Body font>
  {0pt}% <Indent amount>
  {\bfseries}% <Theorem head font>
  {.}% <Punctuation after theorem head>
  {5pt plus 1pt minus 1pt}% <Space after theorem headi>
  {}% <Theorem head spec (can be left empty, meaning `normal')>

% Environments which make the text inside them slanted:
\theoremstyle{plainsl}
  \newtheorem{theorem}{Theorem}[section]
  \newtheorem{proposition}[theorem]{Proposition}
  \newtheorem{lemma}[theorem]{Lemma}
  \newtheorem{corollary}[theorem]{Corollary}
  \newtheorem{conjecture}[theorem]{Conjecture}
% Environments that don't:
\theoremstyle{definition}
  \newtheorem{definition}[theorem]{Definition}
  \newtheorem{example}[theorem]{Example}
  \newtheorem{exercise}[theorem]{Exercise}
  \newtheorem{examples}[theorem]{Examples}
  \newtheorem{algorithm}[theorem]{Algorithm}
  \newtheorem{question}[theorem]{Question}
 \theoremstyle{remark}
  \newtheorem{remark}[theorem]{Remark}
\newenvironment{statement}{\begin{quote}}{\end{quote}}
\newenvironment{fineprint}{\begin{small}}{\end{small}}

%----------------------------------------------------------------------------------------
%	METADATA
%----------------------------------------------------------------------------------------
\newcommand{\myname}{Darij Grinberg} % ENTER YOUR NAME HERE
\newcommand{\myid}{00000000} % ENTER YOUR UMN ID HERE
\newcommand{\mymail}{dgrinber@umn.edu} % ENTER YOUR EMAIL HERE
\newcommand{\psetnumber}{4} % ENTER THE NUMBER OF THIS PSET HERE

%----------------------------------------------------------------------------------------
%	HEADER AND FOOTER
%----------------------------------------------------------------------------------------
\ihead{Solutions to homework set \#\psetnumber} % Page header left
\ohead{page \thepage\ of \pageref{LastPage}} % Page header right
\ifoot{\myname, \myid} % left footer
\ofoot{\mymail} % right footer

%----------------------------------------------------------------------------------------
%	TITLE SECTION
%----------------------------------------------------------------------------------------
\title{	
\normalfont \normalsize 
\textsc{University of Minnesota, School of Mathematics} \\ [25pt] % Your university, school and/or department name(s)
\horrule{0.5pt} \\[0.4cm] % Thin top horizontal rule
\huge Math 4281: Introduction to Modern Algebra, \\
Spring 2019:
Homework \psetnumber\\% The assignment title
\horrule{2pt} \\[0.5cm] % Thick bottom horizontal rule
}
\author{\myname}

\begin{document}

\maketitle % Print the title

\begin{center} % Delete this if you want to save space!
{\large due date: \textbf{Wednesday, 13 March 2019} at the beginning of class, \\
or before that by email or canvas.

Please solve \textbf{at most 3 of the 6 exercises}!}
\end{center}

%----------------------------------------------------------------------------------------
%	EXERCISE 1
%----------------------------------------------------------------------------------------
\horrule{0.3pt} \\[0.4cm]

\section{Exercise 1: Equivalence relations always come from maps}

\subsection{Problem}

Let $S$ be a set.

Recall that if $T$ is a further set, and if $f : S \to T$
is a map, then $\mapeq{f}$
% The "\mapeq" command is not standard LaTeX.
% It is a macro I've defined in the header of this file.
% "\mapeq{f}" expands to "\underset{f}{\equiv}", which is
% a congruence sign with an "f" under it.
denotes the relation on $S$ defined by
\[
\tup{ a \mapeq{f} b }
\iff % This is the "if and only if" sign.
\tup{ f\tup{a} = f\tup{b} }.
\]
This is an equivalence relation, called ``equality upon
applying $f$'' or ``equality under $f$''.

Now, let $\sim$ be \textbf{any} equivalence relation on $S$.
Prove that $\sim$ has the form $\mapeq{f}$ for a properly
chosen set $T$ and a properly chosen $f : S \to T$.

More precisely, prove that $\sim$ equals $\mapeq{f}$,
where $T$ is the quotient set $S / \sim$ and where
$f : S \to T$ is the projection map
$\pi_\sim : S \to S / \sim$.

[\textbf{Hint:} To prove that two relations $R_1$ and $R_2$
on $S$ are equal, you need to check that every pair
$\tup{a, b}$ of elements of $S$ satisfies the equivalence
$\tup{a R_1 b} \iff \tup{a R_2 b}$.]

\subsection{Solution}

[...]

%----------------------------------------------------------------------------------------
%	EXERCISE 2
%----------------------------------------------------------------------------------------
\horrule{0.3pt} \\[0.4cm]

\section{Exercise 2: Totient-related sum}

\subsection{Problem}

Let $n > 1$ be an integer. Prove that
\[
\sum_{\substack{i \in \set{1, 2, \ldots, n}; \\ i \perp n}} i
= n \phi\tup{n} / 2 .
\]

\subsection{Solution}

[...]

%----------------------------------------------------------------------------------------
%	EXERCISE 3
%----------------------------------------------------------------------------------------
\horrule{0.3pt} \\[0.4cm]

\section{Exercise 3: Dual numbers}

\subsection{Problem}

Recall that complex numbers were defined as pairs
$\tup{a, b}$ of real numbers, with entrywise addition
and subtraction and a certain weird-looking
multiplication.

Let me define a different kind of ``numbers'': the
\textit{dual numbers}.
(The word ``numbers'' may appear a bit inappropriate
for them, but it is not exactly a trademark...)

We define a \textit{dual number} to be a pair
$\tup{a, b}$ of two real numbers $a$ and $b$.

We let $\DD$ % a macro for $\mathbb{D}$
be the set of all dual numbers.

For each real number $r$, we denote the dual number
$\tup{r, 0}$ by $r_\DD$.

We let $\eps$ denote the dual number $\tup{0, 1}$.

Define three binary operations $+$, $-$ and $\cdot$
on $\DD$ by setting
\begin{align}
\tup{a, b} + \tup{c, d} &= \tup{a+c, b+d}, \\
\tup{a, b} - \tup{c, d} &= \tup{a-c, b-d}, \\
\tup{a, b} \cdot \tup{c, d} &= \tup{ac, ad+bc}
\end{align}
for all $\tup{a, b} \in \DD$ and $\tup{c, d} \in \DD$.

(Note that the only difference to complex numbers
is the definition of $\cdot$, which is lacking
a $-bd$ term.)

Again, we are following standard PEMDAS rules\footnote{
\url{https://en.wikipedia.org/wiki/Order_of_operations}} for
the order of operations, and we abbreviate $\alpha \cdot \beta$
as $\alpha \beta$.

\begin{enumerate}

\item[\textbf{(a)}]
Prove that
$\alpha \cdot \tup{\beta \cdot \gamma}
= \tup{\alpha \cdot \beta} \cdot \gamma$
for any $\alpha, \beta, \gamma \in \DD$.

\end{enumerate}

You can use the following properties of dual numbers
without proof (they are all essentially obvious):

\begin{itemize}

\item We have $\alpha + \beta = \beta + \alpha$
for any $\alpha, \beta \in \DD$.

\item We have $\alpha + \tup{\beta + \gamma}
= \tup{\alpha + \beta} + \gamma$
for any $\alpha, \beta, \gamma \in \DD$.

\item We have $\alpha + 0_\DD = \alpha$
for any $\alpha \in \DD$.

\item We have $\alpha \cdot 1_\DD = \alpha$
for any $\alpha \in \DD$.

\item We have $\alpha \cdot \beta = \beta \cdot \alpha$
for any $\alpha, \beta \in \DD$.

\item We have $\alpha \cdot \tup{\beta + \gamma}
= \alpha \beta + \alpha \gamma$
and $\tup{\alpha + \beta} \cdot \gamma
= \alpha \gamma + \beta \gamma$
for any $\alpha, \beta, \gamma \in \DD$.

\item We have $\alpha \cdot 0_\DD = 0_\DD$
for any $\alpha \in \DD$.

\item If $\alpha, \beta, \gamma \in \DD$,
then we have the equivalence
$\tup{\alpha - \beta = \gamma}
\iff \tup{\alpha = \beta + \gamma}$.
\end{itemize}

We shall identify each real number $r$ with the dual
number $r_\DD = \tup{r, 0}$.

\begin{enumerate}

\item[\textbf{(b)}]
Prove that $a + b\eps = \tup{a, b}$
for any $a, b \in \RR$.

\end{enumerate}

An \textit{inverse} of a dual number
$\alpha \in \DD$ means a dual number $\beta$ such
that $\alpha \beta = 1_\DD$.
This inverse is unique, and is called $\alpha^{-1}$.

\begin{enumerate}

\item[\textbf{(c)}]
Prove that a dual number $\alpha = a + b\eps$
(with $a, b \in \RR$) has an inverse if and only
if $a \neq 0$.

\item[\textbf{(d)}]
If $a, b \in \RR$ satisfy $a \neq 0$, prove
that the inverse of the dual number $a + b\eps$
is $\dfrac{1}{a} - \dfrac{b}{a^2} \eps$.

\end{enumerate}

We define finite sums and products of dual numbers
in the usual way (i.e., just as finite sums and products
of real numbers were defined).
Here, an empty sum of dual numbers is always understood
to be $0_\DD$, whereas an empty product of dual numbers
is always understood to be $1_\DD$.

If $\alpha$ is a dual number and $k \in \NN$, then
the \textit{$k$-th power of $\alpha$} is defined to be the
dual number
$\underbrace{\alpha \alpha \cdots \alpha}_{k \text{ factors}}$.
This $k$-th power is denoted by $\alpha^k$.
Thus, in particular,
$\alpha^0 =
 \underbrace{\alpha \alpha \cdots \alpha}_{0 \text{ factors}}
= \tup{\text{empty product}} = 1_\DD$.

\begin{enumerate}

\item[\textbf{(e)}]
Let $P\tup{x} = a_k x^k + a_{k-1} x^{k-1} + \cdots + a_0$
be a polynomial with real coefficients.
Prove that
\[
P \tup{a + b \eps}
= P\tup{a} + b P^\prime\tup{a} \eps
\qquad \text{for any } a, b \in \RR .
\]
Here, $P^\prime$ denotes the derivative of $P$, which
is defined by
\[
P^\prime\tup{x}
= k a_k x^{k-1} + \tup{k-1} a_{k-1} x^{k-2} + \cdots + 1 a_1 x^0 .
\]

\end{enumerate}

\subsection{Remark}

The dual number $\eps$ is one of the simplest
``rigorous infinitesimals'' that appear in mathematics.
Part \textbf{(e)} of the exercise shows that we can
literally write
$P \tup{a + \eps}
= P\tup{a} + P^\prime\tup{a} \eps$
when $P$ is a polynomial,
without having to compute any limits.
It is tempting to ``solve'' this equation for
$P^\prime\tup{a}$, thus obtaining something like
$P^\prime\tup{a} = \dfrac{P \tup{a + \eps} - P\tup{a}}{\eps}$.
However, this needs to be taken with a grain of salt,
since $\eps$ has no inverse and the fraction
$\dfrac{P \tup{a + \eps} - P\tup{a}}{\eps}$
is not uniquely determined.
There are other, subtler ways to put infinitesimals
on a firm algebraic footing, but dual numbers are already
useful in some situations.

Note that dual numbers have \textit{zero-divisors}:
i.e., there exist nonzero dual numbers $a$ and $b$ such
that $ab = 0$. The simplest example is probably
$\eps^2 = 0$ (despite $\eps \neq 0$).

\subsection{Solution}

[...]

%----------------------------------------------------------------------------------------
%	EXERCISE 4
%----------------------------------------------------------------------------------------
\horrule{0.3pt} \\[0.4cm]

\section{Exercise 4: $\ZZ\ive{\sqrt2}$}

\subsection{Problem}

Let $\ZZ\ive{\sqrt2}$ denote the set of all reals of the
form $a + b \sqrt2$ with $a, b \in \ZZ$.
We shall call such reals \textit{$\sqrt2$-integers}.

\begin{enumerate}

\item[\textbf{(a)}]
Prove that any $\alpha, \beta \in \ZZ\ive{\sqrt2}$ satisfy
$\alpha + \beta \in \ZZ\ive{\sqrt2}$ and
$\alpha - \beta \in \ZZ\ive{\sqrt2}$ and
$\alpha \beta \in \ZZ\ive{\sqrt2}$.

\item[\textbf{(b)}]
Prove that every element of $\ZZ\ive{\sqrt2}$ can be
written as $a + b \sqrt2$ for a \textbf{unique} pair
$\tup{a, b}$ of integers.
(In other words, if four integers $a, b, c, d$
satisfy $a + b \sqrt2 = c + d \sqrt2$, then $a = c$
and $b = d$.)

\end{enumerate}

For any $\alpha\in \ZZ\ive{\sqrt2}$, define the
\textit{$\sqrt2$-norm} $N_2\tup{\alpha}$ of $\alpha$ by
$N_2\tup{\alpha} = a^2 - 2b^2$,
where $\alpha$ is written in the form $\alpha = a + b \sqrt2$
with $a, b \in \ZZ$.
This is well-defined by part \textbf{(b)} of this
exercise.

\begin{enumerate}

\item[\textbf{(c)}]
Prove that
$N_2\tup{\alpha \beta} = N_2\tup{\alpha} N_2\tup{\beta}$
for all $\alpha, \beta \in \ZZ\ive{\sqrt2}$.

\end{enumerate}

Let $\tup{p_0, p_1, p_2, \ldots}$ be the sequence of
nonnegative integers defined recursively by
\[
p_0 = 0, \qquad
p_1 = 1, \qquad
\text{and } \qquad
p_n = 2p_{n-1} + p_{n-2} \text{ for all } n \geq 2.
\]
(Thus, $p_2 = 2$ and $p_3 = 5$ and $p_4 = 12$ and so on.)

\begin{enumerate}

\item[\textbf{(d)}]
Prove that $p_{n+1} p_{n-1} - p_n^2 = \tup{-1}^n$
for each $n \geq 1$.

\item[\textbf{(e)}]
Prove that
$\tup{p_{n-1} + p_n + p_n \sqrt2} \cdot \tup{p_{n-1} + p_n - p_n \sqrt2} = \tup{-1}^n$
for each $n \geq 1$.

\end{enumerate}

[\textbf{Hint:} For \textbf{(d)}, use induction.]

\subsection{Remark}

The set $\ZZ\ive{\sqrt2}$ of $\sqrt2$-integers
is rather similar to the set
$\ZZ\ive{i}$ of Gaussian integers: the former has elements
of the form $a + b\sqrt2$ with $a, b \in \ZZ$,
while the latter has elements of the form
$a + b\sqrt{-1}$ with $a, b \in \ZZ$.
The $\sqrt2$-norm on $\ZZ\ive{\sqrt2}$ is an analogue of
the (usual) norm on $\ZZ\ive{i}$.
However, visually speaking, the latter set is ``spread
out'' in the Euclidean plane, while the former is
``concentrated'' on the real line (and actually everywhere
dense on it -- i.e., every little interval on the real
line has a $\sqrt2$-integer inside it).
The difference has algebraic consequences; in particular,
there are only four units ($1, -1, i, -i$) in $\ZZ\ive{i}$,
whereas $\ZZ\ive{\sqrt2}$ has infinitely many units
(namely, part \textbf{(e)} of the exercise shows that
$p_{n-1} + p_n + p_n \sqrt2$ is a unit for each $n \geq 1$).

\subsection{Solution}

[...]

%----------------------------------------------------------------------------------------
%	EXERCISE 5
%----------------------------------------------------------------------------------------
\horrule{0.3pt} \\[0.4cm]

\section{Exercise 5: Euler's theorem for non-coprime integers}

\subsection{Problem}

Let $a$ be an integer, and let $n$ be a positive integer.
Prove that
$a^n \equiv a^{n - \phi\tup{n}} \mod n$.

\subsection{Solution}

[...]

%----------------------------------------------------------------------------------------
%	EXERCISE 6
%----------------------------------------------------------------------------------------
\horrule{0.3pt} \\[0.4cm]

\section{Exercise 6: Wilson strikes again}

\subsection{Problem}

Let $p$ be an odd prime.
Write $p$ in the form $p = 2k+1$ for some $k \in \NN$.
Prove that $k!^2 \equiv - \tup{-1}^k \mod p$.

[\textbf{Hint:} Each $j \in \ZZ$ satisfies $j \tup{p-j} \equiv -j^2 \mod p$.]

\subsection{Solution}

[...]

\begin{thebibliography}{99999999}                                                                                         %

% Feel free to add your sources -- or copy some from the source code
% of the class notes ( http://www-users.math.umn.edu/~dgrinber/19s/notes.tex ).

% This is the bibliography: The list of papers/books/articles/blogs/...
% cited. The syntax is: "\bibitem[name]{tag}Reference",
% where "name" is the name that will appear in the compiled
% bibliography, and "tag" is the tag by which you will refer to
% the source in the TeX file. For example, the following source
% has name "Grinbe19" (so you will see it referenced as
% "[Grinbe19]" in the compiled PDF) and tag "detnotes" (so you
% can cite it by writing "\cite{detnotes}").

\bibitem[Grinbe19]{detnotes}Darij Grinberg,
\textit{Notes on the combinatorial fundamentals of algebra},
10 January 2019. \\
\url{http://www.cip.ifi.lmu.de/~grinberg/primes2015/sols.pdf}
\\
The numbering of theorems and formulas in this link might shift
when the project gets updated; for a ``frozen'' version whose
numbering is guaranteed to match that in the citations above, see
\url{https://github.com/darijgr/detnotes/releases/tag/2019-01-10} .

\end{thebibliography}

\end{document}

