\documentclass[numbers=enddot,12pt,final,onecolumn,notitlepage]{scrartcl}%
\usepackage[headsepline,footsepline,manualmark]{scrlayer-scrpage}
\usepackage[all,cmtip]{xy}
\usepackage{amssymb}
\usepackage{amsmath}
\usepackage{amsthm}
\usepackage{framed}
\usepackage{comment}
\usepackage{color}
\usepackage{hyperref}
\usepackage[sc]{mathpazo}
\usepackage[T1]{fontenc}
\usepackage{tikz}
\usepackage{needspace}
\usepackage{tabls}
\usepackage{wasysym}
%TCIDATA{OutputFilter=latex2.dll}
%TCIDATA{Version=5.50.0.2960}
%TCIDATA{LastRevised=Wednesday, February 06, 2019 08:37:57}
%TCIDATA{SuppressPackageManagement}
%TCIDATA{<META NAME="GraphicsSave" CONTENT="32">}
%TCIDATA{<META NAME="SaveForMode" CONTENT="1">}
%TCIDATA{BibliographyScheme=Manual}
%TCIDATA{Language=American English}
%BeginMSIPreambleData
\providecommand{\U}[1]{\protect\rule{.1in}{.1in}}
%EndMSIPreambleData
\usetikzlibrary{arrows}
\newcounter{exer}
\newcounter{exera}
\numberwithin{exer}{section}
\theoremstyle{definition}
\newtheorem{theo}{Theorem}[section]
\newenvironment{theorem}[1][]
{\begin{theo}[#1]\begin{leftbar}}
{\end{leftbar}\end{theo}}
\newtheorem{lem}[theo]{Lemma}
\newenvironment{lemma}[1][]
{\begin{lem}[#1]\begin{leftbar}}
{\end{leftbar}\end{lem}}
\newtheorem{prop}[theo]{Proposition}
\newenvironment{proposition}[1][]
{\begin{prop}[#1]\begin{leftbar}}
{\end{leftbar}\end{prop}}
\newtheorem{defi}[theo]{Definition}
\newenvironment{definition}[1][]
{\begin{defi}[#1]\begin{leftbar}}
{\end{leftbar}\end{defi}}
\newtheorem{remk}[theo]{Remark}
\newenvironment{remark}[1][]
{\begin{remk}[#1]\begin{leftbar}}
{\end{leftbar}\end{remk}}
\newtheorem{coro}[theo]{Corollary}
\newenvironment{corollary}[1][]
{\begin{coro}[#1]\begin{leftbar}}
{\end{leftbar}\end{coro}}
\newtheorem{conv}[theo]{Convention}
\newenvironment{convention}[1][]
{\begin{conv}[#1]\begin{leftbar}}
{\end{leftbar}\end{conv}}
\newtheorem{quest}[theo]{Question}
\newenvironment{question}[1][]
{\begin{quest}[#1]\begin{leftbar}}
{\end{leftbar}\end{quest}}
\newtheorem{warn}[theo]{Warning}
\newenvironment{conclusion}[1][]
{\begin{warn}[#1]\begin{leftbar}}
{\end{leftbar}\end{warn}}
\newtheorem{conj}[theo]{Conjecture}
\newenvironment{conjecture}[1][]
{\begin{conj}[#1]\begin{leftbar}}
{\end{leftbar}\end{conj}}
\newtheorem{exam}[theo]{Example}
\newenvironment{example}[1][]
{\begin{exam}[#1]\begin{leftbar}}
{\end{leftbar}\end{exam}}
\newtheorem{exmp}[exer]{Exercise}
\newenvironment{exercise}[1][]
{\begin{exmp}[#1]\begin{leftbar}}
{\end{leftbar}\end{exmp}}
\newenvironment{statement}{\begin{quote}}{\end{quote}}
\newenvironment{fineprint}{\begin{small}}{\end{small}}
\iffalse
\newenvironment{proof}[1][Proof]{\noindent\textbf{#1.} }{\ \rule{0.5em}{0.5em}}
\newenvironment{question}[1][Question]{\noindent\textbf{#1.} }{\ \rule{0.5em}{0.5em}}
\fi
\let\sumnonlimits\sum
\let\prodnonlimits\prod
\let\cupnonlimits\bigcup
\let\capnonlimits\bigcap
\renewcommand{\sum}{\sumnonlimits\limits}
\renewcommand{\prod}{\prodnonlimits\limits}
\renewcommand{\bigcup}{\cupnonlimits\limits}
\renewcommand{\bigcap}{\capnonlimits\limits}
\setlength\tablinesep{3pt}
\setlength\arraylinesep{3pt}
\setlength\extrarulesep{3pt}
\voffset=0cm
\hoffset=-0.7cm
\setlength\textheight{22.5cm}
\setlength\textwidth{15.5cm}
\newcommand\arxiv[1]{\href{http://www.arxiv.org/abs/#1}{\texttt{arXiv:#1}}}
\newenvironment{verlong}{}{}
\newenvironment{vershort}{}{}
\newenvironment{noncompile}{}{}
\excludecomment{verlong}
\includecomment{vershort}
\excludecomment{noncompile}
\newcommand{\CC}{\mathbb{C}}
\newcommand{\RR}{\mathbb{R}}
\newcommand{\QQ}{\mathbb{Q}}
\newcommand{\NN}{\mathbb{N}}
\newcommand{\ZZ}{\mathbb{Z}}
\newcommand{\id}{\operatorname{id}}
\newcommand{\lcm}{\operatorname{lcm}}
\newcommand{\rev}{\operatorname{rev}}
\newcommand{\powset}[2][]{\ifthenelse{\equal{#2}{}}{\mathcal{P}\left(#1\right)}{\mathcal{P}_{#1}\left(#2\right)}}
\newcommand{\set}[1]{\left\{ #1 \right\}}
\newcommand{\abs}[1]{\left| #1 \right|}
\newcommand{\tup}[1]{\left( #1 \right)}
\newcommand{\ive}[1]{\left[ #1 \right]}
\newcommand{\floor}[1]{\left\lfloor #1 \right\rfloor}
\newcommand{\lf}[2]{#1^{\underline{#2}}}
\newcommand{\underbrack}[2]{\underbrace{#1}_{\substack{#2}}}
\newcommand{\horrule}[1]{\rule{\linewidth}{#1}}
\newcommand{\nnn}{\nonumber\\}
\newcommand{\sslash}{\mathbin{/\mkern-6mu/}}
\ihead{Math 4281 notes}
\ohead{page \thepage}
\cfoot{}
\begin{document}

\title{UMN Spring 2019 Math 4281 notes}
\author{Darij Grinberg}
\date{
%TCIMACRO{\TeXButton{today}{\today} }%
%BeginExpansion
\today
%EndExpansion
}
\maketitle
\tableofcontents

\section{Introduction}

This file will contain the notes from the Math 4281 class (``Introduction to
Modern Algebra'') I am teaching at UMN in Spring 2019. I will type the first
draft directly in the classroom, and subsequently expand it into proper
writing. Occasionally, I will also add extra sections not covered in class.

The website of the class is
\url{http://www-users.math.umn.edu/~dgrinber/19s/index.html} ; you will find
homework sets there.

\subsection{Organisation}

See \href{http://www-users.math.umn.edu/~dgrinber/19s/syll.pdf}{the syllabus}
for the organization of this class and for the homework.

\subsection{Literature}

Many books have been written about abstract algebra. I have only a passing
familiarity with most of them. Some of the ``bibles'' of the subject (bulky
texts covering lots of material) are Dummit/Foote \cite{Dummit-Foote}, Knapp
\cite{Knapp1} and \cite{Knapp2} (both freely available), van der Waerden
\cite{Waerden1} and \cite{Waerden2} (one of the oldest texts on modern
algebra, thus rather dated, but still as readable as ever).
%Two other textbooks are Bosch \cite{Bosch} and Artin \cite{Artin}.


Of course, any book longer than 200 pages likely goes further than our course
will (unless it is full of details or solved exercises or printed in really
large letters). Thus, let me recommend some more introductory sources.
Siksek's lecture notes \cite{Siksek} are a readable introduction that is a lot
more amusing than I had ever expected an algebra text to be. Goodman's free
book \cite{Goodman} combines introductory material with geometric motivation
and applications, such as the classification of regular polyhedra and
2-dimensional crystals. In a sense, it is a great complement to our
ungeometric course. Pinter's \cite{Pinter} often gets used in classes like
ours. Armstrong's notes \cite{Armstrong} cover a significant part of what we
do (and he will likely have notes for a second course written up by the end of
this semester).

Keith Conrad's blurbs \cite{Conrad*} are not a book, as they only cover
selected topics. But at pretty much every topic they cover, they are one of
the best sources (clear, full of examples, and often going fairly deep). We
shall follow one of them particularly closely: the one on Gaussian integers
\cite{Conrad-Gauss}.

We will use some basic linear algebra, all of which can be found in Hefferon's
book \cite{Hefferon} (but we won't need all of this book). As far as
determinants are concerned, we will briefly build up their theory; we refer to
\cite[Section 12 \& Appendix B]{Strickland} for proofs (and to \cite[Chapter
6]{detnotes} for a really detailed and formal treatment).

This course will begin (after some motivating questions) with a survey of
elementary number theory. This is in itself a deep subject (despite the name)
with a long history (\href{https://en.wikipedia.org/wiki/Plimpton_322}{perhaps
as old as mathematics}), and of course we will just scratch the surface. Books
like \cite{NiZuMo91}, \cite{Burton} and \cite{Uspensky-Heaslet} cover a lot
more than we can do. The Gallier/Quaintance survey \cite{Gallier-RSA} covers a
good amount of basics and more.

We assume that the reader is familiar with the commonplaces of mathematical
argumentation, such as induction (including strong induction),
\textquotedblleft WLOG\textquotedblright\ arguments, proof by contradiction,
summation signs ($\sum$) and polynomials (a vague notion of polynomials will
suffice; we will give a precise definition when it becomes necessary). If not,
several texts can be helpful in achieving such familiarity: e.g.,
\cite[particularly Chapters 1--5]{LeLeMe}, \cite{Hammack}, \cite{Day}.

\begin{center}
\textbf{2019-01-23 lecture}
\end{center}

\subsection{The plan}

The material I am going to cover is mostly standard. However, the order in
which I will go through it is somewhat unusual: I will spend a lot of time
studying the basic examples before defining abstract notions such as
``group'', ``monoid'', ``ring'' and ``field''. This way, once I come to these
notions, you'll already have many examples to work with. (Don't be fooled by
the word ``example'': We will prove a lot about them, much of which is neither
straightforward nor easy.)

First, I will show some motivating questions that are easy to state yet
require abstract algebra to prove. We will hopefully see their answers by the
end of this class. (Some of them can also be answered elementarily, without
using abstract algebra, but such answers usually take more work and are harder
to find.)

\subsection{\label{subsect.intro.sum-of-2sq}Motivation: $n=x^{2}+y^{2}$}

A \textit{perfect square} means the square of an integer. Thus, the perfect
squares are
\[
0^{2} = 0, \qquad1^{2} = 1, \qquad2^{2} = 4, \qquad3^{2} = 9, \qquad4^{2} =
16, \qquad\ldots.
\]


Here is an old problem (first solved by Pierre de Fermat in 1640, but
apparently already studied by Diophantus in the 3rd Century):

\begin{question}
\label{quest.intro.sum-of-2sq.1} What integers can be written as sums of two
perfect squares?
\end{question}

For example, $5$ can be written in this way, since $5=2^{2}+1^{2}$.

So can $4$, since $4=2^{2}+0^{2}$. (Keep in mind that $0$ is a perfect square.)

However, $7$ cannot be written in this way. In fact, if we had $7 = a^{2} +
b^{2}$ for two integers $a$ and $b$, then $a^{2}$ and $b^{2}$ would have to be
$\leq7$ (since $a^{2}$ and $b^{2}$ are always $\geq0$, no matter what sign $a$
and $b$ have); but the only perfect squares that are $\leq7$ are $0,1,4$, and
there is no way to write $7$ as a sum of two of these perfect squares (just
check all the possibilities).

For a similar but simpler reason, no negative number can be written as a sum
of two perfect squares.

We can of course approach Question~\ref{quest.intro.sum-of-2sq.1} using a
computer: It is very easy to check, for a given integer $n$, whether $n$ is a
sum of two perfect squares. (Just check all possibilities for $a$ and $b$ for
the validity of the equation $n = a^{2} + b^{2}$. You only need to try $a$ and
$b$ belonging to $\left\{  0, 1, \ldots, \left\lfloor \sqrt{n} \right\rfloor
\right\}  $, where $\left\lfloor y \right\rfloor $ (for a real number $y$)
denotes the smallest integer that is less or equal than $y$ (also known as
``$y$ rounded down'').) If you do this, you will see that among the first
$101$ nonnegative integers, the ones that can be written as sums of two
perfect squares are precisely
\begin{align*}
&  0, 1, 2, 4, 5, 8, 9, 10, 13, 16, 17, 18, 20, 25, 26, 29,\\
&  32, 34, 36, 37, 40, 41, 45, 49, 50, 52, 53, 58, 61, 64,\\
&  65, 68, 72, 73, 74, 80, 81, 82, 85, 89, 90, 97, 98, 100 .
\end{align*}
Having this data, you can look up the sequence in \href{https://oeis.org/}{the
Online Encyclopedia of Integer Sequences (short OEIS)}, and see that the
sequence of these integers is known as \href{https://oeis.org/A001481}{OEIS
Sequence A001481}. In the ``Comments'' field, you can read a lot of what is
known about it (albeit in telegraphic style).

For example, one of the comments says ``Closed under multiplication''. This is
short for ``if you multiply two entries of the sequence, then the product will
again be an entry of the sequence''. In other words, if you multiply two
integers that are sums of two perfect squares, then you get another sum of two
perfect squares. Why is this so?

It turns out that there is a \textquotedblleft simple\textquotedblright%
\ reason for this: the identity
\begin{equation}
\left(  a^{2}+b^{2}\right)  \left(  c^{2}+d^{2}\right)  =\left(  ad+bc\right)
^{2}+\left(  ac-bd\right)  ^{2}, \label{eq.intro.sum-of-2sq.sum*sum}%
\end{equation}
which holds for arbitrary reals $a,b,c,d$ (and thus, in particular, for
integers). This is known as
\href{https://en.wikipedia.org/wiki/Brahmagupta-Fibonacci_identity}{the
Brahmagupta-Fibonacci identity}, and of course can easily be proven by
expanding both sides. But how would you come up with such an identity?

If you stare at the above sequence long enough, you may also discover another
pattern: An integer of the form $4k+3$ with integer $k$ (that is, an integer
that is larger by $3$ than a multiple of $4$) can never be written as a sum of
two perfect squares. (Thus, $3,7,11,15,19,23,\ldots$ cannot be written in this
way.) This does not account for all integers that cannot be written in this
way, but it does provide some clues to the answer that we will later see. In
order to prove this observation, we shall need basic modular arithmetic (or at
least division with remainder); we will see this proof very soon (see Exercise
\ref{exe.ent.even-odd-sumsq} \textbf{(c)}).

Further questions can be asked. One of them is: Given an integer $n$, how many
ways are there to represent $n$ as a sum of two perfect squares? This is
actually several questions masquerading as one, since it is not so clear what
a ``way'' is. Do $5 = 1^{2} + 2^{2}$ and $5 = 2^{2} + 1^{2}$ count as two
different ways? What about $5 = 1^{2} + 2^{2}$ versus $5 = \left(  -1 \right)
^{2} + 2^{2}$ (here, the perfect squares are the same, but do we really want
to count the squares or rather the numbers we are squaring?).

Let me formalize the question as follows:

\begin{question}
\label{quest.intro.sum-of-2sq.2} Let $n$ be an integer.

\textbf{(a)} How many pairs $\left(  a, b \right)  \in\mathbb{N}^{2}$ are
there that satisfy $n = a^{2} + b^{2}$ ? Here, and in the following,
$\mathbb{N}$ denotes the set $\left\{  0, 1, 2, \ldots\right\}  $ of all
nonnegative integers.

\textbf{(b)} How many pairs $\left(  a, b \right)  \in\mathbb{Z}^{2}$ are
there that satisfy $n = a^{2} + b^{2}$ ? Here, and in the following,
$\mathbb{Z}$ denotes the set $\left\{  \ldots, -2, -1, 0, 1, 2, \ldots
\right\}  $ of all integers.

\textbf{(c)} How do these counts change if we count \textbf{unordered} pairs
instead (i.e., count $\left(  a, b \right)  $ and $\left(  b, a \right)  $ as
one only)?
\end{question}

Note that when I say ``pair'', I always mean ``ordered pair'' by default,
unless I explicitly say ``unordered pair''.

Again, a little bit of programming easily yields answers to all three parts of
this question for small values of $n$, and the resulting data can be plugged
into the OEIS and yields lots of information.

\begin{proof}
[First steps toward answering Question~\ref{quest.intro.sum-of-2sq.2}%
.]\textbf{(a)} I claim that the number of such pairs is even unless $n$ is
twice a perfect square (i.e., unless $n = 2m^{2}$ for some integer $m$); in
the latter case, this number is odd instead.

Why? Let me define a \textit{solution} to be a pair $\left(  a,b\right)  $
such that $n=a^{2}+b^{2}$. So I want to know whether the number of solutions
is even or odd. But we have $a^{2}+b^{2}=b^{2}+a^{2}$ for all $a$ and $b$.
Thus, if $\left(  a,b\right)  $ is a solution, then so is $\left(  b,a\right)
$. Hence, the solutions themselves \textquotedblleft come in
pairs\textquotedblright, with each solution $\left(  a,b\right)  $ being
matched to the solution $\left(  b,a\right)  $, unless there is a solution
$\left(  a,b\right)  $ with $a=b$ (because such a solution would be matched to
itself, and thus not form an actual pair). But solutions $\left(  a,b\right)
$ with $a=b$ are easy to classify: If $n$ is twice a perfect square, then
there is exactly one such solution (namely, $\left(  \sqrt{n/2},\sqrt
{n/2}\right)  $); otherwise there is none (because $n=a^{2}+b^{2}$ with $a=b$
leads to $n=b^{2}+b^{2}=2b^{2}$, which can only happen when $n$ is twice a
perfect square). Since we know that all the other solutions \textquotedblleft
come in pairs\textquotedblright, we thus conclude that the number of solutions
is odd if $n$ is twice a perfect square and even otherwise. This proves our claim.

Of course, we have not made much headway into
Question~\ref{quest.intro.sum-of-2sq.2}; knowing whether a number is even or
odd is far from knowing the number itself. But I think the argument above was
worth showing; similar reasoning is used a lot in algebra.

\textbf{(b)} By reasoning analogous to the one we used in part \textbf{(a)},
we can see that the number of such pairs will be divisible by $8$ whenever $n$
is neither a perfect square nor twice a perfect square. Indeed, this relies on
the fact that
\begin{align*}
a^{2} + b^{2}  &  = b^{2} + a^{2} = \left(  -a \right)  ^{2} + b^{2} = b^{2} +
\left(  -a \right)  ^{2} = a^{2} + \left(  -b \right)  ^{2} = \left(  -b
\right)  ^{2} + a^{2}\\
&  = \left(  -a \right)  ^{2} + \left(  -b \right)  ^{2} = \left(  -b \right)
^{2} + \left(  -a \right)  ^{2}%
\end{align*}
for all $a$ and $b$. Thus the pairs $\left(  a, b \right)  \in\mathbb{Z}^{2}$
that satisfy $n = a^{2} + b^{2}$ don't just come in pairs; they come in sets
of $8$ (namely, each $\left(  a, b \right)  $ comes in a set with $\left(  b,
a \right)  $, $\left(  -a, b \right)  $, $\left(  b, -a \right)  $, $\left(
a, -b \right)  $, $\left(  -b, a \right)  $, $\left(  -a, -b \right)  $ and
$\left(  -b, -a \right)  $). These sets of $8$ can ``degenerate'' to smaller
sets when some of their elements coincide, but this can only happen when $n$
is a perfect square (in which case we can have $\left(  a, b \right)  =
\left(  -a, b \right)  $ for example) or twice a perfect square (in which case
we can have $\left(  a, b \right)  = \left(  b, a \right)  $ or $\left(  a, b
\right)  = \left(  -b, -a \right)  $ or other such coincidences). (Check this!)

\textbf{(c)} We can reduce this to parts \textbf{(a)} and \textbf{(b)}.
Indeed:\footnote{In the rest of this argument, \textquotedblleft
pair\textquotedblright\ will always mean \textquotedblleft pair $\left(
a,b\right)  $ satisfying $n=a^{2}+b^{2}$\textquotedblright.}

\begin{itemize}
\item When $n$ is not twice a perfect square, the number of unordered pairs
will be half the number of ordered pairs, since each unordered pair $\left(
u,v\right)  _{\text{unordered}}$ corresponds to precisely two ordered pairs
$\left(  u,v\right)  $ and $\left(  v,u\right)  $.

\item When $n$ is twice a perfect square, we have%
\begin{align*}
&  \left(  \text{the number of unordered pairs}\right) \\
&  =\dfrac{\left(  \text{the number of ordered pairs}\right)  +\left(
\text{the number of pairs with }a=b\right)  }{2}.
\end{align*}
Indeed, each unordered pair $\left(  u,v\right)  _{\text{unordered}}$
corresponds to precisely two ordered pairs $\left(  u,v\right)  $ and $\left(
v,u\right)  $ unless $u=v$, in which case it corresponds to only one ordered
pair. Thus, if we multiply the number of unordered pairs by $2$, then we
\textbf{overcount} the number of ordered pairs, because we are counting the
pairs $\left(  u,v\right)  $ with $u=v$ (that is, the pairs with $a=b$) twice.
So we get $\left(  \text{the number of ordered pairs}\right)  +\left(
\text{the number of pairs with }a=b\right)  $. This proves our above formula.

What is the number of pairs with $a=b$ ? If $n=0$, then it is $1$ (and the
only such pair is $\left(  0,0\right)  $). Otherwise, it is $1$ if we are
counting pairs in $\mathbb{N}^{2}$ (and the only such pair is $\left(
\sqrt{n/2},\sqrt{n/2}\right)  $), and is $2$ if we are counting pairs in
$\mathbb{Z}^{2}$ (and the only two such pairs are $\left(  \sqrt{n/2}%
,\sqrt{n/2}\right)  $ and $\left(  -\sqrt{n/2},-\sqrt{n/2}\right)  $).
\qedhere

\end{itemize}
\end{proof}

Note that sums of squares have a geometric meaning (going back to Pythagoras):
Two real numbers $a$ and $b$ satisfy $a^{2}+b^{2}=n$ (for a given integer
$n\geq0$) if and only if the point with Cartesian coordinates $\left(
a,b\right)  $ lies on the circle with center $0$ and radius $\sqrt{n}$. This
will actually prove a valuable insight that will lead us to the answers to the
above questions.

Just as a teaser: There are formulas for all three parts of
Question~\ref{quest.intro.sum-of-2sq.2}, in terms of divisors of $n$ of the
forms $4k+1$ and $4k+3$. We will see these formulas after we have properly
understood the concept of Gaussian integers.

\subsection{\label{subsect.intro.algnum}Motivation: Algebraic numbers}

\begin{noncompile}
Recall how the number system was constructed. In a way, each extension was
done in order to allow a certain operation to proceed: The natural numbers
were extended to the integers in order to allow subtraction (in all cases, not
just when we are subtracting a smaller number from a larger). Then, the
integers were extended to the rational numbers in order to allow division (in
all reasonable cases\footnote{``Reasonable'' in this case means that division
by $0$ is still forbidden. If we allowed division by $0$ as well, then our
``rational numbers'' would all be equal to each other and therefore a huge
step back from the integers.}, not just when the division works out
remainder-less). Then, the rational numbers were extended to the real numbers
in order to allow limits (in all reasonable cases). Finally, the real numbers
were (or will be -- we will see this in more detail) extended to the complex
numbers in order to allow square roots.

From an algebraic point of view, the step from the rational numbers to the
real numbers is somewhat of an overkill. Algebraists often want to work with
roots, particularly roots of polynomials; ideally, every polynomial of degree
$n$ should have ``all'' $n$ roots (counted with multiplicity), so it can be
factored into linear factors. This does indeed happen once you get to complex
numbers (the so-called ``Fundamental Theorem of Algebra''), but the road there
is bumpy and non-algebraic (at the very least, you need continuity to prove
the ``Fundamental Theorem of Algebra''). So algebraists have wondered whether
there is a cheaper way to buy roots for their polynomials -- without having to
pay the price of analysis. (The question became even more relevant when they
started working over arbitrary fields and even commutative rings -- in a
sense, ``alternative number systems'' in which analysis won't help you.)

The answer is ``yes'', and we will eventually see how. But for now, let me
focus on a simple problem that is already interesting if one works inside the
real numbers.
\end{noncompile}

A real number $z$ is said to be \textit{algebraic} if there exists a nonzero
polynomial $P$ with rational coefficients such that $P\left(  z \right)  = 0$.
In other words, a real number $z$ is algebraic if and only if it is a root of
a nonzero polynomial with rational coefficients.

(If you know the complex numbers, you can replace ``real'' by ``complex'' in
this definition; but we shall only see real numbers in this little
motivational subsection.)

Examples:

\begin{itemize}
\item Each rational number $a$ is algebraic (being a root of the nonzero
polynomial $x-a$ with rational coefficients).

\item The number $\sqrt{2}$ is algebraic (being a root of the nonzero
polynomial $x^{2}-2$).

\item The number $\sqrt[3]{5}$ is algebraic (being a root of $x^{3}-5$).

\item All the roots of the polynomial $f\left(  x \right)  := \dfrac{3}%
{2}x^{4}+17x^{3}-12x+\dfrac{9}{4}$ (whatever they are) are algebraic. \newline
Speaking of these roots, what are they? Using a computer, one can show that
this polynomial $f\left(  x \right)  $ has $4$ real roots ($-11.269\ldots,
-0.960\ldots, 0.198\ldots, 0.697\ldots$), which can be written as complicated
expressions with radicals (i.e., $\sqrt[k]{}$ signs), though complex numbers
appear in these expressions (despite the roots being real!). All this does not
matter to the fact that they are algebraic :)

\item All the roots of the polynomial $g\left(  x \right)  := x^{7} - x^{5} +
1$ are algebraic. \newline This polynomial has only one real root. This root
cannot be written as an expression with radicals (as can be proven using
\href{https://en.wikipedia.org/wiki/Galois_theory}{Galois theory} -- indeed,
the discovery of this theory greatly motivated the development of abstract
algebra).
%(Nor can the remaining $6$ complex roots be.)
Nevertheless, it is algebraic, by definition. (The same holds for the
remaining $6$ complex roots of $g$ -- we are working with real numbers here
only for the sake of familiarity.)

\item The most famous number that is not algebraic is $\pi$. This is a famous
result of Lindemann, but it belongs to analysis, not to algebra, because $\pi$
is not defined algebraically in the first place (it is defined as the length
of a curve or as an area of a curved region -- but either of these definitions
boils down to a limit of a sequence).

\item The second most famous number that is not algebraic is
\href{https://en.wikipedia.org/wiki/E_(mathematical_constant)}{Euler's number
$e$} (the basis of the natural logarithm). Again, analysis is needed to define
$e$, and thus also to prove its non-algebraicity.
\end{itemize}

Numbers that are not algebraic are called
\href{https://en.wikipedia.org/wiki/Transcendental_number}{\textit{transcendental}%
}. We shall not study them much, since most of them do not come from algebra.
Instead, we shall try our hands at the following question:

\begin{question}
\label{quest.intro.algnum.1} \textbf{(a)} Is the sum of two (or, more
generally, finitely many) algebraic numbers always algebraic?

\textbf{(b)} What if we replace ``sum'' by ``difference'' or ``product''?
\end{question}

Let me motivate why this is a natural question to ask. The sum of two integers
is still an integer; the sum of two rational numbers is still a rational
number. These facts are fundamental; without them we could hardly work with
integers and rational numbers. If a similar fact would not hold for algebraic
numbers, it would mean that the algebraic numbers are not a good ``number
system'' to work in; on a practical level, it would mean that (e.g.) if we
defined a function on the set of all algebraic numbers, then we could not plug
a sum of algebraic numbers into it.

\begin{proof}
[Attempts at answering Question~\ref{quest.intro.algnum.1} \textbf{(a)}.]Let
us try a particularly simple example of a sum of two algebraic numbers: Let
$w$ be $\sqrt{2} + \sqrt{3}$. Is $w$ algebraic?

To answer this question affirmatively, we need to find a nonzero polynomial
$f\left(  x \right)  $ with rational coefficients that has $w$ as a root.

Just looking at the equality $w = \sqrt{2} + \sqrt{3}$, we cannot directly
eyeball such an $f$. The problem, in a sense, is that there are too many
(namely, two) square roots in this equality.

However, if we square this equality, then we obtain
\[
w^{2}=\left(  \sqrt{2}+\sqrt{3}\right)  ^{2}=2+2\sqrt{2}\cdot\sqrt{3}+3
=5+2\sqrt{6},
\]
which is an equality with only one square root (a sign of progress).
Subtracting $5$ from this equality (in order to ``isolate'' this remaining
square root), we obtain $w^{2}-5=2\sqrt{6}$. If we now square this equality,
then we obtain $\left(  w^{2}-5\right)  ^{2}=\left(  2\sqrt{6}\right)
^{2}=24$. At this point all square roots are gone, and we are left with an
equality that contains rational numbers and $w$ only! We can further rewrite
it as $\left(  w^{2} - 5 \right)  ^{2} - 24 = 0$. Thus, $w$ is a root of the
polynomial $f\left(  x \right)  := \left(  x^{2}-5\right)  ^{2}-24 =
x^{4}-10x^{2}+1$. This means that $w$ is algebraic (since $f$ is nonzero).

Let us try a more complicated example: Let $z$ be the number $\sqrt
{2}+\sqrt[3]{2}$. Is $z$ algebraic? The squaring trick no longer works, since
squaring $\sqrt{2}+\sqrt[3]{2}$ does not reduce the number of radicals (= root
signs). Let's instead try rewriting $z=\sqrt{2}+\sqrt[3]{2}$ as $z-\sqrt
{2}=\sqrt[3]{2}$. Cubing this equality, we obtain $\left(  z-\sqrt{2}\right)
^{3}=2$. In view of
\[
\left(  z-\sqrt{2}\right)  ^{3}=z^{3}-3z^{2}\sqrt{2}+3z\left(  \sqrt
{2}\right)  ^{2}-\left(  \sqrt{2}\right)  ^{3}%
\]
(this is a particular case of the identity $\left(  a-b\right)  ^{3}%
=a^{3}-3a^{2}b+3ab^{2}-b^{3}$, which is one form of the Binomial Theorem for
exponent $3$), this rewrites a
\[
z^{3}-3z^{2}\sqrt{2}+3z\left(  \sqrt{2}\right)  ^{2}-\left(  \sqrt{2}\right)
^{3}=2.
\]
This simplifies to%
\[
z^{3}-3\sqrt{2}z^{2}+6z-2\sqrt{2}=2.
\]
Let us transform this inequality in such a way that all terms with a $\sqrt
{2}$ in them end up on the right hand side while all the remaining terms end
up on the left. We thus obtain
\[
z^{3}+6z-2=\sqrt{2}\left(  3z^{2}+2\right)  .
\]
Now, squaring this equality yields
\[
\left(  z^{3}+6z-2\right)  ^{2}=2\left(  3z^{2}+2\right)  ^{2}.
\]
Hence, $z$ is a root of the polynomial
\[
g\left(  x\right)  :=\left(  x^{3}+6x-2\right)  ^{2}-2\left(  3x^{2}+2\right)
^{2}=x^{6}-6x^{4}-4x^{3}+12x^{2}-24x-4.
\]
This is a nonzero polynomial with rational coefficients; hence, $z$ is algebraic.

We thus have verified that the sum of two algebraic numbers is algebraic in
two cases. What about more complicated cases, such as
\[
\sqrt{2}+\sqrt{3}+\sqrt[7]{11}\text{ ?}%
\]
This is a sum of two algebraic numbers (since we already know that $\sqrt
{2}+\sqrt{3}=w$ is algebraic). Is it algebraic? Neither of our above two
methods properly works here; do we have to come up with new ad-hoc tricks?
\end{proof}

\begin{center}
\textbf{2019-01-25 lecture}
\end{center}

\subsection{Motivation: Shamir's Secret Sharing Scheme}

\subsubsection{The problem}

Adi Shamir is one of the founders of modern mathematical cryptography (famous
in particular for \href{https://en.wikipedia.org/wiki/RSA_(cryptosystem)}{the
RSA cryptosystem}, see later).

Shamir's Secret Sharing Scheme is a way in which a secret $\mathbf{a}$ (a
piece of data -- e.g., nuclear launch codes) can be distributed among $n$
people in such a way that

\begin{itemize}
\item any $k$ of them can (if they come together) reconstruct it uniquely, but

\item any $k-1$ of them (if they come together) cannot gain \textbf{any}
insight about it (i.e., not only cannot they reconstruct it, but they cannot
even tell that some values are more likely than others to be $\mathbf{a}$).
\end{itemize}

Here $n$ and $k$ are fixed positive integers.

Understanding this scheme completely will require some abstract algebra, but
we can already start thinking about the problem and get reasonably far.

So we have $n$ people $1,2,\ldots,n$, a positive integer $k\in\left\{
1,2,\ldots,n\right\}  $ and a secret piece of data $\mathbf{a}$. We assume
that this data $\mathbf{a}$ is encoded as a \textit{bitstring} -- i.e., a
finite sequence of bits. A \textit{bit} is an element of the set $\left\{
0,1\right\}  $. Thus, examples of bitstrings are $\left(  0,1,1,0\right)  $
and $\left(  1,0\right)  $ and $\left(  1,1,0,1,0,0,0\right)  $ as well as the
empty sequence $\left(  {}\right)  $. When writing bitstring, we shall usually
omit both the commas and the parentheses; thus, e.g., the bitstring $\left(
1,1,0,1,0,0,0\right)  $ will become $1101000$. Make sure you don't mistake it
for a number. Our goal is to give each of the $n$ people $1,2,\ldots,n$ some
bitstring in such a way that:

\begin{itemize}
\item \textit{Requirement 1:} Any $k$ of the $n$ people can (if they come
together) reconstruct $\mathbf{a}$ uniquely.

\item \textit{Requirement 2:} Any $k-1$ of the $n$ people are unable to gain
any insight about $\mathbf{a}$ (even if they collaborate).
\end{itemize}

We denote the bitstrings given to the people $1,2,\ldots,n$ by $\mathbf{a}%
_{1},\mathbf{a}_{2},\ldots,\mathbf{a}_{n}$, respectively.

We assume that the length of our secret bitstring $\mathbf{a}$ is known in
advance to all parties; i.e., it is not a secret. Thus, when we say
\textquotedblleft$k-1$ persons cannot gain any insight about $\mathbf{a}%
$\textquotedblright, we do not mean that they don't know the length; and when
we say \textquotedblleft some values are more likely than others to be
$\mathbf{a}$\textquotedblright, we only mean values that fit this length.

\subsubsection{The $k=1$ case}

One simple special case of our problem is when $k=1$. In this case, it
suffices to give each of the $n$ people the full secret $\mathbf{a}$ (that is,
we set $\mathbf{a}_{i}=\mathbf{a}$ for all $i$). Then, Requirement 1 is
satisfied (since any $1$ of the $n$ people already knows $\mathbf{a}$), while
Requirement 2 is satisfied as well ($0$ people know nothing).

\subsubsection{The $k=n$ case: what doesn't work}

Let us now consider the case when $k=n$. This case will not help us solve the
general problem, but it will show some ideas that we will encounter again and
again in abstract algebra.

We want to ensure that all $n$ people needed to reconstruct the secret
$\mathbf{a}$, while any $n-1$ of them will be completely clueless.

It sounds reasonable to split $\mathbf{a}$ into $n$ parts, and give each
person one of these parts\footnote{assuming that $\mathbf{a}$ is long enough
for that} (i.e., we let $\mathbf{a}_{i}$ be the $i$-th part of $\mathbf{a}$
for each $i\in\left\{  1,2,\ldots,n\right\}  $). This method satisfies
Requirement 1 (indeed, all $n$ people together can reconstruct $\mathbf{a}$
simply by fusing the $n$ parts back together), but fails Requirement 2
(indeed, any $n-1$ people know $n-1$ parts of the secret $\mathbf{a}$, which
is a far from being clueless about $\mathbf{a}$). So this method doesn't work.
It is not that easy.

\subsubsection{The $\operatorname*{XOR}$ operations}

One way to solve the $k=n$ case is using the $\operatorname*{XOR}$ operation.

Let us first define some basic language. A \textit{binary operation} on a set
$S$ is (informally speaking) a function that takes two elements of $S$ and
assigns a new element of $S$ to them. More formally:

\begin{definition}
A \textit{binary operation} on a set $S$ is a map $f$ from $S\times S$ to $S$.
When $f$ is a binary operation on $S$ and $a$ and $b$ are two elements of $S$,
we shall write $afb$ for the value $f\left(  a,b\right)  $.
\end{definition}

\begin{example}
Addition, subtraction and multiplication of integers are three binary
operations on the set $\mathbb{Q}$ (the set of all rational numbers). For
example, addition is the map from $\mathbb{Q}\times\mathbb{Q}$ to $\mathbb{Q}$
that sends each pair $\left(  a,b\right)  \in\mathbb{Q}\times\mathbb{Q}$ to
$a+b$.

Division is not a binary operation on the set $\mathbb{Q}$. Indeed, if it was,
then it would send the pair $\left(  1,0\right)  $ to some integer called
$1/0$; but there is no such integer.

There are myriad more complicated binary operations around waiting for someone
to name them. For example, you could define a binary operation $\smiley{}$ on
the set $\mathbb{Q}$ by $a\smiley{}b=\dfrac{a-b}{1+a^{2}+b^{2}}$. Indeed, you
can do this because $1+a^{2}+b^{2}$ is always nonzero when $a,b\in\mathbb{Q}$
(after all, squares are nonnegative, so that $1+\underbrace{a^{2}}_{\geq
0}+\underbrace{b^{2}}_{\geq0}\geq1>0$). I am not saying that you should...
\end{example}

Now, we define some specific binary operations on the set $\left\{
0,1\right\}  $ of all bits, and on the set $\left\{  0,1\right\}  ^{n}$ of all
length-$n$ bitstrings (for a given $n$).

\begin{definition}
We define a binary operation $\operatorname*{XOR}$ on the set $\left\{
0,1\right\}  $ by setting%
\begin{align*}
0\operatorname*{XOR}0  &  =0,\\
0\operatorname*{XOR}1  &  =1,\\
1\operatorname*{XOR}0  &  =1,\\
1\operatorname*{XOR}1  &  =0.
\end{align*}
This is a valid definition, because there are only four pairs $\left(
a,b\right)  \in\left\{  0,1\right\}  \times\left\{  0,1\right\}  $, and we
have just defined $a\operatorname*{XOR}b$ for each of these four options. We
can also rewrite this definition as follows:%
\[
a\operatorname*{XOR}b=%
\begin{cases}
1, & \text{if }a\neq b;\\
0, & \text{if }a=b
\end{cases}
=%
\begin{cases}
1, & \text{if \textbf{exactly} one of }a\text{ and }b\text{ is }1;\\
0, & \text{otherwise.}%
\end{cases}
\]
For lack of a better name, we refer to $a\operatorname*{XOR}b$ as the
\textquotedblleft XOR of $a$ and $b$\textquotedblright.
\end{definition}

The name \textquotedblleft$\operatorname*{XOR}$\textquotedblright\ is short
for \textquotedblleft exclusive or\textquotedblright. In fact, if you identify
bits with boolean truth values (so the bit $0$ stands for \textquotedblleft
False\textquotedblright\ and the bit $1$ stands for \textquotedblleft
True\textquotedblright), then $a\operatorname*{XOR}b$ is precisely the truth
value for \textquotedblleft exactly one of $a$ and $b$ is
True\textquotedblright, which is also known as \textquotedblleft$a$
exclusive-or $b$\textquotedblright.

\begin{definition}
Let $m$ be a nonnegative integer. We define a binary operation
$\operatorname*{XOR}$ on the set $\left\{  0,1\right\}  ^{m}$ (this is the set
of all length-$m$ bitstrings) by%
\[
\left(  a_{1},a_{2},\ldots,a_{m}\right)  \operatorname*{XOR}\left(
b_{1},b_{2},\ldots,b_{m}\right)  =\left(  a_{1}\operatorname*{XOR}b_{1}%
,a_{2}\operatorname*{XOR}b_{2},\ldots,a_{m}\operatorname*{XOR}b_{m}\right)  .
\]
In other words, if $\mathbf{a}$ and $\mathbf{b}$ are two length-$m$
bitstrings, then $\mathbf{a}\operatorname*{XOR}\mathbf{b}$ is obtained by
taking the XOR of each entry of $\mathbf{a}$ with the corresponding entry of
$\mathbf{b}$, and packing these $m$ XORs into a new length-$m$ bitstring.
\end{definition}

For example,%
\begin{align*}
\left(  1001\right)  \operatorname*{XOR}\left(  1100\right)   &  =0101;\\
\left(  11011\right)  \operatorname*{XOR}\left(  10101\right)   &  =01110;\\
\left(  11010\right)  \operatorname*{XOR}\left(  01011\right)   &  =10001;\\
\left(  1\right)  \operatorname*{XOR}\left(  0\right)   &  =1;\\
\left(  {}\right)  \operatorname*{XOR}\left(  {}\right)   &  =\left(
{}\right)  .
\end{align*}


Note that if $\mathbf{a}$ and $\mathbf{b}$ are two length-$m$ bitstrings, then
the $0$'s in the bitstring $\mathbf{a}\operatorname*{XOR}\mathbf{b}$ are at
the positions where $\mathbf{a}$ and $\mathbf{b}$ have equal entries, and the
$1$'s in $\mathbf{a}\operatorname*{XOR}\mathbf{b}$ are at the positions where
$\mathbf{a}$ and $\mathbf{b}$ have different entries. Thus, the operation
$\operatorname*{XOR}$ on bitstring essentially pinpoints the differences
between $\mathbf{a}$ and $\mathbf{b}$.

We observe the following simple properties of these operations
$\operatorname*{XOR}$ on bits and on bitstrings\footnote{As a mnemonic, we
shall try to use boldfaced letters like $\mathbf{a}$ and $\mathbf{b}$ for
bitstrings and regular italic letters like $a$ and $b$ for single bits.}:

\begin{itemize}
\item We have $a\operatorname*{XOR}0=a$ for any bit $a$. (This can be
trivially checked by considering both possibilities for $a$.)

\item Thus, $\mathbf{a}\operatorname*{XOR}\mathbf{0}=\mathbf{a}$ for any
bitstring $\mathbf{a}$, where $\mathbf{0}$ denotes the bitstring
$00\cdots0=\left(  0,0,\ldots,0\right)  $ (of appropriate length -- i.e., of
the same length as $\mathbf{a}$).

\item We have $a\operatorname*{XOR}a=0$ for any bit $a$. (This can be
trivially checked by considering both possibilities for $a$.)

\item Thus, $\mathbf{a}\operatorname*{XOR}\mathbf{a}=\mathbf{0}$ for any
bitstring $\mathbf{a}$. We shall refer to this as the
\textit{self-cancellation law}.

\item We have $a\operatorname*{XOR}b=b\operatorname*{XOR}a$ for any bits
$a,b$. (Again, this is easy to check by going through all four options for $a$
and $b$.)

\item Thus, $\mathbf{a}\operatorname*{XOR}\mathbf{b}=\mathbf{b}%
\operatorname*{XOR}\mathbf{a}$ for any bitstrings $\mathbf{a},\mathbf{b}$.

\item We have $a\operatorname*{XOR}\left(  b\operatorname*{XOR}c\right)
=\left(  a\operatorname*{XOR}b\right)  \operatorname*{XOR}c$ for any bits
$a,b,c$. (Again, this is easy to check by going through all eight options for
$a,b,c$.)

\item Thus, $\mathbf{a}\operatorname*{XOR}\left(  \mathbf{b}%
\operatorname*{XOR}\mathbf{c}\right)  =\left(  \mathbf{a}\operatorname*{XOR}%
\mathbf{b}\right)  \operatorname*{XOR}\mathbf{c}$ for any bitstrings
$\mathbf{a},\mathbf{b},\mathbf{c}$.

\item Thus, for any bitstrings $\mathbf{a}$ and $\mathbf{b}$, we have%
\[
\left(  \mathbf{a}\operatorname*{XOR}\mathbf{b}\right)  \operatorname*{XOR}%
\mathbf{b}=\mathbf{a}\operatorname*{XOR}\underbrace{\left(  \mathbf{b}%
\operatorname*{XOR}\mathbf{b}\right)  }_{\substack{=\mathbf{0}\\\text{(by the
self-cancellation law)}}}=\mathbf{a}\operatorname*{XOR}\mathbf{0}=\mathbf{a}.
\]


This observation gives rise to a primitive cryptosystem (known as a
\textit{\href{https://en.wikipedia.org/wiki/One-time_pad}{\textit{one-time
pad}}}): If you have a secret bitstring $\mathbf{a}$ that you want to encrypt,
and another secret bitstring $\mathbf{b}$ that can be used as a key, then you
can encrypt\ $\mathbf{a}$ by XORing it with $\mathbf{b}$ (that is, you
transform it into $\mathbf{a}\operatorname*{XOR}\mathbf{b}$). Then, you can
decrypt it again by XORing it with $\mathbf{b}$ again; indeed, if you do this,
you will obtain $\left(  \mathbf{a}\operatorname*{XOR}\mathbf{b}\right)
\operatorname*{XOR}\mathbf{b}=\mathbf{a}$. This is a highly safe cryptosystem
as long as you can safely communicate the key $\mathbf{b}$ to whomever needs
to be able to decrypt (or encrypt) your secrets, and as long as you are able
to generate uniformly random keys $\mathbf{b}$ of sufficient length. Its only
weakness is its impracticality (in many situations): If the secret you want to
encrypt is long (say, a whole book), your key will need to be equally long.
Even storing such keys can become difficult.
\end{itemize}

We shall refer to the properties $a\operatorname*{XOR}b=b\operatorname*{XOR}a$
and $\mathbf{a}\operatorname*{XOR}\mathbf{b}=\mathbf{b}\operatorname*{XOR}%
\mathbf{a}$ as \textit{laws of commutativity}, and we shall refer to the
properties $a\operatorname*{XOR}\left(  b\operatorname*{XOR}c\right)  =\left(
a\operatorname*{XOR}b\right)  \operatorname*{XOR}c$ and $\mathbf{a}%
\operatorname*{XOR}\left(  \mathbf{b}\operatorname*{XOR}\mathbf{c}\right)
=\left(  \mathbf{a}\operatorname*{XOR}\mathbf{b}\right)  \operatorname*{XOR}%
\mathbf{c}$ as \textit{laws of associativity}. These are, of course, similar
to well-known facts like $\alpha+\beta=\beta+\alpha$ and $\alpha+\left(
\beta+\gamma\right)  =\left(  \alpha+\beta\right)  +\gamma$ for numbers
$\alpha,\beta,\gamma$ (which is why we are giving them the same name). This
similarity is not coincidental. Just as for addition or multiplication of
numbers, these laws lead to a notion of \textquotedblleft
XOR-products\textquotedblright:

\begin{proposition}
\label{prop.intro.xor.prodm}Let $m$ be a positive integer. Let $\mathbf{a}%
_{1},\mathbf{a}_{2},\ldots,\mathbf{a}_{m}$ be $m$ bitstrings. Then, the
\textquotedblleft$\operatorname*{XOR}$-product\textquotedblright\ expression%
\[
\mathbf{a}_{1}\operatorname*{XOR}\mathbf{a}_{2}\operatorname*{XOR}%
\mathbf{a}_{3}\operatorname*{XOR}\cdots\operatorname*{XOR}\mathbf{a}_{m}%
\]
is well-defined, in the sense that it does not depend on the parenthesization.
\end{proposition}

What do we mean by \textquotedblleft parenthesization\textquotedblright? To
clarify things, let us set $m=4$. In this case, we want to make sense of the
expression $\mathbf{a}_{1}\operatorname*{XOR}\mathbf{a}_{2}\operatorname*{XOR}%
\mathbf{a}_{3}\operatorname*{XOR}\mathbf{a}_{4}$. This expression does not
make sense a priori, since it is a $\operatorname*{XOR}$ of \textbf{four}
bitstrings, whereas we have defined only the $\operatorname*{XOR}$ of
\textbf{two} bitstrings. But there are five ways to put parentheses around
some of its sub-expressions such that the expression becomes meaningful:
\begin{align*}
&  \left(  \mathbf{a}_{1}\operatorname*{XOR}\mathbf{a}_{2}\right)
\operatorname*{XOR}\left(  \mathbf{a}_{3}\operatorname*{XOR}\mathbf{a}%
_{4}\right)  ,\\
&  \left(  \left(  \mathbf{a}_{1}\operatorname*{XOR}\mathbf{a}_{2}\right)
\operatorname*{XOR}\mathbf{a}_{3}\right)  \operatorname*{XOR}\mathbf{a}_{4},\\
&  \mathbf{a}_{1}\operatorname*{XOR}\left(  \left(  \mathbf{a}_{2}%
\operatorname*{XOR}\mathbf{a}_{3}\right)  \operatorname*{XOR}\mathbf{a}%
_{4}\right)  ,\\
&  \mathbf{a}_{1}\operatorname*{XOR}\left(  \mathbf{a}_{2}\operatorname*{XOR}%
\left(  \mathbf{a}_{3}\operatorname*{XOR}\mathbf{a}_{4}\right)  \right)  ,\\
&  \left(  \mathbf{a}_{1}\operatorname*{XOR}\left(  \mathbf{a}_{2}%
\operatorname*{XOR}\mathbf{a}_{3}\right)  \right)  \operatorname*{XOR}%
\mathbf{a}_{4}.
\end{align*}
Each of these five parenthesizations (= placements of parentheses) turns our
expression $\mathbf{a}_{1}\operatorname*{XOR}\mathbf{a}_{2}\operatorname*{XOR}%
\mathbf{a}_{3}\operatorname*{XOR}\mathbf{a}_{4}$ into a combination of
$\operatorname*{XOR}$'s of \textbf{two} bitstrings each, and thus gives it
meaning. The question is: Do these five parenthesizations give it the
\textbf{same} meaning?

Well, let us calculate:%
\begin{align*}
&  \left(  \mathbf{a}_{1}\operatorname*{XOR}\mathbf{a}_{2}\right)
\operatorname*{XOR}\left(  \mathbf{a}_{3}\operatorname*{XOR}\mathbf{a}%
_{4}\right) \\
&  =\mathbf{a}_{1}\operatorname*{XOR}\underbrace{\left(  \mathbf{a}%
_{2}\operatorname*{XOR}\left(  \mathbf{a}_{3}\operatorname*{XOR}\mathbf{a}%
_{4}\right)  \right)  }_{=\left(  \mathbf{a}_{2}\operatorname*{XOR}%
\mathbf{a}_{3}\right)  \operatorname*{XOR}\mathbf{a}_{4}}\\
&  =\mathbf{a}_{1}\operatorname*{XOR}\left(  \left(  \mathbf{a}_{2}%
\operatorname*{XOR}\mathbf{a}_{3}\right)  \operatorname*{XOR}\mathbf{a}%
_{4}\right) \\
&  =\underbrace{\left(  \mathbf{a}_{1}\operatorname*{XOR}\left(
\mathbf{a}_{2}\operatorname*{XOR}\mathbf{a}_{3}\right)  \right)  }_{=\left(
\mathbf{a}_{1}\operatorname*{XOR}\mathbf{a}_{2}\right)  \operatorname*{XOR}%
\mathbf{a}_{3}}\operatorname*{XOR}\mathbf{a}_{4}\\
&  =\left(  \left(  \mathbf{a}_{1}\operatorname*{XOR}\mathbf{a}_{2}\right)
\operatorname*{XOR}\mathbf{a}_{3}\right)  \operatorname*{XOR}\mathbf{a}_{4},
\end{align*}
where we used the law of associativity in each step. This shows that our five
parenthesizations yield the same result. Thus, they all give our
\textquotedblleft$\operatorname*{XOR}$-product\textquotedblright\ expression
$\mathbf{a}_{1}\operatorname*{XOR}\mathbf{a}_{2}\operatorname*{XOR}%
\mathbf{a}_{3}\operatorname*{XOR}\mathbf{a}_{4}$ the same meaning; so we can
say that this expression is well-defined. This confirms Proposition
\ref{prop.intro.xor.prodm} for $m=4$.

Of course, proving Proposition \ref{prop.intro.xor.prodm} is less simple. Such
a proof will appear in Exercise 4 on homework set \#0.

\subsubsection{The $k=n$ case: an answer}

Let us now return to our problem. We have $n$ persons $1,2,\ldots,n$ and a
secret $\mathbf{a}$ (encoded as a bitstring). We want to give each person $i$
some bitstring $\mathbf{a}_{i}$ such that only all $n$ of them can recover
$\mathbf{a}$ but any $n-1$ of them cannot gain any insight about $\mathbf{a}$.

We let $\mathbf{a}_{1},\mathbf{a}_{2},\ldots,\mathbf{a}_{n-1}$ be $n-1$
\textbf{uniformly} random bitstrings of the same length as $\mathbf{a}$.
(Think of them as random gibberish.) Set%
\[
\mathbf{a}_{n}=\mathbf{a}\operatorname*{XOR}\mathbf{a}_{1}\operatorname*{XOR}%
\mathbf{a}_{2}\operatorname*{XOR}\cdots\operatorname*{XOR}\mathbf{a}_{n-1}.
\]
(This expression makes sense because of Proposition \ref{prop.intro.xor.prodm}.)

Then,%
\begin{align*}
&  \mathbf{a}_{n}\operatorname*{XOR}\mathbf{a}_{n-1}\operatorname*{XOR}%
\mathbf{a}_{n-2}\operatorname*{XOR}\cdots\operatorname*{XOR}\mathbf{a}_{1}\\
&  =\left(  \mathbf{a}\operatorname*{XOR}\mathbf{a}_{1}\operatorname*{XOR}%
\mathbf{a}_{2}\operatorname*{XOR}\cdots\operatorname*{XOR}\mathbf{a}%
_{n-1}\right)  \operatorname*{XOR}\mathbf{a}_{n-1}\operatorname*{XOR}%
\mathbf{a}_{n-2}\operatorname*{XOR}\cdots\operatorname*{XOR}\mathbf{a}_{1}\\
&  =\mathbf{a}\operatorname*{XOR}\mathbf{a}_{1}\operatorname*{XOR}%
\mathbf{a}_{2}\operatorname*{XOR}\cdots\operatorname*{XOR}%
\underbrace{\mathbf{a}_{n-1}\operatorname*{XOR}\mathbf{a}_{n-1}}_{=\mathbf{0}%
}\operatorname*{XOR}\mathbf{a}_{n-2}\operatorname*{XOR}\cdots
\operatorname*{XOR}\mathbf{a}_{1}\\
&  =\mathbf{a}\operatorname*{XOR}\mathbf{a}_{1}\operatorname*{XOR}%
\mathbf{a}_{2}\operatorname*{XOR}\cdots\operatorname*{XOR}%
\underbrace{\mathbf{a}_{n-2}\operatorname*{XOR}\mathbf{0}}_{=\mathbf{a}_{n-2}%
}\operatorname*{XOR}\mathbf{a}_{n-2}\operatorname*{XOR}\cdots
\operatorname*{XOR}\mathbf{a}_{1}\\
&  =\mathbf{a}\operatorname*{XOR}\mathbf{a}_{1}\operatorname*{XOR}%
\mathbf{a}_{2}\operatorname*{XOR}\cdots\operatorname*{XOR}%
\underbrace{\mathbf{a}_{n-2}\operatorname*{XOR}\mathbf{a}_{n-2}}_{=\mathbf{0}%
}\operatorname*{XOR}\cdots\operatorname*{XOR}\mathbf{a}_{1}\\
&  =\cdots\\
&  =\mathbf{a}%
\end{align*}
(here, we have been unravelling the big $\operatorname*{XOR}$-product from the
middle on, by cancelling equal bitstrings using the self-cancellation law and
then removing the resulting $\mathbf{0}$ using the $\mathbf{a}%
\operatorname*{XOR}\mathbf{0}=\mathbf{a}$ law). Hence, the $n$ people together
can decrypt the secret $\mathbf{a}$.

Can $n-1$ people gain any insight about it? The $n-1$ people $1,2,\ldots,n-1$
certainly cannot, since all they know are the random bitstrings $\mathbf{a}%
_{1},\mathbf{a}_{2},\ldots,\mathbf{a}_{n-1}$. But the $n-1$ people
$2,3,\ldots,n$ cannot gain any insight about $\mathbf{a}$ either: In fact, all
they know are the random bitstrings $\mathbf{a}_{2},\mathbf{a}_{3}%
,\ldots,\mathbf{a}_{n-1}$ and the bitstring%
\[
\mathbf{a}_{n}=\mathbf{a}\operatorname*{XOR}\mathbf{a}_{1}\operatorname*{XOR}%
\mathbf{a}_{2}\operatorname*{XOR}\cdots\operatorname*{XOR}\mathbf{a}_{n-1};
\]
therefore, all the information they have about $\mathbf{a}$ and $\mathbf{a}%
_{1}$ comes to them through $\mathbf{a}\operatorname*{XOR}\mathbf{a}_{1}$,
which says nothing about $\mathbf{a}$ as long as they know nothing about
$\mathbf{a}_{1}$. (We used a bit of handwaving in this argument, but then
again we never formally defined what it means to \textquotedblleft gain no
insight\textquotedblright; this is done in courses on cryptography and
information theory.) Similar arguments show that any other choice of $n-1$
persons remains equally clueless about $\mathbf{a}$. So we have solved the
problem in the case $k=n$.

\subsubsection{The $k=2$ case}

The next simple case is when $k=2$. So we want to ensure that any $2$ of our
$n$ people can together recover the secret, but no $1$ person can learn
anything about it alone.

A really nice approach was suggested by Nathan in class: We pick $n$ random
bitstrings $\mathbf{x}_{1},\mathbf{x}_{2},\ldots,\mathbf{x}_{n-1}$ of the same
length as $\mathbf{a}$. Set
\[
\mathbf{x}_{n}=\mathbf{a}\operatorname*{XOR}\mathbf{x}_{1}\operatorname*{XOR}%
\mathbf{x}_{2}\operatorname*{XOR}\cdots\operatorname*{XOR}\mathbf{x}_{n-1};
\]
thus, as in the $k=n$ case, we have%
\begin{equation}
\mathbf{x}_{n}\operatorname*{XOR}\mathbf{x}_{n-1}\operatorname*{XOR}%
\mathbf{x}_{n-2}\operatorname*{XOR}\cdots\operatorname*{XOR}\mathbf{x}%
_{1}=\mathbf{a}. \label{eq.intro.shamir.k=2.2}%
\end{equation}


Each person $i$ now receives the bitstring%
\[
\mathbf{a}_{i}=\mathbf{x}_{1}\mathbf{x}_{2}\cdots\mathbf{x}_{i-1}%
\mathbf{x}_{i+1}\mathbf{x}_{i+2}\cdots\mathbf{x}_{n},
\]
where the product stands for \textit{concatenation} (i.e., the bitstring
$\mathbf{a}_{i}$ is formed by writing down all of the bitstring $\mathbf{x}%
_{1},\mathbf{x}_{2},\ldots,\mathbf{x}_{n}$ one after the other but skipping
$\mathbf{x}_{i}$). Thus, each person $i$ can recover all the $n-1$ bitstrings
$\mathbf{x}_{1},\mathbf{x}_{2},\ldots,\mathbf{x}_{i-1},\mathbf{x}%
_{i+1},\mathbf{x}_{i+2},\ldots,\mathbf{x}_{n}$ (because their lengths are the
length of $\mathbf{a}$, which is known), but knows nothing about
$\mathbf{x}_{i}$ (his \textquotedblleft blind spot\textquotedblright). Hence,
$2$ people together can recover all the $n$ bitstrings $\mathbf{x}%
_{1},\mathbf{x}_{2},\ldots,\mathbf{x}_{n}$ and therefore recover the secret
$\mathbf{a}$ (by (\ref{eq.intro.shamir.k=2.2})). On the other hand, each
single person has no insight about $\mathbf{a}$ (this is proven similarly to
the $k=n$ case). So again, the problem is solved in this case.

\subsubsection{The $k=3$ case}

Now, let us come to the case when $k=3$. Now I think the usefulness of the
$\operatorname*{XOR}$ approach has come to its end: at least I don't know how
to make it work here. Instead, out of the blue, I will invoke something
completely different: polynomials (let's say with rational coefficients).

Recall a fact you might have heard in high school: A polynomial $p\left(
x\right)  =cx^{2}+bx+a$ of degree $\leq2$ is uniquely determined by any three
of its values. More precisely: If $u,v,w$ are three fixed distinct numbers,
then a polynomial $p\left(  x\right)  =cx^{2}+bx+a$ of degree $\leq2$ is
uniquely determined by the values $p\left(  u\right)  ,p\left(  v\right)
,p\left(  w\right)  $. We will put this to use now, and sort-of solve the problem.

Also recall that any bitstring of given length $N$ can be encoded as an
integer in $\left\{  0,1,\ldots,2^{N}-1\right\}  $; just read it as a number
in binary. More precisely, any bitstring $a_{N-1}a_{N-2}\cdots a_{0}$ of
length $N$ becomes the integer $a_{N-1}\cdot2^{N-1}+a_{N-2}\cdot2^{N-2}%
+\cdots+a_{0}\cdot2^{0}\in\left\{  0,1,\ldots,2^{N}-1\right\}  $. For example,
the bitstring $010110$ of length $6$ becomes the integer%
\[
0\cdot2^{5}+1\cdot2^{4}+0\cdot2^{3}+1\cdot2^{2}+1\cdot2^{1}+0\cdot2^{0}%
=22\in\left\{  0,1,\ldots,2^{6}-1\right\}  .
\]


Choose two \textbf{uniformly random} bitstrings $\mathbf{c}$ and $\mathbf{b}$
(of the same length as $\mathbf{a}$) and encode them as numbers $c$ and $b$
(as just explained). Encode the secret $\mathbf{a}$ as a number $a$ as well
(in the same way). Define the polynomial $p\left(  x\right)  =cx^{2}+bx+a$.
Reveal to each person $i\in\left\{  1,2,\ldots,n\right\}  $ the value
$p\left(  i\right)  $ -- or, rather, a bitstring that encodes it in binary --
as $\mathbf{a}_{i}$.

As we know, any three of the values $p\left(  i\right)  $ uniquely determine
the polynomial $p$. Thus, any three people can use their bitstrings
$\mathbf{a}_{i}$ to recover three values $p\left(  i\right)  $ and therefore
$p$ and therefore $a$ (as the constant term of $p$) and therefore $\mathbf{a}$
(by decoding $a$). So our method satisfies Requirement 1.

Now, let us see whether it satisfies Requirement 2. Any $2$ people can recover
two values $p\left(  i\right)  $, which generally do not determine $p$
uniquely. It is not hard to show that they do not even determine $a$ uniquely;
thus, they do not determine $\mathbf{a}$ uniquely. What's better: If you know
just two values of $p$, there are infinitely many possible choices for $p$,
and all of them have distinct constant terms (unless one of the two values you
know is $p\left(  0\right)  $, which of course pins down the constant term).
So we get infinitely many possible values for $a$, and thus infinitely many
possible values for $\mathbf{a}$. This means that our $2$ people don't gain
any insight about $\mathbf{a}$, right?

Not so fast! We cannot really have \textquotedblleft infinitely many possible
values for $\mathbf{a}$\textquotedblright, since $\mathbf{a}$ is bound to be a
bitstring of a given length -- there are only finitely many of those! You can
only get infinitely many possible values for $p$ if you forget how $p$ was
constructed (from $c$, $b$ and $a$) and pretend that $p$ is just a
\textquotedblleft uniformly random\textquotedblright\ polynomial (whatever
this means). But no one can force the $2$ people to do this; it is certainly
not in their interest! Here are some things they might do with this knowledge:

\begin{itemize}
\item Let $N$ be the length of $\mathbf{a}$ (which, as we said, is known).
Thus, $\mathbf{c}$ and $\mathbf{b}$ are bitstrings of length $N$, so that $c$
and $b$ are integers in $\left\{  0,1,\ldots,2^{N}-1\right\}  $. Assume that
one of the $2$ people is person $2$. Now, person $2$ knows $p\left(  2\right)
=c2^{2}+b2+a=4c+2b+a$, and thus knows whether $a$ is even or odd (because $a$
is even resp. odd if and only if $4c+2b+a$ is even resp. odd). This means she
knows the last bit of the secret $\mathbf{a}$. This is not \textquotedblleft
clueless\textquotedblright.

\item You might try to fix this by picking $c$ and $b$ to be uniformly random
rational numbers instead (rather than using uniformly random bitstrings
$\mathbf{c}$ and $\mathbf{b}$).

Unfortunately, there is no such thing as a \textquotedblleft uniformly random
rational number\textquotedblright\ (in the sense that, e.g., larger numbers
aren't less likely to be picked than smaller numbers). Any probability
distribution will make some numbers more likely than others, and this will
usually cause information about $\mathbf{a}$ to \textquotedblleft
leak\textquotedblright. For example, if $c$ and $b$ are chosen from the
interval $\left[  0,2^{N}-1\right]  $, then person $1$'s knowledge of
$p\left(  1\right)  =c1^{2}+b1+a=c+b+a$ will sometimes reveal to person $1$
that $a\geq0.5\cdot\left(  2^{N}-1\right)  $ (namely, this will happen when
$p\left(  1\right)  \geq2.5\cdot\left(  2^{N}-1\right)  $, which occasionally
happens). This, again, is nontrivial information about the secret $\mathbf{a}%
$, which a single person (or even two people) should not be having.
\end{itemize}

So we cannot make Requirement 2 hold, and the culprit is that there are too
many numbers (namely, infinitely many). What would help is a finite
\textquotedblleft number system\textquotedblright\ in which we can add,
subtract, multiply and divide (so that we can define polynomials over it, and
a polynomial of degree $\leq2$ is still uniquely determined by any $3$
values). Assuming that this \textquotedblleft number system\textquotedblright%
\ is large enough that we can encode bitstrings using \textquotedblleft
numbers\textquotedblright\ of this system (instead of integers), we can then
play the above game using this \textquotedblleft number
system\textquotedblright\ and obtain actually uniformly random numbers.

It turns out that such \textquotedblleft number systems\textquotedblright%
\ exist. They are called \textit{finite fields}, and we will construct them
later in this course.

Assuming that they can be constructed, we thus obtain a method of solving the
problem for $k=3$. A similar method works for arbitrary $k$, using polynomials
of degree $\leq k-1$. This is called \textit{Shamir's secret sharing scheme}.

\begin{center}
\textbf{2019-01-30 lecture (virtual)}
\end{center}

\section{Elementary number theory}

Let us now begin a systematic introduction to algebra. We start with studying
integers and their divisibility properties -- the beginnings of number theory.
Part of these will be used directly in what will follow; part of these will
inspire more general results and proofs.

\subsection{Notations}

\begin{definition}
Let $\mathbb{N}=\left\{  0,1,2,\ldots\right\}  $ be the set of
\textbf{nonnegative} integers.

Let $\mathbb{P}=\left\{  1,2,3,\ldots\right\}  $ be the set of
\textbf{positive} integers.

Let $\mathbb{Z}=\left\{  \ldots,-1,0,1,\ldots\right\}  $ be the set of integers.

Let $\mathbb{Q}$ be the set of rational numbers.

Let $\mathbb{R}$ be the set of real numbers.
\end{definition}

Be careful with the notation $\mathbb{N}$: While I use it for $\left\{
0,1,2,\ldots\right\}  $, various other authors use it for $\left\{
1,2,3,\ldots\right\}  $ instead. There is no consensus in sight on what
$\mathbb{N}$ should mean.

Same holds for the word \textquotedblleft natural number\textquotedblright%
\ (which I will avoid): It means \textquotedblleft element of $\mathbb{N}%
$\textquotedblright, so again its ultimate meaning depends on the author.

\subsection{Divisibility}

We now go through the basics of divisibility of integers.

\begin{definition}
\label{def.ent.div.div}Let $a$ and $b$ be two integers. We say that $a\mid b$
(or \textquotedblleft$a$ \textit{divides} $b$\textquotedblright\ or
\textquotedblleft$b$ is \textit{divisible by }$a$\textquotedblright\ or
\textquotedblleft$b$ is a \textit{multiple} of $a$\textquotedblright) if there
exists an integer $c$ such that $b=ac$.
\end{definition}

Some authors define the \textquotedblleft divisibility\textquotedblright%
\ relation a bit differently, in that they forbid $a=0$. From the viewpoint of
abstract algebra, this feels like an unnecessary exception, so we don't follow them.

\begin{example}
\label{exa.ent.div.triv}\textbf{(a)} We have $4\mid12$, since $12=4\cdot3$.

\textbf{(b)} We have $a\mid0$ for any $a\in\mathbb{Z}$, since $0=a\cdot0$.

\textbf{(c)} An integer $b$ satisfies $0\mid b$ only when $b=0$, since $0\mid
b$ implies $b=0c=0$ (for some $c\in\mathbb{Z}$).

\textbf{(d)} We have $a\mid a$ for any $a\in\mathbb{Z}$, since $a=a\cdot1$.

\textbf{(e)} We have $1\mid b$ for each $b\in\mathbb{Z}$, since $b=1\cdot b$.
\end{example}

\begin{proposition}
\label{prop.ent.div.1}Let $a$ and $b$ be two integers.

\textbf{(a)} We have $a\mid b$ if and only $\left\vert a\right\vert
\mid\left\vert b\right\vert $.

\textbf{(b)} If $a\mid b$ and $b\neq0$, then $\left\vert a\right\vert
\leq\left\vert b\right\vert $.

\textbf{(c)} Assume that $a\neq0$. Then, $a\mid b$ if and only if $\dfrac
{b}{a}\in\mathbb{Z}$.
\end{proposition}

Before we prove this proposition, let us recall a well-known fact: We have
\begin{equation}
\left\vert xy\right\vert =\left\vert x\right\vert \cdot\left\vert y\right\vert
\label{eq.ent.div.abs(xy)}%
\end{equation}
for any two integers\footnote{or real numbers} $x$ and $y$. (This can be
easily proven by case distinction: $x$ is either nonnegative or negative, and
so is $y$.)

\begin{proof}
[Proof of Proposition \ref{prop.ent.div.1}.]\textbf{(a)} $\Longrightarrow
:$\footnote{If you are unfamiliar with the shorthand notation
\textquotedblleft$\Longrightarrow:$\textquotedblright, let me explain it. Our
goal is to prove that $a\mid b$ if and only if $\left\vert a\right\vert
\mid\left\vert b\right\vert $. In other words, we need to prove the
equivalence $\left(  a\mid b\right)  \Longleftrightarrow\left(  \left\vert
a\right\vert \mid\left\vert b\right\vert \right)  $. In order to prove this
equivalence, it suffices to prove the two implications $\left(  a\mid
b\right)  \Longrightarrow\left(  \left\vert a\right\vert \mid\left\vert
b\right\vert \right)  $ (called the \textquotedblleft forward
implication\textquotedblright\ or the \textquotedblleft$\Longrightarrow$
direction\textquotedblright\ of the equivalence) and $\left(  a\mid b\right)
\Longleftarrow\left(  \left\vert a\right\vert \mid\left\vert b\right\vert
\right)  $ (called the \textquotedblleft backward
implication\textquotedblright\ or the \textquotedblleft$\Longleftarrow$
direction\textquotedblright). The shorthand \textquotedblleft$\Longrightarrow
:$\textquotedblright\ simply marks the beginning of the proof of the forward
implication; similarly, the symbol \textquotedblleft$\Longleftarrow
:$\textquotedblright\ heralds in the proof of the backward implication.}
Assume that $a\mid b$. Thus, there exists an integer $d$ such that $b=ad$ (by
Definition \ref{def.ent.div.div}). Consider\footnote{Me saying
\textquotedblleft Consider this $d$\textquotedblright\ means that I am picking
some integer $d$ such that $b=ad$ (this can be done, since we have just proven
that such a $d$ exists), and will be referring to it as $d$ from now on.} this
$d$. We have $b=ad$ and thus $\left\vert b\right\vert =\left\vert
ad\right\vert =\left\vert a\right\vert \cdot\left\vert d\right\vert $ (by
(\ref{eq.ent.div.abs(xy)})). Thus, there exists an integer $c$ such that
$\left\vert b\right\vert =\left\vert a\right\vert \cdot c$ (namely,
$c=\left\vert d\right\vert $). In other words, $\left\vert a\right\vert
\mid\left\vert b\right\vert $. This proves the \textquotedblleft%
$\Longrightarrow$\textquotedblright\ direction of Proposition
\ref{prop.ent.div.1} \textbf{(a)}.

$\Longleftarrow:$ Assume that $\left\vert a\right\vert \mid\left\vert
b\right\vert $. Thus, there exists an integer $f$ such that $\left\vert
b\right\vert =\left\vert a\right\vert \cdot f$ (by Definition
\ref{def.ent.div.div}). Consider this $f$.

The definition of $\left\vert b\right\vert $ shows that $\left\vert
b\right\vert $ equals either $b$ or $-b$. Hence, $b$ equals either $\left\vert
b\right\vert $ or $-\left\vert b\right\vert $. In other words, $b$ equals
either $1\left\vert b\right\vert $ or $\left(  -1\right)  \left\vert
b\right\vert $. In other words, $b=q\left\vert b\right\vert $ for some
$q\in\left\{  1,-1\right\}  $. Similarly, $a=r\left\vert a\right\vert $ for
some $r\in\left\{  1,-1\right\}  $. Consider these $q$ and $r$.

From $r\in\left\{  1,-1\right\}  $, we obtain $r^{2}=1$. Now, $r\underbrace{a}%
_{=r\left\vert a\right\vert }=\underbrace{rr}_{=r^{2}=1}\left\vert
a\right\vert =\left\vert a\right\vert $.

Now, $b=a\cdot qfr$ (since $a\cdot qfr=qf\underbrace{ra}_{=\left\vert
a\right\vert }=q\underbrace{f\left\vert a\right\vert }_{=\left\vert
a\right\vert \cdot f=\left\vert b\right\vert }=q\left\vert b\right\vert =b$).
Hence, there exists an integer $c$ such that $b=ac$ (namely, $c=qfr$). In
other words, $a\mid b$. This proves the \textquotedblleft$\Longleftarrow
$\textquotedblright\ direction of Proposition \ref{prop.ent.div.1}
\textbf{(a)}.

Thus, the proof of Proposition \ref{prop.ent.div.1} \textbf{(a)} is complete.

\textbf{(b)} Assume that $a\mid b$ and $b\neq0$.

From $a\mid b$, we conclude that there exists an integer $c$ such that $b=ac$.
Consider this $c$. We have $ac=b\neq0$, thus $c\neq0$. Hence, $\left\vert
c\right\vert >0$, and thus $\left\vert c\right\vert \geq1$ (since $\left\vert
c\right\vert $ is an integer). We can multiply this inequality by $\left\vert
a\right\vert $ (since $\left\vert a\right\vert \geq0$), and obtain $\left\vert
a\right\vert \cdot\left\vert c\right\vert \geq\left\vert a\right\vert
\cdot1=\left\vert a\right\vert $.

From $b=ac$, we obtain $\left\vert b\right\vert =\left\vert ac\right\vert
=\left\vert a\right\vert \cdot\left\vert c\right\vert $ (by
(\ref{eq.ent.div.abs(xy)})). Hence, $\left\vert b\right\vert =\left\vert
a\right\vert \cdot\left\vert c\right\vert \geq\left\vert a\right\vert $. This
proves Proposition \ref{prop.ent.div.1} \textbf{(b)}.

\textbf{(c)} $\Longrightarrow:$ Assume that $a\mid b$. Thus, there exists an
integer $d$ such that $b=ad$. Consider this $d$. We can divide the equality
$b=ad$ by $a$ (since $a\neq0$), and thus obtain $\dfrac{b}{a}=d\in\mathbb{Z}$.
This proves the $\Longrightarrow$ direction of Proposition
\ref{prop.ent.div.1} \textbf{(c)}.

$\Longleftarrow:$ Assume that $\dfrac{b}{a}\in\mathbb{Z}$. Thus, there exists
an integer $c$ such that $b=ac$ (namely, $c=\dfrac{b}{a}$). In other words,
$a\mid b$. This proves the $\Longleftarrow$ direction of Proposition
\ref{prop.ent.div.1} \textbf{(c)}. Hence, the proof of Proposition
\ref{prop.ent.div.1} \textbf{(c)} is complete.
\end{proof}

Proposition \ref{prop.ent.div.1} \textbf{(a)} shows that both $a$ and $b$ in
\textquotedblleft the statement $a\mid b$\textquotedblright\ can be replaced
by their absolute values. Thus, when we talk about divisibility of integers,
the sign of the integers does not really matter -- it usually suffices to work
with nonnegative integers. We will often use this (tacitly, after a couple
times) in proofs.

The next proposition shows some basic properties of the divisibility relation:

\begin{proposition}
\label{prop.ent.div.2}\textbf{(a)} We have $a\mid a$ for every $a\in
\mathbb{Z}$. (This is called the \textit{reflexivity of divisibility}.)

\textbf{(b)} If $a,b,c\in\mathbb{Z}$ satisfy $a\mid b$ and $b\mid c$, then
$a\mid c$. (This is called the \textit{transitivity of divisibility}.)

\textbf{(c)} If $a_{1},a_{2},b_{1},b_{2}\in\mathbb{Z}$ satisfy $a_{1}\mid
b_{1}$ and $a_{2}\mid b_{2}$, then $a_{1}a_{2}\mid b_{1}b_{2}$.
\end{proposition}

\begin{proof}
\textbf{(a)} Let $a\in\mathbb{Z}$. Then, there exists an integer $c$ such that
$a=ac$ (namely, $c=1$). In other words, $a\mid a$. This proves Proposition
\ref{prop.ent.div.2} \textbf{(a)}.

\textbf{(b)} Let $a,b,c\in\mathbb{Z}$ satisfy $a\mid b$ and $b\mid c$.

From $a\mid b$, we conclude that there exists an integer $d$ such that $b=ad$.
Consider this $d$.

From $b\mid c$, we conclude that there exists an integer $e$ such that $c=be$.
Consider this $e$.

We have $c=\underbrace{b}_{=ad}e=ade$. Hence, there exists an integer $f$ such
that $c=af$ (namely, $f=de$). In other words, $a\mid c$ (by Definition
\ref{def.ent.div.div}). This proves Proposition \ref{prop.ent.div.2}
\textbf{(b)}.

\textbf{(c)} Let $a_{1},a_{2},b_{1},b_{2}\in\mathbb{Z}$ satisfy $a_{1}\mid
b_{1}$ and $a_{2}\mid b_{2}$.

From $a_{1}\mid b_{1}$, we conclude that there exists an integer $d$ such that
$b_{1}=a_{1}d$. Consider this $d$.

From $a_{2}\mid b_{2}$, we conclude that there exists an integer $e$ such that
$b_{2}=a_{2}e$. Consider this $e$.

We have $\underbrace{b_{1}}_{=a_{1}d}\underbrace{b_{2}}_{=a_{2}e}=a_{1}%
da_{2}e=a_{1}a_{2}de$. Hence, there exists an integer $f$ such that
$b_{1}b_{2}=a_{1}a_{2}f$ (namely, $f=de$). In other words, $a_{1}a_{2}\mid
b_{1}b_{2}$ (by Definition \ref{def.ent.div.div}). This proves Proposition
\ref{prop.ent.div.2} \textbf{(c)}.
\end{proof}

\begin{exercise}
\label{exe.ent.div.abba}Let $a$ and $b$ be two integers such that $a\mid b$
and $b\mid a$. Prove that $\left\vert a\right\vert =\left\vert b\right\vert $.
\end{exercise}

\begin{exercise}
\label{exe.ent.div.acbc}Let $a,b,c$ be three integers such that $c\neq0$.
Prove that $a\mid b$ holds if and only if $ac\mid bc$.
\end{exercise}

\begin{fineprint}
\begin{proof}
[Solution to Exercise \ref{exe.ent.div.acbc}.]$\Longrightarrow:$ Assume that
$a\mid b$ holds. We must prove that $ac\mid bc$.

It is easy to do this straight from the definition of divisibility, but here
is a shorter argument: Proposition \ref{prop.ent.div.2} \textbf{(a)} (applied
to $c$ instead of $a$) yields $c\mid c$. Also, $a\mid b$. Hence, Proposition
\ref{prop.ent.div.2} \textbf{(c)} (applied to $a_{1}=a$, $b_{1}=b$, $a_{2}=c$
and $b_{2}=c$) yields $ac\mid bc$. This proves the \textquotedblleft%
$\Longrightarrow$\textquotedblright\ direction of Exercise
\ref{exe.ent.div.acbc}.

$\Longleftarrow:$ Assume that $ac\mid bc$ holds. We must prove that $a\mid b$.

We have $ac\mid bc$. In other words, there exists an integer $d$ such that
$bc=\left(  ac\right)  d$ (by Definition \ref{def.ent.div.div}). Consider this
$d$. We have $bc=\left(  ac\right)  d=adc$. We can divide both sides of this
equality by $c$ (since $c\neq0$), and thus obtain $b=ad$. Thus, there exists
an integer $e$ such that $b=ae$ (namely, $e=d$). In other words, $a\mid b$ (by
Definition \ref{def.ent.div.div}). This proves the \textquotedblleft%
$\Longleftarrow$\textquotedblright\ direction of Exercise
\ref{exe.ent.div.acbc}.
\end{proof}
\end{fineprint}

\subsection{Congruence modulo $n$}

The next definition is simple but crucial:

\begin{definition}
\label{def.ent.cong}Let $n,a,b\in\mathbb{Z}$. We say that $a$ \textit{is
congruent to }$b$ \textit{modulo }$n$ (or, short, \textquotedblleft$a\equiv
b\operatorname{mod}n$\textquotedblright) if and only if $n\mid a-b$.
\end{definition}

\begin{example}
\label{exa.ent.cong.triv}\textbf{(a)} Is $3\equiv7\operatorname{mod}2$ ? Yes,
since $2\mid3-7=-4$.

\textbf{(b)} Is $3\equiv6\operatorname{mod}2$ ? No, since $2\nmid3-6=-3$.

Now, let $a$ and $b$ be two integers.

\textbf{(c)} We have $a\equiv b\operatorname{mod}0$ if and only if $a=b$.
(Indeed, $a\equiv b\operatorname{mod}0$ is defined to mean $0\mid a-b$, but
the latter divisibility happens only when $a-b=0$, which is tantamount to
saying $a=b$.)

\textbf{(d)} We have $a\equiv b\operatorname{mod}1$ always, since $1\mid a-b$
always holds (remember: $1$ divides everything).
\end{example}

Note that being congruent modulo $2$ means having the same parity: i.e., two
even numbers will be congruent modulo $2$, and two odd numbers will be, but an
even number will never be congruent to an odd number modulo $2$. (To be
rigorous: This is not quite obvious at this point yet; but it will be easy
once we have properly introduced division with remainder. See Exercise
\ref{exe.ent.even-odd.1} \textbf{(i)} below for the proof.)

\href{https://en.wikipedia.org/wiki/Modulo_(jargon)}{The word
\textquotedblleft modulo\textquotedblright}\ in the phrase \textquotedblleft%
$a$ is congruent to $b$ modulo $n$\textquotedblright\ is due to Gauss and
means something like \textquotedblleft with respect to\textquotedblright. You
should think of \textquotedblleft$a$ is congruent to $b$ modulo $n$%
\textquotedblright\ as a relation between all three of the numbers $a$, $b$
and $n$, but $a$ and $b$ are the \textquotedblleft main
characters\textquotedblright\ and $n$ sets the scenery.

\begin{exercise}
\label{exe.ent.mod.a+b=a-b}Let $a,b\in\mathbb{Z}$. Prove that $a+b\equiv
a-b\operatorname{mod}2$.
\end{exercise}

\begin{fineprint}
\begin{proof}
[Solution to Exercise \ref{exe.ent.mod.a+b=a-b}.]According to Definition
\ref{def.ent.cong}, we have $a+b\equiv a-b\operatorname{mod}2$ if and only if
$2\mid\left(  a+b\right)  -\left(  a-b\right)  $. Thus, it remains to prove
that $2\mid\left(  a+b\right)  -\left(  a-b\right)  $. But this follows
immediately from $\left(  a+b\right)  -\left(  a-b\right)  =2b$. Thus Exercise
\ref{exe.ent.mod.a+b=a-b} is solved.
\end{proof}
\end{fineprint}

We begin with a proposition so fundamental that we will always use it without saying:

\begin{proposition}
\label{prop.ent.mod.0}Let $n\in\mathbb{Z}$ and $a\in\mathbb{Z}$. Then,
$a\equiv0\operatorname{mod}n$ if and only if $n\mid a$.
\end{proposition}

\begin{proof}
[Proof of Proposition \ref{prop.ent.mod.0}.]We have the following chain of
equivalences:%
\begin{align*}
\left(  a\equiv0\operatorname{mod}n\right)  \  &  \Longleftrightarrow\ \left(
n\mid a-0\right)  \ \ \ \ \ \ \ \ \ \ \left(  \text{by Definition
\ref{def.ent.cong}}\right) \\
&  \Longleftrightarrow\ \left(  n\mid a\right)  \ \ \ \ \ \ \ \ \ \ \left(
\text{since }a-0=a\right)  .
\end{align*}
This proves Proposition \ref{prop.ent.mod.0}.
\end{proof}

Next come some staple properties of congruences:

\begin{proposition}
\label{prop.ent.mod.basics}Let $n\in\mathbb{Z}$.

\textbf{(a)} We have $a\equiv a\operatorname{mod}n$ for every $a\in\mathbb{Z}$.

\textbf{(b)} If $a,b,c\in\mathbb{Z}$ satisfy $a\equiv b\operatorname{mod}n$
and $b\equiv c\operatorname{mod}n$, then $a\equiv c\operatorname{mod}n$.

\textbf{(c)} If $a,b\in\mathbb{Z}$ satisfy $a\equiv b\operatorname{mod}n$,
then $b\equiv a\operatorname{mod}n$.

\textbf{(d)} If $a_{1},a_{2},b_{1},b_{2}\in\mathbb{Z}$ satisfy $a_{1}\equiv
b_{1}\operatorname{mod}n$ and $a_{2}\equiv b_{2}\operatorname{mod}n$, then%
\begin{align}
a_{1}+a_{2}  &  \equiv b_{1}+b_{2}\operatorname{mod}%
n;\label{eq.prop.ent.mod.basics.d.1}\\
a_{1}-a_{2}  &  \equiv b_{1}-b_{2}\operatorname{mod}%
n;\label{eq.prop.ent.mod.basics.d.2}\\
a_{1}a_{2}  &  \equiv b_{1}b_{2}\operatorname{mod}n.
\label{eq.prop.ent.mod.basics.d.3}%
\end{align}


\textbf{(e)} Let $m\in\mathbb{Z}$ be such that $m\mid n$. If $a,b\in
\mathbb{Z}$ satisfy $a\equiv b\operatorname{mod}n$, then $a\equiv
b\operatorname{mod}m$.
\end{proposition}

\begin{proof}
\textbf{(a)} Let $a\in\mathbb{Z}$. Recall that $a\equiv a\operatorname{mod}n$
is defined to mean $n\mid a-a$. Since $n\mid a-a$ holds (because
$a-a=0=n\cdot0$), we thus see that $a\equiv a\operatorname{mod}n$ holds. This
proves Proposition \ref{prop.ent.mod.basics} \textbf{(a)}.

\textbf{(b)} Let $a,b,c\in\mathbb{Z}$ satisfy $a\equiv b\operatorname{mod}n$
and $b\equiv c\operatorname{mod}n$.

We have $a\equiv b\operatorname{mod}n$. In other words, $n\mid a-b$ (by
Definition \ref{def.ent.cong}). In other words, there exists an integer $p$
such that $a-b=np$ (by Definition \ref{def.ent.div.div}). Consider this $p$.

We have $b\equiv c\operatorname{mod}n$. In other words, $n\mid b-c$ (by
Definition \ref{def.ent.cong}). In other words, there exists an integer $q$
such that $b-c=nq$ (by Definition \ref{def.ent.div.div}). Consider this $q$.

Now,
\[
a-c=\underbrace{\left(  a-b\right)  }_{=np}+\underbrace{\left(  b-c\right)
}_{=nq}=np+nq=n\left(  p+q\right)  .
\]
Hence, there exists an integer $r$ such that $a-c=nr$ (namely, $r=p+q$). In
other words, $n\mid a-c$ (by Definition \ref{def.ent.div.div}). In other
words, $a\equiv c\operatorname{mod}n$ (by Definition \ref{def.ent.cong}). This
proves Proposition \ref{prop.ent.mod.basics} \textbf{(b)}.

\textbf{(c)} Let $a,b\in\mathbb{Z}$ satisfy $a\equiv b\operatorname{mod}n$.

We have $a\equiv b\operatorname{mod}n$. In other words, $n\mid a-b$ (by
Definition \ref{def.ent.cong}). In other words, there exists an integer $p$
such that $a-b=np$ (by Definition \ref{def.ent.div.div}). Consider this $p$.
Now,%
\[
b-a=-\underbrace{\left(  a-b\right)  }_{=np}=-np=n\left(  -p\right)  .
\]
Hence, there exists an integer $c$ such that $b-a=nc$ (namely, $c=-p$). In
other words, $n\mid b-a$ (by Definition \ref{def.ent.div.div}). In other
words, $b\equiv a\operatorname{mod}n$ (by Definition \ref{def.ent.cong}). This
proves Proposition \ref{prop.ent.mod.basics} \textbf{(c)}.

\textbf{(d)} Let $a_{1},a_{2},b_{1},b_{2}\in\mathbb{Z}$ satisfy $a_{1}\equiv
b_{1}\operatorname{mod}n$ and $a_{2}\equiv b_{2}\operatorname{mod}n$.

We have $a_{1}\equiv b_{1}\operatorname{mod}n$. In other words, $n\mid
a_{1}-b_{1}$ (by Definition \ref{def.ent.cong}). In other words, there exists
an integer $p$ such that $a_{1}-b_{1}=np$ (by Definition \ref{def.ent.div.div}%
). Consider this $p$.

We have $a_{2}\equiv b_{2}\operatorname{mod}n$. In other words, $n\mid
a_{2}-b_{2}$ (by Definition \ref{def.ent.cong}). In other words, there exists
an integer $q$ such that $a_{2}-b_{2}=nq$ (by Definition \ref{def.ent.div.div}%
). Consider this $q$.

We have%
\[
\left(  a_{1}+a_{2}\right)  -\left(  b_{1}+b_{2}\right)  =\underbrace{\left(
a_{1}-b_{1}\right)  }_{=np}+\underbrace{\left(  a_{2}-b_{2}\right)  }%
_{=nq}=np+nq=n\left(  p+q\right)  .
\]
Hence, there exists an integer $c$ such that $\left(  a_{1}+a_{2}\right)
-\left(  b_{1}+b_{2}\right)  =nc$ (namely, $c=p+q$). In other words,
$n\mid\left(  a_{1}+a_{2}\right)  -\left(  b_{1}+b_{2}\right)  $ (by
Definition \ref{def.ent.div.div}). In other words, $a_{1}+a_{2}\equiv
b_{1}+b_{2}\operatorname{mod}n$ (by Definition \ref{def.ent.cong}). A similar
argument (using $p-q$ instead of $p+q$) shows that $a_{1}-a_{2}\equiv
b_{1}-b_{2}\operatorname{mod}n$. It thus remains to show that $a_{1}%
a_{2}\equiv b_{1}b_{2}\operatorname{mod}n$.

Let us first show that $a_{1}a_{2}\equiv a_{1}b_{2}\operatorname{mod}n$.
Indeed, $a_{1}a_{2}-a_{1}b_{2}=a_{1}\underbrace{\left(  a_{2}-b_{2}\right)
}_{=nq}=a_{1}nq=n\left(  a_{1}q\right)  $. Hence, there exists an integer $c$
such that $a_{1}a_{2}-a_{1}b_{2}=nc$ (namely, $c=a_{1}q$). In other words,
$n\mid a_{1}a_{2}-a_{1}b_{2}$ (by Definition \ref{def.ent.div.div}). In other
words, $a_{1}a_{2}\equiv a_{1}b_{2}\operatorname{mod}n$ (by Definition
\ref{def.ent.cong}).

Next, let us show that $a_{1}b_{2}\equiv b_{1}b_{2}\operatorname{mod}n$.
Indeed, $a_{1}b_{2}-b_{1}b_{2}=b_{2}\underbrace{\left(  a_{1}-b_{1}\right)
}_{=np}=b_{2}np=n\left(  b_{2}p\right)  $. Hence, there exists an integer $c$
such that $a_{1}b_{2}-b_{1}b_{2}=nc$ (namely, $c=b_{2}p$). In other words,
$n\mid a_{1}b_{2}-b_{1}b_{2}$ (by Definition \ref{def.ent.div.div}). In other
words, $a_{1}b_{2}\equiv b_{1}b_{2}\operatorname{mod}n$ (by Definition
\ref{def.ent.cong}).

From $a_{1}a_{2}\equiv a_{1}b_{2}\operatorname{mod}n$ and $a_{1}b_{2}\equiv
b_{1}b_{2}\operatorname{mod}n$, we now conclude that $a_{1}a_{2}\equiv
b_{1}b_{2}\operatorname{mod}n$ (by Proposition \ref{prop.ent.mod.basics}
\textbf{(c)}, applied to $a=a_{1}a_{2}$, $b=a_{1}b_{2}$ and $c=b_{1}b_{2}$).
This completes the proof of Proposition \ref{prop.ent.mod.basics} \textbf{(d)}.

\textbf{(e)} Let $a,b\in\mathbb{Z}$ satisfy $a\equiv b\operatorname{mod}n$.

We have $a\equiv b\operatorname{mod}n$. In other words, $n\mid a-b$ (by
Definition \ref{def.ent.cong}). From $m\mid n$ and $n\mid a-b$, we obtain
$m\mid a-b$ (by Proposition \ref{prop.ent.div.2} \textbf{(b)}, applied to $m$,
$n$ and $a-b$ instead of $a$, $b$ and $c$). In other words, $a\equiv
b\operatorname{mod}m$ (by Definition \ref{def.ent.cong}). This proves
Proposition \ref{prop.ent.mod.basics} \textbf{(e)}.
\end{proof}

In the above proof, we took care to explicitly cite Definition
\ref{def.ent.div.div} and Definition \ref{def.ent.cong} whenever we used them;
in the following, we will omit references like this.

Proposition \ref{prop.ent.mod.basics} \textbf{(d)} is saying that congruences
modulo $n$ (for a fixed integer $n$) can be added, subtracted and multiplied
together. This does not mean that you can do everything with them that you can
do with equalities. The next exercise shows that dividing congruences and
taking a congruence to the power of another does not generally work:

\begin{exercise}
\label{exe.ent.mod.basics-nope}Let $n,a_{1},a_{2},b_{1},b_{2}\in\mathbb{Z}$
satisfy $a_{1}\equiv b_{1}\operatorname{mod}n$ and $a_{2}\equiv b_{2}%
\operatorname{mod}n$. Then, \textbf{in general}, neither $a_{1}/a_{2}\equiv
b_{1}/b_{2}\operatorname{mod}n$ nor $a_{1}^{a_{2}}\equiv b_{1}^{b_{2}%
}\operatorname{mod}n$ is necessarily true. Of course, this is partly due to
the fact that $a_{1}/a_{2}$, $b_{1}/b_{2}$ and $a_{1}^{a_{2}}$ and
$b_{1}^{b_{2}}$ are not always integers in the first place (and being
congruent modulo $n$ only makes sense for integers, at least for now). But
even when $a_{1}/a_{2}$, $b_{1}/b_{2}$ and $a_{1}^{a_{2}}$ and $b_{1}^{b_{2}}$
are integers, the congruences $a_{1}/a_{2}\equiv b_{1}/b_{2}\operatorname{mod}%
n$ nor $a_{1}^{a_{2}}\equiv b_{1}^{b_{2}}\operatorname{mod}n$ are often false.
Find examples of $n,a_{1},a_{2},b_{1},b_{2}$ such that $a_{1}/a_{2}$,
$b_{1}/b_{2}$ and $a_{1}^{a_{2}}$ and $b_{1}^{b_{2}}$ are integers but the
congruences $a_{1}/a_{2}\equiv b_{1}/b_{2}\operatorname{mod}n$ and
$a_{1}^{a_{2}}\equiv b_{1}^{b_{2}}\operatorname{mod}n$ are false.
\end{exercise}

\begin{fineprint}
\begin{proof}
[Solution to Exercise \ref{exe.ent.mod.basics-nope}.]There are many such
examples. Here is one:%
\[
n=8,\ \ \ \ \ \ \ \ \ \ a_{1}=10,\ \ \ \ \ \ \ \ \ \ a_{2}%
=2,\ \ \ \ \ \ \ \ \ \ b_{1}=10,\ \ \ \ \ \ \ \ \ \ b_{2}=10.
\]
These satisfy $a_{1}\equiv b_{1}\operatorname{mod}n$ and $a_{2}\equiv
b_{2}\operatorname{mod}n$ but neither $a_{1}/a_{2}\equiv b_{1}/b_{2}%
\operatorname{mod}n$ nor $a_{1}^{a_{2}}\equiv b_{1}^{b_{2}}\operatorname{mod}%
n$.

It is much easier to find examples which fail only one of the two congruences
$a_{1}/a_{2}\equiv b_{1}/b_{2}\operatorname{mod}n$ and $a_{1}^{a_{2}}\equiv
b_{1}^{b_{2}}\operatorname{mod}n$.
\end{proof}
\end{fineprint}

However, we can divide a congruence $a\equiv b\operatorname{mod}n$ by a
nonzero integer $d$ when all of $a,b,n$ are divisible by $d$:

\begin{exercise}
\label{exe.ent.mod.basics.2}Let $n,d,a,b\in\mathbb{Z}$, and assume that
$d\neq0$. Assume that $d$ divides each of $a,b,n$, and assume that $a\equiv
b\operatorname{mod}n$. Prove that $a/d\equiv b/d\operatorname{mod}n/d$.
\end{exercise}

\begin{fineprint}
\begin{proof}
[Solution to Exercise \ref{exe.ent.mod.basics.2}.]We have $a\equiv
b\operatorname{mod}n$. In other words, $n\mid a-b$ (by the definition of
congruence). Note that all of $a/d$, $b/d$ and $n/d$ are integers (since $d$
divides each of $a,b,n$). Hence, $\left(  a-b\right)  /d=a/d-b/d$ is an
integer as well. Hence, Exercise \ref{exe.ent.div.acbc} (applied to $n/d$,
$\left(  a-b\right)  /d$ and $d$ instead of $a$, $b$ and $c$) shows that
$n/d\mid\left(  a-b\right)  /d$ holds if and only if $\left(  n/d\right)
d\mid\left(  \left(  a-b\right)  /d\right)  d$. Since $\left(  n/d\right)
d\mid\left(  \left(  a-b\right)  /d\right)  d$ does hold (indeed, this is just
a complicated way to say $n\mid a-b$), we thus conclude that $n/d\mid\left(
a-b\right)  /d$ holds. In other words, $n/d\mid a/d-b/d$ (since $\left(
a-b\right)  /d=a/d-b/d$). In other words, $a/d\equiv b/d\operatorname{mod}n/d$
(by the definition of congruence). This solves Exercise
\ref{exe.ent.mod.basics.2}.
\end{proof}
\end{fineprint}

We can also take a congruence to the $k$-th power when $k\in\mathbb{N}$:

\begin{exercise}
\label{exe.ent.mod.basics.k-power}Let $n,a,b\in\mathbb{Z}$ be such that
$a\equiv b\operatorname{mod}n$. Prove that $a^{k}\equiv b^{k}%
\operatorname{mod}n$ for each $k\in\mathbb{N}$.
\end{exercise}

(Note that the \textquotedblleft$n$\textquotedblright\ is not being taken to
the $k$-th power here.)

\begin{fineprint}
\begin{proof}
[First solution to Exercise \ref{exe.ent.mod.basics.k-power}.]We want to prove
that
\begin{equation}
a^{k}\equiv b^{k}\operatorname{mod}n\ \ \ \ \ \ \ \ \ \ \text{for each }%
k\in\mathbb{N}. \label{sol.ent.mod.basics.k-power.goal}%
\end{equation}
We shall prove this by induction on $k$:

\textit{Induction base:} Proposition \ref{prop.ent.mod.basics} \textbf{(a)}
yields $1\equiv1\operatorname{mod}n$. In view of $a^{0}=1$ and $b^{0}=1$, this
rewrites as $a^{0}\equiv b^{0}\operatorname{mod}n$. In other words,
(\ref{sol.ent.mod.basics.k-power.goal}) holds for $k=0$. This completes the
induction base.

\textit{Induction step:} Let $\ell\in\mathbb{N}$. Assume that
(\ref{sol.ent.mod.basics.k-power.goal}) holds for $k=\ell$. We must prove that
(\ref{sol.ent.mod.basics.k-power.goal}) holds for $k=\ell+1$.

We have assumed that (\ref{sol.ent.mod.basics.k-power.goal}) holds for
$k=\ell$. In other words, we have $a^{\ell}\equiv b^{\ell}\operatorname{mod}%
n$. Also, recall that $a\equiv b\operatorname{mod}n$. Hence,
(\ref{eq.prop.ent.mod.basics.d.3}) (applied to $c=a^{\ell}$ and $d=b^{\ell}$)
yields $aa^{\ell}\equiv bb^{\ell}\operatorname{mod}n$. In other words,
$a^{\ell+1}\equiv b^{\ell+1}\operatorname{mod}n$ (since $aa^{\ell}=a^{\ell+1}$
and $bb^{\ell}=b^{\ell+1}$). In other words,
(\ref{sol.ent.mod.basics.k-power.goal}) holds for $k=\ell+1$. This completes
the induction step. Thus, (\ref{sol.ent.mod.basics.k-power.goal}) is proven by
induction. Therefore, Exercise \ref{exe.ent.mod.basics.k-power} is solved.
\end{proof}

\begin{proof}
[Second solution to Exercise \ref{exe.ent.mod.basics.k-power}.]Recall that%
\begin{equation}
\left(  a-b\right)  \left(  a^{k-1}+a^{k-2}b+a^{k-3}b^{2}+\cdots
+ab^{k-2}+b^{k-1}\right)  =a^{k}-b^{k}
\label{sol.exe.ent.mod.basics.k-power.2nd.gs}%
\end{equation}
for every $k\in\mathbb{N}$. (This is a well-known identity, and it appears
(with $k$ renamed as $n$) as the first half of Exercise 1 on
\href{http://www-users.math.umn.edu/~dgrinber/19s/hw0s.pdf}{homework set \#0}.)

Now, let $k\in\mathbb{N}$. We have assumed that $a\equiv b\operatorname{mod}%
n$. In other words, $n\mid a-b$. In other words, there exists an integer $c$
such that $a-b=nc$. Consider this $c$. Now,
(\ref{sol.exe.ent.mod.basics.k-power.2nd.gs}) yields%
\begin{align*}
a^{k}-b^{k}  &  =\underbrace{\left(  a-b\right)  }_{=nc}\left(  a^{k-1}%
+a^{k-2}b+a^{k-3}b^{2}+\cdots+ab^{k-2}+b^{k-1}\right) \\
&  =nc\left(  a^{k-1}+a^{k-2}b+a^{k-3}b^{2}+\cdots+ab^{k-2}+b^{k-1}\right)  .
\end{align*}
The right hand side of this equality is clearly divisible by $n$. Hence, so is
the left hand side. In other words, $n\mid a^{k}-b^{k}$. In other words,
$a^{k}\equiv b^{k}\operatorname{mod}n$. Hence, Exercise
\ref{exe.ent.mod.basics.k-power} is solved again.
\end{proof}
\end{fineprint}

We can add not just two, but any number of congruences (where
\textquotedblleft number\textquotedblright\ means \textquotedblleft finite
number\textquotedblright):

\begin{exercise}
\label{exe.ent.mod.k-sum}Let $n$ be an integer. Let $S$ be a finite set. For
each $s\in S$, let $a_{s}$ and $b_{s}$ be two integers. Assume that%
\begin{equation}
a_{s}\equiv b_{s}\operatorname{mod}n\ \ \ \ \ \ \ \ \ \ \text{for each }s\in
S. \label{eq.exe.ent.mod.k-sum.ass}%
\end{equation}


\textbf{(a)} Prove that%
\begin{equation}
\sum_{s\in S}a_{s}\equiv\sum_{s\in S}b_{s}\operatorname{mod}n.
\label{eq.exe.ent.mod.k-sum.a}%
\end{equation}


\textbf{(b)} Prove that
\begin{equation}
\prod_{s\in S}a_{s}\equiv\prod_{s\in S}b_{s}\operatorname{mod}n.
\label{eq.exe.ent.mod.k-sum.b}%
\end{equation}


(Keep in mind that if the set $S$ is empty, then $\sum_{s\in S}a_{s}%
=\sum_{s\in S}b_{s}=0$ and $\prod_{s\in S}a_{s}=\prod_{s\in S}b_{s}=1$; this
holds by the definition of empty sums and of empty products.)
\end{exercise}

\begin{fineprint}
\begin{proof}
[Solution to Exercise \ref{exe.ent.mod.k-sum}.]\textbf{(a)} We shall solve
Exercise \ref{exe.ent.mod.k-sum} \textbf{(a)} by induction on $\left\vert
S\right\vert $:

\textit{Induction base:} Exercise \ref{exe.ent.mod.k-sum} \textbf{(a)} holds
whenever $\left\vert S\right\vert =0$\ \ \ \ \footnote{\textit{Proof.} Let
$n$, $S$, $a_{s}$ and $b_{s}$ be as in Exercise \ref{exe.ent.mod.k-sum}, and
assume that $\left\vert S\right\vert =0$. Then, the set $S$ is empty (since
$\left\vert S\right\vert =0$), and thus we have $\sum_{s\in S}a_{s}=\left(
\text{empty sum}\right)  =0$. Similarly, $\sum_{s\in S}b_{s}=0$. Now,
Proposition \ref{prop.ent.mod.basics} \textbf{(a)} yields $0\equiv
0\operatorname{mod}n$. In view of $\sum_{s\in S}a_{s}=0$ and $\sum_{s\in
S}b_{s}=0$, this rewrites as $\sum_{s\in S}a_{s}\equiv\sum_{s\in S}%
b_{s}\operatorname{mod}n$. Thus, Exercise \ref{exe.ent.mod.k-sum} \textbf{(a)}
holds in our case.
\par
So we have shown that Exercise \ref{exe.ent.mod.k-sum} \textbf{(a)} holds
whenever $\left\vert S\right\vert =0$.}. This completes the induction base.

\textit{Induction step:} Fix $k\in\mathbb{N}$. Assume that Exercise
\ref{exe.ent.mod.k-sum} \textbf{(a)} holds whenever $\left\vert S\right\vert
=k$. We must prove that Exercise \ref{exe.ent.mod.k-sum} \textbf{(a)} holds
whenever $\left\vert S\right\vert =k+1$.

We have assumed that Exercise \ref{exe.ent.mod.k-sum} \textbf{(a)} holds
whenever $\left\vert S\right\vert =k$. In other words, the following statement
is true:

\begin{statement}
\textit{Statement 1:} Let $n$, $S$, $a_{s}$ and $b_{s}$ be as in Exercise
\ref{exe.ent.mod.k-sum}. Assume that $\left\vert S\right\vert =k$. Then,
$\sum_{s\in S}a_{s}\equiv\sum_{s\in S}b_{s}\operatorname{mod}n$.
\end{statement}

Now, we must prove that Exercise \ref{exe.ent.mod.k-sum} \textbf{(a)} holds
whenever $\left\vert S\right\vert =k+1$. In other words, we must prove the
following statement:

\begin{statement}
\textit{Statement 2:} Let $n$, $S$, $a_{s}$ and $b_{s}$ be as in Exercise
\ref{exe.ent.mod.k-sum}. Assume that $\left\vert S\right\vert =k+1$. Then,
$\sum_{s\in S}a_{s}\equiv\sum_{s\in S}b_{s}\operatorname{mod}n$.
\end{statement}

[\textit{Proof of Statement 2:} We have $\left\vert S\right\vert =k+1>k\geq0$;
thus, the set $S$ is nonempty. Hence, there exists some $t\in S$. Pick such a
$t$. Thus, $\left\vert S\setminus\left\{  t\right\}  \right\vert =\left\vert
S\right\vert -1=k$ (since $\left\vert S\right\vert =k+1$). Moreover, from
(\ref{eq.exe.ent.mod.k-sum.ass}), we immediately obtain that
\[
a_{s}\equiv b_{s}\operatorname{mod}n\ \ \ \ \ \ \ \ \ \ \text{for each }s\in
S\setminus\left\{  t\right\}
\]
(since each $s\in S\setminus\left\{  t\right\}  $ belongs to $S$). Hence, we
can apply Statement 1 to $S\setminus\left\{  t\right\}  $ instead of $S$. We
thus obtain
\[
\sum_{s\in S\setminus\left\{  t\right\}  }a_{s}\equiv\sum_{s\in S\setminus
\left\{  t\right\}  }b_{s}\operatorname{mod}n.
\]
Also, we have
\[
a_{t}\equiv b_{t}\operatorname{mod}n
\]
(by (\ref{eq.exe.ent.mod.k-sum.ass}), applied to $s=t$). Adding these two
congruences together, we obtain%
\[
\sum_{s\in S\setminus\left\{  t\right\}  }a_{s}+a_{t}\equiv\sum_{s\in
S\setminus\left\{  t\right\}  }b_{s}+b_{t}\operatorname{mod}n.
\]
In view of%
\[
\sum_{s\in S}a_{s}=\sum_{s\in S\setminus\left\{  t\right\}  }a_{s}%
+a_{t}\ \ \ \ \ \ \ \ \ \ \left(
\begin{array}
[c]{c}%
\text{here, we have split off the addend}\\
\text{for }s=t\text{ from the sum}%
\end{array}
\right)
\]
and%
\[
\sum_{s\in S}b_{s}=\sum_{s\in S\setminus\left\{  t\right\}  }b_{s}%
+b_{t}\ \ \ \ \ \ \ \ \ \ \left(
\begin{array}
[c]{c}%
\text{here, we have split off the addend}\\
\text{for }s=t\text{ from the sum}%
\end{array}
\right)  ,
\]
this can be rewritten as%
\[
\sum_{s\in S}a_{s}\equiv\sum_{s\in S}b_{s}\operatorname{mod}n.
\]
This proves Statement 2.]

We have now proven Statement 2; this means that Exercise
\ref{exe.ent.mod.k-sum} \textbf{(a)} holds whenever $\left\vert S\right\vert
=k+1$. This completes the induction step; thus, Exercise
\ref{exe.ent.mod.k-sum} \textbf{(a)} is solved.

\textbf{(b)} The solution to Exercise \ref{exe.ent.mod.k-sum} \textbf{(b)} is
analogous to the one we gave above for Exercise \ref{exe.ent.mod.k-sum}
\textbf{(a)}; the main difference is that we have to replace sums by products
(and $0$ by $1$).
\end{proof}
\end{fineprint}

\begin{exercise}
\label{exe.ent.mod.prod-wrong}Is it true that if $a_{1},a_{2},b_{1}%
,b_{2},n_{1},n_{2}\in\mathbb{Z}$ satisfy $a_{1}\equiv b_{1}\operatorname{mod}%
n_{1}$ and $a_{2}\equiv b_{2}\operatorname{mod}n_{2}$, then $a_{1}a_{2}\equiv
b_{1}b_{2}\operatorname{mod}n_{1}n_{2}$ ?
\end{exercise}

\begin{fineprint}
\begin{proof}
[Solution to Exercise \ref{exe.ent.mod.prod-wrong}.]No, it is not true. For
example, $a_{1}=1$, $a_{2}=1$, $b_{1}=1$, $b_{2}=0$, $n_{1}=0$ and $n_{2}=1$
yield a counterexample.
\end{proof}
\end{fineprint}

\subsection{\label{sect.ent.subst-chain}Chains of congruences}

For this whole Section \ref{sect.ent.subst-chain}, we fix an integer $n$.

Chains of equalities are a fundamental piece of notation used throughout
mathematics. For example, here is a chain of equalities:%
\begin{align*}
&  \left(  ad+bc\right)  ^{2}+\left(  ac-bd\right)  ^{2}\\
&  =\left(  ad\right)  ^{2}+2ad\cdot bc+\left(  bc\right)  ^{2}+\left(
ac\right)  ^{2}-2ac\cdot bd+\left(  bd\right)  ^{2}\\
&  =a^{2}d^{2}+2abcd+b^{2}c^{2}+a^{2}c^{2}-2abcd+b^{2}d^{2}\\
&  =a^{2}c^{2}+a^{2}d^{2}+b^{2}c^{2}+b^{2}d^{2}\\
&  =\left(  a^{2}+b^{2}\right)  \left(  c^{2}+d^{2}\right)
\end{align*}
(where $a,b,c,d$ are arbitrary numbers). This chain proves the equality
(\ref{eq.intro.sum-of-2sq.sum*sum}). But why does it really? If we look
closely at this chain of equalities, we see that it has the form
\textquotedblleft$A=B=C=D=E$\textquotedblright, where $A,B,C,D,E$ are five
numbers (namely, $A=\left(  ad+bc\right)  ^{2}+\left(  ac-bd\right)  ^{2}$ and
$B=\left(  ad\right)  ^{2}+2ad\cdot bc+\left(  bc\right)  ^{2}+\left(
ac\right)  ^{2}-2ac\cdot bd+\left(  bd\right)  ^{2}$ and so on). This kind of
statement is called a \textquotedblleft chain of equalities\textquotedblright,
and, a priori, it simply means that any two \textbf{adjacent} numbers in this
chain are equal: $A=B$ and $B=C$ and $C=D$ and $D=E$. Without as much as
noticing it, we have concluded that \textbf{any} two numbers in this chain are
equal; thus, in particular, $A=E$, which is precisely the equality
(\ref{eq.intro.sum-of-2sq.sum*sum}) we wanted to prove.

That this kind of \textquotedblleft chaining\textquotedblright\ is possible is
one of the most basic facts in mathematics. Let us define a chain of
equalities formally:

\begin{definition}
If $a_{1},a_{2},\ldots,a_{k}$ are $k$ objects\footnotemark, then the statement
\textquotedblleft$a_{1}=a_{2}=\cdots=a_{k}$\textquotedblright\ shall mean
that
\[
a_{i}=a_{i+1}\text{ holds for each }i\in\left\{  1,2,\ldots,k-1\right\}  .
\]
(In other words, it shall mean that $a_{1}=a_{2}$ and $a_{2}=a_{3}$ and
$a_{3}=a_{4}$ and $\cdots$ and $a_{k-1}=a_{k}$. This is vacuously true when
$k\leq1$. If $k=2$, then it simply means that $a_{1}=a_{2}$.)

Such a statement will be called a \textit{chain of equalities}.
\end{definition}

\footnotetext{\textquotedblleft Objects\textquotedblright\ can be numbers,
sets, tuples or any other well-defined things in mathematics.}

\begin{proposition}
\label{prop.mod.chain-eq}Let $a_{1},a_{2},\ldots,a_{k}$ be $k$ objects such
that $a_{1}=a_{2}=\cdots=a_{k}$. Let $u$ and $v$ be two elements of $\left\{
1,2,\ldots,k\right\}  $. Then, $a_{u}=a_{v}$.
\end{proposition}

So we have defined a chain of equalities to be true if and only if any two
adjacent terms in this chain are equal (i.e., if \textquotedblleft each
equality sign in the chain is satisfied\textquotedblright). Proposition
\ref{prop.mod.chain-eq} shows that in such a chain, \textbf{any two} terms are
equal. This is intuitively rather clear, but can also be formally proven by
induction using the basic properties of equality
(transitivity\footnote{\textit{Transitivity of equality} says that if $a,b,c$
are three objects satisfying $a=b$ and $b=c$, then $a=c$.},
reflexivity\footnote{\textit{Reflexivity of equality} says that every object
$a$ satisfies $a=a$.} and symmetry\footnote{\textit{Symmetry of equality} says
that if $a,b$ are two objects satisfying $a=b$, then $b=a$.}).

But our goal is to understand basic number theory, not to scrutinize the
foundations of mathematics. So let us recall that we have fixed an integer
$n$, and consider congruences modulo $n$. We claim that these can be chained
just as equalities:

\begin{definition}
If $a_{1},a_{2},\ldots,a_{k}$ are $k$ integers, then the statement
\textquotedblleft$a_{1}\equiv a_{2}\equiv\cdots\equiv a_{k}\operatorname{mod}%
n$\textquotedblright\ shall mean that
\[
a_{i}\equiv a_{i+1}\operatorname{mod}n\text{ holds for each }i\in\left\{
1,2,\ldots,k-1\right\}  .
\]
(In other words, it shall mean that $a_{1}\equiv a_{2}\operatorname{mod}n$ and
$a_{2}\equiv a_{3}\operatorname{mod}n$ and $a_{3}\equiv a_{4}%
\operatorname{mod}n$ and $\cdots$ and $a_{k-1}\equiv a_{k}\operatorname{mod}%
n$. This is vacuously true when $k\leq1$. If $k=2$, then it simply means that
$a_{1}\equiv a_{2}\operatorname{mod}n$.)

Such a statement will be called a \textit{chain of congruences modulo }$n$.
\end{definition}

\begin{proposition}
\label{prop.mod.chain}Let $a_{1},a_{2},\ldots,a_{k}$ be $k$ integers such that
$a_{1}\equiv a_{2}\equiv\cdots\equiv a_{k}\operatorname{mod}n$. Let $u$ and
$v$ be two elements of $\left\{  1,2,\ldots,k\right\}  $. Then, $a_{u}\equiv
a_{v}\operatorname{mod}n$.
\end{proposition}

Proposition \ref{prop.mod.chain} shows that any two terms in a chain of
congruences modulo $n$ must be congruent to each other modulo $n$. Again, this
can be formally proven by induction; see \cite[proof of Proposition
2.16]{detnotes}. The ingredients of the proof are basic properties of
congruence modulo $n$: transitivity, reflexivity and symmetry. These are fancy
names for parts \textbf{(b)}, \textbf{(a)} and \textbf{(c)} of Proposition
\ref{prop.ent.mod.basics}.

We will use Proposition \ref{prop.mod.chain} tacitly (just as you would use
Proposition \ref{prop.mod.chain-eq}): i.e., every time we prove a chain of
congruences like $a_{1}\equiv a_{2}\equiv\cdots\equiv a_{k}\operatorname{mod}%
n$, we assume that the reader will automatically conclude that any two of its
terms are congruent to each other modulo $n$ (and will remember this
conclusion). For instance, if we show that $1\equiv4\equiv34\equiv
334\equiv304\operatorname{mod}3$, then we automatically get the congruences
$1\equiv304\operatorname{mod}3$ and $334\equiv1\operatorname{mod}3$ and
$4\equiv334\operatorname{mod}3$ and several others out of this chain.

Chains of congruences can also include equality signs. For example, if
$a,b,c,d$ are integers, then \textquotedblleft$a\equiv b=c\equiv
d\operatorname{mod}n$\textquotedblright\ means that $a\equiv
b\operatorname{mod}n$ and $b=c$ and $c\equiv d\operatorname{mod}n$. Such a
chain is still a chain of congruences, because $b=c$ implies $b\equiv
c\operatorname{mod}n$ (by Proposition \ref{prop.ent.mod.basics} \textbf{(a)}).

Just as there are chains of equalities and chains of congruences, there are
chains of divisibilities:

\begin{definition}
If $a_{1},a_{2},\ldots,a_{k}$ are $k$ integers, then the statement
\textquotedblleft$a_{1}\mid a_{2}\mid\cdots\mid a_{k}$\textquotedblright%
\ shall mean that
\[
a_{i}\mid a_{i+1}\text{ holds for each }i\in\left\{  1,2,\ldots,k-1\right\}
.
\]
(In other words, it shall mean that $a_{1}\mid a_{2}$ and $a_{2}\mid a_{3}$
and $a_{3}\mid a_{4}$ and $\cdots$ and $a_{k-1}\mid a_{k}$. This is vacuously
true when $k\leq1$. If $k=2$, then it simply means that $a_{1}\mid a_{2}$.)

Such a statement will be called a \textit{chain of divisibilities}.
\end{definition}

\begin{proposition}
\label{prop.ent.div.chain}Let $a_{1},a_{2},\ldots,a_{k}$ be $k$ integers such
that $a_{1}\mid a_{2}\mid\cdots\mid a_{k}$. Let $u$ and $v$ be two elements of
$\left\{  1,2,\ldots,k\right\}  $ such that $u\leq v$. Then, $a_{u}\mid a_{v}$.
\end{proposition}

Note that we had to require $u\leq v$ in this proposition, unlike the
analogous propositions for chains of equalities and chains of congruences,
because there is no \textquotedblleft symmetry of
divisibility\textquotedblright\ (i.e., if $a\mid b$, then we don't generally
have $b\mid a$). The proof of Proposition \ref{prop.ent.div.chain} relies on
the reflexivity of divisibility (Proposition \ref{prop.ent.div.2}
\textbf{(a)}) and on the transitivity of divisibility (Proposition
\ref{prop.ent.div.2} \textbf{(b)}).

Again, chains of divisibilities can include equality signs.

\begin{center}
\textbf{2019-02-01 lecture}
\end{center}

\subsection{\label{sect.ent.quorem}Division with remainder}

The following fact you likely remember from high school:

\begin{theorem}
\label{thm.ent.quorem.full}Let $n$ be a positive integer. Let $u\in\mathbb{Z}%
$. Then, there exists a unique pair $\left(  q,r\right)  \in\mathbb{Z}%
\times\left\{  0,1,\ldots,n-1\right\}  $ such that $u=qn+r$.
\end{theorem}

Before we prove this theorem, let us introduce the notations that it justifies:

\begin{definition}
\label{def.ent.quorem}Let $n$ be a positive integer. Let $u\in\mathbb{Z}$.
Theorem \ref{thm.ent.quorem.full} shows that there exists a unique pair
$\left(  q,r\right)  \in\mathbb{Z}\times\left\{  0,1,\ldots,n-1\right\}  $
such that $u=qn+r$. Consider this pair.

\textbf{(a)} We denote the integer $q$ by $u//n$, and refer to it as the
\textit{quotient of the division of }$u$ \textit{by }$n$.

\textbf{(b)} We denote the integer $r$ by $u\%n$, and refer to it as the
\textit{remainder of the division of }$u$ \textit{by }$n$.
\end{definition}

The words \textquotedblleft quotient\textquotedblright\ and \textquotedblleft
remainder\textquotedblright\ are standard, but the notations \textquotedblleft%
$u//n$\textquotedblright\ and \textquotedblleft$u\%n$\textquotedblright\ are
not (I have taken them from the Python programming language); be prepared to
see other notations in the literature (e.g., the notations \textquotedblleft%
$\operatorname*{quo}\left(  u,n\right)  $\textquotedblright\ and
\textquotedblleft$\operatorname*{rem}\left(  u,n\right)  $\textquotedblright%
\ for $u//n$ and $u\%n$, respectively).

\begin{example}
\textbf{(a)} We have $14//3=4$ and $14\%3=2$, because $\left(  4,2\right)  $
is the unique pair $\left(  q,r\right)  \in\mathbb{Z}\times\left\{
0,1,2\right\}  $ satisfying $14=q\cdot3+r$.

\textbf{(b)} We have $18//3=6$ and $18\%3=0$, because $\left(  6,0\right)  $
is the unique pair $\left(  q,r\right)  \in\mathbb{Z}\times\left\{
0,1,2\right\}  $ satisfying $18=q\cdot3+r$.

\textbf{(c)} We have $\left(  -2\right)  //3=-1$ and $\left(  -2\right)
\%3=1$, because $\left(  -1,1\right)  $ is the unique pair $\left(
q,r\right)  \in\mathbb{Z}\times\left\{  0,1,2\right\}  $ satisfying
$-2=q\cdot3+r$.

\textbf{(d)} For each $u\in\mathbb{Z}$, we have $u//1=u$ and $u\%1=0$, because
$\left(  u,0\right)  $ is the unique pair $\left(  q,r\right)  \in
\mathbb{Z}\times\left\{  0\right\}  $ satisfying $u=q\cdot1+r$.
\end{example}

But we have gotten ahead of ourselves: We need to prove Theorem
\ref{thm.ent.quorem.full} before we can use the notations \textquotedblleft%
$u//n$\textquotedblright\ and \textquotedblleft$u\%n$\textquotedblright.

Let us split Theorem \ref{thm.ent.quorem.full} into two parts: existence and uniqueness:

\begin{lemma}
\label{lem.ent.quorem.exist}Let $n$ be a positive integer. Let $u\in
\mathbb{Z}$. Then, there exists \textbf{at least one} pair $\left(
q,r\right)  \in\mathbb{Z}\times\left\{  0,1,\ldots,n-1\right\}  $ such that
$u=qn+r$.
\end{lemma}

\begin{lemma}
\label{lem.ent.quorem.unique}Let $n$ be a positive integer. Let $u\in
\mathbb{Z}$. Then, there exists \textbf{at most one} pair $\left(  q,r\right)
\in\mathbb{Z}\times\left\{  0,1,\ldots,n-1\right\}  $ such that $u=qn+r$.
\end{lemma}

We begin by proving Lemma \ref{lem.ent.quorem.unique} (which is the easier one):

\begin{proof}
[Proof of Lemma \ref{lem.ent.quorem.unique}.]Let $\left(  q_{1},r_{1}\right)
$ and $\left(  q_{2},r_{2}\right)  $ be two pairs $\left(  q,r\right)
\in\mathbb{Z}\times\left\{  0,1,\ldots,n-1\right\}  $ such that $u=qn+r$. We
shall show that $\left(  q_{1},r_{1}\right)  =\left(  q_{2},r_{2}\right)  $.

We know that $\left(  q_{1},r_{1}\right)  $ is a pair $\left(  q,r\right)
\in\mathbb{Z}\times\left\{  0,1,\ldots,n-1\right\}  $ such that $u=qn+r$. In
other words, $\left(  q_{1},r_{1}\right)  \in\mathbb{Z}\times\left\{
0,1,\ldots,n-1\right\}  $ and $u=q_{1}n+r_{1}$. Similarly, $\left(
q_{2},r_{2}\right)  \in\mathbb{Z}\times\left\{  0,1,\ldots,n-1\right\}  $ and
$u=q_{2}n+r_{2}$.

From $\left(  q_{1},r_{1}\right)  \in\mathbb{Z}\times\left\{  0,1,\ldots
,n-1\right\}  $, we obtain $q_{1}\in\mathbb{Z}$ and $r_{1}\in\left\{
0,1,\ldots,n-1\right\}  $. Similarly, $q_{2}\in\mathbb{Z}$ and $r_{2}%
\in\left\{  0,1,\ldots,n-1\right\}  $. Thus, in particular, $q_{1},q_{2}%
,r_{1},r_{2}$ are integers.

From $r_{1}\in\left\{  0,1,\ldots,n-1\right\}  $ and $r_{2}\in\left\{
0,1,\ldots,n-1\right\}  $, we can easily derive%
\begin{equation}
\left\vert r_{2}-r_{1}\right\vert \leq n-1.
\label{pf.lem.ent.quorem.unique.ineq}%
\end{equation}


\begin{fineprint}
[\textit{Proof of (\ref{pf.lem.ent.quorem.unique.ineq}):} Intuitively, this
should be clear: Both $r_{1}$ and $r_{2}$ belong to the integer interval
$\left\{  0,1,\ldots,n-1\right\}  $, and thus the unsigned distance between
$r_{1}$ and $r_{2}$ is at most $n-1$ (with the worst case being when $r_{1}$
and $r_{2}$ are at opposite ends of this interval).

Here is a formal restatement of this argument: We have $r_{1}\in\left\{
0,1,\ldots,n-1\right\}  $, thus $r_{1}\geq0$. Also, $r_{2}\in\left\{
0,1,\ldots,n-1\right\}  $, hence $r_{2}\leq n-1$. Hence, $\underbrace{r_{2}%
}_{\leq n-1}-\underbrace{r_{1}}_{\geq0}\leq\left(  n-1\right)  -0=n-1$.
Similarly, $r_{1}-r_{2}\leq n-1$. But recall that $\left\vert x\right\vert
\in\left\{  x,-x\right\}  $ for each $x\in\mathbb{Z}$. Applying this to
$x=r_{2}-r_{1}$, we obtain
\[
\left\vert r_{2}-r_{1}\right\vert \in\left\{  r_{2}-r_{1},\underbrace{-\left(
r_{2}-r_{1}\right)  }_{=r_{1}-r_{2}}\right\}  =\left\{  r_{2}-r_{1}%
,r_{1}-r_{2}\right\}  .
\]
In other words, $\left\vert r_{2}-r_{1}\right\vert $ is one of the two numbers
$r_{2}-r_{1}$ and $r_{1}-r_{2}$. Since both of these numbers $r_{2}-r_{1}$ and
$r_{1}-r_{2}$ are $\leq n-1$ (as we have just shown), we thus conclude that
$\left\vert r_{2}-r_{1}\right\vert \leq n-1$. This proves
(\ref{pf.lem.ent.quorem.unique.ineq}).]
\end{fineprint}

We have $q_{1}n+r_{1}=u=q_{2}n+r_{2}$, thus $q_{1}n-q_{2}n=r_{2}-r_{1}$.
Hence,%
\begin{equation}
r_{2}-r_{1}=q_{1}n-q_{2}n=\left(  q_{1}-q_{2}\right)  n.
\label{pf.lem.ent.quorem.unique.1}%
\end{equation}


Assume (for the sake of contradiction) that $q_{1}\neq q_{2}$. Thus,
$q_{1}-q_{2}\neq0$, so that $\left\vert q_{1}-q_{2}\right\vert >0$ and
therefore $\left\vert q_{1}-q_{2}\right\vert \geq1$ (since $\left\vert
q_{1}-q_{2}\right\vert $ is an integer). We can multiply this inequality by
$n$ (since $n$ is positive) and thus obtain $\left\vert q_{1}-q_{2}\right\vert
n\geq1n=n$. But from (\ref{pf.lem.ent.quorem.unique.1}), we obtain%
\begin{align*}
\left\vert r_{2}-r_{1}\right\vert  &  =\left\vert \left(  q_{1}-q_{2}\right)
n\right\vert =\left\vert q_{1}-q_{2}\right\vert \cdot\underbrace{\left\vert
n\right\vert }_{\substack{=n\\\text{(since }n\text{ is positive)}%
}}\ \ \ \ \ \ \ \ \ \ \left(  \text{by (\ref{eq.ent.div.abs(xy)})}\right) \\
&  =\left\vert q_{1}-q_{2}\right\vert n\geq n>n-1.
\end{align*}
This contradicts (\ref{pf.lem.ent.quorem.unique.ineq}). This contradiction
shows that our assumption (that $q_{1}\neq q_{2}$) was false. Hence, we have
$q_{1}=q_{2}$. Thus, $q_{1}-q_{2}=0$, so that
(\ref{pf.lem.ent.quorem.unique.1}) becomes $r_{2}-r_{1}=\underbrace{\left(
q_{1}-q_{2}\right)  }_{=0}n=0$ and thus $r_{2}=r_{1}$, so that $r_{1}=r_{2}$.
Combining this with $q_{1}=q_{2}$, we obtain $\left(  q_{1},r_{1}\right)
=\left(  q_{2},r_{2}\right)  $.

Now, forget that we have fixed $\left(  q_{1},r_{1}\right)  $ and $\left(
q_{2},r_{2}\right)  $. We thus have proven that if $\left(  q_{1}%
,r_{1}\right)  $ and $\left(  q_{2},r_{2}\right)  $ are two pairs $\left(
q,r\right)  \in\mathbb{Z}\times\left\{  0,1,\ldots,n-1\right\}  $ such that
$u=qn+r$, then $\left(  q_{1},r_{1}\right)  =\left(  q_{2},r_{2}\right)  $. In
other words, any two pairs $\left(  q,r\right)  \in\mathbb{Z}\times\left\{
0,1,\ldots,n-1\right\}  $ such that $u=qn+r$ must be equal. In other words,
there exists at most one such pair. This proves Lemma
\ref{lem.ent.quorem.unique}.
\end{proof}

But we also need to prove Lemma \ref{lem.ent.quorem.exist}. This lemma can be
proven by induction on $u$, but not without some complications: Since it is
stated for all integers $u$ (rather than just for nonnegative or positive
integers), the classical induction principle (with an induction base and a
\textquotedblleft$u$ to $u+1$\textquotedblright\ step) cannot prove it
directly. Instead, we have to either add a \textquotedblleft$u$ to
$u-1$\textquotedblright\ step to our induction (resulting in a
\textquotedblleft two-sided induction\textquotedblright\ or \textquotedblleft
up- and down-induction\textquotedblright\ argument), or to treat the case of
negative $u$ separately. A proof using the first of these two methods can be
found in \cite[proof of Proposition 2.150]{detnotes} (where $n$ and $u$ are
denoted by $N$ and $n$). We shall instead give a proof using the second
method; thus, we first state the particular case of Lemma
\ref{lem.ent.quorem.exist} when $u$ is nonnegative:

\begin{lemma}
\label{lem.ent.quorem.existN}Let $n$ be a positive integer. Let $u\in
\mathbb{N}$. Then, there exists \textbf{at least one} pair $\left(
q,r\right)  \in\mathbb{Z}\times\left\{  0,1,\ldots,n-1\right\}  $ such that
$u=qn+r$.
\end{lemma}

This lemma can be proven by induction on $u$ as in \cite[proof of Proposition
2.150]{detnotes}. Let us instead prove this by \textbf{strong} induction on
$u$. See \cite[\S 2.8]{detnotes} for an introduction to strong induction; in
particular, recall that a strong induction needs no induction base (but often
contains a case distinction in its \textquotedblleft induction
step\textquotedblright\ that, in some way, does give the first few values a
special treatment). The proof of Lemma \ref{lem.ent.quorem.existN} that we
give below follows a stupid but valid method of finding the pair $\left(
q,r\right)  $: Keep subtracting $n$ from $u$ until $u$ becomes $<n$; then $r$
will be the resulting number, whereas $q$ will be the number of times you have
subtracted $n$.

\begin{proof}
[Proof of Lemma \ref{lem.ent.quorem.existN}.]We proceed by strong induction on
$u$.

Let $U\in\mathbb{N}$. Assume (as the induction hypothesis) that Lemma
\ref{lem.ent.quorem.existN} holds for every $u\in\mathbb{N}$ satisfying $u<U$.
We must prove that Lemma \ref{lem.ent.quorem.existN} also holds for $u=U$. In
other words, we must prove that there exists \textbf{at least one} pair
$\left(  q,r\right)  \in\mathbb{Z}\times\left\{  0,1,\ldots,n-1\right\}  $
such that $U=qn+r$.

We are in one of the following two cases:

\textit{Case 1:} We have $U<n$.

\textit{Case 2:} We have $U\geq n$.

Let us first consider Case 1. In this case, we have $U<n$. Thus, $U\leq n-1$
(since $U$ and $n$ are integers), so that $U\in\left\{  0,1,\ldots
,n-1\right\}  $ (since $U\in\mathbb{N}$). Combining this with $0\in\mathbb{Z}%
$, we obtain $\left(  0,U\right)  \in\mathbb{Z}\times\left\{  0,1,\ldots
,n-1\right\}  $. Hence, $\left(  0,U\right)  $ is a pair $\left(  q,r\right)
\in\mathbb{Z}\times\left\{  0,1,\ldots,n-1\right\}  $ such that $U=qn+r$
(since $U=0n+U$). Thus, there exists \textbf{at least one} pair $\left(
q,r\right)  \in\mathbb{Z}\times\left\{  0,1,\ldots,n-1\right\}  $ such that
$U=qn+r$ (namely, $\left(  q,r\right)  =\left(  0,U\right)  $).

Let us now consider Case 2. In this case, we have $U\geq n$. Hence, $U-n\geq
0$, so that $U-n\in\mathbb{N}$ (remember that $\mathbb{N}=\left\{
0,1,2,\ldots\right\}  $). Also, $U-n<U$ (since $n$ is positive). But our
induction hypothesis said that Lemma \ref{lem.ent.quorem.existN} holds for
every $u\in\mathbb{N}$ satisfying $u<U$. Hence, in particular, Lemma
\ref{lem.ent.quorem.existN} holds for $u=U-n$ (since $U-n\in\mathbb{N}$ and
$U-n<U$). In other words, there exists \textbf{at least one} pair $\left(
q,r\right)  \in\mathbb{Z}\times\left\{  0,1,\ldots,n-1\right\}  $ such that
$U-n=qn+r$. Fix such a pair and denote it by $\left(  q_{0},r_{0}\right)  $.
Thus, $\left(  q_{0},r_{0}\right)  \in\mathbb{Z}\times\left\{  0,1,\ldots
,n-1\right\}  $ and $U-n=q_{0}n+r_{0}$.

From $U-n=q_{0}n+r_{0}$, we obtain $U=n+\left(  q_{0}n+r_{0}\right)  =\left(
q_{0}+1\right)  n+r_{0}$. Also, from $\left(  q_{0},r_{0}\right)
\in\mathbb{Z}\times\left\{  0,1,\ldots,n-1\right\}  $, we obtain $q_{0}%
\in\mathbb{Z}$ and $r_{0}\in\left\{  0,1,\ldots,n-1\right\}  $, and thus
$\left(  q_{0}+1,r_{0}\right)  \in\mathbb{Z}\times\left\{  0,1,\ldots
,n-1\right\}  $. Thus, the pair $\left(  q_{0}+1,r_{0}\right)  $ is a pair
$\left(  q,r\right)  \in\mathbb{Z}\times\left\{  0,1,\ldots,n-1\right\}  $
such that $U=qn+r$ (since $U=\left(  q_{0}+1\right)  n+r_{0}$). Therefore,
there exists \textbf{at least one} pair $\left(  q,r\right)  \in
\mathbb{Z}\times\left\{  0,1,\ldots,n-1\right\}  $ such that $U=qn+r$ (namely,
$\left(  q,r\right)  =\left(  q_{0}+1,r_{0}\right)  $).

Now, in each of the two Cases 1 and 2, we have shown that there exists
\textbf{at least one} pair $\left(  q,r\right)  \in\mathbb{Z}\times\left\{
0,1,\ldots,n-1\right\}  $ such that $U=qn+r$. Hence, this holds always. In
other words, Lemma \ref{lem.ent.quorem.existN} holds for $u=U$. This completes
the induction step; thus, Lemma \ref{lem.ent.quorem.existN} is proven by
strong induction.
\end{proof}

In order to derive Lemma \ref{lem.ent.quorem.exist} from Lemma
\ref{lem.ent.quorem.existN} (that is, to extend Lemma
\ref{lem.ent.quorem.existN} to the case of negative $u$), we shall need a
simple but important trick:

\begin{lemma}
\label{lem.ent.cong-to-nonneg}Let $n$ be a positive integer. Let
$u\in\mathbb{Z}$. Then, there exists a $v\in\mathbb{N}$ such that $u\equiv
v\operatorname{mod}n$.
\end{lemma}

\begin{proof}
[Proof of Lemma \ref{lem.ent.cong-to-nonneg}.]We are in one of the following
two cases:

\textit{Case 1:} We have $u\geq0$.

\textit{Case 2:} We have $u<0$.

Let us first consider Case 1. In this case, we have $u\geq0$. Thus,
$u\in\mathbb{N}$. Also, $u\equiv u\operatorname{mod}n$ (by Proposition
\ref{prop.ent.mod.basics} \textbf{(a)}). Thus, there exists a $v\in\mathbb{N}$
such that $u\equiv v\operatorname{mod}n$ (namely, $v=u$). This proves Lemma
\ref{lem.ent.cong-to-nonneg} in Case 1.

Let us now consider Case 2. In this case, we have $u<0$. Hence, $-u>0$. Now,
$u-\left(  n-1\right)  \left(  -u\right)  =nu$ is divisible by $n$ (since
$u\in\mathbb{Z}$). In other words, $n\mid u-\left(  n-1\right)  \left(
-u\right)  $. In other words, $u\equiv\left(  n-1\right)  \left(  -u\right)
\operatorname{mod}n$. Moreover, $n\geq1$ (since $n$ is a positive integer), so
that $n-1\geq0$. We can multiply this inequality with $-u$ (since $-u>0$), and
thus obtain $\left(  n-1\right)  \left(  -u\right)  \geq0\left(  -u\right)
=0$. In other words, $\left(  n-1\right)  \left(  -u\right)  \in\mathbb{N}$.
Thus, there exists a $v\in\mathbb{N}$ such that $u\equiv v\operatorname{mod}n$
(namely, $v=\left(  n-1\right)  \left(  -u\right)  $). This proves Lemma
\ref{lem.ent.cong-to-nonneg} in Case 2.

We have now proven Lemma \ref{lem.ent.cong-to-nonneg} in both Cases 1 and 2;
hence, Lemma \ref{lem.ent.cong-to-nonneg} always holds.
\end{proof}

\begin{proof}
[Proof of Lemma \ref{lem.ent.quorem.exist}.]Lemma \ref{lem.ent.cong-to-nonneg}
shows that there exists a $v\in\mathbb{N}$ such that $u\equiv
v\operatorname{mod}n$. Consider this $v$.

Note that $n\mid u-v$ (since $u\equiv v\operatorname{mod}n$). In other words,
there exists an integer $c$ such that $u-v=nc$. Consider this $c$. From
$u-v=nc$, we obtain $u=v+nc$.

Lemma \ref{lem.ent.quorem.existN} (applied to $v$ instead of $u$) yields that
there exists \textbf{at least one} pair $\left(  q,r\right)  \in
\mathbb{Z}\times\left\{  0,1,\ldots,n-1\right\}  $ such that $v=qn+r$. Fix
such a pair, and denote it by $\left(  q_{0},r_{0}\right)  $. Thus, $\left(
q_{0},r_{0}\right)  \in\mathbb{Z}\times\left\{  0,1,\ldots,n-1\right\}  $ and
$v=q_{0}n+r_{0}$. Now,%
\[
u=\underbrace{v}_{=q_{0}n+r_{0}}+nc=\left(  q_{0}n+r_{0}\right)  +nc=\left(
q_{0}+c\right)  n+r_{0}.
\]
Also, from $\left(  q_{0},r_{0}\right)  \in\mathbb{Z}\times\left\{
0,1,\ldots,n-1\right\}  $, we obtain $q_{0}\in\mathbb{Z}$ and $r_{0}%
\in\left\{  0,1,\ldots,n-1\right\}  $, and thus $\left(  q_{0}+c,r_{0}\right)
\in\mathbb{Z}\times\left\{  0,1,\ldots,n-1\right\}  $. Thus, the pair $\left(
q_{0}+c,r_{0}\right)  $ is a pair $\left(  q,r\right)  \in\mathbb{Z}%
\times\left\{  0,1,\ldots,n-1\right\}  $ such that $u=qn+r$ (since $u=\left(
q_{0}+c\right)  n+r_{0}$). Therefore, there exists \textbf{at least one} pair
$\left(  q,r\right)  \in\mathbb{Z}\times\left\{  0,1,\ldots,n-1\right\}  $
such that $u=qn+r$ (namely, $\left(  q,r\right)  =\left(  q_{0}+c,r_{0}%
\right)  $). This proves Lemma \ref{lem.ent.quorem.exist}.
\end{proof}

\begin{proof}
[Proof of Theorem \ref{thm.ent.quorem.full}.]Theorem \ref{thm.ent.quorem.full}
follows by combining Lemma \ref{lem.ent.quorem.exist} with Lemma
\ref{lem.ent.quorem.unique}.
\end{proof}

The following properties of the quotient and the remainder are simple but will
be used all the time:

\begin{corollary}
\label{cor.ent.quo-rem.remmod}Let $n$ be a positive integer. Let
$u\in\mathbb{Z}$.

\textbf{(a)} Then, $u\%n\in\left\{  0,1,\ldots,n-1\right\}  $ and $u\%n\equiv
u\operatorname{mod}n$.

\textbf{(b)} We have $n\mid u$ if and only if $u\%n=0$.

\textbf{(c)} If $c\in\left\{  0,1,\ldots,n-1\right\}  $ is such that $c\equiv
u\operatorname{mod}n$, then $c=u\%n$.

\textbf{(d)} We have $u=\left(  u//n\right)  n+\left(  u\%n\right)  $.
\end{corollary}

Before we prove this corollary, let us explain its purpose. Corollary
\ref{cor.ent.quo-rem.remmod} \textbf{(a)} says that $u\%n$ is a number in the
set $\left\{  0,1,\ldots,n-1\right\}  $ that is congruent to $u$ modulo $n$.
Corollary \ref{cor.ent.quo-rem.remmod} \textbf{(c)} says that $u\%n$ is the
\textbf{only} such number (as it says that any further such number $c$ must be
equal to $u\%n$). Corollary \ref{cor.ent.quo-rem.remmod} \textbf{(b)} gives an
algorithm to check whether $n\mid u$ holds (namely, compute $u\%n$ and check
whether $u\%n=0$). Corollary \ref{cor.ent.quo-rem.remmod} \textbf{(d)} is a
trivial consequence of the definition of quotient and remainder.

\begin{proof}
[Proof of Corollary \ref{cor.ent.quo-rem.remmod}.]Theorem
\ref{thm.ent.quorem.full} says that there is a unique pair $\left(
q,r\right)  \in\mathbb{Z}\times\left\{  0,1,\ldots,n-1\right\}  $ such that
$u=qn+r$. Consider this pair $\left(  q,r\right)  $. The uniqueness of this
pair can be restated as follows: If $\left(  q^{\prime},r^{\prime}\right)
\in\mathbb{Z}\times\left\{  0,1,\ldots,n-1\right\}  $ is any further pair such
that $u=q^{\prime}n+r^{\prime}$, then%
\begin{equation}
\left(  q^{\prime},r^{\prime}\right)  =\left(  q,r\right)  .
\label{pf.cor.ent.quo-rem.remmod.uni}%
\end{equation}


Recall that $u\%n$ was defined to be $r$ (in Definition \ref{def.ent.quorem}
\textbf{(b)}). Thus, $u\%n=r$. Now, $n\mid qn=u-r$ (since $u=qn+r$). In other
words, $u\equiv r\operatorname{mod}n$. Hence, $r\equiv u\operatorname{mod}n$
(by Proposition \ref{prop.ent.mod.basics} \textbf{(c)}). This rewrites as
$u\%n\equiv u\operatorname{mod}n$ (since $r=u\%n$).

Furthermore, $u\%n=r\in\left\{  0,1,\ldots,n-1\right\}  $ (since $\left(
q,r\right)  \in\mathbb{Z}\times\left\{  0,1,\ldots,n-1\right\}  $). This
completes the proof of Corollary \ref{cor.ent.quo-rem.remmod} \textbf{(a)}.

Also, $u//n$ was defined to be $q$ (in Definition \ref{def.ent.quorem}
\textbf{(a)}). Hence, $u//n=q$. Now,%
\[
u=\underbrace{q}_{=u//n}n+\underbrace{r}_{=u\%n}=\left(  u//n\right)
n+\left(  u\%n\right)  .
\]
This proves Corollary \ref{cor.ent.quo-rem.remmod} \textbf{(d)}.

\textbf{(b)} $\Longrightarrow:$ Assume that $n\mid u$. We must prove that
$u\%n=0$.

We have $n\mid u$. In other words, there exists some integer $w$ such that
$u=nw$. Consider this $w$.

We have $n-1\in\mathbb{N}$ (since $n$ is a positive integer), thus
$0\in\left\{  0,1,\ldots,n-1\right\}  $. Hence, $\left(  w,0\right)
\in\mathbb{Z}\times\left\{  0,1,\ldots,n-1\right\}  $ (since $w\in\mathbb{Z}%
$). Also, $u=nw=wn=wn+0$. Hence, (\ref{pf.cor.ent.quo-rem.remmod.uni})
(applied to $\left(  q^{\prime},r^{\prime}\right)  =\left(  w,0\right)  $)
yields $\left(  w,0\right)  =\left(  q,r\right)  $. In other words, $w=q$ and
$0=r$. Hence, $r=0$, so that $u\%n=r=0$. This proves the \textquotedblleft%
$\Longrightarrow$\textquotedblright\ implication of Corollary
\ref{cor.ent.quo-rem.remmod} \textbf{(b)}.

$\Longleftarrow:$ Assume that $u\%n=0$. We must prove that $n\mid u$.

We have $u=qn+\underbrace{r}_{=u\%n=0}=qn=nq$. Thus, $n\mid u$. This proves
the \textquotedblleft$\Longleftarrow$\textquotedblright\ implication of
Corollary \ref{cor.ent.quo-rem.remmod} \textbf{(b)}.

\textbf{(c)} Let $c\in\left\{  0,1,\ldots,n-1\right\}  $ be such that $c\equiv
u\operatorname{mod}n$.

We have $c\equiv u\operatorname{mod}n$. In other words, $n\mid c-u$. In other
words, there exists some integer $w$ such that $c-u=nw$. Consider this $w$.

From $-w\in\mathbb{Z}$ and $c\in\left\{  0,1,\ldots,n-1\right\}  $, we obtain
$\left(  -w,c\right)  \in\mathbb{Z}\times\left\{  0,1,\ldots,n-1\right\}  $.
Also, from $c-u=nw$, we obtain $u=c-nw=\left(  -w\right)  n+c$. Hence,
(\ref{pf.cor.ent.quo-rem.remmod.uni}) (applied to $\left(  q^{\prime
},r^{\prime}\right)  =\left(  -w,c\right)  $) yields $\left(  -w,c\right)
=\left(  q,r\right)  $. In other words, $-w=q$ and $c=r$. Hence, $c=r=u\%n$.
This proves Corollary \ref{cor.ent.quo-rem.remmod} \textbf{(c)}.
\end{proof}

\begin{exercise}
\label{exe.ent.quo-rem.mod=rem}Let $n$ be a positive integer. Let $u$ and $v$
be integers. Prove that $u\equiv v\operatorname{mod}n$ if and only if
$u\%n=v\%n$.
\end{exercise}

\subsection{Even and odd numbers}

Recall the following:

\begin{definition}
\label{def.ent.even-odd}Let $u$ be an integer.

\textbf{(a)} We say that $u$ is \textit{even} if $u$ is divisible by $2$.

\textbf{(b)} We say that $u$ is \textit{odd }if $u$ is not divisible by $2$.
\end{definition}

So an integer is either even or odd (but not both at the same time). The
following exercise collects various properties of even and odd integers:

\begin{exercise}
\label{exe.ent.even-odd.1}Let $u$ be an integer.

\textbf{(a)} Prove that $u$ is even if and only if $u\%2=0$.

\textbf{(b)} Prove that $u$ is odd if and only if $u\%2=1$.

\textbf{(c)} Prove that $u$ is even if and only if $u\equiv0\operatorname{mod}%
2$.

\textbf{(d)} Prove that $u$ is odd if and only if $u\equiv1\operatorname{mod}%
2$.

\textbf{(e)} Prove that $u$ is odd if and only if $u+1$ is even.

\textbf{(f)} Prove that exactly one of the two numbers $u$ and $u+1$ is even.

\textbf{(g)} Prove that $u\left(  u+1\right)  \equiv0\operatorname{mod}2$.

\textbf{(h)} Prove that $u^{2}\equiv-u\equiv u\operatorname{mod}2$.

\textbf{(i)} Let $v$ be a further integer. Prove that $u\equiv
v\operatorname{mod}2$ holds if and only if $u$ and $v$ are either both odd or
both even.
\end{exercise}

\begin{proof}
[Solution to Exercise \ref{exe.ent.even-odd.1}.]TODO.
\end{proof}

\begin{exercise}
\label{exe.ent.even-odd-sumsq}\textbf{(a)} Prove that each even integer $u$
satisfies $u^{2}\equiv0\operatorname{mod}4$.

\textbf{(b)} Prove that each odd integer $u$ satisfies $u^{2}\equiv
1\operatorname{mod}4$.

\textbf{(c)} Prove that no two integers $x$ and $y$ satisfy $x^{2}+y^{2}%
\equiv3\operatorname{mod}4$.

\textbf{(d)} Prove that if $x$ and $y$ are two integers satisfying
$x^{2}+y^{2}\equiv2\operatorname{mod}4$, then $x$ and $y$ are both odd.
\end{exercise}

\begin{fineprint}
\begin{proof}
[Solution to Exercise \ref{exe.ent.even-odd-sumsq}.]TODO.
\end{proof}
\end{fineprint}

Exercise \ref{exe.ent.even-odd-sumsq} \textbf{(c)} establishes our previous
experimental observation that an integer of the form $4k+3$ with integer $k$
(that is, an integer that is larger by $3$ than a multiple of $4$) can never
be written as a sum of two perfect squares.

\begin{center}
\textbf{2019-02-04 lecture}
\end{center}

\subsection{The floor function}

\begin{definition}
\label{def.ent.floor}Let $x$ be a real number. Then, $\left\lfloor
x\right\rfloor $ is defined to be the unique integer $n$ satisfying $n\leq
x<n+1$. This integer $\left\lfloor x\right\rfloor $ is called the
\textit{floor} of $x$, or the \textit{integer part} of $x$.
\end{definition}

\begin{remark}
\label{rmk.ent.floor}\textbf{(a)} Why is $\left\lfloor x\right\rfloor $
well-defined? I mean, why does the unique integer $n$ in Definition
\ref{def.ent.floor} exist, and why is it unique? This question is trickier
than it sounds and relies on the construction of real numbers. However, in the
case when $x$ is rational, the well-definedness of $\left\lfloor
x\right\rfloor $ follows from Proposition \ref{prop.ent.floor.quorem} below.

\textbf{(b)} What we call $\left\lfloor x\right\rfloor $ is typically called
$\left[  x\right]  $ in older books (such as \cite{NiZuMo91}). I suggest
avoiding the notation $\left[  x\right]  $ wherever possible; it has too many
different meanings (whereas $\left\lfloor x\right\rfloor $ almost always means
the floor of $x$).

\textbf{(c)} The map $\mathbb{R}\rightarrow\mathbb{Z},\ x\mapsto\left\lfloor
x\right\rfloor $ is called the \textit{floor function} or the \textit{greatest
integer function}.

There is also a \textit{ceiling function}, which sends each $x\in\mathbb{R}$
to the unique integer $n$ satisfying $n-1<x\leq n$; this latter integer is
called $\left\lceil x\right\rceil $. The two functions are connected by the
rule $\left\lceil x\right\rceil =-\left\lfloor -x\right\rfloor $ (for all
$x\in\mathbb{R}$).

The floor and the ceiling functions are some of the simplest examples of
discontinuous functions.

\textbf{(d)} Here are some examples of floors:%
\begin{align*}
\left\lfloor n\right\rfloor  &  =n\ \ \ \ \ \ \ \ \ \ \text{for every }%
n\in\mathbb{Z};\\
\left\lfloor 1.32\right\rfloor  &  =1;\ \ \ \ \ \ \ \ \ \ \left\lfloor
\pi\right\rfloor =3;\ \ \ \ \ \ \ \ \ \ \left\lfloor 0.98\right\rfloor =0;\\
\left\lfloor -2.3\right\rfloor  &  =-3;\ \ \ \ \ \ \ \ \ \ \left\lfloor
-0.4\right\rfloor =-1.
\end{align*}


\textbf{(e)} You might have the impression that $\left\lfloor x\right\rfloor $
is \textquotedblleft what remains from $x$ if the digits behind the comma are
removed\textquotedblright. This impression is highly imprecise. For one, it is
completely broken for negative $x$ (for example, $\left\lfloor
-2.3\right\rfloor $ is $-3$, not $-2$). But more importantly, the operation of
\textquotedblleft removing the digits behind the comma\textquotedblright\ from
a number is not well-defined; the periodic decimal representations
$0.999\ldots$ and $1.000\ldots$ belong to the same real number ($1$), but
removing their digits behind the comma leaves us with different integers.

\textbf{(f)} A related map is the map $\mathbb{R}\rightarrow\mathbb{Z}%
,\ x\mapsto\left\lfloor x+\dfrac{1}{2}\right\rfloor $. It sends each real $x$
to the integer that is closest to $x$, choosing the larger one in the case of
a tie. This is one of the many things that are commonly known as
\textquotedblleft rounding\textquotedblright\ a number.
\end{remark}

\begin{proposition}
\label{prop.ent.floor.quorem}Let $a$ and $b$ be integers such that $b>0$.
Then, $\left\lfloor \dfrac{a}{b}\right\rfloor $ is well-defined and equals
$a//b$.
\end{proposition}

\begin{proof}
[Proof of Proposition \ref{prop.ent.floor.quorem}.]This is a rather easy and
neat exercise. A full proof can be found in \cite[proof of Proposition
1.1.3]{floor}.
\end{proof}

\subsection{Common divisors, the Euclidean algorithm and the Bezout theorem}

\subsubsection{Divisors}

\begin{definition}
Let $b\in\mathbb{Z}$. The \textit{divisors} of $b$ are defined as the integers
that divide $b$.
\end{definition}

Be aware that some authors use a mildly different definition of
\textquotedblleft divisors\textquotedblright; namely, they additionally
require them to be positive. We don't make such a requirement.

For example, the divisors of $6$ are $-6,-3,-2,-1,1,2,3,6$. Of course, the
negative divisors of an integer $b$ are merely the reflections of the positive
divisors through the origin\footnote{\textquotedblleft Reflection through the
origin\textquotedblright\ is just a poetic way to say \textquotedblleft
negative\textquotedblright; i.e., the reflection of a number $a$ through the
origin is $-a$.} (this follows easily from Proposition \ref{prop.ent.div.1}
\textbf{(a)}); thus, the positive divisors are usually the only ones of interest.

Here are some basic properties of divisors:

\begin{proposition}
\label{prop.ent.divisors.find}\textbf{(a)} If $b\in\mathbb{Z}$, then $1$ and
$b$ are divisors of $b$.

\textbf{(b)} The divisors of $0$ are all the integers.

\textbf{(c)} Let $b\in\mathbb{Z}$ be nonzero. Then, all divisors of $b$ belong
to the set $\left\{  -\left\vert b\right\vert ,-\left\vert b\right\vert
+1,\ldots,\left\vert b\right\vert \right\}  \setminus\left\{  0\right\}  $.
\end{proposition}

\begin{proof}
[Proof of Proposition \ref{prop.ent.divisors.find}.]\textbf{(a)} Clearly,
$1\mid b$ (since $b=1b$), so that $1$ is a divisor of $b$. Also, $b\mid b$
(since $b=b\cdot1$), so that $b$ is a divisor of $b$.

\textbf{(b)} Each integer $a$ divides $0$ (since $0=a\cdot0$) and thus is a
divisor of $0$. This proves Proposition \ref{prop.ent.divisors.find}
\textbf{(b)}.

\textbf{(c)} Let $a$ be a divisor of $b$. Then, $a$ divides $b$. In other
words, $a\mid b$. Hence, Proposition \ref{prop.ent.div.1} \textbf{(b)} yields
$\left\vert a\right\vert \leq\left\vert b\right\vert $ (since $b\neq0$). But
$\left\vert a\right\vert \geq a$ (since $\left\vert x\right\vert \geq x$ for
each $x\in\mathbb{R}$), so that $a\leq\left\vert a\right\vert \leq\left\vert
b\right\vert $. Also, $\left\vert a\right\vert \geq-a$ (since $\left\vert
x\right\vert \geq-x$ for each $x\in\mathbb{R}$) and thus $-a\leq\left\vert
a\right\vert \leq\left\vert b\right\vert $, so that $a\geq-\left\vert
b\right\vert $. Combining this with $a\leq\left\vert b\right\vert $, we obtain
$-\left\vert b\right\vert \leq a\leq\left\vert b\right\vert $ and thus
$a\in\left\{  -\left\vert b\right\vert ,-\left\vert b\right\vert
+1,\ldots,\left\vert b\right\vert \right\}  $ (since $a$ is an integer).

From Example \ref{exa.ent.div.triv} \textbf{(c)}, we know that $0\mid b$ only
when $b=0$. Thus, we don't have $0\mid b$ (since $b\neq0$).

If we had $a=0$, then we would have $0=a\mid b$, which would contradict the
fact that we don't have $0\mid b$. Thus, we cannot have $a=0$. Hence, $a\neq
0$. Combining $a\in\left\{  -\left\vert b\right\vert ,-\left\vert b\right\vert
+1,\ldots,\left\vert b\right\vert \right\}  $ with $a\neq0$, we obtain
$a\in\left\{  -\left\vert b\right\vert ,-\left\vert b\right\vert
+1,\ldots,\left\vert b\right\vert \right\}  \setminus\left\{  0\right\}  $.

We have proven this for each divisor $a$ of $b$. Thus, we conclude that all
divisors of $b$ belong to the set $\left\{  -\left\vert b\right\vert
,-\left\vert b\right\vert +1,\ldots,\left\vert b\right\vert \right\}
\setminus\left\{  0\right\}  $. This proves Proposition
\ref{prop.ent.divisors.find} \textbf{(c)}.
\end{proof}

Thanks to Proposition \ref{prop.ent.divisors.find}, we have a method to find
all divisors of an integer $b$: If $b=0$, then Proposition
\ref{prop.ent.divisors.find} \textbf{(b)} directly yields the result;
otherwise, Proposition \ref{prop.ent.divisors.find} \textbf{(c)} shows that
there is only a finite set of numbers we have to check. When $b$ is large,
this is slow, but to some extent that is because the problem is
computationally hard (or at least suspected to be hard).

\subsubsection{Common divisors}

It is somewhat more interesting to consider the common divisors of two or more integers:

\begin{definition}
\label{def.ent.Div}Let $b_{1},b_{2},\ldots,b_{k}$ be finitely many integers.
Then, the \textit{common divisors} of $b_{1},b_{2},\ldots,b_{k}$ are defined
to be the integers $a$ that satisfy%
\begin{equation}
\left(  a\mid b_{i}\text{ for all }i\in\left\{  1,2,\ldots,k\right\}  \right)
\label{eq.def.ent.Div.cond}%
\end{equation}
(in other words, that divide all of the integers $b_{1},b_{2},\ldots,b_{k}$).
We let $\operatorname*{Div}\left(  b_{1},b_{2},\ldots,b_{k}\right)  $ denote
the set of these common divisors.
\end{definition}

Note that the concept of common divisors encompasses the concept of divisors:
The common divisors of a single integer $b$ are merely the divisors of $b$.
Thus, $\operatorname*{Div}\left(  b\right)  $ is the set of all divisors of
$b$ whenever $b\in\mathbb{Z}$. (Of course, speaking of \textquotedblleft
common divisors\textquotedblright\ of just one integer is like speaking of a
conspiracy of just one person. But the definition fits, and we algebraists
don't exclude cases just because they are ridiculous.)

(Also, the common divisors of an empty list of integers are all the integers,
because the requirement (\ref{eq.def.ent.Div.cond}) is vacuously true for
$k=0$. In other words, $\operatorname*{Div}\left(  {}\right)  =\mathbb{Z}$.)

Here are some more interesting examples of common divisors:

\begin{example}
\textbf{(a)} The common divisors of $6$ and $8$ are $-2,-1,1,2$. (In order to
see this, just observe that the divisors of $6$ are $-6,-3,-2,-1,1,2,3,6$,
whereas the divisors of $8$ are $-8,-4,-2,-1,1,2,4,8$; now you can find the
common divisors of $6$ and $8$ by taking the numbers common to these two
lists.) Thus,%
\[
\operatorname*{Div}\left(  6,8\right)  =\left\{  -2,-1,1,2\right\}  .
\]


\textbf{(b)} The common divisors of $6$ and $14$ are $-2,-1,1,2$ again. (In
order to see this, just observe that the divisors of $6$ are
$-6,-3,-2,-1,1,2,3,6$, whereas the divisors of $14$ are
$-14,-7,-2,-1,1,2,7,14$.)

\textbf{(c)} The common divisors of $6$, $10$ and $15$ are $-1$ and $1$. (In
order to see this, note that:

\begin{itemize}
\item The divisors of $6$ are $-6,-3,-2,-1,1,2,3,6$.

\item The divisors of $10$ are $-10,-5,-2,-1,1,2,5,10$.

\item The divisors of $15$ are $-15,-5,-3,-1,1,3,5,15$.
\end{itemize}

The only numbers common to these three lists are $-1$ and $1$.) However:

\begin{itemize}
\item The common divisors of $6$ and $10$ are $-2,-1,1,2$.

\item The common divisors of $6$ and $15$ are $-3,-1,1,3$.

\item The common divisors of $10$ and $15$ are $-5,-1,1,5$.
\end{itemize}

This illustrates the fact that three numbers can have pairwise nontrivial
common divisors (where \textquotedblleft nontrivial\textquotedblright\ means
\textquotedblleft distinct from $1$ and $-1$\textquotedblright), but the
common divisor of all three of them may still be just $1$ and $-1$.
\end{example}

\begin{proposition}
\label{prop.ent.Div.fin}Let $b_{1},b_{2},\ldots,b_{k}$ be finitely many
integers that are not all $0$. Then, the set $\operatorname*{Div}\left(
b_{1},b_{2},\ldots,b_{k}\right)  $ has a largest element, and this largest
element is a positive integer.
\end{proposition}

\begin{proof}
[Proof of Proposition \ref{prop.ent.Div.fin}.]The integer $1$ satisfies
$\left(  1\mid b_{i}\text{ for all }i\in\left\{  1,2,\ldots,k\right\}
\right)  $. Thus, $1$ is a common divisor of $b_{1},b_{2},\ldots,b_{k}$ (by
the definition of a \textquotedblleft common divisor\textquotedblright). In
other words, $1\in\operatorname*{Div}\left(  b_{1},b_{2},\ldots,b_{k}\right)
$ (by the definition of $\operatorname*{Div}\left(  b_{1},b_{2},\ldots
,b_{k}\right)  $). Hence, the set $\operatorname*{Div}\left(  b_{1}%
,b_{2},\ldots,b_{k}\right)  $ is nonempty.

Moreover, it is easy to see that the set $\operatorname*{Div}\left(
b_{1},b_{2},\ldots,b_{k}\right)  $ is finite.

\begin{fineprint}
[\textit{Proof:} We have assumed that $b_{1},b_{2},\ldots,b_{k}$ are not all
$0$. In other words, there exists a $j\in\left\{  1,2,\ldots,k\right\}  $ such
that $b_{j}$ is nonzero. Consider such a $j$.

Let $d\in\operatorname*{Div}\left(  b_{1},b_{2},\ldots,b_{k}\right)  $. Thus,
$d$ is a common divisor of $b_{1},b_{2},\ldots,b_{k}$ (by the definition of
$\operatorname*{Div}\left(  b_{1},b_{2},\ldots,b_{k}\right)  $). In other
words, $d\mid b_{i}$ for all $i\in\left\{  1,2,\ldots,k\right\}  $ (by the
definition of \textquotedblleft common divisor\textquotedblright). Applying
this to $i=j$, we obtain $d\mid b_{j}$. Hence, $d$ is a divisor of $b_{j}$.
But Proposition \ref{prop.ent.divisors.find} \textbf{(c)} (applied to
$b=b_{j}$) shows that all divisors of $b_{j}$ belong to the set $\left\{
-\left\vert b_{j}\right\vert ,-\left\vert b_{j}\right\vert +1,\ldots
,\left\vert b_{j}\right\vert \right\}  \setminus\left\{  0\right\}  $. Hence,
$d$ must belong to this set (since $d$ is a divisor of $b_{j}$). In other
words, $d\in\left\{  -\left\vert b_{j}\right\vert ,-\left\vert b_{j}%
\right\vert +1,\ldots,\left\vert b_{j}\right\vert \right\}  \setminus\left\{
0\right\}  $.

Now, forget that we fixed $d$. We thus have shown that $d\in\left\{
-\left\vert b_{j}\right\vert ,-\left\vert b_{j}\right\vert +1,\ldots
,\left\vert b_{j}\right\vert \right\}  \setminus\left\{  0\right\}  $ for each
$d\in\operatorname*{Div}\left(  b_{1},b_{2},\ldots,b_{k}\right)  $. In other
words,
\[
\operatorname*{Div}\left(  b_{1},b_{2},\ldots,b_{k}\right)  \subseteq\left\{
-\left\vert b_{j}\right\vert ,-\left\vert b_{j}\right\vert +1,\ldots
,\left\vert b_{j}\right\vert \right\}  \setminus\left\{  0\right\}  .
\]
Thus, the set $\operatorname*{Div}\left(  b_{1},b_{2},\ldots,b_{k}\right)  $
is finite (since the set $\left\{  -\left\vert b_{j}\right\vert ,-\left\vert
b_{j}\right\vert +1,\ldots,\left\vert b_{j}\right\vert \right\}
\setminus\left\{  0\right\}  $ is finite).]
\end{fineprint}

Now we know that the set $\operatorname*{Div}\left(  b_{1},b_{2},\ldots
,b_{k}\right)  $ is a nonempty finite set of integers. Thus, this set
$\operatorname*{Div}\left(  b_{1},b_{2},\ldots,b_{k}\right)  $ has a largest
element (since every nonempty finite set of integers has a largest element).
It remains to prove that this largest element is a positive integer.

Let $g$ be this largest element. Thus, we must prove that $g$ is a positive
integer. Clearly, $g$ is an integer (since all elements of
$\operatorname*{Div}\left(  b_{1},b_{2},\ldots,b_{k}\right)  $ are integers);
it thus remains to show that $g$ is positive.

The element $g$ is the largest element of the set $\operatorname*{Div}\left(
b_{1},b_{2},\ldots,b_{k}\right)  $, and thus is $\geq$ to every element of
this set. In other words, $g\geq x$ for each $x\in\operatorname*{Div}\left(
b_{1},b_{2},\ldots,b_{k}\right)  $. Applying this to $x=1$, we obtain $g\geq1$
(since $1\in\operatorname*{Div}\left(  b_{1},b_{2},\ldots,b_{k}\right)  $).
Hence, $g$ is positive. This completes the proof of Proposition
\ref{prop.ent.Div.fin}.
\end{proof}

\subsubsection{Greatest common divisors}

Proposition \ref{prop.ent.Div.fin} allows us to make a crucial definition:

\begin{definition}
\label{def.ent.gcd.gcd}Let $b_{1},b_{2},\ldots,b_{k}$ be finitely many
integers. The \textit{greatest common divisor} of $b_{1},b_{2},\ldots,b_{k}$
is defined as follows:

\begin{itemize}
\item If $b_{1},b_{2},\ldots,b_{k}$ are not all $0$, then it is defined as the
largest element of the set $\operatorname*{Div}\left(  b_{1},b_{2}%
,\ldots,b_{k}\right)  $. This largest element is well-defined (by Proposition
\ref{prop.ent.Div.fin}), and is a positive integer (by Proposition
\ref{prop.ent.Div.fin} again).

\item If $b_{1},b_{2},\ldots,b_{k}$ are all $0$, then it is defined to be $0$.
(This is a slight abuse of the word \textquotedblleft greatest common
divisor\textquotedblright, because $0$ is not actually the greatest among the
common divisors of $b_{1},b_{2},\ldots,b_{k}$ in this case. In fact, when
$b_{1},b_{2},\ldots,b_{k}$ are all $0$, \textbf{every} integer is a common
divisor of $b_{1},b_{2},\ldots,b_{k}$, so that there is no greatest among
these common divisors, because there is no \textquotedblleft greatest
integer\textquotedblright. Nevertheless, defining the greatest common divisor
of $b_{1},b_{2},\ldots,b_{k}$ to be $0$ in this case will prove to be a good
decision, as it will greatly reduce the number of exceptions in our results.)
\end{itemize}

Thus, in either case, this greatest common divisor is a nonnegative integer.
We denote it by $\gcd\left(  b_{1},b_{2},\ldots,b_{k}\right)  $.

We shall also use the word \textquotedblleft\textit{gcd}\textquotedblright\ as
shorthand for \textquotedblleft greatest common divisor\textquotedblright.
\end{definition}

The greatest common divisors you will most commonly see are those of two
integers. Indeed, any other gcd can be rewritten in terms of these: for
example,%
\[
\gcd\left(  a,b,c,d,e\right)  =\gcd\left(  a,\gcd\left(  b,\gcd\left(
c,\gcd\left(  d,e\right)  \right)  \right)  \right)
\]
for all $a,b,c,d,e\in\mathbb{Z}$. We may prove this later in the homework.

TODO: Do this?

We observe several properties of greatest common divisors:

\begin{proposition}
\label{prop.ent.gcd.props1}\textbf{(a)} We have $\gcd\left(  a,0\right)
=\gcd\left(  a\right)  =\left\vert a\right\vert $ for all $a\in\mathbb{Z}$.

\textbf{(b)} We have $\gcd\left(  a,b\right)  =\gcd\left(  b,a\right)  $ for
all $a,b\in\mathbb{Z}$.

\textbf{(c)} We have $\gcd\left(  a,ua+b\right)  =\gcd\left(  a,b\right)  $
for all $a,b,u\in\mathbb{Z}$.

\textbf{(d)} If $a,b,c\in\mathbb{Z}$ satisfy $b\equiv c\operatorname{mod}a$,
then $\gcd\left(  a,b\right)  =\gcd\left(  a,c\right)  $.

\textbf{(e)} If $a,b\in\mathbb{Z}$ are such that $a$ is positive, then
$\gcd\left(  a,b\right)  =\gcd\left(  a,b\%a\right)  $.

\textbf{(f)} We have $\gcd\left(  a,b\right)  \mid a$ and $\gcd\left(
a,b\right)  \mid b$ for all $a,b\in\mathbb{Z}$.

\textbf{(g)} We have $\gcd\left(  -a,b\right)  =\gcd\left(  a,b\right)  $ for
all $a,b\in\mathbb{Z}$.

\textbf{(h)} We have $\gcd\left(  a,-b\right)  =\gcd\left(  a,b\right)  $ for
all $a,b\in\mathbb{Z}$.

\textbf{(i)} If $a,b\in\mathbb{Z}$ satisfy $a\mid b$, then $\gcd\left(
a,b\right)  =\left\vert a\right\vert $.
\end{proposition}

Proposition \ref{prop.ent.divisors.find} is not difficult and we could start
proving it right away. However, such a proof would require some annoying case
distinctions due to the special treatment that the \textquotedblleft%
$b_{1},b_{2},\ldots,b_{k}$ are all $0$\textquotedblright\ case required in
Definition \ref{def.ent.gcd.gcd}. Fortunately, we can circumnavigate these
annoyances by stating a simple rule how the gcd of $k$ integers $b_{1}%
,b_{2},\ldots,b_{k}$ can be computed from their set of common divisors
(including the case when $b_{1},b_{2},\ldots,b_{k}$ are all $0$):

\begin{lemma}
\label{lem.ent.gcd.through-Div}Let $b_{1},b_{2},\ldots,b_{k}$ be finitely many
integers. Then,%
\[
\gcd\left(  b_{1},b_{2},\ldots,b_{k}\right)  =%
\begin{cases}
\max\left(  \operatorname*{Div}\left(  b_{1},b_{2},\ldots,b_{k}\right)
\right)  , & \text{if }0\notin\operatorname*{Div}\left(  b_{1},b_{2}%
,\ldots,b_{k}\right)  ;\\
0, & \text{if }0\in\operatorname*{Div}\left(  b_{1},b_{2},\ldots,b_{k}\right)
.
\end{cases}
\]
(Here, $\max S$ denotes the largest element of a set $S$ of integers, whenever
this largest element exists.)
\end{lemma}

\begin{proof}
[Proof of Lemma \ref{lem.ent.gcd.through-Div}.]We are in one of the following
two cases:

\textit{Case 1:} The integers $b_{1},b_{2},\ldots,b_{k}$ are not all $0$.

\textit{Case 2:} The integers $b_{1},b_{2},\ldots,b_{k}$ are all $0$.

Let us consider Case 1 first. In this case, the integers $b_{1},b_{2}%
,\ldots,b_{k}$ are not all $0$. Hence, $\gcd\left(  b_{1},b_{2},\ldots
,b_{k}\right)  $ is defined as the largest element of the set
$\operatorname*{Div}\left(  b_{1},b_{2},\ldots,b_{k}\right)  $ (by Definition
\ref{def.ent.gcd.gcd}). In other words,
\begin{equation}
\gcd\left(  b_{1},b_{2},\ldots,b_{k}\right)  =\max\left(  \operatorname*{Div}%
\left(  b_{1},b_{2},\ldots,b_{k}\right)  \right)  .
\label{pf.lem.ent.gcd.through-Div.c1.1}%
\end{equation}


On the other hand, $0\notin\operatorname*{Div}\left(  b_{1},b_{2},\ldots
,b_{k}\right)  $\ \ \ \ \footnote{\textit{Proof.} Assume the contrary. Thus,
$0\in\operatorname*{Div}\left(  b_{1},b_{2},\ldots,b_{k}\right)  $. In other
words, $0$ is a common divisor of $b_{1},b_{2},\ldots,b_{k}$ (by the
definition of $\operatorname*{Div}\left(  b_{1},b_{2},\ldots,b_{k}\right)  $).
In other words, $0\mid b_{i}$ for all $i\in\left\{  1,2,\ldots,k\right\}  $
(by the definition of \textquotedblleft common divisor\textquotedblright).
Thus, for all $i\in\left\{  1,2,\ldots,k\right\}  $, we have $b_{i}=0$ (since
$0\mid b_{i}$, so that $b_{i}=0c$ for some integer $c$; but this yields
$b_{i}=0c=0$). In other words, $b_{1},b_{2},\ldots,b_{k}$ are all $0$. But
this contradicts the fact that $b_{1},b_{2},\ldots,b_{k}$ are not all $0$.
This contradiction shows that our assumption was false, qed.}. Hence,%
\[%
\begin{cases}
\max\left(  \operatorname*{Div}\left(  b_{1},b_{2},\ldots,b_{k}\right)
\right)  , & \text{if }0\notin\operatorname*{Div}\left(  b_{1},b_{2}%
,\ldots,b_{k}\right)  ;\\
0, & \text{if }0\in\operatorname*{Div}\left(  b_{1},b_{2},\ldots,b_{k}\right)
\end{cases}
=\max\left(  \operatorname*{Div}\left(  b_{1},b_{2},\ldots,b_{k}\right)
\right)  .
\]
Comparing this with (\ref{pf.lem.ent.gcd.through-Div.c1.1}), we obtain%
\[
\gcd\left(  b_{1},b_{2},\ldots,b_{k}\right)  =%
\begin{cases}
\max\left(  \operatorname*{Div}\left(  b_{1},b_{2},\ldots,b_{k}\right)
\right)  , & \text{if }0\notin\operatorname*{Div}\left(  b_{1},b_{2}%
,\ldots,b_{k}\right)  ;\\
0, & \text{if }0\in\operatorname*{Div}\left(  b_{1},b_{2},\ldots,b_{k}\right)
.
\end{cases}
\]
Hence, Lemma \ref{lem.ent.gcd.through-Div} is proven in Case 1.

Let us now consider Case 2. In this case, the integers $b_{1},b_{2}%
,\ldots,b_{k}$ are all $0$. Hence, $\gcd\left(  b_{1},b_{2},\ldots
,b_{k}\right)  $ is defined as $0$ (by Definition \ref{def.ent.gcd.gcd}). In
other words,
\begin{equation}
\gcd\left(  b_{1},b_{2},\ldots,b_{k}\right)  =0.
\label{pf.lem.ent.gcd.through-Div.c2.1}%
\end{equation}


On the other hand, $0\in\operatorname*{Div}\left(  b_{1},b_{2},\ldots
,b_{k}\right)  $\ \ \ \ \footnote{\textit{Proof.} The integers $b_{1}%
,b_{2},\ldots,b_{k}$ are all $0$. In other words, $b_{i}=0$ for all
$i\in\left\{  1,2,\ldots,k\right\}  $. Hence, $0\mid b_{i}$ for all
$i\in\left\{  1,2,\ldots,k\right\}  $ (since each $i\in\left\{  1,2,\ldots
,k\right\}  $ satisfies $b_{i}=0=0\cdot0$). In other words, $0$ is a common
divisor of $b_{1},b_{2},\ldots,b_{k}$ (by the definition of \textquotedblleft
common divisor\textquotedblright). In other words, $0\in\operatorname*{Div}%
\left(  b_{1},b_{2},\ldots,b_{k}\right)  $ (by the definition of
$\operatorname*{Div}\left(  b_{1},b_{2},\ldots,b_{k}\right)  $).}. Hence,%
\[%
\begin{cases}
\max\left(  \operatorname*{Div}\left(  b_{1},b_{2},\ldots,b_{k}\right)
\right)  , & \text{if }0\notin\operatorname*{Div}\left(  b_{1},b_{2}%
,\ldots,b_{k}\right)  ;\\
0, & \text{if }0\in\operatorname*{Div}\left(  b_{1},b_{2},\ldots,b_{k}\right)
\end{cases}
=0.
\]
Comparing this with (\ref{pf.lem.ent.gcd.through-Div.c2.1}), we obtain%
\[
\gcd\left(  b_{1},b_{2},\ldots,b_{k}\right)  =%
\begin{cases}
\max\left(  \operatorname*{Div}\left(  b_{1},b_{2},\ldots,b_{k}\right)
\right)  , & \text{if }0\notin\operatorname*{Div}\left(  b_{1},b_{2}%
,\ldots,b_{k}\right)  ;\\
0, & \text{if }0\in\operatorname*{Div}\left(  b_{1},b_{2},\ldots,b_{k}\right)
.
\end{cases}
\]
Hence, Lemma \ref{lem.ent.gcd.through-Div} is proven in Case 2.

We have now proven Lemma \ref{lem.ent.gcd.through-Div} in both Cases 1 and 2.
Thus, Lemma \ref{lem.ent.gcd.through-Div} always holds.
\end{proof}

A corollary of Lemma \ref{lem.ent.gcd.through-Div} is the following:

\begin{lemma}
\label{lem.ent.gcd.through-Divc}Let $b_{1},b_{2},\ldots,b_{k}$ be finitely
many integers. Let $c_{1},c_{2},\ldots,c_{\ell}$ be finitely many integers. If%
\[
\operatorname*{Div}\left(  b_{1},b_{2},\ldots,b_{k}\right)
=\operatorname*{Div}\left(  c_{1},c_{2},\ldots,c_{\ell}\right)  ,
\]
then%
\[
\gcd\left(  b_{1},b_{2},\ldots,b_{k}\right)  =\gcd\left(  c_{1},c_{2}%
,\ldots,c_{\ell}\right)  .
\]

\end{lemma}

\begin{proof}
[Proof of Lemma \ref{lem.ent.gcd.through-Div}.]Assume that
$\operatorname*{Div}\left(  b_{1},b_{2},\ldots,b_{k}\right)
=\operatorname*{Div}\left(  c_{1},c_{2},\ldots,c_{\ell}\right)  $. Lemma
\ref{lem.ent.gcd.through-Div} yields%
\begin{align*}
\gcd\left(  b_{1},b_{2},\ldots,b_{k}\right)   &  =%
\begin{cases}
\max\left(  \operatorname*{Div}\left(  b_{1},b_{2},\ldots,b_{k}\right)
\right)  , & \text{if }0\notin\operatorname*{Div}\left(  b_{1},b_{2}%
,\ldots,b_{k}\right)  ;\\
0, & \text{if }0\in\operatorname*{Div}\left(  b_{1},b_{2},\ldots,b_{k}\right)
\end{cases}
\\
&  =%
\begin{cases}
\max\left(  \operatorname*{Div}\left(  c_{1},c_{2},\ldots,c_{\ell}\right)
\right)  , & \text{if }0\notin\operatorname*{Div}\left(  c_{1},c_{2}%
,\ldots,c_{\ell}\right)  ;\\
0, & \text{if }0\in\operatorname*{Div}\left(  c_{1},c_{2},\ldots,c_{\ell
}\right)
\end{cases}
\end{align*}
(since $\operatorname*{Div}\left(  b_{1},b_{2},\ldots,b_{k}\right)
=\operatorname*{Div}\left(  c_{1},c_{2},\ldots,c_{\ell}\right)  $). But Lemma
\ref{lem.ent.gcd.through-Div} (applied to $c_{1},c_{2},\ldots,c_{\ell}$
instead of $b_{1},b_{2},\ldots,b_{k}$) yields%
\[
\gcd\left(  c_{1},c_{2},\ldots,c_{\ell}\right)  =%
\begin{cases}
\max\left(  \operatorname*{Div}\left(  c_{1},c_{2},\ldots,c_{\ell}\right)
\right)  , & \text{if }0\notin\operatorname*{Div}\left(  c_{1},c_{2}%
,\ldots,c_{\ell}\right)  ;\\
0, & \text{if }0\in\operatorname*{Div}\left(  c_{1},c_{2},\ldots,c_{\ell
}\right)  .
\end{cases}
\]
Comparing these two equalities, we obtain $\gcd\left(  b_{1},b_{2}%
,\ldots,b_{k}\right)  =\gcd\left(  c_{1},c_{2},\ldots,c_{\ell}\right)  $. This
proves Lemma \ref{lem.ent.gcd.through-Div}.
\end{proof}

\begin{proof}
[Proof of Proposition \ref{prop.ent.gcd.props1}.]\textbf{(a)} Here is a sketch
of the proof: The number $0$ is a \textquotedblleft joker\textquotedblright%
\ when it comes to common divisors: For example, if $a\in\mathbb{Z}$, then the
common divisors of $a$ and $0$ are the same as the divisors of $a$, because
every integer divides $0$. Thus, if $a\in\mathbb{Z}$ is nonzero, then the
greatest common divisor of $a$ and $0$ is the greatest divisor of $a$, which
is $\left\vert a\right\vert $ (an easy consequence of Proposition
\ref{prop.ent.divisors.find} \textbf{(b)}).

For the sake of completeness, let us give a detailed proof of Proposition
\ref{prop.ent.gcd.props1} \textbf{(a)}:

\begin{fineprint}
Let $a\in\mathbb{Z}$. Definition \ref{def.ent.gcd.gcd} (specifically, its case
when $b_{1},b_{2},\ldots,b_{k}$ are all $0$) shows that $\gcd\left(
0,0\right)  =0$ and $\gcd\left(  0\right)  =0$. Combining this with
$\left\vert 0\right\vert =0$, we obtain $\gcd\left(  0,0\right)  =\gcd\left(
0\right)  =\left\vert 0\right\vert $. In other words, Proposition
\ref{prop.ent.gcd.props1} \textbf{(a)} holds if $a=0$. Thus, for the rest of
this proof, we WLOG assume that $a\neq0$. Hence, the two integers $a,0$ are
not all zero. Thus, $\gcd\left(  a,0\right)  $ is defined to be the largest
element of the set $\operatorname*{Div}\left(  a,0\right)  $ (by Definition
\ref{def.ent.gcd.gcd}). Likewise, $\gcd\left(  a\right)  $ is the largest
element of the set $\operatorname*{Div}\left(  a\right)  $.

We shall now prove that $\operatorname*{Div}\left(  a,0\right)
=\operatorname*{Div}\left(  a\right)  $. Indeed, for any integer $x$, we have
the following chain of equivalences:%
\begin{align*}
&  \ \left(  x\in\operatorname*{Div}\left(  a,0\right)  \right) \\
&  \Longleftrightarrow\ \left(  x\text{ is a common divisor of }a\text{ and
}0\right)  \ \ \ \ \ \ \ \ \ \ \left(  \text{by the definition of
}\operatorname*{Div}\left(  a,0\right)  \right) \\
&  \Longleftrightarrow\ \left(  x\mid a\text{ and }x\mid0\right)
\ \ \ \ \ \ \ \ \ \ \left(  \text{by the definition of a \textquotedblleft
common divisor\textquotedblright}\right) \\
&  \Longleftrightarrow\ \left(  x\mid a\right)  \ \ \ \ \ \ \ \ \ \ \left(
\text{since }x\mid0\text{ always holds (since }0=x\cdot0\text{)}\right) \\
&  \Longleftrightarrow\ \left(  x\text{ is a common divisor of }a\right)
\ \ \ \ \ \ \ \ \ \ \left(  \text{by the definition of a \textquotedblleft
common divisor\textquotedblright}\right) \\
&  \Longleftrightarrow\ \left(  x\in\operatorname*{Div}\left(  a\right)
\right)  \ \ \ \ \ \ \ \ \ \ \left(  \text{by the definition of }%
\operatorname*{Div}\left(  a\right)  \right)  .
\end{align*}
In other words, an integer belongs to $\operatorname*{Div}\left(  a,0\right)
$ if and only if it belongs to $\operatorname*{Div}\left(  a\right)  $. Thus,
$\operatorname*{Div}\left(  a,0\right)  =\operatorname*{Div}\left(  a\right)
$ (since both $\operatorname*{Div}\left(  a,0\right)  $ and
$\operatorname*{Div}\left(  a\right)  $ are sets of integers). Thus, Lemma
\ref{lem.ent.gcd.through-Div} (applied to $\left(  a,0\right)  $ and $\left(
a\right)  $ instead of $\left(  b_{1},b_{2},\ldots,b_{k}\right)  $ and
$\left(  c_{1},c_{2},\ldots,c_{\ell}\right)  $) yields $\gcd\left(
a,0\right)  =\gcd\left(  a\right)  $.

For any integer $x$, we have the following chain of equivalences:
\begin{align*}
&  \ \left(  x\in\operatorname*{Div}\left(  a\right)  \right) \\
&  \Longleftrightarrow\ \left(  x\text{ is a common divisor of }a\right)
\ \ \ \ \ \ \ \ \ \ \left(  \text{by the definition of }\operatorname*{Div}%
\left(  a\right)  \right) \\
&  \Longleftrightarrow\ \left(  x\mid a\right)  \ \ \ \ \ \ \ \ \ \ \left(
\text{by the definition of a \textquotedblleft common
divisor\textquotedblright}\right) \\
&  \Longleftrightarrow\ \left(  x\text{ is a divisor of }a\right)  .
\end{align*}
Thus, $\operatorname*{Div}\left(  a\right)  $ is the set of all divisors of
$a$.

Proposition \ref{prop.ent.div.1} \textbf{(a)} (applied to $\left\vert
a\right\vert $ and $a$ instead of $a$ and $b$) shows that we have $\left\vert
a\right\vert \mid a$ if and only if $\left\vert \left\vert a\right\vert
\right\vert \mid\left\vert a\right\vert $. Since $\left\vert \left\vert
a\right\vert \right\vert \mid\left\vert a\right\vert $ is true (because
$\left\vert \left\vert a\right\vert \right\vert =\left\vert a\right\vert $),
we thus have $\left\vert a\right\vert \mid a$. In other words, $\left\vert
a\right\vert $ is a divisor of $a$.

Moreover, $a$ is nonzero (since $a\neq0$). Hence, Proposition
\ref{prop.ent.divisors.find} \textbf{(b)} (applied to $b=a$) shows that all
divisors of $a$ belong to the set $\left\{  -\left\vert a\right\vert
,-\left\vert a\right\vert +1,\ldots,\left\vert a\right\vert \right\}
\setminus\left\{  0\right\}  $. Hence, they belong to the set $\left\{
-\left\vert a\right\vert ,-\left\vert a\right\vert +1,\ldots,\left\vert
a\right\vert \right\}  $, and thus are $\leq\left\vert a\right\vert $.

Recall that $\left\vert a\right\vert $ is a divisor of $a$. Since we also know
that all divisors of $a$ are $\leq\left\vert a\right\vert $, we can thus
conclude that $\left\vert a\right\vert $ is the \textbf{largest} divisor of
$a$. In other words, $\left\vert a\right\vert $ is the largest element of the
set $\operatorname*{Div}\left(  a\right)  $ (since $\operatorname*{Div}\left(
a\right)  $ is the set of all divisors of $a$). In other words, $\left\vert
a\right\vert $ is $\gcd\left(  a\right)  $ (since $\gcd\left(  a\right)  $ is
the largest element of the set $\operatorname*{Div}\left(  a\right)  $). Thus,
$\gcd\left(  a\right)  =\left\vert a\right\vert $. Combining this with
$\gcd\left(  a,0\right)  =\gcd\left(  a\right)  $, this yields $\gcd\left(
a,0\right)  =\gcd\left(  a\right)  =\left\vert a\right\vert $. Thus,
Proposition \ref{prop.ent.gcd.props1} \textbf{(a)} is finally proven.
\end{fineprint}

\textbf{(b)} For any integer $x$, we have the following chain of equivalences:%
\begin{align*}
&  \ \left(  x\in\operatorname*{Div}\left(  a,b\right)  \right) \\
&  \Longleftrightarrow\ \left(  x\text{ is a common divisor of }a\text{ and
}b\right)  \ \ \ \ \ \ \ \ \ \ \left(  \text{by the definition of
}\operatorname*{Div}\left(  a,b\right)  \right) \\
&  \Longleftrightarrow\ \left(  x\mid a\text{ and }x\mid b\right)
\ \ \ \ \ \ \ \ \ \ \left(  \text{by the definition of a \textquotedblleft
common divisor\textquotedblright}\right) \\
&  \Longleftrightarrow\ \left(  x\mid b\text{ and }x\mid a\right) \\
&  \Longleftrightarrow\ \left(  x\text{ is a common divisor of }b\text{ and
}a\right)  \ \ \ \ \ \ \ \ \ \ \left(  \text{by the definition of a
\textquotedblleft common divisor\textquotedblright}\right) \\
&  \Longleftrightarrow\ \left(  x\in\operatorname*{Div}\left(  b,a\right)
\right)  \ \ \ \ \ \ \ \ \ \ \left(  \text{by the definition of }%
\operatorname*{Div}\left(  b,a\right)  \right)  .
\end{align*}
In other words, an integer belongs to $\operatorname*{Div}\left(  a,b\right)
$ if and only if it belongs to $\operatorname*{Div}\left(  b,a\right)  $.
Thus, $\operatorname*{Div}\left(  a,b\right)  =\operatorname*{Div}\left(
b,a\right)  $ (since both $\operatorname*{Div}\left(  a,b\right)  $ and
$\operatorname*{Div}\left(  b,a\right)  $ are sets of integers). Thus, Lemma
\ref{lem.ent.gcd.through-Div} (applied to $\left(  a,b\right)  $ and $\left(
b,a\right)  $ instead of $\left(  b_{1},b_{2},\ldots,b_{k}\right)  $ and
$\left(  c_{1},c_{2},\ldots,c_{\ell}\right)  $) yields $\gcd\left(
a,b\right)  =\gcd\left(  b,a\right)  $. This proves Proposition
\ref{prop.ent.gcd.props1} \textbf{(b)}.

Let us prove part \textbf{(d)} now, and then derive part \textbf{(c)} from it.

\textbf{(d)} Let $a,b,c\in\mathbb{Z}$ satisfy $b\equiv c\operatorname{mod}a$.
We must prove that $\gcd\left(  a,b\right)  =\gcd\left(  a,c\right)  $. To do
so, we shall first prove that $\operatorname*{Div}\left(  a,b\right)
=\operatorname*{Div}\left(  a,c\right)  $.

From $b\equiv c\operatorname{mod}a$, we obtain $c\equiv b\operatorname{mod}a$
(by Proposition \ref{prop.ent.mod.basics} \textbf{(c)}). Hence, our situation
is symmetric with respect to $b$ and $c$.

We shall now show that $\operatorname*{Div}\left(  a,b\right)  \subseteq
\operatorname*{Div}\left(  a,c\right)  $. Indeed, let $x\in\operatorname*{Div}%
\left(  a,b\right)  $. Then, $x$ is a common divisor of $a$ and $b$ (by the
definition of $\operatorname*{Div}\left(  a,b\right)  $). In other words,
$x\mid a$ and $x\mid b$ (by the definition of a \textquotedblleft common
divisor\textquotedblright). From $x\mid b$, we obtain $b\equiv
0\operatorname{mod}x$. Thus, $c\equiv b\equiv0\operatorname{mod}x$, so that
$x\mid c$. Combining $x\mid a$ and $x\mid c$, we see that $x$ is a common
divisor of $a$ and $c$ (by the definition of a \textquotedblleft common
divisor\textquotedblright). In other words, $x\in\operatorname*{Div}\left(
a,c\right)  $ (by the definition of $\operatorname*{Div}\left(  a,c\right)  $).

Now, forget that we fixed $x$. We thus have proven that $x\in
\operatorname*{Div}\left(  a,c\right)  $ for each $x\in\operatorname*{Div}%
\left(  a,b\right)  $. In other words, $\operatorname*{Div}\left(  a,b\right)
\subseteq\operatorname*{Div}\left(  a,c\right)  $.

The same argument (but with the roles of $b$ and $c$ swapped) shows that
$\operatorname*{Div}\left(  a,c\right)  \subseteq\operatorname*{Div}\left(
a,b\right)  $ (since our situation is symmetric with respect to $b$ and $c$).
Combining this with $\operatorname*{Div}\left(  a,b\right)  \subseteq
\operatorname*{Div}\left(  a,c\right)  $, we obtain $\operatorname*{Div}%
\left(  a,b\right)  =\operatorname*{Div}\left(  a,c\right)  $. Thus, Lemma
\ref{lem.ent.gcd.through-Div} (applied to $\left(  a,b\right)  $ and $\left(
a,c\right)  $ instead of $\left(  b_{1},b_{2},\ldots,b_{k}\right)  $ and
$\left(  c_{1},c_{2},\ldots,c_{\ell}\right)  $) yields $\gcd\left(
a,b\right)  =\gcd\left(  a,c\right)  $. This proves Proposition
\ref{prop.ent.gcd.props1} \textbf{(d)}.

\textbf{(c)} Let $a,b,u\in\mathbb{Z}$. Then, $ua+b\equiv b\operatorname{mod}a$
(since $\left(  ua+b\right)  -b=ua$ is clearly divisible by $a$). Thus,
Proposition \ref{prop.ent.gcd.props1} \textbf{(d)} (applied to $ua+b$ and $b$
instead of $b$ and $c$) yields $\gcd\left(  a,ua+b\right)  =\gcd\left(
a,b\right)  $. This proves Proposition \ref{prop.ent.gcd.props1} \textbf{(c)}.

\textbf{(e)} Let $a,b\in\mathbb{Z}$ be such that $a$ is positive. Then,
$b\%a\equiv b\operatorname{mod}a$ (by Corollary \ref{cor.ent.quo-rem.remmod}
\textbf{(a)}), thus $b\equiv b\%a\operatorname{mod}a$. Hence, $\gcd\left(
a,b\right)  =\gcd\left(  a,b\%a\right)  $ (by Proposition
\ref{prop.ent.gcd.props1} \textbf{(d)}, applied to $c=b\%a$). This proves
Proposition \ref{prop.ent.gcd.props1} \textbf{(e)}.

\textbf{(f)} Let $a,b\in\mathbb{Z}$. We must prove that $\gcd\left(
a,b\right)  \mid a$ and $\gcd\left(  a,b\right)  \mid b$.

If the two integers $a,b$ are all $0$, then this is
obvious\footnote{\textit{Proof.} Assume that $a,b$ are all $0$. Then,
$a=0=\gcd\left(  a,b\right)  \cdot0$, so that $\gcd\left(  a,b\right)  \mid
a$; similarly, $\gcd\left(  a,b\right)  \mid b$. Hence, we have proven that
$\gcd\left(  a,b\right)  \mid a$ and $\gcd\left(  a,b\right)  \mid b$ if the
integers $a,b$ are all $0$.}. Hence, for the rest of this proof, we WLOG
assume that $a,b$ are not all $0$. Thus, $\gcd\left(  a,b\right)  $ is defined
to be the largest element of the set $\operatorname*{Div}\left(  a,b\right)  $
(by Definition \ref{def.ent.gcd.gcd}). Hence, $\gcd\left(  a,b\right)  $ is an
element of this set $\operatorname*{Div}\left(  a,b\right)  $. In other words,
$\gcd\left(  a,b\right)  $ is a common divisor of $a$ and $b$ (by the
definition of $\operatorname*{Div}\left(  a,b\right)  $). In other words,
$\gcd\left(  a,b\right)  \mid a$ and $\gcd\left(  a,b\right)  \mid b$. This
proves Proposition \ref{prop.ent.gcd.props1} \textbf{(f)}.

\textbf{(g)} Let $a,b\in\mathbb{Z}$. We must prove that $\gcd\left(
-a,b\right)  =\gcd\left(  a,b\right)  $. Again, we shall achieve this via
showing that $\operatorname*{Div}\left(  -a,b\right)  =\operatorname*{Div}%
\left(  a,b\right)  $.

First, we will show that $\operatorname*{Div}\left(  a,b\right)
\subseteq\operatorname*{Div}\left(  -a,b\right)  $. Indeed, let $x\in
\operatorname*{Div}\left(  a,b\right)  $. Then, $x$ is a common divisor of $a$
and $b$ (by the definition of $\operatorname*{Div}\left(  a,b\right)  $). In
other words, $x\mid a$ and $x\mid b$ (by the definition of a \textquotedblleft
common divisor\textquotedblright). We have $a\mid-a$ (since $-a=a\cdot\left(
-1\right)  $). Thus, $x\mid a\mid-a$. Combining $x\mid-a$ and $x\mid b$, we
see that $x$ is a common divisor of $-a$ and $b$ (by the definition of a
\textquotedblleft common divisor\textquotedblright). In other words,
$x\in\operatorname*{Div}\left(  -a,b\right)  $ (by the definition of
$\operatorname*{Div}\left(  -a,b\right)  $).

Now, forget that we fixed $x$. We thus have proven that $x\in
\operatorname*{Div}\left(  -a,b\right)  $ for each $x\in\operatorname*{Div}%
\left(  a,b\right)  $. In other words, $\operatorname*{Div}\left(  a,b\right)
\subseteq\operatorname*{Div}\left(  -a,b\right)  $.

The same argument (but applied to $-a$ instead of $a$) shows that
$\operatorname*{Div}\left(  -a,b\right)  \subseteq\operatorname*{Div}\left(
-\left(  -a\right)  ,b\right)  $. Since $-\left(  -a\right)  =a$, this
rewrites as $\operatorname*{Div}\left(  -a,b\right)  \subseteq
\operatorname*{Div}\left(  a,b\right)  $. Combining this with
$\operatorname*{Div}\left(  a,b\right)  \subseteq\operatorname*{Div}\left(
-a,b\right)  $, we obtain $\operatorname*{Div}\left(  -a,b\right)
=\operatorname*{Div}\left(  a,b\right)  $. Thus, Lemma
\ref{lem.ent.gcd.through-Div} (applied to $\left(  -a,b\right)  $ and $\left(
a,b\right)  $ instead of $\left(  b_{1},b_{2},\ldots,b_{k}\right)  $ and
$\left(  c_{1},c_{2},\ldots,c_{\ell}\right)  $) yields $\gcd\left(
-a,b\right)  =\gcd\left(  a,b\right)  $. This proves Proposition
\ref{prop.ent.gcd.props1} \textbf{(g)}.

\textbf{(h)} We can prove this similarly to how we just proved Proposition
\ref{prop.ent.gcd.props1} \textbf{(g)}, but it is easier to derive it from
what was already shown.

Let $a,b\in\mathbb{Z}$. Proposition \ref{prop.ent.gcd.props1} \textbf{(b)}
(applied to $-b$ instead of $b$) yields%
\begin{align*}
\gcd\left(  a,-b\right)   &  =\gcd\left(  -b,a\right)  =\gcd\left(
b,a\right)  \ \ \ \ \ \ \ \ \ \ \left(
\begin{array}
[c]{c}%
\text{by Proposition \ref{prop.ent.gcd.props1} \textbf{(g),}}\\
\text{applied to }b\text{ and }a\text{ instead of }a\text{ and }b
\end{array}
\right) \\
&  =\gcd\left(  a,b\right)  \ \ \ \ \ \ \ \ \ \ \left(  \text{by Proposition
\ref{prop.ent.gcd.props1} \textbf{(b)}}\right)  .
\end{align*}
This proves Proposition \ref{prop.ent.gcd.props1} \textbf{(h)}.

\textbf{(i)} Let $a,b\in\mathbb{Z}$ satisfy $a\mid b$. From $a\mid b$, we
obtain $b\equiv0\operatorname{mod}a$. Hence, Proposition
\ref{prop.ent.gcd.props1} \textbf{(d)} (applied to $c=0$) yields $\gcd\left(
a,b\right)  =\gcd\left(  a,0\right)  =\left\vert a\right\vert $ (by
Proposition \ref{prop.ent.gcd.props1} \textbf{(a)}). This proves Proposition
\ref{prop.ent.gcd.props1} \textbf{(i)}.
\end{proof}

\begin{remark}
Proposition \ref{prop.ent.gcd.props1} \textbf{(c)} says that if we add a
multiple of $a$ to $b$, then $\gcd\left(  a,b\right)  $ does not change.
Similarly, if we add a multiple of $b$ to $a$, then $\gcd\left(  a,b\right)  $
does not change (i.e., we have $\gcd\left(  vb+a,b\right)  =\gcd\left(
a,b\right)  $ for all $a,b,v\in\mathbb{Z}$).

However, if we \textbf{simultaneously} add a multiple of $a$ to $b$ and a
multiple of $b$ to $a$, then $\gcd\left(  a,b\right)  $ may well change: i.e.,
we may have $\gcd\left(  vb+a,ua+b\right)  \neq\gcd\left(  a,b\right)  $ for
all $a,b,u,v\in\mathbb{Z}$. Examples are easy to find (just take $v=1$ and
$u=1$).
\end{remark}

Proposition \ref{prop.ent.gcd.props1} gives a quick way to compute
$\gcd\left(  a,b\right)  $ for two nonnegative integers $a$ and $b$, by
repeatedly applying division with remainder. For example, let us compute
$\gcd\left(  210,45\right)  $ as follows:%
\begin{align*}
\gcd\left(  210,45\right)   &  =\gcd\left(  45,210\right)
\ \ \ \ \ \ \ \ \ \ \left(  \text{by Proposition \ref{prop.ent.gcd.props1}
\textbf{(b)}}\right) \\
&  =\gcd\left(  45,\underbrace{210\%45}_{=30}\right)
\ \ \ \ \ \ \ \ \ \ \left(  \text{by Proposition \ref{prop.ent.gcd.props1}
\textbf{(e)}}\right) \\
&  =\gcd\left(  45,30\right) \\
&  =\gcd\left(  30,45\right)  \ \ \ \ \ \ \ \ \ \ \left(  \text{by Proposition
\ref{prop.ent.gcd.props1} \textbf{(b)}}\right) \\
&  =\gcd\left(  30,\underbrace{45\%30}_{=15}\right)
\ \ \ \ \ \ \ \ \ \ \left(  \text{by Proposition \ref{prop.ent.gcd.props1}
\textbf{(e)}}\right) \\
&  =\gcd\left(  30,15\right) \\
&  =\gcd\left(  15,30\right)  \ \ \ \ \ \ \ \ \ \ \left(  \text{by Proposition
\ref{prop.ent.gcd.props1} \textbf{(b)}}\right) \\
&  =\gcd\left(  15,\underbrace{30\%15}_{=0}\right)
\ \ \ \ \ \ \ \ \ \ \left(  \text{by Proposition \ref{prop.ent.gcd.props1}
\textbf{(e)}}\right) \\
&  =\gcd\left(  15,0\right)  =\left\vert 15\right\vert
\ \ \ \ \ \ \ \ \ \ \left(  \text{by Proposition \ref{prop.ent.gcd.props1}
\textbf{(a)}}\right) \\
&  =15.
\end{align*}
This method of computing $\gcd\left(  a,b\right)  $ is called the
\textit{Euclidean algorithm}, and is usually much faster than the divisors of
$a$ or the divisors of $b$ can be found!

\subsubsection{Bezout's theorem}

The following fact about gcds is one of the most important facts in number theory:

\begin{theorem}
\label{thm.ent.gcd.bezout}Let $a$ and $b$ be two integers. Then, there exist
integers $x\in\mathbb{Z}$ and $y\in\mathbb{Z}$ such that%
\[
\gcd\left(  a,b\right)  =xa+yb.
\]

\end{theorem}

Theorem \ref{thm.ent.gcd.bezout} is often stated as follows: \textquotedblleft
If $a$ and $b$ are two integers, then $\gcd\left(  a,b\right)  $ is a
$\mathbb{Z}$-linear combination of $a$ and $b$\textquotedblright. The notion
\textquotedblleft$\mathbb{Z}$-linear combination of $a$ and $b$%
\textquotedblright\ simply means \textquotedblleft a number of the form
$xa+yb$ with $x\in\mathbb{Z}$ and $y\in\mathbb{Z}$\textquotedblright\ (this is
exactly the notion of a \textquotedblleft linear combination\textquotedblright%
\ in linear algebra, except that now the scalars must come from $\mathbb{Z}$),
so this is just a restatement of Theorem \ref{thm.ent.gcd.bezout}.

Theorem \ref{thm.ent.gcd.bezout} is known as \textit{Bezout's theorem} (or
\textit{Bezout's identity})\footnote{or \textit{Bezout's theorem for integers}
if you want to be more precise (as there are similar theorems for other
objects)}. We shall prove it in several steps. The first step is to show it
when $a$ and $b$ are nonnegative:

\begin{lemma}
\label{lem.ent.gcd.bezout.++}Let $a\in\mathbb{N}$ and $b\in\mathbb{N}$. Then,
there exist integers $x\in\mathbb{Z}$ and $y\in\mathbb{Z}$ such that%
\[
\gcd\left(  a,b\right)  =xa+yb.
\]

\end{lemma}

\begin{proof}
[Proof of Lemma \ref{lem.ent.gcd.bezout.++}.]The following proof uses a
strategy similar to the Euclidean algorithm (making one of $a$ and $b$ smaller
repeatedly until one of $a$ and $b$ becomes $0$), and can in fact be viewed as
a \textquotedblleft protocol\textquotedblright\ of the algorithm.

We use strong induction on $a+b$. Thus, we fix an $n\in\mathbb{N}$, and assume
(as induction hypothesis) that Lemma \ref{lem.ent.gcd.bezout.++} holds
whenever $a+b<n$. We must now prove that Lemma \ref{lem.ent.gcd.bezout.++}
holds whenever $a+b=n$.

We have assumed that Lemma \ref{lem.ent.gcd.bezout.++} holds whenever $a+b<n$.
In other words, the following statement holds:

\begin{statement}
\textit{Statement 1:} Let $a\in\mathbb{N}$ and $b\in\mathbb{N}$ be such that
$a+b<n$. Then, there exist integers $x\in\mathbb{Z}$ and $y\in\mathbb{Z}$ such
that $\gcd\left(  a,b\right)  =xa+yb$.
\end{statement}

Now, we must prove that Lemma \ref{lem.ent.gcd.bezout.++} holds whenever
$a+b=n$. Let us first prove this in the case when $b\geq a$:

\begin{statement}
\textit{Statement 2:} Let $a\in\mathbb{N}$ and $b\in\mathbb{N}$ be such that
$a+b=n$ and $b\geq a$. Then, there exist integers $x\in\mathbb{Z}$ and
$y\in\mathbb{Z}$ such that $\gcd\left(  a,b\right)  =xa+yb$.
\end{statement}

[\textit{Proof of Statement 2:} We are in one of the following two cases:

\textit{Case 1:} We have $a=0$.

\textit{Case 2:} We have $a\neq0$.

Let us first consider Case 1. In this case, we have $a=0$. Now, Proposition
\ref{prop.ent.gcd.props1} \textbf{(a)} (applied to $b$ instead of $a$) yields
$\gcd\left(  b,0\right)  =\gcd\left(  b\right)  =\left\vert b\right\vert
\in\left\{  b,-b\right\}  $. In other words, $\gcd\left(  b,0\right)  =ub$ for
some $u\in\left\{  1,-1\right\}  $. Consider this $u$. Now, Proposition
\ref{prop.ent.gcd.props1} \textbf{(b)} yields%
\[
\gcd\left(  a,b\right)  =\gcd\left(  b,\underbrace{a}_{=0}\right)
=\gcd\left(  b,0\right)  =ub=0a+ub.
\]
Hence, there exist integers $x\in\mathbb{Z}$ and $y\in\mathbb{Z}$ such that
$\gcd\left(  a,b\right)  =xa+yb$ (namely, $x=0$ and $y=u$). Thus, Statement 2
is proven in Case 1.

Let us next consider Case 2. In this case, we have $a\neq0$. Hence, $a>0$
(since $a\in\mathbb{N}$), so that $a+b>b$. Hence, $b<a+b=n$.

From $b\geq a$, we obtain $b-a\in\mathbb{N}$. Moreover, $a\in\mathbb{N}$ and
$b-a\in\mathbb{N}$ satisfy $a+\left(  b-a\right)  =b<n$. Therefore, we can
apply Statement 1 \textbf{to }$b-a$ \textbf{instead of }$b$. Thus we obtain
that there exist integers $x\in\mathbb{Z}$ and $y\in\mathbb{Z}$ such that
$\gcd\left(  a,b-a\right)  =xa+y\left(  b-a\right)  $. Fix two such integers
$x$ and $y$, and denote them by $x_{0}$ and $y_{0}$. Thus, $x_{0}$ and $y_{0}$
are two integers such that $\gcd\left(  a,b-a\right)  =x_{0}a+y_{0}\left(
b-a\right)  $.

Also, Proposition \ref{prop.ent.gcd.props1} \textbf{(c)} (applied to $u=-1$)
yields $\gcd\left(  a,\left(  -1\right)  a+b\right)  =\gcd\left(  a,b\right)
$. Hence,%
\begin{align*}
\gcd\left(  a,b\right)   &  =\gcd\left(  a,\underbrace{\left(  -1\right)
a+b}_{=b-a}\right)  =\gcd\left(  a,b-a\right)  =x_{0}a+y_{0}\left(  b-a\right)
\\
&  =x_{0}a+y_{0}b-y_{0}a=\left(  x_{0}-y_{0}\right)  a+y_{0}b.
\end{align*}
Hence, there exist integers $x\in\mathbb{Z}$ and $y\in\mathbb{Z}$ such that
$\gcd\left(  a,b\right)  =xa+yb$ (namely, $x=x_{0}-y_{0}$ and $y=y_{0}$).
Thus, Statement 2 is proven in Case 2.

We have now proven Statement 2 in both Cases 1 and 2. Hence, Statement 2 is
always proven.]

Now, we can prove that Lemma \ref{lem.ent.gcd.bezout.++} holds whenever
$a+b=n$:

\begin{statement}
\textit{Statement 3:} Let $a\in\mathbb{N}$ and $b\in\mathbb{N}$ be such that
$a+b=n$. Then, there exist integers $x\in\mathbb{Z}$ and $y\in\mathbb{Z}$ such
that $\gcd\left(  a,b\right)  =xa+yb$.
\end{statement}

[\textit{Proof of Statement 3:} We are in one of the following two cases:

\textit{Case 1:} We have $b\geq a$.

\textit{Case 2:} We have $b<a$.

Let us first consider Case 1. In this case, we have $b\geq a$. Hence,
Statement 2 shows that there exist integers $x\in\mathbb{Z}$ and
$y\in\mathbb{Z}$ such that $\gcd\left(  a,b\right)  =xa+yb$. Thus, Statement 3
is proven in Case 1.

Let us next consider Case 2. In this case, we have $b<a$. Hence, $a>b$, so
that $a\geq b$. This shows that we can apply Statement 2 \textbf{to }%
$b$\textbf{ and }$a$ \textbf{instead of }$a$ \textbf{and }$b$. Thus we obtain
that there exist integers $x\in\mathbb{Z}$ and $y\in\mathbb{Z}$ such that
$\gcd\left(  b,a\right)  =xb+ya$. Fix two such integers $x$ and $y$, and
denote them by $x_{0}$ and $y_{0}$. Thus, $x_{0}$ and $y_{0}$ are two integers
such that $\gcd\left(  b,a\right)  =x_{0}b+y_{0}a$. Now, Proposition
\ref{prop.ent.gcd.props1} \textbf{(b)} yields $\gcd\left(  a,b\right)
=\gcd\left(  b,a\right)  =x_{0}b+y_{0}a=y_{0}a+x_{0}b$. Hence, there exist
integers $x\in\mathbb{Z}$ and $y\in\mathbb{Z}$ such that $\gcd\left(
a,b\right)  =xa+yb$ (namely, $x=y_{0}$ and $y=x_{0}$). Thus, Statement 3 is
proven in Case 2.

We have now proven Statement 3 in both Cases 1 and 2. Hence, Statement 3 is
always proven.]

By proving Statement 3, we have shown that Lemma \ref{lem.ent.gcd.bezout.++}
holds whenever $a+b=n$. This completes the induction step. Thus, Lemma
\ref{lem.ent.gcd.bezout.++} is proven by strong induction.
\end{proof}

Next, we shall prove Theorem \ref{thm.ent.gcd.bezout} when $a\in\mathbb{N}$
but $b$ may be negative:

\begin{lemma}
\label{lem.ent.gcd.bezout.+}Let $a\in\mathbb{N}$ and $b\in\mathbb{Z}$. Then,
there exist integers $x\in\mathbb{Z}$ and $y\in\mathbb{Z}$ such that%
\[
\gcd\left(  a,b\right)  =xa+yb.
\]

\end{lemma}

\begin{proof}
[Proof of Lemma \ref{lem.ent.gcd.bezout.+}.]We are in one of the following two cases:

\textit{Case 1:} We have $b\geq0$.

\textit{Case 2:} We have $b<0$.

Let us first consider Case 1. In this case, we have $b\geq0$. Thus,
$b\in\mathbb{N}$ (since $b\in\mathbb{Z}$). Therefore, Lemma
\ref{lem.ent.gcd.bezout.++} shows that there exist integers $x\in\mathbb{Z}$
and $y\in\mathbb{Z}$ such that $\gcd\left(  a,b\right)  =xa+yb$. Thus, Lemma
\ref{lem.ent.gcd.bezout.+} is proven in Case 1.

Let us now consider Case 2. In this case, we have $b<0$. Hence, $-b>0$, so
that $-b\in\mathbb{N}$ (since $-b\in\mathbb{Z}$). Therefore, Lemma
\ref{lem.ent.gcd.bezout.++} (applied to $-b$ instead of $b$) shows that there
exist integers $x\in\mathbb{Z}$ and $y\in\mathbb{Z}$ such that $\gcd\left(
a,-b\right)  =xa+y\left(  -b\right)  $. Fix such integers, and denote them by
$x_{0}$ and $y_{0}$. Thus, $x_{0}\in\mathbb{Z}$ and $y_{0}\in\mathbb{Z}$ are
integers such that $\gcd\left(  a,-b\right)  =x_{0}a+y_{0}\left(  -b\right)  $.

Now, Proposition \ref{prop.ent.gcd.props1} \textbf{(h)} yields $\gcd\left(
a,-b\right)  =\gcd\left(  a,b\right)  $. Hence,%
\[
\gcd\left(  a,b\right)  =\gcd\left(  a,-b\right)  =x_{0}a+y_{0}\left(
-b\right)  =x_{0}a+\left(  -y_{0}\right)  b.
\]
Hence, there exist integers $x\in\mathbb{Z}$ and $y\in\mathbb{Z}$ such that
$\gcd\left(  a,b\right)  =xa+yb$ (namely, $x=x_{0}$ and $y=-y_{0}$). Thus,
Lemma \ref{lem.ent.gcd.bezout.+} is proven in Case 2.

We have now proven Lemma \ref{lem.ent.gcd.bezout.+} in both Cases 1 and 2.
Hence, Lemma \ref{lem.ent.gcd.bezout.+} is proven.
\end{proof}

Now, we can prove the whole Theorem \ref{thm.ent.gcd.bezout}:

\begin{proof}
[Proof of Theorem \ref{thm.ent.gcd.bezout}.]Theorem \ref{thm.ent.gcd.bezout}
can be derived from Lemma \ref{lem.ent.gcd.bezout.+} in the same way as Lemma
\ref{lem.ent.gcd.bezout.+} was derived from Lemma \ref{lem.ent.gcd.bezout.++}
(except that this time, we have to distinguish between the cases $a\geq0$ and
$a<0$, and we have to use Proposition \ref{prop.ent.gcd.props1} \textbf{(g)}
instead of Proposition \ref{prop.ent.gcd.props1} \textbf{(h)}). Again, let us
give the detailed argument for the sake of completeness:

\begin{fineprint}
We are in one of the following two cases:

\textit{Case 1:} We have $a\geq0$.

\textit{Case 2:} We have $a<0$.

Let us first consider Case 1. In this case, we have $a\geq0$. Thus,
$a\in\mathbb{N}$ (since $a\in\mathbb{Z}$). Therefore, Lemma
\ref{lem.ent.gcd.bezout.+} shows that there exist integers $x\in\mathbb{Z}$
and $y\in\mathbb{Z}$ such that $\gcd\left(  a,b\right)  =xa+yb$. Thus, Theorem
\ref{thm.ent.gcd.bezout} is proven in Case 1.

Let us now consider Case 2. In this case, we have $a<0$. Hence, $-a>0$, so
that $-a\in\mathbb{N}$ (since $-a\in\mathbb{Z}$). Therefore, Lemma
\ref{lem.ent.gcd.bezout.+} (applied to $-a$ instead of $a$) shows that there
exist integers $x\in\mathbb{Z}$ and $y\in\mathbb{Z}$ such that $\gcd\left(
-a,b\right)  =x\left(  -a\right)  +yb$. Fix such integers, and denote them by
$x_{0}$ and $y_{0}$. Thus, $x_{0}\in\mathbb{Z}$ and $y_{0}\in\mathbb{Z}$ are
integers such that $\gcd\left(  -a,b\right)  =x_{0}\left(  -a\right)  +y_{0}b$.

Now, Proposition \ref{prop.ent.gcd.props1} \textbf{(g)} yields $\gcd\left(
-a,b\right)  =\gcd\left(  a,b\right)  $. Hence,%
\[
\gcd\left(  a,b\right)  =\gcd\left(  -a,b\right)  =x_{0}\left(  -a\right)
+y_{0}b=\left(  -x_{0}\right)  a+y_{0}b.
\]
Hence, there exist integers $x\in\mathbb{Z}$ and $y\in\mathbb{Z}$ such that
$\gcd\left(  a,b\right)  =xa+yb$ (namely, $x=-x_{0}$ and $y=y_{0}$). Thus,
Theorem \ref{thm.ent.gcd.bezout} is proven in Case 2.

We have now proven Theorem \ref{thm.ent.gcd.bezout} in both Cases 1 and 2.
Hence, Theorem \ref{thm.ent.gcd.bezout} is proven.
\end{fineprint}
\end{proof}

\subsubsection{First applications of Bezout's theorem}

An important corollary of Theorem \ref{thm.ent.gcd.bezout} is the following fact:

\begin{theorem}
\label{thm.ent.gcd.uniprop}Let $a,b\in\mathbb{Z}$. Then:

\textbf{(a)} For each $m\in\mathbb{Z}$, we have the following logical
equivalence:%
\begin{equation}
\left(  m\mid a\ \text{and }m\mid b\right)  \ \Longleftrightarrow\ \left(
m\mid\gcd\left(  a,b\right)  \right)  .\label{eq.thm.ent.gcd.uniprop.equiv}%
\end{equation}


\textbf{(b)} The common divisors of $a$ and $b$ are precisely the divisors of
$\gcd\left(  a,b\right)  $.

\textbf{(c)} We have $\operatorname*{Div}\left(  a,b\right)
=\operatorname*{Div}\left(  \gcd\left(  a,b\right)  \right)  $.
\end{theorem}

The three parts of this theorem are saying the same thing from slightly
different perspectives; the importance of the theorem nevertheless justifies
this repetition. To prove the theorem, we first show the following:

\begin{lemma}
\label{lem.ent.gcd.uniprop}Let $m,a,b\in\mathbb{Z}$ be such that $m\mid a$ and
$m\mid b$. Then, $m\mid\gcd\left(  a,b\right)  $.
\end{lemma}

\begin{proof}
[Proof of Lemma \ref{lem.ent.gcd.uniprop}.]Theorem \ref{thm.ent.gcd.bezout}
shows that there exist integers $x\in\mathbb{Z}$ and $y\in\mathbb{Z}$ such
that%
\begin{equation}
\gcd\left(  a,b\right)  =xa+yb. \label{pf.lem.ent.gcd.uniprop.1}%
\end{equation}
Consider these $x$ and $y$. Now, $m\mid a\mid xa$, so that $xa\equiv
0\operatorname{mod}m$. Also, $m\mid b\mid yb$, thus $yb\equiv
0\operatorname{mod}m$. Adding the congruences $xa\equiv0\operatorname{mod}m$
and $yb\equiv0\operatorname{mod}m$ together, we find $xa+yb\equiv
0+0=0\operatorname{mod}m$; in other words, $m\mid xa+yb$. In view of
(\ref{pf.lem.ent.gcd.uniprop.1}), this rewrites as $m\mid\gcd\left(
a,b\right)  $. This proves Lemma \ref{lem.ent.gcd.uniprop}.
\end{proof}

\begin{proof}
[Proof of Theorem \ref{thm.ent.gcd.uniprop}.]\textbf{(a)} Let $m\in\mathbb{Z}%
$. In order to prove (\ref{eq.thm.ent.gcd.uniprop.equiv}), we need to prove
the \textquotedblleft$\Longrightarrow$\textquotedblright\ and
\textquotedblleft$\Longleftarrow$\textquotedblright\ directions of the
equivalence (\ref{eq.thm.ent.gcd.uniprop.equiv}). But this is easy: The
\textquotedblleft$\Longrightarrow$\textquotedblright\ direction is just the
statement of Lemma \ref{lem.ent.gcd.uniprop}, whereas the \textquotedblleft%
$\Longleftarrow$\textquotedblright\ direction is trivial (to wit: if
$m\mid\gcd\left(  a,b\right)  $, then
\[
m\mid\gcd\left(  a,b\right)  \mid a\ \ \ \ \ \ \ \ \ \ \left(  \text{by
Proposition \ref{prop.ent.gcd.props1} \textbf{(e)}}\right)
\]
and%
\[
m\mid\gcd\left(  a,b\right)  \mid b\ \ \ \ \ \ \ \ \ \ \left(  \text{by
Proposition \ref{prop.ent.gcd.props1} \textbf{(e)}}\right)  ,
\]
and thus $\left(  m\mid a\ \text{and }m\mid b\right)  $). Hence, the
equivalence (\ref{eq.thm.ent.gcd.uniprop.equiv}) is proven. This proves
Theorem \ref{thm.ent.gcd.uniprop} \textbf{(a)}.

\textbf{(b)} The common divisors of $a$ and $b$ are precisely the integers $m$
that satisfy $\left(  m\mid a\text{ and }m\mid b\right)  $ (by the definition
of \textquotedblleft common divisor\textquotedblright). In view of the
equivalence (\ref{eq.thm.ent.gcd.uniprop.equiv}), this rewrites as follows:
The common divisors of $a$ and $b$ are precisely the integers $m$ that satisfy
$m\mid\gcd\left(  a,b\right)  $. In other words, the common divisors of $a$
and $b$ are precisely the divisors of $\gcd\left(  a,b\right)  $. This proves
Theorem \ref{thm.ent.gcd.uniprop} \textbf{(b)}.

\textbf{(c)} Recall that each $c\in\mathbb{Z}$ satisfies%
\begin{align*}
\operatorname*{Div}\left(  c\right)   &  =\left\{  \text{the common divisors
of }c\right\}  \ \ \ \ \ \ \ \ \ \ \left(  \text{by the definition of
}\operatorname*{Div}\left(  c\right)  \right) \\
&  =\left\{  \text{the integers }x\text{ such that }x\mid c\right\} \\
&  \ \ \ \ \ \ \ \ \ \ \left(  \text{by the definition of \textquotedblleft
common divisors\textquotedblright}\right) \\
&  =\left\{  \text{the divisors of }c\right\}  .
\end{align*}
Applying this to $c=\gcd\left(  a,b\right)  $, we obtain%
\begin{equation}
\operatorname*{Div}\left(  \gcd\left(  a,b\right)  \right)  =\left\{
\text{the divisors of }\gcd\left(  a,b\right)  \right\}  .
\label{pf.thm.ent.gcd.uniprop.c.1a}%
\end{equation}


The definition of $\operatorname*{Div}\left(  a,b\right)  $ yields%
\begin{align*}
\operatorname*{Div}\left(  a,b\right)   &  =\left\{  \text{the common divisors
of }a\text{ and }b\right\} \\
&  =\left\{  \text{the divisors of }\gcd\left(  a,b\right)  \right\}
\ \ \ \ \ \ \ \ \ \ \left(  \text{by Theorem \ref{thm.ent.gcd.uniprop}
\textbf{(b)}}\right) \\
&  =\operatorname*{Div}\left(  \gcd\left(  a,b\right)  \right)
\ \ \ \ \ \ \ \ \ \ \left(  \text{by (\ref{pf.thm.ent.gcd.uniprop.c.1a}%
)}\right)  .
\end{align*}
This proves Theorem \ref{thm.ent.gcd.uniprop} \textbf{(c)}.
\end{proof}

The following corollary of Theorem \ref{thm.ent.gcd.bezout} let us
\textquotedblleft combine\textquotedblright\ two divisibilities $a\mid c$ and
$b\mid c$. In fact, Proposition \ref{prop.ent.div.2} \textbf{(c)} would
already allow us to \textquotedblleft combine\textquotedblright\ them to form
$ab\mid cc=c^{2}$; but we can also \textquotedblleft combine\textquotedblright%
\ them to $ab\mid\gcd\left(  a,b\right)  \cdot c$ using the following fact:

\begin{theorem}
\label{thm.ent.gcd.combine}Let $a,b,c\in\mathbb{Z}$ satisfy $a\mid c$ and
$b\mid c$. Then, $ab\mid\gcd\left(  a,b\right)  \cdot c$.
\end{theorem}

\begin{example}
Let $a=6$ and $b=10$ and $c=30$. Then, $a=6\mid30=c$ and $b=10\mid30=c$. Thus,
Theorem \ref{thm.ent.gcd.combine} yields $ab\mid\gcd\left(  a,b\right)  \cdot
c$. And indeed, this is true, since $ab=6\cdot10\mid2\cdot30=\gcd\left(
a,b\right)  \cdot c$ (because $\gcd\left(  a,b\right)  =\gcd\left(
6,10\right)  =2$). Note that this latter divisibility is actually an equality:
we have $6\cdot10=2\cdot30$. Note also that we do \textbf{not} obtain $ab\mid
c$ (and indeed, this does not hold).
\end{example}

\begin{proof}
[Proof of Theorem \ref{thm.ent.gcd.combine}.]Theorem \ref{thm.ent.gcd.bezout}
yields that there exist integers $x\in\mathbb{Z}$ and $y\in\mathbb{Z}$ such
that $\gcd\left(  a,b\right)  =xa+yb$. Consider these $x$ and $y$.

We have $a\mid c$. In other words, there exists an integer $u$ such that
$c=au$. Consider this $u$.

We have $b\mid c$. In other words, there exists an integer $v$ such that
$c=bv$. Consider this $v$.

Now,%
\[
\underbrace{\gcd\left(  a,b\right)  }_{=xa+yb}\cdot c=\left(  xa+yb\right)
c=xa\underbrace{c}_{=bv}+yb\underbrace{c}_{=au}=xabv+ybau=ab\left(
xv+yu\right)  .
\]
Thus, there exists an integer $d$ such that $\gcd\left(  a,b\right)  \cdot
c=abd$ (namely, $d=xv+yu$). In other words, $ab\mid\gcd\left(  a,b\right)
\cdot c$. This proves Theorem \ref{thm.ent.gcd.combine}.
\end{proof}

Here is another corollary of Theorem \ref{thm.ent.gcd.bezout} whose usefulness
will become clearer later on:

\begin{theorem}
\label{thm.ent.gcd.cancel}Let $a,b,c\in\mathbb{Z}$ satisfy $a\mid bc$. Then,
$a\mid\gcd\left(  a,b\right)  \cdot c$.
\end{theorem}

At this point, you should see that Theorem \ref{thm.ent.gcd.cancel} allows
\textquotedblleft strengthening\textquotedblright\ divisibilities: You give it
a \textquotedblleft weak\textquotedblright\ divisibility $a\mid bc$, and
obtain a \textquotedblleft stronger\textquotedblright\ divisibility $a\mid
\gcd\left(  a,b\right)  \cdot c$ from it (stronger because $\gcd\left(
a,b\right)  $ is usually smaller than $b$).

\begin{proof}
[Proof of Theorem \ref{thm.ent.coprime.cancel}.]Theorem
\ref{thm.ent.gcd.bezout} yields that there exist integers $x\in\mathbb{Z}$ and
$y\in\mathbb{Z}$ such that $\gcd\left(  a,b\right)  =xa+yb$. Consider these
$x$ and $y$.

We have $a\mid bc\mid ybc$; in other words, $ybc\equiv0\operatorname{mod}a$.
Also, $a\mid axc$, so that $axc\equiv0\operatorname{mod}a$. Adding the two
congruences $axc\equiv0\operatorname{mod}a$ and $ybc\equiv0\operatorname{mod}%
a$ together, we obtain $axc+ybc\equiv0+0=0\operatorname{mod}a$. In view of
$axc+ybc=\underbrace{\left(  xa+yb\right)  }_{=\gcd\left(  a,b\right)  }%
c=\gcd\left(  a,b\right)  \cdot c$, this rewrites as $\gcd\left(  a,b\right)
\cdot c\equiv0\operatorname{mod}a$. In other words, $a\mid\gcd\left(
a,b\right)  \cdot c$. This proves Theorem \ref{thm.ent.gcd.cancel}.
\end{proof}

\begin{corollary}
\label{cor.ent.gcd.sa,sb}Let $s,a,b\in\mathbb{Z}$. Then,
\[
\gcd\left(  sa,sb\right)  =\left\vert s\right\vert \gcd\left(  a,b\right)  .
\]

\end{corollary}

\begin{proof}
[Proof of Corollary \ref{cor.ent.gcd.sa,sb}.]We shall prove that the two
integers $\gcd\left(  sa,sb\right)  $ and $s\gcd\left(  a,b\right)  $ mutually
divide each other (i.e., they satisfy $\gcd\left(  sa,sb\right)  \mid
s\gcd\left(  a,b\right)  $ and $s\gcd\left(  a,b\right)  \mid\gcd\left(
sa,sb\right)  $). Then, Exercise \ref{exe.ent.div.abba} will let us conclude
that $\left\vert \gcd\left(  sa,sb\right)  \right\vert =\left\vert
s\gcd\left(  a,b\right)  \right\vert $. This will then rewrite as $\gcd\left(
sa,sb\right)  =\left\vert s\right\vert \gcd\left(  a,b\right)  $, and we will
be done. (This trick is actually a common strategy for proving equalities
between gcds.)

For the sake of brevity, let us set $g=\gcd\left(  sa,sb\right)  $ and
$h=s\gcd\left(  a,b\right)  $. So our first goal is to prove that $g\mid h$
and $h\mid g$.

\textit{Proof of }$g\mid h$\textit{:} Theorem \ref{thm.ent.gcd.bezout} yields
that there exist integers $x\in\mathbb{Z}$ and $y\in\mathbb{Z}$ such that
$\gcd\left(  a,b\right)  =xa+yb$. Consider these $x$ and $y$.

Proposition \ref{prop.ent.gcd.props1} \textbf{(f)} (applied to $sa$ and $sb$
instead of $a$ and $b$) yields that $\gcd\left(  sa,sb\right)  \mid sa$ and
$\gcd\left(  sa,sb\right)  \mid sb$. From $g=\gcd\left(  sa,sb\right)  \mid
sa$, we obtain $g\mid sa\mid xsa$, thus $xsa\equiv0\operatorname{mod}g$.
Similarly, $ysb\equiv0\operatorname{mod}g$. Adding these two congruences
together, we find $xsa+ysb\equiv0\operatorname{mod}g$. Now,%
\[
h=s\underbrace{\gcd\left(  a,b\right)  }_{=xa+yb}=s\left(  xa+yb\right)
=xsa+ysb\equiv0\operatorname{mod}g.
\]
In other words, $g\mid h$. Thus, we have proven $g\mid h$.

\textit{Proof of }$h\mid g$\textit{:} Proposition \ref{prop.ent.gcd.props1}
\textbf{(f)} yields $\gcd\left(  a,b\right)  \mid a$ and $\gcd\left(
a,b\right)  \mid b$. Also, $s\mid s$. Hence, Proposition \ref{prop.ent.div.2}
\textbf{(c)} (applied to $s,\gcd\left(  a,b\right)  ,s,a$ instead of
$a_{1},a_{2},b_{1},b_{2}$) yields $s\gcd\left(  a,b\right)  \mid sa$.
Similarly, $s\gcd\left(  a,b\right)  \mid sb$. Hence, Lemma
\ref{lem.ent.gcd.uniprop} (applied to $s\gcd\left(  a,b\right)  $, $sa$ and
$sb$ instead of $m$, $a$ and $b$) yields $s\gcd\left(  a,b\right)  \mid
\gcd\left(  sa,sb\right)  $. In view of $g=\gcd\left(  sa,sb\right)  $ and
$h=s\gcd\left(  a,b\right)  $, this rewrites as $h\mid g$. So we have proven
$h\mid g$.

Now, Exercise \ref{exe.ent.div.abba} (applied to $g$ and $h$ instead of $a$
and $b$) yields $\left\vert g\right\vert =\left\vert h\right\vert $.

But recall that a gcd of any finitely many integers is nonnegative (by
Definition \ref{def.ent.gcd.gcd}). Hence, in particular, $\gcd\left(
a,b\right)  $ and $\gcd\left(  sa,sb\right)  $ are nonnegative. From
$g=\gcd\left(  sa,sb\right)  $, we obtain%
\[
\left\vert g\right\vert =\left\vert \gcd\left(  sa,sb\right)  \right\vert
=\gcd\left(  sa,sb\right)
\]
(since $\gcd\left(  sa,sb\right)  $ is nonnegative). Also, from $h=s\gcd
\left(  a,b\right)  $, we obtain%
\begin{align*}
\left\vert h\right\vert  &  =\left\vert s\gcd\left(  a,b\right)  \right\vert
=\left\vert s\right\vert \cdot\underbrace{\left\vert \gcd\left(  a,b\right)
\right\vert }_{\substack{=\gcd\left(  a,b\right)  \\\text{(since }\gcd\left(
a,b\right)  \\\text{is nonnegative)}}}\ \ \ \ \ \ \ \ \ \ \left(  \text{by
(\ref{eq.ent.div.abs(xy)})}\right) \\
&  =\left\vert s\right\vert \gcd\left(  a,b\right)  .
\end{align*}
Hence,
\[
\gcd\left(  sa,sb\right)  =\left\vert g\right\vert =\left\vert h\right\vert
=\left\vert s\right\vert \gcd\left(  a,b\right)  .
\]
This proves Corollary \ref{cor.ent.gcd.sa,sb}.
\end{proof}

\begin{exercise}
\label{exe.ent.gcd.div}Let $a_{1},a_{2},b_{1},b_{2}\in\mathbb{Z}$ satisfy
$a_{1}\mid b_{1}$ and $a_{2}\mid b_{2}$. Prove that%
\[
\gcd\left(  a_{1},a_{2}\right)  \mid\gcd\left(  b_{1},b_{2}\right)  .
\]

\end{exercise}

\begin{proof}
[Solution to Exercise \ref{exe.ent.gcd.div}.]TODO. (HW?)
\end{proof}

\begin{exercise}
\label{exe.ent.gcd.abs}Let $a,b\in\mathbb{Z}$. Prove that $\gcd\left(
a,b\right)  =\gcd\left(  \left\vert a\right\vert ,\left\vert b\right\vert
\right)  $.
\end{exercise}

\begin{proof}
[Solution to Exercise \ref{exe.ent.gcd.abs}.] TODO.
\end{proof}

\subsubsection{gcds of multiple numbers}

The following theorem generalizes some of the previous facts to gcds of
multiple integers:

\begin{theorem}
\label{thm.ent.gcd.uniprop-mul}Let $b_{1},b_{2},\ldots,b_{k}$ be some integers.

\textbf{(a)} For each $m\in\mathbb{Z}$, we have the following logical
equivalence:%
\[
\left(  m\mid b_{i}\text{ for all }i\in\left\{  1,2,\ldots,k\right\}  \right)
\ \Longleftrightarrow\ \left(  m\mid\gcd\left(  b_{1},b_{2},\ldots
,b_{k}\right)  \right)  .
\]


\textbf{(b)} The common divisors of $b_{1},b_{2},\ldots,b_{k}$ are precisely
the divisors of $\gcd\left(  b_{1},b_{2},\ldots,b_{k}\right)  $.

\textbf{(c)} We have $\operatorname*{Div}\left(  b_{1},b_{2},\ldots
,b_{k}\right)  =\operatorname*{Div}\left(  \gcd\left(  b_{1},b_{2}%
,\ldots,b_{k}\right)  \right)  $.

\textbf{(d)} We have%
\[
\gcd\left(  b_{1},b_{2},\ldots,b_{k}\right)  =\gcd\left(  \gcd\left(
b_{1},b_{2},\ldots,b_{k-1}\right)  ,b_{k}\right)  .
\]


\textbf{(e)} There exist integers $x_{1},x_{2},\ldots,x_{k}$ such that
$x_{1}b_{1}+x_{2}b_{2}+\cdots+x_{k}b_{k}=1$.
\end{theorem}

\begin{proof}
[Proof of Theorem \ref{thm.ent.gcd.uniprop-mul}.] TODO!
\end{proof}

\begin{theorem}
\label{thm.ent.gcd.split}Let $b_{1},b_{2},\ldots,b_{k}$ be some integers, and
let $c_{1},c_{2},\ldots,c_{\ell}$ be some integers. Then,%
\begin{align*}
& \gcd\left(  b_{1},b_{2},\ldots,b_{k},c_{1},c_{2},\ldots,c_{\ell}\right)  \\
& =\gcd\left(  \gcd\left(  b_{1},b_{2},\ldots,b_{k}\right)  ,\gcd\left(
c_{1},c_{2},\ldots,c_{\ell}\right)  \right)  .
\end{align*}

\end{theorem}

\begin{proof}
[Proof of Theorem \ref{thm.ent.gcd.split}.] TODO! Prove left$\leq$right and
right$\leq$left, I think.
\end{proof}

Theorem \ref{thm.ent.gcd.split} is the reason why most properties of gcds of
multiple numbers can be derived from corresponding properties of gcds of two
numbers. For example, we can easily prove the following analogue of Corollary
\ref{cor.ent.gcd.sa,sb} for gcds of three numbers:

\begin{exercise}
\label{exe.ent.gcd.sa,sb,sc}Let $s,a,b,c\in\mathbb{Z}$. Prove that
$\gcd\left(  sa,sb,sc\right)  =\left\vert s\right\vert \gcd\left(
a,b,c\right)  $.
\end{exercise}

\begin{proof}
[Solution to Exercise \ref{exe.ent.gcd.sa,sb,sc}.] TODO!
\end{proof}

\begin{center}
\textbf{2019-02-06 lecture}
\end{center}

\subsection{Coprime integers}

\subsubsection{Definition}

The concept of a gcd leads to one of the most important notions of number theory:

\begin{definition}
\label{def.ent.coprime.coprime}Let $a$ and $b$ be two integers. We say that
$a$ is \textit{coprime} to $b$ if and only if $\gcd\left(  a,b\right)  =1$.
\end{definition}

Instead of \textquotedblleft coprime\textquotedblright, some authors say
\textquotedblleft relatively prime\textquotedblright. Instead of saying
\textquotedblleft$a$ and $b$

\begin{example}
\textbf{(a)} The number $2$ is coprime to $3$, since $\gcd\left(  2,3\right)
=1$.

\textbf{(b)} More generally, for any integer $a$, the number $a$ is coprime to
$a+1$. To prove this, we note that%
\begin{align*}
\gcd\left(  a,\underbrace{a}_{=1a}+1\right)   &  =\gcd\left(  a,1a+1\right)
=\gcd\left(  a,1\right) \\
&  \ \ \ \ \ \ \ \ \ \ \left(  \text{by Proposition \ref{prop.ent.gcd.props1}
\textbf{(c)}, applied to }u=1\text{ and }b=1\right) \\
&  \mid1\ \ \ \ \ \ \ \ \ \ \left(  \text{by Proposition
\ref{prop.ent.gcd.props1} \textbf{(e)}, applied to }b=1\right)  ,
\end{align*}
and thus $\gcd\left(  a,a+1\right)  =1$ (since $\gcd\left(  a,a+1\right)  $ is
a positive integer), which means that $a$ is coprime to $a+b$.

\textbf{(c)} The number $6$ is not coprime to $15$, since $\gcd\left(
6,15\right)  =3\neq1$.
\end{example}

Following Donald Knuth, we introduce a slightly quaint notation:

\begin{definition}
\label{def.ent.coprime.perp}Let $a$ and $b$ be two integers. We write
\textquotedblleft$a\perp b$\textquotedblright\ to signify that $a$ is coprime
to $b$.
\end{definition}

Note that the \textquotedblleft$\perp$\textquotedblright\ relation is symmetric:

\begin{proposition}
\label{prop.ent.coprime.perp-symm}Let $a$ and $b$ be two integers. Then,
$a\perp b$ if and only if $b\perp a$.
\end{proposition}

\begin{proof}
[Proof of Proposition \ref{prop.ent.coprime.perp-symm}.]We have the following
chain of equivalences:%
\begin{align*}
\left(  a\perp b\right)  \  &  \Longleftrightarrow\ \left(  a\text{ is coprime
to }b\right)  \ \ \ \ \ \ \ \ \ \ \left(  \text{by the definition of
\textquotedblleft}\perp\text{\textquotedblright}\right) \\
&  \Longleftrightarrow\ \left(  \gcd\left(  a,b\right)  =1\right)
\ \ \ \ \ \ \ \ \ \ \left(  \text{by the definition of \textquotedblleft
coprime\textquotedblright}\right) \\
&  \Longleftrightarrow\ \left(  \gcd\left(  b,a\right)  =1\right)
\ \ \ \ \ \ \ \ \ \ \left(
\begin{array}
[c]{c}%
\text{since Proposition \ref{prop.ent.gcd.props1} \textbf{(b)}}\\
\text{yields }\gcd\left(  a,b\right)  =\gcd\left(  b,a\right)
\end{array}
\right) \\
&  \Longleftrightarrow\ \left(  b\text{ is coprime to }a\right)
\ \ \ \ \ \ \ \ \ \ \left(  \text{by the definition of \textquotedblleft
coprime\textquotedblright}\right) \\
&  \Longleftrightarrow\ \left(  b\perp a\right)  \ \ \ \ \ \ \ \ \ \ \left(
\text{by the definition of \textquotedblleft}\perp\text{\textquotedblright%
}\right)  .
\end{align*}
This proves Proposition \ref{prop.ent.coprime.perp-symm}.
\end{proof}

\begin{definition}
Let $a$ and $b$ be two integers. Proposition \ref{prop.ent.coprime.perp-symm}
shows that $a$ is coprime to $b$ if and only if $b$ is coprime to $a$. Hence,
we shall sometimes use a more symmetric terminology for this situation: We
shall say that \textquotedblleft$a$ and $b$ \textit{are coprime}%
\textquotedblright\ to mean that $a$ is coprime to $b$ (or, equivalently, that
$b$ is coprime to $a$).
\end{definition}

\subsubsection{Properties of coprime integers}

We can now state multiple theorems about coprime numbers. The first one states
that we can \textquotedblleft cancel\textquotedblright\ a factor $b$ from a
divisibility $a\mid bc$ as long as this factor is coprime to $a$:

\begin{theorem}
\label{thm.ent.coprime.cancel}Let $a,b,c\in\mathbb{Z}$ satisfy $a\mid bc$ and
$a\perp b$. Then, $a\mid c$.
\end{theorem}

\begin{proof}
[Proof of Theorem \ref{thm.ent.coprime.cancel}.]We have $a\perp b$; in other
words, $a$ is coprime to $b$ (by Definition \ref{def.ent.coprime.perp}). In
other words, $\gcd\left(  a,b\right)  =1$ (by the definition of
\textquotedblleft coprime\textquotedblright). Now, Theorem
\ref{thm.ent.gcd.cancel} yields $a\mid\underbrace{\gcd\left(  a,b\right)
}_{=1}\cdot c=c$. This proves Theorem \ref{thm.ent.coprime.cancel}.
\end{proof}

The next one lets us \textquotedblleft combine\textquotedblright\ two
divisibilities $a\mid c$ and $b\mid c$ to $ab\mid c$ as long as $a$ and $b$
are coprime:

\begin{theorem}
\label{thm.ent.coprime.combine}Let $a,b,c\in\mathbb{Z}$ satisfy $a\mid c$ and
$b\mid c$ and $a\perp b$. Then, $ab\mid c$.
\end{theorem}

\begin{proof}
[Proof of Theorem \ref{thm.ent.coprime.combine}.]We have $a\perp b$; in other
words, $a$ is coprime to $b$ (by Definition \ref{def.ent.coprime.perp}). In
other words, $\gcd\left(  a,b\right)  =1$ (by the definition of
\textquotedblleft coprime\textquotedblright). Now, Theorem
\ref{thm.ent.gcd.combine} yields $ab\mid\underbrace{\gcd\left(  a,b\right)
}_{=1}\cdot c=c$. This proves Theorem \ref{thm.ent.coprime.combine}.
\end{proof}

\begin{theorem}
\label{thm.ent.coprime.modinv}Let $a,n\in\mathbb{Z}$.

\textbf{(a)} There exists a $b\in\mathbb{Z}$ such that $ab\equiv\gcd\left(
a,n\right)  \operatorname{mod}n$.

\textbf{(b)} If $a\perp n$, then there exists a $b\in\mathbb{Z}$ such that
$ab\equiv1\operatorname{mod}n$.
\end{theorem}

\begin{proof}
[Proof of Theorem \ref{thm.ent.coprime.modinv}.] \textbf{(a)} Theorem
\ref{thm.ent.gcd.bezout} (applied to $b=n$) yields that there exist integers
$x\in\mathbb{Z}$ and $y\in\mathbb{Z}$ such that $\gcd\left(  a,n\right)
=xa+yn$. Consider these $x$ and $y$. We have $ax=xa\equiv
xa+yn\operatorname{mod}n$ (since $xa-\left(  xa+yn\right)  =-yn=n\left(
-y\right)  $ is clearly divisible by $n$). Thus, $ax\equiv xa+yn=\gcd\left(
a,n\right)  \operatorname{mod}n$. Thus, there exists a $b\in\mathbb{Z}$ such
that $ab\equiv\left(  a,n\right)  \operatorname{mod}n$ (namely, $b=x$). This
proves Theorem \ref{thm.ent.coprime.modinv} \textbf{(a)}.

\textbf{(b)} Assume that $a\perp n$. In other words, $a$ is coprime to $n$ (by
Definition \ref{def.ent.coprime.perp}). In other words, $\gcd\left(
a,n\right)  =1$ (by the definition of \textquotedblleft
coprime\textquotedblright). Now, Theorem \ref{thm.ent.coprime.modinv}
\textbf{(a)} yields that there exists a $b\in\mathbb{Z}$ such that
$ab\equiv\gcd\left(  a,n\right)  \operatorname{mod}n$. In view of $\gcd\left(
a,n\right)  =1$, this rewrites as follows: There exists a $b\in\mathbb{Z}$
such that $ab\equiv1\operatorname{mod}n$. This proves Theorem
\ref{thm.ent.coprime.modinv} \textbf{(b)}.
\end{proof}

If $a,n\in\mathbb{Z}$, then an integer $b\in\mathbb{Z}$ satisfying
$ab\equiv1\operatorname{mod}n$ is called a \textit{modular inverse} of $a$
modulo $n$. Theorem \ref{thm.ent.coprime.modinv} \textbf{(b)} shows that such
a modular inverse always exists when $a\perp n$; it is easy to show that the
converse also holds (i.e., if a modular inverse of $a$ modulo $n$ exists, then
$a\perp n$). The word \textquotedblleft modular inverse\textquotedblright\ is
chosen in analogy to the usual concept of an \textquotedblleft
inverse\textquotedblright\ (which stands for an integer $b\in\mathbb{Z}$
satisfying $ab=1$; this exists if and only if $a$ equals $1$ or $-1$).

\begin{theorem}
\label{thm.ent.coprime.ab-to-c}Let $a,b,c\in\mathbb{Z}$ such that $a\perp c$
and $b\perp c$. Then, $ab\perp c$.
\end{theorem}

\begin{proof}
[Proof of Theorem \ref{thm.ent.coprime.ab-to-c}.] TODO!
\end{proof}

\subsubsection{An application}

\begin{exercise}
\label{exe.ent.coprime.1+2+...+n}Let $n\in\mathbb{N}$. Let $k$ be an odd
positive integer. Prove that%
\[
1+2+\cdots+n\mid1^{k}+2^{k}+\cdots+n^{k}.
\]

\end{exercise}

\begin{proof}
[Solution to Exercise \ref{exe.ent.coprime.1+2+...+n}.] Recall Little Gauss
$1+2+\cdots+n=\dfrac{n\left(  n+1\right)  }{2}$. Multiply by $2$ and treat $n$
and $n+1$ separately. TODO
\end{proof}

TODO:

\begin{itemize}
\item $a\mid bc,\ b\perp c\ \Longrightarrow\ a=uv$ with $u\mid b$ and $v\mid
c$.

Better: if $b\perp c$, then $\gcd\left(  a,bc\right)  =\gcd\left(  a,b\right)
\cdot\gcd\left(  a,c\right)  $. Rather, hw.
\end{itemize}

\subsection{Lowest common multiples}

\begin{itemize}
\item lcm.

\item gcd*lcm.
\end{itemize}

\subsection{The Chinese remainder theorem (elementary form)}

\begin{theorem}
\label{thm.ent.crt1}Let $m$ and $n$ be two coprime integers. Let
$a,b\in\mathbb{Z}$.

\textbf{(a)} There exists an integer $x\in\mathbb{Z}$ such that%
\[
\left(  x\equiv a\operatorname{mod}m\text{ and }x\equiv b\operatorname{mod}%
n\right)  .
\]


\textbf{(b)} If $x_{1}$ and $x_{2}$ are two such integers $x$, then
$x_{1}\equiv x_{2}\operatorname{mod}mn$.
\end{theorem}

Theorem \ref{thm.ent.crt1} is known as the \textit{Chinese remainder theorem}.
(Sunzi, 3rd Century AD.)

\begin{proof}
[Proof of Theorem \ref{thm.ent.crt1}.]
\end{proof}

TODO:

\begin{itemize}
\item primes

\item Erathosthenes

\item $p>1$ prime $\Longleftrightarrow$ $p$ coprime to $1,2,\ldots,p-1$

\item each $n>0$ is prod of primes (allow empty)

\item $p\mid ab$ $\Longrightarrow$ $p\mid a$ or $p\mid b$

\item unique prod of primes

\item factorization with exponents

\item Division/$9$ criterion. This leads to...
\end{itemize}

\subsection{\label{sect.ent.subst-mod}Substitutivity for congruences}

In\ Section \ref{sect.ent.subst-chain}, we have learnt that congruences modulo
an integer $n$ can be chained together like equalities. A further important
feature of congruences is the principle of \textit{substitutivity for
congruences}. This is yet another way in which congruences behave like
equalities. We are not going to state it fully formally (as it is a
meta-mathematical principle), but merely explain its meaning. Later on, once
we understand what the rings $\mathbb{Z}/n$ (for integer $n$) are, we will no
longer need this principle, since it will just boil down to \textquotedblleft
equal things can be substituted for one another\textquotedblright\ (the whole
point of $\mathbb{Z}/n$ is to \textquotedblleft make congruent numbers
equal\textquotedblright); but for now, we cannot treat \textquotedblleft
congruent modulo $n$\textquotedblright\ as \textquotedblleft
equal\textquotedblright, so we have to state it.

You are probably used to making computations like these:%
\begin{align*}
\underbrace{\left(  a+b\right)  ^{2}}_{=a^{2}+2ab+b^{2}}+\underbrace{\left(
a-b\right)  ^{2}}_{=a^{2}-2ab+b^{2}}  &  =\left(  a^{2}+2ab+b^{2}\right)
+\left(  a^{2}-2ab+b^{2}\right) \\
&  =\underbrace{a^{2}+a^{2}}_{=2a^{2}}+\underbrace{b^{2}+b^{2}}_{=2b^{2}%
}=2a^{2}+2b^{2}%
\end{align*}
(for any two numbers $a$ and $b$). What is going on in these underbraces (like
\textquotedblleft$\underbrace{\left(  a+b\right)  ^{2}}_{=a^{2}+2ab+b^{2}}%
$\textquotedblright)? Something pretty simple is going on: You are replacing a
number (in this case, $\left(  a+b\right)  ^{2}$) by an equal number (in this
case, $a^{2}+2ab+b^{2}$). This relies on a fundamental principle of
mathematics (called the \textit{principle of substitutivity for equalities}),
which says that an object in an expression can indeed be replaced by any
object equal to it (without changing the value of the expression). (This is
also known as \textit{Leibniz's equality law}.) To be precise, we are using
this principle twice in some of our equality signs above, since we are making
several replacements at the same time; but this is fine (we can just do the
replacement one by one instead).

We would like to have a similar principle for congruences modulo $n$: We would
like to be able to replace any integer by an integer congruent to it modulo
$n$. For example, we would like to be able to say that if seven integers
$a,a^{\prime},b,b^{\prime},c,c^{\prime},n$ satisfy $a\equiv a^{\prime
}\operatorname{mod}n$ and $b\equiv b^{\prime}\operatorname{mod}n$ and $c\equiv
c^{\prime}\operatorname{mod}n$, then%
\[
\underbrace{b}_{\equiv b^{\prime}\operatorname{mod}n}\ \ \underbrace{c}%
_{\equiv c^{\prime}\operatorname{mod}n}+\underbrace{c}_{\equiv c^{\prime
}\operatorname{mod}n}\ \ \underbrace{a}_{\equiv a^{\prime}\operatorname{mod}%
n}+\underbrace{a}_{\equiv a^{\prime}\operatorname{mod}n}\ \ \underbrace{b}%
_{\equiv b^{\prime}\operatorname{mod}n}\equiv b^{\prime}c^{\prime}+c^{\prime
}a^{\prime}+a^{\prime}b^{\prime}\operatorname{mod}n.
\]


We have to be careful with this: For example, we run into troubles if division
is involved in our expressions. For example, we have $6\equiv
9\operatorname{mod}3$, but we do not have $\underbrace{6}_{\equiv
9\operatorname{mod}3}/3\equiv9/3\operatorname{mod}3$. Similarly,
exponentiation can be problematic. So we need to state the principle we are
using here in clearer terms, so that we know what we can do.

For this whole Section \ref{sect.ent.subst-mod}, we fix an integer $n$.

The \textit{principle of substitutivity for equalities} says the following:

\begin{statement}
\textit{Principle of substitutivity for equalities (PSE):} If two objects $x$
and $x^{\prime}$ are equal, and if we have any expression $A$ that involves
the object $x$, then we can replace this $x$ (or, more precisely, any
arbitrary appearance of $x$ in $A$) in $A$ by $x^{\prime}$; the resulting
expression $A^{\prime}$ will be equal to $A$.
\end{statement}

Here are two examples of how this principle can be used:

\begin{itemize}
\item If $a,b,c,d,e,c^{\prime}$ are numbers such that $c=c^{\prime}$, then the
PSE says that we can replace $c$ by $c^{\prime}$ in the expression $a\left(
b-\left(  c+d\right)  e\right)  $, and the resulting expression $a\left(
b-\left(  c^{\prime}+d\right)  e\right)  $ will be equal to $a\left(
b-\left(  c+d\right)  e\right)  $; that is, we have%
\begin{equation}
a\left(  b-\left(  c+d\right)  e\right)  =a\left(  b-\left(  c^{\prime
}+d\right)  e\right)  . \label{eq.mod.substitutivity-nums.1}%
\end{equation}


\item If $a,b,c,a^{\prime}$ are numbers such that $a=a^{\prime}$, then
\begin{equation}
\left(  a-b\right)  \left(  a+b\right)  =\left(  a^{\prime}-b\right)  \left(
a+b\right)  , \label{eq.mod.substitutivity-nums.2}%
\end{equation}
because the PSE allows us to replace the first $a$ appearing in the expression
$\left(  a-b\right)  \left(  a+b\right)  $ by an $a^{\prime}$. (We can also
replace the second $a$ by $a^{\prime}$, of course.)
\end{itemize}

More generally, we can make several such replacements at the same time.

The PSE is one of the headstones of mathematical logic; it is the essence of
what it means for two objects to be equal.

The \textit{principle of substitutivity for congruences} is similar, but far
less fundamental; it says the following:

\begin{statement}
\textit{Principle of substitutivity for congruences (PSC):} If two numbers $x$
and $x^{\prime}$ are congruent to each other modulo $n$ (that is, $x\equiv
x^{\prime}\operatorname{mod}n$), and if we have any expression $A$ that
involves only integers, addition, subtraction and multiplication, and involves
the object $x$, then we can replace this $x$ (or, more precisely, any
arbitrary appearance of $x$ in $A$) in $A$ by $x^{\prime}$; the resulting
expression $A^{\prime}$ will be congruent to $A$ modulo $n$.
\end{statement}

This principle is less general than the PSE, since it only applies to
expressions that are built from integers and certain operations (note that
division is not one of these operations). But it still lets us prove analogues
of our above examples (\ref{eq.mod.substitutivity-nums.1}) and
(\ref{eq.mod.substitutivity-nums.2}):

\begin{itemize}
\item If $a,b,c,d,e,c^{\prime}$ are integers such that $c\equiv c^{\prime
}\operatorname{mod}n$, then the PSC says that we can replace $c$ by
$c^{\prime}$ in the expression $a\left(  b-\left(  c+d\right)  e\right)  $,
and the resulting expression $a\left(  b-\left(  c^{\prime}+d\right)
e\right)  $ will be congruent to $a\left(  b-\left(  c+d\right)  e\right)  $
modulo $n$; that is, we have%
\begin{equation}
a\left(  b-\left(  c+d\right)  e\right)  \equiv a\left(  b-\left(  c^{\prime
}+d\right)  e\right)  \operatorname{mod}n.
\label{eq.mod.substitutivity-congs.1}%
\end{equation}


\item If $a,b,c,a^{\prime}$ are integers such that $a\equiv a^{\prime
}\operatorname{mod}n$, then
\begin{equation}
\left(  a-b\right)  \left(  a+b\right)  \equiv\left(  a^{\prime}-b\right)
\left(  a+b\right)  \operatorname{mod}n, \label{eq.mod.substitutivity-congs.2}%
\end{equation}
because the PSC allows us to replace the first $a$ appearing in the expression
$\left(  a-b\right)  \left(  a+b\right)  $ by an $a^{\prime}$. (We can also
replace the second $a$ by $a^{\prime}$, of course.)
\end{itemize}

We shall not prove the PSC, since we have not formalized it (after all, we
have not defined what an \textquotedblleft expression\textquotedblright\ is).
But we shall prove the specific congruences
(\ref{eq.mod.substitutivity-congs.1}) and (\ref{eq.mod.substitutivity-congs.2}%
) using Proposition \ref{prop.ent.mod.basics}; the way in which we prove these
congruences is symptomatic: Every congruence obtained from the PSC can be
proven in a manner like these. Thus, the proofs of
(\ref{eq.mod.substitutivity-congs.1}) and (\ref{eq.mod.substitutivity-congs.2}%
) given below can serve as templates which can easily be adapted to any other
situation in which an application of the PSC needs to be justified.

\begin{proof}
[Proof of (\ref{eq.mod.substitutivity-congs.1}).]Let $n$ be any integer, and
let $a,b,c,d,e,c^{\prime}$ be integers such that $c\equiv c^{\prime
}\operatorname{mod}n$.

Adding the congruence\footnote{Proposition \ref{prop.ent.mod.basics}
\textbf{(d)} shows that we can add, subtract and multiply congruences modulo
$n$ at will. We are using this freedom here and will use it many times below.}
$c\equiv c^{\prime}\operatorname{mod}n$ with the congruence $d\equiv
d\operatorname{mod}n$ (which follows from Proposition
\ref{prop.ent.mod.basics} \textbf{(a)}), we obtain $c+d\equiv c^{\prime
}+d\operatorname{mod}n$. Multiplying this congruence with the congruence
$e\equiv e\operatorname{mod}n$ (which follows from Proposition
\ref{prop.ent.mod.basics} \textbf{(a)}), we obtain $\left(  c+d\right)
e\equiv\left(  c^{\prime}+d\right)  e\operatorname{mod}n$. Subtracting this
congruence from the congruence $b\equiv b\operatorname{mod}n$ (which, again,
follows from Proposition \ref{prop.ent.mod.basics} \textbf{(a)}), we obtain
$b-\left(  c+d\right)  e\equiv b-\left(  c^{\prime}+d\right)
e\operatorname{mod}n$. Multiplying the congruence $a\equiv a\operatorname{mod}%
n$ (which follows from Proposition \ref{prop.ent.mod.basics} \textbf{(a)})
with this congruence, we obtain $a\left(  b-\left(  c+d\right)  e\right)
\equiv a\left(  b-\left(  c^{\prime}+d\right)  e\right)  \operatorname{mod}n$.
This proves (\ref{eq.mod.substitutivity-congs.1}).
\end{proof}

\begin{proof}
[Proof of (\ref{eq.mod.substitutivity-congs.2}).]Let $n$ be any integer, and
let $a,b,c,a^{\prime}$ be integers such that $a\equiv a^{\prime}%
\operatorname{mod}n$.

Subtracting the congruence $b\equiv b\operatorname{mod}n$ (which follows from
Proposition \ref{prop.ent.mod.basics} \textbf{(a)}) from the congruence
$a\equiv a^{\prime}\operatorname{mod}n$, we obtain $a-b\equiv a^{\prime
}-b\operatorname{mod}n$. Multiplying this congruence with the congruence
$a+b\equiv a+b\operatorname{mod}n$ (which follows from Proposition
\ref{prop.ent.mod.basics} \textbf{(a)}), we obtain $\left(  a-b\right)
\left(  a+b\right)  \equiv\left(  a^{\prime}-b\right)  \left(  a+b\right)
\operatorname{mod}n$. This proves (\ref{eq.mod.substitutivity-congs.2}).
\end{proof}

As we said, these two proofs are exemplary: Any congruence obtained from the
PSC can be proven in such a way (starting with the congruence $x\equiv
x^{\prime}\operatorname{mod}n$, and then \textquotedblleft
wrapping\textquotedblright\ it up in the expression $A$ by repeatedly adding,
multiplying and subtracting congruences that follow from Proposition
\ref{prop.ent.mod.basics} \textbf{(a)}).

When we apply the PSC, we shall use underbraces to point out which integers we
are replacing. For example, when deriving (\ref{eq.mod.substitutivity-congs.1}%
) from this principle, we shall write%
\[
a\left(  b-\left(  \underbrace{c}_{\equiv c^{\prime}\operatorname{mod}%
n}+d\right)  e\right)  \equiv a\left(  b-\left(  c^{\prime}+d\right)
e\right)  \operatorname{mod}n,
\]
in order to stress that we are replacing $c$ by $c^{\prime}$. Likewise, when
deriving (\ref{eq.mod.substitutivity-congs.2}) from the PSC, we shall write%
\[
\left(  \underbrace{a}_{\equiv a^{\prime}\operatorname{mod}n}-b\right)
\left(  a+b\right)  \equiv\left(  a^{\prime}-b\right)  \left(  a+b\right)
\operatorname{mod}n,
\]
in order to stress that we are replacing the first $a$ (but not the second
$a$) by $a^{\prime}$.

The PSC allows us to replace a \textbf{single} integer $x$ appearing in an
expression by another integer $x^{\prime}$ that is congruent to $x$ modulo
$n$. Applying this principle many times, we thus conclude that we can also
replace \textbf{several} integers at the same time (because we can get to the
same result by performing these replacements one at a time, and Proposition
\ref{prop.mod.chain} shows that the final result will be congruent to the
original result). For example, if seven integers $a,a^{\prime},b,b^{\prime
},c,c^{\prime},n$ satisfy $a\equiv a^{\prime}\operatorname{mod}n$ and $b\equiv
b^{\prime}\operatorname{mod}n$ and $c\equiv c^{\prime}\operatorname{mod}n$,
then%
\begin{equation}
bc+ca+ab\equiv b^{\prime}c^{\prime}+c^{\prime}a^{\prime}+a^{\prime}b^{\prime
}\operatorname{mod}n, \label{eq.mod.substitutivity-congs.3}%
\end{equation}
because we can replace all the six integers $b,c,c,a,a,b$ in the expression
$bc+ca+ab$ (listed in the order of their appearance in this expression) by
$b^{\prime},c^{\prime},c^{\prime},a^{\prime},a^{\prime},b^{\prime}$,
respectively. If we want to derive this from the principle of substitutivity
for congruences, we must perform the replacements one at a time, e.g., as
follows:%
\begin{align*}
\underbrace{b}_{\equiv b^{\prime}\operatorname{mod}n}c+ca+ab  &  \equiv
b^{\prime}\underbrace{c}_{\equiv c^{\prime}\operatorname{mod}n}+ca+ab\equiv
b^{\prime}c^{\prime}+\underbrace{c}_{\equiv c^{\prime}\operatorname{mod}%
n}a+ab\\
&  \equiv b^{\prime}c^{\prime}+c^{\prime}\underbrace{a}_{\equiv a^{\prime
}\operatorname{mod}n}+ab\equiv b^{\prime}c^{\prime}+c^{\prime}a^{\prime
}+\underbrace{a}_{\equiv a^{\prime}\operatorname{mod}n}b\\
&  \equiv b^{\prime}c^{\prime}+c^{\prime}a^{\prime}+a^{\prime}\underbrace{b}%
_{\equiv b^{\prime}\operatorname{mod}n}\equiv b^{\prime}c^{\prime}+c^{\prime
}a^{\prime}+a^{\prime}b^{\prime}\operatorname{mod}n.
\end{align*}
Of course, we shall always just show the replacements as a single step:%
\[
\underbrace{b}_{\equiv b^{\prime}\operatorname{mod}n}\ \ \underbrace{c}%
_{\equiv c^{\prime}\operatorname{mod}n}+\underbrace{c}_{\equiv c^{\prime
}\operatorname{mod}n}\ \ \underbrace{a}_{\equiv a^{\prime}\operatorname{mod}%
n}+\underbrace{a}_{\equiv a^{\prime}\operatorname{mod}n}\ \ \underbrace{b}%
_{\equiv b^{\prime}\operatorname{mod}n}\equiv b^{\prime}c^{\prime}+c^{\prime
}a^{\prime}+a^{\prime}b^{\prime}\operatorname{mod}n.
\]


\begin{thebibliography}{999999999}                                                                                        %


\bibitem[Armstr18]{Armstrong}Drew Armstrong, \textit{Abstract Algebra I},
2018.\newline\url{http://www.math.miami.edu/~armstrong/561fa18.php}

\bibitem[Artin10]{Artin}Michael Artin, \textit{Algebra}, 2nd edition, Pearson 2010.

\bibitem[Bosch18]{Bosch}Siegfried Bosch, \textit{Algebra -- From the Viewpoint
of Galois Theory}, Springer 2018. \newline\url{https://www.springer.com/la/book/9783319951768}

\bibitem[Burton10]{Burton}David M. Burton, \textit{Elementary Number Theory},
7th edition, McGraw-Hill 2010.

\bibitem[Conrad*]{Conrad*}Keith Conrad, \textit{Expository notes
(\textquotedblleft blurbs\textquotedblright)}.\newline\url{https://kconrad.math.uconn.edu/blurbs/}

\bibitem[ConradG]{Conrad-Gauss}Keith Conrad, \textit{The Gaussian
integers}.\newline\url{http://www.math.uconn.edu/~kconrad/blurbs/ugradnumthy/Zinotes.pdf}

\bibitem[Day16]{Day}Martin V. Day, \textit{An Introduction to Proofs and the
Mathematical Vernacular}, 7 December 2016.\newline%
\url{https://www.math.vt.edu/people/day/ProofsBook/IPaMV.pdf} .

\bibitem[DumFoo04]{Dummit-Foote}David S. Dummit, Richard M. Foote,
\textit{Abstract Algebra}, 3rd edition, Wiley 2004. \newline See
\url{http://www.cems.uvm.edu/~rfoote/errata_3rd_edition.pdf} for errata.

\bibitem[GalQua17]{Gallier-RSA}Jean Gallier, Jocelyn Quaintance, \textit{Notes
on Primality Testing And Public Key Cryptography, Part 1}, 8 November
2017.\newline\url{https://www.cis.upenn.edu/~jean/RSA-primality-testing.pdf}

\bibitem[Goodma16]{Goodman}Frederick M. Goodman, \textit{Algebra: Abstract and
Concrete}, edition 2.6, 12 October 2016.\newline\url{http://homepage.divms.uiowa.edu/~goodman/algebrabook.dir/algebrabook.html}

\bibitem[Grinbe15]{detnotes}Darij Grinberg, \textit{Notes on the combinatorial
fundamentals of algebra}, 10 January 2019.\newline%
\url{http://www.cip.ifi.lmu.de/~grinberg/primes2015/sols.pdf} \newline The
numbering of theorems and formulas in this link might shift when the project
gets updated; for a \textquotedblleft frozen\textquotedblright\ version whose
numbering is guaranteed to match that in the citations above, see
\url{https://github.com/darijgr/detnotes/releases/tag/2019-01-10} .

\bibitem[Grinbe16]{floor}Darij Grinberg, \textit{18.781 (Spring 2016): Floor
and arithmetic functions}, 19 June 2016.\newline\url{http://www.cip.ifi.lmu.de/~grinberg/floor.pdf}

\bibitem[Hammac18]{Hammack}Richard Hammack, \textit{Book of Proof}, 3rd
edition 2018.\newline\url{http://www.people.vcu.edu/~rhammack/BookOfProof/}

\bibitem[Heffer17]{Hefferon}Jim Hefferon, \textit{Linear Algebra}, 3rd edition
2017.\newline\url{http://joshua.smcvt.edu/linearalgebra/}

\bibitem[Knapp16a]{Knapp1}Anthony W. Knapp, \textit{Basic Algebra}, digital
2nd edition 2016. \newline\url{http://www.math.stonybrook.edu/~aknapp/download.html}

\bibitem[Knapp16b]{Knapp2}Anthony W. Knapp, \textit{Advanced Algebra}, digital
2nd edition 2016. \newline\url{http://www.math.stonybrook.edu/~aknapp/download.html}

\bibitem[LeLeMe18]{LeLeMe}Eric Lehman, F. Thomson Leighton, Albert R. Meyer,
\textit{Mathematics for Computer Science}, revised Tuesday 6th June
2018.\newline\url{https://courses.csail.mit.edu/6.042/spring18/mcs.pdf} .

\bibitem[NiZuMo91]{NiZuMo91}Ivan Niven, Herbert S. Zuckerman, Hugh L.
Montgomery, \textit{An Introduction to the Theory of Numbers}, 5th edition 1991.

\bibitem[Pinter10]{Pinter}Charles C. Pinter, \textit{A book of abstract
algebra}, 2nd edition, Dover 2010.\newline\url{https://www.amazon.com/Book-Abstract-Algebra-Second-Mathematics/dp/0486474178}

\bibitem[Siksek15]{Siksek}Samir Siksek, \textit{Introduction to Abstract
Algebra}, 2015.\newline\url{http://homepages.warwick.ac.uk/staff/S.Siksek/teaching/aa/aanotes.pdf}

\bibitem[Strick13]{Strickland}Neil Strickland, \textit{Linear mathematics for
applications}, 2013.\newline\url{https://neil-strickland.staff.shef.ac.uk/courses/MAS201/MAS201.pdf}

\bibitem[UspHea39]{Uspensky-Heaslet}J. V. Uspensky, M. A. Heaslet,
\textit{Elementary Number Theory}, McGraw-Hill 1939.

\bibitem[Waerde91a]{Waerden1}B.L. van der Waerden, \textit{Algebra, Volume I},
translated 7th edition, Springer 1991.

\bibitem[Waerde91b]{Waerden2}B.L. van der Waerden, \textit{Algebra, Volume
II}, translated 5th edition, Springer 1991.
\end{thebibliography}


\end{document}