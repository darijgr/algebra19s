% Like most advanced LaTeX files, this one begins with a lot of
% boilerplate. You don't need to understand (or even read) most of it.
% All you need to do is fill in your name, UMN ID, email address,
% and the number of the pset. (Search for "METADATA" to find the place
% for this.) Then, you can go straight to the "EXERCISE 1"
% section and start writing your solutions.
% The "VARIOUS USEFUL COMMANDS" section is probably worth taking a
% look at at some point.

%----------------------------------------------------------------------------------------
%	PACKAGES AND OTHER DOCUMENT CONFIGURATIONS
%----------------------------------------------------------------------------------------
\documentclass[paper=a4, fontsize=12pt]{scrartcl} % A4 paper and 12pt font size
\usepackage[T1]{fontenc} % Use 8-bit encoding that has 256 glyphs
\usepackage[english]{babel} % English language/hyphenation
\usepackage{amsmath,amsfonts,amsthm,amssymb} % Math packages
\usepackage{mathrsfs}    % More math packages
\usepackage{sectsty}  % Allows customizing section commands
\allsectionsfont{\centering \normalfont\scshape} % Make all section titles centered, the default font and small caps %remove this to left align section tites
\usepackage{hyperref} % Turns cross-references into hyperlinks,
                      % and defines \url and \href commands.
\usepackage{graphicx} % For embedding graphics files.
\usepackage{framed}   % For the "leftbar" environment used below.
\usepackage{ifthen}   % Used for the \powset command below.
\usepackage{lastpage} % for counting the number of pages
\usepackage[headsepline,footsepline,manualmark]{scrlayer-scrpage}
\usepackage[height=10in,a4paper,hmargin={1in,0.8in}]{geometry}
\usepackage[usenames,dvipsnames]{xcolor}
\usepackage{tikz}     % This is a powerful tool to draw vector
                      % graphics inside LaTeX. In particular, you can
                      % use it to draw graphs.
\usepackage{verbatim} % For the "verbatim" environment, in which
                      % special symbols can be used freely without
                      % confusing the compiler. (And it's typeset in
                      % a constant-width font.)
                      % Useful, e.g., for quoting code (or ASCII art).

%\numberwithin{table}{section} % Number tables within sections (i.e. 1.1, 1.2, 2.1, 2.2 instead of 1, 2, 3, 4)

\setlength\parindent{20pt} % Makes indentation for paragraphs longer.
                           % This makes paragraphs stand out more.

%----------------------------------------------------------------------------------------
%	VARIOUS USEFUL COMMANDS
%----------------------------------------------------------------------------------------
% The commands below might be convenient. For example, you probably
% prefer to write $\powset[2]{V}$ for the set of $2$-element subsets
% of $V$, rather than writing $\mathcal{P}_2(V)$.
% Notice that you can easily define your own commands like this.
% Caveat: Some of these commands need to be properly "guarded" when
% they occur in subscripts or superscripts. So you should not write
% $K_\CC$, but rather $K_{\CC}$.
\newcommand{\CC}{\mathbb{C}} % complex numbers
\newcommand{\RR}{\mathbb{R}} % real numbers
\newcommand{\QQ}{\mathbb{Q}} % rational numbers
\newcommand{\NN}{\mathbb{N}} % nonnegative integers
\newcommand{\DD}{{\mathbb{D}}} % dual numbers
\newcommand{\PP}{\mathbb{P}} % positive integers
\newcommand{\Z}[1]{\mathbb{Z}/#1\mathbb{Z}} % integers modulo k
                                            % (syntax: "\Z{k}")
\newcommand{\ZZ}{\mathbb{Z}} % integers
\newcommand{\id}{\operatorname{id}} % identity map
\newcommand{\lcm}{\operatorname{lcm}}
% Lowest common multiple. For historical reasons, LaTeX has a \gcd
% command built in, but not an \lcm command. The preceding line
% rectifies that.
\newcommand{\set}[1]{\left\{ #1 \right\}}
% $\set{...}$ compiles to {...} (set-brackets).
\newcommand{\abs}[1]{\left| #1 \right|}
% $\abs{...}$ compiles to |...| (absolute value, or size of a set).
\newcommand{\tup}[1]{\left( #1 \right)}
% $\tup{...}$ compiles to (...) (parentheses, or tuple-brackets).
\newcommand{\ive}[1]{\left[ #1 \right]}
% $\ive{...}$ compiles to [...] (Iverson bracket, aka truth value; also, set of first n integers).
\newcommand{\floor}[1]{\left\lfloor #1 \right\rfloor}
% $\floor{...}$ compiles to |_..._| (floor function).
\newcommand{\underbrack}[2]{\underbrace{#1}_{\substack{#2}}}
% $\underbrack{...1}{...2}$ yields
% $\underbrace{...1}_{\substack{...2}}$. This is useful for doing
% local rewriting transformations on mathematical expressions with
% justifications. For example, try this out:
% $ \underbrack{(a+b)^2}{= a^2 + 2ab + b^2 \\ \text{(by the binomial formula)}} $
\newcommand{\powset}[2][]{\ifthenelse{\equal{#2}{}}{\mathcal{P}\left(#1\right)}{\mathcal{P}_{#1}\left(#2\right)}}
% $\powset[k]{S}$ stands for the set of all $k$-element subsets of
% $S$. The argument $k$ is optional, and if not provided, the result
% is the whole powerset of $S$.
\newcommand{\mapeq}[1]{\underset{#1}{\equiv}}
% $\mapeq{f}$ yields $\underset{f}{\equiv}$.
% This is the "equality upon applying $f$" relation.
\newcommand{\eps}{\varepsilon}
% Just a shorthand for a commonly-used symbol.
\newcommand{\N}{\operatorname{N}} % norm of a complex number
\newcommand{\horrule}[1]{\rule{\linewidth}{#1}} % Create horizontal rule command with 1 argument of height
\newcommand{\nnn}{\nonumber\\} % Don't number this line in an "align" environment, and move on to the next line.

%----------------------------------------------------------------------------------------
%	MAKING SUMMATION SIGNS ALWAYS PUT THEIR BOUNDS ABOVE AND BELOW
%	THE SIGN
%----------------------------------------------------------------------------------------
% The following are hacks to ensure that sums (such as
% $\sum_{k=1}^n k$) always put their bounds (i.e., the $k=1$ and the
% $n$) underneath and above the sign, as opposed to on its right.
% Same for products (\prod), set unions (\bigcup) and set
% intersections (\bigcap). Remove the 8 lines below if you do not want
% this behavior.
\let\sumnonlimits\sum
\let\prodnonlimits\prod
\let\cupnonlimits\bigcup
\let\capnonlimits\bigcap
\renewcommand{\sum}{\sumnonlimits\limits}
\renewcommand{\prod}{\prodnonlimits\limits}
\renewcommand{\bigcup}{\cupnonlimits\limits}
\renewcommand{\bigcap}{\capnonlimits\limits}

%----------------------------------------------------------------------------------------
%	ENVIRONMENTS
%----------------------------------------------------------------------------------------
% The incantations below define how theorem environments
% (\begin{theorem} ... \end{theorem}) and their likes will look like.
\newtheoremstyle{plainsl}% <name>
  {8pt plus 2pt minus 4pt}% <Space above>
  {8pt plus 2pt minus 4pt}% <Space below>
  {\slshape}% <Body font>
  {0pt}% <Indent amount>
  {\bfseries}% <Theorem head font>
  {.}% <Punctuation after theorem head>
  {5pt plus 1pt minus 1pt}% <Space after theorem headi>
  {}% <Theorem head spec (can be left empty, meaning `normal')>

% Environments which make the text inside them slanted:
\theoremstyle{plainsl}
  \newtheorem{theorem}{Theorem}[section]
  \newtheorem{proposition}[theorem]{Proposition}
  \newtheorem{lemma}[theorem]{Lemma}
  \newtheorem{corollary}[theorem]{Corollary}
  \newtheorem{conjecture}[theorem]{Conjecture}
% Environments that don't:
\theoremstyle{definition}
  \newtheorem{definition}[theorem]{Definition}
  \newtheorem{example}[theorem]{Example}
  \newtheorem{exercise}[theorem]{Exercise}
  \newtheorem{examples}[theorem]{Examples}
  \newtheorem{algorithm}[theorem]{Algorithm}
  \newtheorem{question}[theorem]{Question}
 \theoremstyle{remark}
  \newtheorem{remark}[theorem]{Remark}
\newenvironment{statement}{\begin{quote}}{\end{quote}}
\newenvironment{fineprint}{\begin{small}}{\end{small}}

%----------------------------------------------------------------------------------------
%	METADATA
%----------------------------------------------------------------------------------------
\newcommand{\myname}{Darij Grinberg} % ENTER YOUR NAME HERE
\newcommand{\myid}{00000000} % ENTER YOUR UMN ID HERE
\newcommand{\mymail}{dgrinber@umn.edu} % ENTER YOUR EMAIL HERE
\newcommand{\psetnumber}{5} % ENTER THE NUMBER OF THIS PSET HERE

%----------------------------------------------------------------------------------------
%	HEADER AND FOOTER
%----------------------------------------------------------------------------------------
\ihead{Solutions to homework set \#\psetnumber} % Page header left
\ohead{page \thepage\ of \pageref{LastPage}} % Page header right
\ifoot{\myname, \myid} % left footer
\ofoot{\mymail} % right footer

%----------------------------------------------------------------------------------------
%	TITLE SECTION
%----------------------------------------------------------------------------------------
\title{	
\normalfont \normalsize 
\textsc{University of Minnesota, School of Mathematics} \\ [25pt] % Your university, school and/or department name(s)
\horrule{0.5pt} \\[0.4cm] % Thin top horizontal rule
\huge Math 4281: Introduction to Modern Algebra, \\
Spring 2019:
Homework \psetnumber\\% The assignment title
\horrule{2pt} \\[0.5cm] % Thick bottom horizontal rule
}
\author{\myname}

\begin{document}

\maketitle % Print the title

\begin{center} % Delete this if you want to save space!
{\large due date: \textbf{Monday, 1 April 2019} at the beginning of class, \\
or before that by email or canvas.

Please solve \textbf{at most 3 of the 6 exercises}!}
\end{center}

%----------------------------------------------------------------------------------------
%	EXERCISE 1
%----------------------------------------------------------------------------------------
\horrule{0.3pt} \\[0.4cm]

\section{Exercise 1: Sums of powers of divisors}

\subsection{Problem}

Let $n$ be a positive integer. Let $k \in \NN$.
Prove that
\[
\sum_{d \mid n} d^k
= \prod_{p \text{ prime}} \tup{p^{0k} + p^{1k} + \cdots + p^{v_p\tup{n}\cdot k} }.
\]
Here, the summation sign ``$\sum_{d \mid n}$'' means a sum over
all \textbf{positive} divisors $d$ of $n$.

\subsection{Solution}

[...]

%----------------------------------------------------------------------------------------
%	EXERCISE 2
%----------------------------------------------------------------------------------------
\horrule{0.3pt} \\[0.4cm]

\section{Exercise 2: Another version of Jacobi's two-squares theorem}

\subsection{Problem}

Let $n$ be a positive integer.
Prove that
\begin{align*}
& \tup{\text{the number of pairs $\tup{x, y} \in \ZZ^2$ such that $n = x^2 + y^2$}} \\
& = 4 \tup{\text{the number of positive divisors $d$ of $n$ such that $d \equiv 1 \mod 4$}} \\
& \qquad - 4 \tup{\text{the number of positive divisors $d$ of $n$ such that $d \equiv 3 \mod 4$}} .
\end{align*}

[\textbf{Hint:} The formula for the left hand side
that we proved in class can be freely used.]

\subsection{Solution}

[...]

%----------------------------------------------------------------------------------------
%	EXERCISE 3
%----------------------------------------------------------------------------------------
\horrule{0.3pt} \\[0.4cm]

\section{Exercise 3: Characterizing Gaussian primes}

\subsection{Problem}

Let $\pi$ be a Gaussian prime.

Prove the following:

\begin{enumerate}

\item[\textbf{(a)}]
If $\pi$ is unit-equivalent to an integer, then $\pi$
is unit-equivalent to a prime\footnote{%
The unqualified word ``prime'' always means a prime in the
original sense, i.e., an integer $p > 1$ whose only positive
divisors are $1$ and $p$.}
of Type 3.

\end{enumerate}

\noindent
(Recall that a prime $p$ is said to be \textit{of Type 3}
if it is congruent to $3$ modulo $4$.)

Assume, from now on, that $\pi$ is \textbf{not}
unit-equivalent to any integer.
Let $\tup{p_1, p_2, \ldots, p_k}$ be a prime factorization
of the positive integer $\N\tup{\pi}$.
(Thus, $p_1, p_2, \ldots, p_k$ are primes such that
$\N\tup{\pi} = p_1 p_2 \cdots p_k$.)

\begin{enumerate}

\item[\textbf{(b)}]
Prove that $\pi \mid p_i$ for some $i \in \set{1, 2, \ldots, k}$.

\end{enumerate}

Fix an $i \in \set{1, 2, \ldots, k}$ such that $\pi \mid p_i$.

\begin{enumerate}

\item[\textbf{(c)}]
Prove that $p_i = \pi \overline{\pi}$.

\item[\textbf{(d)}]
Prove that $p_i$ is a prime of Type 1 or of Type 2.

\end{enumerate}

\noindent
(Recall that a prime $p$ is said to be \textit{of Type 1} if
it is congruent to $1$ modulo $4$,
and is said to be \textit{of Type 2} if it
equals $2$.)

\subsection{Remark}

This exercise yields that the Gaussian primes are the
primes of Type 3 and the Gaussian prime divisors of the
primes of Types 1 and 2 (up to unit-equivalence).
Conversely, any of the latter are indeed Gaussian primes
(as we proved in class).
This completes the characterization of Gaussian primes.

\subsection{Solution}

[...]

%----------------------------------------------------------------------------------------
%	EXERCISE 4
%----------------------------------------------------------------------------------------
\horrule{0.3pt} \\[0.4cm]

\section{Exercise 4: Gaussian integers modulo a Gaussian integer}

\subsection{Problem}

For any Gaussian integer $\tau$, we
let $\mapeq{\tau}$ be the binary relation on
$\ZZ\ive{i}$ defined by
\[
\tup{\alpha \mapeq{\tau} \beta}
\ \iff \ \tup{\alpha \equiv \beta \mod \tau} .
\]
It is straightforward to see (just as in the case of
integers) that this relation $\mapeq{\tau}$ is an
equivalence relation.
(You don't need to prove this.)
We shall refer to the equivalence classes of this
relation $\mapeq{\tau}$ as the \textit{Gaussian
residue classes modulo $\tau$};
let $\ZZ\ive{i} / \tau$ be the set of all these classes.

Let $n$ be a nonzero integer.

Prove that the equivalence classes of the relation
$\mapeq{n}$ (on $\ZZ\ive{i}$) are the $n^2$ classes
$\ive{a + bi}_{\mapeq{n}}$ for
$a, b \in \set{0, 1, \ldots, \abs{n}-1}$,
and that these $n^2$ classes are all distinct.

\subsection{Remark}

This exercise yields
$\abs{\ZZ\ive{i} / n} = n^2 = \N\tup{n}$ for any nonzero
integer $n$.
This is \cite[Lemma 7.15]{Conrad-Gauss}.
(Conrad proves this ``by example''; you can follow the
argument but you should write it up in full generality.)

More generally,
$\abs{\ZZ\ive{i} / \tau} = \N\tup{\tau}$
for any nonzero Gaussian integer $\tau$.
This is proven in \cite[Theorem 7.14]{Conrad-Gauss}
(using the above exercise as a stepping stone).

\subsection{Solution}

[...]

%----------------------------------------------------------------------------------------
%	EXERCISE 5
%----------------------------------------------------------------------------------------
\horrule{0.3pt} \\[0.4cm]

\section{Exercise 5: A Fibonacci divisibility}

\subsection{Problem}

Let $\phi = \dfrac{1+\sqrt5}{2}$ and $\psi = \dfrac{1-\sqrt5}{2}$
be the two (real) roots of the polynomial $x^2 - x - 1$.
(The number $\phi$ is known as the \textit{golden ratio}.)
It is easy to see that $\phi + \psi = 1$ and $\phi \cdot \psi = -1$.

Let $\ZZ\ive{\phi}$ be the set of all reals of the form
$a + b \phi$ with $a, b \in \ZZ$.

\begin{enumerate}

\item[\textbf{(a)}]
Prove that any $\alpha, \beta \in \ZZ\ive{\phi}$ satisfy
$\alpha + \beta \in \ZZ\ive{\phi}$ and
$\alpha - \beta \in \ZZ\ive{\phi}$ and
$\alpha \beta \in \ZZ\ive{\phi}$.

\end{enumerate}

(In the terminology of abstract algebra, this is saying
that $\ZZ\ive{\phi}$ is a subring of $\RR$.)

\begin{enumerate}

\item[\textbf{(b)}]
Prove that every element of $\ZZ\ive{\phi}$ can be
written as $a + b \phi$ for a \textbf{unique} pair
$\tup{a, b}$ of integers.
(In other words, if four integers $a, b, c, d$
satisfy $a + b \phi = c + d \phi$, then $a = c$
and $b = d$.)

\end{enumerate}

Given two elements $\alpha$ and $\beta$ of $\ZZ\ive{\phi}$,
we say that \textit{$\alpha \mid \beta$ in $\ZZ\ive{\phi}$}
if and only if there exists some $\gamma \in \ZZ\ive{\phi}$
such that $\beta = \alpha \gamma$.
Thus, we have defined divisibility in $\ZZ\ive{\phi}$.
Basic properties of divisibility of integers (such as
Proposition 2.2.4) still apply
to divisibility in $\ZZ\ive{\phi}$ (with the same proofs).

\begin{enumerate}

\item[\textbf{(c)}]
If $a$ and $b$ are two elements of
$\ZZ$ such that $a \mid b$ in $\ZZ\ive{\phi}$,
then prove that $a \mid b$ in $\ZZ$.

\end{enumerate}

Let $\tup{f_0, f_1, f_2, \ldots}$ be the sequence of nonnegative
integers defined recursively by
\[
f_0 = 0, \qquad
f_1 = 1, \qquad \text{and} \qquad
f_n = f_{n-1} + f_{n-2} \text{ for all } n \geq 2 .
\]
This is the so-called \textit{Fibonacci sequence} (and continues
with $f_2 = 1$, $f_3 = 2$, $f_4 = 3$, $f_5 = 5$ etc.).

It is well-known (\textit{Binet's formula}) that
\[
f_n = \dfrac{\phi^n - \psi^n}{\sqrt5}
\qquad \text{for all } n \geq 0 .
\]
(You don't need to prove this; there is a completely
straightforward proof by induction on $n$.)

\begin{enumerate}

\item[\textbf{(d)}]
Prove that $f_d \mid f_{dn}$ for any nonnegative integers
$d$ and $n$.

\end{enumerate}

[\textbf{Hint:} Lemma 2.10.11 \textbf{(a)} holds not just
for integers.]

\subsection{Remark}

This exercise (specifically its part \textbf{(d)}) is an
example of how a property of integers (here, $f_d \mid f_{dn}$)
can often be proved by working in a larger domain
(in our case, $\ZZ\ive{\phi}$).
Another example is our study of sums of two perfect squares
using Gaussian integers (done in class).
There are various others.
While part \textbf{(d)} of this exercise has fairly simple
solutions using integer arithmetic alone, some other properties
of Fibonacci numbers are best understood by means of working
in $\ZZ\ive{\phi}$.
For example, if $p \neq 5$ is a prime, then one of the two
Fibonacci numbers $f_{p-1}$ and $f_{p+1}$ is divisible by
$p$, while the other is $\equiv 1 \mod p$.

\subsection{Solution}

[...]

%----------------------------------------------------------------------------------------
%	EXERCISE 6
%----------------------------------------------------------------------------------------
\horrule{0.3pt} \\[0.4cm]

\section{Exercise 6: Non-unique factorization in $\ZZ\ive{\sqrt{-3}}$}

\subsection{Problem}

We let $\sqrt{-3}$ denote the complex number $\sqrt3 i$.

Let $\ZZ\ive{\sqrt{-3}}$ be the set of all complex numbers
of the form $a + b \sqrt{-3}$ with $a, b \in \ZZ$.
These complex numbers are called the
\textit{$3$-Gaussian integers}.

It is easy to see that the set $\ZZ\ive{\sqrt{-3}}$
is closed under addition, subtraction and multiplication
(i.e., that any $\alpha, \beta \in \ZZ\ive{\sqrt{-3}}$ satisfy
$\alpha + \beta \in \ZZ\ive{\sqrt{-3}}$ and
$\alpha - \beta \in \ZZ\ive{\sqrt{-3}}$ and
$\alpha \beta \in \ZZ\ive{\sqrt{-3}}$).
(In the terminology of abstract algebra, this is saying
that $\ZZ\ive{\sqrt{-3}}$ is a subring of $\CC$.)

It is also easy to see that each element of $\ZZ\ive{\sqrt{-3}}$
can be written as $a + b \sqrt{-3}$ for a \textbf{unique} pair
$\tup{a, b}$ of integers.

\begin{enumerate}

\item[\textbf{(a)}]
Prove that each $3$-Gaussian integer $\alpha$ satisfies
$\N\tup{\alpha} \in \NN$ and
$\N\tup{\alpha} \not\equiv 2 \mod 3$.

\end{enumerate}

\noindent
(Recall that $\N\tup{\alpha}$ is defined for every complex
number $\alpha$, and thus for every $3$-Gaussian integer
$\alpha$, since $3$-Gaussian integers are complex numbers.)

Given two elements $\alpha$ and $\beta$ of $\ZZ\ive{\sqrt{-3}}$,
we say that \textit{$\alpha \mid \beta$ in $\ZZ\ive{\sqrt{-3}}$}
if and only if there exists some $\gamma \in \ZZ\ive{\sqrt{-3}}$
such that $\beta = \alpha \gamma$.
Thus, we have defined divisibility in $\ZZ\ive{\sqrt{-3}}$.
Basic properties of divisibility of integers (such as
Proposition 2.2.4) still apply
to divisibility in $\ZZ\ive{\sqrt{-3}}$ (with the same proofs).

If $\alpha \in \ZZ\ive{\sqrt{-3}}$, then a
\textit{$3$-Gaussian divisor of $\alpha$} shall mean a
$\beta \in \ZZ\ive{\sqrt{-3}}$ such that $\beta \mid \alpha$
in $\ZZ\ive{\sqrt{-3}}$.

We define the notions of ``inverse'', ``unit'' and
``unit-equivalent'' for $3$-Gaussian integers
as we did for Gaussian integers.

A nonzero $3$-Gaussian integer $\pi$ that is not a unit is
called a \textit{$3$-Gaussian prime} if each $3$-Gaussian
divisor of $\pi$ is either a unit or unit-equivalent to $\pi$.

\begin{enumerate}

\item[\textbf{(b)}]
List all the $3$-Gaussian integers having norms $< 4$.

\item[\textbf{(c)}]
List all units in $\ZZ\ive{\sqrt{-3}}$.

\item[\textbf{(d)}]
Prove that $2$, $1 + \sqrt{-3}$ and $1 - \sqrt{-3}$ are
$3$-Gaussian primes.

\item[\textbf{(e)}]
Prove that
$2 \cdot 2 = \tup{1 + \sqrt{-3}} \cdot \tup{1 - \sqrt{-3}}$.

\item[\textbf{(f)}]
Define two $3$-Gaussian integers $\alpha$ and $\beta$
by $\alpha = 2$ and $\beta = 1 + \sqrt{-3}$.
Prove that there exist no $3$-Gaussian integers $\gamma$
and $\rho$ such that $\alpha = \gamma \beta + \rho$
and $\N\tup{\rho} < \N\tup{\beta}$.

\end{enumerate}

[\textbf{Hint:} Your list in part \textbf{(b)} should
contain $5$ entries.
Your list in part \textbf{(c)} should contain $2$
entries: Unlike the ring $\ZZ\ive{i}$ with its $4$ units,
the ring $\ZZ\ive{\sqrt{-3}}$
has only $2$ units.

For \textbf{(d)}, discuss the norm
of any possible $3$-Gaussian divisor.]

\subsection{Remark}

Parts \textbf{(d)} and \textbf{(e)} of this exercise
show that unique factorization into primes is not
automatically preserved when we extend a number system.
Neither is division with remainder, as part \textbf{(f)}
illustrates (though we already have seen the geometric
reason for this in class).
(Neither is the existence of a well-behaved greatest
common divisor.)

\subsection{Solution}

[...]

\begin{thebibliography}{99999999}                                                                                         %

% Feel free to add your sources -- or copy some from the source code
% of the class notes ( http://www.cip.ifi.lmu.de/~grinberg/t/19s/notes.tex ).

% This is the bibliography: The list of papers/books/articles/blogs/...
% cited. The syntax is: "\bibitem[name]{tag}Reference",
% where "name" is the name that will appear in the compiled
% bibliography, and "tag" is the tag by which you will refer to
% the source in the TeX file. For example, the following source
% has name "GrKnPa94" (so you will see it referenced as
% "[GrKnPa94]" in the compiled PDF) and tag "GKP" (so you
% can cite it by writing "\cite{GKP}").

% I've commented the below references out, since I'm not citing
% them above.

% \bibitem[GrKnPa94]{GKP}Ronald L. Graham, Donald E. Knuth, Oren Patashnik,
% \textit{Concrete Mathematics, Second Edition}, Addison-Wesley 1994.\\
% See \url{https://www-cs-faculty.stanford.edu/~knuth/gkp.html} for errata.

% \bibitem[Grinbe19]{detnotes}Darij Grinberg,
% \textit{Notes on the combinatorial fundamentals of algebra},
% 10 January 2019. \\
% \url{http://www.cip.ifi.lmu.de/~grinberg/primes2015/sols.pdf}
% \\
% The numbering of theorems and formulas in this link might shift
% when the project gets updated; for a ``frozen'' version whose
% numbering is guaranteed to match that in the citations above, see
% \url{https://github.com/darijgr/detnotes/releases/tag/2019-01-10} .

\bibitem[ConradG]{Conrad-Gauss}Keith Conrad, \textit{The Gaussian
integers}.\newline\url{http://www.math.uconn.edu/~kconrad/blurbs/ugradnumthy/Zinotes.pdf}

\end{thebibliography}

\end{document}
